\let\negmedspace\undefined
\let\negthickspace\undefined

\documentclass[journal]{IEEEtran}
\usepackage[a5paper, margin=10mm, onecolumn]{geometry}
%\usepackage{lmodern} % Ensure lmodern is loaded for pdflatex
\usepackage{tfrupee} % Include tfrupee package

\setlength{\headheight}{1cm} % Set the height of the header box
\setlength{\headsep}{0mm}     % Set the distance between the header box and the top of the text

\usepackage{gvv-book}
\usepackage{gvv}
\usepackage{cite}
\usepackage{amsmath,amssymb,amsfonts,amsthm}
\usepackage{algorithmic}
\usepackage{graphicx}
\usepackage{textcomp}
\usepackage{xcolor}
\usepackage{txfonts}
\usepackage{listings}
\usepackage{enumitem}
\usepackage{mathtools}
\usepackage{gensymb}
\usepackage{comment}
\usepackage[breaklinks=true]{hyperref}
\usepackage{tkz-euclide} 
\usepackage{listings}
% \usepackage{gvv}                                        
\def\inputGnumericTable{}                                 
\usepackage[latin1]{inputenc}                                
\usepackage{color}                                            
\usepackage{array}                                            
\usepackage{longtable}                                       
\usepackage{calc}                                             
\usepackage{multirow}                                         
\usepackage{hhline}                                           
\usepackage{ifthen}                                           
\usepackage{lscape}
\begin{document}

\bibliographystyle{IEEEtran}
\vspace{3cm}

\title{2.3.13}
\author{EE25BTECH11010 - Arsh Dhoke}
{\let\newpage\relax\maketitle}

\renewcommand{\thefigure}{\theenumi}
\renewcommand{\thetable}{\theenumi}
\setlength{\intextsep}{10pt}
\numberwithin{equation}{enumi}
\numberwithin{figure}{enumi}
\renewcommand{\thetable}{\theenumi}

\parindent 0px
\textbf{Question:} \\
Find the angle which the line $\frac{x}{1}=\frac{y}{-1}=\frac{z}{2}$ makes with the positive direction of the Y axis.

\solution \\

The line can be represented as
$k \myvec{1\\-1\\2}$ 
\\

Hence its direction vector is

\begin{align}
\vec{v} &= \myvec{1\\-1\\2} \\
\vec{e_2} &= \myvec{0\\1\\0} \\
\vec{v}^T\vec{e_2} &= 
\myvec{1 & -1 & 2}\myvec{0\\1\\0}=-1 \\
\|\vec{v}\| &= 
\sqrt{\vec{v}^T\vec{v}}
=\sqrt{\myvec{1 & -1 & 2}\myvec{1\\-1\\2}}=\sqrt{6} \\
\|\vec{e_2}\| &=1 \\
\cos\theta &= 
\frac{\vec{v}^T\vec{e_2}}{\|\vec{v}\|\|\vec{e_2}\|}
=\frac{-1}{\sqrt{6}} \\
\theta &= \cos^{-1}\!\brak{-\frac{1}{\sqrt{6}}}
\end{align}

Therefore,
$
\boxed{\theta=\cos^{-1}\!\brak{-\frac{1}{\sqrt{6}}}\;\approx\;114.09\degree}
$

\begin{figure}[ht!]
\centering
\includegraphics[height=0.6\textheight, keepaspectratio]{figs/q3.png}
\captionof{figure}{Graph}
\end{figure}


\end{document}