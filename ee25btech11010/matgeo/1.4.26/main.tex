\let\negmedspace\undefined
\let\negthickspace\undefined
\documentclass[journal]{IEEEtran}
\usepackage[a5paper, margin=10mm, onecolumn]{geometry}
\usepackage{tfrupee}

\setlength{\headheight}{1cm}
\setlength{\headsep}{0mm}

\usepackage{cite}
\usepackage{amsmath,amssymb,amsfonts}
\usepackage{algorithmic}
\usepackage{graphicx}
\usepackage{textcomp}
\usepackage{xcolor}
\usepackage{float}
\usepackage{txfonts}
\usepackage{listings}
\usepackage{enumitem}
\usepackage{mathtools}
\usepackage{gensymb}
\usepackage{comment}
\usepackage[breaklinks=true]{hyperref}
\usepackage{tkz-euclide} 
\usepackage{longtable,multirow,array,hhline}
\renewcommand{\vec}[1]{\mathbf{#1}}
\newcommand{\solution}{\textbf{Solution: }}

% Define brak for matrices
\newcommand{\brak}[1]{\begin{pmatrix}#1\end{pmatrix}}

\begin{document}

\bibliographystyle{IEEEtran}
\vspace{3cm}

\title{1.4.26}
\author{EE25BTECH11010 - Arsh Dhoke}
{\let\newpage\relax\maketitle}

\renewcommand{\thefigure}{\theenumi}
\renewcommand{\thetable}{\theenumi}
\setlength{\intextsep}{10pt}
\numberwithin{equation}{enumi}
\numberwithin{figure}{enumi}

\parindent 0px
\textbf{Question:} \\
The position vector of the point which divides the join of points $2\vec{a} - 3\vec{b}$ and $\vec{a} + \vec{b}$ in the ratio $3:1$ is \underline{\hspace{2cm}}.

\solution \\

\begin{align}
\vec{P} &= 2\vec{a} - 3\vec{b}
   = \brak{2a\\-3b}, \\
\vec{Q} &= \vec{a} + \vec{b}
   = \brak{a\\b}.
\end{align}

Using section formula, the point $R$ dividing $PQ$ in ratio $3:1$ is
\begin{align}
\vec{R} &= \frac{3\vec{Q} + 1\vec{P}}{3+1}. \\
\vec{R} &= \brak{\begin{matrix}Q & P\end{matrix}}\brak{\begin{matrix}\frac{3}{4} \\[6pt]\frac{1}{4}\end{matrix}} \\
\vec{R} &= \brak{\begin{matrix}\brak{a \\ b} & \brak{2a \\ -3b}\end{matrix}}\brak{\begin{matrix}\frac{3}{4} \\[6pt]\frac{1}{4}\end{matrix}} \\
\vec{R}
&= \frac{1}{4}\Big(3\brak{a\\b}+\brak{2a\\-3b}\Big) \\[6pt]
&= \frac{1}{4}\brak{3a+2a\\3b-3b} \\[6pt]
&= \frac{1}{4}\brak{5a\\0} \\[6pt]
&= \brak{\dfrac{5a}{4}\\0}.
\end{align}

\begin{align}
\boxed{\,\vec{R} = \brak{\dfrac{5a}{4}\\0}\,}
\end{align}

Let a=1 and b=0.

\begin{figure}[ht!]
\centering
\includegraphics[height=0.5\textheight, keepaspectratio]{figs/q1.png}
\caption{Graph for question 1}
\end{figure}

\end{document}
