\documentclass{beamer}

\usetheme{Madrid}
\usecolortheme{default}

\usepackage{tfrupee}
\usepackage{amsmath,amssymb,amsfonts}
\usepackage{algorithmic}
\usepackage{graphicx}
\usepackage{textcomp}
\usepackage{xcolor}
\usepackage{float}
\usepackage{txfonts}
\usepackage{listings}
\usepackage{enumitem}
\usepackage{mathtools}
\usepackage{gensymb}
\usepackage{comment}
\usepackage{tkz-euclide} 
\usepackage{longtable,multirow,array,hhline}

\definecolor{codeblue}{rgb}{0,0,1}

\lstdefinestyle{CStyle}{
  language=C,
  basicstyle=\ttfamily\footnotesize,
  keywordstyle=\color{blue}\bfseries,
  stringstyle=\color{orange},
  commentstyle=\color{green!50!black},
  numbers=none,
  breaklines=true,
  frame=single, % THIS adds the box
  showstringspaces=false,
  tabsize=2
}

\lstset{style=CStyle}


\newcommand{\brak}[1]{\begin{pmatrix}#1\end{pmatrix}}

\title{1.4.26}
\subtitle{Vector Section Formula}
\author{EE25BTECH11010 - Arsh Dhoke}
\date{}

\begin{document}

\begin{frame}
\titlepage
\end{frame}

\begin{frame}{Question}
The position vector of the point which divides the join of points 
\[
2\mathbf{a} - 3\mathbf{b} \text{ and } \mathbf{a} + \mathbf{b}
\]
in the ratio \( 3:1 \) is \underline{\hspace{2cm}}.
\end{frame}

\begin{frame}{Theoretical Solution}
\[
P = 2\mathbf a - 3\mathbf b = \brak{2\\-3}, \quad 
Q = \mathbf a + \mathbf b = \brak{1\\1}.
\]
\end{frame}

\begin{frame}{Equation}
Using section formula,
the point \(R\) dividing \(PQ\) in ratio \(3:1\) is:
\[
R=\frac{3Q+1P}{3+1}.
\]
\begin{align*}
R &= \frac{1}{4}\Big(3\brak{1\\1}+\brak{2\\-3}\Big) \\
  &= \frac{1}{4}\brak{3+2\\3-3} \\
  &= \frac{1}{4}\brak{5\\0} \\
  &= \brak{\dfrac{5}{4}\\0}.
\end{align*}
\[
\boxed{R=\brak{\dfrac{5}{4}\\0}}
\]
\end{frame}

\begin{frame}[fragile]
    \frametitle{C Code - Section formula}

    \begin{lstlisting}
#include <stdio.h>

void sectionFormula(int m, int n, float a, float b, float *x) {
    *x = (m * b + n * a) / (float)(m + n);
}

    \end{lstlisting}
\end{frame}

\begin{frame}[fragile]
    \frametitle{Python Code}

    \begin{lstlisting}
import numpy as np
import matplotlib.pyplot as plt

# Define values for a and b
a = 2  # Example value
b = 3  # Example value

# Define points as NumPy arrays
P = np.array([2*a, -3*b])  # Point P
Q = np.array([a, b])       # Point Q

# Ratio m:n
m = 3
n = 1

# Section formula for internal division
R = (m * Q + n * P) / (m + n)

# Print the result
print("Position vector of the point R:", R)

    \end{lstlisting}
\end{frame}

\begin{frame}[fragile]
    \frametitle{Python Code}

    \begin{lstlisting}

# Plotting
plt.figure(figsize=(6, 6))
plt.axhline(0, color='black', linewidth=0.8)
plt.axvline(0, color='black', linewidth=0.8)

# Plot P, Q, and R
plt.scatter(*P, color='blue', label='P (2a, -3b)')
plt.scatter(*Q, color='green', label='Q (a, b)')
plt.scatter(*R, color='red', label=f'R (ratio {m}:{n})')

# Draw line between P and Q
plt.plot([P[0], Q[0]], [P[1], Q[1]], color='gray', linestyle='--')

# Annotate points
plt.text(P[0]+0.2, P[1]+0.2, 'P')
plt.text(Q[0]+0.2, Q[1]+0.2, 'Q')
plt.text(R[0]+0.2, R[1]+0.2, 'R')

\end{lstlisting}
\end{frame}


\begin{frame}[fragile]
    \frametitle{Python Code}

    \begin{lstlisting}

plt.xlabel('X-axis')
plt.ylabel('Y-axis')
plt.title('Section Formula Visualization')
plt.legend()
plt.grid(True)
plt.savefig("/home/arsh-dhoke/ee1030-2025/ee25btech11010/matgeo/1.4.26/figs/q1.png")
plt.show()

    \end{lstlisting}
\end{frame}

\begin{frame}[fragile]
    \frametitle{Python+ C Code}

    \begin{lstlisting}

import ctypes
import numpy as np
import matplotlib.pyplot as plt

# Load the shared library
lib = ctypes.CDLL("./libsection_int.so")

# Define argument and return types
lib.sectionFormula.argtypes = [
    ctypes.c_int, ctypes.c_int,
    ctypes.POINTER(ctypes.c_double), ctypes.POINTER(ctypes.c_double),
    ctypes.POINTER(ctypes.c_double)
]
lib.sectionFormula.restype = None

# Values for a and b
a = 2
b = 3


    \end{lstlisting}
\end{frame}

\begin{frame}[fragile]
    \frametitle{Python+ C Code}

    \begin{lstlisting}
    
# Points P and Q
P = (ctypes.c_double * 2)(2 * a, -3 * b)
Q = (ctypes.c_double * 2)(a, b)
R = (ctypes.c_double * 2)(0.0, 0.0)

# Ratio m:n
m, n = 3, 1

# Call the C function
lib.sectionFormula(m, n, P, Q, R)

# Convert to NumPy arrays for plotting
P_np = np.array([P[0], P[1]])
Q_np = np.array([Q[0], Q[1]])
R_np = np.array([R[0], R[1]])

print("Position vector of the point R (from C):", R_np)

  \end{lstlisting}
\end{frame}

\begin{frame}[fragile]
    \frametitle{Python+ C Code}

    \begin{lstlisting}

# Plotting
plt.figure(figsize=(6, 6))
plt.axhline(0, color='black', linewidth=0.8)
plt.axvline(0, color='black', linewidth=0.8)

# Plot P, Q, and R
plt.scatter(*P_np, color='blue', label='P (2a, -3b)')
plt.scatter(*Q_np, color='green', label='Q (a, b)')
plt.scatter(*R_np, color='red', label=f'R (ratio {m}:{n})')

# Draw line between P and Q
plt.plot([P_np[0], Q_np[0]], [P_np[1], Q_np[1]], color='gray', linestyle='--')

# Annotate points
plt.text(P_np[0]+0.2, P_np[1]+0.2, 'P')
plt.text(Q_np[0]+0.2, Q_np[1]+0.2, 'Q')
plt.text(R_np[0]+0.2, R_np[1]+0.2, 'R')

\end{lstlisting}
\end{frame}

\begin{frame}[fragile]
    \frametitle{Python+ C Code}

    \begin{lstlisting}

plt.xlabel('X-axis')
plt.ylabel('Y-axis')
plt.title('Section Formula Visualization (Using C & Python)')
plt.legend()
plt.grid(True)
plt.axis('equal')

# Save the plot
plt.savefig("/home/arsh-dhoke/ee1030-2025/ee25btech11010/matgeo/1.4.26/figs/q1.png")

# Show plot
plt.show()


      \end{lstlisting}
\end{frame}

\begin{frame}{Plot}
\centering
\includegraphics[height=0.7\textheight, keepaspectratio]{figs/q1.png}
\end{frame}

\end{document}
