\let\negmedspace\undefined
\let\negthickspace\undefined

\documentclass[journal]{IEEEtran}
\usepackage[a5paper, margin=10mm, onecolumn]{geometry}
%\usepackage{lmodern} % Ensure lmodern is loaded for pdflatex
\usepackage{tfrupee} % Include tfrupee package

\setlength{\headheight}{1cm} % Set the height of the header box
\setlength{\headsep}{0mm}     % Set the distance between the header box and the top of the text

\usepackage{gvv-book}
\usepackage{gvv}
\usepackage{cite}
\usepackage{amsmath,amssymb,amsfonts,amsthm}
\usepackage{algorithmic}
\usepackage{graphicx}
\usepackage{textcomp}
\usepackage{xcolor}
\usepackage{txfonts}
\usepackage{listings}
\usepackage{enumitem}
\usepackage{mathtools}
\usepackage{gensymb}
\usepackage{comment}
\usepackage[breaklinks=true]{hyperref}
\usepackage{tkz-euclide} 
\usepackage{listings}
% \usepackage{gvv}                                        
\def\inputGnumericTable{}                                 
\usepackage[latin1]{inputenc}                                
\usepackage{color}                                            
\usepackage{array}                                            
\usepackage{longtable}                                       
\usepackage{calc}                                             
\usepackage{multirow}                                         
\usepackage{hhline}                                           
\usepackage{ifthen}                                           
\usepackage{lscape}
\begin{document}

\bibliographystyle{IEEEtran}
\vspace{3cm}

\title{2.7.22}
\author{EE25BTECH11010 - Arsh Dhoke}
{\let\newpage\relax\maketitle}

\renewcommand{\thefigure}{\theenumi}
\renewcommand{\thetable}{\theenumi}
\setlength{\intextsep}{10pt}
\numberwithin{equation}{enumi}
\numberwithin{figure}{enumi}
\renewcommand{\thetable}{\theenumi}

\parindent 0px
\textbf{Question}:\\
The area of a triangle whose vertices are (5,0), (8,0), (8,4) (in sq.units) is

\solution \\

\begin{tabular}[12pt]{ |c| c|}
    \hline
    \textbf{Name} & \textbf{Point}\\ 
    \hline
	Point A &\myvec{h \\ k}\\
    \hline 
 Point B &\myvec{x1 \\ y1}\\
    \hline
	  Point R &\myvec{x2 \\ y2}\\
    \hline
    
    \end{tabular}


\begin{align}
\vec{A} &= \myvec{5\\0}, \quad 
\vec{B} = \myvec{8\\0}, \quad 
\vec{C} = \myvec{8\\4}
\end{align}

\begin{align}
	\vec{A}-\vec{B}=\myvec{-3\\0},\quad \vec{A}-\vec{C}=\myvec{-3\\-4}
\end{align}
\begin{align}
	\brak{ABC}=\frac{1}{2}\norm{\brak{\vec{A}-\vec{B}}\times\brak{\vec{A}-\vec{C}}}=6
\end{align}

Hence, the area of $\triangle ABC$ is $\textbf{6}$.

\begin{figure}[ht!]
\centering
\includegraphics[height=0.6\textheight, keepaspectratio]{figs/q4.png}
\captionof{figure}{Graph}
\end{figure}

\end{document}