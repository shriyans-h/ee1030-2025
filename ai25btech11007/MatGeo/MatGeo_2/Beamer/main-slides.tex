\documentclass{beamer}
\usepackage[utf8]{inputenc}

\usetheme{Madrid}
\usecolortheme{default}
\usepackage{amsmath,amssymb,amsfonts,amsthm}
\usepackage{txfonts}
\usepackage{tkz-euclide}
\usepackage{listings}
\usepackage{adjustbox}
\usepackage{array}
\usepackage{tabularx}
\usepackage{gvv}
\usepackage{lmodern}
\usepackage{circuitikz}
\usepackage{tikz}
\usepackage{graphicx}

\setbeamertemplate{page number in head/foot}[totalframenumber]

\usepackage{tcolorbox}
\tcbuselibrary{minted,breakable,xparse,skins}



\definecolor{bg}{gray}{0.95}
\DeclareTCBListing{mintedbox}{O{}m!O{}}{%
  breakable=true,
  listing engine=minted,
  listing only,
  minted language=#2,
  minted style=default,
  minted options={%
    linenos,
    gobble=0,
    breaklines=true,
    breakafter=,,
    fontsize=\small,
    numbersep=8pt,
    #1},
  boxsep=0pt,
  left skip=0pt,
  right skip=0pt,
  left=25pt,
  right=0pt,
  top=3pt,
  bottom=3pt,
  arc=5pt,
  leftrule=0pt,
  rightrule=0pt,
  bottomrule=2pt,
  toprule=2pt,
  colback=bg,
  colframe=orange!70,
  enhanced,
  overlay={%
    \begin{tcbclipinterior}
    \fill[orange!20!white] (frame.south west) rectangle ([xshift=20pt]frame.north west);
    \end{tcbclipinterior}},
  #3,
}
\lstset{
    language=C,
    basicstyle=\ttfamily\small,
    keywordstyle=\color{blue},
    stringstyle=\color{orange},
    commentstyle=\color{green!60!black},
    numbers=left,
    numberstyle=\tiny\color{gray},
    breaklines=true,
    showstringspaces=false,
}
\title 
{MatGeo Assignment 1.6.22}

\author
{AI25BTECH11007}
\begin{document}

\frame{\titlepage}
\begin{frame}{Question}
Show that the points A(2, -3, 4), B(-1, 2, 1) and C(0, 1/3, 2) are collinear.
\end{frame}
\begin{frame}{Solution}
  Let us solve the given equation theoretically and then verify the solution
computationally
According to the question,


\textbf{Show that the points } 
$$A( 2, -3, 4), B( -1, 2, 1), C( 0, 1/3, 2) $$
\textbf{ are collinear (rank method).}
\end{frame}

\begin{frame}{Theoretical Solution}
\begin{equation}
A=\myvec{2 \\ -3 \\ 4}, \quad 
B=\myvec{-1 \\ 2 \\ 1}, \quad 
C=\myvec{0 \\ \tfrac{1}{3} \\ 2}
\end{equation}

\begin{equation}
    B-A=\myvec{-1 \\ 2 \\ 1}-\myvec{2 \\ -3 \\ 4}=\myvec{-3 \\ 5 \\ -3}
\end{equation}

\begin{equation}
   C-A=\myvec{0 \\ \tfrac{1}{3} \\ 2}-\myvec{2 \\ -3 \\ 4}
=\myvec{-2 \\ \tfrac{10}{3} \\ -2}
\end{equation}


Form the $3\times 2$ matrix whose columns are $B-A$ and $C-A$:
\begin{equation}
    M=\big[B-A\ \ C-A\big]
=\myvec{ -3 & -2 \\[4pt] 5 & \tfrac{10}{3} \\[4pt] -3 & -2 }
\end{equation}
\end{frame}
\begin{frame}
The vectors $B-A$ and $C-A$ are linearly dependent (hence the three points are collinear) iff the rank of $M$ is $1$. For a $3\times 2$ matrix this is equivalent to all $2\times 2$ minors of $M$ vanishing. Compute the minors:

Minor from rows 1 and 2:
\begin{equation}
    \begin{vmatrix}
-3 & -2\\[4pt]
5 & \tfrac{10}{3}
\end{vmatrix}
=(-3)\cdot\left(\tfrac{10}{3}\right)-(-2)\cdot 5
=-10-(-10)=0
\end{equation}


Minor from rows 1 and 3:
\begin{equation}
    \begin{vmatrix}
-3 & -2\\[4pt]
-3 & -2
\end{vmatrix}
=(-3)(-2)-(-2)(-3)=6-6=0
\end{equation}


Minor from rows 2 and 3:
\begin{equation}
    \begin{vmatrix}
5 & \tfrac{10}{3}\\[4pt]
-3 & -2
\end{vmatrix}
=5\cdot(-2)-\left(\tfrac{10}{3}\right)\cdot(-3)
=-10-(-10)=0
\end{equation}

\end{frame}
\begin{frame}
All $2\times 2$ minors are zero, so $\operatorname{rank}(M)=1$. Therefore the columns are linearly dependent, i.e,
\begin{equation}
    \operatorname{rank}\big[B-A\ \ C-A\big]=1.
\end{equation}

Hence the vectors $B-A$ and $C-A$ are proportional and the points.

\end{frame}

\begin{frame}{Plot}
   \begin{figure}[H]
    \centering
    \includegraphics[width=0.6\linewidth]{figs/image.png}
    \caption{Image Visual}
    \label{fig:figs/image.png}
\end{figure}
\end{frame}
\begin{frame}{Conclusion}
    As the rank of the matrix M is 1, the three given points are collinear.

From the figure it is clearly verified that theoritical solution matches with the computational solution.
\end{frame}
\end{document}

