\let\negmedspace\undefined
\let\negthickspace\undefined
\documentclass[journal]{IEEEtran}
\usepackage[a5paper, margin=10mm, onecolumn]{geometry}
%\usepackage{lmodern} % Ensure lmodern is loaded for pdflatex
\usepackage{tfrupee} % Include tfrupee package

\setlength{\headheight}{1cm} % Set the height of the header box
\setlength{\headsep}{0mm}     % Set the distance between the header box and the top of the text

\usepackage{gvv-book}
\usepackage{gvv}
\usepackage{cite}
\usepackage{amsmath,amssymb,amsfonts,amsthm}
\usepackage{algorithmic}
\usepackage{graphicx}
\usepackage{textcomp}
\usepackage{xcolor}
\usepackage{txfonts}
\usepackage{listings}
\usepackage{enumitem}
\usepackage{mathtools}
\usepackage{gensymb}
\usepackage{comment}
\usepackage[breaklinks=true]{hyperref}
\usepackage{tkz-euclide} 
\usepackage{listings}
% \usepackage{gvv}                                        
\def\inputGnumericTable{}                                 
\usepackage[latin1]{inputenc}                                
\usepackage{color}                                            
\usepackage{array}                                            
\usepackage{longtable}                                       
\usepackage{calc}                                             
\usepackage{multirow} 
\usepackage{hhline}                                           
\usepackage{ifthen}                                           
\usepackage{lscape}
\usepackage{circuitikz}
\tikzstyle{block} = [rectangle, draw, fill=blue!20, 
    text width=4em, text centered, rounded corners, minimum height=3em]
\tikzstyle{sum} = [draw, fill=blue!10, circle, minimum size=1cm, node distance=1.5cm]
\tikzstyle{input} = [coordinate]
\tikzstyle{output} = [coordinate]

\begin{document}
\bibliographystyle{IEEEtran}
\vspace{3cm}

\title{MatGeo Assignment 3.2.23}
\author{AI25BTECH11007}
 \maketitle
% \newpage
% \bigskip
{\let\newpage\relax\maketitle}

\renewcommand{\thefigure}{\theenumi}
\renewcommand{\thetable}{\theenumi}
\setlength{\intextsep}{10pt} % Space between text and floats


\numberwithin{equation}{enumi}
\numberwithin{figure}{enumi}
\renewcommand{\thetable}{\theenumi}
\textbf{Question:}\\

Construct a triangle $ABC$ in which 
\[
BC = 5 \,\text{cm}, \quad \angle B = 60^\circ, \quad \text{and} \quad AC + AB = 7.5 \,\text{cm}.
\]


\textbf{Solution:}\\
\bigskip
 Set up points and given data

Let
\[
\vec{B}=\begin{pmatrix}0\\0\end{pmatrix}, \qquad
\vec{C}=\begin{pmatrix}5\\0\end{pmatrix}.
\]


\[
\vec{AC} = \vec{c}, \qquad \vec{AC} = \vec{b}, \qquad \vec{b+c}=7.5.
\]

 Position vector of point $A$

Since $\angle B=60^\circ$, the vector $\vec{BA}$ has length $c$ and direction $60^\circ$ above the $x$-axis. Thus
\[
\vec{A} = \vec{c}
\begin{pmatrix}\cos 60^\circ \\ \sin 60^\circ\end{pmatrix}
= \begin{pmatrix}\tfrac{\vec{c}}{2}\\[4pt]\tfrac{\vec{c}\sqrt{3}}{2}\end{pmatrix}.
\]

 Expression for $AC$

\[
\vec{AC} = \vec{C} - \vec{A}
= \begin{pmatrix}5-\tfrac{\vec{c}}{2}\\[4pt]-\tfrac{\vec{c}\sqrt{3}}{2}\end{pmatrix},
\]
and
\[
b^2 = \Big(5-\tfrac{\vec{c}}{2}\Big)^2 + \tfrac{3\vec{c}^2}{4}.
\]

  Apply $b+c=7.5$

Since $b=7.5-c$, we have
\[
(7.5-c)^2 = \Big(5-\tfrac{c}{2}\Big)^2 + \tfrac{3c^2}{4}.
\]
Expanding and simplifying gives,
\[
56.25 -15c + c^2 = 25 -5c + c^2,
\]
\[
\quad c=3.125.
\]
Hence
\[
b = 7.5 - 3.125 = 4.375.
\]

 Coordinates of vertices

\[
\vec{A} = \begin{pmatrix}\tfrac{3.125}{2}\\[4pt]\tfrac{3.125\sqrt{3}}{2}\end{pmatrix}
= \begin{pmatrix}1.5625\\[4pt]2.7050\ldots\end{pmatrix},
\]
\[
\vec{B}=\begin{pmatrix}0\\0\end{pmatrix}, \quad
\vec{C}=\begin{pmatrix}5\\0\end{pmatrix}.
\]

Verification

\[
\vec{BA}\cdot \vec{BC} = \tfrac{5c}{2}, \qquad
|\vec{BA}|=c, \quad |\vec{BC}|=5,
\]
\[
\cos \angle B = \frac{\tfrac{5c}{2}}{5c} = \tfrac{1}{2} = \cos 60^\circ.
\]

Final Answer,

\[
\boxed{
\vec{A} = \begin{pmatrix}1.5625\\[4pt]2.7050\end{pmatrix},\quad
\vec{B} = \begin{pmatrix}0\\0\end{pmatrix},\quad
\vec{C} = \begin{pmatrix}5\\0\end{pmatrix}
}
\]
with $AB=3.125$ cm, $AC=4.375$ cm, and $BC=5$ cm.\\


\begin{figure}[H]
    \centering
    \includegraphics[width=1.0\linewidth]{figs/image.png}
    \caption{Construction Plot}
    \label{fig:placeholder}
\end{figure}

\end{document}