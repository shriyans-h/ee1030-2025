\documentclass{beamer}
\usepackage[utf8]{inputenc}
  
\usetheme{Madrid}
\usecolortheme{default}
\usepackage{amsmath,amssymb,amsfonts,amsthm}
\usepackage{txfonts}
\usepackage{tkz-euclide}
\usepackage{listings}
\usepackage{adjustbox}
\usepackage{array}
\usepackage{tabularx}
\usepackage{gvv}
\usepackage{lmodern}
\usepackage{circuitikz}
\usepackage{tikz}
\usepackage{graphicx}
\usepackage[T1]{fontenc}
\UseRawInputEncoding

\setbeamertemplate{page number in head/foot}[totalframenumber]

\usepackage{tcolorbox}
\tcbuselibrary{minted,breakable,xparse,skins}



\definecolor{bg}{gray}{0.95}
\DeclareTCBListing{mintedbox}{O{}m!O{}}{%
  breakable=true,
  listing engine=minted,
  listing only,
  minted language=#2,
  minted style=default,
  minted options={%
    linenos,
    gobble=0,
    breaklines=true,
    breakafter=,,
    fontsize=\small,
    numbersep=8pt,
    #1},
  boxsep=0pt,
  left skip=0pt,
  right skip=0pt,
  left=25pt,
  right=0pt,
  top=3pt,
  bottom=3pt,
  arc=5pt,
  leftrule=0pt,
  rightrule=0pt,
  bottomrule=2pt,
  toprule=2pt,
  colback=bg,
  colframe=orange!70,
  enhanced,
  overlay={%
    \begin{tcbclipinterior}
    \fill[orange!20!white] (frame.south west) rectangle ([xshift=20pt]frame.north west);
    \end{tcbclipinterior}},
  #3,
}
\lstset{
    language=C,
    basicstyle=\ttfamily\small,
    keywordstyle=\color{blue},
    stringstyle=\color{orange},
    commentstyle=\color{green!60!black},
    numbers=left,
    numberstyle=\tiny\color{gray},
    breaklines=true,
    showstringspaces=false,
}



\title 
{MatGeo Assignment 1.2.13}

\author
{AI25BTECH11007}
\begin{document}

\frame{\titlepage}
\begin{frame}{Question}
Construct a triangle $ABC$ in which 
\[
BC = 5 \,\text{cm}, \quad\angle B = 45^\circ, \quad \text{and} \quad AC + AB = 7.5 \,\text{cm}.
\]

\end{frame}

\begin{frame}{Solution}
    \bigskip
 Set up points and given data

Let
\[
\vec{B}=\begin{pmatrix}0\\0\end{pmatrix}, \qquad
\vec{C}=\begin{pmatrix}5\\0\end{pmatrix}.
\]


\[
\vec{AC} = \vec{c}, \qquad \vec{AC} = \vec{b}, \qquad \vec{b+c}=7.5.
\]

 Position vector of point $A$

Since $\angle B=60^\circ$, the vector $\vec{BA}$ has length $c$ and direction $60^\circ$ above the $x$-axis. Thus
\[
\vec{A} = \vec{c}
\begin{pmatrix}\cos 60^\circ \\ \sin 60^\circ\end{pmatrix}
= \begin{pmatrix}\tfrac{\vec{c}}{2}\\[4pt]\tfrac{\vec{c}\sqrt{3}}{2}\end{pmatrix}.
\]

 Expression for $AC$

\[
\vec{AC} = \vec{C} - \vec{A}
= \begin{pmatrix}5-\tfrac{\vec{c}}{2}\\[4pt]-\tfrac{\vec{c}\sqrt{3}}{2}\end{pmatrix},
\]
\end{frame}
\begin{frame}
and
\[
b^2 = \Big(5-\tfrac{\vec{c}}{2}\Big)^2 + \tfrac{3\vec{c}^2}{4}.
\]

  Apply $b+c=7.5$

Since $b=7.5-c$, we have
\[
(7.5-c)^2 = \Big(5-\tfrac{c}{2}\Big)^2 + \tfrac{3c^2}{4}.
\]
Expanding and simplifying gives,
\[
56.25 -15c + c^2 = 25 -5c + c^2,
\]
\[
\quad c=3.125.
\]
Hence
\[
b = 7.5 - 3.125 = 4.375.
\]
\end{frame}
\begin{frame}
 Coordinates of vertices

\[
\vec{A} = \begin{pmatrix}\tfrac{3.125}{2}\\[4pt]\tfrac{3.125\sqrt{3}}{2}\end{pmatrix}
= \begin{pmatrix}1.5625\\[4pt]2.7050\ldots\end{pmatrix},
\]
\[
\vec{B}=\begin{pmatrix}0\\0\end{pmatrix}, \quad
\vec{C}=\begin{pmatrix}5\\0\end{pmatrix}.
\]

Verification

\[
\vec{BA}\cdot \vec{BC} = \tfrac{5c}{2}, \qquad
|\vec{BA}|=c, \quad |\vec{BC}|=5,
\]
\[
\cos \angle B = \frac{\tfrac{5c}{2}}{5c} = \tfrac{1}{2} = \cos 60^\circ.
\]

Final Answer,

\[
\boxed{
\vec{A} = \begin{pmatrix}1.5625\\[4pt]2.7050\end{pmatrix},\quad
\vec{B} = \begin{pmatrix}0\\0\end{pmatrix},\quad
\vec{C} = \begin{pmatrix}5\\0\end{pmatrix}
}
\]

with $AB=3.125$ cm, $AC=4.375$ cm, and $BC=5$ cm.\\
\end{frame}

\begin{frame}{Construction Plot}
\begin{figure}
    \centering
    \includegraphics[width=1\linewidth]{figs/image.png}
    \caption{Construction Plot}
    \label{fig:figs/image.png}
\end{figure}
\end{frame}


\begin{frame}[fragile]{C Code: Dot Product and Magnitude}
\begin{lstlisting}[language=C]
#include <stdio.h>
#include <math.h>

// Function to compute dot product of 2D vectors
double dotProduct(double A[], double B[]) {
    return A[0]*B[0] + A[1]*B[1];
}

// Function to compute magnitude of a 2D vector
double magnitude(double V[]) {
    return sqrt(V[0]*V[0] + V[1]*V[1]);
}
\end{lstlisting}
\end{frame}

% C code part 2
\begin{frame}[fragile]{C Code: Angle Calculation and Main}
\begin{lstlisting}[language=C]
// Function to calculate angle (in degrees) between two vectors
double angleBetweenVectors(double A[], double B[]) {
    double dot = dotProduct(A, B);
    double magA = magnitude(A);
    double magB = magnitude(B);
    double cosTheta = dot / (magA * magB);

    // Clamp for numerical stability
    if (cosTheta > 1.0) cosTheta = 1.0;
    else if (cosTheta < -1.0) cosTheta = -1.0;

    return acos(cosTheta) * (180.0 / M_PI);
}
\end{lstlisting}
\end{frame}

\begin{frame}[fragile]{C Code}
\begin{lstlisting}[language=C]
int main() {
    // Coordinates of vertices
    double A[2] = {1.5625, 2.705};
    double B[2] = {0.0, 0.0};
    double C[2] = {5.0, 0.0};

    // Vectors BA and BC
    double BA[2] = {A[0] - B[0], A[1] - B[1]};
    double BC[2] = {C[0] - B[0], C[1] - B[1]};

    printf("Angle at B: %.2f degrees\n", angleBetweenVectors(BA, BC));
    return 0;
}
\end{lstlisting}
\end{frame}


% Python code part 1
\begin{frame}[fragile]{Python Code: Setup and Points}
\begin{lstlisting}[language=Python]
import matplotlib.pyplot as plt
import numpy as np

# Coordinates of vertices
A = np.array([1.5625, 2.705])   # Computed intersection
B = np.array([0, 0])            # Origin
C = np.array([5, 0])            # On x-axis
\end{lstlisting}
\end{frame}

% Python code part 2
\begin{frame}[fragile]{Python Code: Plot Triangle}
\begin{lstlisting}[language=Python]
fig, ax = plt.subplots()

# Plot the triangle edges
triangle_points = np.array([A, B, C, A])
ax.plot(triangle_points[:, 0], triangle_points[:, 1], 'b-', marker='o')

# Annotate vertices
ax.text(A[0], A[1], 'A', fontsize=12, ha='right', va='bottom')
ax.text(B[0], B[1], 'B', fontsize=12, ha='right', va='top')
ax.text(C[0], C[1], 'C', fontsize=12, ha='left', va='top')
\end{lstlisting}
\end{frame}

% Python code part 3
\begin{frame}[fragile]{Python Code: Final Touches and Save}
\begin{lstlisting}[language=Python]
# Formatting and labels
ax.set_aspect('equal', 'box')
ax.grid(True, linestyle='--', alpha=0.6)
ax.set_xlabel('x (cm)')
ax.set_ylabel('y (cm)')
ax.set_title('Triangle ABC: BC=5 cm, ∠B=60°, AB+AC=7.5 cm')

# Axis limits with padding
padding = 1
min_x, max_x = min(A[0], B[0], C[0]) - padding, max(A[0], B[0], C[0]) + padding
min_y, max_y = min(A[1], B[1], C[1]) - padding, max(A[1], B[1], C[1]) + padding
ax.set_xlim(min_x, max_x)
ax.set_ylim(min_y, max_y)

# Save and display
plt.savefig('triangle_plot.png', dpi=300)
plt.show()
\end{lstlisting}
\end{frame}
\end{document}