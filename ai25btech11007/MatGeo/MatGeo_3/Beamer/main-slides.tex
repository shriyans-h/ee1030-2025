\documentclass{beamer}
\usepackage[utf8]{inputenc}

\usetheme{Madrid}
\usecolortheme{default}
\usepackage{amsmath,amssymb,amsfonts,amsthm}
\usepackage{txfonts}
\usepackage{tkz-euclide}
\usepackage{listings}
\usepackage{adjustbox}
\usepackage{array}
\usepackage{tabularx}
\usepackage{gvv}
\usepackage{lmodern}
\usepackage{circuitikz}
\usepackage{tikz}
\usepackage{graphicx}

\setbeamertemplate{page number in head/foot}[totalframenumber]

\usepackage{tcolorbox}
\tcbuselibrary{minted,breakable,xparse,skins}



\definecolor{bg}{gray}{0.95}
\DeclareTCBListing{mintedbox}{O{}m!O{}}{%
  breakable=true,
  listing engine=minted,
  listing only,
  minted language=#2,
  minted style=default,
  minted options={%
    linenos,
    gobble=0,
    breaklines=true,
    breakafter=,,
    fontsize=\small,
    numbersep=8pt,
    #1},
  boxsep=0pt,
  left skip=0pt,
  right skip=0pt,
  left=25pt,
  right=0pt,
  top=3pt,
  bottom=3pt,
  arc=5pt,
  leftrule=0pt,
  rightrule=0pt,
  bottomrule=2pt,
  toprule=2pt,
  colback=bg,
  colframe=orange!70,
  enhanced,
  overlay={%
    \begin{tcbclipinterior}
    \fill[orange!20!white] (frame.south west) rectangle ([xshift=20pt]frame.north west);
    \end{tcbclipinterior}},
  #3,
}
\lstset{
    language=C,
    basicstyle=\ttfamily\small,
    keywordstyle=\color{blue},
    stringstyle=\color{orange},
    commentstyle=\color{green!60!black},
    numbers=left,
    numberstyle=\tiny\color{gray},
    breaklines=true,
    showstringspaces=false,
}
\title 
{MatGeo Assignment 1.11.9}

\author
{AI25BTECH11007}
\begin{document}

\frame{\titlepage}
\begin{frame}{Question}
If
\[
\vec{a} = \hat{i} - 7\hat{j} + 7\hat{k}
\quad \text{and} \quad
\vec{b} = 3\hat{i} - 2\hat{j} + 2\hat{k},
\]
find a unit vector perpendicular to both the vectors $\vec{a}$ and $\vec{b}$.\\
\end{frame}
\begin{frame}{Solution}

We need a vector $\vec{n}$ such that
\begin{equation}
\vec{n} \cdot \vec{a} = 0, \quad \vec{n} \cdot \vec{b} = 0
\end{equation}
where
\begin{equation}
\vec{a} = \myvec{1 \\ -7 \\ 7}, \quad 
\vec{b} = \myvec{3 \\ -2 \\ 2}.
\end{equation}

Let
\begin{equation}
\vec{n} = \myvec{x \\ y \\ z}.
\end{equation}

Orthogonality conditions,
\begin{align}
\vec{n} \cdot \vec{a} &= x - 7y + 7z = 0 \label{eq:1} \\
\vec{n} \cdot \vec{b} &= 3x - 2y + 2z = 0 \label{eq:2}
\end{align}
\end{frame}
\begin{frame}
    

Solve equations,

From \eqref{eq:1},
\begin{equation}
x = 7y - 7z. \label{eq:3}
\end{equation}

Substitute \eqref{eq:3} into \eqref{eq:2}:
\begin{align}
3(7y - 7z) - 2y + 2z &= 0 \\
21y - 21z - 2y + 2z &= 0 \\
19y - 19z &= 0 \\
y &= z. \label{eq:4}
\end{align}

From \eqref{eq:3} and \eqref{eq:4}:
\begin{equation}
x = 7y - 7y = 0.
\end{equation}

Thus,
\begin{equation}
\vec{n} = \myvec{0 \\ y \\ y} = y \myvec{0 \\ 1 \\ 1}.
\end{equation}
\end{frame}
\begin{frame}
Normalize,
\begin{equation}
\hat{n} = \frac{\myvec{0 \\ 1 \\ 1}}{\sqrt{0^2 + 1^2 + 1^2}}
= \myvec{0 \\ \tfrac{1}{\sqrt{2}} \\ \tfrac{1}{\sqrt{2}}}.
\end{equation}

Hence, a unit vector perpendicular to both $\vec{a}$ and $\vec{b}$ is
\begin{equation}
\hat{n} = \frac{1}{\sqrt{2}}(\hat{j} + \hat{k}),
\end{equation}
or its negative.
\end{frame}
\begin{frame}{Plot}
    \begin{figure}[H]
    \centering
    \includegraphics[width=0.75\columnwidth]{figs/image.png}
    \caption{Image Visual}
    \label{fig:figs/image.png}
    \end{figure}
\end{frame}

\begin{frame}{Conclusion}
    Therefore, a unit vector perpendicular to both $\vec{a}$ and $\vec{b}$ is
\[
\hat{n} = \frac{1}{\sqrt{2}}(\hat{j} + \hat{k}),
\]
or its negative.
\end{frame}
\end{document}