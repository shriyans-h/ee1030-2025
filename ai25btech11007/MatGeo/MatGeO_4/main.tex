\let\negmedspace\undefined
\let\negthickspace\undefined
\documentclass[journal]{IEEEtran}
\usepackage[a5paper, margin=10mm, onecolumn]{geometry}
%\usepackage{lmodern} % Ensure lmodern is loaded for pdflatex
\usepackage{tfrupee} % Include tfrupee package

\setlength{\headheight}{1cm} % Set the height of the header box
\setlength{\headsep}{0mm}     % Set the distance between the header box and the top of the text

\usepackage{gvv-book}
\usepackage{gvv}
\usepackage{cite}
\usepackage{amsmath,amssymb,amsfonts,amsthm}
\usepackage{algorithmic}
\usepackage{graphicx}
\usepackage{textcomp}
\usepackage{xcolor}
\usepackage{txfonts}
\usepackage{listings}
\usepackage{enumitem}
\usepackage{mathtools}
\usepackage{gensymb}
\usepackage{comment}
\usepackage[breaklinks=true]{hyperref}
\usepackage{tkz-euclide} 
\usepackage{listings}
% \usepackage{gvv}                                        
\def\inputGnumericTable{}                                 
\usepackage[latin1]{inputenc}                                
\usepackage{color}                                            
\usepackage{array}                                            
\usepackage{longtable}                                       
\usepackage{calc}                                             
\usepackage{multirow} 
\usepackage{hhline}                                           
\usepackage{ifthen}                                           
\usepackage{lscape}
\usepackage{circuitikz}
\tikzstyle{block} = [rectangle, draw, fill=blue!20, 
    text width=4em, text centered, rounded corners, minimum height=3em]
\tikzstyle{sum} = [draw, fill=blue!10, circle, minimum size=1cm, node distance=1.5cm]
\tikzstyle{input} = [coordinate]
\tikzstyle{output} = [coordinate]

\begin{document}
\bibliographystyle{IEEEtran}
\vspace{3cm}

\title{MatGeo Assignment 2.6.13}
\author{AI25BTECH11007}
 \maketitle
% \newpage
% \bigskip
{\let\newpage\relax\maketitle}

\renewcommand{\thefigure}{\theenumi}
\renewcommand{\thetable}{\theenumi}
\setlength{\intextsep}{10pt} % Space between text and floats


\numberwithin{equation}{enumi}
\numberwithin{figure}{enumi}
\renewcommand{\thetable}{\theenumi}
\textbf{Question:}\\
Using vectors, find the area of $\triangle ABC$ with vertices A(1, 2, 3),B(2, -1, 4) and C(4, 5, -1).

\textbf{Solution:}\\
 Compute vectors $\vec{B}-\vec{A}$ and $\vec{C}-\vec{A}$:\\
 
\begin{align}
\vec{B}-\vec{A} = \myvec{1\\-3\\1}, \label{eq:AB}\\
\vec{C}-\vec{A} = \myvec{3\\3\\-4}. \label{eq:AC}
\end{align}

Recall the identity:
\begin{equation}
\|\vec{B}-\vec{A}\\ \times \vec{C}-\vec{A}\|^2
= \|\vec{B}-\vec{A}\|^2 \|\vec{C}-\vec{A}\|^2 - ((\vec{B}-\vec{A})^T(\vec{C}-\vec{A}))^2. \label{eq:identity}
\end{equation}

Compute the inner products:
\begin{align}
\|\vec{B}-\vec{A}\|^2 &= (\vec{B}-\vec{A})^T (\vec{B}-\vec{A}) = 1^2 + (-3)^2 + 1^2 = 11, \label{eq:ABsq}\\
\|\vec{C}-\vec{A}\|^2 &= (\vec{C}-\vec{A})^T (\vec{C}-\vec{A}) = 3^2 + 3^2 + (-4)^2 = 34, \label{eq:ACsq}\\
(\vec{B}-\vec{A})^T(\vec{C}-\vec{A}) &= (1)(3) + (-3)(3) + (1)(-4) = -10. \label{eq:dot}
\end{align}

Substitute into \eqref{eq:identity}:
\begin{align}
\|\vec{B}-\vec{A} \times \vec{C}-\vec{A}\|^2 &= (11)(34) - (-10)^2 \notag\\
&= 374 - 100 \notag\\
&= 274. \label{eq:crosssq}
\end{align}

 Hence
\begin{equation}
\|\vec{B}-\vec{A}\times \vec{C}-\vec{A}\| = \sqrt{274}. \label{eq:crossnorm}
\end{equation}

 The area of $\triangle ABC$ is
\begin{equation}
\text{Area}(\triangle ABC) 
= \tfrac{1}{2}\|\vec{B}-\vec{A}\times\vec{C}-\vec{A}\|
= \frac{\sqrt{274}}{2}. \label{eq:area}
\end{equation}
\begin{figure}[H]
    \centering
    \includegraphics[width=0.5\linewidth]{figs/image.png}
    \caption{Image Visual}
    \label{fig:figs/image.png}
\end{figure}
\end{document}















