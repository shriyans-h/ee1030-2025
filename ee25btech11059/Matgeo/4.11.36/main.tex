\let\negmedspace\undefined
\let\negthickspace\undefined
\documentclass[journal]{IEEEtran}
\usepackage[a5paper, margin=10mm, onecolumn]{geometry}
\usepackage{lmodern} 
\usepackage{tfrupee} 
\setlength{\headheight}{1cm}
\setlength{\headsep}{0mm}   

\usepackage{gvv-book}
\usepackage{gvv}
\usepackage{cite}
\usepackage{amsmath,amssymb,amsfonts,amsthm}
\usepackage{algorithmic}
\usepackage{graphicx}
\usepackage{textcomp}
\usepackage{xcolor}
\usepackage{txfonts}
\usepackage{listings}
\usepackage{enumitem}
\usepackage{mathtools}
\usepackage{gensymb}
\usepackage{comment}
\usepackage[breaklinks=true]{hyperref}
\usepackage{tkz-euclide} 
\usepackage{listings}                             
\def\inputGnumericTable{}                                 
\usepackage[latin1]{inputenc}                                
\usepackage{color}                                            
\usepackage{array}                                            
\usepackage{longtable}                                       
\usepackage{calc}                                             
\usepackage{multirow}                                         
\usepackage{hhline}                                           
\usepackage{ifthen}                                           
\usepackage{lscape}
\usepackage{xparse}

\bibliographystyle{IEEEtran}

\title{4.11.36}
\author{EE25BTECH11059 - Vaishnavi Ramkrishna Anantheertha}

\begin{document}
\maketitle

\renewcommand{\thefigure}{\theenumi}
\renewcommand{\thetable}{\theenumi}

\numberwithin{equation}{enumi}
\numberwithin{figure}{enumi} 

\textbf{Question}:
Find the coordinates of the point where the line through the points $(3, -4, -5)$ and $(2, -3, 1)$ crosses the plane determined by the points $(1, 2, 3)$, $(4, 2, -3)$ and $(0, 4, 3)$.

\textbf{Solution }
\begin{table}[H]    
  \centering
  \begin{tabular}[12pt]{ |c| c|}
    \hline
    \textbf{Name} & \textbf{Point}\\ 
    \hline
	Point A &\myvec{h \\ k}\\
    \hline 
 Point B &\myvec{x1 \\ y1}\\
    \hline
	  Point R &\myvec{x2 \\ y2}\\
    \hline
    
    \end{tabular}

  \caption{Variables Used}
  \label{tab:4.7.56}
\end{table}

Let eq of plane be
\begin{align}
    \vec{n^T}\vec{x}=1
\end{align}
As $\vec{P},\vec{Q},\vec{R}$ lie on the plane
\begin{align}
 \vec{n^T}
 \myvec{1
        \\
        2
        \\
        3
 }=1\\
  \vec{n^T}
 \myvec{4
        \\
        2
        \\
        -3
 }=1\\
  \vec{n^T}
 \myvec{0
        \\
        4
        \\
        3
 }=1
 \end{align}
From eq $(0.2),(0.3) and (0.4)$
\begin{align}
\begin{pmatrix}
\begin{array}{ccc|c}
1 & 2 & 3 & 1\\
4 & 2 & -3 & 1\\
0 & 4 & 3 & 1
\end{array}
\end{pmatrix}
&\xrightarrow{R_2 \to R_2 - 4R_1}
\begin{pmatrix}
\begin{array}{ccc|c}
1 & 2 & 3 & 1\\
0 & -6 & -15 & -3\\
0 & 4 & 3 & 1
\end{array}
\end{pmatrix} \\[4pt]
&\xrightarrow{R_2 \to -\tfrac{1}{3}R_2}
\begin{pmatrix}
\begin{array}{ccc|c}
1 & 2 & 3 & 1\\
0 & 2 & 5 & 1\\
0 & 4 & 3 & 1
\end{array}
\end{pmatrix} \\[4pt]
&\xrightarrow{R_3 \to R_3 - 2R_2}
\begin{pmatrix}
\begin{array}{ccc|c}
1 & 2 & 3 & 1\\
0 & 2 & 5 & 1\\
0 & 0 & -7 & -1
\end{array}
\end{pmatrix} \\[4pt]
&\xrightarrow{R_3 \to -\tfrac{1}{7}R_3}
\begin{pmatrix}
\begin{array}{ccc|c}
1 & 2 & 3 & 1\\
0 & 2 & 5 & 1\\
0 & 0 & 1 & \tfrac{1}{7}
\end{array}
\end{pmatrix} \\[4pt]
&\xrightarrow{R_2 \to R_2 - 5R_3}
\begin{pmatrix}
\begin{array}{ccc|c}
1 & 2 & 3 & 1\\
0 & 2 & 0 & \tfrac{2}{7}\\
0 & 0 & 1 & \tfrac{1}{7}
\end{array}
\end{pmatrix} \\[4pt]
&\xrightarrow{R_2 \to \tfrac{1}{2}R_2}
\begin{pmatrix}
\begin{array}{ccc|c}
1 & 2 & 3 & 1\\
0 & 1 & 0 & \tfrac{1}{7}\\
0 & 0 & 1 & \tfrac{1}{7}
\end{array}
\end{pmatrix} \\[4pt]
&\xrightarrow{R_1 \to R_1 - 3R_3}
\begin{pmatrix}
\begin{array}{ccc|c}
1 & 2 & 0 & \tfrac{4}{7}\\
0 & 1 & 0 & \tfrac{1}{7}\\
0 & 0 & 1 & \tfrac{1}{7}
\end{array}
\end{pmatrix} \\[4pt]
&\xrightarrow{R_1 \to R_1 - 2R_2}
\begin{pmatrix}
\begin{array}{ccc|c}
1 & 0 & 0 & \tfrac{2}{7}\\
0 & 1 & 0 & \tfrac{1}{7}\\
0 & 0 & 1 & \tfrac{1}{7}
\end{array}
\end{pmatrix}
\end{align}

\begin{align}
\vec{n}=
\myvec{\frac{2}{7}
       \\
       \frac{1}{7}
       \\
       \frac{1}{7}
}
\end{align}





hence eq of plane is
\begin{align}
 \myvec{\tfrac{2}{7} & \tfrac{1}{7} & \tfrac{1}{7}} 
 \vec{x}
 =1
\end{align}   


let a point on line $\vec{A}\vec{B}$ be 
\begin{align}
\vec{c}=k\vec{A}+(1-k)\vec{B}\\
\vec{c}=\myvec{2+k
               \\
              -3-k
              \\
              1-6k
 }\\
\myvec{\tfrac{2}{7} & \tfrac{1}{7} & \tfrac{1}{7}} 
\myvec{2+k
               \\
              -3-k
              \\
              1-6k
              }
 =1
 4+2k-3-k+1-6k=7\\
2-5k=7\\
k=-1
\end{align}
The point $\vec{c}$ is
\begin{align}
    \vec{c}=\myvec{1
                   \\
                   -2
                   \\
                   7
             }
\end{align}
Refer to Figure

\begin{figure}[H]
\begin{center}
\includegraphics[width=0.6\columnwidth]{figs/graph8.png}
\end{center}
\caption{}
\label{fig:Fig}
\end{figure}
\end{document}