\let\negmedspace\undefined
\let\negthickspace\undefined
\documentclass[journal]{IEEEtran}
\usepackage[a5paper, margin=10mm, onecolumn]{geometry}
\usepackage{lmodern} 
\usepackage{tfrupee} 
\setlength{\headheight}{1cm}
\setlength{\headsep}{0mm}   

\usepackage{gvv-book}
\usepackage{gvv}
\usepackage{cite}
\usepackage{amsmath,amssymb,amsfonts,amsthm}
\usepackage{algorithmic}
\usepackage{graphicx}
\usepackage{textcomp}
\usepackage{xcolor}
\usepackage{txfonts}
\usepackage{listings}
\usepackage{enumitem}
\usepackage{mathtools}
\usepackage{gensymb}
\usepackage{comment}
\usepackage[breaklinks=true]{hyperref}
\usepackage{tkz-euclide} 
\usepackage{listings}                             
\def\inputGnumericTable{}                                 
\usepackage[latin1]{inputenc}                                
\usepackage{color}                                            
\usepackage{array}                                            
\usepackage{longtable}                                       
\usepackage{calc}                                             
\usepackage{multirow}                                         
\usepackage{hhline}                                           
\usepackage{ifthen}                                           
\usepackage{lscape}
\usepackage{xparse}

\bibliographystyle{IEEEtran}

\title{5.5.11}
\author{EE25BTECH11059 - Vaishnavi Ramkrishna Anantheertha}

\begin{document}
\maketitle

\renewcommand{\thefigure}{\theenumi}
\renewcommand{\thetable}{\theenumi}

\numberwithin{equation}{enumi}
\numberwithin{figure}{enumi} 

\textbf{Question}:
Find inverse of the following matrix,using elementary transformation\\
A=\myvec{2 & 0 & -1
          \\
        5 & 1 & 0
         \\
        0 & 1 & 3
}


\textbf{Solution }


Construct the augmented matrix of A and I
\begin{align}
&\left(
\begin{array}{ccc|ccc}
2 & 0 & -1 & 1 & 0 & 0 \\
5 & 1 & 0 & 0 & 1 & 0 \\
0 & 1 & 3 & 0 & 0 & 1
\end{array}
\right)
\xrightarrow{R_1 \gets \frac{1}{2}R_1}
\left(
\begin{array}{ccc|ccc}
1 & 0 & -\frac{1}{2} & \frac{1}{2} & 0 & 0 \\
5 & 1 & 0 & 0 & 1 & 0 \\
0 & 1 & 3 & 0 & 0 & 1
\end{array}
\right) \\[12pt]
&\xrightarrow{R_2 \gets R_2 - 5R_1}
\left(
\begin{array}{ccc|ccc}
1 & 0 & -\frac{1}{2} & \frac{1}{2} & 0 & 0 \\
0 & 1 & \frac{5}{2} & -\frac{5}{2} & 1 & 0 \\
0 & 1 & 3 & 0 & 0 & 1
\end{array}
\right) \\[12pt]
&\xrightarrow{R_3 \gets R_3 - R_2}
\left(
\begin{array}{ccc|ccc}
1 & 0 & -\frac{1}{2} & \frac{1}{2} & 0 & 0 \\
0 & 1 & \frac{5}{2} & -\frac{5}{2} & 1 & 0 \\
0 & 0 & \frac{1}{2} & \frac{5}{2} & -1 & 1
\end{array}
\right) \\[12pt]
&\xrightarrow{R_3 \gets 2R_3}
\left(
\begin{array}{ccc|ccc}
1 & 0 & -\frac{1}{2} & \frac{1}{2} & 0 & 0 \\
0 & 1 & \frac{5}{2} & -\frac{5}{2} & 1 & 0 \\
0 & 0 & 1 & 5 & -2 & 2
\end{array}
\right) \\[12pt]
&\xrightarrow{R_2 \gets R_2 - \frac{5}{2}R_3}
\left(
\begin{array}{ccc|ccc}
1 & 0 & -\frac{1}{2} & \frac{1}{2} & 0 & 0 \\
0 & 1 & 0 & -15 & 6 & -5 \\
0 & 0 & 1 & 5 & -2 & 2
\end{array}
\right) \\[12pt]
&\xrightarrow{R_1 \gets R_1 + \frac{1}{2}R_3}
\left(
\begin{array}{ccc|ccc}
1 & 0 & 0 & 3 & -1 & 1 \\
0 & 1 & 0 & -15 & 6 & -5 \\
0 & 0 & 1 & 5 & -2 & 2
\end{array}
\right)
\end{align}
\begin{align}
A^{-1}
   =\myvec{3 & -1 & 1
          \\
        -15 & 6 & -5
         \\
        5 & -2 & 2
}
\end{align}
\end{document}




