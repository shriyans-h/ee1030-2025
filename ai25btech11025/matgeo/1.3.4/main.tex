\let\negmedspace\undefined
\let\negthickspace\undefined
\documentclass[journal]{IEEEtran}
%\usepackage{lmodern} % Ensure lmodern is loaded for pdflatex
%\usepackage{tfrupee} % Include tfrupee package

\setlength{\headheight}{1cm} % Set the height of the header box
\setlength{\headsep}{0mm}     % Set the distance between the header box and the top of the text

\usepackage{gvv-book}
\usepackage{gvv}
\usepackage{cite}
\usepackage{amsmath,amssymb,amsfonts,amsthm}
\usepackage{algorithmic}
\usepackage{graphicx}
\usepackage{textcomp}
\usepackage{xcolor}
\usepackage{txfonts}
\usepackage{listings}
\usepackage{enumitem}
\usepackage{mathtools}
\usepackage{gensymb}
\usepackage{comment}
\usepackage[breaklinks=true]{hyperref}
\usepackage{tkz-euclide} 
\usepackage{listings}
% \usepackage{gvv}                                        
\def\inputGnumericTable{}                                 
\usepackage[latin1]{inputenc}                                
\usepackage{color}                                            
\usepackage{array}  
\usepackage{longtable}                                       
\usepackage{calc}                                             
\usepackage{multirow}                                         
\usepackage{hhline}                                           
\usepackage{ifthen}                                           
\usepackage{lscape}
\usepackage{circuitikz}
\tikzstyle{block} = [rectangle, draw, fill=blue!20, 
    text width=4em, text centered, rounded corners, minimum height=3em]
\tikzstyle{sum} = [draw, fill=blue!10, circle, minimum size=1cm, node distance=1.5cm]
\tikzstyle{input} = [coordinate]
\tikzstyle{output} = [coordinate]


\begin{document}



\bibliographystyle{IEEEtran}
\vspace{3cm}

\title{1.3.4}
\author{AI25BTECH11025-R Nikhil}
 \maketitle
% \newpage
% \bigskip
{\let\newpage\relax\maketitle}

\renewcommand{\thefigure}{\theenumi}
\renewcommand{\thetable}{\theenumi}
\setlength{\intextsep}{10pt} % Space between text and floats


\numberwithin{equation}{enumi}
\numberwithin{figure}{enumi}
\renewcommand{\thetable}{\theenumi}  
\textbf{Question 1.3.4} \\
If $ A(1, 3) $, $ B(4, 2) $, $ C(x, 5) $, and $ D(x, 4) $ are the vertices of a parallelogram $ABCD$, then the value of $x$ is \underline{\hspace{2cm}}. \hfill (10, 2012)

\vspace{1em}
\textbf{Solution:}

In a parallelogram, opposite sides are equal and parallel. Since $ABCD$ is a parallelogram, vectors $ \vec{AB} $ and $ \vec{CD} $ must be equal.

\begin{align}
\vec{AB} = \vec{B} -\vec{A} = \myvec{4 - 1, 2 - 3} = \myvec{3, -1}
\end{align}
\begin{align}
\vec{CD} = \vec{D} - \vec{C} = \myvec{x - x, 4 - 5} = \myvec{0, -1}
\end{align}

Clearly, $ \vec{AB} \neq \vec{CD} $, so let's try using diagonals. In a parallelogram, the diagonals bisect each other.

Midpoint of diagonal $AC$:
\begin{align}
\Vec{M_{AC}} = \myvec{\frac{1 + x}{2} \\ \frac{3 + 5}{2} } = \myvec{ \frac{1 + x}{2}\\ 4 }
\end{align}

Midpoint of diagonal $BD$:
\begin{align}
\vec{M_{BD}} = \myvec{\frac{4 + x}{2} \\ \frac{2 + 4}{2} } = \myvec{ \frac{4 + x}{2}\\ 3 }
\end{align}

Equating midpoints:
\begin{align}
\frac{1 + x}{2} = \frac{4 + x}{2} \quad \text{and} \quad 4 = 3
\end{align}

The second equation is false, so diagonals do not bisect each other. Let's try using opposite sides again, but this time equating $ \vec{AD} $ and $ \vec{BC} $:

\begin{align}
\vec{AD} = \vec{D} -\vec{A} = \begin{myvec}
    {x-1 \\ 4-3 } \end{myvec} =\begin{myvec}
        {x-1 \\ 1}
    \end{myvec}
 \end{align}   
 \begin{align}
\vec{BC} =\vec{C} -\vec{B} = \begin{myvec}
    {x - 4 \\ 5 - 2} \end{myvec} =\begin{myvec}
         {x - 4 \\ 3}
     \end{myvec}
\end{align}

Equating vectors:
\begin{align}
x - 1 = x - 4 \quad \text{and} \quad 1 = 3
\end{align}

Again, contradiction. So let's try using the property that opposite sides are equal in length.

\begin{align}
\text{Length of } \vec{AD}: \quad |AD| &= \sqrt{(x - 1)^2 + (4 - 3)^2} = \sqrt{(x - 1)^2 + 1} \\
\text{Length of } \vec{BC}: \quad |BC| &= \sqrt{(x - 4)^2 + (5 - 2)^2} = \sqrt{(x - 4)^2 + 9}
\end{align}

Equating lengths:
\begin{align}
\sqrt{(x - 1)^2 + 1} = \sqrt{(x - 4)^2 + 9}
\end{align}

Squaring both sides:
\begin{align}
(x - 1)^2 + 1 = (x - 4)^2 + 9
\end{align}
\begin{align}
x^2 - 2x + 1 + 1 = x^2 - 8x + 16 + 9
\end{align}
\begin{align}
x^2 - 2x + 2 = x^2 - 8x + 25
\end{align}

Subtract $x^2$ from both sides:

\begin{align}
-2x + 2 = -8x + 25
\end{align}

\begin{align}
6x = 23 \Rightarrow x = \frac{23}{6}
\end{align}

\textbf{Answer:} $ \boxed{\frac{23}{6}} $

\begin{figure}[h!]
    \centering
    \includegraphics[height=0.6\textheight, keepaspectratio]{figs/plotc.png}
    \label{figure_1}
\end{figure}

\end{document}
