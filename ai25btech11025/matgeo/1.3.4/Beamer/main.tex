\documentclass{beamer}
\usepackage[utf8]{inputenc}

\usetheme{Madrid}
\usecolortheme{default}
\usepackage{amsmath,amssymb,amsfonts,amsthm}
\usepackage{txfonts}
\usepackage{tkz-euclide}
\usepackage{listings}
\usepackage{adjustbox}
\usepackage{array}
\usepackage{tabularx}
\usepackage{gvv}
\usepackage{lmodern}
\usepackage{circuitikz}
\usepackage{tikz}
\usepackage{graphicx}

\setbeamertemplate{page number in head/foot}[totalframenumber]

\usepackage{tcolorbox}
\tcbuselibrary{minted,breakable,xparse,skins}

\definecolor{bg}{gray}{0.95}
\DeclareTCBListing{mintedbox}{O{}m!O{}}{%
  breakable=true,
  listing engine=minted,
  listing only,
  minted language=#2,
  minted style=default,
  minted options={%
    linenos,
    gobble=0,
    breaklines=true,
    breakafter=,,
    fontsize=\small,
    numbersep=8pt,
    #1},
  boxsep=0pt,
  left skip=0pt,
  right skip=0pt,
  left=25pt,
  right=0pt,
  top=3pt,
  bottom=3pt,
  arc=5pt,
  leftrule=0pt,
  rightrule=0pt,
  bottomrule=2pt,
  toprule=2pt,
  colback=bg,
  colframe=orange!70,
  enhanced,
  overlay={%
    \begin{tcbclipinterior}
    \fill[orange!20!white] (frame.south west) rectangle ([xshift=20pt]frame.north west);
    \end{tcbclipinterior}},
  #3,
}
\lstset{
    language=C,
    basicstyle=\ttfamily\small,
    keywordstyle=\color{blue},
    stringstyle=\color{orange},
    commentstyle=\color{green!60!black},
    numbers=left,
    numberstyle=\tiny\color{gray},
    breaklines=true,
    showstringspaces=false,
}
%------------------------------------------------------------
%This block of code defines the information to appear in the
%Title page
\title %optional
{1.3.4}
%\subtitle{A short story}

\author % (optional)
{R Nikhil-AI25BTECH11025}

\begin{document}

\frame{\titlepage}
\begin{frame}{Question}
If $ A(1, 3) $, $ B(4, 2) $, $ C(x, 5) $, and $ D(x, 4) $ are the vertices of a parallelogram $ABCD$, then the value of $x$ is \underline{\hspace{2cm}}. \hfill (10, 2012)
\end{frame}

\begin{frame}{Theoretical Solution}
In a parallelogram, opposite sides are equal and parallel. Since $ABCD$ is a parallelogram, vectors $ \vec{AB} $ and $ \vec{CD} $ must be equal.

\begin{align}
\vec{B} -\vec{A} = \myvec{4 - 1\\ 2 - 3} = \myvec{3, -1}
\end{align}
\begin{align}
\vec{D} - \vec{C} = \myvec{x - x\\ 4 - 5} = \myvec{0, -1}
\end{align}

	Clearly, $ \vec{B}-\vec{A} \neq \vec{D}-\vec{C} $, so let's try using diagonals. In a parallelogram, the diagonals bisect each other.

Midpoint of diagonal $AC$:
\begin{align}
\myvec{\frac{1 + x}{2} \\ \frac{3 + 5}{2} } = \myvec{ \frac{1 + x}{2}\\ 4 }
\end{align}
\end{frame}


\begin{frame}{Theoretical Solution}
Midpoint of diagonal $BD$
\begin{align}
\myvec{\frac{4 + x}{2} \\ \frac{2 + 4}{2} } = \myvec{ \frac{4 + x}{2}\\ 3 }
\end{align}

Equating midpoints:
\begin{align}
\frac{1 + x}{2} = \frac{4 + x}{2} \quad \text{and} \quad 4 = 3
\end{align}
\end{frame}
\begin{frame}{Theoretical Solution}
The second equation is false, so diagonals do not bisect each other. Let's try using opposite sides again, but this time equating $ \vec{AD} $ and $ \vec{BC} $:

\begin{align}
\vec{D} -\vec{A} = \begin{myvec}
    {x-1 \\ 4-3 } \end{myvec} =\begin{myvec}
        {x-1 \\ 1}
    \end{myvec}
 \end{align}   
 \begin{align}
\vec{C} -\vec{B} = \begin{myvec}
    {x - 4 \\ 5 - 2} \end{myvec} =\begin{myvec}
         {x - 4 \\ 3}
     \end{myvec}
\end{align}

Equating vectors:
\begin{align}
x - 1 = x - 4 \quad \text{and} \quad 1 = 3
\end{align}

Again, contradiction. So let's try using the property that opposite sides are equal in length.
\end{frame}



\begin{frame}{Theoretical Solution}
\begin{align}
	\text{Length of } \vec{D}-\vec{A}: \quad |AD| &= \sqrt{(x - 1)^2 + (4 - 3)^2} = \sqrt{(x - 1)^2 + 1} \\
	\text{Length of } \vec{C}-\vec{B}: \quad |BC| &= \sqrt{(x - 4)^2 + (5 - 2)^2} = \sqrt{(x - 4)^2 + 9}
\end{align}

Equating lengths:
\begin{align}
\sqrt{(x - 1)^2 + 1} = \sqrt{(x - 4)^2 + 9}
\end{align}

Squaring both sides:
\begin{align}
(x - 1)^2 + 1 = (x - 4)^2 + 9
\end{align}
\begin{align}
x^2 - 2x + 1 + 1 = x^2 - 8x + 16 + 9
\end{align}
\begin{align}
x^2 - 2x + 2 = x^2 - 8x + 25
\end{align}

\end{frame}

\begin{frame}{Theoretical Solution}
Subtract $x^2$ from both sides:

\begin{align}
-2x + 2 = -8x + 25
\end{align}

\begin{align}
6x = 23 \Rightarrow x = \frac{23}{6}
\end{align}

\textbf{Answer:} $ \boxed{\frac{23}{6}} $
\end{frame}

\begin{frame}[fragile]
\frametitle{Main C Code}
   \begin{lstlisting}
include <stdio.h>

// Function declarations (prototypes)
double dx_from_abc(double ax, double ay, double bx, double by, double cx, double cy);
double dy_from_abc(double ax, double ay, double bx, double by, double cx, double cy);
int write_points_file(const char *filepath,
                      double ax, double ay,
                      double bx, double by,
                      double cx, double cy);

   int main(void) {

   \end{lstlisting}
	\end{frame}
\begin{frame}[fragile]
\frametitle{Main C Code}
   \begin{lstlisting}
    // Given A(1,3), B(-1,2), C(2,5)
    double ax=1, ay=3, bx=-1, by=2, cx=2, cy=5;

    double dx = dx_from_abc(ax, ay, bx, by, cx, cy);
    double dy = dy_from_abc(ax, ay, bx, by, cx, cy);

    printf("Computed x for D: %.10g\n", dx);
    printf("Computed y for D (for consistency): %.10g\n", dy);

    // Write coordinates to a file
    if (write_points_file("points.dat", ax, ay, bx, by, cx, cy) != 0) {
        fprintf(stderr, "Failed to write points.dat\n");
        return 1;
    }
    printf("Wrote points to points.dat\n");

    return 0;
}
   \end{lstlisting}
\end{frame}

\begin{frame}[fragile]
\frametitle{C Function}
   \begin{lstlisting}
   #include <stdio.h>

// Function to calculate Dx (x-coordinate of D)
double dx_from_abc(double ax, double ay, double bx, double by, double cx, double cy) {
    (void)ay; (void)by; (void)cy; // unused
    return ax + cx - bx;
}
   \end{lstlisting}
\end{frame}

\begin{frame}[fragile]
\frametitle{C Function}
     \begin{lstlisting}

// Function to calculate Dy (y-coordinate of D)
double dy_from_abc(double ax, double ay, double bx, double by, double cx, double cy) {
    (void)ax; (void)bx; (void)cx; // unused
    return ay + cy - by;
}

// Function to write points into a file
int write_points_file(const char *filepath,
                      double ax, double ay,
                      double bx, double by,
                      double cx, double cy) {
    double dx = dx_from_abc(ax, ay, bx, by, cx, cy);
    double dy = dy_from_abc(ax, ay, bx, by, cx, cy);
     \end{lstlisting}
     \end{frame}

\begin{frame}[fragile]
\frametitle{C Function}
   \begin{lstlisting}

 FILE *fp = fopen(filepath, "w");
    if (!fp) return 1;
    fprintf(fp, "A %.10g %.10g\n", ax, ay);
    fprintf(fp, "B %.10g %.10g\n", bx, by);
    fprintf(fp, "C %.10g %.10g\n", cx, cy);
    fprintf(fp, "D %.10g %.10g\n", dx, dy);
    fclose(fp);
    return 0;
}

   \end{lstlisting}
\end{frame}

\begin{frame}[fragile]
\frametitle{Python Code}
   \begin{lstlisting}

   import ctypes
import pandas as pd
import matplotlib.pyplot as plt

# Load the shared library
lib = ctypes.CDLL("./libparallelogram.so")
lib.dx_from_abc.argtypes = [ctypes.c_double]*6
lib.dx_from_abc.restype = ctypes.c_double
lib.dy_from_abc.argtypes = [ctypes.c_double]*6
lib.dy_from_abc.restype = ctypes.c_double

# Given points
ax, ay = 1.0, 3.0
bx, by = -1.0, 2.0
cx, cy = 2.0, 5.0

\end{lstlisting}
\end{frame}

\begin{frame}[fragile]
\frametitle{Python Code}
   \begin{lstlisting}
dx = lib.dx_from_abc(ax, ay, bx, by, cx, cy)
dy = lib.dy_from_abc(ax, ay, bx, by, cx, cy)

print("From Python via .so:")
print("D =", dx, dy)

# Read the points written by C main
df = pd.read_csv("points.dat", sep=r"\s+", header=None, names=["label","x","y"])

# Plot
order = ["A","B","C","D","A"]
xs = [df.loc[df["label"]==lbl,"x"].values[0] for lbl in order]
ys = [df.loc[df["label"]==lbl,"y"].values[0] for lbl in order]
   \end{lstlisting}
\end{frame}
\begin{frame}[fragile]
\frametitle{Python Code}
   \begin{lstlisting}
plt.plot(xs, ys, marker="o")
for lbl in ["A","B","C","D"]:
    x = df.loc[df["label"]==lbl,"x"].values[0]
    y = df.loc[df["label"]==lbl,"y"].values[0]
    plt.text(x, y, f"{lbl}({x:.0f},{y:.0f})")

plt.title("Parallelogram ABCD")
plt.xlabel("x")
plt.ylabel("y")
plt.grid(True)
plt.savefig("/home/r-nikhil/ee1030-2025/ai25btech11025/matgeo/1.3.4/figs/plotc.png")
plt.show()
\end{lstlisting}
\end{frame}

\begin{frame}{Plot}
    \begin{figure}[h!]
    \centering
    \includegraphics[height=0.6\textheight, keepaspectratio]{figs/plotc.png}
    \label{figure_1}
    \end{figure}
\end{frame}

\end{document}
