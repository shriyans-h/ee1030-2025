\documentclass{beamer}
\mode<presentation>
\usepackage{amsmath}
\usepackage{amssymb}
%\usepackage{advdate}
\usepackage{graphicx}
\graphicspath{{../Figs/}}
\usepackage{adjustbox}
\usepackage{subcaption}
\usepackage{enumitem}
\usepackage{multicol}
\usepackage{mathtools}
\usepackage{listings}
\usepackage{url}
\def\UrlBreaks{\do\/\do-}
\usetheme{Boadilla}
\usecolortheme{lily}
\setbeamertemplate{footline}
{
  \leavevmode%
  \hbox{%
  \begin{beamercolorbox}[wd=\paperwidth,ht=2.25ex,dp=1ex,right]{author in head/foot}%
    \insertframenumber{} / \inserttotalframenumber\hspace*{2ex} 
  \end{beamercolorbox}}%
  \vskip0pt%
}
\setbeamertemplate{navigation symbols}{}

\providecommand{\nCr}[2]{\,^{#1}C_{#2}} % nCr
\providecommand{\nPr}[2]{\,^{#1}P_{#2}} % nPr
\providecommand{\mbf}{\mathbf}
\providecommand{\pr}[1]{\ensuremath{\Pr\left(#1\right)}}
\providecommand{\qfunc}[1]{\ensuremath{Q\left(#1\right)}}
\providecommand{\sbrak}[1]{\ensuremath{{}\left[#1\right]}}
\providecommand{\lsbrak}[1]{\ensuremath{{}\left[#1\right.}}
\providecommand{\rsbrak}[1]{\ensuremath{{}\left.#1\right]}}
\providecommand{\brak}[1]{\ensuremath{\left(#1\right)}}
\providecommand{\lbrak}[1]{\ensuremath{\left(#1\right.}}
\providecommand{\rbrak}[1]{\ensuremath{\left.#1\right)}}
\providecommand{\cbrak}[1]{\ensuremath{\left\{#1\right\}}}
\providecommand{\lcbrak}[1]{\ensuremath{\left\{#1\right.}}
\providecommand{\rcbrak}[1]{\ensuremath{\left.#1\right\}}}
\theoremstyle{remark}
\newtheorem{rem}{Remark}
\newcommand{\sgn}{\mathop{\mathrm{sgn}}}
\providecommand{\abs}[1]{\left\vert#1\right\vert}
\providecommand{\res}[1]{\Res\displaylimits_{#1}} 
\providecommand{\norm}[1]{\lVert#1\rVert}
\providecommand{\mtx}[1]{\mathbf{#1}}
\providecommand{\mean}[1]{E\left[ #1 \right]}
\providecommand{\fourier}{\overset{\mathcal{F}}{ \rightleftharpoons}}
%\providecommand{\hilbert}{\overset{\mathcal{H}}{ \rightleftharpoons}}
\providecommand{\system}[1]{\overset{\mathcal{#1}}{ \longleftrightarrow}}
%\providecommand{\system}{\overset{\mathcal{H}}{ \longleftrightarrow}}
	%\newcommand{\solution}[2]{\textbf{Solution:}{#1}}
%\newcommand{\solution}{\noindent \textbf{Solution: }}
\providecommand{\dec}[2]{\ensuremath{\overset{#1}{\underset{#2}{\gtrless}}}}
\newcommand{\myvec}[1]{\ensuremath{\begin{pmatrix}#1\end{pmatrix}}}
\let\vec\mathbf

\lstset{
%language=C,
frame=single, 
breaklines=true,
columns=fullflexible
}

\numberwithin{equation}{section}

\title{1.6.17}
\author{AI25BTECH11002 - Ayush Sunil Labhade}
\begin{document}
\maketitle
		\textbf{Question}:\newline

		Using vectors, find the value of $k$ such that the points $(k , -10, 3)$, $\vec{B}(1, -1, 3)$ and $(3, 5, 3)$ are collinear.

		\textbf{Solution:}
		Given:
		\begin{table}[H]
			\centering
			\begin{tabular}{|c|c|}
\hline
Point & Vector \\
\hline
$\vec{a}$ & $\myvec{k \\ -10 \\ 3}$ \\
\hline
$\vec{b}$ & $\myvec{1 \\ -1 \\ 3}$ \\
\hline
$\vec{c}$ & $\myvec{3 \\ 5 \\ 3}$ \\
\hline
\end{tabular}

			\label{}
			\caption{Given data}
		\end{table}
		Since the points are collinear, we can use 
		\begin{align}
			rank\myvec{\vec{B}-\vec{A} & \vec{C}-\vec{B}} = 1	
		\end{align}
		Therefore,
		\begin{align}
			\myvec{\vec{B}-\vec{A} & \vec{C}-\vec{B}}^T = \myvec{1-k & 3-1\\-1-(-10) & 5-(-1)\\3-3 & 3-3}	
			\\
			\myvec{1-k & 2\\9 & 6\\0 & 0}
			\xRightarrow[]{R_2 \leftrightarrow R_1} 
			\myvec{9 & 6 \\1-k & 2\\0 & 0} 
			\\
			\myvec{9 & 6\\1-k & 2\\0 & 0}
			\xRightarrow[]{R_2= R_2-\frac{1/3}R_1} 
			\myvec{9 & 6\\ -k-2 & 0 \\ 0 & 0} 
		\end{align}
		The rank of the matrix will be 1 when 
		\begin{align}
		-k-2 = 0
		\end{align}
		\begin{align}
		\Rightarrow k = -2
		\end{align}
		Graph:
\begin{figure}[H]
	\centering
	\includegraphics[scale=0.5]{plot}
	\caption*{}
	\label{fig}
\end{figure}
Therefore, k = -2.
\end{document}
