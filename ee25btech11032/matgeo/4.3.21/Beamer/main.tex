\documentclass{beamer}
\usepackage[utf8]{inputenc}

\usetheme{Madrid}
\usecolortheme{default}
\usepackage{extarrows}
\usepackage{amsmath}
\usepackage{extarrows}
\usepackage{amssymb,amsfonts,amsthm}
\usepackage{txfonts}
\usepackage{tkz-euclide}
\usepackage{listings}
\usepackage{adjustbox}
\usepackage{array}
\usepackage{tabularx}
\usepackage{gvv}
\usepackage{lmodern}
\usepackage{circuitikz}
\usepackage{tikz}
\usepackage{graphicx}
\usepackage{amsmath} 

\setbeamertemplate{page number in head/foot}[totalframenumber]

\usepackage{tcolorbox}
\tcbuselibrary{minted,breakable,xparse,skins}

\definecolor{bg}{gray}{0.95}
\DeclareTCBListing{mintedbox}{O{}m!O{}}{%
  breakable=true,
  listing engine=minted,
  listing only,
  minted language=#2,
  minted style=default,
  minted options={%
    linenos,
    gobble=0,
    breaklines=true,
    breakafter=,,
    fontsize=\small,
    numbersep=8pt,
    #1},
  boxsep=0pt,
  left skip=0pt,
  right skip=0pt,
  left=25pt,
  right=0pt,
  top=3pt,
  bottom=3pt,
  arc=5pt,
  leftrule=0pt,
  rightrule=0pt,
  bottomrule=2pt,
  toprule=2pt,
  colback=bg,
  colframe=orange!70,
  enhanced,
  overlay={%
    \begin{tcbclipinterior}
    \fill[orange!20!white] (frame.south west) rectangle ([xshift=20pt]frame.north west);
    \end{tcbclipinterior}},
  #3,
}
\lstset{
    language=C,
    basicstyle=\ttfamily\small,
    keywordstyle=\color{blue},
    stringstyle=\color{orange},
    commentstyle=\color{green!60!black},
    numbers=left,
    numberstyle=\tiny\color{gray},
    breaklines=true,
    showstringspaces=false,
}
\title %optional
{4.3.21}
\date{September 9 ,2025}


\author 
{Kartik Lahoti - EE25BTECH11032}



\begin{document}


\frame{\titlepage}
\begin{frame}{Question}
In what ratio does the $X$-axis divide the line segment joining the points $\brak{-4,-6}$ and $\brak{-1,7}?$ Find the coordinates of the point of division.
\end{frame}

\begin{frame}{Theoretical Solution}
Given :
\begin{table}[H]
    \centering
    \begin{tabular}[12pt]{ |c| c|}
    \hline
    \textbf{Name} & \textbf{Point}\\ 
    \hline
	Point A &\myvec{h \\ k}\\
    \hline 
 Point B &\myvec{x1 \\ y1}\\
    \hline
	  Point R &\myvec{x2 \\ y2}\\
    \hline
    
    \end{tabular}

    \caption{4.3.21}
    \label{tab:placeholder_1}
\end{table}

Let , $\vec{P}$ be the point on $x$-axis dividing the  line $\vec{AB}$ in the ratio say $k\colon1$.
\end{frame}
\begin{frame}{Theoretical Solution}

Now, $\vec{A}$ , $\vec{B}$ and $\vec{P}$ are collinear , 

\begin{align}
    \therefore rank\myvec{\vec{B} - \vec{A} & \vec{P} - \vec{B}} = 1
\end{align}
\begin{align}
    \myvec{3 & x+1 \\ 13 & -7 } \xleftrightarrow[]{R_2 \rightarrow {\frac{-13}{3}R_1 + R_2}} \myvec{3 & x+1 \\ 0 & -\frac{13x}{3}-\frac{34}{3} } 
\end{align}

\end{frame}

\begin{frame}{Theoretical Solution}
The number of nonzero rows in the echelon form is defined as the rank. For the above matrix to be of rank 1 ,
\begin{align}
    -\frac{13x}{3}-\frac{34}{3} &= 0  \\ 
    x &= -\frac{34}{13}
\end{align}

$\therefore$ The coordinates of the point of intersection is 
\begin{align}
    \vec{P} = \myvec{-\frac{34}{13}  \\ 0}
\end{align}
\end{frame}

\begin{frame}{Theoretical Solution}
Now, 
\begin{align}
    k = \frac{\brak{\vec{A} - \vec{P}}^{\top}\brak{\vec{P} - \vec{B}}}{\norm{\vec{P} - \vec{B}}^2}
\end{align}

Substituting the values of $\vec{A}$ , $\vec{B}$ and $\vec{P}$ , we get,
\begin{align}
    k = \frac{\myvec{-\frac{18}{13}& -6} \myvec{-\frac{21}{13} \\ -7 }}{\norm{\myvec{-\frac{21}{13} \\ -7}}^2}
\end{align}
\begin{align}
    k = \frac{6}{7}
\end{align}
Thus,  the ratio in which the point $\vec{P}$ divides the line segment $\vec{AB}$ is $6\colon7$
\end{frame}

\begin{frame}[fragile]
    \frametitle{C Code (1)}

    \begin{lstlisting}
#include <math.h>
double norm_sq (double *P , double *Q , int m )
{
	double sum = 0 ; 
	for ( int i = 0 ; i  < m ; i++ )
		sum += pow(P[i] - Q[i] , 2);
	return sum ;
}
double ratio ( double *A , double *B , double *P , double norm  )
{
		double k = (A[0]-P[0]) * (P[0] - B[0]) + (A[1] - P[1]) * (P[1] - B[1]);
		k = k / norm;
		return k ; 
}
    \end{lstlisting}
\end{frame}

\begin{frame}[fragile]
    \frametitle{C Code (2) - Function to Generate Points on Line}
    \begin{lstlisting}
#include <math.h>
void line_cre(double *X , double *Y , double *A , double *B , int n , int m )
{
	double temp[m] ; 
	for(int i = 0 ; i < m ; i++)
	{
		temp [i] = (B[i] - A[i])/(double)n ; 
	}
	for(int i = 0 ; i <= n ; i++)
	{
		X[i] = A[0] + temp[0] * i ; 
		Y[i] = A[1] + temp[1] * i ;
	}
}

\end{lstlisting}
\end{frame}

\begin{frame}[fragile]
    \frametitle{Python Code - Using Shared Object}
    \begin{lstlisting}
import ctypes as ct
import numpy as np
import matplotlib.pyplot as plt

handc1 = ct.CDLL("./func.so")
handc1.norm_sq.argtypes = [
    ct.POINTER(ct.c_double),
    ct.POINTER(ct.c_double),
    ct.c_int
]
handc1.norm_sq.restype = ct.c_double
A = np.array([[-4],[-6]] , dtype = np.float64).reshape(-1,1)
B = np.array([[-1],[7]] , dtype = np.float64).reshape(-1,1)
P = np.array([[-34/13] ,[ 0 ]] , dtype = np.float64).reshape(-1,1)
\end{lstlisting}
\end{frame}

\begin{frame}[fragile]
    \frametitle{Python Code - Using Shared Object}
    \begin{lstlisting}

norm = handc1.norm_sq(
    B.ctypes.data_as(ct.POINTER(ct.c_double)),
    P.ctypes.data_as(ct.POINTER(ct.c_double)),
    2)
handc1.ratio.argtypes = [
    ct.POINTER(ct.c_double),
    ct.POINTER(ct.c_double),
    ct.POINTER(ct.c_double),
    ct.c_double]
    
handc1.ratio.restype = ct.c_double
k = handc1.ratio(
    A.ctypes.data_as(ct.POINTER(ct.c_double)),
    B.ctypes.data_as(ct.POINTER(ct.c_double)),
    P.ctypes.data_as(ct.POINTER(ct.c_double)),norm)

\end{lstlisting}
\end{frame}

\begin{frame}[fragile]
    \frametitle{Python Code - Using Shared Object}
    \begin{lstlisting}
def line(P : np.ndarray , Q : np.ndarray , str ) :

    handc2 = ct.CDLL("./line_gen.so")
    handc2.line_cre.argtypes = [
        ct.POINTER(ct.c_double),
        ct.POINTER(ct.c_double),
        ct.POINTER(ct.c_double),
        ct.POINTER(ct.c_double),
        ct.c_int,ct.c_int
    ]
    n = 200
    X_l = np.zeros(n,dtype=np.float64)
    Y_l = np.zeros(n,dtype=np.float64)

    handc2.line_cre.restype = None
\end{lstlisting}
\end{frame}
\begin{frame}[fragile]
    \frametitle{Python Code - Using Shared Object}
    \begin{lstlisting}
handc2.line_cre(
        X_l.ctypes.data_as(ct.POINTER(ct.c_double)),
        Y_l.ctypes.data_as(ct.POINTER(ct.c_double)),
        P.ctypes.data_as(ct.POINTER(ct.c_double)),
        Q.ctypes.data_as(ct.POINTER(ct.c_double)),
        n,2
    )

    plt.plot([X_l[0],X_l[-1]],[Y_l[0],Y_l[-1]], str , label = "Line Segment AB")
    \end{lstlisting}
\end{frame}

\begin{frame}[fragile]
    \frametitle{Python Code - Using Shared Object}
    \begin{lstlisting}
plt.figure()

line(A,B,"g--")

coords = np.block([[A,B,P]])

plt.scatter(coords[0,:] , coords[1,:])
vert_label = ['A', 'B' , 'P' ]

for i , txt in enumerate(vert_label) :
    plt.annotate(f"{txt}\n({coords[0,i]:.1f},{coords[1,i]:.1f})",
                 (coords[0,i], coords[1,i]),
                 textcoords = "offset points" ,
                 xytext = (10,20),ha = "center")
\end{lstlisting}
\end{frame}

\begin{frame}[fragile]
    \frametitle{Python Code - Using Shared Object}
    \begin{lstlisting}
plt.xlabel("$x$")
plt.ylabel("$y$")
plt.grid()

plt.legend(loc="best")

plt.title("4.3.21")

#plt.savefig("../figs/section1.png")
#plt.show()
plt.savefig('../figs/section1.png')
subprocess.run(shlex.split("termux-open ../figs/section1.png"))


\end{lstlisting}
\end{frame}

\begin{frame}[fragile]
    \frametitle{Python Code}
    \begin{lstlisting}
import math
import sys 
sys.path.insert(0, '/home/kartik-lahoti/matgeo/codes/CoordGeo')
import numpy as np
import numpy.linalg as LA
import matplotlib.pyplot as plt

from line.funcs import *
#from triangle.funcs import *
#from conics.funcs import circ_gen

#if using termux
#import subprocess
#import shlex
\end{lstlisting}
\end{frame}

\begin{frame}[fragile]
    \frametitle{Python Code }
    \begin{lstlisting}

def norm_sq(P,Q) :
    return pow(LA.norm(P-Q),2)

def ratio(A,B,P,norm) :
    k =  (A[0]-P[0]) * (P[0] - B[0]) + (A[1] - P[1]) * (P[1] - B[1])
    return k /norm 

\end{lstlisting}
\end{frame}

\begin{frame}[fragile]
    \frametitle{Python Code }
    \begin{lstlisting}
A = np.array([-4,-6]).reshape(-1,1)
B = np.array([-1,7]).reshape(-1,1)
P = np.array([-34/13 , 0 ]).reshape(-1,1)

norm = norm_sq(P,B)

k = ratio(A,B,P,norm) 
k = np.squeeze(k) 
print("Ratio = " , k )
\end{lstlisting}
\end{frame}

\begin{frame}[fragile]
    \frametitle{Python Code }
    \begin{lstlisting}

def plot_it(P,Q,str):
    x_l = line_gen_num(P,Q,20)
    plt.plot(x_l[0,:],x_l[1,:] , str , label = "Line Segment AB" )

plt.figure()

plot_it(A,B,"g--")

coords = np.block([[A,B,P]])
plt.scatter(coords[0,:],coords[1,:])
vert_labels = ['A','B','P']

    \end{lstlisting}
\end{frame}
\begin{frame}[fragile]
    \frametitle{Python Code }
    \begin{lstlisting}
for i, txt in enumerate(vert_labels):
    plt.annotate(f'{txt}\n({coords[0,i]:.1f}, {coords[1,i]:.1f})',
                 (coords[0,i], coords[1,i]),
                 textcoords="offset points",
                 xytext=(20,15),
                 ha='center')
plt.xlabel('$x$')
plt.ylabel('$y$')
plt.grid()
plt.legend(loc = "best")
plt.title("Fig:4.3.21")
#plt.savefig("../figs/section2.png")
#plt.show()
plt.savefig('../figs/section2.png')
subprocess.run(shlex.split("termux-open ../figs/section2.png"))

    \end{lstlisting}
\end{frame}


\begin{frame}{Plot}
    \centering
    \includegraphics[width=\columnwidth, height=0.8\textheight, keepaspectratio]{../figs/section1.png}   
\end{frame}


\end{document}
