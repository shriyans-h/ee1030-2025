\documentclass{beamer}
\usepackage[utf8]{inputenc}

\usetheme{Madrid}
\usecolortheme{default}
\usepackage{extarrows}
\usepackage{amsmath}
\usepackage{extarrows}
\usepackage{amssymb,amsfonts,amsthm}
\usepackage{txfonts}
\usepackage{tkz-euclide}
\usepackage{listings}
\usepackage{adjustbox}
\usepackage{array}
\usepackage{tabularx}
\usepackage{gvv}
\usepackage{lmodern}
\usepackage{circuitikz}
\usepackage{tikz}
\usepackage{graphicx}
\usepackage{amsmath} 

\setbeamertemplate{page number in head/foot}[totalframenumber]

\usepackage{tcolorbox}
\tcbuselibrary{minted,breakable,xparse,skins}

\definecolor{bg}{gray}{0.95}
\DeclareTCBListing{mintedbox}{O{}m!O{}}{%
  breakable=true,
  listing engine=minted,
  listing only,
  minted language=#2,
  minted style=default,
  minted options={%
    linenos,
    gobble=0,
    breaklines=true,
    breakafter=,,
    fontsize=\small,
    numbersep=8pt,
    #1},
  boxsep=0pt,
  left skip=0pt,
  right skip=0pt,
  left=25pt,
  right=0pt,
  top=3pt,
  bottom=3pt,
  arc=5pt,
  leftrule=0pt,
  rightrule=0pt,
  bottomrule=2pt,
  toprule=2pt,
  colback=bg,
  colframe=orange!70,
  enhanced,
  overlay={%
    \begin{tcbclipinterior}
    \fill[orange!20!white] (frame.south west) rectangle ([xshift=20pt]frame.north west);
    \end{tcbclipinterior}},
  #3,
}
\lstset{
    language=C,
    basicstyle=\ttfamily\small,
    keywordstyle=\color{blue},
    stringstyle=\color{orange},
    commentstyle=\color{green!60!black},
    numbers=left,
    numberstyle=\tiny\color{gray},
    breaklines=true,
    showstringspaces=false,
}
\title %optional
{5.4.27}


\author 
{Kartik Lahoti - EE25BTECH11032}



\begin{document}


\frame{\titlepage}
\begin{frame}{Question}
Using elementary transformations, find the inverse of the following matrix.
\begin{align*}
    \myvec{2 & 0 & -1 \\ 5 & 1 & 0 \\ 0 & 1 & 3}
\end{align*}
\end{frame}

\begin{frame}{Theoretical Solution}
Given the matrix,
\begin{align}
    \vec{A} = \myvec{2 & 0 & -1 \\ 5 & 1 & 0 \\ 0 & 1 & 3}
\end{align}

Let $\vec{A}^{-1}$ be the inverse of the matrix $\vec{A}$

We know that,
\begin{align}
\vec{A}\vec{A}^{-1} = \vec{I}
\end{align}

\end{frame}

\begin{frame}{Theoretical Solution}
The augmented matrix of $\augvec{1}{1}{\vec{A} & \vec{I}}$ is given by , 

\begin{align}
    \augvec{3}{3}{2 & 0 & -1 & 1 & 0 & 0\\ 5 & 1 & 0 & 0 & 1 & 0 \\ 0 & 1 & 3 & 0 & 0 & 1}
\end{align}
\end{frame}

\begin{frame}{Theoretical Solution}
\begin{align}
    \augvec{3}{3}{2 & 0 & -1 & 1 & 0 & 0\\ 5 & 1 & 0 & 0 & 1 & 0 \\ 0 & 1 & 3 & 0 & 0 & 1}\xleftrightarrow[]{R_1\rightarrow\frac{1}{2}R_1}\augvec{3}{3}{1 & 0 & \frac{-1}{2}& \frac{1}{2} & 0 & 0\\ 5 & 1 & 0 & 0 & 1 & 0 \\ 0 & 1 & 3 & 0 & 0 & 1}
\end{align}
\begin{align}
    \augvec{3}{3}{1 & 0 & \frac{-1}{2}& \frac{1}{2} & 0 & 0\\ 5 & 1 & 0 & 0 & 1 & 0 \\ 0 & 1 & 3 & 0 & 0 & 1}\xleftrightarrow[]{R_2\rightarrow R_2 - 5R_1}\augvec{3}{3}{1 & 0 & \frac{-1}{2}& \frac{1}{2} & 0 & 0\\ 0 & 1 & \frac{5}{2} & \frac{-5}{2} & 1 & 0 \\ 0 & 1 & 3 & 0 & 0 & 1}
\end{align}
\end{frame}

\begin{frame}{Theoretical Solution}

\begin{align}
    \augvec{3}{3}{1 & 0 & \tfrac{-1}{2}& \tfrac{1}{2} & 0 & 0\\ 0 & 1 & \tfrac{5}{2} & \tfrac{-5}{2} & 1 & 0 \\ 0 & 1 & 3 & 0 & 0 & 1}\xleftrightarrow[]{R_3\rightarrow R_3 - R_2 }\augvec{3}{3}{1 & 0 & \tfrac{-1}{2}& \tfrac{1}{2} & 0 & 0\\ 0 & 1 & \tfrac{5}{2} & \tfrac{-5}{2} & 1 & 0 \\ 0 & 0 & \tfrac{1}{2} & \tfrac{5}{2} & -1 & 1}
\end{align}
\begin{align}
    \augvec{3}{3}{1 & 0 & \tfrac{-1}{2}& \tfrac{1}{2} & 0 & 0\\ 0 & 1 & \tfrac{5}{2} & \tfrac{-5}{2} & 1 & 0 \\ 0 & 0 & \tfrac{1}{2} & \tfrac{5}{2} & -1 & 1}\xleftrightarrow[]{R_3\rightarrow 2R_3}\augvec{3}{3}{1 & 0 & \tfrac{-1}{2}& \tfrac{1}{2} & 0 & 0\\ 0 & 1 & \tfrac{5}{2} & \tfrac{-5}{2} & 1 & 0 \\ 0 & 0 & 1 & 5 & -2 & 2}
\end{align}
\end{frame}
\begin{frame}{Theoretical Solution}
\begin{align}
    \augvec{3}{3}{1 & 0 & \tfrac{-1}{2}& \tfrac{1}{2} & 0 & 0\\ 0 & 1 & \tfrac{5}{2} & \tfrac{-5}{2} & 1 & 0 \\ 0 & 0 & 1 & 5 & -2 & 2}\xleftrightarrow[]{R_2\rightarrow R_2 - \frac{5}{2}R_3 }\augvec{3}{3}{1 & 0 & \tfrac{-1}{2}& \tfrac{1}{2} & 0 & 0\\ 0 & 1 & 0 & -15 & 6 & -5 \\ 0 & 0 & 1 & 5 & -2 & 2}
\end{align}
\begin{align}
    \augvec{3}{3}{1 & 0 & \tfrac{-1}{2}& \tfrac{1}{2} & 0 & 0\\ 0 & 1 & 0 & -15 & 6 & -5 \\ 0 & 0 & 1 & 5 & -2 & 2}\xleftrightarrow[]{R_1\rightarrow R_1 + \frac{1}{2}R_3}\augvec{3}{3}{1 & 0 & 0 & 3 & -1 & 1\\ 0 & 1 & 0 & -15 & 6 & -5 \\ 0 & 0 & 1 & 5 & -2 & 2}
\end{align}
\end{frame}

\begin{frame}{Theoretical Solution}
Hence ,
\begin{align}
    \vec{A}^{-1} = \myvec{3 & -1 & 1 \\ -15 & 6 & -5 \\ 5 & -2 & 2}
\end{align}
\end{frame}

\begin{frame}[fragile]
    \frametitle{C Code - To find inverse of a Matrix }
    \begin{lstlisting}
#include <stdio.h>
void row_mal(double A[3][6] , int n , int m, double k ){
	for(int i = 0 ; i < 6 ;i++)
	{
		A[m][i] -= A[n][i]*k; 
	}
}
void row_div(double A[3][6] , int n , int m){
	double k = A[n][m]; 
	for(int i = 0 ; i  <6 ; i++)
	{
		A[n][i] /= k ;
	}
}

    \end{lstlisting}
\end{frame}

\begin{frame}[fragile]
    \frametitle{C Code  }
    \begin{lstlisting}
void inv( double *A , double *B , double *C  ){
	double K[3][6]; 
	for(int i = 0 ; i < 3 ; i++)	{
		K[i][0] = A[i]; 
		K[i][1] = B[i];
		K[i][2] = C[i];
	}
	for(int i = 0 ; i < 3 ;  i++)	{
	   // K[i][i] = 1 ;
		for(int j = 3 ; j < 6 ; j++){
			if( j-3  == i )
				K[i][j] = 1 ;
			else
				K[i][j] = 0 ; 
	}
}
\end{lstlisting}
\end{frame}
\begin{frame}[fragile]
    \frametitle{C Code  }
    \begin{lstlisting}
	//print
	for(int i = 0 ; i  < 3 ; i++)
	{
	    
		for(int j = 0 ; j < 6; j++)
		{
		    if( j < 3){
			printf("%.1f ",K[i][j]);}
		}
		printf("\n");
	}	
\end{lstlisting}
\end{frame}

\begin{frame}[fragile]
    \frametitle{C Code  }
    \begin{lstlisting}
    if(K[0][0] != 0  )
    {
        row_div(K , 0 , 0 );
	row_mal(K , 0 , 1 , K[1][0]);
	row_mal(K , 0 , 2 , K[2][0]);		
    }
    else
    {
	if(K[1][0] != 0)
		row_mal(K,0,1,-1);
	else if(K[2][0] != 0)
		row_mal(K,0,2,-1);
	  row_div(K , 0 , 0 );
	  row_mal(K , 0 , 1 , K[1][0]);
	  row_mal(K , 0 , 2 , K[2][0]);		
    }


\end{lstlisting}
\end{frame}
\begin{frame}[fragile]
    \frametitle{C Code  }
    \begin{lstlisting}
    if ( K[1][1] != 0  )
    {
	row_div(K , 1, 1);
	row_mal(K, 1, 0 , K[0][1]);
	row_mal(K , 1, 2 , K[2][1]);
    }
    else
    {
	if(K[0][1] != 0 )
		row_mal(K, 1 , 0 , -1);
	else if(K[2][1] != 0 )
		row_mal(K, 1 , 2 ,-1);
	row_div(K , 1, 1);
	row_mal(K, 1, 0 , K[0][1]);
	row_mal(K , 1, 2 , K[2][1]);
    }
\end{lstlisting}
\end{frame}
\begin{frame}[fragile]
    \frametitle{C Code  }
    \begin{lstlisting}
    if (K[2][2] != 0  )
    {
	row_div(K , 2, 2);
	row_mal(K, 2, 0 , K[0][2]);
        row_mal(K , 2, 1 , K[1][2]);
    }
    else
    {
	if(K[0][2] != 0 )
		row_mal(K,2 , 0 , -1);
        else if(K[1][2] != 0 )
		row_mal(K,2,1,-1);
	row_div(K , 2, 2);
	row_mal(K, 2, 0 , K[0][2]);
	row_mal(K , 2, 1 , K[1][2]);
    }

\end{lstlisting}
\end{frame}
\begin{frame}[fragile]
    \frametitle{C Code  }
    \begin{lstlisting}
    printf("_____________________\n");
	for(int i = 0 ; i  < 3 ; i++)
	{   
	    
		for(int j = 0 ; j < 6; j++)
		{
		    if ( j >= 3){
			printf("%.3f ",K[i][j]);}
		}
		printf("\n");
	}	
}
\end{lstlisting}
\end{frame}

\begin{frame}[fragile]
    \frametitle{Python Code}
    \begin{lstlisting}
import ctypes as ct
import numpy as np

handc1 = ct.CDLL("./func.so")

handc1.inv.argtypes = [
    ct.POINTER(ct.c_double),
    ct.POINTER(ct.c_double),
    ct.POINTER(ct.c_double)
]

\end{lstlisting}
\end{frame}

\begin{frame}[fragile]
    \frametitle{Python Code}
    \begin{lstlisting}
A = np.array([2 , 5 , 0 ], dtype =np.float64).reshape(-1,1)
B = np.array([0 , 1 , 1 ], dtype =np.float64).reshape(-1,1)
C = np.array([-1 , 0 , 3 ], dtype =np.float64).reshape(-1,1)

handc1.inv.restype = None

handc1.inv(
    A.ctypes.data_as(ct.POINTER(ct.c_double)),
    B.ctypes.data_as(ct.POINTER(ct.c_double)),
    C.ctypes.data_as(ct.POINTER(ct.c_double))
)
\end{lstlisting}
\end{frame}

\end{document}
