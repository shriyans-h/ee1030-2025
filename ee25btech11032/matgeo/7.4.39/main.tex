\let\negmedspace\undefined
\let\negthickspace\undefined
\documentclass[journal]{IEEEtran}
\usepackage[a5paper, margin=10mm, onecolumn]{geometry}
\usepackage{tfrupee} 
\setlength{\headheight}{1cm} 
\setlength{\headsep}{0mm}     

\usepackage{gvv-book}
\usepackage{gvv}
\usepackage{cite}
\usepackage{amsmath,amssymb,amsfonts,amsthm}
\usepackage{algorithmic}
\usepackage{graphicx}
\usepackage{textcomp}
\usepackage{xcolor}
\usepackage{txfonts}
\usepackage{listings}
\usepackage{enumitem}
\usepackage{mathtools}
\usepackage{gensymb}
%\usepackage{wasysym}
\usepackage{comment}
\usepackage[breaklinks=true]{hyperref}
\usepackage{tkz-euclide} 
\usepackage{listings}
\def\inputGnumericTable{}                                 
\usepackage[latin1]{inputenc}                                
\usepackage{color}                                            
\usepackage{array}                                            
\usepackage{longtable}                                       
\usepackage{calc}                                             
\usepackage{multirow}                                         
\usepackage{hhline}                                           
\usepackage{ifthen}                                           
\usepackage{lscape}
\usepackage{circuitikz}
\tikzstyle{block} = [rectangle, draw, fill=blue!20, 
    text width=4em, text centered, rounded corners, minimum height=3em]
\tikzstyle{sum} = [draw, fill=blue!10, circle, minimum size=1cm, node distance=1.5cm]
\tikzstyle{input} = [coordinate]
\tikzstyle{output} = [coordinate]
\renewcommand{\thefigure}{\theenumi}
\renewcommand{\thetable}{\theenumi}
\setlength{\intextsep}{10pt} % Space between text and floats
\numberwithin{equation}{enumi}
\numberwithin{figure}{enumi}
\renewcommand{\thetable}{\theenumi}

\begin{document}

\bibliographystyle{IEEEtran}
\vspace{3cm}

\title{7.4.39}
\author{EE25BTECH11032 - Kartik Lahoti}
\maketitle

\subsection*{Question: } 

If $\brak{m_i , \frac{1}{m_i}}$ , $m_i > 0 , i = 1, 2,3,4$ are four distinct points on a circle, then show that $m_1m_2m_3m_4 = 1 $

\textbf{Solution}:\\


Let the circle equation be 
\begin{align}
    \norm{\vec{x}}^2 + 2\vec{u}^{\top}\vec{x} + f = 0 
\end{align}

where, $\vec{u} = \myvec{a \\ b}$ with $a$ and $b$ as constants.

Let $\vec{P} = \myvec{m \\ \frac{1}{m}}$ be a arbitrary vector in space.

Putting $\vec{P}$ in the circle , we get 


\begin{align}
    \norm{\vec{P}}^2 + 2\vec{u}^{\top}\vec{P} + f = 0 
\end{align}

\begin{align}
    m^2 + \frac{1}{m^2} + 2am + \frac{2b}{m} + f = 0
\end{align}

\begin{align}
    m^4 + 2am^3 + fm^2 + 2bm + 1 = 0 \label{eq_1}
\end{align}

For a general polynomial of degree $n$

\begin{align}
    a_0x^n + a_1x^{n-1} + a_2x^{n-2} \dots + a_nx^0 = 0  
\end{align}

Product of roots is given by 
\begin{align}
    \brak{-1}^n\frac{a_n}{a_0}
\end{align}

Since , $m_i , $ where $ i \in \cbrak{1,2,3,4} $ satisfies the equation \ref{eq_1}.

we can say 

\begin{align}
    m_1m_2m_3m_4 = 1 
\end{align}

Hence Proved

\begin{figure}[H]
    \centering
    \includegraphics[width=1.0\columnwidth]{figs/graph1.png}
    \caption*{}
    \label{fig:placeholder}
\end{figure}

\end{document}


