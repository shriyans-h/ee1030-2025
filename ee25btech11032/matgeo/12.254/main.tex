\let\negmedspace\undefined
\let\negthickspace\undefined
\documentclass[journal]{IEEEtran}
\usepackage[a5paper, margin=10mm, onecolumn]{geometry}
\usepackage{tfrupee} 
\setlength{\headheight}{1cm} 
\setlength{\headsep}{0mm}     

\usepackage{gvv-book}
\usepackage{gvv}
\usepackage{cite}
\usepackage{amsmath,amssymb,amsfonts,amsthm}
\usepackage{algorithmic}
\usepackage{graphicx}
\usepackage{textcomp}
\usepackage{xcolor}
\usepackage{txfonts}
\usepackage{listings}
\usepackage{enumitem}
\usepackage{mathtools}
\usepackage{gensymb}
%\usepackage{wasysym}
\usepackage{comment}
\usepackage[breaklinks=true]{hyperref}
\usepackage{tkz-euclide} 
\usepackage{listings}
\def\inputGnumericTable{}                                 
\usepackage[latin1]{inputenc}                                
\usepackage{color}                                            
\usepackage{array}                                            
\usepackage{longtable}                                       
\usepackage{calc}                                             
\usepackage{multirow}                                         
\usepackage{hhline}                                           
\usepackage{ifthen}                                           
\usepackage{lscape}
\usepackage{circuitikz}
\tikzstyle{block} = [rectangle, draw, fill=blue!20, 
    text width=4em, text centered, rounded corners, minimum height=3em]
\tikzstyle{sum} = [draw, fill=blue!10, circle, minimum size=1cm, node distance=1.5cm]
\tikzstyle{input} = [coordinate]
\tikzstyle{output} = [coordinate]
\renewcommand{\thefigure}{\theenumi}
\renewcommand{\thetable}{\theenumi}
\setlength{\intextsep}{10pt} % Space between text and floats
\numberwithin{equation}{enumi}
\numberwithin{figure}{enumi}
\renewcommand{\thetable}{\theenumi}

\begin{document}

\bibliographystyle{IEEEtran}
\vspace{3cm}

\title{12.254}
\author{EE25BTECH11032 - Kartik Lahoti}
\maketitle

\subsection*{Question: } 

The two vectors $\sbrak{1,1,1}$ and $\sbrak{1,a,a^2}$, where $a = \brak{\frac{-1}{2} + j\frac{\sqrt{3}}{2}}$

\begin{enumerate}
    \begin{multicols}{4}
        \item orthonormal
        \item orthogonal
        \item paralle
        \item collinear
    \end{multicols}
\end{enumerate}

\textbf{Solution}:\\

Given , 
\begin{align}
    \vec{P} = \myvec{1\\1\\1} 
\end{align}
\begin{align}
    \vec{Q} = \myvec{1\\a\\a^2} 
\end{align}

Let,

\begin{align}
    \vec{z_1}=x_1+jy_1 \longrightarrow{} \myvec{x_1 & -y_1 \\ y_1 & x_1 }\\ 
    \vec{z_2}=x_2+jy_2 \longrightarrow{} \myvec{x_2 & -y_2 \\ y_2 & x_2 } 
\end{align}

Look at 

\begin{align} 
    \vec{z_1} + \vec{z_2} = \brak{x_1+x_2} + j \brak{y_1 + y_2}
\end{align}
Which is equivalent to 
\begin{align}
    \myvec{x_1 & -y_1 \\ y_1 & x_1 } + \myvec{x_2 & -y_2 \\ y_2 & x_2 } = \myvec{x_1+x_2 & -y_1-y_2 \\ y_1+y_2 & x_1+x_1}
\end{align}

Also, 

\begin{align}
    \vec{z_1}\vec{z_2} = \brak{x_1x_2 - y_1y_2} + j\brak{x_1y_2 + x_2y_1}
\end{align}

This is equivalent to 

\begin{align}
    \myvec{x_1 & -y_1 \\ y_1 & x_1}\myvec{x_2 & -y_2 \\ y_2 & x_2 } = \myvec{\brak{x_1x_2 - y_1y_2} & -\brak{x_1y_2 + x_2y_1} \\ \brak{x_1y_2 + x_2y_1} & \brak{x_1x_2 - y_1y_2}}
\end{align}

$\therefore$ Complex Numbers can be represented as this matrix form since it satisfies Addition and Multiplication properties. 
 
\begin{align}
    x+jy \longrightarrow{} \myvec{x & -y \\ y & x }
\end{align}

\begin{align}
    \therefore a = \brak{\frac{-1}{2} + j\frac{\sqrt{3}}{2}} \longrightarrow \vec{A} = \myvec{\frac{-1}{2}&-\frac{\sqrt{3}}{2}\\\frac{-1}{2} & \frac{\sqrt{3}}{2}}
\end{align}

Similarly 

\begin{align}
    a^2 = \brak{\frac{-1}{2} - j\frac{\sqrt{3}}{2}} \longrightarrow \vec{A}^2 = \myvec{\frac{-1}{2}&\frac{\sqrt{3}}{2}\\\frac{-1}{2} & -\frac{\sqrt{3}}{2}}
\end{align}

\begin{align}
    1 \longrightarrow \vec{I} = \myvec{1&0\\0&1}
\end{align}

Now, 
\begin{align}
    \vec{I} + \vec{A} + \vec{A^2} &= \myvec{1&0\\0&1} + \myvec{\frac{-1}{2}&-\frac{\sqrt{3}}{2}\\\frac{-1}{2} & \frac{\sqrt{3}}{2}} + \myvec{\frac{-1}{2}&\frac{\sqrt{3}}{2}\\\frac{-1}{2} & -\frac{\sqrt{3}}{2}}\\ 
    &= \myvec{0&0\\0&0}
\end{align}

\begin{align}
    \implies 1 + a + a^2 = 0 
\end{align}

Now, Look At , 

\begin{align}
    \vec{P}^{\top}\vec{Q} = \myvec{1&1&1}\myvec{1\\a\\a^2} = 1 + a + a^2 = 0 
\end{align}

Hence $\vec{P}$ and $\vec{Q}$ are orthogonal. 

Answer  : Option $\brak{2}$

\end{document}


