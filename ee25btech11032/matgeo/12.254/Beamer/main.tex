\documentclass{beamer}
\usepackage[utf8]{inputenc}

\usetheme{Madrid}
\usecolortheme{default}
\usepackage{extarrows}
\usepackage{amsmath}
\usepackage{extarrows}
\usepackage{amssymb,amsfonts,amsthm}
\usepackage{txfonts}
\usepackage{tkz-euclide}
\usepackage{listings}
\usepackage{adjustbox}
\usepackage{array}
\usepackage{tabularx}
\usepackage{gvv}
\usepackage{lmodern}
\usepackage{circuitikz}
\usepackage{tikz}
\usepackage{graphicx}
\usepackage{amsmath} 

\setbeamertemplate{page number in head/foot}[totalframenumber]

\usepackage{tcolorbox}
\tcbuselibrary{minted,breakable,xparse,skins}

\definecolor{bg}{gray}{0.95}
\DeclareTCBListing{mintedbox}{O{}m!O{}}{%
  breakable=true,
  listing engine=minted,
  listing only,
  minted language=#2,
  minted style=default,
  minted options={%
    linenos,
    gobble=0,
    breaklines=true,
    breakafter=,,
    fontsize=\small,
    numbersep=8pt,
    #1},
  boxsep=0pt,
  left skip=0pt,
  right skip=0pt,
  left=25pt,
  right=0pt,
  top=3pt,
  bottom=3pt,
  arc=5pt,
  leftrule=0pt,
  rightrule=0pt,
  bottomrule=2pt,
  toprule=2pt,
  colback=bg,
  colframe=orange!70,
  enhanced,
  overlay={%
    \begin{tcbclipinterior}
    \fill[orange!20!white] (frame.south west) rectangle ([xshift=20pt]frame.north west);
    \end{tcbclipinterior}},
  #3,
}
\lstset{
    language=C,
    basicstyle=\ttfamily\small,
    keywordstyle=\color{blue},
    stringstyle=\color{orange},
    commentstyle=\color{green!60!black},
    numbers=left,
    numberstyle=\tiny\color{gray},
    breaklines=true,
    showstringspaces=false,
}
\title %optional
{12.254}


\author 
{Kartik Lahoti - EE25BTECH11032}

\begin{document}


\frame{\titlepage}
\begin{frame}{Question}
The two vectors $\sbrak{1,1,1}$ and $\sbrak{1,a,a^2}$, where $a = \brak{\frac{-1}{2} + j\frac{\sqrt{3}}{2}}$

\begin{multicols}
\begin{enumerate}
    \item orthonormal
        \item orthogonal
        \item paralle
        \item collinear
\end{enumerate}
\end{multicols}

\end{frame}

\begin{frame}{Theoretical Solution}
Given , 
\begin{align}
    \vec{P} = \myvec{1\\1\\1} 
\end{align}
\begin{align}
    \vec{Q} = \myvec{1\\a\\a^2} 
\end{align}
\end{frame}

\begin{frame}{Theoretical Solution}
we know, 

\begin{align}
    x+jy \longrightarrow{} \myvec{x & -y \\ y & x }
\end{align}

\begin{align}
    a = \brak{\frac{-1}{2} + j\frac{\sqrt{3}}{2}} \longrightarrow \vec{A} = \myvec{\frac{-1}{2}&-\frac{\sqrt{3}}{2}\\\frac{-1}{2} & \frac{\sqrt{3}}{2}}
\end{align}

Similarly 

\begin{align}
    a^2 = \brak{\frac{-1}{2} - j\frac{\sqrt{3}}{2}} \longrightarrow \vec{A}^2 = \myvec{\frac{-1}{2}&\frac{\sqrt{3}}{2}\\\frac{-1}{2} & -\frac{\sqrt{3}}{2}}
\end{align}

\begin{align}
    1 \longrightarrow \vec{I} = \myvec{1&0\\0&1}
\end{align}
\end{frame}

\begin{frame}{Theoretical Solution}

Now, 
\begin{align}
    \vec{I} + \vec{A} + \vec{A^2} &= \myvec{1&0\\0&1} + \myvec{\frac{-1}{2}&-\frac{\sqrt{3}}{2}\\\frac{-1}{2} & \frac{\sqrt{3}}{2}} + \myvec{\frac{-1}{2}&\frac{\sqrt{3}}{2}\\\frac{-1}{2} & -\frac{\sqrt{3}}{2}}\\ 
    &= \myvec{0&0\\0&0}
\end{align}

\begin{align}
    \implies 1 + a + a^2 = 0 
\end{align}
\end{frame}
\begin{frame}{Theoretical Solution}
Now, Look At , 

\begin{align}
    \vec{P}^{\top}\vec{Q} = \myvec{1&1&1}\myvec{1\\a\\a^2} = 1 + a + a^2 = 0 
\end{align}

Hence $\vec{P}$ and $\vec{Q}$ are orthogonal. 

Answer  : Option $\brak{2}$
\end{frame}

\end{document}
