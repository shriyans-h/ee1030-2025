\let\negmedspace\undefined
\let\negthickspace\undefined
\documentclass[journal]{IEEEtran}
\usepackage[a5paper, margin=10mm, onecolumn]{geometry}
\usepackage{tfrupee} 
\setlength{\headheight}{1cm} 
\setlength{\headsep}{0mm}     

\usepackage{gvv-book}
\usepackage{gvv}
\usepackage{cite}
\usepackage{amsmath,amssymb,amsfonts,amsthm}
\usepackage{algorithmic}
\usepackage{graphicx}
\usepackage{textcomp}
\usepackage{xcolor}
\usepackage{txfonts}
\usepackage{listings}
\usepackage{enumitem}
\usepackage{mathtools}
\usepackage{gensymb}
%\usepackage{wasysym}
\usepackage{comment}
\usepackage[breaklinks=true]{hyperref}
\usepackage{tkz-euclide} 
\usepackage{listings}
\def\inputGnumericTable{}                                 
\usepackage[latin1]{inputenc}                                
\usepackage{color}                                            
\usepackage{array}                                            
\usepackage{longtable}                                       
\usepackage{calc}                                             
\usepackage{multirow}                                         
\usepackage{hhline}                                           
\usepackage{ifthen}                                           
\usepackage{lscape}
\usepackage{circuitikz}
\tikzstyle{block} = [rectangle, draw, fill=blue!20, 
    text width=4em, text centered, rounded corners, minimum height=3em]
\tikzstyle{sum} = [draw, fill=blue!10, circle, minimum size=1cm, node distance=1.5cm]
\tikzstyle{input} = [coordinate]
\tikzstyle{output} = [coordinate]
\renewcommand{\thefigure}{\theenumi}
\renewcommand{\thetable}{\theenumi}
\setlength{\intextsep}{10pt} % Space between text and floats
\numberwithin{equation}{enumi}
\numberwithin{figure}{enumi}
\renewcommand{\thetable}{\theenumi}

\begin{document}

\bibliographystyle{IEEEtran}
\vspace{3cm}

\title{4.11.10}
\author{EE25BTECH11032 - Kartik Lahoti}
\maketitle

\subsection*{Question: } 
Point $\vec{A}$ lies on the line segment $\vec{XY}$  joining $\vec{X}\brak{6,-6}$ and $\vec{Y}\brak{-4,-1}$ in such a way that $\frac{\vec{XA}}{\vec{XY}} = \frac{2}{5}$. if point $\vec{A}$ also lies on the line $3x + k(y+1) = 0$, find the value of $k$ .

\solution 

Given : 
\begin{table}[H]
    \centering
    \begin{tabular}[12pt]{ |c| c|}
    \hline
    \textbf{Name} & \textbf{Point}\\ 
    \hline
	Point A &\myvec{h \\ k}\\
    \hline 
 Point B &\myvec{x1 \\ y1}\\
    \hline
	  Point R &\myvec{x2 \\ y2}\\
    \hline
    
    \end{tabular}

    \caption*{}
    \label{tab:placeholder_1}
\end{table}


Using Section Formula, 
\begin{align}
    \vec{A} = \frac{1}{1+p}\brak{\vec{X} + p\vec{Y}}
\end{align}

From the question , $p = \frac{2}{3}$

Substituting the values, we get

\begin{align}
    \vec{A} =  \frac{1}{1 + \frac{2}{3}}\brak{\myvec{6 \\ -6} + \frac{2}{3}\myvec{-4\\-1}} = \myvec{2 \\ -4}
\end{align}

Given Line Equation, 
\begin{align}
    \myvec{3 & k}\vec{x} + k &= 0  
\end{align}
Putting $\vec{A}$ in this equation,

\begin{align}
    \myvec{3 & k}\myvec{2 \\ -4} + k &= 0 
\end{align}
\begin{align}
    6 - 4k + k = 0 
\end{align}
\begin{align}
    k &= 2
\end{align}

Hence, 
\begin{align}
    \vec{A} = \myvec{2 \\ -4 } \text{ and } k = 2
\end{align}


\begin{figure}[H]
    \centering
    \includegraphics[width=1.0\columnwidth]{figs/section1.png}
    \caption*{}
    \label{fig:}
\end{figure}

\end{document}

