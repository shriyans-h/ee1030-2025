\let\negmedspace\undefined
\let\negthickspace\undefined
\documentclass[journal]{IEEEtran}
\usepackage[a5paper, margin=10mm, onecolumn]{geometry}
\usepackage{tfrupee} 
\setlength{\headheight}{1cm} 
\setlength{\headsep}{0mm}     

\usepackage{gvv-book}
\usepackage{gvv}
\usepackage{cite}
\usepackage{amsmath,amssymb,amsfonts,amsthm}
\usepackage{algorithmic}
\usepackage{graphicx}
\usepackage{textcomp}
\usepackage{xcolor}
\usepackage{txfonts}
\usepackage{listings}
\usepackage{enumitem}
\usepackage{mathtools}
\usepackage{gensymb}
\usepackage{comment}
\usepackage[breaklinks=true]{hyperref}
\usepackage{tkz-euclide} 
\usepackage{listings}
\def\inputGnumericTable{}                                 
\usepackage[latin1]{inputenc}                                
\usepackage{color}                                            
\usepackage{array}                                            
\usepackage{longtable}                                       
\usepackage{calc}                                             
\usepackage{multirow}                                         
\usepackage{hhline}                                           
\usepackage{ifthen}                                           
\usepackage{lscape}
\usepackage{circuitikz}
\tikzstyle{block} = [rectangle, draw, fill=blue!20, 
    text width=4em, text centered, rounded corners, minimum height=3em]
\tikzstyle{sum} = [draw, fill=blue!10, circle, minimum size=1cm, node distance=1.5cm]
\tikzstyle{input} = [coordinate]
\tikzstyle{output} = [coordinate]
\renewcommand{\thefigure}{\theenumi}
\renewcommand{\thetable}{\theenumi}
\setlength{\intextsep}{10pt} % Space between text and floats
\numberwithin{equation}{enumi}
\numberwithin{figure}{enumi}
\renewcommand{\thetable}{\theenumi}

\begin{document}

\bibliographystyle{IEEEtran}
\vspace{3cm}

\title{2.10.61}
\author{EE25BTECH11032 - Kartik Lahoti}
\maketitle

\subsection*{Question: } 
If $\vec{a}$ and $\vec{b}$ are vectors such that $\mydet{\vec{a} + \vec{b}} = \sqrt{29}$ and
\begin{align*}
    \vec{a} \times \brak{2\hat{i} + 3\hat{j} + 4\hat{k}} = \brak{2\hat{i} + 3\hat{j} + 4\hat{k}} \times \vec{b}
\end{align*}
then a possible value of $\brak{\vec{a}+\vec{b}}\cdot\brak{-7\hat{i} + 2\hat{j} + 3\hat{k}}$ is 
\begin{enumerate}
    \begin{multicols}{4}
    \item 0
    \item 3
    \item 4
    \item 8
    \end{multicols}
\end{enumerate}
\solution 

Given : 
\begin{table}[H]
    \centering
    \begin{tabular}[12pt]{ |c| c|}
    \hline
    \textbf{Name} & \textbf{Point}\\ 
    \hline
	Point A &\myvec{h \\ k}\\
    \hline 
 Point B &\myvec{x1 \\ y1}\\
    \hline
	  Point R &\myvec{x2 \\ y2}\\
    \hline
    
    \end{tabular}

    \caption*{}
    \label{tab:placeholder_1}
\end{table}
\begin{align}
    \vec{a} \times \vec{c} &= \vec{c} \times \vec{b}\\
    \vec{a} \times \vec{c} &= - \brak{\vec{b} \times \vec{c}} \\
    \brak{\vec{a} + \vec{b}} \times \vec{c} &= \vec{0}  
\end{align}
If cross product of 2 vectors is zero ,  this implies both the vectors are parallel. 
Thus , 
\begin{align}
    \brak{\vec{a} + \vec{b}} \parallel \vec{c} 
\end{align}

\begin{align}
   \therefore\brak{\vec{a} + \vec{b}} = \lambda\vec{c}  \text{, where } \lambda \in \mathbb{R}
\end{align}
Equating the magnitudes , we get 
\begin{align}
    \norm{\brak{\vec{a} + \vec{b}}}^2 &= \lambda^2 \norm{\vec{c}}^2 \\
    29 &= \lambda^2 29 \\ 
    \lambda &= \pm 1 
 \end{align}
Thus, 
\begin{align}
    \brak{\vec{a} + \vec{b}} = \myvec{2 \\ 3 \\ 4} \text{ or } \brak{\vec{a} + \vec{b}} = \myvec{-2 \\ -3 \\ -4}
\end{align}
Hence, 
\begin{align}
    \brak{\vec{a} + \vec{b}}^{\top}\vec{d} = 4 \text{ or } -4
\end{align}

Answer : Option $\brak{3}$

\begin{figure}[H]
    \centering
    \includegraphics[width=1\columnwidth]{figs/vector1.png}
    \caption*{}
    \label{fig:}
\end{figure}

\end{document}

