\documentclass{beamer}
\usepackage[utf8]{inputenc}

\usetheme{Madrid}
\usecolortheme{default}
\usepackage{amsmath,amssymb,amsfonts,amsthm}
\usepackage{txfonts}
\usepackage{tkz-euclide}
\usepackage{listings}
\usepackage{adjustbox}
\usepackage{array}
\usepackage{tabularx}
\usepackage{gvv}
\usepackage{lmodern}
\usepackage{circuitikz}
\usepackage{tikz}
\usepackage{graphicx}
%\usepackage{enumerate}
%\usepackage{multicols}

\setbeamertemplate{page number in head/foot}[totalframenumber]

\usepackage{tcolorbox}
\tcbuselibrary{minted,breakable,xparse,skins}



\definecolor{bg}{gray}{0.95}
\DeclareTCBListing{mintedbox}{O{}m!O{}}{%
  breakable=true,
  listing engine=minted,
  listing only,
  minted language=#2,
  minted style=default,
  minted options={%
    linenos,
    gobble=0,
    breaklines=true,
    breakafter=,,
    fontsize=\small,
    numbersep=8pt,
    #1},
  boxsep=0pt,
  left skip=0pt,
  right skip=0pt,
  left=25pt,
  right=0pt,
  top=3pt,
  bottom=3pt,
  arc=5pt,
  leftrule=0pt,
  rightrule=0pt,
  bottomrule=2pt,
  toprule=2pt,
  colback=bg,
  colframe=orange!70,
  enhanced,
  overlay={%
    \begin{tcbclipinterior}
    \fill[orange!20!white] (frame.south west) rectangle ([xshift=20pt]frame.north west);
    \end{tcbclipinterior}},
  #3,
}
\lstset{
    language=C,
    basicstyle=\ttfamily\small,
    keywordstyle=\color{blue},
    stringstyle=\color{orange},
    commentstyle=\color{green!60!black},
    numbers=left,
    numberstyle=\tiny\color{gray},
    breaklines=true,
    showstringspaces=false,
}
%------------------------------------------------------------
%This block of code defines the information to appear in the
%Title page
\title %optional
{2.10.61}
\date{September 5, 2025}


\author 
{Kartik Lahoti - EE25BTECH11032}



\begin{document}


\frame{\titlepage}
\begin{frame}{Question}
If $\vec{a}$ and $\vec{b}$ are vectors such that $\mydet{\vec{a} + \vec{b}} = \sqrt{29}$ and
\begin{align*}
    \vec{a} \times \brak{2\hat{i} + 3\hat{j} + 4\hat{k}} = \brak{2\hat{i} + 3\hat{j} + 4\hat{k}} \times \vec{b}
\end{align*}
then a possible value of $\brak{\vec{a}+\vec{b}}\cdot\brak{-7\hat{i} + 2\hat{j} + 3\hat{k}}$ is 


  $\brak{1}$ 0 \hspace{2cm} $\brak{2}$ 3 \hspace{2cm} $\brak{3}$ 4 \hspace{2cm} $\brak{4}$ 8
\end{frame}



\begin{frame}{Theoretical Solution}
Given :
\begin{table}[H]
    \centering
    \begin{tabular}[12pt]{ |c| c|}
    \hline
    \textbf{Name} & \textbf{Point}\\ 
    \hline
	Point A &\myvec{h \\ k}\\
    \hline 
 Point B &\myvec{x1 \\ y1}\\
    \hline
	  Point R &\myvec{x2 \\ y2}\\
    \hline
    
    \end{tabular}

    \caption{2.10.61}
    \label{tab:placeholder_1}
\end{table}
\end{frame}

\begin{frame}{Theory}
\begin{align}
    \vec{a} \times \vec{c} &= \vec{c} \times \vec{b}\\
    \vec{a} \times \vec{c} &= - \brak{\vec{b} \times \vec{c}} \\
    \brak{\vec{a} + \vec{b}} \times \vec{c} &= \vec{0}  
\end{align}
\end{frame}
\begin{frame}{Theoretical Solution}

If cross product of 2 vectors is zero ,  this implies both the vectors are parallel. 
Thus , 
\begin{align}
    \brak{\vec{a} + \vec{b}} \parallel \vec{c} 
\end{align}

\begin{align}
   \therefore\brak{\vec{a} + \vec{b}} = \lambda\vec{c}  \text{, where } \lambda \in \mathbb{R}
\end{align}

\end{frame}

\begin{frame}{Theoretical Solution}
Equating the magnitudes , we get 
\begin{align}
  \norm{\brak{\vec{a} + \vec{b}}}^2 &= \lambda^2 \norm{\vec{c}}^2 \\
    29 &= \lambda^2 29 \\ 
    \lambda &= \pm 1 
 \end{align}
\end{frame}

\begin{frame}{Theoretical Solution}
Thus, 
\begin{align}
    \brak{\vec{a} + \vec{b}} = \myvec{2 \\ 3 \\ 4} \text{ or } \brak{\vec{a} + \vec{b}} = \myvec{-2 \\ -3 \\ -4}
\end{align}
Hence, 
\begin{align}
    \brak{\vec{a} + \vec{b}}^{\top}\vec{d} = 4 \text{ or } -4
\end{align}

Answer : Option $\brak{3}$
\end{frame}


\begin{frame}[fragile]
    \frametitle{C Code (1) }

    \begin{lstlisting}
#include <math.h>
double norm_vec_sq(double *A , int m )
{
    double sum = 0.0; 
    for ( int i = 0 ; i < m ; i++ )
    {
        sum += pow(A[i] , 2 );
    }
    return sum; 
}
    \end{lstlisting}
\end{frame}



\begin{frame}[fragile]
    \frametitle{C Code (2) - Function to Generate Points on Line}
    \begin{lstlisting}

void linegen(double *X, double *Y , double *Z , double *A , double *B , int n , int m )
{
    double temp[m] ; 
    for (int i = 0 ; i < m ; i++)
    {
        temp [ i ] = (B[i]- A[i]) /(double) n ; 
    }
    for (int i = 0 ; i <= n ; i++ )
    {
        X[i] = A[0] + temp[0] * i ; 
        Y[i] = A[1] + temp[1] * i ;
        Z[i] = A[2] + temp[2] * i ; 
    }
}

\end{lstlisting}
\end{frame}

\begin{frame}[fragile]
    \frametitle{Python Code - Using Shared Object}
    \begin{lstlisting}
import ctypes
import numpy as np
import matplotlib.pyplot as plt
handc1 = ctypes.CDLL("./func.so")

handc1.norm_vec_sq.argtypes = [
    ctypes.POINTER(ctypes.c_double),
    ctypes.c_int]

handc1.norm_vec_sq.restype = ctypes.c_double
C = np.array([[2],[3],[4]], dtype= np.float64).reshape(-1,1)
ab_sq = 29
m = 3
c_sq = handc1.norm_vec_sq(
    C.ctypes.data_as(ctypes.POINTER(ctypes.c_double)),m)
\end{lstlisting}
\end{frame}

\begin{frame}[fragile]
    \frametitle{Python Code - Using Shared Object}
    \begin{lstlisting}
l = ab_sq / c_sq
ab1 = np.sqrt(l) * C
ab2 = -np.sqrt(l) * C

def line_cre(P: np.ndarray , Q: np.ndarray, str):
    handc2 = ctypes.CDLL("./line_gen.so")

    handc2.linegen.argtypes = [
        ctypes.POINTER(ctypes.c_double),
        ctypes.POINTER(ctypes.c_double),
        ctypes.POINTER(ctypes.c_double),
        ctypes.POINTER(ctypes.c_double),
        ctypes.POINTER(ctypes.c_double),
        ctypes.c_int , ctypes.c_int
    ]
\end{lstlisting}
\end{frame}

\begin{frame}[fragile]
    \frametitle{Python Code - Using Shared Object}
    \begin{lstlisting}
 handc2.linegen.restype = None
    n = 200
    X_l = np.zeros(n,dtype=np.float64)
    Y_l = np.zeros(n,dtype=np.float64)
    Z_l = np.zeros(n,dtype=np.float64)
    handc2.linegen (
        X_l.ctypes.data_as(ctypes.POINTER(ctypes.c_double)),
        Y_l.ctypes.data_as(ctypes.POINTER(ctypes.c_double)),
        Z_l.ctypes.data_as(ctypes.POINTER(ctypes.c_double)),
        P.ctypes.data_as(ctypes.POINTER(ctypes.c_double)),
        Q.ctypes.data_as(ctypes.POINTER(ctypes.c_double)),
        n,3
    )
    ax.plot([X_l[0],X_l[-1]],[Y_l[0],Y_l[-1]],[Z_l[0],Z_l[-1]],str)
\end{lstlisting}
\end{frame}
\begin{frame}[fragile]
    \frametitle{Python Code - Using Shared Object}
    \begin{lstlisting}
O = np.array([[0],[0],[0]]).reshape(-1,1)
fig = plt.figure()
ax = fig.add_subplot(111,projection="3d")

line_cre(ab1,O,"g-")
line_cre(ab2,O,"r-")

coords = np.block([[ab1,ab2,O]])
ax.scatter(coords[0,:],coords[1,:],coords[2,:])
vert_labels = [r'$(a+b)_1$',r'$(a+b)_2$','O']
    \end{lstlisting}
\end{frame}

\begin{frame}[fragile]
    \frametitle{Python Code - Using Shared Object}
    \begin{lstlisting}
for i, txt in enumerate(vert_labels):
    if (coords[0,i] == 0 ) :
        ax.text(coords[0,i], coords[1,i] , coords[2,i],txt , ha='center', va = 'bottom')
    else :
        ax.text(coords[0,i], coords[1,i] , coords[2,i],f'{txt}\n({coords[0,i]:.1f}, {coords[1,i]:.1f}, {coords[2,i]:.1f})',ha='center', va = 'bottom')
ax.scatter(coords[0,2], coords[1,2], coords[2,2], color="b", label="O : ORIGIN")
\end{lstlisting}
\end{frame}

\begin{frame}[fragile]
    \frametitle{Python Code - Using Shared Object}
    \begin{lstlisting}
ax.legend(loc = "best")
ax.set_xlabel('$x$')
ax.set_ylabel('$y$')
ax.set_zlabel('$z$')
ax.grid()
plt.title("Fig:2.10.61")
ax.set_box_aspect([1,1,1])

fig.savefig("../figs/vector1.png")
fig.show()

#plt.savefig('figs/triangle/ang-bisect.pdf')
#subprocess.run(shlex.split("termux-open figs/triangle/ang-bisect.pdf"))
\end{lstlisting}
\end{frame}

\begin{frame}[fragile]
    \frametitle{Python Code}
    \begin{lstlisting}
import math
import sys 
sys.path.insert(0, '/home/kartik-lahoti/matgeo/codes/CoordGeo')
import numpy as np
import numpy.linalg as LA
import matplotlib.pyplot as plt
import matplotlib.image as mpimg

from line.funcs import *
#from triangle.funcs import *
#from conics.funcs import circ_gen


#if using termux
#import subprocess
#import shlex
\end{lstlisting}
\end{frame}

\begin{frame}[fragile]
    \frametitle{Python Code }
    \begin{lstlisting}

C = np.array([[2],[3],[4]], dtype = np.float64 ).reshape(-1,1)
ab_sq= 29

c_sq = LA.norm(C)**2

l = ab_sq / c_sq

ab1 = np.sqrt(l) * C
ab2 = - np.sqrt(l) * C
O = np.array([0,0,0]).reshape(-1,1)

\end{lstlisting}
\end{frame}

\begin{frame}[fragile]
    \frametitle{Python Code }
    \begin{lstlisting}
def plot_it(P,Q,str):
    x_l = line_gen_num(P,Q,20)
    ax.plot(x_l[0,:],x_l[1,:],x_l[2,:] , str )

fig = plt.figure()
ax = fig.add_subplot(111,projection = "3d")

plot_it(ab1,O,"g-")
plot_it(ab2,O,"r-")

coords = np.block([[ab1,ab2,O]])
plt.scatter(coords[0,:],coords[1,:],coords[2,:])
vert_labels = [r'$(a+b)_1$',r'$(a+b)_2$','O']
\end{lstlisting}
\end{frame}

\begin{frame}[fragile]
    \frametitle{Python Code }
    \begin{lstlisting}

for i, txt in enumerate(vert_labels):
    if (coords[0,i] == 0 ) :
        ax.text(coords[0,i], coords[1,i] , coords[2,i],txt , ha='center', va = 'bottom')
    else :
        ax.text(coords[0,i], coords[1,i] , coords[2,i],f'{txt}\n({coords[0,i]:.1f}, {coords[1,i]:.1f}, {coords[2,i]:.1f})',ha='center', va = 'bottom')

ax.scatter(coords[0,2], coords[1,2], coords[2,2], color="b", label="O : ORIGIN")
ax.legend(loc = "best")


    \end{lstlisting}
\end{frame}
\begin{frame}[fragile]
    \frametitle{Python Code }
    \begin{lstlisting}

    
ax.set_xlabel('$x$')
ax.set_ylabel('$y$')
ax.set_zlabel('$z$')

ax.grid()
plt.title("Fig:2.8.15")
ax.set_box_aspect([1,1,1])

fig.savefig("../figs/vector2.png")
fig.show()
#plt.savefig('figs/triangle/ang-bisect.pdf')
#subprocess.run(shlex.split("termux-open figs/triangle/ang-bisect.pdf"))

    \end{lstlisting}
\end{frame}


\begin{frame}{Plot}
    \centering
    \includegraphics[width=\columnwidth, height=0.8\textheight, keepaspectratio]{figs/vector1.png}   
\end{frame}


\end{document}
