\documentclass{beamer}
\mode<presentation>
\usepackage{amsmath}
\usepackage{amssymb}
%\usepackage{advdate}
\usepackage{graphicx}
\graphicspath{{../figs/}}
\usepackage{adjustbox}
\usepackage{subcaption}
\usepackage{enumitem}
\usepackage{multicol}
\usepackage{mathtools}
\usepackage{listings}
\usepackage{url}
\def\UrlBreaks{\do\/\do-}
\usetheme{Boadilla}
\usecolortheme{lily}
\setbeamertemplate{footline}
{
  \leavevmode%
  \hbox{%
  \begin{beamercolorbox}[wd=\paperwidth,ht=2.25ex,dp=1ex,right]{author in head/foot}%
    \insertframenumber{} / \inserttotalframenumber\hspace*{2ex} 
  \end{beamercolorbox}}%
  \vskip0pt%
}
\setbeamertemplate{navigation symbols}{}
\let\solution\relax
\usepackage{gvv}
\lstset{
%language=C,
frame=single, 
breaklines=true,
columns=fullflexible
}

\numberwithin{equation}{section}



\begin{document}

\title{8.4.23}
\author{EE25BTECH11020 - Darsh Pankaj Gajare}
% \maketitle
% \newpage
% \bigskip
%\begin{document}
{\let\newpage\relax\maketitle}
%\renewcommand{\thefigure}{\theenumi}
%\renewcommand{\thetable}{\theenumi}

Question:\\
The curve described parametrically by $x = t^2 + t + 1$ and $y = t^2 - t + 1$ represents:
\begin{multicols}{2}
	\begin{enumerate}[label=(\Alph*)]
\item a pair of straight lines
\item an ellipse
\item a parabola
\item a hyperbola
\end{enumerate}
\end{multicols}


\solution
\begin{table}[H]
	\centering
	\caption{}
	\begin{tabular}[12pt]{ |c| c|}
    \hline
    \textbf{Name} & \textbf{Point}\\ 
    \hline
	Point A &\myvec{h \\ k}\\
    \hline 
 Point B &\myvec{x1 \\ y1}\\
    \hline
	  Point R &\myvec{x2 \\ y2}\\
    \hline
    
    \end{tabular}

	\label{}
\end{table}

The parametric form can be written as
\begin{align}
\vec{x} &= \vec{a}t^2 + \vec{b}t + \vec{c}.
\end{align}
\begin{align}
	\vec{x}=\myvec{\vec{a}\\\vec{b}}^\top\myvec{t^2\\t}+\vec{c}
\end{align}
\begin{align}
\myvec{x\\y}
&= 
\myvec{1 & 1\\[4pt]1 & -1}
\myvec{t^2\\t} + \myvec{1\\1}.
\end{align}

Solving for $\myvec{t^2\\t}$ using the inverse matrix,
\begin{align}
\myvec{t^2\\t}
&= 
\frac{1}{2}
\myvec{1 & 1\\[4pt]1 & -1}
^{-1}
\brak{\myvec{x\\y}-\myvec{1\\1}}\\
&= 
\frac{1}{2}
\myvec{1 & 1\\[4pt]1 & -1}
\brak{\myvec{x\\y}-\myvec{1\\1}}.
\end{align}

Multiplying,
\begin{align}
\myvec{t^2\\t}
&= 
\frac{1}{2}
\myvec{x+y-2\\[4pt]x-y}.
\end{align}

\text{Eliminating } t:
\begin{align}
\myvec{t^2\\t}
= \frac{1}{2}\myvec{x+y-2\\x-y}
\implies
\frac{1}{2}(x+y-2)
= \brak{\frac{1}{2}(x-y)}^2
\end{align}
\begin{align}
\Rightarrow (x-y)^2 = 2(x+y-2)
\end{align}

\begin{align}
\text{Write as quadratic form:}\quad
\vec{x}^\top \, \vec{V}\, \vec{x} + 2\vec{u}^\top\vec{x} + f &= 0
\end{align}

\begin{align}
\vec{V} &= \myvec{1 & -1\\[4pt]-1 & 1}, &
\vec{u} &= \myvec{-1\\[4pt]-1}, &
f = 4
\end{align}

Extract quadratic coefficients:
\begin{align}
A &= V_{11} = 1, & B &= 2V_{12} = -2, & C &= V_{22} = 1
\end{align}
Discriminant:
\begin{align}
\Delta &= B^2 - 4AC = (-2)^2 - 4(1)(1) = 0
\end{align}

Since $\Delta=0$ the conic is a parabola.
Plot using C libraries:
\begin{figure}[H]
	\centering
	\includegraphics[scale=0.5]{img1}
	\caption*{}
	\label{img1}
\end{figure}
Plot using Python:
\begin{figure}[H]
	\centering
	\includegraphics[scale=0.5]{img2}
	\caption*{}
	\label{img2}
\end{figure}
\end{document}

