\let\negmedspace\undefined
\let\negthickspace\undefined
\documentclass[journal,12pt,onecolumn]{IEEEtran}
\usepackage{cite}
\usepackage{amsmath,amssymb,amsfonts,amsthm}
\usepackage{algorithmic}
\usepackage{graphicx}
\graphicspath{{./figs/}}
\usepackage{textcomp}
\usepackage{xcolor}
\usepackage{txfonts}
\usepackage{listings}
\usepackage{enumitem}
\usepackage{mathtools}
\usepackage{gensymb}
\usepackage{comment}
\usepackage{caption}
\usepackage[breaklinks=true]{hyperref}
\usepackage{tkz-euclide} 
\usepackage{listings}
\usepackage{gvv}                                        
%\def\inputGnumericTable{}                                 
\usepackage[latin1]{inputenc}     
\usepackage{xparse}
\usepackage{color}                                            
\usepackage{array}                                            
\usepackage{longtable}                                       
\usepackage{calc}                                             
\usepackage{multirow}
\usepackage{multicol}
\usepackage{hhline}                                           
\usepackage{ifthen}                                           
\usepackage{lscape}
\usepackage{tabularx}
\usepackage{array}
\usepackage{float}
%\newtheorem{theorem}{Theorem}[section]
%\newtheorem{theorem}{Theorem}[section]
%\newtheorem{problem}{Problem}
%\newtheorem{proposition}{Proposition}[section]
%\newtheorem{lemma}{Lemma}[section]
%\newtheorem{corollary}[theorem]{Corollary}
%\newtheorem{example}{Example}[section]
%\newtheorem{definition}[problem]{Definition}
\usepackage{listings}
\usepackage{xcolor}

\lstset{
  language=C,
  basicstyle=\ttfamily\footnotesize,
  keywordstyle=\color{blue},
  commentstyle=\color{green!50!black},
  numbers=left,
  numberstyle=\tiny\color{gray},
  stepnumber=1,
  frame=single
}

\begin{document}

\title{5.13.46}
\author{EE25BTECH11020 - Darsh Pankaj Gajare}
% \maketitle
% \newpage
% \bigskip
%\begin{document}
{\let\newpage\relax\maketitle}
%\renewcommand{\thefigure}{\theenumi}
%\renewcommand{\thetable}{\theenumi}
Question:\\Consider the set $A$ of all determinants of order $3$ with entries $0$ or $1$ only. Let $B$ be the subset of $A$ consisting of all determinants with value $1$. Let $C$ be the subset of $A$ consisting of all determinants with value $-1$. 

Then

\begin{multicols}{2}
\begin{enumerate}
    \item $C$ is empty
    \item $B$ has as many elements as $C$
    \item $A = B \cup C$
    \item $B$ has twice as many elements as $C$
\end{enumerate}
\end{multicols}
\solution

Let $\vec{A}$ be
\begin{align}
	\vec{A}=\myvec{a_{11}&a_{12}&a_{13}\\a_{21}&a_{22}&a_{23}\\a_{31}&a_{32}&a_{33}}
\end{align}
where $a_{ij}\in \cbrak{0,1}$ 
\begin{align}
\det(A)\in\{-2,-1,0,1,2\}.
\end{align}


\textbf{ Cases}
\begin{align}
|\det|=2 &\;\;\Rightarrow\;\;
\myvec{0&1&1\\1&0&1\\1&1&0},\;
\myvec{1&0&1\\1&1&0\\0&1&1},\;
\myvec{1&1&0\\0&1&1\\1&0&1}, \\[1ex]
&\quad 3 \;\text{with } \det=2,\;
3 \;\text{with } \det=-2.
\end{align}

\begin{align}
|\det|=1 &\;\;\Rightarrow\;\;
	\brak{2^3-1}\brak{2^3-2}\brak{2^3-4}=168
=84 \;(+1),\; 84 \;(-1).
\end{align}

\begin{align}
\det=0 &\;\;\Rightarrow\;\; 512-(168+6)=338.
\end{align}

\textbf{ Distribution}
\begin{align}
-2 \implies 3 \\
-1 \implies 84 \\
0 \implies 338 \\
1 \implies 84 \\
2 \implies 3
\end{align}
\textbf{Answer:} (b),
\begin{lstlisting}[caption={C code}]
#include <stdio.h>
int det3(int m[3][3]) {
    return m[0][0]*m[1][1]*m[2][2]
         + m[0][1]*m[1][2]*m[2][0]
         + m[0][2]*m[1][0]*m[2][1]
         - m[0][2]*m[1][1]*m[2][0]
         - m[0][0]*m[1][2]*m[2][1]
         - m[0][1]*m[1][0]*m[2][2];
}

void compute_counts(int counts[7]) {
    int mat[3][3];
    for (int i = 0; i < 7; i++) counts[i] = 0;

    for (int mask = 0; mask < (1<<9); mask++) {
        for (int i = 0; i < 9; i++) {
            mat[i/3][i%3] = (mask >> i) & 1;
        }
        int d = det3(mat);
        counts[d+3]++;
    }
}
\end{lstlisting}
\end{document}

