\documentclass{beamer}
\mode<presentation>
\usepackage{amsmath}
\usepackage{amssymb}
%\usepackage{advdate}
\usepackage{graphicx}
\graphicspath{{../figs/}}
\usepackage{adjustbox}
\usepackage{subcaption}
\usepackage{enumitem}
\usepackage{multicol}
\usepackage{mathtools}
\usepackage{listings}
\usepackage{url}
\def\UrlBreaks{\do\/\do-}
\usetheme{Boadilla}
\usecolortheme{lily}
\setbeamertemplate{footline}
{
  \leavevmode%
  \hbox{%
  \begin{beamercolorbox}[wd=\paperwidth,ht=2.25ex,dp=1ex,right]{author in head/foot}%
    \insertframenumber{} / \inserttotalframenumber\hspace*{2ex} 
  \end{beamercolorbox}}%
  \vskip0pt%
}
\setbeamertemplate{navigation symbols}{}
\let\solution\relax
\usepackage{gvv}
\lstset{
%language=C,
frame=single, 
breaklines=true,
columns=fullflexible
}

\numberwithin{equation}{section}



\begin{document}

\title{7.4.27}
\author{EE25BTECH11020 - Darsh Pankaj Gajare}
% \maketitle
% \newpage
% \bigskip
%\begin{document}
{\let\newpage\relax\maketitle}
%\renewcommand{\thefigure}{\theenumi}
%\renewcommand{\thetable}{\theenumi}


Question:\\
The triangle $PQR$ is inscribed in the circle $x^2+y^2=25$.If $\vec{Q}$ and $\vec{R}$ have co-ordinates $\brak{3,4}$ and $\brak{-4,3}$ respectively then $\angle QPR$ is equal to
\begin{multicols}{4}
	\begin{enumerate}[label=(\Alph*)]
		\item $\frac{\pi}{2}$
		\item $\frac{\pi}{3}$
		\item $\frac{\pi}{4}$
		\item $\frac{\pi}{6}$
	\end{enumerate}
\end{multicols}
\solution
\begin{table}[H]
	\centering
	\caption{}
	\begin{tabular}[12pt]{ |c| c|}
    \hline
    \textbf{Name} & \textbf{Point}\\ 
    \hline
	Point A &\myvec{h \\ k}\\
    \hline 
 Point B &\myvec{x1 \\ y1}\\
    \hline
	  Point R &\myvec{x2 \\ y2}\\
    \hline
    
    \end{tabular}

	\label{}
\end{table}
\begin{align}
\vec{x}^\top\vec{x} &= 25
\end{align}

The given points (position vectors) are
\begin{align}
\vec{q} &= \myvec{3\\4}, & \vec{r} &= \myvec{-4\\3}
\end{align}

Verify they lie on the circle:
\begin{align}
\vec{q}^\top\vec{q} &= 3^2+4^2 = 25, \\
\vec{r}^\top\vec{r} &= (-4)^2+3^2 = 25.
\end{align}

Compute the inner product (matrix/dot product):
\begin{align}
\vec{q}^\top\vec{r} &= \myvec{3 & 4}\myvec{-4\\3} \\
&= 3\cdot(-4) + 4\cdot 3 = -12 + 12 = 0.
\end{align}

Compute norms (using matrix notation) and the central angle $\theta$:
\begin{align}
\norm{\vec{q}} &= \sqrt{\vec{q}^\top\vec{q}} = 5, & 
\norm{\vec{r}} &= \sqrt{\vec{r}^\top\vec{r}} = 5, \\
\cos\theta &= \dfrac{\vec{q}^\top\vec{r}}{\norm{\vec{q}}\norm{\vec{r}}}
= \dfrac{0}{5\cdot 5} = 0 \\
\implies \theta &= \frac{\pi}{2}.
\end{align}

Since $\angle QPR$ is the angle subtended at the circumference by chord $QR$, it equals half the central angle:
\begin{align}
\angle QPR &= \frac{\theta}{2} = \frac{\pi}{4}.
\end{align}

\begin{figure}[H]
	\centering
	\includegraphics[scale=0.25]{img1}
	\caption*{Plot using C libraries}
	\label{img1}
\end{figure}
\begin{figure}[H]
	\centering
	\includegraphics[scale=0.25]{img2}
	\caption*{Plot using Python}
	\label{img2}
\end{figure}
\end{document}

