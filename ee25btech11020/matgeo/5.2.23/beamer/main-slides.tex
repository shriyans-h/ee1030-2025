\documentclass{beamer}
\mode<presentation>
\usepackage{amsmath}
\usepackage{amssymb}
%\usepackage{advdate}
\usepackage{graphicx}
\graphicspath{{../figs/}}
\usepackage{adjustbox}
\usepackage{subcaption}
\usepackage{enumitem}
\usepackage{multicol}
\usepackage{mathtools}
\usepackage{listings}
\usepackage{url}
\def\UrlBreaks{\do\/\do-}
\usetheme{Boadilla}
\usecolortheme{lily}
\setbeamertemplate{footline}
{
  \leavevmode%
  \hbox{%
  \begin{beamercolorbox}[wd=\paperwidth,ht=2.25ex,dp=1ex,right]{author in head/foot}%
    \insertframenumber{} / \inserttotalframenumber\hspace*{2ex} 
  \end{beamercolorbox}}%
  \vskip0pt%
}
\setbeamertemplate{navigation symbols}{}
\let\solution\relax
\usepackage{gvv}
\lstset{
%language=C,
frame=single, 
breaklines=true,
columns=fullflexible
}

\numberwithin{equation}{section}



\begin{document}

\title{5.2.23}
\author{EE25BTECH11020 - Darsh Pankaj Gajare}
% \maketitle
% \newpage
% \bigskip
%\begin{document}
{\let\newpage\relax\maketitle}
%\renewcommand{\thefigure}{\theenumi}
%\renewcommand{\thetable}{\theenumi}
Question:\\
Solve the following system of linear equations
\begin{align}
	\frac{3x}{2} - \frac{5y}{2} = -2, \frac{x}{3} + \frac{y}{2}=\frac{13}{6}
\end{align}
\solution
\begin{table}[H]
	\centering
	\caption{}
	\begin{tabular}[12pt]{ |c| c|}
    \hline
    \textbf{Name} & \textbf{Point}\\ 
    \hline
	Point A &\myvec{h \\ k}\\
    \hline 
 Point B &\myvec{x1 \\ y1}\\
    \hline
	  Point R &\myvec{x2 \\ y2}\\
    \hline
    
    \end{tabular}

	\label{}
\end{table}
Let the point of intersection be $\vec{P}$
\begin{align}
	\vec{n_1}^\top\vec{P}=-2
\end{align}
\begin{align}
	\vec{n_2}^\top\vec{P}=\frac{13}{6}
\end{align}
\begin{align}
	\myvec{\vec{n_1}^\top\\\vec{n_2}^\top}\vec{P}=\myvec{-2\\\frac{13}{6}}
\end{align}
\begin{align}
	\augvec{2}{1}{\frac{3}{2} & \frac{-5}{2} & -2\\ \frac{1}{3} &\frac{1}{2} &\frac{13}{6}}
\end{align}
$R_1=2R_1$, $R_2=6R_2$
\begin{align}
	\augvec{2}{1}{3&-5&-4\\2&3&13}
\end{align}
$R_2=R_2-\frac{2}{3}R_1$
\begin{align}
	\augvec{2}{1}{3&-5&-4\\0&\frac{19}{3}&\frac{47}{3}}
\end{align}
\begin{align}
	\frac{19}{3}y=\frac{47}{3} \implies y=\frac{47}{19}
\end{align}
\begin{align}
	3x - 5\cdot\frac{47}{19} = -4 \implies x= \frac{53}{19}
\end{align}
\begin{align}
	\vec{P}=\myvec{\frac{53}{19}\\\frac{47}{19}}
\end{align}
\begin{figure}[H]
	\centering
	\includegraphics[scale=0.25]{img1}
	\caption*{Plot using C libraries}
	\label{img1}
\end{figure}
\begin{figure}[H]
	\centering
	\includegraphics[scale=0.25]{img2}
	\caption*{Plot using Python}
	\label{img2}
\end{figure}
\end{document}

