\let\negmedspace\undefined
\let\negthickspace\undefined
\documentclass[journal,12pt,onecolumn]{IEEEtran}
\usepackage{cite}
\usepackage{amsmath,amssymb,amsfonts,amsthm}
\usepackage{algorithmic}
\usepackage{graphicx}
\graphicspath{{./figs/}}
\usepackage{textcomp}
\usepackage{xcolor}
\usepackage{txfonts}
\usepackage{listings}
\usepackage{enumitem}
\usepackage{mathtools}
\usepackage{gensymb}
\usepackage{comment}
\usepackage{caption}
\usepackage[breaklinks=true]{hyperref}
\usepackage{tkz-euclide} 
\usepackage{listings}
\usepackage{gvv}                                        
%\def\inputGnumericTable{}                                 
\usepackage[latin1]{inputenc}     
\usepackage{xparse}
\usepackage{color}                                            
\usepackage{array}                                            
\usepackage{longtable}                                       
\usepackage{calc}                                             
\usepackage{multirow}
\usepackage{multicol}
\usepackage{hhline}                                           
\usepackage{ifthen}                                           
\usepackage{lscape}
\usepackage{tabularx}
\usepackage{array}
\usepackage{float}
%\newtheorem{theorem}{Theorem}[section]
%\newtheorem{theorem}{Theorem}[section]
%\newtheorem{problem}{Problem}
%\newtheorem{proposition}{Proposition}[section]
%\newtheorem{lemma}{Lemma}[section]
%\newtheorem{corollary}[theorem]{Corollary}
%\newtheorem{example}{Example}[section]
%\newtheorem{definition}[problem]{Definition}
\usepackage{listings}
\usepackage{xcolor}

\lstset{
  language=C,
  basicstyle=\ttfamily\footnotesize,
  keywordstyle=\color{blue},
  commentstyle=\color{green!50!black},
  numbers=left,
  numberstyle=\tiny\color{gray},
  stepnumber=1,
  frame=single
}

\begin{document}

\title{12.34}
\author{EE25BTECH11020 - Darsh Pankaj Gajare}
% \maketitle
% \newpage
% \bigskip
%\begin{document}
{\let\newpage\relax\maketitle}
%\renewcommand{\thefigure}{\theenumi}
%\renewcommand{\thetable}{\theenumi}
Question:\\ Let $\vec{A}=10\vec{I}_3$ where $\vec{I}_3$ is the $3\times3$ identity matrix. Find the nullity of $5\vec{A}\brak{\vec{I}_3+\vec{A}+\vec{A}^2}$.
\solution
\begin{align}
	\vec{A}^2=100\vec{I}_3
\end{align}
\begin{align}
	\vec{M}=	5\vec{A}\brak{\vec{I}_3+\vec{A}+\vec{A}^2}=5\brak{10\vec{I}_3}\brak{\vec{I}_3+10\vec{A}+100\vec{A}}=5550\vec{I}_3
\end{align}
Solve $\vec{M}\vec{x}=0$
\begin{align}
	5550\vec{I}_3\vec{x}=0 \implies 5550\vec{x}=0
\end{align}
Since 5550 is a nonzero scalar, the only vector satisfying this is the zero vector:
\begin{align}
	\vec{x}=0
\end{align}
The nullspace contains only the zero vector, so its dimension is 0. Therefore
\begin{align}
	nullity\brak{5\vec{A}\brak{\vec{I}_3+\vec{A}+\vec{A}^2}}=0
\end{align}
\begin{lstlisting}[caption=nullity.c]
#include <stdio.h>

int find_nullity(double k) {
    double scalar = 5 * k * (1 + k + k * k);
    int n = 3;  // 3x3 matrix
    int nullity = 0;

    if (scalar == 0)
        nullity = n;
    else
        nullity = 0;

    return nullity;
}
\end{lstlisting}
\end{document}

