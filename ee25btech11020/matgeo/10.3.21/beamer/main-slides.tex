\documentclass{beamer}
\mode<presentation>
\usepackage{amsmath}
\usepackage{amssymb}
%\usepackage{advdate}
\usepackage{graphicx}
\graphicspath{{../figs/}}
\usepackage{adjustbox}
\usepackage{subcaption}
\usepackage{enumitem}
\usepackage{multicol}
\usepackage{mathtools}
\usepackage{listings}
\usepackage{url}
\def\UrlBreaks{\do\/\do-}
\usetheme{Boadilla}
\usecolortheme{lily}
\setbeamertemplate{footline}
{
  \leavevmode%
  \hbox{%
  \begin{beamercolorbox}[wd=\paperwidth,ht=2.25ex,dp=1ex,right]{author in head/foot}%
    \insertframenumber{} / \inserttotalframenumber\hspace*{2ex} 
  \end{beamercolorbox}}%
  \vskip0pt%
}
\setbeamertemplate{navigation symbols}{}
\let\solution\relax
\usepackage{gvv}
\lstset{
%language=C,
frame=single, 
breaklines=true,
columns=fullflexible
}

\numberwithin{equation}{section}



\begin{document}

\title{10.3.21}
\author{EE25BTECH11020 - Darsh Pankaj Gajare}
% \maketitle
% \newpage
% \bigskip
%\begin{document}
{\let\newpage\relax\maketitle}
%\renewcommand{\thefigure}{\theenumi}
%\renewcommand{\thetable}{\theenumi}


Question:\\
Find the point at which the line $y=x+1$ is a tangent to the curve $y^2=4x$.\\
\solution
The given conic can be expressed as
\begin{align}
\vec{x}^\top V\vec{x} + 2\vec{u}^\top\vec{x} + f = 0
\end{align}
where
\begin{align}
V = \myvec{0 & 0 \\ 0 & 1}, \quad
\vec{u} = \myvec{-2 \\ 0}, \quad
f = 0
\end{align}
and $\vec{x} = \myvec{x \\ y}$.

The given line is 
\begin{align}
y = x + 1
\end{align}
which can be parameterized as
\begin{align}
\vec{x} = \vec{h} + t\vec{m}
\end{align}
where
\begin{align}
\vec{h} = \myvec{0 \\ 1}, \quad
\vec{m} = \myvec{1 \\ 1}.
\end{align}

Substituting $\vec{x} = \vec{h} + t\vec{m}$ in the conic equation,
\begin{align}
(\vec{h} + t\vec{m})^\top V (\vec{h} + t\vec{m})
+ 2\vec{u}^\top (\vec{h} + t\vec{m}) + f = 0
\end{align}

Expanding,
\begin{align}
t^2\brak{\vec{m}^\top V\vec{m}} 
+ 2t\brak{\vec{m}^\top V\vec{h} + \vec{u}^\top\vec{m}} 
+ \brak{\vec{h}^\top V\vec{h} + 2\vec{u}^\top\vec{h} + f} = 0
\end{align}

Compute each term:
\begin{align}
\vec{m}^\top V\vec{m} &= 1, &
\vec{m}^\top V\vec{h} &= 1, &
\vec{u}^\top\vec{m} &= -2, \\
\vec{h}^\top V\vec{h} &= 1, &
\vec{u}^\top\vec{h} &= 0
\end{align}

Substituting,
\begin{align}
t^2 - 2t + 1 = 0
\end{align}

\begin{align}
\Rightarrow (t - 1)^2 = 0 \implies t = 1
\end{align}

Hence, the point of contact is
\begin{align}
\vec{q} = \vec{h} + t\vec{m} 
\end{align}

\begin{align}
 \vec{q} = \myvec{1 \\ 2}.
\end{align}

\begin{figure}[H]
	\centering
	\includegraphics[scale=0.5]{img1}
	\caption*{Plot using C libraries:}
	\label{img1}
\end{figure}
\begin{figure}[H]
	\centering
	\includegraphics[scale=0.5]{img2}
	\caption*{
		Plot using Python:}
	\label{img2}
\end{figure}
\end{document}

