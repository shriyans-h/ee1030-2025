\documentclass{beamer}
\mode<presentation>
\usepackage{amsmath}
\usepackage{amssymb}
%\usepackage{advdate}
\usepackage{graphicx}
\graphicspath{{../figs/}}
\usepackage{adjustbox}
\usepackage{subcaption}
\usepackage{enumitem}
\usepackage{multicol}
\usepackage{mathtools}
\usepackage{listings}
\usepackage{url}
\def\UrlBreaks{\do\/\do-}
\usetheme{Boadilla}
\usecolortheme{lily}
\setbeamertemplate{footline}
{
  \leavevmode%
  \hbox{%
  \begin{beamercolorbox}[wd=\paperwidth,ht=2.25ex,dp=1ex,right]{author in head/foot}%
    \insertframenumber{} / \inserttotalframenumber\hspace*{2ex} 
  \end{beamercolorbox}}%
  \vskip0pt%
}
\setbeamertemplate{navigation symbols}{}

\providecommand{\nCr}[2]{\,^{#1}C_{#2}} % nCr
\providecommand{\nPr}[2]{\,^{#1}P_{#2}} % nPr
\providecommand{\mbf}{\mathbf}
\providecommand{\pr}[1]{\ensuremath{\Pr\left(#1\right)}}
\providecommand{\qfunc}[1]{\ensuremath{Q\left(#1\right)}}
\providecommand{\sbrak}[1]{\ensuremath{{}\left[#1\right]}}
\providecommand{\lsbrak}[1]{\ensuremath{{}\left[#1\right.}}
\providecommand{\rsbrak}[1]{\ensuremath{{}\left.#1\right]}}
\providecommand{\brak}[1]{\ensuremath{\left(#1\right)}}
\providecommand{\lbrak}[1]{\ensuremath{\left(#1\right.}}
\providecommand{\rbrak}[1]{\ensuremath{\left.#1\right)}}
\providecommand{\cbrak}[1]{\ensuremath{\left\{#1\right\}}}
\providecommand{\lcbrak}[1]{\ensuremath{\left\{#1\right.}}
\providecommand{\rcbrak}[1]{\ensuremath{\left.#1\right\}}}
\theoremstyle{remark}
\newtheorem{rem}{Remark}
\newcommand{\sgn}{\mathop{\mathrm{sgn}}}
\providecommand{\abs}[1]{$\left\vert#1\right\vert$}
\providecommand{\res}[1]{\Res\displaylimits_{#1}} 
\providecommand{\norm}[1]{\lVert#1\rVert}
%\providecommand{\mtx}[1]{\mathbf{#1}}
\providecommand{\mean}[1]{E$\left[ #1 \right]$}
\providecommand{\fourier}{\overset{\mathcal{F}}{ \rightleftharpoons}}
%\providecommand{\hilbert}{\overset{\mathcal{H}}{ \rightleftharpoons}}
\providecommand{\system}[1]{\overset{\mathcal{#1}}{ \longleftrightarrow}}
%\providecommand{\system}{\overset{\mathcal{H}}{ \longleftrightarrow}}
	%\newcommand{\solution}[2]{\textbf{Solution:}{#1}}
%\newcommand{\solution}{\noindent \textbf{Solution: }}
\providecommand{\dec}[2]{\ensuremath{\overset{#1}{\underset{#2}{\gtrless}}}}
\newcommand{\myvec}[1]{\ensuremath{\begin{pmatrix}#1\end{pmatrix}}}
\let\vec\mathbf

\lstset{
%language=C,
frame=single, 
breaklines=true,
columns=fullflexible
}

\numberwithin{equation}{section}

\title{2.10.49}
\author{EE25BTECH11020 - Darsh Pankaj Gajare}
\begin{document}
\maketitle
% \newpage
% \bigskip
Question:\\
The unit vector which is orthogonal to the vector $3\hat{i}+2\hat{j} +6\hat{k}$ and is coplanar with vectors $2\hat{i}+\hat{j}+\hat{k}$ and $\hat{i}-\hat{j}+\hat{k}$ is
\begin{multicols}{4}
	\begin{enumerate}[label=(\Alph*)]
\item $ \frac{2 \hat{i} - 6 \hat{j} + \hat{k}}{\sqrt{41}}$
\item $\frac{2\hat{i}-3\hat{j}}{\sqrt{13}}$
\item $\frac{3\hat{i}-\hat{k}}{\sqrt{10}}$
\item $\frac{4\hat{i}+3\hat{j}-3\hat{k}}{\sqrt{34}}$
\end{enumerate}
\end{multicols}
\textbf{Solution:}
Given:
\begin{table}[H]
	\centering
	\label{}
	\caption{Given data}
	\begin{tabular}[12pt]{ |c| c|}
    \hline
    \textbf{Name} & \textbf{Point}\\ 
    \hline
	Point A &\myvec{h \\ k}\\
    \hline 
 Point B &\myvec{x1 \\ y1}\\
    \hline
	  Point R &\myvec{x2 \\ y2}\\
    \hline
    
    \end{tabular}

\end{table}
To find: $\vec{P}$

\begin{align}
\vec{P} = \alpha\vec{A} + \beta\vec{B} 
\end{align}

\begin{align}
\vec{P}^T \vec{C} &= 0 
\end{align}
\begin{align}
    \myvec{\vec{A} & \vec{B}}^T \vec{C} \myvec{\alpha\\\beta} &= 0
\end{align}
\begin{align}
\myvec{\vec{A}^T \vec{C} & \vec{B}^T \vec{C}}\myvec{\alpha\\\beta} = 0
\end{align}
\begin{align}
 \myvec{\vec{A}^T \vec{C} & \vec{B}^T \vec{C}} = \myvec{14 & 7}
\end{align}

\begin{align}
 2\alpha+\beta = 0 \implies \beta = -2\alpha
\end{align}
\begin{align}
\vec{P}=\myvec{2\alpha+\beta\\\alpha-\beta\\\alpha+\beta}
\end{align}
Therefore,
\begin{align}
\vec{P} &= \alpha\myvec{0\\3\\-1}
\end{align}

Normalizing,
\begin{align}
\vec{P} &= \pm \frac{1}{\sqrt{10}} \myvec{0\\3\\-1}
\end{align}
\begin{figure}[H]
	\centering
	\includegraphics[scale=0.5]{img}
	\caption*{Plot using C functions}
	\label{img}
\end{figure}
\begin{figure}[H]
	\centering
	\includegraphics[scale=0.5]{fig}
	\caption*{Plot using Python}
	\label{fig}
\end{figure}
\end{document}
