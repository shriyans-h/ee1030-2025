\let\negmedspace\undefined
\let\negthickspace\undefined
\documentclass[journal,12pt,onecolumn]{IEEEtran}
\usepackage{cite}
\usepackage{amsmath,amssymb,amsfonts,amsthm}
\usepackage{algorithmic}
\usepackage{graphicx}
\graphicspath{{./figs/}}
\usepackage{textcomp}
\usepackage{xcolor}
\usepackage{txfonts}
\usepackage{listings}
\usepackage{enumitem}
\usepackage{mathtools}
\usepackage{gensymb}
\usepackage{comment}
\usepackage{caption}
\usepackage[breaklinks=true]{hyperref}
\usepackage{tkz-euclide} 
\usepackage{listings}
\usepackage{gvv}                                        
%\def\inputGnumericTable{}                                 
\usepackage[latin1]{inputenc}     
\usepackage{xparse}
\usepackage{color}                                            
\usepackage{array}                                            
\usepackage{longtable}                                       
\usepackage{calc}                                             
\usepackage{multirow}
\usepackage{multicol}
\usepackage{hhline}                                           
\usepackage{ifthen}                                           
\usepackage{lscape}
\usepackage{tabularx}
\usepackage{array}
\usepackage{float}
%\newtheorem{theorem}{Theorem}[section]
%\newtheorem{theorem}{Theorem}[section]
%\newtheorem{problem}{Problem}
%\newtheorem{proposition}{Proposition}[section]
%\newtheorem{lemma}{Lemma}[section]
%\newtheorem{corollary}[theorem]{Corollary}
%\newtheorem{example}{Example}[section]
%\newtheorem{definition}[problem]{Definition}

\begin{document}

\title{2.10.49}
\author{EE25BTECH11020 - Darsh Pankaj Gajare}
% \maketitle
% \newpage
% \bigskip
%\begin{document}
{\let\newpage\relax\maketitle}
%\renewcommand{\thefigure}{\theenumi}
%\renewcommand{\thetable}{\theenumi}
Question:\\
The unit vector which is orthogonal to the vector $3\hat{i}+2\hat{j} +6\hat{k}$ and is coplanar with vectors $2\hat{i}+\hat{j}+\hat{k}$ and $\hat{i}-\hat{j}+\hat{k}$ is
\begin{multicols}{4}
\begin{enumerate}
\item$\frac{2\hat{i}-6\hat{j}+\hat{k}}{\sqrt{41}}$
\item $\frac{2\hat{i}-3\hat{j}}{\sqrt{13}}$
\item $\frac{3\hat{i}-\hat{k}}{\sqrt{10}}$
\item $\frac{4\hat{i}+3\hat{j}-3\hat{k}}{\sqrt{34}}$
\end{enumerate}
\end{multicols}
\textbf{Solution:}
Given:
\begin{table}[H]
	\centering
	\label{}
	\caption{Given data}
	\begin{tabular}[12pt]{ |c| c|}
    \hline
    \textbf{Name} & \textbf{Point}\\ 
    \hline
	Point A &\myvec{h \\ k}\\
    \hline 
 Point B &\myvec{x1 \\ y1}\\
    \hline
	  Point R &\myvec{x2 \\ y2}\\
    \hline
    
    \end{tabular}

\end{table}
To find: $\vec{P}$
\solution

\begin{align}
\vec{P} = \alpha\vec{A} + \beta\vec{B} 
\end{align}

\begin{align}
\vec{P}^T \vec{C} &= 0 
\end{align}
\begin{align}
    \myvec{\vec{A} & \vec{B}}^T \vec{C} \myvec{\alpha\\\beta} &= 0
\end{align}
\begin{align}
\myvec{\vec{A}^T \vec{C} & \vec{B}^T \vec{C}}\myvec{\alpha\\\beta} = 0
\end{align}
\begin{align}
 \myvec{\vec{A}^T \vec{C} & \vec{B}^T \vec{C}} = \myvec{14 & 7}
\end{align}

\begin{align}
 2\alpha+\beta = 0 \implies \beta = -2\alpha
\end{align}
\begin{align}
\vec{P}=\myvec{2\alpha+\beta\\\alpha-\beta\\\alpha+\beta}
\end{align}
Therefore,
\begin{align}
\vec{P} &= \alpha\myvec{0\\3\\-1}
\end{align}

Normalizing,
\begin{align}
\vec{P} &= \pm \frac{1}{\sqrt{10}} \myvec{0\\3\\-1}
\end{align}
\begin{figure}[H]
	\centering
	\includegraphics[scale=0.5]{img}
	\caption*{Plot using C function}
	\label{img}
\end{figure}
\begin{figure}[H]
	\centering
	\includegraphics[scale=0.5]{fig}
	\caption*{Plot using Python}
	\label{fig}
\end{figure}
\end{document}
