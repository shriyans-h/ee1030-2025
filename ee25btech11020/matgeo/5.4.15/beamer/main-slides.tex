\documentclass{beamer}
\mode<presentation>
\usepackage{amsmath}
\usepackage{amssymb}
%\usepackage{advdate}
\usepackage{graphicx}
\graphicspath{{../figs/}}
\usepackage{adjustbox}
\usepackage{subcaption}
\usepackage{enumitem}
\usepackage{multicol}
\usepackage{mathtools}
\usepackage{listings}
\usepackage{url}
\def\UrlBreaks{\do\/\do-}
\usetheme{Boadilla}
\usecolortheme{lily}
\setbeamertemplate{footline}
{
  \leavevmode%
  \hbox{%
  \begin{beamercolorbox}[wd=\paperwidth,ht=2.25ex,dp=1ex,right]{author in head/foot}%
    \insertframenumber{} / \inserttotalframenumber\hspace*{2ex} 
  \end{beamercolorbox}}%
  \vskip0pt%
}
\setbeamertemplate{navigation symbols}{}
\let\solution\relax
\usepackage{gvv}
\lstset{
language=C,
frame=single, 
breaklines=true,
columns=fullflexible
}

\numberwithin{equation}{section}



\begin{document}

\title{5.2.23}
\author{EE25BTECH11020 - Darsh Pankaj Gajare}
% \maketitle
% \newpage
% \bigskip
%\begin{document}
{\let\newpage\relax\maketitle}
%\renewcommand{\thefigure}{\theenumi}
%\renewcommand{\thetable}{\theenumi}
Question:\\
Using elementary transformations, find inverse of the matrix $\myvec{2&1\\7&4}$
\solution
\begin{table}[H]
	\centering
	\caption{}
	\begin{tabular}[12pt]{ |c| c|}
    \hline
    \textbf{Name} & \textbf{Point}\\ 
    \hline
	Point A &\myvec{h \\ k}\\
    \hline 
 Point B &\myvec{x1 \\ y1}\\
    \hline
	  Point R &\myvec{x2 \\ y2}\\
    \hline
    
    \end{tabular}

	\label{}
\end{table}
\begin{align}
	\vec{A}\vec{A}^{-1}=\vec{I}
\end{align}
Using Augmented matrix,
\begin{align}
	\augvec{2}{2}{2&1&1&0\\7&4&0&1}
\end{align}
$R_2=R_2-3R_1$
\begin{align}
	\augvec{2}{2}{2&1&1&0\\1&1&-3&1}
\end{align}
$R_1=R_1-R_2$
\begin{align}
	\augvec{2}{2}{1&0&4&-1\\1&1&-3&1}
\end{align}
$R_2=R_2-R_1$
\begin{align}
	\augvec{2}{2}{1&0&4&-1\\0&1&-7&2}
\end{align}
\begin{align}
	\vec{A}^{-1}=\myvec{4&-1\\-7&2}
\end{align}
C function to find inverse of a matrix:
\begin{lstlisting}

int inverse2x2(double mat[4], double inv[4]) {
    double a = mat[0], b = mat[1];
    double c = mat[2], d = mat[3];

    double det = a*d - b*c;
    if(det == 0.0) {
        return -1;  // not invertible
    }

    inv[0] =  d / det;
    inv[1] = -b / det;
    inv[2] = -c / det;
    inv[3] =  a / det;

    return 0; // success
}
\end{lstlisting}

\end{document}

