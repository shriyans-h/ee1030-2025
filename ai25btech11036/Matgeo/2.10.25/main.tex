\let\negmedspace\undefined
\let\negthickspace\undefined
\documentclass[journal]{IEEEtran}
\usepackage[a5paper, margin=10mm, onecolumn]{geometry}
%\usepackage{lmodern} % Ensure lmodern is loaded for pdflatex
\usepackage{tfrupee} % Include tfrupee package

\setlength{\headheight}{1cm} % Set the height of the header box
\setlength{\headsep}{0mm}     % Set the distance between the header box and the top of the text

\usepackage{gvv-book}
\usepackage{gvv}
\usepackage{cite}
\usepackage{amsmath,amssymb,amsfonts,amsthm}
\usepackage{algorithmic}
\usepackage{graphicx}
\usepackage{textcomp}
\usepackage{xcolor}
\usepackage{txfonts}
\usepackage{listings}
\usepackage{enumitem}
\usepackage{mathtools}
\usepackage{gensymb}
\usepackage{comment}
\usepackage[breaklinks=true]{hyperref}
\usepackage{tkz-euclide} 
\usepackage{listings}
% \usepackage{gvv}                                        
\def\inputGnumericTable{}                                 
\usepackage[latin1]{inputenc}                                
\usepackage{color}                                            
\usepackage{array}                                            
\usepackage{longtable}                                       
\usepackage{calc}                                             
\usepackage{multirow}                                         
\usepackage{hhline}                                           
\usepackage{ifthen}                                           
\usepackage{lscape}
\usepackage{circuitikz}
\tikzstyle{block} = [rectangle, draw, fill=blue!20, 
    text width=4em, text centered, rounded corners, minimum height=3em]
\tikzstyle{sum} = [draw, fill=blue!10, circle, minimum size=1cm, node distance=1.5cm]
\tikzstyle{input} = [coordinate]
\tikzstyle{output} = [coordinate]


\begin{document}
\bibliographystyle{IEEEtran}
\vspace{3cm}
\title{2.10.25}
\author{AI25BTECH11036--SNEHAMRUDULA}
 \maketitle
% \newpage
% \bigskip
{\let\newpage\relax\maketitle}

\renewcommand{\thefigure}{\theenumi}
\renewcommand{\thetable}{\theenumi}
\setlength{\intextsep}{10pt} % Space between text and floats


\numberwithin{equation}{enumi}
\numberwithin{figure}{enumi}
\renewcommand{\thetable}{\theenumi}


\title{Solution: Collinearity Problem}
\author{AI25BTECH11036-- SNEHAMRUDULA}
\date{}


\maketitle
\begin{enumerate}
\item[2.10.25.] In $\triangle PQR$
\end{enumerate}, let 
  $\vec{a} = \overrightarrow{QR}, \; 
   \vec{b} = \overrightarrow{RP}, \; 
   \vec{c} = \overrightarrow{PQ}$.  

  If $|\vec{a}| = 12,\; |\vec{b}| = 4\sqrt{3},\; \vec{b}\cdot\vec{c} = 24$,  
  then which of the following is (are) true?  

  \begin{enumerate}[label=(\alph*), leftmargin=*]
    \item $\dfrac{|\vec{c}|^2}{2} - |\vec{a}| = 12$
    \item $\dfrac{|\vec{c}|^2}{2} + |\vec{a}| = 30$
    \item $|\vec{a} \times \vec{b} + \vec{c} \times \vec{a}| = 48\sqrt{3}$
    \item $\vec{a}\cdot\vec{b} = -72$
    \end{enumerate}

\textbf{solution}
$
\norm{\vec a} = 12,\quad \norm{\vec b} = 4\sqrt{3},\quad \vec b \cdot \vec c = 24,
\vec a + \vec b + \vec c = \vec 0$
Thus, $\vec c = -(\vec a + \vec b)$.

Find:
\begin{enumerate}
    \item[(a)] $\norm{\vec c}^2$
    \item[(b)] Check $\frac{\norm{\vec c}^2}{2} \pm \norm{\vec a}$
    \item[(c)] $\norm{\vec a \times \vec b + \vec c \times \vec a}$
    \item[(d)] $\vec a \cdot \vec b$
\end{enumerate}
\textbf{(d)} Compute $\vec a \cdot \vec b$:
\begin{align}
\vec b \cdot \vec c = -\vec a \cdot \vec b - \norm{\vec b}^2
\quad \Rightarrow \quad 24 = -\vec a \cdot \vec b - 48,
\end{align}
\begin{align}
\vec a \cdot \vec b = -72.
\end{align}

\textbf{(a), (b)} Compute $\norm{\vec c}^2$:
\begin{align}
\norm{\vec c}^2 = \norm{\vec a + \vec b}^2 = 144 + 48 + 2(-72) = 48,
\end{align}
\begin{align}
\frac{48}{2} - 12 = 12\ (\text{True}),\quad \frac{48}{2} + 12 = 36\ (\text{False}).
\end{align}

\textbf{(c)} Compute $\norm{\vec a \times \vec b + \vec c \times \vec a}$:
\begin{align}
\vec a \times \vec b + \vec c \times \vec a = 2(\vec a \times \vec b),
\end{align}
\begin{align}
\cos\theta = \frac{-72}{12 \cdot 4\sqrt{3}} = -\frac{\sqrt{3}}{2},\quad \sin\theta = \frac{1}{2},
\norm{\cos\theta} = 2 \cdot 12 \cdot 4\sqrt{3} \cdot \frac{1}{2} = 48\sqrt{3}
\end{align}
\begin{enumerate}
    \item $\vec a \cdot \vec b = -72$
    \item $\norm{\vec c}^2 = 48$
    \item $\frac{\norm{\vec c}^2}{2} - \norm{\vec a} = 12 (True)$
    \item $\frac{\norm{\vec c}^2}{2} + \norm{\vec a} = 36 (False)$
    \item $\norm{\vec a \times \vec b + \vec c}$ 
\end{enumerate}
\begin{figure}[ht!]
    \centering
    \includegraphics[width=0.9\textwidth]{figs/fig.2.10.25.png}
    \caption{}
    \label{fig:1.2.27.jpg}
\end{figure}

\end{document}