\documentclass{beamer}
\usepackage[utf8]{inputenc}
\usetheme{Madrid}
\usecolortheme{default}
\usepackage{amsmath,amssymb,amsfonts,amsthm}
\usepackage{txfonts}
\usepackage{tk\documentclass{beamer}
\usepackage[utf8]{inputenc}
\usetheme{Madrid}
\usecolortheme{default}
\usepackage{amsmath,amssymb,amsfonts,amsthm}
\usepackage{txfonts}
\usepackage{tkz-euclide}
\usepackage{listings}
\usepackage{adjustbox}
\usepackage{array}
\usepackage{tabularx}
\usepackage{gvv}
\usepackage{lmodern}
\usepackage{circuitikz}
\usepackage{tikz}
\usepackage{graphicx}

\setbeamertemplate{page number in head/foot}[totalframenumber]

\usepackage{tcolorbox}
\tcbuselibrary{minted,breakable,xparse,skins}



\definecolor{bg}{gray}{0.95}
\DeclareTCBListing{mintedbox}{O{}m!O{}}{%
  breakable=true,
  listing engine=minted,
  listing only,
  minted language=#2,
  minted style=default,
  minted options={%
    linenos,
    gobble=0,
    breaklines=true,
    breakafter=,,
    fontsize=\small,
    numbersep=8pt,
    #1},
  boxsep=0pt,
  left skip=0pt,
  right skip=0pt,
  left=25pt,
  right=0pt,
  top=3pt,
  bottom=3pt,
  arc=5pt,
  leftrule=0pt,
  rightrule=0pt,
  bottomrule=2pt,
  toprule=2pt,
  colback=bg,
  colframe=orange!70,
  enhanced,
  overlay={%
    \begin{tcbclipinterior}
    \fill[orange!20!white] (frame.south west) rectangle ([xshift=20pt]frame.north west);
    \end{tcbclipinterior}},
  #3,
}
\lstset{
    language=C,
    basicstyle=\ttfamily\small,
    keywordstyle=\color{blue},
    stringstyle=\color{orange},
    commentstyle=\color{green!60!black},
    numbers=left,
    numberstyle=\tiny\color{gray},
    breaklines=true,
    showstringspaces=false,
}
%------------------------------------------------------------
%This block of code defines the information to appear in the
%Title page
\title %optional
{2.2.26}

%\subtitle{A short story}

\author % (optional)
{BALU-ai25btech11017}



\begin{document}


\frame{\titlepage}
\begin{frame}{Question}
Find the area of the triangle formed by the points $P(-1.5,3)$, $Q(6,-2)$ and $R(-3,4)$.
\\ 
\end{frame}



\begin{frame}{Theoretical Solution}

Given three points\\
\begin{align}
  \vec{P}=\begin{myvec}{-1.5\\3}\end{myvec}\;
  \vec{Q}=\begin{myvec}{6\\-2}\end{myvec}\;
  \vec{R}=\begin{myvec}{-3\\4}\end{myvec}\
   \end{align}
   \begin{align}
 \vec{Q}-\vec{P}=\begin{myvec}{7.5\\-5}\end{myvec}\
\end{align}
\begin{align}
  \vec{R}-\vec{P}=\begin{myvec}{-1.5\\1}\end{myvec}\
\end{align}
\begin{align}
ar(PQR) &= \frac{1}{2} \, \|(\vec{Q} - \vec{P}) \times (\vec{R} - \vec{P}) \|
\end{align}
\begin{align}
ar(PQR) &= \frac{1}{2} \, \|(\vec{Q} - \vec{P}) \times (\vec{R} - \vec{P}) \|=0
\end{align}
points are collinear
\end{frame}
\begin{frame}[fragile]
    \frametitle{C Code}

    \begin{lstlisting}
#include <stdio.h>
#include <math.h>

// Function to calculate area of triangle using cross product
double triangle_area(double P[2], double Q[2], double R[2]) {
    double x1 = Q[0] - P[0];
    double y1 = Q[1] - P[1];
    double x2 = R[0] - P[0];
    double y2 = R[1] - P[1];
    
    // Cross product magnitude in 2D
    double cross = fabs(x1 * y2 - y1 * x2);
    
    return 0.5 * cross;
}



   

     \end{lstlisting}
\end{frame}
\begin{frame}[fragile]
    \frametitle{C Code - Resultant velocity}

    \begin{lstlisting}
  int main() {
    double P[2] = {-1.5, 3};
    double Q[2] = {6, -2};
    double R[2] = {-3, 4};

    double area = triangle_area(P, Q, R);
    
    printf("Area of triangle PQR = %.2f\n", area);
    
    return 0;
}

    \end{lstlisting}
\end{frame}
\begin{frame}[fragile]
    \frametitle{Python Code}
    \begin{lstlisting}
import numpy as np
import matplotlib.pyplot as plt
from mpl_toolkits.mplot3d import Axes3D

# Points
P = np.array([-1.5, 3, 0])
Q = np.array([6, -2, 0])
R = np.array([-3, 4, 0])

# Function to compute area of triangle
def triangle_area(A, B, C):
    return 0.5 * np.linalg.norm(np.cross(B - A, C - A))

# Calculate area
area = triangle_area(P, Q, R)

# Plot the triangle in 3D













    \end{lstlisting}
\end{frame}

\begin{frame}[fragile]
    \frametitle{Python Code}
    \begin{lstlisting}
fig = plt.figure(figsize=(6,6))
ax = fig.add_subplot(111, projection='3d')

# Plot points
ax.scatter(*P, color='r', s=50)
ax.scatter(*Q, color='g', s=50)
ax.scatter(*R, color='b', s=50)

# Plot triangle edges
ax.plot([P[0], Q[0]], [P[1], Q[1]], [P[2], Q[2]], 'k-')
ax.plot([Q[0], R[0]], [Q[1], R[1]], [Q[2], R[2]], 'k-')
ax.plot([R[0], P[0]], [R[1], P[1]], [R[2], P[2]], 'k-')


    \end{lstlisting}
\end{frame}

\begin{frame}[fragile]
    \frametitle{Python Code}
    \begin{lstlisting}
# Labels for points
ax.text(*P, "P(-1.5,3)", color='r')
ax.text(*Q, "Q(6,-2)", color='g')
ax.text(*R, "R(-3,4)", color='b')

# Axis labels
ax.set_xlabel('X-axis')
ax.set_ylabel('Y-axis')
ax.set_zlabel('Z-axis')
ax.set_title(f"Area of Triangle PQR = {area:.2f}")

# Save and show
plt.savefig("triangle_area.png")
plt.show()


    \end{lstlisting}
\end{frame}

\begin{frame}{Plot}
    \centering
    \includegraphics[width=\columnwidth, height=0.8\textheight, keepaspectratio]{Beamer/figs/fig4.png}     
\end{frame}




\end{document}z-euclide}
\usepackage{listings}
\usepackage{adjustbox}
\usepackage{array}
\usepackage{tabularx}
\usepackage{gvv}
\usepackage{lmodern}
\usepackage{circuitikz}
\usepackage{tikz}
\usepackage{graphicx}

\setbeamertemplate{page number in head/foot}[totalframenumber]

\usepackage{tcolorbox}
\tcbuselibrary{minted,breakable,xparse,skins}



\definecolor{bg}{gray}{0.95}
\DeclareTCBListing{mintedbox}{O{}m!O{}}{%
  breakable=true,
  listing engine=minted,
  listing only,
  minted language=#2,
  minted style=default,
  minted options={%
    linenos,
    gobble=0,
    breaklines=true,
    breakafter=,,
    fontsize=\small,
    numbersep=8pt,
    #1},
  boxsep=0pt,
  left skip=0pt,
  right skip=0pt,
  left=25pt,
  right=0pt,
  top=3pt,
  bottom=3pt,
  arc=5pt,
  leftrule=0pt,
  rightrule=0pt,
  bottomrule=2pt,
  toprule=2pt,
  colback=bg,
  colframe=orange!70,
  enhanced,
  overlay={%
    \begin{tcbclipinterior}
    \fill[orange!20!white] (frame.south west) rectangle ([xshift=20pt]frame.north west);
    \end{tcbclipinterior}},
  #3,
}
\lstset{
    language=C,
    basicstyle=\ttfamily\small,
    keywordstyle=\color{blue},
    stringstyle=\color{orange},
    commentstyle=\color{green!60!black},
    numbers=left,
    numberstyle=\tiny\color{gray},
    breaklines=true,
    showstringspaces=false,
}

%\numberwithin{equation}{section}

\title{Matgeo-q.4.4.32}
\author{AI25BTECH11036-SNEHAMRUDULA}

\date{\today} 
\begin{document}

\begin{frame}
\titlepage
\end{frame}

\section*{Outline}
\begin{frame}{question}
   \begin{question}
Consider the lines given by
\begin{align*}
L_1 &: x + 3y - 5 = 0, \\
L_2 &: 3x - ky - 1 = 0, \\
L_3 &: 5x + 2y - 12 = 0.
\end{align*}
Match the Statements/Expressions in Column I with the Statements/Expressions in Column II.

\begin{center}
\begin{tabular}{p{0.45\linewidth} p{0.45\linewidth}}
\textbf{Column I} & \textbf{Column II} \\
\begin{enumerate}[label=(\Alph*)]
    \item $L_1, L_2, L_3$ are concurrent, if
    \item One of $L_1, L_2, L_3$ is parallel to at least one of the other two, if
    \item $L_1, L_2, L_3$ form a triangle, if
    \item $L_1, L_2, L_3$ do not form a triangle, if
\end{enumerate}
&
\begin{enumerate}[label=(\alph*)]
    \item $k = 9$
    \item $k = \dfrac{-6}{5}$
    \item $k = \dfrac{5}{6}$
    \item $k = 5$
\end{enumerate}
\end{tabular}
\end{center}
\end{question}
\end{frame}
\begin{frame}{solution}
\[
\left(\vec{n}^{\,T}\right)\vec{x} = 1
\]
The general form of a line is $\vec{n}^\top \vec{x} = p$, where $\vec{n}$ is the normal vector.
For the given lines:
\begin{align*}
L_1 &: \vec{n}_1 = \begin{pmatrix} 1 \\ 3 \end{pmatrix}, \quad p_1 = 5, \\
L_2 &: \vec{n}_2 = \begin{pmatrix} 3 \\ -k \end{pmatrix}, \quad p_2 = 1, \\
L_3 &: \vec{n}_3 = \begin{pmatrix} 5 \\ 2 \end{pmatrix}, \quad p_3 = 12.
\end{align*}
    \item \textbf{Concurrent:} The lines are concurrent if the matrix
    $\begin{pmatrix} 1 & 3 & -5 \\ 3 & -k & -1 \\ 5 & 2 & -12 \end{pmatrix}$
    has determinant zero.
\end{frame}
\begin{frame}{solution}
 Computing the determinant:
    \begin{align*}
    &1((-k)(-12) - (-1)(2)) - 3(3(-12) - (-1)(5)) + (-5)(3(2) - (-k)(5)) \\
    &= 1(12k + 2) - 3(-36 + 5) - 5(6 + 5k) \\
    &= 12k + 2 + 93 - 30 - 25k \\
    &= -13k + 65.
    \end{align*}
    Setting it equal to zero: $-13k + 65 = 0 \Rightarrow k = 5$.

    So concurrent if $k = 5$. Hence (A) matches with (s).

    \item \textbf{Parallel:} Two lines are parallel if their normal vectors are scalar multiples.

    Checking pairs:

    \begin{itemize}
        \item $\vec{n}_1$ and $\vec{n}_2$: $\frac{3}{1} = \frac{-k}{3} \Rightarrow k = -9$.

        \item $\vec{n}_2$ and $\vec{n}_3$: $\frac{3}{5} = \frac{-k}{2} \Rightarrow k = \frac{-6}{5}$.

        \item $\vec{n}_1$ and $\vec{n}_3$: $\frac{1}{5} = \frac{3}{2}$, which is false ⇒ not parallel.
    \end{itemize}

    So parallel if $k = \frac{-6}{5}$. Hence (B) matches with (q).

\end{frame}
\begin{frame}{solution}

\item \textbf{Form a triangle:} The lines form a triangle if they are neither concurrent nor parallel.

    So $k \neq 5$ and $k \neq \frac{-6}{5}$.

    A typical example is $k = 9$. Hence (C) matches with (p).

    \item \textbf{Do not form a triangle:} The lines do not form a triangle if they are either concurrent or two of them are parallel.

    Hence possible when $k = 5$ or $k = \frac{-6}{5}$.

    Here the option is $k = 5$. Hence (D) matches with (s).
    \end{frame}
    \begin{frame}{Graphical Representation}
   \begin{figure}[h!]
\centering
\includegraphics[width=0.6\linewidth]{fig4.13.1.png}
\end{figure}
\end{frame}
\end{document}

