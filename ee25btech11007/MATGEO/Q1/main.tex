%iffalse
\let\negmedspace\undefined
\let\negthickspace\undefined
\documentclass[article,12pt,onecolumn]{IEEEtran}
\usepackage{cite}
\usepackage{amsmath,amssymb,amsfonts,amsthm}
\usepackage{algorithmic}
\usepackage{graphicx}
\usepackage{textcomp}
\usepackage{xcolor}
\graphicspath{{figs/}}
\usepackage{txfonts}
\usepackage{listings}
\usepackage{enumitem}
\usepackage{mathtools}
\usepackage{gensymb}
\usepackage{comment}
\usepackage[breaklinks=true]{hyperref}
\usepackage{tkz-euclide} 
\usepackage{listings}
\usepackage{gvv}
\def\inputGnumericTable{}                                 
\usepackage[utf8]{inputenc}                              
\usepackage{color}                                         
\usepackage{array}                                        
\usepackage{longtable}                                     
\usepackage{calc}                                          
\usepackage{multirow}                                      
\usepackage{hhline}                                        
\usepackage{ifthen}                                        
\usepackage{lscape}
\newtheorem{theorem}{Theorem}[section]
\newtheorem{problem}{Problem}
\newtheorem{proposition}{Proposition}[section]
\newtheorem{lemma}{Lemma}[section]
\newtheorem{corollary}[theorem]{Corollary}
\newtheorem{example}{Example}[section]
\newtheorem{definition}[problem]{Definition}
\newcommand{\BEQA}{\begin{eqnarray}}
\newcommand{\EEQA}{\end{eqnarray}}
\newcommand{\define}{\stackrel{\triangle}{=}}
\theoremstyle{remark}
\newtheorem{rem}{Remark}
\graphicspath{ {./Figures/} }
\usepackage{float} % For the [H] float option
\usepackage{textcomp}
\usepackage{multicol}

% --- myvec macro (wraps a bmatrix so your document keeps the same visual output
%     while using the requested \myvec command) ---

\begin{document}
\textbf{Question}\\[2pt]
Consider two points $P$ and $Q$ with position vectors
\[
\vec{OP}=3\vec{a}-2\vec{b},
\qquad
\vec{OQ}=\vec{a}+\vec{b}.
\]
Find the position vector of a point $R$ which divides the line joining $P$ and $Q$ in the ratio $2:1$,
\begin{enumerate}
    \item[(a)] internally, and
    \item[(b)] externally.
\end{enumerate}

\vspace{6pt}
\textbf{Solution}\\[2pt]
In the basis $\{\vec a,\vec b\}$, we can write
\[
\vec A=\myvec{3\\-2},
\qquad
\vec B=\myvec{1\\1}. \tag{1}
\]

\textbf{(a) Internal Division.} If $R$ divides $AB$ in the ratio $k:1$ internally, then
\[
\vec R=\frac{k\vec B+\vec A}{k+1}. \tag{2}
\]
With $k=2$ 
\begin{align}
\vec R
&=\frac{2\myvec{1\\1}+\myvec{3\\-2}}{2+1} \tag{3}\\
&=\frac{\myvec{2\\2}+\myvec{3\\-2}}{3} \tag{4}\\
&=\frac{\myvec{5\\0}}{3} \tag{5}\\
&=\myvec{\tfrac{5}{3}\\0}. \tag{6}
\end{align}
\[
\boxed{\;\vec R_{\text{internal}}=\myvec{\tfrac{5}{3}\\[2pt]0}\;}
\]

\bigskip
\textbf{(b) External Division.} If $R$ divides $AB$ in the ratio $k:1$ externally, then
\[
\vec R=\frac{k\vec B-\vec A}{k-1}. \tag{7}
\]
With $k=2$ 
\begin{align}
\vec R
&=\frac{2\myvec{1\\1}-\myvec{3\\-2}}{2-1} \tag{8}\\
&=\myvec{2\\2}-\myvec{3\\-2} \tag{9}\\
&=\myvec{-1\\4} \tag{10}
\end{align}
\[
\boxed{\;\vec R_{\text{external}}=\myvec{-1\\[2pt]4}\;}
\]
From the calculations above, we obtain:

\[
\boxed{\;\;\Large \vec{R}_{\text{internal}} = \tfrac{5}{3}\vec{a}\;\;}
\]

\[
\boxed{\;\;\Large \vec{R}_{\text{external}} = -\vec{a} + 4\vec{b}\;\;}
\]

\begin{figure}[H]
    \centering
    \includegraphics[width=\columnwidth]{figs/mg1plot.png}
\end{figure}

\end{document}
