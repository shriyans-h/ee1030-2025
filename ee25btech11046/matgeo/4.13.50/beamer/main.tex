\documentclass{beamer}
\usepackage[utf8]{inputenc}

\usetheme{Madrid}
\usecolortheme{default}
\usepackage{amsmath,amssymb,amsfonts,amsthm}
\usepackage{txfonts}
\usepackage{tkz-euclide}
\usepackage{listings}
\usepackage{adjustbox}
\usepackage{array}
\usepackage{tabularx}
\usepackage{gvv}
\usepackage{lmodern}
\usepackage{circuitikz}
\usepackage{tikz}
\usepackage{graphicx}
\usepackage{mathtools}

\setbeamertemplate{page number in head/foot}[totalframenumber]

\usepackage{tcolorbox}
\tcbuselibrary{minted,breakable,xparse,skins}



\definecolor{bg}{gray}{0.95}
\DeclareTCBListing{mintedbox}{O{}m!O{}}{%
  breakable=true,
  listing engine=minted,
  listing only,
  minted language=#2,
  minted style=default,
  minted options={%
    linenos,
    gobble=0,
    breaklines=true,
    breakafter=,,
    fontsize=\small,
    numbersep=8pt,
    #1},
  boxsep=0pt,
  left skip=0pt,
  right skip=0pt,
  left=25pt,
  right=0pt,
  top=3pt,
  bottom=3pt,
  arc=5pt,
  leftrule=0pt,
  rightrule=0pt,
  bottomrule=2pt,
  toprule=2pt,
  colback=bg,
  colframe=orange!70,
  enhanced,
  overlay={%
    \begin{tcbclipinterior}
    \fill[orange!20!white] (frame.south west) rectangle ([xshift=20pt]frame.north west);
    \end{tcbclipinterior}},
  #3,
}
\lstset{
    language=C,
    basicstyle=\ttfamily\small,
    keywordstyle=\color{blue},
    stringstyle=\color{orange},
    commentstyle=\color{green!60!black},
    numbers=left,
    numberstyle=\tiny\color{gray},
    breaklines=true,
    showstringspaces=false,
}
\title{4.13.50}
\date{27th September, 2025}
\author{Puni Aditya - EE25BTECH11046}

\begin{document}

\frame{\titlepage}
\begin{frame}{Question}
Two equal sides of an isosceles triangle are given by the equations $7x - y + 3 = 0$ and $x + y - 3 = 0$ and its third side passes through the point $(1, -10)$. Determine the equation of the third side.
\end{frame}

\begin{frame}{Theoretical Solution}
Let the two equal sides of the isosceles triangle be represented by
\begin{align*}
    \vec{n_1}^\top\vec{x} &= c_1 \\
    \vec{n_2}^\top\vec{x} &= c_2
\end{align*}
and the third side by the line
\begin{align*}
    \vec{n}^\top\vec{x} &= c
\end{align*}
The third side of the isosceles, the base, is perpendicular to the angle bisector of the two equal sides.
\end{frame}

\begin{frame}{Theoretical Solution}
\begin{align}
    \frac{\abs{\vec{n}^\top\vec{n_1}}}{\norm{\vec{n}}\norm{\vec{n_1}}} &= \frac{\abs{\vec{n}^\top\vec{n_2}}}{\norm{\vec{n}}\norm{\vec{n_2}}} \\
    \frac{\abs{\vec{n}^\top\vec{u_1}}}{\norm{\vec{n}}} &= \frac{\abs{\vec{n}^\top\vec{u_2}}}{\norm{\vec{n}}} \\
    \abs{\vec{n}^\top\vec{u_1}} &= \abs{\vec{n}^\top\vec{u_2}} \\
    \vec{n}^\top\vec{u_1} &= \pm \vec{n}^\top\vec{u_2} \\
    \vec{n}^\top(\vec{u_1} \mp \vec{u_2}) &= 0
\end{align}
Here, $\vec{u_1}$ and $\vec{u_2}$ represent the unit vectors of $\vec{n_1}$ and $\vec{n_2}$ respectively.
\end{frame}

\begin{frame}{Theoretical Solution}
A vector perpendicular to given vector $\myvec{1 \\ m}$ is
\begin{align}
    \vec{n} = \myvec{-m \\ 1} \label{eq:40}
\end{align}
For the given question,
\begin{align}
    \vec{n_1} &= \myvec{7 \\ -1}\text{ and }\vec{n_2} = \myvec{1 \\ 1} \\
    \norm{\vec{n_1}} &= \sqrt{50} = 5\sqrt{2} \\
    \norm{\vec{n_2}} &= \sqrt{2} \\
    \vec{u_1} = \frac{1}{5\sqrt{2}}\myvec{7 \\ -1},&\text{ }\vec{u_2} = \frac{1}{\sqrt{2}}\myvec{1 \\ 1} = \frac{5}{5\sqrt{2}}\myvec{1 \\ 1}
\end{align}
\end{frame}

\begin{frame}{Theoretical Solution}
\begin{align}
    \vec{u_1} - \vec{u_2} &= \frac{1}{5\sqrt{2}}\left(\myvec{7 \\ -1} - \myvec{5 \\ 5}\right) = \frac{1}{5\sqrt{2}}\myvec{2 \\ -6} \\
    \vec{n_a} &= \myvec{1 \\ -3} \\
    \vec{u_1} + \vec{u_2} &= \frac{1}{5\sqrt{2}}\left(\myvec{7 \\ -1} + \myvec{5 \\ 5}\right) = \frac{1}{5\sqrt{2}}\myvec{12 \\ 4} \\
    \vec{n_b} &= \myvec{1 \\ \frac{1}{3}}
\end{align}
For the bisector parallel to $\vec{n_a}$, using \eqref{eq:40},
\begin{align}
    \vec{n_p} = \myvec{3 \\ 1}
\end{align}
\end{frame}

\begin{frame}{Theoretical Solution}
For the bisector parallel to $\vec{n_b}$, using \eqref{eq:40},
\begin{align}    
    \vec{n_q} = \myvec{-\frac{1}{3} \\ 1}
\end{align}
For a line passing through a given point $\vec{p}$,
\begin{align}
    \vec{p} &= \myvec{1 \\ -10} \\
    \vec{n}^\top\vec{x} &= \vec{n}^\top\vec{p}
\end{align}
\end{frame}

\begin{frame}{Theoretical Solution}
For $\vec{n_p}$,
\begin{align}
    \vec{n} &= \myvec{3 \\ 1} \\
    \myvec{3 & 1}\vec{x} &= \myvec{3 & 1}\myvec{1 \\ -10} \\
    \myvec{3 & 1}\vec{x} &= -7
\end{align}
For $\vec{n_q}$,
\begin{align}
    \vec{n} &= \myvec{1 \\ -3} \\
    \myvec{1 & -3}\vec{x} &= \myvec{1 & -3}\myvec{1 \\ -10} \\
    \myvec{1 & -3}\vec{x} &= 31
\end{align}
\end{frame}

\begin{frame}{Plot}
    \begin{figure}
        \centering
        \includegraphics[width=0.5\columnwidth]{../figs/plot_c.jpg}
        \caption{Isosceles Triangle}
        \label{fig:fig}
    \end{figure}
\end{frame}

\end{document}
