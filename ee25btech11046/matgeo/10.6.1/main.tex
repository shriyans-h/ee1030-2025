\documentclass[journal]{IEEEtran}
\usepackage[a5paper, margin:10mm, onecolumn]{geometry}
\usepackage{amsmath,amssymb,amsfonts,amsthm}
\usepackage{mathtools}
\usepackage{gvv-book}
\usepackage{gvv}
\usepackage{hyperref}

\begin{document}

\title{10.6.1}
\author{Puni Aditya - EE25BTECH11046}
\maketitle

\textbf{Question:}

Draw a circle of radius 2.5cm. Take a point P outside the circle at a distance of 7cm from the center. Then construct a pair of tangents to the circle from point P.

\textbf{Solution:}

We first derive the formula for the chord of contact from the general tangent equation.
\begin{align*}
    g\brak{\vec{x}} = \vec{x}^\top\vec{V}\vec{x} + 2\vec{u}^\top\vec{x} + f = 0
\end{align*}
The equation of the tangent at a point of contact $\vec{q}$ is:
\begin{align}
    \brak{\vec{V}\vec{q} + \vec{u}}^\top\vec{x} + \vec{u}^\top\vec{q} + f = 0 \label{eq:tangent}
\end{align}
Since the tangent passes through the external point $\vec{h}$:
\begin{align}
    \brak{\vec{V}\vec{q} + \vec{u}}^\top\vec{h} + \vec{u}^\top\vec{q} + f = 0 \label{eq:point_on_tangent}
\end{align}
\begin{align}
    \vec{h}^\top\vec{V}\vec{q} + \vec{u}^\top\vec{q} + \vec{u}^\top\vec{h} + f = 0 \label{eq:rearranged}
\end{align}
The previous equation shows that any point of contact $\vec{q}$ lies on the following line, the chord of contact:
\begin{align}
    \brak{\vec{V}\vec{h} + \vec{u}}^\top\vec{x} + \vec{u}^\top\vec{h} + f = 0 \label{eq:chord_of_contact}
\end{align}
For the given circle, the circle is centered at the origin, so its conic parameters are:
\begin{align}
    \vec{V} = \vec{I}, \text{ } \vec{u} = \vec{0}, \text{ } f = -r^2, \text{ } \vec{h} = d\vec{e_1}
\end{align}
Substituting these into \eqref{eq:chord_of_contact}:
\begin{align}
    \brak{\vec{I}\brak{d\vec{e_1}} + \vec{0}}^\top\vec{x} + \vec{0}^\top\brak{d\vec{e_1}} - r^2 &= 0 \\
    \brak{d\vec{e_1}}^\top\vec{x} - r^2 &= 0
\end{align}
This line, $L$, contains the points of contact. Its parametric form is $\vec{x} = \vec{h_L} + \kappa\vec{m_L}$.
\begin{align}
    d x - r^2 = 0 \implies \vec{h_L} = \frac{r^2}{d}\vec{e_1}, \text{ } \vec{m_L} = \vec{e_2}
\end{align}
The points of contact are the intersection of line $L$ with the circle $g\brak{\vec{x}} = \vec{x}^\top\vec{x}-r^2=0$.
We use the intersection formula for the parameter $\kappa$:
\begin{align}
    \kappa_{1,2} = \frac{-\vec{m_L}^\top\brak{\vec{V}\vec{h_L}+\vec{u}} \pm \sqrt{\brak{\vec{m_L}^\top\brak{\vec{V}\vec{h_L}+\vec{u}}}^2 - \brak{\vec{m_L}^\top\vec{V}\vec{m_L}}g\brak{\vec{h_L}}}}{\vec{m_L}^\top\vec{V}\vec{m_L}} \label{eq:57}
\end{align}
Calculating the terms with $\vec{V}=\vec{I}, \vec{u}=\vec{0}$:
\begin{align}
    \vec{m_L}^\top\vec{V}\vec{m_L} &= \vec{e_2}^\top\vec{I}\vec{e_2} = 1 \\
    \vec{m_L}^\top\brak{\vec{V}\vec{h_L}+\vec{u}} &= \vec{e_2}^\top\brak{\vec{I}\frac{r^2}{d}\vec{e_1}+\vec{0}} = 0 \\
    g\brak{\vec{h_L}} &= \brak{\frac{r^2}{d}\vec{e_1}}^\top\brak{\frac{r^2}{d}\vec{e_1}} - r^2 = \frac{r^4}{d^2} - r^2
\end{align}
Substituting these into \eqref{eq:57},
\begin{align}
    \kappa = \frac{0 \pm \sqrt{0 - 1\brak{\frac{r^4}{d^2}-r^2}}}{1} = \pm\sqrt{r^2 - \frac{r^4}{d^2}} = \pm \frac{r}{d}\sqrt{d^2-r^2}
\end{align}
The points of contact are $\vec{q} = \vec{h_L} + \kappa\vec{m_L}$.
\begin{align}
    \vec{q} = \frac{r^2}{d}\vec{e_1} \pm \frac{r}{d}\sqrt{d^2-r^2}\vec{e_2}
\end{align}
Substituting the given values $r=2.5$ and $d=7$:
\begin{align}
    \vec{q} &= \frac{\brak{2.5}^2}{7}\vec{e_1} \pm \frac{2.5}{7}\sqrt{7^2 - \brak{2.5}^2}\vec{e_2} \\
    &= \frac{6.25}{7}\vec{e_1} \pm \frac{2.5}{7}\sqrt{42.75}\vec{e_2}
\end{align}
The coordinates of the two points of contact are:
\begin{align}
    \vec{q_{1,2}} = \myvec{\frac{25}{28} \\ \pm \frac{2.5\sqrt{42.75}}{7}}
\end{align}

\begin{figure}[h!]
	\centering
	\includegraphics[width=\columnwidth]{figs/plot_c.jpg}
	\caption*{Plot}
	\label{fig:fig}
\end{figure}

\end{document}
