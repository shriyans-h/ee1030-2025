\documentclass{beamer}
\usepackage[utf8]{inputenc}

\usetheme{Madrid}
\usecolortheme{default}
\usepackage{amsmath,amssymb,amsfonts,amsthm}
\usepackage{txfonts}
\usepackage{tkz-euclide}
\usepackage{listings}
\usepackage{adjustbox}
\usepackage{array}
\usepackage{tabularx}
\usepackage{gvv}
\usepackage{lmodern}
\usepackage{circuitikz}
\usepackage{tikz}
\usepackage{graphicx}

\setbeamertemplate{page number in head/foot}[totalframenumber]

\usepackage{tcolorbox}
\tcbuselibrary{minted,breakable,xparse,skins}



\definecolor{bg}{gray}{0.95}
\DeclareTCBListing{mintedbox}{O{}m!O{}}{%
  breakable=true,
  listing engine=minted,
  listing only,
  minted language=#2,
  minted style=default,
  minted options={%
    linenos,
    gobble=0,
    breaklines=true,
    breakafter=,,
    fontsize=\small,
    numbersep=8pt,
    #1},
  boxsep=0pt,
  left skip=0pt,
  right skip=0pt,
  left=25pt,
  right=0pt,
  top=3pt,
  bottom=3pt,
  arc=5pt,
  leftrule=0pt,
  rightrule=0pt,
  bottomrule=2pt,
  toprule=2pt,
  colback=bg,
  colframe=orange!70,
  enhanced,
  overlay={%
    \begin{tcbclipinterior}
    \fill[orange!20!white] (frame.south west) rectangle ([xshift=20pt]frame.north west);
    \end{tcbclipinterior}},
  #3,
}
\lstset{
    language=C,
    basicstyle=\ttfamily\small,
    keywordstyle=\color{blue},
    stringstyle=\color{orange},
    commentstyle=\color{green!60!black},
    numbers=left,
    numberstyle=\tiny\color{gray},
    breaklines=true,
    showstringspaces=false,
}
\title{2.8.36}
\date{9th September, 2025}
\author{Puni Aditya - EE25BTECH11046}

\begin{document}

\frame{\titlepage}
\begin{frame}{Question}
The value of the expression $\norm{\vec{a} \times \vec{b}} + \brak{\vec{a}^\top\vec{b}}$ is \rule{2cm}{0.4pt}
\end{frame}

\begin{frame}{Theoretical Solution}
Let $\vec{a}$ and $\vec{b}$ be two vectors, and let $\theta$ be the angle between them.

The magnitude of the cross product is
\begin{align}
    \norm{\vec{a} \times \vec{b}} = \norm{\vec{a}} \norm{\vec{b}} \sin\brak{\theta}
\end{align}
The dot product or inner product is
\begin{align}
    \vec{a}^\top\vec{b} = \norm{\vec{a}} \norm{\vec{b}} \cos\brak{\theta}
\end{align}
\end{frame}

\begin{frame}{Theoretical Solution}
Now, we substitute these definitions into the given expression:
\begin{align}
    \norm{\vec{a} \times \vec{b}} + \brak{\vec{a}^\top\vec{b}} = \norm{\vec{a}} \norm{\vec{b}} \sin\brak{\theta} + \norm{\vec{a}} \norm{\vec{b}} \cos\brak{\theta}
\end{align}
\begin{align}
\implies \norm{\vec{a} \times \vec{b}} + \brak{\vec{a}^\top\vec{b}} = \norm{\vec{a}} \norm{\vec{b}} \brak{\sin\brak{\theta} + \cos\brak{\theta}} \\
\sin\brak{\theta} + \cos\brak{\theta} = \sqrt{2}\sin\brak{\theta + \frac{\pi}{4}}
\end{align}
\begin{align*}
\therefore \norm{\vec{a} \times \vec{b}} + \brak{\vec{a}^\top\vec{b}} = \sqrt{2} \norm{\vec{a}} \norm{\vec{b}} \sin\brak{\theta + \frac{\pi}{4}}
\end{align*}
\end{frame}

\begin{frame}{Example}
Example: Let
\begin{align*}
    \vec{a}=\myvec{0 \\ 1 \\ 0}\text{ and }\vec{b}=\myvec{-1 \\ 1 \\ 0} \\
    \norm{\vec{a}} = 1\text{ and }\norm{\vec{b}} = \sqrt{2} \\
\end{align*}
\begin{align}
    \norm{\vec{a} \times \vec{b}} + \brak{\vec{a}^\top\vec{b}} &= \norm{\myvec{0 \\ 0 \\ 1}} + \myvec{0 & 1 &  0} \myvec{-1 \\ 1 \\ 0} \\
    \norm{\vec{a} \times \vec{b}} + \brak{\vec{a}^\top\vec{b}} &= 1 + 1 = 2 \label{eq:2}
\end{align}
\end{frame}

\begin{frame}{Example}
\begin{align}
    \cos\brak{\theta} &= \frac{\myvec{0 & 1 &  0} \myvec{-1 \\ 1 \\ 0}}{\norm{\myvec{0 \\ 1 \\  0}}\norm{\myvec{-1 \\ 1 \\ 0}}} \\
    \cos\brak{\theta} &= \frac{1}{\sqrt{2}} \\
    \theta &= \frac{\pi}{4} \label{eq:3} \\
    \text{From \eqref{eq:3}} \\
    \sqrt{2} \norm{\vec{a}} \norm{\vec{b}} \sin\brak{\theta + \frac{\pi}{4}} &= \sqrt{2} \times 1 \times \sqrt{2} \times \sin\brak{\frac{\pi}{2}}
\end{align}
\end{frame}

\begin{frame}{Example}
\begin{align}
    \sqrt{2} \norm{\vec{a}} \norm{\vec{b}} \sin\brak{\theta + \frac{\pi}{4}} &= 2 \label{eq:4}
\end{align}
From \eqref{eq:2} and \eqref{eq:4},
\begin{align*}
\norm{\vec{a} \times \vec{b}} + \brak{\vec{a}^\top\vec{b}} = \sqrt{2} \norm{\vec{a}} \norm{\vec{b}} \sin\brak{\theta + \frac{\pi}{4}}
\end{align*}
\end{frame}

\begin{frame}{allowframebreaks}
\frametitle{Theoretical Solution}
\begin{figure}
    \centering
    \includegraphics[width=0.5\columnwidth]{../figs/plot_c.jpg}
    \caption{Example}
    \label{fig:fig}
\end{figure}
\end{frame}

\end{document}
