\documentclass[journal]{IEEEtran}
\usepackage[a5paper, margin=10mm, onecolumn]{geometry}
\usepackage{amsmath,amssymb,amsfonts,amsthm}
\usepackage{mathtools}
\usepackage{gvv-book}
\usepackage{gvv}
\usepackage{hyperref}

\begin{document}

\title{10.7.94}
\author{Puni Aditya - EE25BTECH11046}
\maketitle

\textbf{Question:}

A circle touches the X axis and also touches the circle with centre at \brak{0, 3} and radius 2. The locus of the centre of the circle is
\begin{enumerate}
    \item an ellipse
    \item a circle
    \item a hyperbola
    \item a parabola
\end{enumerate}

\textbf{Solution:}

Let the center of the moving circle be 
\begin{align*} 
    \vec{c} = \myvec{x \\ y}
\end{align*}
and its radius be $r$.
The circle touches the X-axis, so its radius is the y-coordinate of its center.
\begin{align}
    r = y = \vec{e_2}^\top\vec{c} \text{ } \brak{y>0}
\end{align}
The fixed circle has center 
\begin{align} 
    \vec{c}_f = 3\vec{e_2}
\end{align} 
and radius 
\begin{align}
    r_f=2
\end{align}
The distance between the centers of two touching circles is the sum of their radii (for external tangency).
\begin{align}
    \norm{\vec{c}-\vec{c}_f} &= r + r_f \\
    \norm{\vec{c}-3\vec{e_2}} &= \vec{e_2}^\top\vec{c} + 2
\end{align}
Squaring both sides,
\begin{align}
    \brak{\vec{c}-3\vec{e_2}}^\top\brak{\vec{c}-3\vec{e_2}} &= \brak{\vec{e_2}^\top\vec{c} + 2}^2 \\
    \vec{c}^\top\vec{c} - 6\vec{e_2}^\top\vec{c} + 9 &= \brak{\vec{e_2}^\top\vec{c}}^2 + 4\vec{e_2}^\top\vec{c} + 4
\end{align}
\begin{align}
    x^2 + y^2 - 6y + 9 &= y^2 + 4y + 4 \\
    x^2 - 10y + 5 &= 0
\end{align}
The locus in the standard form of the conic is
\begin{align}
    \myvec{x & y}\myvec{1 & 0 \\ 0 & 0}\myvec{x \\ y} + 2\myvec{0 & -5}\myvec{x \\ y} + 5 = 0
\end{align}
The matrix of the quadratic part is
\begin{align}
    \vec{V} = \myvec{1 & 0 \\ 0 & 0}
\end{align}
The type of conic section is determined by the eigenvalues of $\vec{V}$. For a diagonal matrix, the eigenvalues are the diagonal entries.
\begin{align}
    \lambda_1 = 1, \text{ } \lambda_2 = 0
\end{align}
\begin{align}
    \mydet{\vec{V}} = \lambda_1\lambda_2 = 1 \cdot 0 = 0
\end{align}
Since one of the eigenvalues is zero, the locus is a parabola. \\
The correct option is \textbf{4) a parabola}.

\begin{figure}[h!]
	\centering
	\includegraphics[width=0.8\columnwidth]{figs/plot_c.jpg}
	\caption*{Plot}
	\label{fig:fig}
\end{figure}

\end{document}
