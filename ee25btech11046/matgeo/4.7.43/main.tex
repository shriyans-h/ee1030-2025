\documentclass[journal]{IEEEtran}
\usepackage[a5paper, margin=10mm, onecolumn]{geometry}
\usepackage{amsmath,amssymb,amsfonts,amsthm}
\usepackage{gvv-book}
\usepackage{gvv}
\usepackage{hyperref}

\begin{document}

\title{4.7.43}
\author{Puni Aditya - EE25BTECH11046}
\maketitle

\textbf{Question:}

Show that the points $\brak{\hat{i}-\hat{j}+3\hat{k}}$ and $3\brak{\hat{i}+\hat{j}+\hat{k}}$ are equidistant from the plane $\vec{r}\cdot\brak{5\hat{i}+2\hat{j}-7\hat{k}} + 9 = 0$ and lie on opposite sides of it.

\textbf{Solution:}

Let the given points be $\vec{P_1} = \myvec{1 \\ -1 \\ 3}$ and $\vec{P_2} = \myvec{3 \\ 3 \\ 3}$.
The equation of the given plane is
\begin{align}
    \myvec{5 & 2 & -7}\vec{x} + 9 = 0
\end{align}
This can be written in the standard form $\vec{n}^\top\vec{x} = k$. Here, $\vec{n} = \myvec{5 \\ 2 \\ -7}$ and $k = -9$.
\begin{align}
    \myvec{5 & 2 & -7}\vec{x} = -9 \label{eq:plane_eqn}
\end{align}

The perpendicular distance of a point with position vector $\vec{P}$ from the plane $\vec{n}^\top\vec{x} = k$ is given by the formula
\begin{align}
    D = \frac{\abs{\vec{n}^\top\vec{P} - k}}{\norm{\vec{n}}}
\end{align}
\begin{align}
    \norm{\vec{n}} &= \sqrt{5^2 + 2^2 + \brak{-7}^2} \\
    &= \sqrt{25 + 4 + 49} = \sqrt{78}
\end{align}
Distance $D_1$ of the point $\vec{P_1}$ from the plane is
\begin{align}
    D_1 &= \frac{\abs{\myvec{5 & 2 & -7}\myvec{1 \\ -1 \\ 3} - \brak{-9}}}{\sqrt{78}} \\
    &= \frac{\abs{\brak{5}\brak{1} + \brak{2}\brak{-1} + \brak{-7}\brak{3} + 9}}{\sqrt{78}} \\
    &= \frac{\abs{5 - 2 - 21 + 9}}{\sqrt{78}} \\
    &= \frac{\abs{-9}}{\sqrt{78}} = \frac{9}{\sqrt{78}} \label{eq:distance1}
\end{align}
Distance $D_2$ of the point $\vec{P_2}$ from the plane is
\begin{align}
    D_2 &= \frac{\abs{\myvec{5 & 2 & -7}\myvec{3 \\ 3 \\ 3} - \brak{-9}}}{\sqrt{78}} \\
    &= \frac{\abs{\brak{5}\brak{3} + \brak{2}\brak{3} + \brak{-7}\brak{3} + 9}}{\sqrt{78}} \\
    &= \frac{\abs{15 + 6 - 21 + 9}}{\sqrt{78}} \\
    &= \frac{\abs{9}}{\sqrt{78}} = \frac{9}{\sqrt{78}} \label{eq:distance2}
\end{align}
From \eqref{eq:distance1} and \eqref{eq:distance2}, $D_1 = D_2$. Thus, the points are equidistant from the plane. \\

\begin{align}
    \vec{n}^\top\vec{P_1} - k &= \myvec{5 & 2 & -7}\myvec{1 \\ -1 \\ 3} - \brak{-9} = -9 \label{eq:3} \\
    \vec{n}^\top\vec{P_2} - k &= \myvec{5 & 2 & -7}\myvec{3 \\ 3 \\ 3} - \brak{-9} = 9 \label{eq:4}
\end{align}

From \eqref{eq:3} and \eqref{eq:4}, 
\begin{align}
    \brak{\vec{n}^\top\vec{P_1}-k}\brak{\vec{n}^\top\vec{P_2}-k} = -81 < 0
\end{align}

$\because \brak{\vec{n}^\top\vec{P_1}-k}\brak{\vec{n}^\top\vec{P_2}-k} < 0$, the points $\vec{P_1}$ and $\vec{P_2}$ lie on opposite sides of the plane.

\begin{figure}
    \centering
    \includegraphics[width=\columnwidth]{figs/plot_c.jpg}
    \caption*{Plot}
    \label{fig:fig}
\end{figure}

\end{document}
