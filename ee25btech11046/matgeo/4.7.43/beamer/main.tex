\documentclass{beamer}
\usepackage[utf8]{inputenc}

\usetheme{Madrid}
\usecolortheme{default}
\usepackage{amsmath,amssymb,amsfonts,amsthm}
\usepackage{txfonts}
\usepackage{tkz-euclide}
\usepackage{listings}
\usepackage{adjustbox}
\usepackage{array}
\usepackage{tabularx}
\usepackage{gvv}
\usepackage{lmodern}
\usepackage{circuitikz}
\usepackage{tikz}
\usepackage{graphicx}

\setbeamertemplate{page number in head/foot}[totalframenumber]

\usepackage{tcolorbox}
\tcbuselibrary{minted,breakable,xparse,skins}



\definecolor{bg}{gray}{0.95}
\DeclareTCBListing{mintedbox}{O{}m!O{}}{%
  breakable=true,
  listing engine=minted,
  listing only,
  minted language=#2,
  minted style=default,
  minted options={%
    linenos,
    gobble=0,
    breaklines=true,
    breakafter=,,
    fontsize=\small,
    numbersep=8pt,
    #1},
  boxsep=0pt,
  left skip=0pt,
  right skip=0pt,
  left=25pt,
  right=0pt,
  top=3pt,
  bottom=3pt,
  arc=5pt,
  leftrule=0pt,
  rightrule=0pt,
  bottomrule=2pt,
  toprule=2pt,
  colback=bg,
  colframe=orange!70,
  enhanced,
  overlay={%
    \begin{tcbclipinterior}
    \fill[orange!20!white] (frame.south west) rectangle ([xshift=20pt]frame.north west);
    \end{tcbclipinterior}},
  #3,
}
\lstset{
    language=C,
    basicstyle=\ttfamily\small,
    keywordstyle=\color{blue},
    stringstyle=\color{orange},
    commentstyle=\color{green!60!black},
    numbers=left,
    numberstyle=\tiny\color{gray},
    breaklines=true,
    showstringspaces=false,
}
\title{4.3.36}
\date{12th September, 2025}
\author{Puni Aditya - EE25BTECH11046}

\begin{document}

\frame{\titlepage}
\begin{frame}{Question}
Show that the points $\brak{\hat{i}-\hat{j}+3\hat{k}}$ and $3\brak{\hat{i}+\hat{j}+\hat{k}}$ are equidistant from the plane $\vec{r}\cdot\brak{5\hat{i}+2\hat{j}-7\hat{k}} + 9 = 0$ and lie on opposite sides of it.
\end{frame}

\begin{frame}{Theoretical Solution}
Let the given points be $\vec{P_1} = \myvec{1 \\ -1 \\ 3}$ and $\vec{P_2} = \myvec{3 \\ 3 \\ 3}$.
The equation of the given plane is
\begin{align}
    \myvec{5 & 2 & -7}\vec{x} + 9 = 0
\end{align}
This can be written in the standard form $\vec{n}^\top\vec{x} = k$. Here, $\vec{n} = \myvec{5 \\ 2 \\ -7}$ and $k = -9$.
\begin{align}
    \myvec{5 & 2 & -7}\vec{x} = -9 \label{eq:plane_eqn}
\end{align}
\end{frame}

\begin{frame}{Theoretical Solution}
The perpendicular distance of a point with position vector $\vec{P}$ from the plane $\vec{n}^\top\vec{x} = k$ is given by the formula
\begin{align}
    D = \frac{\abs{\vec{n}^\top\vec{P} - k}}{\norm{\vec{n}}}
\end{align}
\begin{align}
    \norm{\vec{n}} &= \sqrt{5^2 + 2^2 + \brak{-7}^2} \\
    &= \sqrt{25 + 4 + 49} = \sqrt{78}
\end{align}
Distance $D_1$ of the point $\vec{P_1}$ from the plane is
\begin{align}
    D_1 &= \frac{\abs{\myvec{5 & 2 & -7}\myvec{1 \\ -1 \\ 3} - \brak{-9}}}{\sqrt{78}} \\
    &= \frac{\abs{5 - 2 - 21 + 9}}{\sqrt{78}} \\
    &= \frac{\abs{-9}}{\sqrt{78}} = \frac{9}{\sqrt{78}} \label{eq:distance1}
\end{align}
\end{frame}

\begin{frame}{Theoretical Solution}
Distance $D_2$ of the point $\vec{P_2}$ from the plane is
\begin{align}
    D_2 &= \frac{\abs{\myvec{5 & 2 & -7}\myvec{3 \\ 3 \\ 3} - \brak{-9}}}{\sqrt{78}} \\
    &= \frac{\abs{15 + 6 - 21 + 9}}{\sqrt{78}} \\
    &= \frac{\abs{9}}{\sqrt{78}} = \frac{9}{\sqrt{78}} \label{eq:distance2}
\end{align}
From \eqref{eq:distance1} and \eqref{eq:distance2}, $D_1 = D_2$. Thus, the points are equidistant from the plane.
\begin{align}
    \vec{n}^\top\vec{P_1} - k &= \myvec{5 & 2 & -7}\myvec{1 \\ -1 \\ 3} - \brak{-9} = -9 \label{eq:3}
\end{align}
\end{frame}

\begin{frame}{Theoretical Solution}
\begin{align}
    \vec{n}^\top\vec{P_2} - k &= \myvec{5 & 2 & -7}\myvec{3 \\ 3 \\ 3} - \brak{-9} = 9 \label{eq:4}
\end{align}
From \eqref{eq:3} and \eqref{eq:4}, 
\begin{align}
    \brak{\vec{n}^\top\vec{P_1}-k}\brak{\vec{n}^\top\vec{P_2}-k} = -81 < 0
\end{align}

$\because \brak{\vec{n}^\top\vec{P_1}-k}\brak{\vec{n}^\top\vec{P_2}-k} < 0$, the points $\vec{P_1}$ and $\vec{P_2}$ lie on opposite sides of the plane.
\end{frame}

\begin{frame}{Plot}
    \begin{figure}
        \centering
        \includegraphics[width=0.5\columnwidth]{../figs/plot_c.jpg}
        \caption{Plot}
        \label{fig:fig}
    \end{figure}
\end{frame}

\end{document}