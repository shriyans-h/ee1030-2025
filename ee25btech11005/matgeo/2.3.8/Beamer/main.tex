\documentclass{beamer}
\usepackage[utf8]{inputenc}
\usetheme{Madrid}
\usecolortheme{default}
\usepackage{amsmath,amssymb,amsfonts,amsthm}
\usepackage{txfonts}
\usepackage{tkz-euclide}
\usepackage{listings}
\usepackage{adjustbox}
\usepackage{array}
\usepackage{tabularx}
\usepackage{gvv}
\usepackage{lmodern}
\usepackage{circuitikz}
\usepackage{tikz}
\usepackage{graphicx}
\setbeamertemplate{page number in head/foot}[totalframenumber]
\definecolor{bg}{gray}{0.95}
\lstset{
    language=C,
        basicstyle=\ttfamily\small,
            keywordstyle=\color{blue},
                stringstyle=\color{orange},
                    commentstyle=\color{green!60!black},
                        numbers=left,
                            numberstyle=\tiny\color{gray},
                                breaklines=true,
                                    showstringspaces=false,
                                    }
\title{2.3.8}
                                    \author{Aditya Mishra - EE25BTECH11005}
                                    \date{September 15, 2025}
                                    \begin{document}
                                    \frame{\titlepage}

                                    \begin{frame}{Question}
                                    If $\hat{i} + \hat{j} + \hat{k},\ 2\hat{i} + 5\hat{j},\ 3\hat{i} + 2\hat{j} - 3\hat{k},\ \hat{i} - 6\hat{j} - \hat{k}$ respectively are the position vectors of points $A, B, C,$ and $D$, then find the angle between the straight lines $\vec{AB}$ and $\vec{CD}$. Also, determine whether $\vec{AB}$ and $\vec{CD}$ are collinear.
                                    \end{frame}

                                    \begin{frame}{Given Information}
                                    \centering
                                    \label{tab:parameters}
                                    Let $\vec{A} = \begin{pmatrix}1\\1\\1\end{pmatrix}$, $\vec{B} = \begin{pmatrix}2\\5\\0\end{pmatrix}$, $\vec{C} = \begin{pmatrix}3\\2\\-3\end{pmatrix}$, $\vec{D} = \begin{pmatrix}1\\-6\\-1\end{pmatrix}$

                                    Direction vectors:
                                    \begin{align*}
                                    \vec{AB} &= \vec{B} - \vec{A} = \begin{pmatrix}1\\4\\-1\end{pmatrix} \\
                                    \vec{CD} &= \vec{D} - \vec{C} = \begin{pmatrix}-2\\-8\\2\end{pmatrix}
                                    \end{align*}
                                    \end{frame}

                                    \begin{frame}{Formula}
                                   \begin{align}
                                    \cos \theta = \frac{\vec{AB}^T \vec{CD}}{|\vec{AB}|\,|\vec{CD}|}
                                    \end{align}
                                    \begin{align}
                                    \theta = \cos^{-1}\left(\frac{\vec{AB}^T \vec{CD}}{|\vec{AB}|\,|\vec{CD}|}\right)
                                    \end{align}
                                    Vectors are collinear if $\theta = 0^\circ$ or $180^\circ$.
                                    \end{frame}

                                    \begin{frame}{Solution}
                                    \[
                                    \vec{AB}^T \vec{CD}
                                    = \begin{pmatrix}1 & 4 & -1\end{pmatrix}
                                    \begin{pmatrix}-2 \\ -8 \\ 2\end{pmatrix}
                                    = 1\times(-2) + 4\times(-8) + (-1)\times 2 = -2 -32 -2 = -36
                                    \]
                                    \[
                                    |\vec{AB}| = \sqrt{1^2 + 4^2 + (-1)^2} = \sqrt{18}
                                    \]
                                    \[
                                    |\vec{CD}| = \sqrt{(-2)^2 + (-8)^2 + 2^2} = \sqrt{4 + 64 + 4} = \sqrt{72}
                                    \]
                                    \[
                                    \cos\theta = \frac{-36}{\sqrt{18}\times\sqrt{72}} = \frac{-36}{36} = -1
                                    \]
                                    \[
                                    \theta = \cos^{-1}(-1) = 180^\circ
                                    \]
                                    Therefore, $\vec{AB}$ and $\vec{CD}$ are collinear but point in opposite directions.
                                    \end{frame}


\begin{frame}[fragile]
\frametitle{Python Plot Code}
\begin{lstlisting}[language=Python]
import numpy as np
from mpl_toolkits.mplot3d import Axes3D
import matplotlib.pyplot as plt
A = np.array([1, 1, 1])
B = np.array([2, 5, 0])
C = np.array([3, 2, -3])
D = np.array([1, -6, -1])
fig = plt.figure()
ax = fig.add_subplot(111, projection='3d')
ax.quiver(*A, *(B-A), color='b', label='AB')
ax.quiver(*C, *(D-C), color='r', label='CD')
\end{lstlisting} 
\end{frame}
\begin{frame}[fragile]
\frametitle{Python Plot Code}

\begin{lstlisting}
ax.scatter(*A, color='blue')
ax.scatter(*B, color='blue')
ax.scatter(*C, color='red')
ax.scatter(*D, color='red')
ax.text(*A, 'A')
ax.text(*B, 'B')
ax.text(*C, 'C')
ax.text(*D, 'D')
plt.show()
\end{lstlisting}
\end{frame}

\begin{frame}[fragile]
\frametitle{C Code}
\begin{lstlisting}[language=C]
#include<stdio.h>
#include<math.h>
double dot(double *a, double *b) {
double r=0;
for(int i=0;i<3;i++)
    r += a[i]*b[i];
return r;
}
double norm(double *a) {
return sqrt(a[0]*a[0] + a[1]*a[1] + a[2]*a[2]);}
\end{lstlisting} 
\end{frame}
\begin{frame}[fragile]
\frametitle{C Code}

\begin{lstlisting}
double angle_deg(double *a, double *b) {
double d = dot(a, b);
double n1 = norm(a);
double n2 = norm(b);
double ratio = d/(n1*n2);
if (ratio > 1) ratio=1;
if (ratio < -1) ratio=-1;
return acos(ratio)*180.0/M_PI;}
int collinear(double *a, double *b) {
    double ang = angle_deg(a, b);
    return (ang < 1e-8 || fabs(ang - 180) < 1e-8);
    }
\end{lstlisting}
\end{frame}

\begin{frame}[fragile]
\frametitle{Python and C Code}
\begin{lstlisting}[language=Python]
    import numpy as np
    import ctypes
    lib = ctypes.CDLL('./mat3.so')
    lib.angle_deg.argtypes = [ctypes.POINTER(ctypes.c_double), ctypes.POINTER(ctypes.c_double)]
    lib.angle_deg.restype = ctypes.c_double
    A = np.array([1., 1., 1.])
    B = np.array([2., 5., 0.])
    C = np.array([3., 2., -3.])
    D = np.array([1., -6., -1.])                          
    AB = B - A
    CD = D - C
    AB_c = np.ascontiguousarray(AB, dtype=np.double).ctypes.data_as(ctypes.POINTER(ctypes.c_double))
    \end{lstlisting} 
    \end{frame}
    \begin{frame}[fragile]
    \frametitle{Python and C Code}

    \begin{lstlisting}
    CD_c = np.ascontiguousarray(CD, dtype=np.double).ctypes.data_as(ctypes.POINTER(ctypes.c_double))
    angle = lib.angle_deg(AB_c, CD_c)
    print("Angle between AB and CD (degrees):", angle)
    lib.collinear.argtypes = [ctypes.POINTER(ctypes.c_double), ctypes.POINTER(ctypes.c_double)]
    
    lib.collinear.restype = ctypes.c_int
    is_collinear = lib.collinear(AB_c, CD_c)
    print("Are AB and CD collinear?", "Yes" if is_collinear else "No")
    \end{lstlisting}
    \end{frame}

                                                                                                                \begin{frame}{Plot}
                                                                                                                \begin{figure}
                                                                                                                    \centering
                                                                                                                        \includegraphics[width=0.8\columnwidth]{Figs/Figure.png}
                                                                                                                        \end{figure}
                                                                                                                        \end{frame}

                                                                                                                                \end{document}
