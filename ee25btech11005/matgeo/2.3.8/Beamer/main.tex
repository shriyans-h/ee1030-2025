\documentclass{beamer}
\usepackage[utf8]{inputenc}
\usetheme{Madrid}
\usecolortheme{default}
\usepackage{amsmath,amssymb,amsfonts,amsthm}
\usepackage{txfonts}
\usepackage{tkz-euclide}
\usepackage{listings}
\usepackage{adjustbox}
\usepackage{array}
\usepackage{tabularx}
\usepackage{gvv}
\usepackage{lmodern}
\usepackage{circuitikz}
\usepackage{tikz}
\usepackage{graphicx}
\setbeamertemplate{page number in head/foot}[totalframenumber]
\definecolor{bg}{gray}{0.95}
\lstset{
    language=C,
    basicstyle=\ttfamily\small,
    keywordstyle=\color{blue},
    stringstyle=\color{orange},
    commentstyle=\color{green!60!black},
    numbers=left,
    numberstyle=\tiny\color{gray},
    breaklines=true,
    showstringspaces=false,
}
\title{2.3.8}
\author{Aditya Mishra - EE25BTECH11005}
\date{September 15, 2025}
\begin{document}
\frame{\titlepage}
\begin{frame}{Question}
If $\hat{i} + \hat{j} + \hat{k},\ 2\hat{i} + 5\hat{j},\ 3\hat{i} + 2\hat{j} - 3\hat{k},\ \hat{i} - 6\hat{j} - \hat{k}$ respectively are the position vectors of points $A, B, C,$ and $D$, then find the angle between the straight lines $(\vec{B}-\vec{A})$ and $(\vec{D}-\vec{C})$. Find whether $(\vec{B}-\vec{A})$ and $(\vec{D}-\vec{C})$ are collinear or not.
\end{frame}
\begin{frame}{Given Information}
\centering
\label{tab:parameters}
Let $\vec{A}$ = $\begin{pmatrix}1\\1\\1\end{pmatrix}$,\ 
$\vec{B}$ = $\begin{pmatrix}2\\5\\0\end{pmatrix}$,\
$\vec{C}$ = $\begin{pmatrix}3\\2\\-3\end{pmatrix}$,\
$\vec{D}$ = $\begin{pmatrix}1\\-6\\-1\end{pmatrix}$

Direction vectors:
\begin{align}
\vec{B}-\vec{A} = \begin{pmatrix}1\\4\\-1\end{pmatrix} \\
\vec{D}-\vec{C} = \begin{pmatrix}-2\\-8\\2\end{pmatrix}
\end{align}
\end{frame}

\begin{frame}{Formula}
The angle $\theta$ between $(\vec{B}-\vec{A})$ and $(\vec{D}-\vec{C})$ is
\begin{align}
\cos \theta = \frac{(\vec{B}-\vec{A})^T (\vec{D}-\vec{C})}
{|(\vec{B}-\vec{A})| |(\vec{D}-\vec{C})|}
\end{align}
\begin{align}
\theta = \cos^{-1}\left(
\frac{(\vec{B}-\vec{A})^T (\vec{D}-\vec{C})}
{|(\vec{B}-\vec{A})| |(\vec{D}-\vec{C})|}\right)
\end{align}
Vectors are collinear if $\theta = 0^\circ$ or $180^\circ$.
\end{frame}
\begin{frame}{Solution}
\[
(\vec{B}-\vec{A})^T (\vec{D}-\vec{C})
= \begin{pmatrix}1 & 4 & -1\end{pmatrix}
\begin{pmatrix}-2 \\ -8 \\ 2\end{pmatrix}
= 1\times(-2) + 4\times(-8) + (-1)\times 2 = -36
\]
\[
|(\vec{B}-\vec{A})| = \sqrt{1^2 + 4^2 + (-1)^2} = \sqrt{18}
\]
\[
|(\vec{D}-\vec{C})| = \sqrt{(-2)^2 + (-8)^2 + 2^2} = \sqrt{72}
\]
\[
\cos\theta = \frac{-36}{\sqrt{18} \times \sqrt{72}} = \frac{-36}{36} = -1
\]
\[
\theta = \cos^{-1}(-1) = 180^\circ
\]
Therefore, $(\vec{B}-\vec{A})$ and $(\vec{D}-\vec{C})$ are collinear but in opposite directions.
\end{frame}
\begin{frame}{Plot}
\begin{figure}
    \centering
    \includegraphics[width=0.8\columnwidth]{Figs/Figure.png}
\end{figure}
\end{frame}
\begin{frame}{Codes}
\centering
    For Codes, refer to the URL below:
    https://github.com/Aditya-Mishra11005/ee1030-2025/tree/main/ee25btech11005/matgeo/2.3.8/Codes
\end{frame}
\end{document}

