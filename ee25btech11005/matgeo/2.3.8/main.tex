\documentclass[12pt]{article}
\usepackage{graphicx}
\usepackage{enumitem}
\usepackage{amsmath}
\usepackage{gvv-book}
\usepackage{gvv}
\title{\textbf{2.3.8}}
\author{\textbf{EE25BTECH11005 - Aditya Mishra}}
\date{September 12, 2025}
\begin{document}
\maketitle
\section*{Question}
If $ \vec{A} = \hat{i} + \hat{j} + \hat{k},\ \vec{B} = 2\hat{i} + 5\hat{j},\ \vec{C} = 3\hat{i} + 2\hat{j} - 3\hat{k},\ \vec{D} = \hat{i} - 6\hat{j} - \hat{k}$ are the position vectors of points A, B, C and D, then find the angle between the straight lines $AB$ and $CD$. Find whether $(\vec{B}-\vec{A})$ and $(\vec{D}-\vec{C})$ are collinear or not.

\section*{Solution}
Let the direction vectors be
\begin{align}
\vec{B} - \vec{A} &= \myvec{2\\5\\0} - \myvec{1\\1\\1} = \myvec{1\\4\\-1} \\
\vec{D} - \vec{C} &= \myvec{1\\-6\\-1} - \myvec{3\\2\\-3} = \myvec{-2\\-8\\2}
\end{align}

The angle $\theta$ between $(\vec{B}-\vec{A})$ and $(\vec{D}-\vec{C})$ is given by
\begin{align}
\cos\theta = \frac{(\vec{B}-\vec{A})^\top (\vec{D}-\vec{C})}{|(\vec{B}-\vec{A})|\,|(\vec{D}-\vec{C})|}
\end{align}

Substitute the values:
\begin{align}
(\vec{B}-\vec{A})^T (\vec{D}-\vec{C}) &= \myvec{1\\4\\-1}^\top \myvec{-2\\-8\\2} \\
&= (1)(-2) + (4)(-8) + (-1)(2) \\
&= -2 - 32 - 2 = -36 
\end{align}

Magnitudes:
\begin{align}
|(\vec{B}-\vec{A})| &= \sqrt{1^2 + 4^2 + (-1)^2} = \sqrt{1 + 16 + 1} = \sqrt{18} \\
|(\vec{D}-\vec{C})| &= \sqrt{(-2)^2 + (-8)^2 + 2^2} = \sqrt{4 + 64 + 4} = \sqrt{72}
\end{align}

Thus,
\begin{align}
\cos\theta &= \frac{-36}{\sqrt{18}\sqrt{72}}
= \frac{-36}{\sqrt{1296}}
= \frac{-36}{36}
= -1
\end{align}
\begin{align}
\theta = \cos^{-1}(-1) = \pi \text{ radians } = 180^\circ
\end{align}

So, $(\vec{B}-\vec{A})$ and $(\vec{D}-\vec{C})$ are collinear but point in opposite directions, i.e., they are anti-parallel.

\vspace{0.5cm}
\centering
The lines are collinear (anti-parallel).

\begin{figure}[H]
    \centering
    \includegraphics[width=0.7\columnwidth]{Figs/Figure.png}
    \caption{Line directions}
    \label{fig:ab_cd}
\end{figure}
\end{document}

