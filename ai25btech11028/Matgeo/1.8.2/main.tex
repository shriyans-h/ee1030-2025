\let\negmedspace\undefined
\let\negthickspace\undefined
\documentclass[journal,12pt,onecolumn]{IEEEtran}
\usepackage{cite}
\usepackage{amsmath,amssymb,amsfonts,amsthm}
\usepackage{algorithmic}
\usepackage{graphicx}
\graphicspath{{./figs/}}
\usepackage{textcomp}
\usepackage{xcolor}
\usepackage{txfonts}
\usepackage{listings}
\usepackage{enumitem}
\usepackage{mathtools}
\usepackage{gensymb}
\usepackage{comment}
\usepackage{caption}
\usepackage[breaklinks=true]{hyperref}
\usepackage{tkz-euclide} 
\usepackage{listings}
\usepackage{gvv}                                        
\def\inputGnumericTable{}                                 
\usepackage[latin1]{inputenc}     
\usepackage{xparse}
\usepackage{color}                                            
\usepackage{array}                                            
\usepackage{longtable}                                       
\usepackage{calc}                                             
\usepackage{multirow}
\usepackage{multicol}
\usepackage{hhline}                                           
\usepackage{ifthen}                                           
\usepackage{lscape}
\usepackage{tabularx}
\usepackage{array}
\usepackage{float}
\newtheorem{theorem}{Theorem}[section]

\newtheorem{problem}{Problem}
\newtheorem{proposition}{Proposition}[section]
\newtheorem{lemma}{Lemma}[section]
\newtheorem{corollary}[theorem]{Corollary}
\newtheorem{example}{Example}[section]
\newtheorem{definition}[problem]{Definition}

\title{1.8.2}
\author{AI25BTECH11028 -R.Manohar}
\begin{document}
\maketitle
{\let\newpage\relax\maketitle}
\renewcommand{\thefigure}{\theenumi}\renewcommand{\thetable}{\theenumi}
 \bigskip

\textbf{Question:}
Find the distance between the following pairs of points:
\begin{enumerate}
\item (2,3,5) and (4,3,1)
\item (-3,7,2) and (2,4,-1)
\item (-1,3,-4) and (1,-3,4)
\item (2,-1,3) and (-2,1,3)
\end{enumerate}

\textbf{Solution:}
We know that,

The length of a vector is defined as
\begin{align}
    \|\vec{x}\| = \sqrt{\vec{x}^\top \vec{x}}
\end{align}

Therefore,

 distance between $\vec{P}$ and $\vec{Q}$ is
\begin{align}
d(\vec{P},\vec{Q}) = \|\vec{P}-\vec{Q}\|
= \sqrt{(\vec{P}-\vec{Q})^\top(\vec{P}-\vec{Q})}.
\end{align}

Let,
\begin{enumerate}
\item $\vec{A} = \myvec{2\\3\\5}, \;\vec{B} = \myvec{4\\3\\1}$
\begin{align*}
d(\vec{A},\vec{B}) &= \|\myvec{2\\3\\5}-\myvec{4\\3\\1}\|
= \|\myvec{-2\\0\\4}\| = \sqrt{\myvec{-2\\0\\4}^\top\myvec{-2\\0\\4}} = \sqrt{(-2)^2+0^2+4^2} = \sqrt{20} = 2\sqrt{5}. \\[10pt]
\end{align*}

\item $\vec{C} = \myvec{-3\\7\\2}, \;\vec{D} = \myvec{2\\4\\-1}$
\begin{align*}
d(\vec{C},\vec{D}) &= \|\myvec{-3\\7\\2}-\myvec{2\\4\\-1}\|
= \|\myvec{-5\\3\\3}\| = \sqrt{\myvec{-5\\3\\3}^\top\myvec{-5\\3\\3}} = \sqrt{(-5)^2+3^2+3^2} = \sqrt{43}. \\[10pt]
\end{align*}

\item $\vec{E} = \myvec{-1\\3\\-4}, \;\vec{F} = \myvec{1\\-3\\4}$
\begin{align*}
d(\vec{E},\vec{F}) &= \|\myvec{-1\\3\\-4}-\myvec{1\\-3\\4}\|
= \|\myvec{-2\\6\\-8}\| = \sqrt{\myvec{-2\\6\\-8}^\top\myvec{-2\\6\\-8}} = \sqrt{(-2)^2+6^2+(-8)^2} = \sqrt{104} = 2\sqrt{26}. \\[10pt]
\end{align*}
\item $\vec{G} = \myvec{2\\-1\\3}, \;\vec{H} = \myvec{-2\\1\\3}$
\begin{align*}
d(\vec{G},\vec{H}) &= \|\myvec{2\\-1\\3}-\myvec{-2\\1\\3}\|
= \|\myvec{4\\-2\\0}\| = \sqrt{\myvec{4\\-2\\0}^\top\myvec{4\\-2\\0}} = \sqrt{4^2+(-2)^2+0^2} = \sqrt{20} = 2\sqrt{5}.
\end{align*}
\end{enumerate}

\section*{Figure}
\begin{center}
\begin{figure}[H]
    \centering
    \includegraphics[scale=0.4]{figs/Figure_1.png}
    \caption{}
    \label{fig:1}
\end{figure}
\end{center}

\end{document}


