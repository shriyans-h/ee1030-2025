\documentclass{beamer}
\usepackage[utf8]{inputenc}

\usetheme{Madrid}
\usecolortheme{default}
\usepackage{amsmath,amssymb,amsfonts,amsthm}
\usepackage{txfonts}
\usepackage{tkz-euclide}
\usepackage{listings}
\usepackage{adjustbox}
\usepackage{array}
\usepackage{tabularx}
\usepackage{gvv}
\usepackage{lmodern}
\usepackage{circuitikz}
\usepackage{tikz}
\usepackage{graphicx}

\setbeamertemplate{page number in head/foot}[totalframenumber]

\usepackage{tcolorbox}
\tcbuselibrary{minted,breakable,xparse,skins}



\definecolor{bg}{gray}{0.95}
\DeclareTCBListing{mintedbox}{O{}m!O{}}{
  breakable=true,
  listing engine=minted,
  listing only,
  minted language=#2,
  minted style=default,
  minted options={
    linenos,
    gobble=0,
    breaklines=true,
    breakafter=,,
    fontsize=\small,
    numbersep=8pt,
    #1},
  boxsep=0pt,
  left skip=0pt,
  right skip=0pt,
  left=25pt,
  right=0pt,
  top=3pt,
  bottom=3pt,
  arc=5pt,
  leftrule=0pt,
  rightrule=0pt,
  bottomrule=2pt,
  toprule=2pt,
  colback=bg,
  colframe=orange!70,
  enhanced,
  overlay={
    \begin{tcbclipinterior}
    \fill[orange!20!white] (frame.south west) rectangle ([xshift=20pt]frame.north west);
    \end{tcbclipinterior}},
  #3,
}
\lstset{
    language=C,
    basicstyle=\ttfamily\small,
    keywordstyle=\color{blue},
    stringstyle=\color{orange},
    commentstyle=\color{green!60!black},
    numbers=left,
    numberstyle=\tiny\color{gray},
    breaklines=true,
    showstringspaces=false,
}

\title 
{1.8.2}

\author 
{R.Manohar-AI25BTECH11028}



\begin{document}


\frame{\titlepage}
\begin{frame}{Question}
Find the distance between the following pairs of points:
\begin{enumerate}
\item (2,3,5) and (4,3,1)
\item (-3,7,2) and (2,4,-1)
\item (-1,3,-4) and (1,-3,4)
\item (2,-1,3) and (-2,1,3)
\end{enumerate}
\end{frame}



\begin{frame}{Theoretical Solution}

We know that,

The length of a vector is defined as
\begin{align}
    \|\vec{x}\| = \sqrt{\vec{x}^\top \vec{x}}
\end{align}

Therefore,

 distance between $\vec{P}$ and $\vec{Q}$ is
\begin{align}
d(\vec{P},\vec{Q}) = \|\vec{P}-\vec{Q}\|
= \sqrt{(\vec{P}-\vec{Q})^\top(\vec{P}-\vec{Q})}.
\end{align}

\end{frame}

\begin{frame}{Theoretical Solution}
Let,

$\vec{A} = \myvec{2\\3\\5}, \;\vec{B} = \myvec{4\\3\\1}$
\begin{align*}
d(\vec{A},\vec{B}) &= \|\myvec{2\\3\\5}-\myvec{4\\3\\1}\|
= \|\myvec{-2\\0\\4}\| \\
&= \sqrt{\myvec{-2\\0\\4}^\top\myvec{-2\\0\\4}} = \sqrt{(-2)^2+0^2+4^2} = \sqrt{20} = 2\sqrt{5}. \\[10pt]
\end{align*}


\end{frame}
\begin{frame}{Theoretical Solution}
$\vec{C} = \myvec{-3\\7\\2}, \;\vec{D} = \myvec{2\\4\\-1}$
\begin{align*}
d(\vec{C},\vec{D}) &= \|\myvec{-3\\7\\2}-\myvec{2\\4\\-1}\|
= \|\myvec{-5\\3\\3}\|\\ 
&= \sqrt{\myvec{-5\\3\\3}^\top\myvec{-5\\3\\3}} = \sqrt{(-5)^2+3^2+3^2} = \sqrt{43}. \\[10pt]
\end{align*}
\end{frame}

\begin{frame}{Theoretical Solution}

$\vec{E} = \myvec{-1\\3\\-4}, \;\vec{F} = \myvec{1\\-3\\4}$
\begin{align*}
d(\vec{E},\vec{F}) &= \|\myvec{-1\\3\\-4}-\myvec{1\\-3\\4}\|
= \|\myvec{-2\\6\\-8}\| \\
&= \sqrt{\myvec{-2\\6\\-8}^\top\myvec{-2\\6\\-8}} = \sqrt{(-2)^2+6^2+(-8)^2} = \sqrt{104} = 2\sqrt{26}. \\[10pt]
\end{align*}


\end{frame}


\begin{frame}{Theoritical Solution}

$\vec{G} = \myvec{2\\-1\\3}, \;\vec{H} = \myvec{-2\\1\\3}$
\begin{align*}
d(\vec{G},\vec{H}) = \|\myvec{2\\-1\\3}-\myvec{-2\\1\\3}\|
= \|\myvec{4\\-2\\0}\|\\
= \sqrt{\myvec{4\\-2\\0}^\top\myvec{4\\-2\\0}} = \sqrt{4^2+(-2)^2+0^2} = \sqrt{20} = 2\sqrt{5}.
\end{align*}
\end{frame}

\begin{frame}[fragile]
    \frametitle{C Code - Sum of vectors and Magnitude of vectors}

    \begin{lstlisting}
#include<stdio.h>
#include<math.h>

#define DIST(p,q) sqrt( ((p[0]-q[0])*(p[0]-q[0])) + \
                        ((p[1]-q[1])*(p[1]-q[1])) + \
                        ((p[2]-q[2])*(p[2]-q[2])) )
int main()
{
    int A[3] = {2,3,5}, B[3] = {4,3,1};
    int C[3] = {-3,7,2}, D[3] = {2,4,-1};
    int E[3] = {-1,3,-4}, F[3] = {1,-3,4};
    int G[3] = {2,-1,3}, H[3] = {-2,1,3};
    printf("AB = %.3f\n", DIST(A,B));
    printf("CD = %.3f\n", DIST(C,D));
    printf("EF = %.3f\n", DIST(E,F));
    printf("GH = %.3f\n", DIST(G,H));

    return 0;
}

    \end{lstlisting}
\end{frame}

\begin{frame}[fragile]
\frametitle{Python Code - Distance calculation}
\begin{lstlisting}
import numpy as np
import matplotlib.pyplot as plt

def distance3D(p1, p2):
    return np.linalg.norm(np.array(p1) - np.array(p2))

# Points
A, B = (2,3,5), (4,3,1)
C, D = (-3,7,2), (2,4,-1)
E, F = (-1,3,-4), (1,-3,4)
G, H = (2,-1,3), (-2,1,3)

print("AB =", distance3D(A,B))
print("CD =", distance3D(C,D))
print("EF =", distance3D(E,F))
print("GH =", distance3D(G,H))
\end{lstlisting}
\end{frame}

\begin{frame}[fragile]
\frametitle{Python Code - 3D Plot}
\begin{lstlisting}

fig = plt.figure(figsize=(8,6))
ax = fig.add_subplot(111, projection='3d')

pairs = [(A,B,'r'),(C,D,'g'),(E,F,'b'),(G,H,'k')]
for P, Q, c in pairs:
    ax.plot([P[0],Q[0]], [P[1],Q[1]], [P[2],Q[2]], c+"-o")

ax.set_xlabel("X")
ax.set_ylabel("Y")
ax.set_zlabel("Z")
ax.set_title("3D Segments between given points")
plt.savefig("/home/user/Matrix/Matgeo_assignments/1.9.15/figs/Figure_1.png", dpi=300, bbox_inches='tight')
plt.show()

\end{lstlisting}
\end{frame}


\begin{frame}{Plot}
    \centering
    \includegraphics[width=\columnwidth, height=0.8\textheight, keepaspectratio]{figs/Figure_1.png}     
\end{frame}




\end{document}