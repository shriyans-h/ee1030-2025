\let\negmedspace\undefined
\let\negthickspace\undefined
\documentclass[journal]{IEEEtran}
\usepackage[a5paper, margin=10mm, onecolumn]{geometry}
%\usepackage{lmodern} % Ensure lmodern is loaded for pdflatex
\usepackage{tfrupee} % Include tfrupee package

\setlength{\headheight}{1cm} % Set the height of the header box
\setlength{\headsep}{0mm}     % Set the distance between the header box and the top of the text

\usepackage{gvv-book}
\usepackage{gvv}
\usepackage{cite}
\usepackage{amsmath,amssymb,amsfonts,amsthm}
\usepackage{algorithmic}
\usepackage{graphicx}
\usepackage{textcomp}
\usepackage{xcolor}
\usepackage{txfonts}
\usepackage{listings}
\usepackage{enumitem}
\usepackage{mathtools}
\usepackage{gensymb}
\usepackage{comment}
\usepackage[breaklinks=true]{hyperref}
\usepackage{tkz-euclide} 
\usepackage{listings}
% \usepackage{gvv}                                        
\def\inputGnumericTable{}                                 
\usepackage[latin1]{inputenc}                                
\usepackage{color}                                            
\usepackage{array}                                            
\usepackage{longtable}                                       
\usepackage{calc}                                             
\usepackage{multirow}                                         
\usepackage{hhline}                                           
\usepackage{ifthen}                                           
\usepackage{lscape}
\begin{document}

\bibliographystyle{IEEEtran}
\vspace{3cm}

\title{1.6.8}
\author{EE25BTECH11058 - Tangellapalli Mohana Krishna Sushma
}
% \maketitle
% \newpage
% \bigskip
{\let\newpage\relax\maketitle}

\renewcommand{\thefigure}{\theenumi}
\renewcommand{\thetable}{\theenumi}
\setlength{\intextsep}{10pt} % Space between text and floats


\numberwithin{equation}{enumi}
\numberwithin{figure}{enumi}
\renewcommand{\thetable}{\theenumi}


\textbf{Question}:\\
\text {If three points } (x, -1),\ (2, 1)\ \text{and}\ (4, 5)\ \text{are collinear, find the value of}\ x.
\\
\textbf{Solution: }
\begin{table}[h!]    
  \centering
  \begin{tabular}[12pt]{ |c| c|}
    \hline
    \textbf{Name} & \textbf{Point}\\ 
    \hline
	Point A &\myvec{h \\ k}\\
    \hline 
 Point B &\myvec{x1 \\ y1}\\
    \hline
	  Point R &\myvec{x2 \\ y2}\\
    \hline
    
    \end{tabular}

  \caption{Variables Used}
  \label{tab10.5.3.9.1}
\end{table}

\[
\begin{align}
\Rightarrow
\begin{pmatrix}
a - b & a - c \\
\end{pmatrix}
\]
\end{align}
\[
\begin{pmatrix}
x - 2 & x - 4 \\
-1-1 & -1-5
\end{pmatrix}
\Rightarrow
\begin{pmatrix}
x - 2 & x - 4 \\
-2 & -6
\end{pmatrix}
\]

Changing the matrix into row echelon form, 
using row operations,

\[
R_2 \leftrightarrow R_1
\Rightarrow
\begin{pmatrix}
-2 & -6 \\
x - 2 & x - 4
\end{pmatrix}
\]

Again,

\[
R_2 \rightarrow R_2 + \left(\frac{x - 2}{2}\right) R_1
\]

\[
\begin{pmatrix}
-2 & -6 \\
0 & x - 4 + 6 - 3x
\end{pmatrix}
=
\begin{pmatrix}
-2 & -6 \\
0 & -2x + 2
\end{pmatrix}
\]

As the condition for three points to be collinear, Rank of the Matrix should be 1

\[
-2x + 2 = 0
\Rightarrow x = 1
\]
Hence, the value of x  is '1'.

\begin{figure}[H]
    \centering
    \includegraphics[width=0.5\linewidth]{Figs/Fig.png}
    \caption{PLOT}
    \label{fig:placeholder}
\end{figure}

\end{document}  
\end{document}