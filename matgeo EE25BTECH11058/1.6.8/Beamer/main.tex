\documentclass{beamer}
\usepackage[utf8]{inputenc}

\usetheme{Madrid}
\usecolortheme{default}
\usepackage{amsmath,amssymb,amsfonts,amsthm}
\usepackage{txfonts}
\usepackage{tkz-euclide}
\usepackage{listings}
\usepackage{adjustbox}
\usepackage{array}
\usepackage{tabularx}
\usepackage{gvv}
\usepackage{lmodern}
\usepackage{circuitikz}
\usepackage{tikz}
\usepackage{graphicx}

\setbeamertemplate{page number in head/foot}[totalframenumber]

\usepackage{tcolorbox}
\tcbuselibrary{minted,breakable,xparse,skins}



\definecolor{bg}{gray}{0.95}
\DeclareTCBListing{mintedbox}{O{}m!O{}}{%
  breakable=true,
  listing engine=minted,
  listing only,
  minted language=#2,
  minted style=default,
  minted options={%
    linenos,
    gobble=0,
    breaklines=true,
    breakafter=,,
    fontsize=\small,
    numbersep=8pt,
    #1},
  boxsep=0pt,
  left skip=0pt,
  right skip=0pt,
  left=25pt,
  right=0pt,
  top=3pt,
  bottom=3pt,
  arc=5pt,
  leftrule=0pt,
  rightrule=0pt,
  bottomrule=2pt,

  colback=bg,
  colframe=orange!70,
  enhanced,
  overlay={%
    \begin{tcbclipinterior}
    \fill[orange!20!white] (frame.south west) rectangle ([xshift=20pt]frame.north west);
    \end{tcbclipinterior}},
  #3,
}
\lstset{
    language=C,
    basicstyle=\ttfamily\small,
    keywordstyle=\color{blue},
    stringstyle=\color{orange},
    commentstyle=\color{green!60!black},
    numbers=left,
    numberstyle=\tiny\color{gray},
    breaklines=true,
    showstringspaces=false,
}
%------------------------------------------------------------
%This block of code defines the information to appear in the
%Title page
\title %optional
{1.4.19}
\date{August  2025}
%\subtitle{A short story}

\author % (optional)
{Tangellapalli Mohana Krishna Sushma- EE25BTECH11058}



\begin{document}


\frame{\titlepage}
\begin{frame}{Question}
 \text {If three points } (x, -1),\ (2, 1)\ \text{and}\ (4, 5)\ \text{are collinear, find the value of}\ x.
\\

\end{frame}
 
\begin{frame}{given data}
 

\[
\begin{array}{|c|c|c|c|}
\hline
\textbf{Point} & \textbf{x} & \textbf{y} \\
\hline
a & x & -1 \\
b & 2 & 1 \\
c & 4 & 5 \\
\hline
\end{array}
\]
a,b,c are collinear
   
\end{frame}

\begin{frame}{Formula}
collinearity matrix can be expressed as 
  
\begin{align*}
  \brak{a-b\;\;\;a-c}
 \end{align*}
\end{frame}
 


 \begin{frame}{allowframebreaks}
\frametitle{Row reduction}

\[
R_2 \leftrightarrow R_1
\Rightarrow
\begin{pmatrix}
-2 & -6 \\
x - 2 & x - 4
\end{pmatrix}
\]

Again,

\[
R_2 \rightarrow R_2 + \left(\frac{x - 2}{2}\right) R_1
\]
As the condition for three points to be collinear, Rank of the Matrix should be 1

\[
-2x + 2 = 0
\Rightarrow x = 1
\]
Hence, the value of x  is '1'.

\end{frame}

 

\begin{frame}[fragile]
    \frametitle{Python Code}
    \begin{lstlisting}
 import matplotlib.pyplot as plt

# Find x such that the points are collinear
# Area formula = 0 for collinear points:
# | x1(y2-y3) + x2(y3-y1) + x3(y1-y2) | = 0

# Points: (x, -1), (2, 1), (4, 5)
# Substitute:
# x*(-1 - 5) + 2*(5 + 1) + 4*(-1 - 1) = 0
# x*(-6) + 12 + (-8) = 0
# -6x + 4 = 0 --> x = 2/3

x = 2  # Correct value as derived
\end{lstlisting}
\end{frame}

\begin{frame}[fragile]
    \frametitle{Python Code}

    \begin{lstlisting}
 # Prepare points for plotting
points_x = [x, 2, 4]
points_y = [-1, 1, 5]

    \end{lstlisting}
\end{frame}

\begin{frame}[fragile]
    \frametitle{Python Code}

    \begin{lstlisting}
 # Plotting
plt.figure(figsize=(6, 6))
plt.scatter(points_x, points_y, color='red', zorder=5)

# Draw the straight line through all points
plt.plot(points_x, points_y, '--b', label='Collinear points')

# Annotate each point
for px, py in zip(points_x, points_y):
    plt.annotate(f'({px},{py})', (px, py), textcoords="offset points", xytext=(10,5), ha='center')

plt.xlabel('x')
plt.ylabel('y')
plt.title('Graph of Collinear Points')
plt.grid(True)
plt.legend()
plt.xlim(0, 5)
plt.ylim(-2, 6)
plt.show()


\end{lstlisting}
\end{frame}

 



 


\begin{frame}[fragile]
\frametitle{C Code}
\begin{lstlisting}
  #include <stdio.h>

int main() {

    double y1 = -1.0;
    double x2 = 2.0, y2 = 1.0;
    double x3 = 4.0, y3 = 5.0;
    double x1;
    double numerator = y1 * (x2 - x3) - (x2 * y3 - x3 * y2);

    double denominator = y2 - y3;

    x1 = numerator / denominator;

    printf("Using the matrix determinant method for collinear points:\n");
    printf("The value of x is: %.1f\n", x1);

return 0;
}





\end{lstlisting}

\end{frame}


\begin{frame}[fragile]
\frametitle{Python and C Code}

\begin{lstlisting}
 import subprocess

# 1. Compile the C program
subprocess.run(["gcc", "collinear.c", "-o", "collinear"])

# 2. Run the compiled C program
result = subprocess.run(["./collinear"], capture_output=True, text=True)

# 3. Print the output from the C program
print(result.stdout)
\end{lstlisting}

\end{frame}

 


\begin{figure}
    \centering
    \includegraphics[width=0.8\columnwidth]{Fig.png}
    \caption{Plot}
    \label{fig:placeholder}
\end{figure}


\end{document}