\documentclass{beamer}
\usepackage[utf8]{inputenc}

\usetheme{Madrid}
\usecolortheme{default}
\usepackage{amsmath,amssymb,amsfonts,amsthm}
\usepackage{txfonts}
\usepackage{tkz-euclide}
\usepackage{listings}
\usepackage{adjustbox}
\usepackage{array}
\usepackage{tabularx}
\usepackage{gvv}
\usepackage{lmodern}
\usepackage{circuitikz}
\usepackage{tikz}
\usepackage{graphicx}

\setbeamertemplate{page number in head/foot}[totalframenumber]

\usepackage{tcolorbox}
\tcbuselibrary{minted,breakable,xparse,skins}



\definecolor{bg}{gray}{0.95}
\DeclareTCBListing{mintedbox}{O{}m!O{}}{%
  breakable=true,
  listing engine=minted,
  listing only,
  minted language=#2,
  minted style=default,
  minted options={%
    linenos,
    gobble=0,
    breaklines=true,
    breakafter=,,
    fontsize=\small,
    numbersep=8pt,
    #1},
  boxsep=0pt,
  left skip=0pt,
  right skip=0pt,
  left=25pt,
  right=0pt,
  top=3pt,
  bottom=3pt,
  arc=5pt,
  leftrule=0pt,
  rightrule=0pt,
  bottomrule=2pt,
  toprule=2pt,
  colback=bg,
  colframe=orange!70,
  enhanced,
  overlay={%
    \begin{tcbclipinterior}
    \fill[orange!20!white] (frame.south west) rectangle ([xshift=20pt]frame.north west);
    \end{tcbclipinterior}},
  #3,
}
\lstset{
    language=C,
    basicstyle=\ttfamily\small,
    keywordstyle=\color{blue},
    stringstyle=\color{orange},
    commentstyle=\color{green!60!black},
    numbers=left,
    numberstyle=\tiny\color{gray},
    breaklines=true,
    showstringspaces=false,
}
%------------------------------------------------------------
%This block of code defines the information to appear in the
%Title page
\title %optional
{1.7.12}
\date{August 27,2025}
%\subtitle{A short story}

\author % (optional)
{Rathlavath Jeevan-AI25BTECH11026}



\begin{document}


\frame{\titlepage}
\begin{frame}{Question}
\textbf{} Find the value of $k$, if the points $P(5,4)$, $Q(7,k)$ and $R(9,-2)$ are collinear. 

\textit{Hint:} Three points $P(x_1,y_1)$, $Q(x_2,y_2)$, $R(x_3,y_3)$ are collinear if the area of the triangle formed by them is zero.\\
\\
\end{frame}


\begin{frame}{Theoretical Solution}
\textbf{Solution:} \\
\begin{align}
\vec{A} &= \myvec{-3\\-14}, \quad
\vec{B} = \myvec{a\\-5}
\end{align}

\begin{align}
\|\vec{A}-\vec{B}\| &= 9 \\[2mm]
\implies \left\|\myvec{-3\\-14} - \myvec{a\\-5}\right\| &= 9 \\[1mm]
\implies \left\|\myvec{-3-a\\-9}\right\| &= 9 \\[1mm]
\implies (-3-a)^2 + (-9)^2 &= 9^2 \\[1mm]
(a+3)^2 + 81 &= 81 \\[1mm]
(a+3)^2 &= 0 \\[1mm]
a &= -3
\end{align}

\begin{align}
\boxed{\,a = -3\,}
\end{align}

\end{frame}




\begin{frame}[fragile]
    \frametitle{C Code }

    \begin{lstlisting}
#include <stdio.h>

int main() {
    int x1 = 5, y1 = 4;
    int x2 = 7, y2;   // y2 = k
    int x3 = 9, y3 = -2;
    int k;

    // Equation: x1(y2 - y3) + x2(y3 - y1) + x3(y1 - y2) = 0
    // Substituting values
    // 5(k - (-2)) + 7((-2) - 4) + 9(4 - k) = 0
    // Solve manually inside program:

    // Simplified form: -4k + 4 = 0 => k = 1
    k = 1;
      \end{lstlisting}
\end{frame}

    \begin{frame}[fragile]
    \frametitle{C Code }

    \begin{lstlisting}

    printf("The value of k is: %d\n", k);

    return 0;
}

    \end{lstlisting}
\end{frame}





\begin{frame}[fragile]
    \frametitle{Python Code }
    \begin{lstlisting}
    
import numpy as np
import matplotlib.pyplot as plt
from mpl_toolkits.mplot3d import Axes3D

# Points
x1, y1 = 5, 4
x2, y2 = 7, 1  # k = 1 (solution)
x3, y3 = 9, -2

# Create figure
fig = plt.figure()
ax = fig.add_subplot(111, projection='3d')
 \end{lstlisting}

\end{frame}
\begin{frame}[fragile]
    \frametitle{Python Code }
    \begin{lstlisting}
# Plot the points in 3D (z = 0 for 2D points)
ax.scatter([x1, x2, x3], [y1, y2, y3], [0, 0, 0], c='r', s=100, label='Points')

# Draw line through the points
xs = np.array([x1, x2, x3])
ys = np.array([y1, y2, y3])
zs = np.array([0, 0, 0])
ax.plot(xs, ys, zs, label='Collinear Line')

# Labels and title
ax.set_xlabel('X axis')
ax.set_ylabel('Y axis')
ax.set_zlabel('Z axis')
ax.set_title('3D Visualization of Collinear Points')
ax.legend()
\end{lstlisting}

\end{frame}
\begin{frame}[fragile]
    \frametitle{Python Code }
    \begin{lstlisting}
# Save plot as picture
plt.savefig("collinear_points.png", dpi=300)

# Show the plot
plt.show()

print("Graph saved as collinear_points.png")
     \end{lstlisting}

\end{frame}

\begin{frame}{Plot}
    \centering
    \includegraphics[width=\columnwidth, height=0.8\textheight, keepaspectratio]{beamer2/figs/assignment2.jpeg}     
\end{frame}




\end{document}