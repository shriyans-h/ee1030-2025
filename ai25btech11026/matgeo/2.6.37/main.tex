 \let\negmedspace\undefined
\let\negthickspace\undefined
\documentclass[journal]{IEEEtran}
\usepackage[a5paper, margin=10mm, onecolumn]{geometry}
\usepackage{lmodern} % Ensure lmodern is loaded for pdflatex
\usepackage{tfrupee} % Include tfrupee package

\setlength{\headheight}{1cm} % Set the height of the header box
\setlength{\headsep}{0mm}     % Set the distance between the header box and the top of the text

\usepackage{gvv-book}
\usepackage{gvv}
\usepackage{cite}
\usepackage{amsmath,amssymb,amsfonts,amsthm}
\usepackage{algorithmic}
\usepackage{graphicx}
\usepackage{textcomp}
\usepackage{xcolor}
\usepackage{txfonts}
\usepackage{listings}
\usepackage{enumitem}
\usepackage{mathtools}
\usepackage{gensymb}
\usepackage{comment}
\usepackage[breaklinks=true]{hyperref}
\usepackage{tkz-euclide} 
\usepackage{listings}
\usepackage{gvv}                                        
\def\inputGnumericTable{}                       
\usepackage[latin1]{inputenc}                                
\usepackage{color}                                            
\usepackage{array}                                            
\usepackage{longtable}                                       
\usepackage{calc}                                             
\usepackage{multirow}                                         
\usepackage{hhline}                                           
\usepackage{ifthen}                                           
\usepackage{lscape}  
\usetikzlibrary{patterns}
\begin{document}
\bibliographystyle{IEEEtran}


\textbf{Question 2.6.37:} \\
The vector from origin to the points $A$ and $B$ are \begin{align}
\mathbf{a} = 2\hat{i} - 3\hat{j} + 2\hat{k} \quad \text{and} 
 \quad \mathbf{b} = 2\hat{i} + 3\hat{j} + \hat{k},
 \end{align}respectively, then the area of $\triangle OAB$ is \underline{\hspace{2cm}}.

\textbf{Solution:}
Given

The area of the triangle $OAB$ is given by
\begin{align}
\text{Area}(OAB) = \frac{1}{2} \|\vec{a} \times \vec{b}\|.
\end{align}
We have
\begin{align}
\mathbf{a} = (2,-3,2), \quad \vec{b} = (2,3,1).
\end{align}

Using the cross product definition,
\begin{align}
\begin{myvec}{
\vec{a} \times \vec{b} =
\begin{myvec}{
a_2 & b_2 \\
a_3 & b_3}
\end{myvec} \\
\\
\begin{myvec}{
a_3 & b_3 \\
a_1 & b_1}
\end{myvec} \\
\\
\begin{myvec}{
a_1 & b_1 \\
a_2 & b_2}
\end{myvec}}
\end{myvec}
\end{align}

Substituting values:
\begin{align}
\vec{a} \times \vec{b} =
\begin{myvec}{
\begin{myvec}{
-3 & 3 \\
2 & 1}
\end{myvec} \\
\\
\begin{myvec}{
2 & 1 \\
2 & 2}
\end{myvec} \\
\\
\begin{myvec}{
2 & 2 \\
-3 & 3}
\end{myvec}}
\end{myvec}
=
\begin{myvec}{
(-3)(1) - (3)(2) \\
(2)(2) - (1)(2) \\
(2)(3) - (2)(-3)}
\end{myvec}.
\end{align}

\begin{align}
\vec{a} \times \vec{b} = (-9,\, 2,\, 12).
\end{align}

Now, its magnitude is
\begin{align}
\|\vec{a} \times \vec{b}\| = \sqrt{(-9)^2 + (2)^2 + (12)^2}
= \sqrt{81 + 4 + 144} = \sqrt{229}.
\end{align}

Therefore, the required area is
\begin{align}
\text{Area}(OAB) = \frac{1}{2}\|\vec{a} \times \vec{b}\|
= \frac{1}{2}\sqrt{229}.
\end{align}

\begin{align}
\boxed{\text{Area}(OAB) = \frac{\sqrt{229}}{2}}
\end{align}
\begin{figure}[h!]
    \centering
    \includegraphics[width=0.5\linewidth]{figs/matg4.jpeg}
    \caption{Caption}
    \label{fig:placeholder}
\end{figure}

\end{document}

