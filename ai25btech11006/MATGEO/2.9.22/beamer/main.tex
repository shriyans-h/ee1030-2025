\documentclass{beamer}
\usepackage[utf8]{inputenc}

\usetheme{Madrid}
\usecolortheme{default}
\usepackage{amsmath,amssymb,amsfonts,amsthm}
\usepackage{txfonts}
\usepackage{tkz-euclide}
\usepackage{listings}
\usepackage{adjustbox}
\usepackage{array}
\usepackage{tabularx}
\usepackage{gvv}
\usepackage{lmodern}
\usepackage{circuitikz}
\usepackage{tikz}
\usepackage{graphicx}

\setbeamertemplate{page number in head/foot}[totalframenumber]

\usepackage{tcolorbox}
\tcbuselibrary{minted,breakable,xparse,skins}



\definecolor{bg}{gray}{0.95}
\DeclareTCBListing{mintedbox}{O{}m!O{}}{%
  breakable=true,
  listing engine=minted,
  listing only,
  minted language=#2,
  minted style=default,
  minted options={%
    linenos,
    gobble=0,
    breaklines=true,
    breakafter=,,
    fontsize=\small,
    numbersep=8pt,
    #1},
  boxsep=0pt,
  left skip=0pt,
  right skip=0pt,
  left=25pt,
  right=0pt,
  top=3pt,
  bottom=3pt,
  arc=5pt,
  leftrule=0pt,
  rightrule=0pt,
  bottomrule=2pt,
  toprule=2pt,
  colback=bg,
  colframe=orange!70,
  enhanced,
  overlay={%
    \begin{tcbclipinterior}
    \fill[orange!20!white] (frame.south west) rectangle ([xshift=20pt]frame.north west);
    \end{tcbclipinterior}},
  #3,
}
\lstset{
    language=C,
    basicstyle=\ttfamily\small,
    keywordstyle=\color{blue},
    stringstyle=\color{orange},
    commentstyle=\color{green!60!black},
    numbers=left,
    numberstyle=\tiny\color{gray},
    breaklines=true,
    showstringspaces=false,
}
%------------------------------------------------------------

\title
{2.9.22}
\date{September 8,2025}
\author 
{AI25BTECH11006 - Nikhila}



\begin{document}


\frame{\titlepage}
\begin{frame}{Question}
 Let $\overrightarrow{a}$,
$\overrightarrow{b}$, and $\overrightarrow{c}$ be three vectors such that $|\overrightarrow{a}|$ = 1, $|\overrightarrow{b}|$ = 2, and $|\overrightarrow{c}|$ = 3. If the
projection of $\overrightarrow{b}$ along $\overrightarrow{a}$ is equal to the projection of $\overrightarrow{c}$ along $\overrightarrow{a}$, and $\overrightarrow{b}$ and $\overrightarrow{c}$ are perpendicular to each other, then find $|3\overrightarrow{a} - 2\overrightarrow{b} + 2\overrightarrow{c}|$.
\end{frame}


\begin{frame}[fragile]
    \frametitle{Theoretical Solution}
Given: 
\begin{equation}
\norm{\vec{a}} = 1, \, \norm{\vec{b}} = 2, \, \norm{\vec{c}} = 3\\
\end{equation}

\begin{equation}
\text{The projection of} \, \vec{b} \, \text{along} \, \vec{a} = \vec{b}^T\dfrac{\vec{a}}{\norm{\vec{a}}^2}\vec{a}    
\end{equation}

\begin{equation}
\text{The projection of} \, \vec{c} \, \text{along} \, \vec{a} = \vec{c}^T\dfrac{\vec{a}}{\norm{\vec{a}}^2}\vec{a}    
\end{equation}


\begin{equation}
\vec{b}^T\dfrac{\vec{a}}{\norm{\vec{a}}}\vec{a} = \vec{c}^T\dfrac{\vec{a}}{\norm{\vec{a}}}\vec{a}\\
\end{equation}

\begin{equation}
\text{Since, } \norm{\vec{a}} = 1
\Rightarrow \quad
\therefore \, \, \vec{b}^T\vec{a} = \vec{c}^T\vec{a}
\end{equation}

Since $\vec{b}$ and $\vec{c}$ are perpendicular: 
\begin{equation}
\vec{b}^T\vec{c} = 0\\
\end{equation}

\end{frame}

\begin{frame}[fragile]
    \frametitle{Theoretical Solution}
\begin{equation}
\text{Let} \, \,\vec{v} = 3\vec{a} - 2\vec{b} + 2\vec{c}
\end{equation}


\begin{equation}
\norm{\vec{v}}^2 = (3\vec{a} - 2\vec{b} + 2\vec{c})^T(3\vec{a} - 2\vec{b} + 2\vec{c})
\end{equation}

\begin{equation}
\norm{\vec{v}}^2 = 9(\vec{a}^T\vec{a}) -6(\vec{a}^T\vec{b}) +6(\vec{a}^T\vec{c}) - 6(\vec{b}^T\vec{a}) + 4(\vec{b}^T\vec{b}) -4(\vec{b}^T\vec{c})+ 6(\vec{c}^T\vec{a}) - 4(\vec{c}^T\vec{b}) + 4(\vec{c}^T\vec{c})\\
\end{equation}


\begin{equation}
    \text{Since} \,  \vec{a}^T\vec{b} = \vec{b}^T\vec{a} \, \text{\&}
\, \vec{a}^T\vec{c} = \vec{c}^T\vec{a}
\end{equation}


\begin{equation}
\norm{\vec{v}}^2 = 9(\vec{a}^T\vec{a}) + 4(\vec{b}^T\vec{b}) + 4(\vec{c}^T\vec{c}) - 12(\vec{a}^T\vec{b}) + 12(\vec{a}^T\vec{c}) - 8(\vec{b}^T\vec{c})
\end{equation}\\
\end{frame}

\begin{frame}[fragile]
\frametitle{Theoretical  Solution}
From Equation 1 \& 6,
\begin{equation}
\vec{a}^T\vec{a} = \norm{\vec{a}}^2 = 1, \,\,
\vec{b}^T\vec{b} = \norm{\vec{b}}^2 = 4, \,\,
\vec{c}^T\vec{c} = \norm{\vec{c}}^2 = 9, \,\,
\vec{b}^T\vec{c} = 0
\end{equation}


\begin{equation}
\norm{\vec{v}}^2 = 9 + 16 + 36
\end{equation}


\begin{align}
    \centering
    \norm{\vec{v}}^2 = 61 
    \quad \Rightarrow \quad
    \norm{\vec{v}} = \sqrt{61}
\end{align}

\begin{align}
\centering
\boxed{\norm{3\vec{a} - 2\vec{b} + 2\vec{c}} = \sqrt{61}}
\end{align}
\end{frame}


% Graphical representation
\begin{frame}
\frametitle{Graphical Representation}
\begin{center}
\includegraphics[width=0.7\linewidth]{fig1.png}
\end{center}
\end{frame}


\end{document}
