\documentclass{beamer}
\usepackage[utf8]{inputenc}

\usetheme{Madrid}
\usecolortheme{default}
\usepackage{amsmath,amssymb,amsfonts,amsthm}
\usepackage{txfonts}
\usepackage{tkz-euclide}
\usepackage{listings}
\usepackage{adjustbox}
\usepackage{array}
\usepackage{tabularx}
\usepackage{gvv}
\usepackage{lmodern}
\usepackage{circuitikz}
\usepackage{tikz}
\usepackage{graphicx}

\setbeamertemplate{page number in head/foot}[totalframenumber]

\usepackage{tcolorbox}
\tcbuselibrary{minted,breakable,xparse,skins}



\definecolor{bg}{gray}{0.95}
\DeclareTCBListing{mintedbox}{O{}m!O{}}{%
	breakable=true,
	listing engine=minted,
	listing only,
	minted language=#2,
	minted style=default,
	minted options={%
		linenos,
		gobble=0,
		breaklines=true,
		breakafter=,,
		fontsize=\small,
		numbersep=8pt,
		#1},
	boxsep=0pt,
	left skip=0pt,
	right skip=0pt,
	left=25pt,
	right=0pt,
	top=3pt,
	bottom=3pt,
	arc=5pt,
	leftrule=0pt,
	rightrule=0pt,
	bottomrule=2pt,
	toprule=2pt,
	colback=bg,
	colframe=orange!70,
	enhanced,
	overlay={%
		\begin{tcbclipinterior}
			\fill[orange!20!white] (frame.south west) rectangle ([xshift=20pt]frame.north west);
	\end{tcbclipinterior}},
	#3,
}
\lstset{
	language=C,
	basicstyle=\ttfamily\small,
	keywordstyle=\color{blue},
	stringstyle=\color{orange},
	commentstyle=\color{green!60!black},
	numbers=left,
	numberstyle=\tiny\color{gray},
	breaklines=true,
	showstringspaces=false,
}
%------------------------------------------------------------
%This block of code defines the information to appear in the
%Title page
\title %optional
{2.8.37}
%\subtitle{A short story}

\author % (optional)
{RAVULA SHASHANK REDDY - EE25BTECH11047}

 \begin{document}
	
	
	\frame{\titlepage}
	\begin{frame}{Question}
If $|\vec{a}\times \vec{b}|^2 + (\vec{a}^T \vec{b})^2 = 144$ and $||\vec{a}|| = 4$, then $||\vec{b}||$ is equal to \underline{\hspace{2cm}}\\
\end{frame}

\begin{frame}{Equation}
\begin{align*}
|\vec{a}\times \vec{b}|^2 + (\vec{a}^T \vec{b})^2
&= ||\vec{a}||^2||\vec{b}||^2.
\end{align*}
    
\end{frame}
\begin{frame}{Theoretical Solution}
We know that
    \begin{align}
|\vec{a}\times \vec{b}|^2 + (\vec{a}^T \vec{b})^2
&= ||\vec{a}||^2||\vec{b}||^2.
\end{align}

Given :
\begin{align}
|\vec{a}\times \vec{b}|^2 + (\vec{a}^T \vec{b})^2 &= 144, \\
||\vec{a}|| &= 4,
\end{align}
\end{frame}
\begin{frame}{Theoretical Solution}
\begin{align}
144 &= ||\vec{a}||^2||\vec{b}||^2 \\
144 &= 4^2||\vec{b}||^2 \\
144 &= 16||\vec{b}||^2 \\
||\vec{b}||^2 &= \frac{144}{16} = 9 \\
||\vec{b}|| &= 3.
\end{align}

\end{frame}
\begin{frame}[fragile]
     \frametitle{C Code}
      \begin{lstlisting}
          #include <stdio.h>
#include <math.h>

int main() {
    // Given values
    double a = 4.0;   // |a| = 4
    double lhs = 144; // |a × b|^2 + (a · b)^2 = 144

    // From identity: |a × b|^2 + (a · b)^2 = |a|^2 * |b|^2
    // => |b|^2 = lhs / (|a|^2)
    double b_squared = lhs / (a * a);
    double b = sqrt(b_squared);

    printf("|b| = %.2f\n", b);

    return 0;
}
      \end{lstlisting}
\end{frame}

\begin{frame}[fragile]
     \frametitle{Python Shared Output}
      \begin{lstlisting}
      import numpy as np
import ctypes
# Local imports
from libs.line.funcs import *
from libs.triangle.funcs import *
from libs.conics.funcs import circ_gen

# Load C math library for sqrt
libm = ctypes.CDLL("libm.so.6")
libm.sqrt.restype = ctypes.c_double
libm.sqrt.argtypes = [ctypes.c_double]
  \end{lstlisting}
\end{frame}

\begin{frame}[fragile]
     \frametitle{Python Shared Output}
      \begin{lstlisting}
# Given values
a = 4.0        # |a|
lhs = 144.0    # |a × b|^2 + (a · b)^2

# Compute b^2 using NumPy
b_squared = lhs / np.power(a, 2)

# Compute b using C's sqrt via ctypes
b = libm.sqrt(ctypes.c_double(b_squared))

print(f"|b| = {b}")
\end{lstlisting}
\end{frame}

\begin{frame}[fragile]
     \frametitle{Python Direct}
      \begin{lstlisting}

import numpy as np

# Given values
a = 4.0        # |a|
lhs = 144.0    # |a × b|^2 + (a · b)^2

# Using numpy
b_squared = lhs / np.power(a, 2)
b = np.sqrt(b_squared)

print(f"|b| = {b}")
\end{lstlisting}
\end{frame}
\end{document}