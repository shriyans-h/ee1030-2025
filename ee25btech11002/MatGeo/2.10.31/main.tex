\let\negmedspace\undefined
\let\negthickspace\undefined
\documentclass[journal]{IEEEtran}
\usepackage[a5paper, margin=10mm, onecolumn]{geometry}
%\usepackage{lmodern} % Ensure lmodern is loaded for pdflatex
\usepackage{tfrupee} % Include tfrupee package

\setlength{\headheight}{1cm} % Set the height of the header box
\setlength{\headsep}{0mm}     % Set the distance between the header box and the top of the text

\usepackage{gvv-book}
\usepackage{gvv}
\usepackage{cite}
\usepackage{amsmath,amssymb,amsfonts,amsthm}
\usepackage{algorithmic}
\usepackage{graphicx}
\usepackage{textcomp}
\usepackage{xcolor}
\usepackage{txfonts}
\usepackage{listings}
\usepackage{enumitem}
\usepackage{mathtools}
\usepackage{gensymb}
\usepackage{comment}
\usepackage[breaklinks=true]{hyperref}
\usepackage{tkz-euclide} 
\usepackage{listings}
% \usepackage{gvv}                                        
\def\inputGnumericTable{}                                 
\usepackage[latin1]{inputenc}                                
\usepackage{color}                                            
\usepackage{array}                                            
\usepackage{longtable}                                       
\usepackage{calc}                                             
\usepackage{multirow}                                         
\usepackage{hhline}                                           
\usepackage{ifthen}                                           
\usepackage{lscape}
\begin{document}
\bibliographystyle{IEEEtran}
\title{2.10.31}
\author{EE25BTECH11002 - Achat Parth Kalpesh }
{\let\newpage\relax\maketitle}
\renewcommand{\thefigure}{\theenumi}
\renewcommand{\thetable}{\theenumi}
\setlength{\intextsep}{10pt} % Space between text and floats
\numberwithin{equation}{enumi}
\numberwithin{figure}{enumi}
\renewcommand{\thetable}{\theenumi}
\parindent 0px


\textbf{Question:}\\
Let $\vec{a}$, $\vec{b}$, $\vec{c}$ be three non coplanar vectors and 
$\vec{p}, \vec{q},\vec{r}$ are vectors defined by the relations 
\begin{align}
\vec{p}=\frac{\vec{b}\times\vec{c}}{\sbrak{\vec{a}\ \vec{b}\ \vec{c}}}, \vec{q}=\frac{\vec{c}\times\vec{a}}{\sbrak{\vec{a}\ \vec{b}\ \vec{c}}},\vec{r}=\frac{\vec{a}\times\vec{b}}{\sbrak{\vec{a}\ \vec{b}\ \vec{c}}}
\end{align}
then the value of the expression $\brak{\vec{a}+\vec{b}}\cdot\vec{p}+\brak{\vec{b}+\vec{c}}\cdot\vec{q}+\brak{\vec{c}+\vec{a}}\cdot\vec{r}$ is equal to
  \begin{multicols}{4}
\begin{enumerate}
\item $0$
\item $1$
\item $2$
\item $3$
\end{enumerate}
  \end{multicols}
\textbf{Solution:}\\
Let the given expression be:
\begin{align}
    E &= \brak{\vec{a} + \vec{b}} \cdot \vec{p} + \brak{\vec{b} + \vec{c}} \cdot \vec{q} + \brak{\vec{c} + \vec{a}} \cdot \vec{r}\\
    &= \brak{\vec{a} \cdot \vec{p} + \vec{b} \cdot \vec{p}} + \brak{\vec{b} \cdot \vec{q} + \vec{c} \cdot \vec{q}} + \brak{\vec{c} \cdot \vec{r} + \vec{a} \cdot \vec{r}}
\end{align}
Let ,
\begin{align}
    \vec{V} &= \myvec{\vec{a}&\vec{b}&\vec{c}}\\
    \vec{P} &= \myvec{\vec{p}&\vec{q}&\vec{r}}
\end{align}
\begin{align}
    \vec{V}^\top\vec{P} &= \myvec{\vec{a}\\ \vec{b}\\ \vec{c}} \myvec{\vec{p}&\vec{q}&\vec{r}} 
    = \myvec{\vec{a} \cdot \vec{p} & \vec{a} \cdot \vec{q} & \vec{a} \cdot \vec{r} \\
        \vec{b} \cdot \vec{p} & \vec{b} \cdot \vec{q} & \vec{b} \cdot \vec{r} \\
        \vec{c} \cdot \vec{p} & \vec{c} \cdot \vec{q} & \vec{c} \cdot \vec{r}}
\end{align}

\begin{align}
    \vec{a} \cdot \vec{p} = \vec{a} \cdot \frac{\vec{b} \times \vec{c}}{\sbrak{\vec{a}\ \vec{b}\ \vec{c}}} = \frac{\sbrak{\vec{a}\ \vec{b}\ \vec{c}}}{\sbrak{\vec{a}\ \vec{b}\ \vec{c}}} = 1
\end{align}

\begin{align}
    \vec{a} \cdot \vec{q} = \vec{a} \cdot \frac{\vec{c} \times \vec{a}}{\sbrak{\vec{a}\ \vec{b}\ \vec{c}}} = \frac{\sbrak{\vec{a}\ \vec{c}\ \vec{a}}}{\sbrak{\vec{a}\ \vec{b}\ \vec{c}}} = 0
\end{align}
since the scalar triple product with a repeated vector is zero.
Thus, the matrix product becomes the identity matrix:
\begin{align}
    \myvec{\vec{a} \cdot \vec{p} & \vec{a} \cdot \vec{q} & \vec{a} \cdot \vec{r} \\
        \vec{b} \cdot \vec{p} & \vec{b} \cdot \vec{q} & \vec{b} \cdot \vec{r} \\
        \vec{c} \cdot \vec{p} & \vec{c} \cdot \vec{q} & \vec{c} \cdot \vec{r}}
    =\myvec{1 & 0 & 0 \\
        0 & 1 & 0 \\
        0 & 0 & 1} = \vec{I}
\end{align}
Substituting these results back into the expanded expression for $E$:
\begin{align}
    E &= \brak{1 + 0} + \brak{1 + 0} + \brak{1 + 0} \\
      &= 1 + 1 + 1 \\
      &= 3
\end{align}
The value of the expression is 3.

\end{document}