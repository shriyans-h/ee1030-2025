\documentclass{beamer}
\usepackage[utf8]{inputenc}

\usetheme{Madrid}
\usecolortheme{default}
\usepackage{amsmath,amssymb,amsfonts,amsthm}
\usepackage{txfonts}
\usepackage{tkz-euclide}
\usepackage{listings}
\usepackage{adjustbox}
\usepackage{array}
\usepackage{tabularx}
\usepackage{gvv}
\usepackage{lmodern}
\usepackage{circuitikz}
\usepackage{lmodern}
\usepackage{multicol}
\usepackage{tikz}
\usepackage{graphicx}

\setbeamertemplate{page number in head/foot}[totalframumber]

\usepackage{tcolorbox}
\tcbuselibrary{minted,breakable,xparse,skins}



\definecolor{bg}{gray}{0.95}
\DeclareTCBListing{mintedbox}{O{}m!O{}}{%
  breakable=true,
  listing engine=minted,
  listing only,
  minted language=#2,
  minted style=default,
  minted options={%
    linenos,
    gobble=0,
    breaklines=true,
    breakafter=,,
    fontsize=\small,
    numbersep=8pt,
    #1},
  boxsep=0pt,
  left skip=0pt,
  right skip=0pt,
  left=25pt,
  right=0pt,
  top=3pt,
  bottom=3pt,
  arc=5pt,
  leftrule=0pt,
  rightrule=0pt,
  bottomrule=2pt,
  toprule=2pt,
  colback=bg,
  colframe=orange!70,
  enhanced,
  overlay={%
    \begin{tcbclipinterior}
    \fill[orange!20!white] (frame.south west) rectangle ([xshift=20pt]frame.north west);
    \end{tcbclipinterior}},
  #3,
}
\lstset{
    language=C,
    basicstyle=\ttfamily\small,
    keywordstyle=\color{blue},
    stringstyle=\color{orange},
    commentstyle=\color{green!60!black},
    numbers=left,
    numberstyle=\tiny\color{gray},
    breaklines=true,
    showstringspaces=false,
}
%------------------------------------------------------------
%This block of code defines the information to appear in the
%Title page
\title %optional
{2.10.31}
\date{September 19,2025}
%\subtitle{A short story}

\author % (optional)
{EE25BTECH11002 - Achat Parth Kalpesh}



\begin{document}


\frame{\titlepage}

\begin{frame}{Question}
Let $\vec{a}$, $\vec{b}$, $\vec{c}$ be three non coplanar vectors and 
$\vec{p}, \vec{q},\vec{r}$ are vectors defined by the relations 
\begin{align}
\vec{p}=\frac{\vec{b}\times\vec{c}}{\sbrak{\vec{a}\ \vec{b}\ \vec{c}}}, \vec{q}=\frac{\vec{c}\times\vec{a}}{\sbrak{\vec{a}\ \vec{b}\ \vec{c}}},\vec{r}=\frac{\vec{a}\times\vec{b}}{\sbrak{\vec{a}\ \vec{b}\ \vec{c}}}
\end{align}
then the value of the expression $\brak{\vec{a}+\vec{b}}\cdot\vec{p}+\brak{\vec{b}+\vec{c}}\cdot\vec{q}+\brak{\vec{c}+\vec{a}}\cdot\vec{r}$ is equal to
  \begin{multicols}{2}
\begin{enumerate}
\item $0$
\item $1$
\item $2$
\item $3$
\end{enumerate}
\end{multicols}
\end{frame}
\begin{frame}{Theoretical Solution}
Let the given expression be:
\begin{align}
    E &= \brak{\vec{a} + \vec{b}} \cdot \vec{p} + \brak{\vec{b} + \vec{c}} \cdot \vec{q} + \brak{\vec{c} + \vec{a}} \cdot \vec{r}\\
    &= \brak{\vec{a} \cdot \vec{p} + \vec{b} \cdot \vec{p}} + \brak{\vec{b} \cdot \vec{q} + \vec{c} \cdot \vec{q}} + \brak{\vec{c} \cdot \vec{r} + \vec{a} \cdot \vec{r}}
\end{align}
Let ,
\begin{align}
    \vec{V} &= \myvec{\vec{a}&\vec{b}&\vec{c}}\\
    \vec{P} &= \myvec{\vec{p}&\vec{q}&\vec{r}}
\end{align}
\begin{align}
    \vec{V}^\top\vec{P} &= \myvec{\vec{a}\\ \vec{b}\\ \vec{c}} \myvec{\vec{p}&\vec{q}&\vec{r}} 
    = \myvec{\vec{a} \cdot \vec{p} & \vec{a} \cdot \vec{q} & \vec{a} \cdot \vec{r} \\
        \vec{b} \cdot \vec{p} & \vec{b} \cdot \vec{q} & \vec{b} \cdot \vec{r} \\
        \vec{c} \cdot \vec{p} & \vec{c} \cdot \vec{q} & \vec{c} \cdot \vec{r}}
\end{align}

\end{frame}
\begin{frame}{Theoretical Solution}

\begin{align}
    \vec{a} \cdot \vec{p} = \vec{a} \cdot \frac{\vec{b} \times \vec{c}}{\sbrak{\vec{a}\ \vec{b}\ \vec{c}}} = \frac{\sbrak{\vec{a}\ \vec{b}\ \vec{c}}}{\sbrak{\vec{a}\ \vec{b}\ \vec{c}}} = 1
\end{align}
\begin{align}
    \vec{a} \cdot \vec{q} = \vec{a} \cdot \frac{\vec{c} \times \vec{a}}{\sbrak{\vec{a}\ \vec{b}\ \vec{c}}} = \frac{\sbrak{\vec{a}\ \vec{c}\ \vec{a}}}{\sbrak{\vec{a}\ \vec{b}\ \vec{c}}} = 0
\end{align}
\end{frame}
\begin{frame}{Theoretical Solution}
since the scalar triple product with a repeated vector is zero.
Thus, the matrix product becomes the identity matrix:
\begin{align}
    \myvec{\vec{a} \cdot \vec{p} & \vec{a} \cdot \vec{q} & \vec{a} \cdot \vec{r} \\
        \vec{b} \cdot \vec{p} & \vec{b} \cdot \vec{q} & \vec{b} \cdot \vec{r} \\
        \vec{c} \cdot \vec{p} & \vec{c} \cdot \vec{q} & \vec{c} \cdot \vec{r}}
    =\myvec{1 & 0 & 0 \\
        0 & 1 & 0 \\
        0 & 0 & 1} = \vec{I}
\end{align}
Substituting these results back into the expanded expression for $E$:
\begin{align}
    E &= \brak{1 + 0} + \brak{1 + 0} + \brak{1 + 0} \\
      &= 1 + 1 + 1 \\
      &= 3
\end{align}
The value of the expression is 3.

\end{frame}

\end{document}