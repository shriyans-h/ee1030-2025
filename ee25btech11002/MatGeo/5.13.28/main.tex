\let\negmedspace\undefined
\let\negthickspace\undefined
\documentclass[journal]{IEEEtran}
\usepackage[a5paper, margin=10mm, onecolumn]{geometry}
%\usepackage{lmodern} % Ensure lmodern is loaded for pdflatex
\usepackage{tfrupee} % Include tfrupee package

\setlength{\headheight}{1cm} % Set the height of the header box
\setlength{\headsep}{0mm}     % Set the distance between the header box and the top of the text

\usepackage{gvv-book}
\usepackage{gvv}
\usepackage{cite}
\usepackage{amsmath,amssymb,amsfonts,amsthm}
\usepackage{algorithmic}
\usepackage{graphicx}
\usepackage{textcomp}
\usepackage{xcolor}
\usepackage{txfonts}
\usepackage{listings}
\usepackage{enumitem}
\usepackage{mathtools}
\usepackage{gensymb}
\usepackage{comment}
\usepackage[breaklinks=true]{hyperref}
\usepackage{tkz-euclide} 
\usepackage{listings}
% \usepackage{gvv}                                        
\def\inputGnumericTable{}                                 
\usepackage[latin1]{inputenc}                                
\usepackage{color}                                            
\usepackage{array}                                            
\usepackage{longtable}                                       
\usepackage{calc}                                             
\usepackage{multirow}                                         
\usepackage{hhline}                                           
\usepackage{ifthen}                                           
\usepackage{lscape}
\begin{document}
\bibliographystyle{IEEEtran}
\title{5.13.28}
\author{EE25BTECH11002 - Achat Parth Kalpesh }
{\let\newpage\relax\maketitle}
\renewcommand{\thefigure}{\theenumi}
\renewcommand{\thetable}{\theenumi}
\setlength{\intextsep}{10pt} % Space between text and floats
\numberwithin{equation}{enumi}
\numberwithin{figure}{enumi}
\renewcommand{\thetable}{\theenumi}
\parindent 0px


\textbf{Question:}\\
 Let $\vec{A}$ be a $2 \times 2$ matrix with real entries. Let 
 $\vec{I}$ be the $2 \times 2$ identity matrix. Denote by 
 tr\brak{\vec{A}},the sum of diagonal entries of $\vec{A}$. Assume
 that $\vec{A}^2 = \vec{I}$. \\
		Statement-1 : If $\vec{A} \neq \vec{I}$ and $\vec{A} \neq -
        \vec{I}$,then det\brak{\vec{A}}=-1 \\
		Statement-2 : If $\vec{A} \neq \vec{I}$ and $\vec{A} \neq -
        \vec{I}$,then $tr\brak{\vec{A}} \neq 0$.
		\begin{enumerate}
			\item Statement-1 is false, Statement-2 is true
			\item Statement-1 is true, Statement-2 is 
            true;Statement-2 is a correct explanation for 
            Statement-1
			\item Statement-1 is true, Statement-2 is 
            true;Statement-2 is not a correct explanation for 
            Statement-1
			\item Statement-1 is true, Statement-2 is false
		\end{enumerate}

\textbf{Solution:}\\
Let the eigenvalue of  $\vec{A}$  be  $\lambda$.
\begin{align}
    \vec{A}\vec{v} &= \lambda \vec{v} \\
    \vec{A}^2 \vec{v} &= \lambda^2 \vec{v}\\
    \vec{A}^2 &= \vec{I}\\
    \lambda^2 &= 1 \\
    \lambda &= \pm 1
\end{align}
Thus, eigenvalues $\lambda_1$ , $\lambda_2$ of  $\vec{A}$  are chosen from  \cbrak{1,-1}\\
As it is given as $\vec{A} \neq \vec{I}$ and $\vec{A} \neq -\vec{I}$ , so the possible case is
\begin{align}
    \lambda_1 = 1,\lambda_2 = -1
\end{align}
Thereby,
\begin{align}
    det\brak{\vec{A}} &= \lambda_1 \lambda_2\\
    &= -1 \\
    tr\brak{\vec{A}} &= \lambda_1 + \lambda_2 \\
    &= 0
\end{align}
Thus Statement-1 is true, Statement-2 is false
\end{document}