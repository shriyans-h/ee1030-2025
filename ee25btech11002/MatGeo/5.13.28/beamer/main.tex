\documentclass{beamer}
\usepackage[utf8]{inputenc}

\usetheme{Madrid}
\usecolortheme{default}
\usepackage{amsmath,amssymb,amsfonts,amsthm}
\usepackage{txfonts}
\usepackage{tkz-euclide}
\usepackage{listings}
\usepackage{adjustbox}
\usepackage{array}
\usepackage{tabularx}
\usepackage{gvv}
\usepackage{lmodern}
\usepackage{circuitikz}
\usepackage{lmodern}
\usepackage{multicol}
\usepackage{tikz}
\usepackage{graphicx}

\setbeamertemplate{page number in head/foot}[totalframumber]

\usepackage{tcolorbox}
\tcbuselibrary{minted,breakable,xparse,skins}



\definecolor{bg}{gray}{0.95}
\DeclareTCBListing{mintedbox}{O{}m!O{}}{%
  breakable=true,
  listing engine=minted,
  listing only,
  minted language=#2,
  minted style=default,
  minted options={%
    linenos,
    gobble=0,
    breaklines=true,
    breakafter=,,
    fontsize=\small,
    numbersep=8pt,
    #1},
  boxsep=0pt,
  left skip=0pt,
  right skip=0pt,
  left=25pt,
  right=0pt,
  top=3pt,
  bottom=3pt,
  arc=5pt,
  leftrule=0pt,
  rightrule=0pt,
  bottomrule=2pt,
  toprule=2pt,
  colback=bg,
  colframe=orange!70,
  enhanced,
  overlay={%
    \begin{tcbclipinterior}
    \fill[orange!20!white] (frame.south west) rectangle ([xshift=20pt]frame.north west);
    \end{tcbclipinterior}},
  #3,
}
\lstset{
    language=C,
    basicstyle=\ttfamily\small,
    keywordstyle=\color{blue},
    stringstyle=\color{orange},
    commentstyle=\color{green!60!black},
    numbers=left,
    numberstyle=\tiny\color{gray},
    breaklines=true,
    showstringspaces=false,
}
%------------------------------------------------------------
%This block of code defines the information to appear in the
%Title page
\title %optional
{5.13.28}
\date{October 2,2025}
%\subtitle{A short story}

\author % (optional)
{EE25BTECH11002 - Achat Parth Kalpesh}



\begin{document}


\frame{\titlepage}

\begin{frame}{Question}
Let $\vec{A}$ be a $2 \times 2$ matrix with real entries. Let 
 $\vec{I}$ be the $2 \times 2$ identity matrix. Denote by 
 tr\brak{\vec{A}},the sum of diagonal entries of $\vec{A}$. Assume
 that $\vec{A}^2 = \vec{I}$. \\
		Statement-1 : If $\vec{A} \neq \vec{I}$ and $\vec{A} \neq -
        \vec{I}$,then det\brak{\vec{A}}=-1 \\
		Statement-2 : If $\vec{A} \neq \vec{I}$ and $\vec{A} \neq -
        \vec{I}$,then $tr\brak{\vec{A}} \neq 0$.
		\begin{enumerate}
			\item Statement-1 is false, Statement-2 is true
			\item Statement-1 is true, Statement-2 is 
            true;Statement-2 is a correct explanation for 
            Statement-1
			\item Statement-1 is true, Statement-2 is 
            true;Statement-2 is not a correct explanation for 
            Statement-1
			\item Statement-1 is true, Statement-2 is false
		\end{enumerate}
\end{frame}
\begin{frame}{Solution}
Let the eigenvalue of  $\vec{A}$  be  $\lambda$.
\begin{align}
    \vec{A}\vec{v} &= \lambda \vec{v} \\
    \vec{A}^2 \vec{v} &= \lambda^2 \vec{v}\\
    \vec{A}^2 &= \vec{I}\\
    \lambda^2 &= 1 \\
    \lambda &= \pm 1
\end{align}
Thus, eigenvalues $\lambda_1$ , $\lambda_2$ of  $\vec{A}$  are chosen from  \cbrak{1,-1}\\
As it is given as $\vec{A} \neq \vec{I}$ and $\vec{A} \neq -\vec{I}$ , so the possible case is
\begin{align}
    \lambda_1 = 1,\lambda_2 = -1
\end{align}
\end{frame}
\begin{frame}{Solution}
Thereby,
\begin{align}
    det\brak{\vec{A}} &= \lambda_1 \lambda_2\\
    &= -1 \\
    tr\brak{\vec{A}} &= \lambda_1 + \lambda_2 \\
    &= 0
\end{align}
Thus Statement-1 is true, Statement-2 is false
\end{frame}

\end{document}