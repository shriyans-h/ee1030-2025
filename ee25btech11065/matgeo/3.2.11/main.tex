\let\negmedspace\undefined
\let\negthickspace\undefined
\documentclass[journal]{IEEEtran}
\usepackage[a5paper, margin=10mm, onecolumn]{geometry}
%\usepackage{lmodern} % Ensure lmodern is loaded for pdflatex
\usepackage{tfrupee} % Include tfrupee package

\setlength{\headheight}{1cm} % Set the height of the header box
\setlength{\headsep}{0mm}     % Set the distance between the header box and the top of the text
\usepackage{gvv-book}
\usepackage{gvv}
\usepackage{cite}
\usepackage{amsmath,amssymb,amsfonts,amsthm}
\usepackage{algorithmic}
\usepackage{graphicx}
\usepackage{textcomp}
\usepackage{xcolor}
\usepackage{txfonts}
\usepackage{listings}
\usepackage{enumitem}
\usepackage{mathtools}
\usepackage{gensymb}
\usepackage{comment}
\usepackage[breaklinks=true]{hyperref}
\usepackage{tkz-euclide} 
\usepackage{listings}
% \usepackage{gvv}                                        
\def\inputGnumericTable{}                                 
\usepackage[latin1]{inputenc}                                
\usepackage{color}                                            
\usepackage{array}                                            
\usepackage{longtable}                                       
\usepackage{calc}                                             
\usepackage{multirow}                                         
\usepackage{hhline}                                           
\usepackage{ifthen}                                           
\usepackage{lscape}
\begin{document}

\bibliographystyle{IEEEtran}
\vspace{3cm}

\title{3.2.11}
\author{EE25BTECH11065 - Yoshita.J}
% \maketitle
% \newpage
% \bigskip
{\let\newpage\relax\maketitle}

\renewcommand{\thefigure}{\theenumi}
\renewcommand{\thetable}{\theenumi}
\setlength{\intextsep}{10pt} % Space between text and floats


\numberwithin{equation}{enumi}
\numberwithin{figure}{enumi}
\renewcommand{\thetable}{\theenumi}


\textbf{Question}:\\
Draw an Right angle  triangle $\triangle ABC$ in which $\boldsymbol{BC} = 12 \text{ cm}$, $\boldsymbol{AB} = 5 \text{ cm}$, and $\angle B = 90^\circ$.
\\ \solution \\
    \begin{table}[h!]    
      \centering
      \begin{tabular}[12pt]{ |c| c|}
    \hline
    \textbf{Name} & \textbf{Point}\\ 
    \hline
	Point A &\myvec{h \\ k}\\
    \hline 
 Point B &\myvec{x1 \\ y1}\\
    \hline
	  Point R &\myvec{x2 \\ y2}\\
    \hline
    
    \end{tabular}

      \caption{}
    \end{table}
   \begin{align}
      AB^2 & = 5^2 = 25, \\
      BC^2 & = 12^2 = 144.
   \end{align}
According to the Pythagorean theorem:
   \begin{align}
      AC^2 & = AB^2 + BC^2 
   \end{align}
Now substituting in the values:

   \begin{align}
      AC^2 & = 25 + 144 \\
      & = 169.
   \end{align}

Thus, we find:

   \begin{align}
      AC & = \sqrt{169} \\
      & = 13 \, \text{cm}.
   \end{align}
    \begin{figure}[h]
       \centering
       \includegraphics[width=0.9\columnwidth]{figs/fig1.png}
       \caption{}
       \label{graph}
    \end{figure}
\end{document}  

