\let\negmedspace\undefined
\let\negthickspace\undefined
\documentclass[journal]{IEEEtran}
\usepackage[a5paper, margin=10mm, onecolumn]{geometry}
\usepackage{tfrupee}

\setlength{\headheight}{1cm}
\setlength{\headsep}{0mm}

\usepackage{gvv-book}
\usepackage{gvv}
\usepackage{cite}
\usepackage{amsmath,amssymb,amsfonts,amsthm}
\usepackage{algorithmic}
\usepackage{graphicx}
\usepackage{textcomp}
\usepackage{xcolor}
\usepackage{txfonts}
\usepackage{listings}
\usepackage{enumitem}
\usepackage{mathtools}
\usepackage{gensymb}
\usepackage{comment}
\usepackage[breaklinks=true]{hyperref}
\usepackage{tkz-euclide}
\usepackage{listings}

\def\inputGnumericTable{}
\usepackage[latin1]{inputenc}
\usepackage{color}
\usepackage{array}
\usepackage{longtable}
\usepackage{calc}
\usepackage{multirow}
\usepackage{hhline}
\usepackage{ifthen}
\usepackage{lscape}

\begin{document}

\bibliographystyle{IEEEtran}
\vspace{3cm}

\title{5.10.3}
\author{EE25BTECH11065 - Yoshita J}
{\let\newpage\relax\maketitle}

\renewcommand{\thefigure}{\theenumi}
\renewcommand{\thetable}{\theenumi}
\setlength{\intextsep}{10pt}

\textbf{Question}:\\
Balance the following chemical equation.
\begin{align*}
    Fe + H_2O \to Fe_3O_4 + H_2
\end{align*}
\bigskip

\textbf{Solution:} \\
Let the balanced version of the equation be
\begin{align}
    x_1 Fe + x_2 H_2O \to x_3 Fe_3O_4 + x_4 H_2
\end{align}
which results in the following equations based on the conservation of each element:
\begin{align}
    \text{For Fe: } &x_1 - 3x_3 = 0 \\
    \text{For H: } &2x_2 - 2x_4 = 0 \implies x_2 - x_4 = 0 \\
    \text{For O: } &x_2 - 4x_3 = 0
\end{align}
This can be expressed as a homogeneous system of linear equations:
\begin{align}
    x_1 + 0x_2 - 3x_3 + 0x_4 &= 0 \\
    0x_1 + x_2 + 0x_3 - x_4 &= 0 \\
    0x_1 + x_2 - 4x_3 + 0x_4 &= 0
\end{align}
This results in the matrix equation $A\mathbf{x} = \mathbf{0}$, where:
\begin{align}
    \myvec{
        1 & 0 & -3 & 0 \\
        0 & 1 & 0 & -1 \\
        0 & 1 & -4 & 0
    } \mathbf{x} = \mathbf{0},
    \quad \mathbf{x} = \myvec{ x_1 \\ x_2 \\ x_3 \\ x_4 }
\end{align}
The coefficient matrix can be reduced as follows using Gaussian elimination to find the null space:
\begin{align*}
\myvec{
1 & 0 & -3 & 0 \\
0 & 1 & 0 & -1 \\
0 & 1 & -4 & 0
}
\xrightarrow{R_3 \to R_3 - R_2}
\myvec{
1 & 0 & -3 & 0 \\
0 & 1 & 0 & -1 \\
0 & 0 & -4 & 1
}
\xrightarrow{R_3 \to -\frac{1}{4}R_3}
\myvec{
1 & 0 & -3 & 0 \\
0 & 1 & 0 & -1 \\
0 & 0 & 1 & -1/4
}
\end{align*}
\begin{align*}
\xrightarrow{R_1 \to R_1 + 3R_3}
\myvec{
1 & 0 & 0 & -3/4 \\
0 & 1 & 0 & -1 \\
0 & 0 & 1 & -1/4
}
\end{align*}
From the reduced row echelon form, we get the solutions in terms of the free variable $x_4$:
\begin{align}
    x_1 = \frac{3}{4}x_4, \quad x_2 = x_4, \quad x_3 = \frac{1}{4}x_4
\end{align}
Thus,
\begin{align}
    \mathbf{x} = x_4 \myvec{ 3/4 \\ 1 \\ 1/4 \\ 1 }
\end{align}
By substituting $x_4 = 4$, the simplest integer solution is found. Hence,
\begin{align}
    \mathbf{x} = 4 \myvec{ 3/4 \\ 1 \\ 1/4 \\ 1 } = \myvec{ 3 \\ 4 \\ 1 \\ 4 }
\end{align}
This gives $x_1 = 3, x_2 = 4, x_3 = 1, \text{ and } x_4 = 4$.
\bigskip
Hence, the balanced equation finally becomes:
\begin{align}
    \boxed{3Fe + 4H_2O \to Fe_3O_4 + 4H_2}
\end{align}
\end{document}


