\let\negmedspace\undefined
\let\negthickspace\undefined
\documentclass[journal]{IEEEtran}
\usepackage[a5paper, margin=10mm, onecolumn]{geometry}
%\usepackage{lmodern} % Ensure lmodern is loaded for pdflatex
\usepackage{tfrupee} % Include tfrupee package

\setlength{\headheight}{1cm} % Set the height of the header box
\setlength{\headsep}{0mm}     % Set the distance between the header box and the top of the text

\usepackage{gvv-book}
\usepackage{gvv}
\usepackage{cite}
\usepackage{amsmath,amssymb,amsfonts,amsthm}
\usepackage{algorithmic}
\usepackage{graphicx}
\usepackage{textcomp}
\usepackage{xcolor}
\usepackage{txfonts}
\usepackage{listings}
\usepackage{enumitem}
\usepackage{mathtools}
\usepackage{gensymb}
\usepackage{comment}
\usepackage[breaklinks=true]{hyperref}
\usepackage{tkz-euclide} 
\usepackage{listings}
% \usepackage{gvv}                                     
\def\inputGnumericTable{}                               
\usepackage[latin1]{inputenc}                           
\usepackage{color}                                      
\usepackage{array}                                      
\usepackage{longtable}                                  
\usepackage{calc}                                       
\usepackage{multirow}                                   
\usepackage{hhline}                                     
\usepackage{ifthen}                                     
\usepackage{lscape}
\begin{document}

\bibliographystyle{IEEEtran}
\vspace{3cm}

\title{8.2.33}
\author{EE25BTECH11065 - Yoshita J}
% \maketitle
% \newpage
% \bigskip
{\let\newpage\relax\maketitle}

\renewcommand{\thefigure}{\theenumi}
\renewcommand{\thetable}{\theenumi}
\setlength{\intextsep}{10pt} % Space between text and floats

\textbf{Question}:\\
Find the equation of the conic with length of major axis 26, foci $(\pm 5, 0)$.
\\

\textbf{Solution:} \\
The given foci are $\mathbf{F}_1 = \myvec{5 \\ 0}$ and $\mathbf{F}_2 = \myvec{-5 \\ 0}$.
\\

The center of the conic is the midpoint of the foci:
\begin{align}
    \mathbf{u} = \frac{\mathbf{F}_1 + \mathbf{F}_2}{2} = \myvec{0 \\ 0}
\end{align}
\\

The length of the major axis is given as $2a = 26$
So, The distance from the center to a focus is $c=5$.

Eccentricity:
\begin{align}
    e = \frac{c}{a} = \frac{5}{13}
\end{align}
\\


The general equation of a conic is given by :
\begin{align}
    g(\mathbf{x}) = \mathbf{x}^T\mathbf{V}\mathbf{x} + 2\mathbf{u}^T\mathbf{x} + f = 0
\end{align}
where, x is a vertex on the major axis, 
\\


Since the center $\mathbf{u}=\mathbf{0}$, the equation simplifies to
\begin{align}
\mathbf{x}^T\mathbf{V}\mathbf{x} + f = 0.
\end{align}
\\


where,
\begin{align}
    \mathbf{V} = ||\mathbf{n}||^2 \mathbf{I} - e^2\mathbf{n}\mathbf{n}^T
\end{align}

This simplifies to:
\begin{align}
    \mathbf{V} = \myvec{1-e^2 & 0 \\ 0 & 1}
\end{align}

Substituting,
\begin{align}
    \mathbf{V} = \myvec{1 - (5/13)^2 & 0 \\ 0 & 1} = \myvec{144/169 & 0 \\ 0 & 1}
\end{align}
\\


Simplifying equation (5) and (4),
\begin{align}
    \myvec{13 & 0} \myvec{144/169 & 0 \\ 0 & 1} \myvec{13 \\ 0} + f &= 0 \nonumber \\
    144 + f = 0 \Rightarrow f &= -144
\end{align}
\\


Final equation of the conic,
\begin{align*}
\mathbf{x}^T \myvec{144/169 & 0 \\ 0 & 1} \mathbf{x} - 144 = 0.
\end{align*}
\begin{figure}[h!]
\begin{center}
\includegraphics[width=\columnwidth]{figs/fig3.png}
\end{center}
\label{fig:Fig.1}
\end{figure}
\end{document}


