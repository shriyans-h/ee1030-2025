
\documentclass{beamer}
\usepackage[utf8]{inputenc}

\usetheme{Madrid}
\usecolortheme{default}
\usepackage{amsmath,amssymb,amsfonts,amsthm}
\usepackage{txfonts}
\usepackage{tkz-euclide}
\usepackage{listings}
\usepackage{adjustbox}
\usepackage{array}
\usepackage{tabularx}
\usepackage{gvv}
\usepackage{lmodern}
\usepackage{circuitikz}
\usepackage{tikz}
\usepackage{graphicx}

\setbeamertemplate{page number in head/foot}[totalframenumber]

\usepackage{tcolorbox}
\tcbuselibrary{minted,breakable,xparse,skins}



\definecolor{bg}{gray}{0.95}
\DeclareTCBListing{mintedbox}{O{}m!O{}}{%
  breakable=true,
  listing engine=minted,
  listing only,
  minted language=#2,
  minted style=default,
  minted options={%
    linenos,
    gobble=0,
    breaklines=true,
    breakafter=,,
    fontsize=\small,
    numbersep=8pt,
    #1},
  boxsep=0pt,
  left skip=0pt,
  right skip=0pt,
  left=25pt,
  right=0pt,
  top=3pt,
  bottom=3pt,
  arc=5pt,
  leftrule=0pt,
  rightrule=0pt,
  bottomrule=2pt,
  toprule=2pt,
  colback=bg,
  colframe=orange!70,
  enhanced,
  overlay={%
    \begin{tcbclipinterior}
    \fill[orange!20!white] (frame.south west) rectangle ([xshift=20pt]frame.north west);
    \end{tcbclipinterior}},
  #3,
}
\lstset{
    language=C,
    basicstyle=\ttfamily\small,
    keywordstyle=\color{blue},
    stringstyle=\color{orange},
    commentstyle=\color{green!60!black},
    numbers=left,
    numberstyle=\tiny\color{gray},
    breaklines=true,
    showstringspaces=false,
}
\begin{document}

\title 
{8.2.33}
\date{September 19,2025}


\author 
{EE25BTECH11065-Yoshita J}






\frame{\titlepage}
\begin{frame}{Question}
Find the equation of the conic with length of major axis 26, foci $(\pm 5, 0)$.

\end{frame}


\begin{frame}{Theoretical Solution}
The given foci are $\mathbf{F}_1 = \myvec{5 \\ 0}$ and $\mathbf{F}_2 = \myvec{-5 \\ 0}$.


The center of the conic is the midpoint of the foci:
\begin{align}
    \mathbf{u} = \frac{\mathbf{F}_1 + \mathbf{F}_2}{2} = \myvec{0 \\ 0}
\end{align}
\\

The length of the major axis is given as $2a = 26$
So, The distance from the center to a focus is $c=5$.

Eccentricity:
\begin{align}
    e = \frac{c}{a} = \frac{5}{13}
\end{align}
\\

\end{frame}

\begin{frame}{Theoretical Solution}
The general equation of a conic is given by :
\begin{align}
    g(\mathbf{x}) = \mathbf{x}^T\mathbf{V}\mathbf{x} + 2\mathbf{u}^T\mathbf{x} + f = 0
\end{align}
where, x is a vertex on the major axis, 
\\
Since the center $\mathbf{u}=\mathbf{0}$, the equation simplifies to
\begin{align}
\mathbf{x}^T\mathbf{V}\mathbf{x} + f = 0.
\end{align}
\\
where,
\begin{align}
    \mathbf{V} = ||\mathbf{n}||^2 \mathbf{I} - e^2\mathbf{n}\mathbf{n}^T
\end{align}

This simplifies to:
\begin{align}
    \mathbf{V} = \myvec{1-e^2 & 0 \\ 0 & 1}
\end{align}

\end{frame}

\begin{frame}{Theoretical Solution}
Substituting,
\begin{align}
    \mathbf{V} = \myvec{1 - (5/13)^2 & 0 \\ 0 & 1} = \myvec{144/169 & 0 \\ 0 & 1}
\end{align}
\\

Simplifying equation (5) and (4),
\begin{align}
    \myvec{13 & 0} \myvec{144/169 & 0 \\ 0 & 1} \myvec{13 \\ 0} + f &= 0 \nonumber \\
    144 + f = 0 \Rightarrow f &= -144
\end{align}
\\

Final equation of the conic,
\begin{align*}
\mathbf{x}^T \myvec{144/169 & 0 \\ 0 & 1} \mathbf{x} - 144 = 0.
\end{align*}
\end{frame}

\begin{frame}[fragile]
    \frametitle{C Code}

    \begin{lstlisting}

#include <stdio.h>
#include <math.h>

#define PI 3.1415926535

double calculate_circular_sector_area() {
    double radius = 2.0;
    double angle_in_radians = PI / 6.0;
    double area = 0.5 * radius * radius * angle_in_radians;
    
    return area;
}


    \end{lstlisting}
\end{frame}


\begin{frame}[fragile]
    \frametitle{Python Code}
    \begin{lstlisting}
import numpy as np
import matplotlib.pyplot as plt

a = 13
b = 12
c = 5
theta = np.linspace(0, 2 * np.pi, 200)

x = a * np.cos(theta)
y = b * np.sin(theta)

plt.figure(figsize=(10, 8))
ax = plt.gca()

ax.plot(x, y, label='Ellipse: $x^2/169 + y^2/144 = 1$')

ax.plot(0, 0, 'ko', label='Center (0, 0)')
ax.plot(c, 0, 'ro', label='Focus 1 (5, 0)')
ax.plot(-c, 0, 'ro', label='Focus 2 (-5, 0)')


    \end{lstlisting}
\end{frame}

\begin{frame}[fragile]
    \frametitle{Python Code}
    \begin{lstlisting}
ax.set_title('Plot of the Ellipse', fontsize=16)
ax.set_xlabel('X-axis')
ax.set_ylabel('Y-axis')

ax.set_aspect('equal', adjustable='box')

ax.grid(True, linestyle='--')
ax.legend()

ax.set_xlim(-a - 2, a + 2)
ax.set_ylim(-b - 2, b + 2)

ax.axhline(0, color='black', linewidth=0.5)
ax.axvline(0, color='black', linewidth=0.5)

plt.show()
   
    \end{lstlisting}
\end{frame}


\begin{frame}{Plot}

    
    \begin{figure}[h!]
\begin{center}
\includegraphics[width=\columnwidth]{figs/fig3.png}
\end{center}
\label{fig:Fig.1}
\end{figure}
    
\end{frame}    
    
  

\end{document}
