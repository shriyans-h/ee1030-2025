\let\negmedspace\undefined
\let\negthickspace\undefined
\documentclass[journal]{IEEEtran}
\usepackage[a5paper, margin=10mm, onecolumn]{geometry}
%\usepackage{lmodern}
\usepackage{tfrupee}

\setlength{\headheight}{1cm}
\setlength{\headsep}{0mm}

\usepackage{gvv-book}
\usepackage{gvv}
\usepackage{cite}
\usepackage{amsmath,amssymb,amsfonts,amsthm}
\usepackage{algorithmic}
\usepackage{graphicx}
\usepackage{textcomp}
\usepackage{xcolor}
\usepackage{txfonts}
\usepackage{listings}
\usepackage{enumitem}
\usepackage{mathtools}
\usepackage{gensymb}
\usepackage{comment}
\usepackage[breaklinks=true]{hyperref}
\usepackage{tkz-euclide} 
\usepackage{listings}
\def\inputGnumericTable{}                                 
\usepackage[latin1]{inputenc}                                
\usepackage{color}                                            
\usepackage{array}                                            
\usepackage{longtable}                                       
\usepackage{calc}                                             
\usepackage{multirow}                                         
\usepackage{hhline}                                           
\usepackage{ifthen}                                           
\usepackage{lscape}
\usepackage{circuitikz}

\begin{document}

\bibliographystyle{IEEEtran}
\vspace{3cm}

\title{1.8.26}
\author{EE25BTECH11009-Anshu kumar ram}
\maketitle
{\let\newpage\relax\maketitle}

\renewcommand{\thefigure}{\theenumi}
\renewcommand{\thetable}{\theenumi}
\setlength{\intextsep}{10pt}

\numberwithin{equation}{enumi}
\numberwithin{figure}{enumi}
\renewcommand{\thetable}{\theenumi}

\textbf{Question}:\\
Find a point on the Y axis which is equidistant from the points $A(6, 5)$ and $B(-4, 3)$.

\solution \\
The input parameters for this problem are available in Table 
\begin{tabular}[12pt]{ |c| c|}
    \hline
    \textbf{Name} & \textbf{Point}\\ 
    \hline
	Point A &\myvec{h \\ k}\\
    \hline 
 Point B &\myvec{x1 \\ y1}\\
    \hline
	  Point R &\myvec{x2 \\ y2}\\
    \hline
    
    \end{tabular}



If $\vec{O}$ lies on the $y$-axis and is equidistant from the points $\vec{A}$ and $\vec{B}$, 
\begin{align}
 \norm{\vec{O}-\vec{A}} &=
\norm{\vec{O}-\vec{B}} 
\\
 \implies \norm{\vec{O}-\vec{A}}^2 &=
\norm{\vec{O}-\vec{B}}^2 
\\
 \implies \norm{\vec{O}}^2-2{\vec{O}}^{\top}\vec{A} + \norm{\vec{A}}^2
	&= \norm{\vec{O}}^2-2{\vec{O}}^{\top}\vec{B} + \norm{\vec{B}}^2,
\end{align}
which can be simplified to obtain
\begin{align}
	  \brak{\vec{A}-\vec{B}}^\top   \vec{O}&=\frac{\norm{\vec{A}}^2 -\norm{\vec{B}}^2 }{2}.
\end{align}
\begin{align}
\because \vec{O} &= y\vec{e}_2,
\end{align}
\begin{align}
 y &=\frac{\norm{\vec{A}}^2 -\norm{\vec{B}}^2 }{2\brak{\vec{A}-\vec{B}}^{\top }\vec{e}_2}.
 \label{eq:10/7/1/75}  
\end{align}
Substituting from table we get ,
 $y =  9$.  Thus, 
\begin{align}
\vec{O} = \myvec{ 0 \\ 9}.
\end{align}

\begin{figure}[H]
 \begin{center}
  \includegraphics[width=0.75\columnwidth]{figs/equidistant_point.png}
 \end{center}
\caption{Point $O(0,9)$ on the $y$-axis is equidistant from $A$ and $B$.}
\label{fig}
\end{figure}

\end{document}
