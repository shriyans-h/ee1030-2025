\documentclass{beamer}
\usepackage[utf8]{inputenc}

\usetheme{Madrid}
\usecolortheme{default}
\usepackage{amsmath,amssymb,amsfonts,amsthm}
\usepackage{txfonts}
\usepackage{tkz-euclide}
\usepackage{listings}
\usepackage{adjustbox}
\usepackage{array}
\usepackage{tabularx}
\usepackage{gvv}
\usepackage{lmodern}
\usepackage{circuitikz}
\usepackage{tikz}
\usepackage{graphicx}

\setbeamertemplate{page number in head/foot}[totalframenumber]

\usepackage{tcolorbox}
\tcbuselibrary{minted,breakable,xparse,skins}

\definecolor{bg}{gray}{0.95}
\DeclareTCBListing{mintedbox}{O{}m!O{}}{%
  breakable=true,
  listing engine=minted,
  listing only,
  minted language=#2,
  minted style=default,
  minted options={%
    linenos,
    gobble=0,
    breaklines=true,
    breakafter=,,
    fontsize=\small,
    numbersep=8pt,
    #1},
  boxsep=0pt,
  left skip=0pt,
  right skip=0pt,
  left=25pt,
  right=0pt,
  top=3pt,
  bottom=3pt,
  arc=5pt,
  leftrule=0pt,
  rightrule=0pt,
  bottomrule=2pt,
  toprule=2pt,
  colback=bg,
  colframe=orange!70,
  enhanced,
  overlay={%
    \begin{tcbclipinterior}
    \fill[orange!20!white] (frame.south west) rectangle ([xshift=20pt]frame.north west);
    \end{tcbclipinterior}},
  #3,
}
\lstset{
    language=C,
    basicstyle=\ttfamily\small,
    keywordstyle=\color{blue},
    stringstyle=\color{orange},
    commentstyle=\color{green!60!black},
    numbers=left,
    numberstyle=\tiny\color{gray},
    breaklines=true,
    showstringspaces=false,
}
%------------------------------------------------------------
\title %optional
{1.8.26}
\date{September 05, 2025}
\author % (optional)
{Anshu kumar ram-EE25BTECH11009}

\begin{document}

\frame{\titlepage}

% ---------------- Question ----------------
\begin{frame}{Question}
Find a point on the $y$-axis which is equidistant from the points $A(6,5)$ and $B(-4,3)$.
\end{frame}

% ---------------- Table ----------------
\begin{frame}{Input Parameters}
The input parameters for this problem are available in the table below.
\begin{tabular}[12pt]{ |c| c|}
    \hline
    \textbf{Name} & \textbf{Point}\\ 
    \hline
	Point A &\myvec{h \\ k}\\
    \hline 
 Point B &\myvec{x1 \\ y1}\\
    \hline
	  Point R &\myvec{x2 \\ y2}\\
    \hline
    
    \end{tabular}

\end{frame}


% ---------------- Solution Step 1 ----------------
\begin{frame}{Solution Step 1}
If $\vec{O}$ lies on the $y$-axis and is equidistant from the points $\vec{A}$ and $\vec{B}$,  
\begin{align}
\norm{\vec{O}-\vec{A}} &= \norm{\vec{O}-\vec{B}} \\
\implies \norm{\vec{O}-\vec{A}}^2 &= \norm{\vec{O}-\vec{B}}^2
\end{align}
Expanding both sides,  
\begin{align}
\norm{\vec{O}}^2 - 2\vec{O}^\top \vec{A} + \norm{\vec{A}}^2
&= \norm{\vec{O}}^2 - 2\vec{O}^\top \vec{B} + \norm{\vec{B}}^2
\end{align}
\end{frame}

% ---------------- Solution Step 2 ----------------
\begin{frame}{Solution Step 2}
Simplifying,  
\begin{align}
(\vec{A}-\vec{B})^\top \vec{O}
= \frac{\norm{\vec{A}}^2 - \norm{\vec{B}}^2}{2}
\end{align}
Since $\vec{O}$ lies on the $y$-axis,  
\begin{align}
\vec{O} &= y \vec{e}_2
\end{align}
Thus,  
\begin{align}
y &= \frac{\norm{\vec{A}}^2 - \norm{\vec{B}}^2}{2(\vec{A}-\vec{B})^\top \vec{e}_2}
\end{align}
\end{frame}

% ---------------- Solution Step 3 ----------------
\begin{frame}{Solution Step 3}
Substituting from the table we get,  
\begin{align}
y &= 9
\end{align}
So the required point is,  
\[
\boxed{\vec{O} = \myvec{0 \\ 9}}
\]
\end{frame}
% ---------------- Plot ----------------
\begin{frame}{Plot}
\centering
\includegraphics[width=\columnwidth, height=0.8\textheight, keepaspectratio]{figs/equidistant_point.png}
\end{frame}



% ---------------- Pure Python ----------------
\begin{frame}[fragile]{Pure Python (Part 1)}
\begin{lstlisting}
import sys
import numpy as np
import matplotlib.pyplot as plt

# path to your external scripts
sys.path.insert(0, '/home/anshu-ram/matgeo/codes/CoordGeo')

# local imports
from line.funcs import line_gen_num

# ========== Given vectors ==========
A = np.array([6, 5]).reshape(-1,1)
B = np.array([-4, 3]).reshape(-1,1)
O = np.array([0, 9]).reshape(-1,1)
\end{lstlisting}
\end{frame}

\begin{frame}[fragile]{Pure Python (Part 2)}
\begin{lstlisting}
# ========== Lines ==========
x_AB = line_gen_num(A, B, 20)
x_OA = line_gen_num(O, A, 20)
x_OB = line_gen_num(O, B, 20)

# ========== Plotting ==========
plt.plot(x_AB[0,:], x_AB[1,:], "g--", label="Line AB")
plt.plot(x_OA[0,:], x_OA[1,:], "r--", label="Line OA")
plt.plot(x_OB[0,:], x_OB[1,:], "b--", label="Line OB")
\end{lstlisting}
\end{frame}

\begin{frame}[fragile]{Pure Python (Part 3)}
\begin{lstlisting}
# Points
tri_coords = np.hstack((A,B,O))   # stack column vectors
plt.scatter(tri_coords[0,:], tri_coords[1,:])

# Labels
vert_labels = [
    f'A({int(A[0,0])},{int(A[1,0])})',
    f'B({int(B[0,0])},{int(B[1,0])})',
    f'O({int(O[0,0])},{int(O[1,0])})'
]

for i, txt in enumerate(vert_labels):
    plt.annotate(txt, (tri_coords[0,i], tri_coords[1,i]),
                 textcoords="offset points", xytext=(0,10), ha='right')
\end{lstlisting}
\end{frame}

\begin{frame}[fragile]{Pure Python (Part 4)}
\begin{lstlisting}
plt.xlabel('$x$')
plt.ylabel('$y$')
plt.legend(loc='best')
plt.grid()
plt.title("Point O(0,9) equidistant from A and B")
plt.axis('equal')

# Save & show
plt.savefig("../figs/equidistant_point.png")
plt.show()
\end{lstlisting}
\end{frame}
% ---------------- C Code ----------------
\begin{frame}[fragile]{C Code (Part 1)}
\begin{lstlisting}
#include <stdio.h>

/* Function to compute O on y-axis equidistant from A and B */
void equidistant_yaxis(const double* A, const double* B, double* O) {
    // A = (x1,y1), B = (x2,y2), O = (0,y)
    double x1 = A[0], y1 = A[1];
    double x2 = B[0], y2 = B[1];

    double num = (x1*x1 + y1*y1) - (x2*x2 + y2*y2);
    double den = 2*(y1 - y2);

    O[0] = 0.0;
    O[1] = num/den;
}
\end{lstlisting}
\end{frame}

\begin{frame}[fragile]{C Code (Part 2)}
\begin{lstlisting}
/* Generate n points on line AB */
void line_gen(double* X, double* Y, const double* A, const double* B, int n, int m) {
    double temp[2];

    for (int i = 0; i < 2; i++) {
        temp[i] = (B[i] - A[i]) / (double)(n-1);
    }

    for (int i = 0; i < n; i++) {
        X[i] = A[0] + temp[0] * i;
        Y[i] = A[1] + temp[1] * i;
    }
}
\end{lstlisting}
\end{frame}

% ---------------- Python + C Integration ----------------
\begin{frame}[fragile]{Python + C Integration (Part 1)}
\begin{lstlisting}
import ctypes
import numpy as np
import matplotlib.pyplot as plt

# Load shared library
handc = ctypes.CDLL("./func.so")

# Define equidistant_yaxis prototype
handc.equidistant_yaxis.argtypes = [
    ctypes.POINTER(ctypes.c_double),
    ctypes.POINTER(ctypes.c_double),
    ctypes.POINTER(ctypes.c_double),
]
handc.equidistant_yaxis.restype = None
\end{lstlisting}
\end{frame}

\begin{frame}[fragile]{Python + C Integration (Part 2)}
\begin{lstlisting}
# Define line_gen prototype
handc.line_gen.argtypes = [
    ctypes.POINTER(ctypes.c_double),
    ctypes.POINTER(ctypes.c_double),
    ctypes.POINTER(ctypes.c_double),
    ctypes.POINTER(ctypes.c_double),
    ctypes.c_int,
    ctypes.c_int
]
handc.line_gen.restype = None

# Dimension
m = 2

# Points A and B
A = np.array([6.0, 5.0], dtype=np.float64)
B = np.array([-4.0, 3.0], dtype=np.float64)

# Placeholder for O
O = np.zeros(m, dtype=np.float64)
\end{lstlisting}
\end{frame}

\begin{frame}[fragile]{Python + C Integration (Part 3)}
\begin{lstlisting}
# Call C function to compute O
handc.equidistant_yaxis(
    A.ctypes.data_as(ctypes.POINTER(ctypes.c_double)),
    B.ctypes.data_as(ctypes.POINTER(ctypes.c_double)),
    O.ctypes.data_as(ctypes.POINTER(ctypes.c_double))
)

print("A =", A)
print("B =", B)
print("O =", O)

# Generate lines
n = 20
X_l = np.zeros(n, dtype=np.float64)
Y_l = np.zeros(n, dtype=np.float64)
\end{lstlisting}
\end{frame}

\begin{frame}[fragile]{Python + C Integration (Part 4)}
\begin{lstlisting}
# AB
handc.line_gen(X_l.ctypes.data_as(ctypes.POINTER(ctypes.c_double)),
               Y_l.ctypes.data_as(ctypes.POINTER(ctypes.c_double)),
               A.ctypes.data_as(ctypes.POINTER(ctypes.c_double)),
               B.ctypes.data_as(ctypes.POINTER(ctypes.c_double)),
               n, m)
x_AB, y_AB = X_l.copy(), Y_l.copy()

# OA
handc.line_gen(X_l.ctypes.data_as(ctypes.POINTER(ctypes.c_double)),
               Y_l.ctypes.data_as(ctypes.POINTER(ctypes.c_double)),
               O.ctypes.data_as(ctypes.POINTER(ctypes.c_double)),
               A.ctypes.data_as(ctypes.POINTER(ctypes.c_double)),
               n, m)
x_OA, y_OA = X_l.copy(), Y_l.copy()
\end{lstlisting}
\end{frame}

\begin{frame}[fragile]{Python + C Integration (Part 5)}
\begin{lstlisting}
# OB
handc.line_gen(X_l.ctypes.data_as(ctypes.POINTER(ctypes.c_double)),
               Y_l.ctypes.data_as(ctypes.POINTER(ctypes.c_double)),
               O.ctypes.data_as(ctypes.POINTER(ctypes.c_double)),
               B.ctypes.data_as(ctypes.POINTER(ctypes.c_double)),
               n, m)
x_OB, y_OB = X_l.copy(), Y_l.copy()

# Plotting
plt.figure()
plt.plot(x_AB, y_AB, "g--", label="Line AB")
plt.plot(x_OA, y_OA, "r--", label="Line OA")
plt.plot(x_OB, y_OB, "b--", label="Line OB")
\end{lstlisting}
\end{frame}

\begin{frame}[fragile]{Python + C Integration (Part 6)}
\begin{lstlisting}
# Points
plt.scatter(A[0], A[1], color="blue", s=50)
plt.scatter(B[0], B[1], color="red", s=50)
plt.scatter(O[0], O[1], color="green", s=50)

# Labels
plt.annotate(f"A({A[0]:.0f},{A[1]:.0f})", (A[0], A[1]), textcoords="offset points", xytext=(-20,10))
plt.annotate(f"B({B[0]:.0f},{B[1]:.0f})", (B[0], B[1]), textcoords="offset points", xytext=(10,-15))
plt.annotate(f"O({O[0]:.0f},{O[1]:.0f})", (O[0], O[1]), textcoords="offset points", xytext=(10,10))
\end{lstlisting}
\end{frame}

\begin{frame}[fragile]{Python + C Integration (Part 7)}
\begin{lstlisting}
# Equal aspect ratio
plt.gca().set_aspect("equal", adjustable="box")
plt.xlim([-8,8])
plt.ylim([0,12])

plt.xlabel("X")
plt.ylabel("Y")
plt.title("Point O on y-axis equidistant from A and B")
plt.legend(loc="upper left")
plt.grid(True)

# Save & show
plt.savefig("../figs/equidistant_graph.png")
plt.show()
\end{lstlisting}
\end{frame}

\end{document}
