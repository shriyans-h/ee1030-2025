\let\negmedspace\undefined
\let\negthickspace\undefined
\documentclass[journal]{IEEEtran}
\usepackage[a5paper, margin=10mm, onecolumn]{geometry}
%\usepackage{lmodern} % Ensure lmodern is loaded for pdflatex
\usepackage{tfrupee} % Include tfrupee package

\setlength{\headheight}{1cm} % Set the height of the header box
\setlength{\headsep}{0mm}     % Set the distance between the header box and the top of the text

\usepackage{gvv-book}
\usepackage{gvv}
\usepackage{cite}
\usepackage{amsmath,amssymb,amsfonts,amsthm}
\usepackage{algorithmic}
\usepackage{graphicx}
\usepackage{textcomp}
\usepackage{xcolor}
\usepackage{txfonts}
\usepackage{listings}
\usepackage{enumitem}
\usepackage{mathtools}
\usepackage{gensymb}
\usepackage{comment}
\usepackage[breaklinks=true]{hyperref}
\usepackage{tkz-euclide} 
\usepackage{listings}
% \usepackage{gvv}                                        
\def\inputGnumericTable{}                                 
\usepackage[latin1]{inputenc}                                
\usepackage{color}                                            
\usepackage{array}                                            
\usepackage{longtable}                                       
\usepackage{calc}                                             
\usepackage{multirow}                                         
\usepackage{hhline}                                           
\usepackage{ifthen}                                           
\usepackage{lscape}
\usepackage{circuitikz}
\tikzstyle{block} = [rectangle, draw, fill=blue!20, 
    text width=4em, text centered, rounded corners, minimum height=3em]
\tikzstyle{sum} = [draw, fill=blue!10, circle, minimum size=1cm, node distance=1.5cm]
\tikzstyle{input} = [coordinate]
\tikzstyle{output} = [coordinate]


\begin{document}

\bibliographystyle{IEEEtran}
\vspace{3cm}

\title{4.5.11}
\author{EE25BTECH11009 - Anshu Kumar Ram }
\maketitle
% \newpage
% \bigskip
{\let\newpage\relax\maketitle}

\renewcommand{\thefigure}{\theenumi}
\renewcommand{\thetable}{\theenumi}
\setlength{\intextsep}{10pt} % Space between text and floats


\numberwithin{equation}{enumi}
\numberwithin{figure}{enumi}
\renewcommand{\thetable}{\theenumi}

\textbf{Question}:\\
Find the cartesian equation of the line which passes through the point $(-2,4,-5)$
and parallel to the line 
\begin{align}
\frac{x+3}{3} = \frac{y-4}{5} = \frac{z+8}{6}
\end{align}

\solution \\

From the given line, the direction vector is
\begin{align}
\vec{m} = \myvec{3 \\ 5 \\ 6}
\end{align}

The required line passes through
\begin{align}
\vec{A} = \myvec{-2 \\ 4 \\ -5}
\end{align}

So, the vector equation is
\begin{align}
\vec{r} = \vec{A} + \lambda \vec{m}, \ \lambda \in \mathbb{R}
\end{align}

In matrix form,
\begin{align}
\myvec{x \\ y \\ z} = \myvec{-2 \\ 4 \\ -5} + \lambda \myvec{3 \\ 5 \\ 6}
\end{align}

Eliminating $\lambda$,
\begin{align}
\frac{x+2}{3} = \frac{y-4}{5} = \frac{z+5}{6}
\end{align}

Hence, the required cartesian equation of the line is
\begin{align}
\boxed{\;\frac{x+2}{3} = \frac{y-4}{5} = \frac{z+5}{6}\;}
\end{align}


\begin{figure}[h!]
    \centering
    \includegraphics[height=0.5\textheight, keepaspectratio]{figs/Figure_1.png}
    \label{figure_1}
\end{figure}

\end{document}
