\let\negmedspace\undefined
\let\negthickspace\undefined
\documentclass[journal]{IEEEtran}
\usepackage[a4paper, margin=10mm, onecolumn]{geometry}
\usepackage{lmodern} % Ensure lmodern is loaded for pdflatex
\usepackage{tfrupee} % Include tfrupee package

\setlength{\headheight}{1cm} % Set the height of the header box
\setlength{\headsep}{0mm}  % Set the distance between the header box and the top of the text

\usepackage{gvv-book}
\usepackage{gvv}
\usepackage{cite}
\usepackage{amsmath,amssymb,amsfonts,amsthm}
\usepackage{algorithmic}
\usepackage{graphicx}
\usepackage{float}
\usepackage{textcomp}
\usepackage{xcolor}
\usepackage{txfonts}
\usepackage{listings}
\usepackage{enumitem}
\usepackage{mathtools}
\usepackage{gensymb}
\usepackage{comment}
\usepackage[breaklinks=true]{hyperref}
\usepackage{tkz-euclide} 
\usepackage{listings}
% \usepackage{gvv}                                        
\def\inputGnumericTable{}                                 
\usepackage[latin1]{inputenc}                                
\usepackage{color}                                            
\usepackage{array}                                            
\usepackage{longtable}                                       
\usepackage{calc}                                             
\usepackage{multirow}                                         
\usepackage{hhline}                                           
\usepackage{ifthen}                                           
\usepackage{lscape}
\usepackage{tikz}
\usetikzlibrary{patterns}

\begin{document}

\bibliographystyle{IEEEtran}
\vspace{3cm}

\title{2.5.32}
\author{EE25BTECH11064 - Yojit Manral}

\maketitle
% \maketitle
% \newpage
% \bigskip
{\let\newpage\relax\maketitle}
\renewcommand{\thefigure}{\theenumi}
\renewcommand{\thetable}{\theenumi}
\setlength{\intextsep}{10pt} % Space between text and float

\textbf{Question:}\\
Show that the points $\brak{7,10}$, $\brak{-2,5}$ and $\brak{3,4}$ are vertices of an isosceles right triangle.\\

\textbf{Solution:}\\
\begin{table}[h!]    
  \centering
  \begin{tabular}[12pt]{ |c| c|}
    \hline
    \textbf{Points} & \textbf{Name}\\ 
    \hline
	\myvec{2\\5\\-3} & Point $\Vec{R}$ \\
    \hline 
	\myvec{-2\\-3\\5} & Point $\Vec{S}$\\
    \hline
	\myvec{5\\3\\-3} & Point $\Vec{T}$\\
    \hline
\end{tabular}
  \caption{List of Points}
  \label{Table_1}
\end{table} \\
$\rightarrow$ Let
\begin{align}
    \vec{a} = \vec{C} - \vec{B} &= \myvec{5\\-1}\\
    \vec{b} = \vec{A} - \vec{C} &= \myvec{4\\6}\\
    \vec{c} = \vec{B} - \vec{A} &= \myvec{-9\\-5}\\
    \angle{A} &= \angle{BAC}\\
    \angle{B} &= \angle{CBA}\\
    \angle{C} &= \angle{ACB}
\end{align}
$\rightarrow$ Then the cosines of the angles are
\begin{align}
    \cos{A} = -\frac{\vec{b}^{T}\vec{c}}{\norm{\vec{b}}\norm{\vec{c}}} = \frac{66}{\sqrt{52}\sqrt{106}} = 0.889\\
    \cos{B} = -\frac{\vec{c}^{T}\vec{a}}{\norm{\vec{c}}\norm{\vec{a}}} = \frac{40}{\sqrt{106}\sqrt{26}} = 0.762\\
    \cos{C} = -\frac{\vec{a}^{T}\vec{b}}{\norm{\vec{a}}\norm{\vec{b}}} = -\frac{14}{\sqrt{26}\sqrt{52}} = -0.381 \\
    \notag
\end{align}
$\rightarrow$ As none of the cosines are equal, $\triangle$ABC is not isosceles. Also, since $\cos{C}$ is negative, $\triangle$ABC forms an obtuse angled triangle. \\
$\implies$ $\triangle$ABC is neither isosceles nor right-angled triangle.

\begin{figure}[h!]
   \centering
   \includegraphics[width=0.7\linewidth]{figs/01.png}
   \caption{Plot of $\triangle$ABC}
   \label{Plot_1}
\end{figure}
\end{document}
