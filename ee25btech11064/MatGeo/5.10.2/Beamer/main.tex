\documentclass{beamer}
\mode<presentation>
\usepackage{amsmath,amssymb,mathtools}
\usepackage{textcomp}
\usepackage{gensymb}
\usepackage{adjustbox}
\usepackage{subcaption}
\usepackage{enumitem}
\usepackage{multicol}
\usepackage{listings}
\usepackage{url}
\usepackage{graphicx} % <-- needed for images
\def\UrlBreaks{\do\/\do-}

\usetheme{Boadilla}
\usecolortheme{lily}
\setbeamertemplate{footline}{
  \leavevmode%
  \hbox{%
  \begin{beamercolorbox}[wd=\paperwidth,ht=2ex,dp=1ex,right]{author in head/foot}%
    \insertframenumber{} / \inserttotalframenumber\hspace*{2ex}
  \end{beamercolorbox}}%
  \vskip0pt%
}
\setbeamertemplate{navigation symbols}{}

\lstset{
  frame=single,
  breaklines=true,
  columns=fullflexible,
  basicstyle=\ttfamily\tiny   % tiny font so code fits
}

\numberwithin{equation}{section}

% ---- your macros ----
\providecommand{\nCr}[2]{\,^{#1}C_{#2}}
\providecommand{\nPr}[2]{\,^{#1}P_{#2}}
\providecommand{\mbf}{\mathbf}
\providecommand{\pr}[1]{\ensuremath{\Pr\left(#1\right)}}
\providecommand{\qfunc}[1]{\ensuremath{Q\left(#1\right)}}
\providecommand{\sbrak}[1]{\ensuremath{{}\left[#1\right]}}
\providecommand{\lsbrak}[1]{\ensuremath{{}\left[#1\right.}}
\providecommand{\rsbrak}[1]{\ensuremath{\left.#1\right]}}
\providecommand{\brak}[1]{\ensuremath{\left(#1\right)}}
\providecommand{\lbrak}[1]{\ensuremath{\left(#1\right.}}
\providecommand{\rbrak}[1]{\ensuremath{\left.#1\right)}}
\providecommand{\cbrak}[1]{\ensuremath{\left\{#1\right\}}}
\providecommand{\lcbrak}[1]{\ensuremath{\left\{#1\right.}}
\providecommand{\rcbrak}[1]{\ensuremath{\left.#1\right\}}}
\theoremstyle{remark}
\newtheorem{rem}{Remark}
\newcommand{\sgn}{\mathop{\mathrm{sgn}}}
\providecommand{\abs}[1]{\left\vert#1\right\vert}
\providecommand{\res}[1]{\Res\displaylimits_{#1}}
\providecommand{\norm}[1]{\lVert#1\rVert}
\providecommand{\mtx}[1]{\mathbf{#1}}
\providecommand{\mean}[1]{E\left[ #1 \right]}
\providecommand{\fourier}{\overset{\mathcal{F}}{ \rightleftharpoons}}
\providecommand{\system}{\overset{\mathcal{H}}{ \longleftrightarrow}}
\providecommand{\dec}[2]{\ensuremath{\overset{#1}{\underset{#2}{\gtrless}}}}
\newcommand{\myvec}[1]{\ensuremath{\begin{pmatrix}#1\end{pmatrix}}}
\let\vec\mathbf

\title{MatGeo Presentation - Problem 5.10.2}
\author{EE25BTECH11064 - Yojit Manral}
\date{}

\begin{document}

\frame{\titlepage}
\begin{frame}{Question}
Balance the following chemical equation:
\begin{align}
    NaOH + H_2SO_4 \rightarrow Na_2SO_4 + H_2O
\end{align}
\end{frame}

\begin{frame}{Solution}
$\rightarrow$ Let the balanced version of (1) be
\begin{align}
    x_1NaOH + x_2H_2SO_4 \rightarrow x_3Na_2SO_4 + x_4H_2O
\end{align}
$\rightarrow$ This results in the following equations
\begin{align}
    (x_1 - 2x_3)Na &= 0 \\
    (x_1 + 4x_2 - 4x_3 - x_4)O &= 0 \\
    (x_1 + 2x_2 - 2x_4)H &= 0 \\
    (x_2 - x_3)S &= 0
\end{align}
$\rightarrow$ Which can further be expressed as
\begin{align}
    (1x_1 + 0x_2 - 2x_3 + 0x_4)Na &= 0 \\
    (1x_1 + 4x_2 - 4x_3 - 1x_4)O &= 0 \\
    (1x_1 + 2x_2 + 0x_3 - 2x_4)H &= 0 \\
    (0x_1 + 1x_2 - 1x_3 + 0x_4)S &= 0
\end{align}
\end{frame}

\begin{frame}{Solution}
$\rightarrow$ Giving us the matrix equation
\begin{align}
    \myvec{1&0&-2&0\\1&4&-4&-1\\1&2&0&-2\\0&1&-1&0}\vec{x} = 0, \hspace{0.5cm} \vec{x} = \myvec{x_1\\x_2\\x_3\\x_4}
\end{align}
$\rightarrow$ Now, (11) can be reduced as follows
\begin{align}
&\myvec{1&0&-2&0\\1&4&-4&-1\\1&2&0&-2\\0&1&-1&0}\xrightarrow[R_3 \leftrightarrow R_3 - R_1]{R_2 \leftrightarrow R_2 - R_1}\myvec{1&0&-2&0\\0&4&-2&-1\\0&2&2&-2\\0&1&-1&0} \\
\xrightarrow{R_2 \leftrightarrow (1/4)R_2}&\myvec{1&0&-2&0\\0&1&-1/2&-1/4\\0&2&2&-2\\0&1&-1&0}\xrightarrow[R_4 \leftrightarrow R_4 - R_2]{R_3 \leftrightarrow R_3 - 2R_2}\myvec{1&0&-2&0\\0&1&-1/2&-1/4\\0&0&3&-3/2\\0&0&-1/2&1/4}
\end{align}
\end{frame}

\begin{frame}{Solution}
\begin{align}
\xrightarrow[R_1 \leftrightarrow R_1 + 2R_3]{R_3 \leftrightarrow (1/3)R_3}\myvec{1&0&0&-1\\0&1&-1/2&-1/4\\0&0&1&-1/2\\0&0&-1/2&1/4}\xrightarrow[R_4 \leftrightarrow R_4 + (1/2)R_3]{R_2 \leftrightarrow R_2 + (1/2)R_3}\myvec{1&0&0&-1\\0&1&0&-1/2\\0&0&1&-1/2\\0&0&0&0}
\end{align}
$\rightarrow$ Thus
\begin{align}
    x_1 = x_4, x_2 = \frac{1}{2}x_4, x_3 = \frac{1}{2}x_4 \\
    \implies \vec{x} = x_4\myvec{1\\1/2\\1/2\\1} = \myvec{2\\1\\1\\2}
\end{align}
\hspace{0.3cm} by substituting $x_4 = 2$. Hence, (2) finally becomes
\begin{align}
    2NaOH + H_2SO_4 \rightarrow Na_2SO_4 + 2H_2O
\end{align}
\end{frame}
\end{document}
