\let\negmedspace\undefined
\let\negthickspace\undefined
\documentclass[journal]{IEEEtran}
\usepackage[a5paper, margin=10mm, onecolumn]{geometry}
\usepackage{lmodern} % Ensure lmodern is loaded for pdflatex
\usepackage{tfrupee} % Include tfrupee package

\setlength{\headheight}{1cm} % Set the height of the header box
\setlength{\headsep}{0mm}     % Set the distance between the header box and the top of the text

\usepackage{gvv-book}
\usepackage{gvv}
\usepackage{cite}
\usepackage{amsmath,amssymb,amsfonts,amsthm}
\usepackage{algorithmic}
\usepackage{graphicx}
\usepackage{textcomp}
\usepackage{xcolor}
\usepackage{txfonts}
\usepackage{listings}
\usepackage{enumitem}
\usepackage{mathtools}
\usepackage{gensymb}
\usepackage{comment}
\usepackage[breaklinks=true]{hyperref}
\usepackage{tkz-euclide} 
\usepackage{listings}
 \usepackage{gvv}                                        
\def\inputGnumericTable{}                                 
\usepackage[latin1]{inputenc}                                
\usepackage{color}                                            
\usepackage{array}                                            
\usepackage{longtable}                                       
\usepackage{calc}                                             
\usepackage{multirow}                                         
\usepackage{hhline}                                           
\usepackage{ifthen}                                           
\usepackage{lscape}
\begin{document}

\bibliographystyle{IEEEtran}


\title{4.6.9}
\author{EE25BTECH11021 - Dhanush Sagar}

{\let\newpage\relax\maketitle}

\renewcommand{\thefigure}{\theenumi}
\renewcommand{\thetable}{\theenumi}
\setlength{\intextsep}{10pt} % Space between text and floats
\numberwithin{equation}{enumi}
\numberwithin{figure}{enumi}
\renewcommand{\thetable}{\theenumi}
\textbf{Question} \\
Find the equation of the plane containing the two parallel lines 
$\frac{x-1}{2} = \frac{y+1}{-1} = \frac{z}{3}$ and 
$\frac{x}{4} = \frac{y-2}{-2} = \frac{z+1}{6}$. 
Also, determine whether the plane thus obtained contains the line 
$\frac{x-2}{3} = \frac{y-1}{1} = \frac{z-2}{5}$.\\


\textbf{Solution} \\
\noindent A plane can be written in matrix form as 
\(\vec{n}^T(\vec{r} - \vec{P}) = 0\), where \(\vec{n}\) is the normal vector, \(\vec{r}\) is a general point on the plane, and \(\vec{P}\) is a point on the plane:
\begin{align}
\vec{n}^T(\vec{r} - \vec{P}) = 0
\end{align}

\noindent The given lines are in symmetric form:
\begin{align}
\text{L1} &= \frac{x-1}{2} = \frac{y+1}{-1} = \frac{z}{3} \\
\text{L2} &= \frac{x}{4} = \frac{y-2}{-2} = \frac{z+1}{6} \\
\text{L3} &= \frac{x-2}{3} = \frac{y-1}{1} = \frac{z-2}{5}
\end{align}

\noindent Extract points and directions from L1 and L2. The vector joining points from the two lines is:
\begin{align}
\vec{P}_1 = \myvec{1\\-1\\0}, \quad \vec{d}_1 = \myvec{2\\-1\\3}, \quad
\vec{P}_2 = \myvec{0\\2\\-1}, \quad \vec{v} = \vec{P}_2 - \vec{P}_1 = \myvec{-1\\3\\-1}
\end{align}

\noindent The plane's normal vector is orthogonal to both \(\vec{d}_1\) and \(\vec{v}\), so it lies in the nullspace of the constraint matrix:
\begin{align}
\vec{A} = \myvec{2 & -1 & 3 \\ -1 & 3 & -1}
\end{align}

\noindent Row-reduction to find the nullspace:
\begin{align}
\vec{A} &\xrightarrow{R_1 \to \tfrac{1}{2}R_1} \myvec{1 & -\tfrac{1}{2} & \tfrac{3}{2} \\ -1 & 3 & -1} \\[1mm]
\vec{A} &\xrightarrow{R_2 \to R_2 + R_1} \myvec{1 & -\tfrac{1}{2} & \tfrac{3}{2} \\ 0 & \tfrac{5}{2} & \tfrac{1}{2}} \\[1mm]
\vec{A} &\xrightarrow{R_2 \to \tfrac{2}{5}R_2} \myvec{1 & -\tfrac{1}{2} & \tfrac{3}{2} \\ 0 & 1 & \tfrac{1}{5}} \\[1mm]
\vec{A} &\xrightarrow{R_1 \to R_1 + \tfrac{1}{2}R_2} \myvec{1 & 0 & \tfrac{8}{5} \\ 0 & 1 & \tfrac{1}{5}}
\end{align}
Now, we need to form a vector $\vec{n}_0$ whose product with $\vec{A}$ gives a null vector.\\
\noindent Express leading variables in terms of the free variable \(n_3\) to get a vector in the nullspace, which is the plane's normal vector:
\begin{align}
n_1 + \frac{8}{5} n_3 &= 0 \quad \Rightarrow \quad n_1 = -\frac{8}{5} n_3 \\
n_2 + \frac{1}{5} n_3 &= 0 \quad \Rightarrow \quad n_2 = -\frac{1}{5} n_3
\end{align}

\noindent Let \(n_3 = 1\) for simplicity:
\begin{align}
\vec{n}_0 = \myvec{-\frac{8}{5} \\[1mm] -\frac{1}{5} \\[1mm] 1}
\end{align}

\noindent Clearing denominators and adjusting the sign gives the normal vector:
\begin{align}
\vec{n} = \myvec{8\\1\\-5}
\end{align}

\noindent Let \(\vec{r}\) be a general point on the plane:
\begin{align}
\vec{r} = \myvec{r_1\\ r_2\\ r_3}
\end{align}

\noindent The plane equation using point \(\vec{P}_1\) and normal vector \(\vec{n}\):
\begin{align}
\vec{n}^T(\vec{r} - \vec{P}_1) = 0
\end{align}

\begin{align}
\vec{n}^T \vec{P}_1 = \myvec{8 & 1 & -5}\myvec{1\\-1\\0} = 7
\end{align}

\begin{align}
\boxed{\myvec{8 & 1 & -5}\,\vec{r} = 7}
\end{align}

\noindent Check if the third line L3 lies in the plane by verifying the point and direction:
\begin{align}
\vec{P}_3 = \myvec{2\\1\\2}, \quad \vec{d}_3 = \myvec{3\\1\\5} \\
\vec{n}^T \vec{P}_3 = \myvec{8 & 1 & -5}\myvec{2\\1\\2} = 7 \\
\vec{n}^T \vec{d}_3 = \myvec{8 & 1 & -5}\myvec{3\\1\\5} = 0
\end{align}

\noindent Therefore, the plane containing the first two lines has the matrix form:
\[
\myvec{8 & 1 & -5}\,\vec{r} = 7
\]  
and it also contains the third line.

\begin{figure}[H]
    \centering
    \includegraphics[width=0.5\columnwidth]{figs/fig1.png}
    \caption{}
    \label{fig:placeholder}
\end{figure}






\end{document}