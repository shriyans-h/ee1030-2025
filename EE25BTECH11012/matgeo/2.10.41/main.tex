\let\negmedspace\undefined
\let\negthickspace\undefined
\documentclass[journal]{IEEEtran}
\usepackage[a5paper, margin=10mm, onecolumn]{geometry}
%\usepackage{lmodern} % Ensure lmodern is loaded for pdflatex
\usepackage{tfrupee} % Include tfrupee package

\setlength{\headheight}{1cm} % Set the height of the header box
\setlength{\headsep}{0mm}     % Set the distance between the header box and the top of the text

\usepackage{gvv-book}
\usepackage{gvv}
\usepackage{cite}
\usepackage{amsmath,amssymb,amsfonts,amsthm}
\usepackage{algorithmic}
\usepackage{graphicx}
\usepackage{textcomp}
\usepackage{xcolor}
\usepackage{txfonts}
\usepackage{listings}
\usepackage{enumitem}
\usepackage{mathtools}
\usepackage{gensymb}
\usepackage{comment}
\usepackage[breaklinks=true]{hyperref}
\usepackage{tkz-euclide} 
\usepackage{listings}
% \usepackage{gvv}                                        
\def\inputGnumericTable{}                                 
\usepackage[latin1]{inputenc}                                
\usepackage{color}                                            
\usepackage{array}                                            
\usepackage{longtable}                                       
\usepackage{calc}                                             
\usepackage{multirow}                                         
\usepackage{hhline}                                           
\usepackage{ifthen}                                           
\usepackage{lscape}
\begin{document}

\bibliographystyle{IEEEtran}
\vspace{3cm}

\title{2.10.41}
\author{EE25BTECH11012-BEERAM MADHURI}
% \maketitle
% \newpage
% \bigskip
{\let\newpage\relax\maketitle}

\renewcommand{\thefigure}{\theenumi}
\renewcommand{\thetable}{\theenumi}
\setlength{\intextsep}{10pt} % Space between text and floats


\numberwithin{equation}{enumi}
\numberwithin{figure}{enumi}
\renewcommand{\thetable}{\theenumi}


\textbf{Question}:\\
Let the vectors $\mathbf{a}, \mathbf{b}, \mathbf{c}$ and $\mathbf{d}$ be such that $(\mathbf{a} \times \mathbf{b}) \times (\mathbf{c} \times \mathbf{d}) = \mathbf{0}$. Let $A$ and $B$ be planes determined by the pairs of vectors $\mathbf{a}, \mathbf{b}$ and $\mathbf{c}, \mathbf{d}$ respectively. Then the angle between $A$ and $B$ is

\begin{enumerate}
\begin{multicols}{4}
\item[a)] $0$
\item[b)] $\frac{\pi}{4}$
\item[c)] $\frac{\pi}{3}$
\item[d)] $\frac{\pi}{2}$
\end{multicols}
\end{enumerate}
\textbf{Solution}:\\
given, 
\begin{align}
(\mathbf{a} \times \mathbf{b}) \times (\mathbf{c} \times \mathbf{d}) = 0
\end{align}
$\Rightarrow$ angle between $\mathbf{a} \times \mathbf{b}$ and $\mathbf{c} \times \mathbf{d}$ is $0$
\begin{align}
\therefore \mathbf{a} \times \mathbf{b} \parallel \mathbf{c} \times \mathbf{d}
\end{align}
Given that,\\
plane A is determined by $\mathbf{a}, \mathbf{b}$\\
plane B is determined by $\mathbf{c}, \mathbf{d}$\\

normals to planes A and B:
\begin{align}
n_A = \mathbf{a} \times \mathbf{b}\\
n_B = \mathbf{c} \times \mathbf{d}
\end{align}

Angle between Planes A and B = Angle between Normals $n_A$ and $n_B$\\
\begin{align}
\mathbf{a} \times \mathbf{b} \parallel \mathbf{c} \times \mathbf{d}\\
\therefore n_A \parallel n_B\\
\therefore plane A \parallel plane B
\end{align}
Hence, Angle between the planes is $0$.\\option (a).
\begin{figure}
    \centering[H]
    \includegraphics[width=0.75\columnwidth]{figs/graph .jpg}
    \caption{Planes A and B}
    \label{fig:placeholder}
\end{figure}

\end{document}