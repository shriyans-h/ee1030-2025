\let\negmedspace\undefined
\let\negthickspace\undefined
\documentclass[journal]{IEEEtran}
\usepackage[a5paper, margin=10mm, onecolumn]{geometry}
%\usepackage{lmodern} % Ensure lmodern is loaded for pdflatex
\usepackage{tfrupee} % Include tfrupee package

\setlength{\headheight}{1cm} % Set the height of the header box
\setlength{\headsep}{0mm}     % Set the distance between the header box and the top of the text

\usepackage{gvv-book}
\usepackage{gvv}
\usepackage{cite}
\usepackage{amsmath,amssymb,amsfonts,amsthm}
\usepackage{algorithmic}
\usepackage{graphicx}
\usepackage{textcomp}
\usepackage{xcolor}
\usepackage{txfonts}
\usepackage{listings}
\usepackage{enumitem}
\usepackage{mathtools}
\usepackage{gensymb}
\usepackage{comment}
\usepackage[breaklinks=true]{hyperref}
\usepackage{tkz-euclide} 
\usepackage{listings}
% \usepackage{gvv}                                        
\def\inputGnumericTable{}                                 
\usepackage[latin1]{inputenc}                                
\usepackage{color}                                            
\usepackage{array}                                            
\usepackage{longtable}                                       
\usepackage{calc}                                             
\usepackage{multirow}                                         
\usepackage{hhline}                                           
\usepackage{ifthen}                                           
\usepackage{lscape}
\usepackage{multicol}
\begin{document}

\bibliographystyle{IEEEtran}
\vspace{3cm}

\title{12.234}
\author{EE25BTECH11012-BEERAM MADHURI}
% \maketitle
% \newpage
% \bigskip
{\let\newpage\relax\maketitle}

\renewcommand{\thefigure}{\theenumi}
\renewcommand{\thetable}{\theenumi}
\setlength{\intextsep}{10pt} % Space between text and floats


\numberwithin{equation}{enumi}
\numberwithin{figure}{enumi}
\renewcommand{\thetable}{\theenumi}


\textbf{Question}:\\
Consider the set of vectors in three-dimensional real vector space $\mathbb{R}^3$,
\[S = \{(1, 1, 1), (1, -1, 1), (1, 1, -1)\}.\]

Which one of the following statements is true?

\begin{enumerate}
\item[a)] $S$ is not a linearly independent set.
\item[b)] $S$ is a basis for $\mathbb{R}^3$.
\item[c)] The vectors in $S$ are orthogonal.
\item[d)] An orthogonal set of vectors cannot be generated from $S$.
\end{enumerate}
\textbf{Solution:}\\
let the vectors in S be:
\begin{table}[H]
    \centering
    \begin{tabular}{|c|c|c|c|}
    \hline
    Value of $p$ & $\rank(\vec{A})$ & $\rank([\vec{A}|\vec{b}])$ &  Solution Type \\
    \hline
    $p \neq 4$ & 2 & 2 & UNIQUE \\
    $p = 4$ & 1 & 2 & NO SOLUTION \\
    \hline
\end{tabular}
    \caption{Variables used}
    \label{table 12.234}
\end{table}
Let $A$ be the matrix with its columns as vectors of $S$

\begin{align}
A = \begin{bmatrix}1 & 1 & 1 \\1 & -1 & 1 \\1 & 1 & -1\end{bmatrix}
\end{align}
these vectors are linearly independent if and only if
\begin{align}
\det(A) \neq 0
\end{align}
\begin{align}
    det(A)=1(0)-1(-2)+1(2)\\
    =4 \neq 0
\end{align}
$\therefore$ Vectors are linearly independent\\
$\therefore$ Since there are 3 linearly independent vectors in $\mathbb{R}^3$ \\they form a basis for $\mathbb{R}^3$
    
    Let the vector be $\vec{v_1, v_2, v_3}$.
    
    \begin{align}
    \vec{v_1^T v_2} &\neq 0 \\
    \vec{v_1^T v_3 }&\neq 0 \\
    \vec{v_2^T v_3 }&\neq 0
    \end{align}
    
    $\therefore$ These vectors are not orthogonal
    
    Applying Gram-Schmidt process :
    
    let the orthogonal vectors be $\vec{u_1, u_2, u_3}$ generated from $\vec{v_1, v_2, v_3}$
    
    \begin{align}\vec{u_1} = \vec{v_1} = \begin{pmatrix} 1 \\ 1\\1 \end{pmatrix}\\
\vec{u_2} &= \vec{v_2} - (\vec{u_1^T v_2}) \hat{\vec{u}}_1 \\
\vec{u_2} &= \vec{v_2} - \left(\frac{\vec{v_2}^T \vec{u_1}}{\vec{u_1}^T \vec{u_1}} \right) \vec{u_1}\\ &= \begin{pmatrix} 2/3 \\ -4/3 \\ 2/3 \end{pmatrix} 
\end{align}
\begin{align}
\vec{u_3} &= \vec{v_3} - (\hat{\vec{u}}_2^T \vec{v_3}) \hat{\vec{u}}_2 \\&= \vec{v_3} - \left( \frac{\vec{u_2}^T \vec{v_3}}{\vec{u_2}^T \vec{u_2}} \right) \vec{u_2} \\&= \begin{pmatrix} 1 \\ 0 \\ -1 \end{pmatrix}\\
\vec{u_1}^T \vec{u_2} &= 0 \\\vec{u_2}^T \vec{u_3} &= 0 \\\vec{u_1}^T \vec{u_3} &= 0
\end{align}
$\therefore$ an orthogonal set of vectors can be generated from S.\\
$\therefore$ Options b and d are correct.
\end{document}
