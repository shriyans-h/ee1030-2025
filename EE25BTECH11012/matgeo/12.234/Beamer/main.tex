\documentclass{beamer}
\usepackage[utf8]{inputenc}

\usetheme{Madrid}
\usecolortheme{default}
\usepackage{amsmath,amssymb,amsfonts,amsthm}
\usepackage{txfonts}
\usepackage{tkz-euclide}
\usepackage{listings}
\usepackage{adjustbox}
\usepackage{array}
\usepackage{tabularx}
\usepackage{gvv}
\usepackage{lmodern}
\usepackage{circuitikz}
\usepackage{tikz}
\usepackage{graphicx}
\usepackage{multicol}
\setbeamertemplate{page number in head/foot}[totalframenumber]

\usepackage{tcolorbox}
\tcbuselibrary{minted,breakable,xparse,skins}



\definecolor{bg}{gray}{0.95}
\DeclareTCBListing{mintedbox}{O{}m!O{}}{%
  breakable=true,
  listing engine=minted,
  listing only,
  minted language=#2,
  minted style=default,
  minted options={%
    linenos,
    gobble=0,
    breaklines=true,
    breakafter=,,
    fontsize=\small,
    numbersep=8pt,
    #1},
  boxsep=0pt,
  left skip=0pt,
  right skip=0pt,
  left=25pt,
  right=0pt,
  top=3pt,
  bottom=3pt,
  arc=5pt,
  leftrule=0pt,
  rightrule=0pt,
  bottomrule=2pt,

  colback=bg,
  colframe=orange!70,
  enhanced,
  overlay={%
    \begin{tcbclipinterior}
    \fill[orange!20!white] (frame.south west) rectangle ([xshift=20pt]frame.north west);
    \end{tcbclipinterior}},
  #3,
}
\lstset{
    language=C,
    basicstyle=\ttfamily\small,
    keywordstyle=\color{blue},
    stringstyle=\color{orange},
    commentstyle=\color{green!60!black},
    numbers=left,
    numberstyle=\tiny\color{gray},
    breaklines=true,
    showstringspaces=false,
}
%------------------------------------------------------------
%This block of code defines the information to appear in the
%Title page
\title %optional
{12.234}
\date{October  2025}
%\subtitle{A short story}

\author % (optional)
{BEERAM MADHURI - EE25BTECH11012}



\begin{document}


\frame{\titlepage}
\begin{frame}{Question-12.234}
Consider the set of vectors in three-dimensional real vector space $\mathbb{R}^3$,
\[S = \{(1, 1, 1), (1, -1, 1), (1, 1, -1)\}.\]

Which one of the following statements is true?

\begin{enumerate}
\item[a)] $S$ is not a linearly independent set.
\item[b)] $S$ is a basis for $\mathbb{R}^3$.
\item[c)] The vectors in $S$ are orthogonal.
\item[d)] An orthogonal set of vectors cannot be generated from $S$.
\end{enumerate}
\end{frame}
 
\begin{frame}{given data}
let the vectors in S be:
\begin{table}[h!]
    \centering
    \begin{tabular}{|c|c|c|c|}
    \hline
    Value of $p$ & $\rank(\vec{A})$ & $\rank([\vec{A}|\vec{b}])$ &  Solution Type \\
    \hline
    $p \neq 4$ & 2 & 2 & UNIQUE \\
    $p = 4$ & 1 & 2 & NO SOLUTION \\
    \hline
\end{tabular}
    \caption{Variables used}
    \label{table 12.234}
\end{table}
\end{frame}

\begin{frame}{solution}
    \frametitle{finding the properties of S:}
Let $A$ be the matrix with its columns as vectors of $S$

\begin{align}
A = \begin{pmatrix}1 & 1 & 1 \\1 & -1 & 1 \\1 & 1 & -1\end{pmatrix}
\end{align}
\textbf{Option a:}\\
The column vectors of a matrix $A$ are \textbf{linearly independent} if and only if the equation
\begin{align}
A\vec{x} = \vec{0}\\
\text{has only the trivial solution}\\
(\vec{x} = \vec{0})
\end{align}
\end{frame}
\begin{frame}
We can find the solution by guassian elimination of $A$.
\begin{align}
\text{Perform row operations to get it into row echelon form:}\\\\
\begin{pmatrix}1 & 1 & 1 \\1 & -1 & 1 \\1 & 1 & -1\end{pmatrix}\xrightarrow[
R_2 \to R_2 - R_1]{R_3 \to R_3 - R_1}
    \begin{pmatrix} 1 & 1 & 1 \\ 0 & -2 & 0 \\ 0 & 0 & -2 \end{pmatrix}\\\\
\begin{pmatrix} 1 & 1 & 1 \\ 0 & -2 & 0 \\ 0 & 0 & -2 \end{pmatrix}\xrightarrow[
R_2 \to R_2 / -2]{R_3 \to R_3 / -2}
    \begin{pmatrix} 1 & 1 & 1 \\ 0 & 1 & 0 \\ 0 & 0 & 1 \end{pmatrix}\\\\
 \begin{pmatrix} 1 & 1 & 1 \\ 0 & 1 & 0 \\ 0 & 0 & 1 \end{pmatrix}\xrightarrow{
    R_1 \to R_1 - R_3}
    \begin{pmatrix} 1 & 1 & 0 \\ 0 & 1 & 0 \\ 0 & 0 & 1 \end{pmatrix}\xrightarrow{  R_1 \to R_1 - R_2}
    \begin{pmatrix} 1 & 0 & 0 \\ 0 & 1 & 0 \\ 0 & 0 & 1 \end{pmatrix}
    \end{align}
\end{frame}
\begin{frame}
The reduced row echelon form of $A$ is the Identity matrix $I$.\\\\
$\therefore$ The only possible solution is the trivialsolution:
$\vec{x}=0$\\
$\therefore$ Vectors are linearly independent\\\\
\textbf{Option b:}\\
Since there are 3 linearly independent vectors in $\mathbb{R}^3$ \\they form a basis for $\mathbb{R}^3$
\end{frame}
\begin{frame}
\textbf{Option c:}\\
    Let the vector be $\vec{v_1, v_2, v_3}$.
    \begin{align}
    \vec{v_1^\top v_2} &\neq 0 \\
    \vec{v_1^\top v_3 }&\neq 0 \\
    \vec{v_2^\top v_3 }&\neq 0
    \end{align}
    $\therefore$ These vectors are not orthogonal
\end{frame}
\begin{frame}
\textbf{Option d:}\\
\text{Applying Gram-Schmidt process:} \\
\text{let the orthogonal vectors be $\vec{u_1, u_2, u_3}$ generated from $\vec{v_1, v_2, v_3}$}
\begin{align}\vec{u_1} = \vec{v_1} = \begin{pmatrix} 1 \\ 1\\1 \end{pmatrix}\\
\vec{u_2} &= \vec{v_2} - (\vec{u_1^\top v_2}) \hat{\vec{u}}_1 \\
\vec{u_2} &= \vec{v_2} - \left(\frac{\vec{v_2}^\top \vec{u_1}}{\vec{u_1}^\top \vec{u_1}} \right) \vec{u_1}\\ &= \begin{pmatrix} 2/3 \\ -4/3 \\ 2/3 \end{pmatrix} 
\end{align}
\end{frame}
\begin{frame}
\begin{align}
\vec{u_3}= \vec{v_3} - (\hat{\vec{u}}_2^\top \vec{v_3}) \hat{\vec{u}}_2 \\&= \vec{v_3} - \left( \frac{\vec{u_2}^\top \vec{v_3}}{\vec{u_2}^\top \vec{u_2}} \right) \vec{u_2} \\&= \begin{pmatrix} 1 \\ 0 \\ -1 \end{pmatrix}\\
\vec{u_1}^\top \vec{u_2} &= 0 \\\vec{u_2}^\top \vec{u_3} &= 0 \\\vec{u_1}^\top \vec{u_3} &= 0
\end{align}
$\therefore$ an orthogonal set of vectors can be generated from S.\\
$\therefore$ Options b and d are correct.
\end{frame}

\begin{frame}[fragile]
    \frametitle{Python Code}
    \begin{lstlisting}
import numpy as np

# Define the vectors
v1 = np.array([1, 1, 1])
v2 = np.array([1, -1, 1])
v3 = np.array([1, 1, -1])

# Form the matrix with given vectors as columns
A = np.column_stack((v1, v2, v3))
\end{lstlisting}
\end{frame}

\begin{frame}[fragile]
\frametitle{Python Code}
\begin{lstlisting}
# 1. Check Linear Independence using determinant
det_A = np.linalg.det(A)
if abs(det_A) > 1e-9:
    print("S is a linearly independent set.")
else:
    print("S is not a linearly independent set.")
\end{lstlisting}
\end{frame}

\begin{frame}[fragile]
\frametitle{Python Code}
\begin{lstlisting}
# 2. Check if S is a basis for R³
if A.shape == (3,3) and abs(det_A) > 1e-9:
    print("S is a basis for R³.")
else:
    print("S is not a basis for R³.")

# 3. Check if vectors are orthogonal
def is_orthogonal(u, v):
    return np.dot(u, v) == 0
\end{lstlisting}
\end{frame}

\begin{frame}[fragile]
\frametitle{Python Code}
\begin{lstlisting}
print("Dot products:")
print("v1·v2 =", np.dot(v1, v2))
print("v1·v3 =", np.dot(v1, v3))
print("v2·v3 =", np.dot(v2, v3))
if is_orthogonal(v1, v2) and is_orthogonal(v1, v3) and is_orthogonal(v2, v3):
    print("The vectors in S are orthogonal.")
else:
    print("The vectors in S are not orthogonal.")
\end{lstlisting}
\end{frame}

\begin{frame}[fragile]
\frametitle{Python Code}
\begin{lstlisting}
# 4. Generate an orthogonal set using Gram-Schmidt process
def gram_schmidt(vectors):
    ortho = []
    for v in vectors:
        for u in ortho:
            v = v - np.dot(v, u) / np.dot(u, u) * u
        ortho.append(v)
    return ortho
orthogonal_set = gram_schmidt([v1, v2, v3])
print("\nOrthogonal set generated using Gram-Schmidt:")
for vec in orthogonal_set:
    print(vec)
\end{lstlisting}
\end{frame}

\begin{frame}[fragile]
\frametitle{C Code}
\begin{lstlisting}
#include <stdio.h>
#include <math.h>

// Function to calculate determinant of 3x3 matrix
float determinant(float a[3][3]) {
    float det;
    det = a[0][0]*(a[1][1]*a[2][2] - a[1][2]*a[2][1])
        - a[0][1]*(a[1][0]*a[2][2] - a[1][2]*a[2][0])
        + a[0][2]*(a[1][0]*a[2][1] - a[1][1]*a[2][0]);
    return det;
}
\end{lstlisting}
\end{frame}

\begin{frame}[fragile]
\frametitle{C Code}
\begin{lstlisting}
// Function to compute dot product of two 3D vectors
float dot_product(float a[3], float b[3]) {
    return a[0]*b[0] + a[1]*b[1] + a[2]*b[2];
}
int main() {
    float v1[3] = {1, 1, 1};
    float v2[3] = {1, -1, 1};
    float v3[3] = {1, 1, -1};
\end{lstlisting}
\end{frame}

\begin{frame}[fragile]
\frametitle{C Code}
\begin{lstlisting}
    // Form matrix with vectors as columns
    float A[3][3] = {
        {v1[0], v2[0], v3[0]},
        {v1[1], v2[1], v3[1]},
        {v1[2], v2[2], v3[2]}
    };
    // 1. Check linear independence using determinant
    float det = determinant(A);
    printf("Determinant = %.2f\n", det);
\end{lstlisting}
\end{frame}

\begin{frame}[fragile]
\frametitle{C Code}
\begin{lstlisting}
    if (fabs(det) > 1e-6)
        printf("S is a linearly independent set.\n");
    else
        printf("S is NOT a linearly independent set.\n");
    // 2. Check if it is a basis for R^3
    if (fabs(det) > 1e-6)
        printf("S is a basis for R^3.\n");
    else
        printf("S is NOT a basis for R^3.\n");
\end{lstlisting}
\end{frame}

\begin{frame}[fragile]
\frametitle{C Code}
\begin{lstlisting}
    // 3. Check orthogonality
    float d12 = dot_product(v1, v2);
    float d13 = dot_product(v1, v3);
    float d23 = dot_product(v2, v3);

    printf("\nDot products:\n");
    printf("v1·v2 = %.2f\n", d12);
    printf("v1·v3 = %.2f\n", d13);
    printf("v2·v3 = %.2f\n", d23);
\end{lstlisting}
\end{frame}

\begin{frame}[fragile]
\frametitle{C Code}
\begin{lstlisting}
    if (fabs(d12) < 1e-6 && fabs(d13) < 1e-6 && fabs(d23) < 1e-6)
        printf("Vectors are orthogonal.\n");
    else
        printf("Vectors are NOT orthogonal.\n");
    // 4. Based on results, print the correct option
    printf("\nCorrect statement:\n");
    if (fabs(det) > 1e-6 && !(fabs(d12) < 1e-6 && fabs(d13) < 1e-6 && fabs(d23) < 1e-6))
        printf("Option (b): S is a basis for R^3.\n");
\end{lstlisting}
\end{frame}

\begin{frame}[fragile]
\frametitle{C Code}
\begin{lstlisting}
    else if (fabs(det) < 1e-6)
        printf("Option (a): S is not a linearly independent set.\n");
    else if (fabs(d12) < 1e-6 && fabs(d13) < 1e-6 && fabs(d23) < 1e-6)
        printf("Option (c): The vectors in S are orthogonal.\n");
    else
        printf("Option (d): An orthogonal set cannot be generated from S.\n");
    return 0;}
\end{lstlisting}
\end{frame}

\begin{frame}[fragile]
\frametitle{Python and C Code}
\begin{lstlisting}
import ctypes
import math
# Define C-like types
c_float = ctypes.c_float
c_float_p = ctypes.POINTER(c_float)
# Define C-like array types
# float[3] is analogous to C float v[3]
FloatArray3 = c_float * 3
# float[3][3] is analogous to C float A[3][3]
FloatMatrix3x3 = (c_float * 3) * 3
\end{lstlisting}
\end{frame}

\begin{frame}[fragile]
\frametitle{Python and C Code}
\begin{lstlisting}
# Function to calculate determinant of 3x3 matrix
# Takes a 3x3 array/matrix of c_float
def determinant(a):
    """Calculates the determinant of a 3x3 matrix represented by a ctypes array."""
    # Access elements using a[row][col]
    det = a[0][0] * (a[1][1] * a[2][2] - a[1][2] * a[2][1]) \
        - a[0][1] * (a[1][0] * a[2][2] - a[1][2] * a[2][0]) \
        + a[0][2] * (a[1][0] * a[2][1] - a[1][1] * a[2][0])
    return det.value if isinstance(det, c_float) else det
\end{lstlisting}
\end{frame}

\begin{frame}[fragile]
\frametitle{Python and C Code}
\begin{lstlisting}
# Function to compute dot product of two 3D vectors
# Takes two 3-element arrays/vectors of c_float
def dot_product(a, b):
    """Computes the dot product of two 3D vectors represented by ctypes arrays."""
    # Access elements using a[index]
    result = a[0] * b[0] + a[1] * b[1] + a[2] * b[2]
    return result.value if isinstance(result, c_float) else result
\end{lstlisting}
\end{frame}

\begin{frame}[fragile]
\frametitle{Python and C Code}
\begin{lstlisting}
# Main execution block
def main():
    # Define vectors using the C-like FloatArray3 type
    v1 = FloatArray3(1.0, 1.0, 1.0)
    v2 = FloatArray3(1.0, -1.0, 1.0)
    v3 = FloatArray3(1.0, 1.0, -1.0)
    # Define the tolerance (epsilon) used for floating-point comparisons
    TOLERANCE = 1e-6
\end{lstlisting}
\end{frame}

\begin{frame}[fragile]
\frametitle{Python and C Code}
\begin{lstlisting}  
    # Form matrix A with vectors as columns using the C-like FloatMatrix3x3 type
    A = FloatMatrix3x3(
        FloatArray3(v1[0], v2[0], v3[0]), # Row 0: x-components
        FloatArray3(v1[1], v2[1], v3[1]), # Row 1: y-components
        FloatArray3(v1[2], v2[2], v3[2])  # Row 2: z-components
    )

    print("--- Linear Algebra Calculations with ctypes ---")
\end{lstlisting}
\end{frame}

\begin{frame}[fragile]
\frametitle{Python and C Code}
\begin{lstlisting}  
    # 1. Check linear independence using determinant
    det = determinant(A)
    print(f"Determinant = {det:.2f}")

    if abs(det) > TOLERANCE:
        print("S is a linearly independent set.")
    else:
        print("S is NOT a linearly independent set.")
\end{lstlisting}
\end{frame}

\begin{frame}[fragile]
\frametitle{Python and C Code}
\begin{lstlisting}
    # 2. Check if it is a basis for R^3
    if abs(det) > TOLERANCE:
        print("S is a basis for R^3.")
    else:
        print("S is NOT a basis for R^3.")
    # 3. Check orthogonality
    d12 = dot_product(v1, v2)
    d13 = dot_product(v1, v3)
    d23 = dot_product(v2, v3)
\end{lstlisting}
\end{frame}

\begin{frame}[fragile]
\frametitle{Python and C Code}
\begin{lstlisting}
    print("\nDot products:")
    print(f"v1·v2 = {d12:.2f}")
    print(f"v1·v3 = {d13:.2f}")
    print(f"v2·v3 = {d23:.2f}")
    is_orthogonal = abs(d12) < TOLERANCE and abs(d13) < TOLERANCE and abs(d23) < TOLERANCE
    if is_orthogonal:
        print("Vectors are orthogonal.")
    else:
        print("Vectors are NOT orthogonal.")
\end{lstlisting}
\end{frame}

\begin{frame}[fragile]
\frametitle{Python and C Code}
\begin{lstlisting}
    # 4. Based on results, print the correct option
    print("\nCorrect statement:")
    if abs(det) > TOLERANCE and not is_orthogonal:
        print("Option (b): S is a basis for R^3.")
    elif abs(det) < TOLERANCE:
        print("Option (a): S is not a linearly independent set.")
\end{lstlisting}
\end{frame}

\begin{frame}[fragile]
\frametitle{Python and C Code}
\begin{lstlisting}
    elif is_orthogonal:
        print("Option (c): The vectors in S are orthogonal.")
    else:
        # This else block should theoretically not be reached if the previous logic is exhaustive
        print("Option (d): An orthogonal set cannot be generated from S.")

if __name__ == "__main__":
    main()
\end{lstlisting}
\end{frame}

\end{document}