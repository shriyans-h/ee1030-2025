\documentclass{beamer}
\usepackage[utf8]{inputenc}

\usetheme{Madrid}
\usecolortheme{default}
\usepackage{amsmath,amssymb,amsfonts,amsthm}
\usepackage{txfonts}
\usepackage{tkz-euclide}
\usepackage{listings}
\usepackage{adjustbox}
\usepackage{array}
\usepackage{tabularx}
\usepackage{gvv}
\usepackage{lmodern}
\usepackage{circuitikz}
\usepackage{tikz}
\usepackage{graphicx}
\usepackage{multicol}
\setbeamertemplate{page number in head/foot}[totalframenumber]

\usepackage{tcolorbox}
\tcbuselibrary{minted,breakable,xparse,skins}



\definecolor{bg}{gray}{0.95}
\DeclareTCBListing{mintedbox}{O{}m!O{}}{%
  breakable=true,
  listing engine=minted,
  listing only,
  minted language=#2,
  minted style=default,
  minted options={%
    linenos,
    gobble=0,
    breaklines=true,
    breakafter=,,
    fontsize=\small,
    numbersep=8pt,
    #1},
  boxsep=0pt,
  left skip=0pt,
  right skip=0pt,
  left=25pt,
  right=0pt,
  top=3pt,
  bottom=3pt,
  arc=5pt,
  leftrule=0pt,
  rightrule=0pt,
  bottomrule=2pt,

  colback=bg,
  colframe=orange!70,
  enhanced,
  overlay={%
    \begin{tcbclipinterior}
    \fill[orange!20!white] (frame.south west) rectangle ([xshift=20pt]frame.north west);
    \end{tcbclipinterior}},
  #3,
}
\lstset{
    language=C,
    basicstyle=\ttfamily\small,
    keywordstyle=\color{blue},
    stringstyle=\color{orange},
    commentstyle=\color{green!60!black},
    numbers=left,
    numberstyle=\tiny\color{gray},
    breaklines=true,
    showstringspaces=false,
}
%------------------------------------------------------------
%This block of code defines the information to appear in the
%Title page
\title %optional
{12.338}
\date{October  2025}
%\subtitle{A short story}

\author % (optional)
{BEERAM MADHURI - EE25BTECH11012}



\begin{document}


\frame{\titlepage}
\begin{frame}{Question}
For a real symmetric matrix $\mathbf{A}$, which of the following statements is true?

\begin{enumerate}
    \item[a)] The matrix is always diagonalizable and invertible.
    \item[b)] The matrix is always invertible but not necessarily diagonalizable.
    \item[c)] The matrix is always diagonalizable but not necessarily invertible.
    \item[d)] The matrix is always neither diagonalizable nor invertible.
\end{enumerate}
\end{frame}

\begin{frame}{solution}
    \frametitle{finding the properties of matrix A:}
Checking for diagonalizability of matrix $A$ \\
given,
\begin{align}
\vec{A} = \vec{A}^\top
\end{align}
$\therefore$ eigenvalues of $\vec{A}$ are real.\\
for distinct eigenvalues $\lambda_i$, $\lambda_j$ corresponding eigenvectors are $\vec{x_i}$, $\vec{x_j}$.
\end{frame}
\begin{frame}
\begin{align}
\vec{Ax_i} &= \lambda_i \vec{x_i} \quad \text{and} \quad \vec{Ax_j} = \lambda_j \vec{x_j} 
\\\vec{x_j}^\top \vec{A x_i} &= \lambda_i \vec{x_j}^\top \vec{x_i} \\
\vec{(Ax_j)}^\top \vec{x_i} &= \lambda_i \vec{x_j}^\top \vec{x_i}\\
\because \quad \vec{Ax_j} &= \lambda_j \vec{x_j}\\
\lambda_j \vec{x_j}^\top \vec{x_i} = \lambda_i \vec{x_j}^\top \vec{x_i}\\
(\lambda_j - \lambda_i) \vec{x_j}^\top \vec{x_i} = 0
\end{align}
$\therefore$ \text{eigenvectors are orthogonal}\\
\end{frame}
\begin{frame}
$\therefore$ \text{We can construct an orthogonal matrix with these eigenvectors}\\
\begin{align}
Q = [\vec{x_1} \ \vec{x_2} \ \vec{x_3} \ \dots \ \vec{x_n}] \\
Q^\top Q = I\\
A = Q MQ^\top
\end{align}
\text{Where $\vec{M}$ is diagonal matrix}\\
$\therefore$ $\vec{A}$ is always diagonalizable.\\\\
\end{frame}
\begin{frame}
Checking for invertibility of Matrix  $\vec{A}$:
\begin{align}
\vec{A} = Q M Q^\top \\|A| = |Q| |M| |Q^\top|
|\vec{A}| = M_1 M_2 \cdots M_n
\end{align}
where $M_1, M_2, \cdots M_n$ are diagonal entries of Matrix M.\\
A is invertible only when 
\begin{align}
\det(A) \neq 0 
\end{align}
that is $M_1, M_2, M_3 \cdots M_n \neq 0$ \\
that is none of its eigenvalues are zero \\
\end{frame}
\begin{frame}
if $\lambda_i = 0$ \\
then $A$ is non-invertible \\

$\therefore$ a real symmetric matrix may or may not be invertible.\\
$\therefore$ Option c is correct.\\\\
Example of a real symmetric matrix $\vec{A}$:
\begin{align}
    \vec{A}=\begin{pmatrix}
        1 & 1\\1& 1
    \end{pmatrix}\\
    \vec{A}=\vec{A^\top}
\end{align}
$\vec{A}$ is symmetric and diagonalizable but not invertible as det(\vec{A}) $=0$
\end{frame}


\begin{frame}[fragile]
\frametitle{Python Code}
\begin{lstlisting}
import numpy as np

# --- Example 1: A real symmetric matrix that IS invertible ---
matrix_A = np.array([
    [3, 1],
    [1, 2]
])
print("## Matrix A ##")
print(matrix_A)
\end{lstlisting}
\end{frame}

\begin{frame}[fragile]
\frametitle{Python Code}
\begin{lstlisting}
# Check for symmetry: A == A.transpose()
is_symmetric_A = np.all(matrix_A == matrix_A.T)
print(f"Is symmetric? {is_symmetric_A}")

# Check for invertibility by calculating the determinant
det_A = np.linalg.det(matrix_A)
print(f"Determinant: {det_A:.2f}")
print(f"Is invertible? {det_A != 0}")
\end{lstlisting}
\end{frame}

\begin{frame}[fragile]
\frametitle{Python Code}
\begin{lstlisting}
print("-" * 20)

# --- Example 2: A real symmetric matrix that is NOT invertible ---
matrix_B = np.array([
    [2, 4],
    [4, 8]
])
print("## Matrix B ##")
print(matrix_B)
\end{lstlisting}
\end{frame}

\begin{frame}[fragile]
\frametitle{Python Code}
\begin{lstlisting}
# Check for symmetry
is_symmetric_B = np.all(matrix_B == matrix_B.T)
print(f"Is symmetric? {is_symmetric_B}")

# Check for invertibility
det_B = np.linalg.det(matrix_B)
print(f"Determinant: {det_B:.2f}")
print(f"Is invertible? {det_B != 0}")
\end{lstlisting}
\end{frame}

\begin{frame}[fragile]
\frametitle{C Code}
\begin{lstlisting}
#include <stdio.h>
#include <stdbool.h>

// Define a 2x2 matrix structure
typedef struct {
    double elements[2][2];
} Matrix2x2;

// Function to print a 2x2 matrix
\end{lstlisting}
\end{frame}

\begin{frame}[fragile]
\frametitle{C Code}
\begin{lstlisting}
void printMatrix(Matrix2x2 m) {
    for (int i = 0; i < 2; i++) {
        for (int j = 0; j < 2; j++) {
            printf("%8.2f", m.elements[i][j]);
        }
        printf("\n");
    }
}
// Function to check if a 2x2 matrix is symmetric
// A matrix A is symmetric if A = A^T (its transpose)
// For a 2x2 matrix, this just means element [0][1] must equal element [1][0]
\end{lstlisting}
\end{frame}

\begin{frame}[fragile]
\frametitle{C Code}
\begin{lstlisting}
bool isSymmetric(Matrix2x2 m) {
    if (m.elements[0][1] == m.elements[1][0]) {
        return true;
    }
    return false;
}
// Function to calculate the determinant of a 2x2 matrix
// For a matrix [[a, b], [c, d]], the determinant is ad - bc
double determinant(Matrix2x2 m) {
    return (m.elements[0][0] * m.elements[1][1]) - (m.elements[0][1] * m.elements[1][0]);
}
\end{lstlisting}
\end{frame}

\begin{frame}[fragile]
\frametitle{C Code}
\begin{lstlisting}
int main() {
    // Example 1: A real symmetric matrix that IS invertible
    Matrix2x2 matrixA = {
        {{3.0, 1.0}, {1.0, 2.0}}
    };
    printf("## Matrix A ##\n");
    printMatrix(matrixA);
    printf("Is symmetric? %s\n", isSymmetric(matrixA) ? "Yes" : "No");
\end{lstlisting}
\end{frame}

\begin{frame}[fragile]
\frametitle{C Code}
\begin{lstlisting}
    double detA = determinant(matrixA);
    printf("Determinant: %.2f\n", detA);
    printf("Is invertible? %s\n\n", (detA != 0) ? "Yes" : "No");
    // Example 2: A real symmetric matrix that is NOT invertible
    Matrix2x2 matrixB = {
        {{2.0, 4.0}, {4.0, 8.0}}
    };
\end{lstlisting}
\end{frame}

\begin{frame}[fragile]
\frametitle{C Code}
\begin{lstlisting}
    printf("## Matrix B ##\n");
    printMatrix(matrixB);
    printf("Is symmetric? %s\n", isSymmetric(matrixB) ? "Yes" : "No");
    double detB = determinant(matrixB);
    printf("Determinant: %.2f\n", detB);
    printf("Is invertible? %s\n", (detB != 0) ? "Yes" : "No");
    return 0;
}
\end{lstlisting}
\end{frame}

\begin{frame}[fragile]
\frametitle{Python and C Code}
\begin{lstlisting}
import ctypes

# Define a 2x2 matrix structure that is compatible with the C struct
class Matrix2x2(ctypes.Structure):
    """A C-compatible 2x2 matrix structure."""
    fields = [
        ("elements", (ctypes.c_double * 2) * 2)
    ]
# --- Python functions that operate on the C-like structure ---
\end{lstlisting}
\end{frame}

\begin{frame}[fragile]
\frametitle{Python and C Code}
\begin{lstlisting}
def print_matrix(m: Matrix2x2):
    """Prints the elements of a Matrix2x2 structure."""
    for i in range(2):
        for j in range(2):
            # Use an f-string for formatted output, similar to printf
            print(f"{m.elements[i][j]:8.2f}", end="")
        print()
\end{lstlisting}
\end{frame}

\begin{frame}[fragile]
\frametitle{Python and C Code}
\begin{lstlisting}
def is_symmetric(m: Matrix2x2) -> bool:
    """
    Checks if a 2x2 matrix is symmetric.
    A matrix A is symmetric if element [0][1] equals element [1][0].
    """
    return m.elements[0][1] == m.elements[1][0]
\end{lstlisting}
\end{frame}

\begin{frame}[fragile]
\frametitle{Python and C Code}
\begin{lstlisting}
def determinant(m: Matrix2x2) -> float:
    """
    Calculates the determinant of a 2x2 matrix.
    For a matrix [[a, b], [c, d]], the determinant is ad - bc.
    """
    return (m.elements[0][0] * m.elements[1][1]) - (m.elements[0][1] * m.elements[1][0])
\end{lstlisting}
\end{frame}

\begin{frame}[fragile]
\frametitle{Python and C Code}
\begin{lstlisting}
if _name_ == "_main_":
    # Example 1: A real symmetric matrix that IS invertible
    # Instantiate the structure using nested tuples for the 2D array
    matrix_a = Matrix2x2(elements=((3.0, 1.0), (1.0, 2.0)))
    print("## Matrix A ##")
    print_matrix(matrix_a)
    # Use a ternary operator in the f-string to mimic C's output
    print(f"Is symmetric? {'Yes' if is_symmetric(matrix_a) else 'No'}")
\end{lstlisting}
\end{frame}

\begin{frame}[fragile]
\frametitle{Python and C Code}
\begin{lstlisting}
    det_a = determinant(matrix_a)
    print(f"Determinant: {det_a:.2f}")
    print(f"Is invertible? {'Yes' if det_a != 0 else 'No'}\n")


    # Example 2: A real symmetric matrix that is NOT invertible
    matrix_b = Matrix2x2(elements=((2.0, 4.0), (4.0, 8.0)))
\end{lstlisting}
\end{frame}

\begin{frame}[fragile]
\frametitle{Python and C Code}
\begin{lstlisting}
    print("## Matrix B ##")
    print_matrix(matrix_b)
    print(f"Is symmetric? {'Yes' if is_symmetric(matrix_b) else 'No'}")

    det_b = determinant(matrix_b)
    print(f"Determinant: {det_b:.2f}")
    print(f"Is invertible? {'Yes' if det_b != 0 else 'No'}")
\end{lstlisting}
\end{frame}

\end{document}