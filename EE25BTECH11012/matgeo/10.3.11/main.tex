\let\negmedspace\undefined
\let\negthickspace\undefined
\documentclass[journal]{IEEEtran}
\usepackage[a5paper, margin=10mm, onecolumn]{geometry}
%\usepackage{lmodern} % Ensure lmodern is loaded for pdflatex
\usepackage{tfrupee} % Include tfrupee package

\setlength{\headheight}{1cm} % Set the height of the header box
\setlength{\headsep}{0mm}     % Set the distance between the header box and the top of the text

\usepackage{gvv-book}
\usepackage{gvv}
\usepackage{cite}
\usepackage{amsmath,amssymb,amsfonts,amsthm}
\usepackage{algorithmic}
\usepackage{graphicx}
\usepackage{textcomp}
\usepackage{xcolor}
\usepackage{txfonts}
\usepackage{listings}
\usepackage{enumitem}
\usepackage{mathtools}
\usepackage{gensymb}
\usepackage{comment}
\usepackage[breaklinks=true]{hyperref}
\usepackage{tkz-euclide} 
\usepackage{listings}
% \usepackage{gvv}                                        
\def\inputGnumericTable{}                                 
\usepackage[latin1]{inputenc}                                
\usepackage{color}                                            
\usepackage{array}                                            
\usepackage{longtable}                                       
\usepackage{calc}                                             
\usepackage{multirow}                                         
\usepackage{hhline}                                           
\usepackage{ifthen}                                           
\usepackage{lscape}
\usepackage{multicol}
\begin{document}

\bibliographystyle{IEEEtran}
\vspace{3cm}

\title{10.3.11}
\author{EE25BTECH11012-BEERAM MADHURI}
% \maketitle
% \newpage
% \bigskip
{\let\newpage\relax\maketitle}

\renewcommand{\thefigure}{\theenumi}
\renewcommand{\thetable}{\theenumi}
\setlength{\intextsep}{10pt} % Space between text and floats


\numberwithin{equation}{enumi}
\numberwithin{figure}{enumi}
\renewcommand{\thetable}{\theenumi}


\textbf{Question}:\\
Find the normal at the point $(1,1)$ on the curve
\begin{align}
2y + x^2 = 3
\end{align}
\textbf{Solution:}\\
Let $F = 2y + x^2 - 3 = 0$

gradient Vector is:
\begin{align}
\begin{bmatrix}
\frac{\partial F}{\partial x} \\
\frac{\partial F}{\partial y}
\end{bmatrix} = \begin{bmatrix}
2x \\
2
\end{bmatrix}
\end{align}

Normal Vector at $(1,1)$ is:
\begin{align}
n = \begin{bmatrix}
2 \\
2
\end{bmatrix}
\end{align}
let $m$ be tangent vector,
\begin{align}
\text{if } n = \begin{bmatrix}
a \\
b
\end{bmatrix}\\
\text{then } m = \begin{bmatrix}
-b \\
a
\end{bmatrix}
\end{align}
\begin{align}
\therefore m = \begin{bmatrix}
-2 \\
2
\end{bmatrix}
\end{align}
let 
\begin{align}
p = \begin{bmatrix}
x \\
y
\end{bmatrix} \& p_0 = \begin{bmatrix}
1 \\
1
\end{bmatrix}
\end{align}

\begin{align}
m^\top (P - P_0) = 0
\end{align}

Substituting the values:-
\begin{align}
\begin{bmatrix}-2 & 2\end{bmatrix}\begin{bmatrix}x-1 \\y-1\end{bmatrix}= 0\\
-2(x-1) + 2(y-1) = 0\\
y = x
\end{align}

Hence equation of normal to $2y + x^2 - 3 = 0$ at $(1,1)$ is $y = x$.

\begin{figure}[H]
    \centering
    \includegraphics[width=0.85\columnwidth]{figs/graph-17.png}
    \caption{10.3.11}
    \label{fig:placeholder}
\end{figure}
\end{document}
