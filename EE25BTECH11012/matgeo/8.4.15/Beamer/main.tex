\documentclass{beamer}
\usepackage[utf8]{inputenc}

\usetheme{Madrid}
\usecolortheme{default}
\usepackage{amsmath,amssymb,amsfonts,amsthm}
\usepackage{txfonts}
\usepackage{tkz-euclide}
\usepackage{listings}
\usepackage{adjustbox}
\usepackage{array}
\usepackage{tabularx}
\usepackage{gvv}
\usepackage{lmodern}
\usepackage{circuitikz}
\usepackage{tikz}
\usepackage{graphicx}
\usepackage{multicol}
\setbeamertemplate{page number in head/foot}[totalframenumber]

\usepackage{tcolorbox}
\tcbuselibrary{minted,breakable,xparse,skins}



\definecolor{bg}{gray}{0.95}
\DeclareTCBListing{mintedbox}{O{}m!O{}}{%
  breakable=true,
  listing engine=minted,
  listing only,
  minted language=#2,
  minted style=default,
  minted options={%
    linenos,
    gobble=0,
    breaklines=true,
    breakafter=,,
    fontsize=\small,
    numbersep=8pt,
    #1},
  boxsep=0pt,
  left skip=0pt,
  right skip=0pt,
  left=25pt,
  right=0pt,
  top=3pt,
  bottom=3pt,
  arc=5pt,
  leftrule=0pt,
  rightrule=0pt,
  bottomrule=2pt,

  colback=bg,
  colframe=orange!70,
  enhanced,
  overlay={%
    \begin{tcbclipinterior}
    \fill[orange!20!white] (frame.south west) rectangle ([xshift=20pt]frame.north west);
    \end{tcbclipinterior}},
  #3,
}
\lstset{
    language=C,
    basicstyle=\ttfamily\small,
    keywordstyle=\color{blue},
    stringstyle=\color{orange},
    commentstyle=\color{green!60!black},
    numbers=left,
    numberstyle=\tiny\color{gray},
    breaklines=true,
    showstringspaces=false,
}
%------------------------------------------------------------
%This block of code defines the information to appear in the
%Title page
\title %optional
{8.4.15}
\date{October  2025}
%\subtitle{A short story}

\author % (optional)
{BEERAM MADHURI - EE25BTECH11012}



\begin{document}


\frame{\titlepage}
\begin{frame}{Question}
In a triangle $ABC$ with fixed base $BC$, the vertex A moves such that
\begin{align*}
\cos B + \cos C = 4 \sin^2 \frac{A}{2}.
\end{align*}
If $a, b$ and $c$ denote the lengths of the sides of the triangle opposite to the angles $A, B$ and $C$, respectively, then

\begin{enumerate}
    \item[a)] $b + c = 4a$
    \item[b)] $b + c = 2a$
    \item[c)] locus of the point A is an ellipse
    \item[d)] locus of the point A is a pair of straight lines
\end{enumerate}
\end{frame}

\begin{frame}{solution}
    \frametitle{finding the locus of A:}
Given,
\begin{align}
\cos B + \cos C = 4 \sin^2 \frac{A}{2}\\
\cos B + \cos C = 4 \frac{(1 - \cos A)}{2}\\
2 \cos A + \cos B + \cos C = 2
\end{align}
By Projection rule:
\begin{align}c \cos B + b \cos C &= a \\c \cos A + a \cos C &= b \\b \cos A + a \cos B &= c\end{align}
\end{frame}
\begin{frame}
Combining all these into a Matrix:
\begin{align}\begin{bmatrix}2 & 1 & 1 \\0 & c & b \\c & 0 & a \\b & a & 0\end{bmatrix}\begin{bmatrix}\cos A \\\cos B \\\cos C\end{bmatrix}=\begin{bmatrix}2 \\a \\b \\c\end{bmatrix}\\
AX = b
\end{align}
for this system to be consistent
\begin{align}
\text{rank}(A) = \text{rank}([A|b]) \leq 3
\end{align}
\end{frame}
\begin{frame}
if $\text{rank}([A|b]) \leq 3$\\ its columns are linearly dependent.
$\therefore \det([A|b]) = 0$
\begin{align}
(2a - b - c)(-a^2 + (b-c)a + (b-c)^2) = 0\\
\text{as, a, b, c are sides of triangle}\\
-a^2 + (b-c)a + (b-c)^2 \neq 0\\
\therefore 2a - b - c &= 0 \\
\therefore b + c &= 2a
\end{align}
\end{frame}
\begin{frame}
\begin{align}
b + c = 2a\\
\| A - C \| + \| A - B \| = 2 \| B - C \|
\end{align}
given $B$ and $C$ are fixed.

$\therefore$ Locus of `A' is an ellipse,

as, the sum of its distances from 2 fixed points is constant (represents an ellipse).\\
$\therefore$ Options b and c are correct.
\end{frame}


\begin{frame}[fragile]
    \frametitle{Python Code}
    \begin{lstlisting}
import numpy as np
import matplotlib.pyplot as plt
# --- 1. Define the fixed base and derive ellipse parameters ---
# Let the fixed base BC be on the x-axis, centered at the origin.
# B = (-k, 0), C = (k, 0). We can choose k=2 for a clear visual.
k = 2
B = np.array([-k, 0])
C = np.array([k, 0])
# The length of the base 'a' is the distance between B and C.
# Note: 'a' here is the side length, not the semi-major axis of the ellipse.
side_a = np.linalg.norm(C - B)  # side_a = 2k = 4
\end{lstlisting}
\end{frame}

\begin{frame}[fragile]
\frametitle{Python Code}
\begin{lstlisting}
# --- 2. Use the derived condition to find the ellipse's properties ---
# From the solution, we found b + c = 2a.
# b = distance(A, C) and c = distance(A, B).
# So, distance(A, C) + distance(A, B) = 2 * side_a = 2 * (4) = 8.
# This is the definition of an ellipse with foci at B and C.
# The constant sum of distances is 2 * a_ellipse (semi-major axis).
constant_sum = 2 * side_a
a_ellipse = constant_sum / 2  # a_ellipse = side_a = 4
\end{lstlisting}
\end{frame}

\begin{frame}[fragile]
\frametitle{Python Code}
\begin{lstlisting}
# The distance from the center to a focus is c_ellipse.
c_ellipse = k  # c_ellipse = 2
# For an ellipse, a_ellipse^2 = b_ellipse^2 + c_ellipse^2
b_ellipse = np.sqrt(a_ellipse**2 - c_ellipse**2)  # Semi-minor axis
# --- 3. Generate points for the ellipse (locus of A) ---
t = np.linspace(0, 2 * np.pi, 300)
x_ellipse = a_ellipse * np.cos(t)
y_ellipse = b_ellipse * np.sin(t)
\end{lstlisting}
\end{frame}

\begin{frame}[fragile]
\frametitle{Python Code}
\begin{lstlisting}
# --- 4. Create the plot ---
plt.style.use('seaborn-v0_8-whitegrid')
fig, ax = plt.subplots(figsize=(10, 8))
# Plot the locus of A
ax.plot(x_ellipse, y_ellipse, label='Locus of Vertex A (Ellipse)', color='dodgerblue', linewidth=2)
# Plot the fixed base BC and foci
ax.plot([B[0], C[0]], [B[1], C[1]], 'o', markersize=8, color='red', label='Foci (Fixed Base BC)')
ax.text(B[0], B[1] - 0.5, f'B({B[0]}, {B[1]})', ha='center', fontsize=12)
ax.text(C[0], C[1] - 0.5, f'C({C[0]}, {C[1]})', ha='center', fontsize=12)
\end{lstlisting}
\end{frame}

\begin{frame}[fragile]
\frametitle{Python Code}
\begin{lstlisting}
# --- 5. Illustrate with an example triangle ---
# Choose an example point A on the ellipse (e.g., at t = 2π/5)
t_example = 2 * np.pi / 5
A_example = np.array([a_ellipse * np.cos(t_example), b_ellipse * np.sin(t_example)])
# Draw the triangle ABC
triangle_x = [A_example[0], B[0], C[0], A_example[0]]
triangle_y = [A_example[1], B[1], C[1], A_example[1]]
ax.plot(triangle_x, triangle_y, 'g--', label='Example Triangle ABC', linewidth=1.5)
\end{lstlisting}
\end{frame}

\begin{frame}[fragile]
\frametitle{Python Code}
\begin{lstlisting}
ax.plot(A_example[0], A_example[1], 'go', markersize=6)
ax.text(A_example[0], A_example[1] + 0.3, 'A(x, y)', ha='center', fontsize=12)

# Verify the condition for the example point and display it
side_b = np.linalg.norm(A_example - C)
side_c = np.linalg.norm(A_example - B)
info_text = (
    f"Condition:  $b + c = 2a$\n\n"
    f"Side $a$ (distance BC) = {side_a:.2f}\n"
    f"Side $b$ (distance AC) = {side_b:.2f}\n"
    f"Side $c$ (distance AB) = {side_c:.2f}\n\n"
\end{lstlisting}
\end{frame}

\begin{frame}[fragile]
\frametitle{Python Code}
\begin{lstlisting}
    f"Check:  ${side_b:.2f} + {side_c:.2f} = {side_b + side_c:.2f}$\n"
    f"Required: $2 \\times a = 2 \\times {side_a:.2f} = {2 * side_a:.2f}$"
)
ax.text(0.95, 0.05, info_text, transform=ax.transAxes, fontsize=11,
        verticalalignment='bottom', horizontalalignment='right',
        bbox=dict(boxstyle='round,pad=0.5', fc='aliceblue', alpha=0.9))
\end{lstlisting}
\end{frame}

\begin{frame}[fragile]
\frametitle{Python Code}
\begin{lstlisting}
# --- 6. Finalize the plot ---
ax.set_aspect('equal', adjustable='box')
ax.set_title('Locus of Vertex A is an Ellipse', fontsize=16)
ax.set_xlabel('x-axis', fontsize=12)
ax.set_ylabel('y-axis', fontsize=12)
ax.legend(loc='upper left')
plt.show()


\end{lstlisting}
\end{frame}

\begin{frame}[fragile]
\frametitle{C Code}
\begin{lstlisting}
#include <stdio.h>
#include <math.h>

// Define PI for trigonometric calculations if not already defined
#ifndef M_PI
#define M_PI 3.14159265358979323846
#endif

double to_radians(double degrees) {
    return degrees * M_PI / 180.0;
}
\end{lstlisting}
\end{frame}

\begin{frame}[fragile]
\frametitle{C Code}
\begin{lstlisting}
void check_triangle_properties(double angle_A_deg) {
    printf("--- Checking for A = %.2f degrees ---\n", angle_A_deg);

    // For a valid triangle, the simplified condition cos((B-C)/2) = 2*sin(A/2) must hold.
    // Since the maximum value of cosine is 1, we must have 2*sin(A/2) <= 1.
    // This implies sin(A/2) <= 0.5, which means A/2 <= 30 degrees, so A <= 60 degrees.
\end{lstlisting}
\end{frame}

\begin{frame}[fragile]
\frametitle{C Code}
\begin{lstlisting}
    if (angle_A_deg > 60.0 || angle_A_deg <= 0) {
        printf("A triangle with this condition cannot exist for A > 60 degrees or A <= 0.\n");
        printf("The expression 2*sin(A/2) would be > 1, which is impossible for a cosine value.\n\n");
        return;
    }
    // Convert angle A to radians for use in math functions
    double A_rad = to_radians(angle_A_deg);
\end{lstlisting}
\end{frame}

\begin{frame}[fragile]
\frametitle{C Code}
\begin{lstlisting}
    // --- Step 1: Find angles B and C that satisfy the condition ---
    // From the simplified relation: cos((B-C)/2) = 2*sin(A/2)
    double val_for_acos = 2.0 * sin(A_rad / 2.0);
    double B_minus_C_half_rad = acos(val_for_acos);   
    // We also know A + B + C = PI radians, so (B+C)/2 = PI/2 - A/2
    double B_plus_C_half_rad = M_PI / 2.0 - A_rad / 2.0;
    // Solve for B and C
    double B_rad = B_plus_C_half_rad + B_minus_C_half_rad;
    double C_rad = B_plus_C_half_rad - B_minus_C_half_rad;
\end{lstlisting}
\end{frame}

\begin{frame}[fragile]
\frametitle{C Code}
\begin{lstlisting}
    // --- Step 2: Verify the original trigonometric identity ---
    double lhs_identity = cos(B_rad) + cos(C_rad);
    double rhs_identity = 4.0 * pow(sin(A_rad / 2.0), 2);
    printf("Verification of given identity:\n");
    printf("  LHS (cos B + cos C)    = %f\n", lhs_identity);
    printf("  RHS (4 * sin^2(A/2)) = %f\n", rhs_identity);
    // --- Step 3: Verify the derived side relationship b + c = 2a ---
\end{lstlisting}
\end{frame}

\begin{frame}[fragile]
\frametitle{C Code}
\begin{lstlisting}
    // Using the Sine Rule, we can use the sines of the angles as relative side lengths.
    double a = sin(A_rad);
    double b = sin(B_rad);
    double c = sin(C_rad);

    double b_plus_c = b + c;
    double two_a = 2.0 * a;
\end{lstlisting}
\end{frame}

\begin{frame}[fragile]
\frametitle{C Code}
\begin{lstlisting}
    printf("\nVerification of the side relationship (b + c = 2a):\n");
    printf("  Relative side lengths (a=sinA, b=sinB, c=sinC):\n");
    printf("    b + c = %f\n", b_plus_c);
    printf("    2 * a = %f\n", two_a);

    // Check for equality using a small tolerance for floating-point errors
\end{lstlisting}
\end{frame}

\begin{frame}[fragile]
\frametitle{C Code}
\begin{lstlisting}
    if (fabs(b_plus_c - two_a) < 1e-9) {
        printf("\nResult: The relationship b + c = 2a holds true. \n\n");
    } else {
        printf("\nResult: The relationship b + c = 2a does NOT hold. \n\n");
    }
}
\end{lstlisting}
\end{frame}

\begin{frame}[fragile]
\frametitle{C Code}
\begin{lstlisting}
int main() {
    printf("## Solution Verification for Triangle Problem ##\n\n");
    printf("This program verifies the two correct conclusions from the problem:\n");
    printf("  b) b + c = 2a\n");
    printf("  c) locus of the point A is an ellipse\n\n");
    // --- Part 1: Numerical Verification of b + c = 2a ---
    printf("### Part 1: Verifying the side relationship b + c = 2a ###\n\n");
    printf("The code will now test the derived relationship for various valid angles of A.\n\n");
\end{lstlisting}
\end{frame}

\begin{frame}[fragile]
\frametitle{C Code}
\begin{lstlisting}
    // Check for a few valid angles of A (where A <= 60 degrees)
    check_triangle_properties(60.0); // Special case: equilateral triangle
    check_triangle_properties(45.0);
    check_triangle_properties(30.0);
    // Check an invalid angle to show the constraint
    check_triangle_properties(90.0);
    // --- Part 2: Explanation of the Locus of A ---
    printf("### Part 2: Determining the Locus of Point A ###\n\n");
    printf("1. The problem states that the base BC is fixed. Let its length be 'a'.\n");
\end{lstlisting}
\end{frame}

\begin{frame}[fragile]
\frametitle{C Code}
\begin{lstlisting}
    printf("2. From the derivation, we found the relationship b + c = 2a.\n");
    printf("3. The side 'b' is the distance AC, and 'c' is the distance AB.\n");
    printf("4. Therefore, the condition is AB + AC = 2a.\n");
    printf("5. Since 'a' is a fixed length, '2a' is a constant value.\n\n");
    printf("This is the geometric definition of an ellipse: the set of all points (A) for which the sum of the distances to two fixed points (the foci, B and C) is a constant (2a).\n\n");
    printf("Conclusion: The locus of point A is an ellipse. \n");
    return 0;
}
\end{lstlisting}
\end{frame}

\begin{frame}[fragile]
\frametitle{Python and C Code}
\begin{lstlisting}
import math

# Define PI for trigonometric calculations
PI = math.pi

def to_radians(degrees: float) -> float:
    return degrees * PI / 180.0

def check_triangle_properties(angle_A_deg: float):
    print(f"--- Checking for A = {angle_A_deg:.2f} degrees ---")
\end{lstlisting}
\end{frame}

\begin{frame}[fragile]
\frametitle{Python and C Code}
\begin{lstlisting}
    if angle_A_deg > 60.0 or angle_A_deg <= 0:
        print("A triangle with this condition cannot exist for A > 60 degrees or A <= 0.")
        print("The expression 2*sin(A/2) would be > 1, which is impossible for a cosine value.\n")
        return

    # Convert angle A to radians
    A_rad = to_radians(angle_A_deg)
\end{lstlisting}
\end{frame}

\begin{frame}[fragile]
\frametitle{Python and C Code}
\begin{lstlisting}
    # Step 1: Find angles B and C that satisfy the condition
    val_for_acos = 2.0 * math.sin(A_rad / 2.0)

    # Check if acos input is in valid domain
    if val_for_acos > 1.0:
        print("acos argument exceeds 1. Invalid configuration.\n")
        return

    B_minus_C_half_rad = math.acos(val_for_acos)
    B_plus_C_half_rad = PI / 2.0 - A_rad / 2.0
\end{lstlisting}
\end{frame}

\begin{frame}[fragile]
\frametitle{Python and C Code}
\begin{lstlisting}
    B_rad = B_plus_C_half_rad + B_minus_C_half_rad
    C_rad = B_plus_C_half_rad - B_minus_C_half_rad

    # Step 2: Verify the original trigonometric identity
    lhs_identity = math.cos(B_rad) + math.cos(C_rad)
    rhs_identity = 4.0 * (math.sin(A_rad / 2.0) ** 2)

    print("Verification of given identity:")
    print(f"  LHS (cos B + cos C)    = {lhs_identity}")
    print(f"  RHS (4 * sin^2(A/2)) = {rhs_identity}")
\end{lstlisting}
\end{frame}

\begin{frame}[fragile]
\frametitle{Python and C Code}
\begin{lstlisting}
    # Step 3: Verify the derived side relationship b + c = 2a
    a = math.sin(A_rad)
    b = math.sin(B_rad)
    c = math.sin(C_rad)

    b_plus_c = b + c
    two_a = 2.0 * a
    print("\nVerification of the side relationship (b + c = 2a):")
    print("  Relative side lengths (a=sinA, b=sinB, c=sinC):")
    print(f"    b + c = {b_plus_c}")
    print(f"    2 * a = {two_a}")
\end{lstlisting}
\end{frame}

\begin{frame}[fragile]
\frametitle{Python and C Code}
\begin{lstlisting}
    # Allow small numerical tolerance
    if abs(b_plus_c - two_a) < 1e-9:
        print("\nResult: The relationship b + c = 2a holds true.\n")
    else:
        print("\nResult: The relationship b + c = 2a does NOT hold.\n")
\end{lstlisting}
\end{frame}

\begin{frame}[fragile]
\frametitle{Python and C Code}
\begin{lstlisting}
def main():
    print("## Solution Verification for Triangle Problem ##\n")
    print("This program verifies the two correct conclusions from the problem:")
    print("  b) b + c = 2a")
    print("  c) locus of the point A is an ellipse\n")
    # Part 1: Numerical Verification
    print("### Part 1: Verifying the side relationship b + c = 2a ###\n")
    print("The code will now test the derived relationship for various valid angles of A.\n")
\end{lstlisting}
\end{frame}

\begin{frame}[fragile]
\frametitle{Python and C Code}
\begin{lstlisting}
    # Test various angles
    check_triangle_properties(60.0)  # Equilateral triangle
    check_triangle_properties(45.0)
    check_triangle_properties(30.0)

    # Invalid case
    check_triangle_properties(90.0)
\end{lstlisting}
\end{frame}

\begin{frame}[fragile]
\frametitle{Python and C Code}
\begin{lstlisting}
    # Part 2: Locus Explanation
    print("### Part 2: Determining the Locus of Point A ###\n")
    print("1. The problem states that the base BC is fixed. Let its length be 'a'.")
    print("2. From the derivation, we found the relationship b + c = 2a.")
    print("3. The side 'b' is the distance AC, and 'c' is the distance AB.")
    print("4. Therefore, the condition is AB + AC = 2a.")
\end{lstlisting}
\end{frame}

\begin{frame}[fragile]
\frametitle{Python and C Code}
\begin{lstlisting}
    print("5. Since 'a' is a fixed length, '2a' is a constant value.\n")
    print("This is the geometric definition of an ellipse:")
    print("The set of all points (A) for which the sum of the distances to two fixed points (the foci, B and C) is a constant (2a).\n")
    print("Conclusion: The locus of point A is an ellipse.\n")

if __name__ == "__main__":
    main()
\end{lstlisting}
\end{frame}
\begin{frame}
\begin{figure}
    \centering
    \includegraphics[width=0.75\columnwidth]{graph15.png}
    \caption{Plot}
    \label{fig:placeholder}
\end{figure}
\end{frame}

\end{document}