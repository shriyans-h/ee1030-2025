\let\negmedspace\undefined
\let\negthickspace\undefined
\documentclass[journal]{IEEEtran}
\usepackage[a5paper, margin=10mm, onecolumn]{geometry}
%\usepackage{lmodern} % Ensure lmodern is loaded for pdflatex
\usepackage{tfrupee} % Include tfrupee package

\setlength{\headheight}{1cm} % Set the height of the header box
\setlength{\headsep}{0mm}     % Set the distance between the header box and the top of the text

\usepackage{gvv-book}
\usepackage{gvv}
\usepackage{cite}
\usepackage{amsmath,amssymb,amsfonts,amsthm}
\usepackage{algorithmic}
\usepackage{graphicx}
\usepackage{textcomp}
\newcommand{\tr}{\operatorname{tr}}
\usepackage{xcolor}
\usepackage{txfonts}
\usepackage{listings}
\usepackage{enumitem}
\usepackage{mathtools}
\usepackage{gensymb}
\usepackage{comment}
\usepackage[breaklinks=true]{hyperref}
\usepackage{tkz-euclide} 
\usepackage{listings}
% \usepackage{gvv}                                        
\def\inputGnumeric\topable{}                                 
\usepackage[latin1]{inputenc}                                
\usepackage{color}                                            
\usepackage{array}                                            
\usepackage{longtable}                                       
\usepackage{calc}                                             
\usepackage{multirow}                                         
\usepackage{hhline}                                           
\usepackage{ifthen}                                           
\usepackage{lscape}
\begin{document}

\bibliographystyle{IEEEtran}

\title{Extra}
\author{EE25BTECH11023 -   Venkata Sai}
% \maketitle
% \newpage
% \bigskip
\maketitle \vspace{-1cm}
\renewcommand{\thefigure}{\theenumi}
\renewcommand{\thetable}{\theenumi}
\setlength{\intextsep}{10pt} % Space between text and floats

\numberwithin{align}{enumi}
\numberwithin{figure}{enumi}
\renewcommand{\thetable}{\theenumi}

\textbf{Question:}  \\
Find the equations of tangents drawn from origin to the circle $x^2+y^2-2rx-2hy+h^2=0$,are
\begin{multicols}{2}
\begin{enumerate}
    \item $x=0$
    \item $y=0$
    \item $\brak{h^2-r^2}x-2rhy=0$
    \item $\brak{h^2-r^2}x+2rhy=0$
\end{enumerate}
\end{multicols}
\textbf{Solution:}  \\
Equation of a circle is given by
\begin{align}
\vec{x}^\top \vec{V} \vec{x} + 2 \vec{u}^\top \vec{x} + f = 0,
\end{align}
The parametric equation of this line is
\begin{align}
\vec{x} = \vec{h} + k \vec{m}.
\end{align}
\begin{align}
\brak{\vec{h}+k\vec{m}}^T \vec{V}\brak{\vec{h}+k\vec{m}}+2\vec{u}^T\brak{\vec{h}+k\vec{m}}+f
\end{align}
\begin{align}
\vec{h}^\top \vec{V} \vec{h}+ k\brak{\vec{m}^\top \vec{V} \vec{h} + \vec{h}^\top \vec{V} \vec{m}}+ k^2 \vec{m}^\top \vec{V} \vec{m}+ 2\vec{u}^\top\vec{h} + 2k\,\vec{u}^\top\vec{m} + f \\
\brak{\vec{m}^\top \vec{V} \vec{m}}k^2+2\brak{\vec{m}^\top \vec{V}\vec{h} + \vec{u}^\top\vec{m}}\,k
   + \brak{\vec{h}^\top \vec{V} \vec{h} + 2\vec{u}^\top\vec{h} + f}
\end{align}
Hence the quadratic in $k$ is
\begin{align}
A k^2 + B k + C = 0,
\end{align}
where
\begin{align}
A &= \vec{m}^{\top} \vec{V} \vec{m}, \\
B &= 2 \brak{ \vec{m}^{\top} \vec{V} \vec{h} + \vec{u}^{\top} \vec{m} }, \\
C &= \brak{ \vec{h}^{\top} \vec{V} \vec{h} + 2 \vec{u}^{\top} \vec{h} + f }
\end{align}
Discriminant is 0 as tangent intersects the circle at only one point
\begin{align}
B^2 - 4AC = 0 
\end{align}
\begin{align}
A &= \vec{m}^{\top} \vec{V} \vec{m}, \\
B &= 2 \brak{ \vec{m}^{\top} \vec{V} \vec{h} + \vec{u}^{\top} \vec{m} }, \\
C &= \brak{ \vec{h}^{\top} \vec{V} \vec{h} + 2 \vec{u}^{\top} \vec{h} + f }
\end{align}
Discriminant is 0 as tangent intersects the circle at only one point
\begin{align}
B^2 - 4AC = 0 
\end{align}
Substituting the expressions for A, B, and C and expanding:
\begin{align}
\left(2 \brak{\vec{m}^{\top}\vec{V}\vec{h} + \vec{u}^{\top}\vec{m}}\right) ^{2}
- 4 \brak{\vec{m}^{\top}\vec{V}\vec{m}}\brak{\vec{h}^{\top}\vec{V}\vec{h} + 2 \vec{u}^{\top}\vec{h} + f}
= 0.
\end{align}
\begin{align}
\brak{\vec{m}^{\top}\vec{V}\vec{h}+\vec{u}^{\top}\vec{m}}\brak{\vec{m}^{\top}\vec{V}\vec{h} + \vec{u}^{\top}\vec{m}}^\top-\brak{\vec{m}^{\top}\vec{V}\vec{m}}\brak{\vec{h}^{\top}\vec{V}\vec{h} + 2 \vec{u}^{\top}\vec{h} + f}
= 0
\end{align}
\begin{align}
\brak{\vec{m}^{\top}\vec{V}\vec{h} + \vec{u}^{\top}\vec{m}}\brak{\vec{h}^{\top}\vec{V}\vec{m}+\vec{m}^{\top}\vec{u}}-\brak{\vec{m}^{\top}\vec{V}\vec{m}}\brak{\vec{h}^{\top}\vec{V}\vec{h} + 2 \vec{u}^{\top}\vec{h} + f}=0
\end{align}
\begin{align}
\brak{\vec{m}^{\top} \vec{V} \vec{h}}\brak{\vec{h}^{\top} \vec{V} \vec{m}} + \brak{\vec{m}^{\top} \vec{V} \vec{h}}\brak{\vec{m}^{\top} \vec{u}} + \brak{\vec{u}^{\top} \vec{m}}\brak{\vec{h}^{\top} \vec{V} \vec{m}} + \brak{\vec{u}^{\top} \vec{m}}\brak{\vec{m}^{\top} \vec{u}} \nonumber \\
- \brak{\vec{m}^{\top}\vec{V}\vec{m}}\brak{\vec{h}^{\top}\vec{V}\vec{h}} + 2\brak{\vec{m}^{\top}\vec{V}\vec{m}}\brak{\vec{u}^{\top}\vec{h}} + f\brak{\vec{m}^{\top}\vec{V}\vec{m}} &= 0
\end{align}
\begin{align}
\brak{\vec{m}^{\top} \vec{V} \vec{h}}\brak{\vec{h}^{\top} \vec{V} \vec{m}} + \brak{\vec{m}^{\top} \vec{V} \vec{h}}\brak{\vec{m}^{\top} \vec{u}} + \brak{\vec{u}^{\top} \vec{m}}\brak{\vec{h}^{\top} \vec{V} \vec{m}} + \brak{\vec{u}^{\top} \vec{m}}\brak{\vec{m}^{\top} \vec{u}} \nonumber \\
- \brak{\vec{m}^{\top}\vec{V}\vec{m}}\brak{\vec{h}^{\top}\vec{V}\vec{h}} - 2\brak{\vec{m}^{\top}\vec{V}\vec{m}}\brak{\vec{u}^{\top}\vec{h}} - f\brak{\vec{m}^{\top}\vec{V}\vec{m}} &= 0
\end{align}
\begin{align}
\vec{m}^{\top} \brak{ \vec{V}\vec{h}\vec{h}^{\top}\vec{V} + \vec{V}\vec{h}\vec{u}^{\top} + \vec{u}\vec{h}^{\top}\vec{V} + \vec{u}\vec{u}^{\top} - \brak{\vec{h}^{\top}\vec{V}\vec{h}}\vec{V} - 2\brak{\vec{u}^{\top}\vec{h}}\vec{V} - f\vec{V} } \vec{m} &= 0
\end{align}
\begin{align}
\vec{m}^{\top} \brak{ \vec{V}\vec{h}\brak{\vec{V}\vec{h}}^\top + \vec{V}\vec{h}\vec{u}^{\top} + \vec{u}\brak{\vec{V}\vec{h}}^{\top} + \vec{u}\vec{u}^{\top} - \brak{\vec{h}^{\top}\vec{V}\vec{h}}\vec{V} - 2\brak{\vec{u}^{\top}\vec{h}}\vec{V} - f\vec{V} } \vec{m} &= 0
\end{align}
\begin{align}
\vec{m}^{\top} \brak{\brak{\vec{V}\vec{h}+\vec{u}}\brak{\vec{V}\vec{h}+\vec{u}}^\top -\brak{ \vec{h}^{\top}\vec{V}\vec{h} + 2\vec{u}^{\top}\vec{h} + f}\vec{V} } \vec{m} &= 0
\end{align}
\begin{align}
\vec{m}^{\top} \brak{\brak{\vec{V}\vec{h}+\vec{u}}\brak{\vec{V}\vec{h}+\vec{u}}^\top - g\brak{\vec{h}}} \vec{m} &= 0
\end{align}
where
\begin{align}
g\brak{\vec{h}}=\brak{ \vec{h}^{\top}\vec{V}\vec{h} + 2\vec{u}^{\top}\vec{h} + f}\vec{V} 
\end{align}
On comparing the given circle with matrix equation
\begin{align}
x^2+y^2=\vec{x}^\top \myvec{1&0\\0&1} \vec{x} \implies \vec{V}=\myvec{1&0\\0&1}  \\
-2rx-2hy=2\vec{u}^\top\vec{x} \implies \vec{u}^\top=\myvec{-r&-h}\implies \vec{u}=\myvec{-r\\-h} \\
h^2=f
\end{align}
Given tangents are drawn from the origin
\begin{align}
    \vec{h}=\myvec{0\\0}
\end{align}
\begin{align}
  \vec{V}\vec{h}+\vec{u}=\myvec{1&0\\0&1}\myvec{0\\0}+\myvec{-r\\-h}=\myvec{-r\\-h}
  \end{align}
  \begin{align}
   \vec{V}\vec{h}=0 \implies  \vec{h}^\top\vec{V}\vec{h}=0
   \end{align}
   \begin{align}
  g\brak{\vec{h}}=\brak{\vec{h}^{\top}\vec{V}\vec{h}+2\vec{u}^{\top}\vec{h} + f}\vec{V}=\brak{0+2\myvec{-r\\-h}\myvec{0\\0}+h^2}\myvec{1&0\\0&1}
\end{align}
\begin{align}
g\brak{\vec{h}}=h^2
\end{align}
\begin{align}
\vec{m}^\top\sum\vec{m}\implies\sum=\myvec{-r\\-h}\myvec{-r&-h}-h^2\vec{I}
\end{align}
\begin{align}
\sum=\myvec{r^2& rh \\rh &h^2}-\myvec{h^2&0\\0&h^2}=\myvec{r^2-h^2&rh\\rh&0}
\end{align}
\begin{align}
\mydet{\sum-\lambda\vec{I}}=0\\
\mydet{\myvec{r^2-h^2-\lambda&rh\\rh&-\lambda}}=0\\
\brak{r^2-h^2-\lambda}\brak{-\lambda}-r^2h^2=0 \\
\lambda^2-\brak{r^2-h^2}\lambda-r^2h^2=0\\
\brak{\lambda-r^2}\brak{\lambda+h^2}=0
\end{align}
the eigen vectors are
\begin{align}
    \lambda=r^2\ \text{or}\ \lambda=-h^2
\end{align}
 Because the tangent passes through  origin
 \begin{align}
     \vec{m}=\vec{x} \\
     \myvec{x&y}\myvec{r^2-h^2&rh\\rh&0}\myvec{x\\y}=0 \\
     \myvec{x&y}\myvec{\brak{r^2-h^2}x+\brak{rh}y\\rhx}=0 \\
     x\myvec{\brak{r^2-h^2}x+rhy}+y\brak{rhx}=0 \\
     x^2\brak{r^2-h^2}+2rhxy=0 \\
     x\brak{\brak{r^2-h^2}x+2rhy}=0 \\
     x=0\ \text{and}\ \brak{r^2-h^2}x+2rhy=0
 \end{align}
\end{document}