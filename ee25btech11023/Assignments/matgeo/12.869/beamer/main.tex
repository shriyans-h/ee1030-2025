\documentclass{beamer}
\usepackage[utf8]{inputenc}

\usetheme{Boadilla}
\usecolortheme{lily}
\usepackage{amsmath,amssymb,amsfonts,amsthm}
\usepackage{mathtools}
\usepackage{txfonts}
\usepackage{tkz-euclide}
\usepackage{listings}
\usepackage{adjustbox}
\usepackage{array}
\usepackage{tabularx}
\usepackage{lmodern}
\usepackage{gvv}
\usepackage{circuitikz}
\usepackage{tikz}
\usepackage{graphicx}

\setbeamertemplate{footline}
{
  \leavevmode%
  \hbox{%
  \begin{beamercolorbox}[wd=\paperwidth,ht=2.25ex,dp=1ex,right]{author in head/foot}%
    \insertframenumber{} / \inserttotalframenumber\hspace*{2ex} 
  \end{beamercolorbox}}%
  \vskip0pt%
}

\usepackage{tcolorbox}
\tcbuselibrary{minted,breakable,xparse,skins}




\providecommand{\nCr}[2]{\,^{#1}C_{#2}} % nCr
\providecommand{\nPr}[2]{\,^{#1}P_{#2}} % nPr
\providecommand{\mbf}{\mathbf}
\providecommand{\pr}[1]{\ensuremath{\Pr\left(#1\right)}}
\providecommand{\qfunc}[1]{\ensuremath{Q\left(#1\right)}}
\providecommand{\sbrak}[1]{\ensuremath{{}\left[#1\right]}}
\providecommand{\lsbrak}[1]{\ensuremath{{}\left[#1\right.}}
\providecommand{\rsbrak}[1]{\ensuremath{{}\left.#1\right]}}
\providecommand{\brak}[1]{\ensuremath{\left(#1\right)}}
\providecommand{\lbrak}[1]{\ensuremath{\left(#1\right.}}
\providecommand{\rbrak}[1]{\ensuremath{\left.#1\right)}}
\providecommand{\cbrak}[1]{\ensuremath{\left\{#1\right\}}}
\providecommand{\lcbrak}[1]{\ensuremath{\left\{#1\right.}}
\providecommand{\rcbrak}[1]{\ensuremath{\left.#1\right\}}}
\theoremstyle{remark}
\newcommand{\sgn}{\mathop{\mathrm{sgn}}}
\providecommand{\abs}[1]{\left\vert#1\right\vert}
\providecommand{\res}[1]{\Res\displaylimits_{#1}} 
\providecommand{\norm}[1]{\lVert#1\rVert}
\providecommand{\mtx}[1]{\mathbf{#1}}
\providecommand{\mean}[1]{E\left[ #1 \right]}
\providecommand{\fourier}{\overset{\mathcal{F}}{ \rightleftharpoons}}
%\providecommand{\hilbert}{\overset{\mathcal{H}}{ \rightleftharpoons}}
\providecommand{\system}{\overset{\mathcal{H}}{ \longleftrightarrow}}
	%\newcommand{\solution}[2]{\textbf{Solution:}{#1}}
%\newcommand{\solution}{\noindent \textbf{Solution: }}
\providecommand{\dec}[2]{\ensuremath{\overset{#1}{\underset{#2}{\gtrless}}}}
\newcommand{\myvec}[1]{\ensuremath{\begin{pmatrix}#1\end{pmatrix}}}
\let\vec\mathbf

\lstset{
%language=C,
frame=single, 
breaklines=true,
columns=fullflexible
}

\numberwithin{equation}{section}

\lstset{
  language=Python,
  basicstyle=\ttfamily\small,
  keywordstyle=\color{blue},
  stringstyle=\color{orange},
  numbers=left,
  numberstyle=\tiny\color{gray},
  breaklines=true,
  showstringspaces=false
}

\title{Problem 12.869}
\author{ee25btech11023-Venkata Sai}

\date{\today} 
\begin{document}

\begin{frame}
\titlepage
\end{frame}

\section*{Outline}
\begin{frame}
\tableofcontents
\end{frame}

\section{Problem}

\begin{frame}
\frametitle{Problem}
For $\vec{X}=\myvec{x_1\\x_2\\x_3}$, quadratic form
\begin{align}
\vec{Q\brak{\vec{X}}} = 2x^2_1 + 2x^2_2 + 3x^2_3 + 4x_1x_2 + 2x_1x_3 + 2x_2x_3
\end{align}
Let $\vec{M}$ be symmetric matrix of Q. For $\vec{Y} \in \mathbb{R}^3$? non-zero define 
\begin{align}
    a_n=\frac{\vec{Y}^\top\brak{\vec{M}+\vec{I}_3}^{n+1}\vec{Y}}{\vec{Y}^\top\brak{\vec{M}+\vec{I}_3}^{n}\vec{Y}}
\end{align}
Then lim$_{n\rightarrow\infty}a_n$ = \dots  \\
\end{frame}
%\subsection{Literature}
\section{Solution}

 
\subsection{Equation}
\begin{frame}
\frametitle{Equation}
Given
 \begin{align}
 \vec{Q\brak{\vec{X}}} = 2x^2_1 + 2x^2_2 + 3x^2_3 &+ 4x_1x_2 + 2x_1x_3 + 2x_2x_3  \\
 \vec{X}&=\myvec{x_1\\x_2\\x_3}
 \end{align}
 $\vec{M}$ is a symmetric matrix of Q
\begin{align}
     \vec{Q}=\vec{X}^\top\vec{M}\vec{X} 
     \end{align}
     \begin{align}
          \vec{Q}  &=\myvec{x_1 & x_2 & x_3}\myvec{M_{11} & M_{12} & M_{13}\\M_{12} &M_{22} &M_{23}\\M_{13} &M_{23}&M_{33}}\myvec{x_1\\x_2\\x_3} \\
            &=\myvec{x_1&x_2&x_3}\myvec{ x_1 M_{11} + x_2 M_{21} + x_3 M_{13} \\ x_1 M_{12} + x_2 M_{22} + x_3 M_{23} \\x_1 M_{13} + x_2 M_{23} + x_3 M_{33}}
            \end{align}
\end{frame}
\subsection{Finding the vector A}
\begin{frame}
\frametitle{Finding $\vec{A}$}
   \begin{align}
            =x_1^2 M_{11} + x_1 x_2 M_{12} + x_1 x_3 M_{13} + x_2 x_1 M_{12} + x_2^2 M_{22} \\ \nonumber+ x_2 x_3 M_{23} + x_3 x_1 M_{13} + x_3 x_2 M_{23} + x_3^2 M_{33}
            \end{align}
            \begin{align}
           =x_1^2 M_{11} + 2x_1 x_2 M_{12} + 2x_1 x_3 M_{13}+2x_2 x_3 M_{23} + x_2^2 M_{22}+x_3^2 M_{33}
            \end{align}
On comparing
\begin{align}
    M_{11}=2,M_{12}=\frac{4}{2}=2,M_{13}=\frac{2}{2}=1,M_{23}=\frac{2}{2}=1,M_{22}=2,M_{33}=3 
\end{align}
\begin{align}
    \vec{M}&=\myvec{2&2&1\\2&2&1\\1&1&3}\\
\vec{M}+\vec{I_3}&=\myvec{2&2&1\\2&2&1\\1&1&3}+\myvec{1&0&0\\0&1&0\\0&0&1}=\myvec{3&2&1\\2&3&1\\1&1&4}=\vec{A}
\end{align}
 \end{frame}
 \subsection{Condition}
\begin{frame}
\frametitle{Condition}
For eigen values of $\vec{A}$
\begin{align}
    \mydet{\vec{A}-\lambda\vec{I}}=0 
    \end{align}
    \begin{align}
    \mydet{\myvec{3&2&1\\2&3&1\\1&1&4}-\lambda\myvec{1&0&0\\0&1&0\\0&0&1}} = 0 \\
 \mydet{\myvec{3-\lambda&2&1\\2&3-\lambda&1\\1&1&4-\lambda}}=0 
 \end{align}
 \begin{align}
 \brak{3-\lambda}\brak{\brak{3-\lambda}\brak{4-\lambda}-1}-2\brak{2\brak{4-\lambda}-1}+1\brak{2-\brak{3-\lambda}}=0 
 \end{align}
 \vspace{-0.7cm} 
 \begin{align}
 \brak{3-\lambda}\brak{\lambda^2-7\lambda+12-1}-2\brak{8-2\lambda-1}+1\brak{2+\lambda-3}=0 
 \end{align}
 \vspace{-0.7cm} 
 \begin{align}
 \brak{3-\lambda}\brak{\lambda^2-7\lambda+11}-2\brak{7-2\lambda}+1\brak{\lambda-1}=0 
 \end{align}
 \end{frame}
\begin{frame}
\subsection{Finding eigen values}
\frametitle{Finding eigen values}
 \vspace{-0.7cm} 
 \begin{align}
 \brak{3\lambda^2-21\lambda+33-\lambda^3+7\lambda^2-11\lambda}-14+4\lambda+\lambda-1=0
 \end{align}
 \vspace{-0.7cm}
 \begin{align}
 -\lambda^3-10\lambda^2-32\lambda+33-14+5\lambda-1=0 
 \end{align}
 \vspace{-0.7cm}
 \begin{align}
 \lambda^3-10\lambda^2+27\lambda-18=0 
\end{align}
\vspace{-0.7cm}
\begin{align}
    \brak{\lambda-1}\brak{\lambda^2-9\lambda+18}=0
\end{align}
\vspace{-0.7cm}
\begin{align}
    \brak{\lambda-1}\brak{\lambda-3}\brak{\lambda-6}=0
\end{align}
The eigen values of $\vec{A}$ are 1,3,6 \\
Given
\begin{align}
a_n&=\frac{\vec{Y}^\top\brak{\vec{M}+\vec{I}_3}^{n+1}\vec{Y}}{\vec{Y}^\top\brak{\vec{M}+\vec{I}_3}^{n}\vec{Y}} \\
    a_n&=\frac{\vec{Y}^\top\vec{A}^{n+1}\vec{Y}}{\vec{Y}^\top\vec{A}^{n}\vec{Y}}
\end{align}
As $\vec{A}$ is symmetric
\begin{align}
    \vec{A}=\vec{P}\vec{D}\vec{P}^\top\ \text{where}\ \vec{D}=\myvec{1&0&0\\0&3&0\\0&0&6} 
    \end{align}
\end{frame}
\subsection{Simplify}
\begin{frame}[fragile]
\frametitle{Simplify $a_n$}
\begin{align}
\vec{A}^n=\vec{P}\myvec{1^n&0&0\\0&3^n&0\\0&0&6^n}\vec{P}^\top 
\end{align}
\begin{align}
\vec{Y}^\top\vec{A}^n\vec{Y}=\brak{\vec{P^\top}\vec{Y}}^\top \myvec{1^n&0&0\\0&3^n&0\\0&0&6^n}\vec{P}^\top\vec{Y}=\vec{v}^\top\myvec{1^n&0&0\\0&3^n&0\\0&0&6^n}\vec{v}
\end{align}
where
\begin{align}
    \vec{v}&=\vec{P}^\top\vec{Y} \\
\vec{Y}^\top\vec{A}^n\vec{Y}&=v_1^2\brak{1}^n+v_2^2\brak{3}^n+v_3^2\brak{6}^n
\end{align}
which will be of the form
\begin{align}
a_n=\frac{v_1^2\brak{1}^{n+1}+v_2^2\brak{3}^{n+1}+v_3^2\brak{6}^{n+1}}{v_1^2\brak{1}^n+v_2^2\brak{3}^n+v_3^2\brak{6}^n}
\end{align}
\end{frame}
 \subsection{Conclusion}
\begin{frame}[fragile]
\frametitle{Conclusion}
\begin{align}
a_n=\frac{v_1^2\brak{1}^{n}1+v_2^2\brak{3}^{n}3+v_3^2\brak{6}^{n}6}{v_1^2\brak{1}^n+v_2^2\brak{3}^n+v_3^2\brak{6}^n} 
\end{align}
\begin{align}
   a_n= \frac{6^n\brak{v_1^2\brak{\frac{1}{6}}^{n}1+v_2^2\brak{\frac{3}{6}}^{n}3+v_3^2\brak{\frac{6}{6}}^{n}6}}{6^n\brak{v_1^2\brak{\frac{1}{6}}^n+v_2^2\brak{\frac{3}{6}}^n+v_3^2\brak{\frac{6}{6}}^n}} 
\end{align}
\begin{align}
    lim_{n\rightarrow\infty}a_n = \frac{0+0+v_3^2\brak{6}}{0+0+v_3^2}=6
\end{align}
where as $n\rightarrow\infty$
\begin{align}
    \frac{1}{6} \rightarrow 0 ,\frac{3}{6} \rightarrow 0
\end{align}
Hence
\begin{align}
    lim_{n\rightarrow\infty}a_n=6
\end{align}
\end{frame}
\end{document}