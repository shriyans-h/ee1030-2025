\documentclass{beamer}
\usepackage[utf8]{inputenc}

\usetheme{Boadilla}
\usecolortheme{lily}
\usepackage{amsmath,amssymb,amsfonts,amsthm}
\usepackage{mathtools}
\usepackage{txfonts}
\usepackage{tkz-euclide}
\usepackage{listings}
\usepackage{multicol}
\usepackage{adjustbox}
\usepackage{array}
\usepackage{tabularx}
\usepackage{lmodern}
\usepackage{gvv}
\usepackage{circuitikz}
\usepackage{tikz}
\usepackage{graphicx}

\setbeamertemplate{footline}
{
  \leavevmode%
  \hbox{%
  \begin{beamercolorbox}[wd=\paperwidth,ht=2.25ex,dp=1ex,right]{author in head/foot}%
    \insertframenumber{} / \inserttotalframenumber\hspace*{2ex}
  \end{beamercolorbox}}%
  \vskip0pt%
}

\usepackage{tcolorbox}
\tcbuselibrary{minted,breakable,xparse,skins}




\providecommand{\nCr}[2]{\,^{#1}C_{#2}} % nCr
\providecommand{\nPr}[2]{\,^{#1}P_{#2}} % nPr
\providecommand{\mbf}{\mathbf}
\providecommand{\pr}[1]{\ensuremath{\Pr\left(#1\right)}}
\providecommand{\qfunc}[1]{\ensuremath{Q\left(#1\right)}}
\providecommand{\sbrak}[1]{\ensuremath{{}\left[#1\right]}}
\providecommand{\lsbrak}[1]{\ensuremath{{}\left[#1\right.}}
\providecommand{\rsbrak}[1]{\ensuremath{{}\left.#1\right]}}
\providecommand{\brak}[1]{\ensuremath{\left(#1\right)}}
\providecommand{\lbrak}[1]{\ensuremath{\left(#1\right.}}
\providecommand{\rbrak}[1]{\ensuremath{\left.#1\right)}}
\providecommand{\cbrak}[1]{\ensuremath{\left\{#1\right\}}}
\providecommand{\lcbrak}[1]{\ensuremath{\left\{#1\right.}}
\providecommand{\rcbrak}[1]{\ensuremath{\left.#1\right\}}}
\theoremstyle{remark}
\newcommand{\sgn}{\mathop{\mathrm{sgn}}}
\providecommand{\abs}[1]{\left\vert#1\right\vert}
\providecommand{\res}[1]{\Res\displaylimits_{#1}}
\providecommand{\norm}[1]{\lVert#1\rVert}
\providecommand{\mtx}[1]{\mathbf{#1}}
\providecommand{\mean}[1]{E\left[ #1 \right]}
\providecommand{\fourier}{\overset{\mathcal{F}}{ \rightleftharpoons}}
%\providecommand{\hilbert}{\overset{\mathcal{H}}{ \rightleftharpoons}}
\providecommand{\system}{\overset{\mathcal{H}}{ \longleftrightarrow}}
	%\newcommand{\solution}[2]{\textbf{Solution:}{#1}}
%\newcommand{\solution}{\noindent \textbf{Solution: }}
\providecommand{\dec}[2]{\ensuremath{\overset{#1}{\underset{#2}{\gtrless}}}}
\newcommand{\myvec}[1]{\ensuremath{\begin{pmatrix}#1\end{pmatrix}}}
\let\vec\mathbf

\lstset{
%language=C,
frame=single,
breaklines=true,
columns=fullflexible
}

\numberwithin{equation}{section}

\lstset{
  language=Python,
  basicstyle=\ttfamily\small,
  keywordstyle=\color{blue},
  stringstyle=\color{orange},
  numbers=left,
  numberstyle=\tiny\color{gray},
  breaklines=true,
  showstringspaces=false
}

\title{Problem 12.453}
\author{ee25btech11023-Venkata Sai}

\date{\today}
\begin{document}

\begin{frame}
\titlepage
\end{frame}

\section*{Outline}
\begin{frame}
\tableofcontents
\end{frame}

\section{Problem}

\begin{frame}
\frametitle{Problem}
A 3 $\times$ 3 matrix $\vec{P}$ is such that, $\vec{P}^3$ = $\vec{P}$. Then the eigenvalues of $\vec{P}$ are\\
\end{frame}
%\subsection{Literature}
\section{Solution}


\subsection{Formula}
\begin{frame}
\frametitle{Formula}
  Let $\lambda$ be the eigen value of the matrix $\vec{P}$ and $\vec{v}$ be the corresponding eigen vector
 \begin{align}
     \vec{P}\vec{v}=\lambda\vec{v} \\
 \end{align}
 Multiplying both sides with vector $\vec{P}$
 \begin{align}
     \vec{P}\brak{\vec{P}\vec{v}}&=\vec{P}\brak{\lambda\vec{v}} \\
     \vec{P}^{2}\vec{v}&=\lambda\brak{\vec{P}\vec{v}}\\
     \implies \vec{P}^{2}\vec{v}&=\lambda\brak{\lambda\vec{v}} =\lambda^2\vec{v}
 \end{align}
 Hence
 \begin{align}
     \vec{P}^{n}\vec{v}=\lambda^n\vec{v} \\
     \vec{P}^{3}\vec{v}=\lambda^3\vec{v}
 \end{align}
\end{frame}
\subsection{Conclusion}
\begin{frame}
\frametitle{Conclusion}
  \begin{align}
     \vec{P}^3&=\vec{P} \\
     \vec{P}^3&-\vec{P}=0 \\
     \lambda^3\vec{v}-\lambda\vec{v}=0  &\implies \brak{\lambda^3-\lambda}\vec{v}=0 \\
     \lambda^3-\lambda=0 &\implies \lambda\brak{\lambda^2-1}=0 \\
     \lambda=0\ &\text{and}\ \lambda^2-1=0 \\
     \lambda=0\ &\text{and}\ \lambda=\pm1
 \end{align}
 Hence the eigen values of $\vec{P} = 0,1,-1$
 \end{frame}
\end{document}
