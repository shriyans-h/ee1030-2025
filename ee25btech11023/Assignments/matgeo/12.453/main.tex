\let\negmedspace\undefined
\let\negthickspace\undefined
\documentclass[journal]{IEEEtran}
\usepackage[a5paper, margin=10mm, onecolumn]{geometry}
%\usepackage{lmodern} % Ensure lmodern is loaded for pdflatex
\usepackage{tfrupee} % Include tfrupee package

\setlength{\headheight}{1cm} % Set the height of the header box
\setlength{\headsep}{0mm}     % Set the distance between the header box and the top of the text

\usepackage{gvv-book}
\usepackage{gvv}
\usepackage{cite}
\usepackage{amsmath,amssymb,amsfonts,amsthm}
\usepackage{algorithmic}
\usepackage{graphicx}
\usepackage{textcomp}
\usepackage{xcolor}
\usepackage{txfonts}
\usepackage{listings}
\usepackage{enumitem}
\usepackage{mathtools}
\usepackage{gensymb}
\usepackage{comment}
\usepackage[breaklinks=true]{hyperref}
\usepackage{tkz-euclide} 
\usepackage{listings}
% \usepackage{gvv}                                        
\def\inputGnumericTable{}                                 
\usepackage[latin1]{inputenc}                                
\usepackage{color}                                            
\usepackage{array}                                            
\usepackage{longtable}                                       
\usepackage{calc}                                             
\usepackage{multirow}                                         
\usepackage{hhline}                                           
\usepackage{ifthen}                                           
\usepackage{lscape}
\begin{document}

\bibliographystyle{IEEEtran}

\title{12.453}
\author{EE25BTECH11023 - Venkata Sai}
% \maketitle
% \newpage
% \bigskip
\maketitle 
\renewcommand{\thefigure}{\theenumi}
\renewcommand{\thetable}{\theenumi}
\setlength{\intextsep}{10pt} % Space between text and floats

\numberwithin{align}{enumi}
\numberwithin{figure}{enumi}
\renewcommand{\thetable}{\theenumi}
\vspace{-1em}
\textbf{Question:}  \\
A 3 $\times$ 3 matrix $\vec{P}$ is such that, $\vec{P}^3$ = $\vec{P}$. Then the eigenvalues of $\vec{P}$ are\\
\textbf{Solution:}  \\
  \begin{align}
     \vec{P}&=\vec{Q}\vec{D}\vec{Q}^{-1} \\
      \vec{P}^2&=\brak{\vec{Q}\vec{D}\vec{Q}^{-1}}^2 \\
      &=\vec{Q}\vec{D}\vec{Q}^{-1}\vec{Q}\vec{D}\vec{Q}^{-1}\\
      &=\vec{Q}\vec{D}\vec{I}\vec{D}\vec{Q}^{-1}\\
      &=\vec{Q}\vec{D}^2\vec{Q}^{-1}
 \end{align}
 where $\vec{D}$ is the Diagonal matrix containing eigen values
 \begin{align}
     \vec{P}^k&=\vec{Q}\vec{D}^k\vec{Q}^{-1} \\
     \vec{P}^3&=\vec{Q}\vec{D}^3\vec{Q}^{-1} 
     \end{align}
Given 
\begin{align}
    \vec{P}^3 &= \vec{P} \\
      \vec{P}^{3}& - \vec{P} = 0
      \end{align}
From \brak{1} and \brak{7}
      \begin{align}
   \vec{Q}\vec{D}^{3}\vec{Q}^{-1}-\vec{Q}\vec{D}\vec{Q}^{-1}=0 \\
   \vec{Q}\brak{\vec{D}^{3}-\vec{D}^{}}\vec{Q}^{-1}=0 \\
   \implies \brak{\vec{D}^{3}-\vec{D}^{}}=0\\
   \implies \brak{\vec{\lambda}^{3}-\vec{\lambda}^{}}=0 
\end{align}
where $\lambda$ are the eigen values
\begin{align}
    \lambda^{}\brak{\lambda^2-1}=0&\implies
    \lambda^{}=0\ \text{or}\ \lambda^{2}-1=0 \\
     \lambda=0\ &\text{or}\ \lambda=\pm1
\end{align}
 \end{document}
