\let\negmedspace\undefined
\let\negthickspace\undefined
\documentclass[journal]{IEEEtran}
\usepackage[a5paper, margin=10mm, onecolumn]{geometry}
%\usepackage{lmodern} % Ensure lmodern is loaded for pdflatex
\usepackage{tfrupee} % Include tfrupee package

\setlength{\headheight}{1cm} % Set the height of the header box
\setlength{\headsep}{0mm}     % Set the distance between the header box and the top of the text

\usepackage{gvv-book}
\usepackage{gvv}
\usepackage{cite}
\usepackage{amsmath,amssymb,amsfonts,amsthm}
\usepackage{algorithmic}
\usepackage{graphicx}
\usepackage{textcomp}
\usepackage{xcolor}
\usepackage{txfonts}
\usepackage{listings}
\usepackage{enumitem}
\usepackage{mathtools}
\usepackage{gensymb}
\usepackage{comment}
\usepackage[breaklinks=true]{hyperref}
\usepackage{tkz-euclide} 
\usepackage{listings}
% \usepackage{gvv}                                        
\def\inputGnumericTable{}                                 
\usepackage[latin1]{inputenc}                                
\usepackage{color}                                            
\usepackage{array}                                            
\usepackage{longtable}                                       
\usepackage{calc}                                             
\usepackage{multirow}                                         
\usepackage{hhline}                                           
\usepackage{ifthen}                                           
\usepackage{lscape}
\begin{document}

\bibliographystyle{IEEEtran}

\title{4.11.1}
\author{EE25BTECH11023 - Venkata Sai}
% \maketitle
% \newpage
% \bigskip
{\let\newpage\relax\maketitle}

\renewcommand{\thefigure}{\theenumi}
\renewcommand{\thetable}{\theenumi}
\setlength{\intextsep}{10pt} % Space between text and floats


\numberwithin{align}{enumi}
\numberwithin{figure}{enumi}
\renewcommand{\thetable}{\theenumi}

\textbf{Question:}  \\
Find the coordinates of the point where the line  $\frac{x-1}{3} = \frac{y+4}{7} = \frac{z+4}{2}$ cuts the XY-plane 

\textbf{Solution:}  
The line equation is  
\begin{align}
\vec{r} = \vec{a} + t \vec{b}
\end{align}
where $\vec{a}$ is the point on line and $\vec{b}$ is the direction vector
\begin{align}
\vec{a} = \myvec{1 \\ -4 \\ -4}\ \text{and}\
\vec{b} = \myvec{3 \\ 7 \\ 2}
\end{align}
The normal vector to XY plane is
\begin{align}
\vec{n} = \myvec{0 \\ 0 \\ 1}
\end{align}
The plane equation of the XY-plane is  
\begin{align}
\vec{n}^T \vec{x} = 0 \implies \myvec{0 & 0 & 1}\vec{x}=0
\end{align}
Substituting the line into the plane equation gives  
\begin{align}
\vec{n}^T\brak{ \vec{a} + t\vec{b}} = 0  \\
\vec{n}^T \vec{a} + t\brak{\vec{n}^T \vec{b}} = 0 \\
t\brak{\vec{n}^T \vec{b}}=-\vec{n}^T \vec{a} 
\end{align}
\begin{align}
t =-\frac{\vec{n}^T \vec{a}}{\vec{n}^T \vec{b}}
\end{align}
\begin{align}
t =-\frac{\myvec{0 & 0 & 1}\myvec{1 \\ -4 \\ -4}}{\myvec{0 & 0 & 1}\myvec{3 \\ 7 \\ 2}} 
\end{align}
\begin{align}
t=\frac{4}{2}\\
t=2
\end{align}
The intersection point is  
\begin{align}
\vec{r} = \vec{a} + t\vec{b} 
= \myvec{1 \\ -4 \\ -4} + 2\myvec{3 \\ 7 \\ 2}
\end{align} 
\begin{align}
\vec{r} = \myvec{7 \\ 10 \\ 0}
\end{align}
\begin{figure}[h!]
   \centering
   \includegraphics[width=0.7\columnwidth]{figs/fig1.png}
   \caption{}
   \label{Figure}
\end{figure}
\end{document}  