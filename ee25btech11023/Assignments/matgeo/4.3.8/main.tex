\let\negmedspace\undefined
\let\negthickspace\undefined
\documentclass[journal]{IEEEtran}
\usepackage[a5paper, margin=10mm, onecolumn]{geometry}
%\usepackage{lmodern} % Ensure lmodern is loaded for pdflatex
\usepackage{tfrupee} % Include tfrupee package

\setlength{\headheight}{1cm} % Set the height of the header box
\setlength{\headsep}{0mm}     % Set the distance between the header box and the top of the text

\usepackage{gvv-book}
\usepackage{gvv}
\usepackage{cite}
\usepackage{amsmath,amssymb,amsfonts,amsthm}
\usepackage{algorithmic}
\usepackage{graphicx}
\usepackage{textcomp}
\usepackage{xcolor}
\usepackage{txfonts}
\usepackage{listings}
\usepackage{enumitem}
\usepackage{mathtools}
\usepackage{gensymb}
\usepackage{comment}
\usepackage[breaklinks=true]{hyperref}
\usepackage{tkz-euclide} 
\usepackage{listings}
% \usepackage{gvv}                                        
\def\inputGnumericTable{}                                 
\usepackage[latin1]{inputenc}                                
\usepackage{color}                                            
\usepackage{array}                                            
\usepackage{longtable}                                       
\usepackage{calc}                                             
\usepackage{multirow}                                         
\usepackage{hhline}                                           
\usepackage{ifthen}                                           
\usepackage{lscape}
\begin{document}

\bibliographystyle{IEEEtran}

\title{4.3.8}
\author{EE25BTECH11023 - Venkata Sai}
% \maketitle
% \newpage
% \bigskip
{\let\newpage\relax\maketitle}

\renewcommand{\thefigure}{\theenumi}
\renewcommand{\thetable}{\theenumi}
\setlength{\intextsep}{10pt} % Space between text and floats


\numberwithin{align}{enumi}
\numberwithin{figure}{enumi}
\renewcommand{\thetable}{\theenumi}


\textbf{Question}:\newline
The vector equation of the line
\begin{align}
\frac{x-5}{3} = \frac{y+4}{7} = \frac{z-6}{2}
\end{align}
is

\textbf{Solution: }
A point on the line  is given as
\begin{align}
\myvec{x-5 \\y+4 \\z-6}=\myvec{0\\0\\0}\\
\vec{a}=\myvec{x\\y\\z}=\myvec{5\\-4\\6}
\end{align}
The direction vectors of given line are
\begin{align}
    \vec{b}=\myvec{3\\7\\2}
\end{align}
The vector equation of a line is given as
\begin{align}
    \vec{r}=\vec{a}+t\vec{b} \\
    \vec{r}=\myvec{5\\-4\\6}+t\myvec{3\\7\\2}
\end{align}

\begin{figure}[h!]
   \centering
   \includegraphics[width=0.7\columnwidth]{figs/fig1.png}
   \caption{}
   \label{Figure}
\end{figure}
\end{document}  
