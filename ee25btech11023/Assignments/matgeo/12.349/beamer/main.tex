\documentclass{beamer}
\usepackage[utf8]{inputenc}

\usetheme{Boadilla}
\usecolortheme{lily}
\usepackage{amsmath,amssymb,amsfonts,amsthm}
\usepackage{mathtools}
\usepackage{txfonts}
\usepackage{tkz-euclide}
\usepackage{listings}
\usepackage{multicol}
\usepackage{adjustbox}
\usepackage{array}
\usepackage{tabularx}
\usepackage{lmodern}
\usepackage{gvv}
\usepackage{circuitikz}
\usepackage{tikz}
\usepackage{graphicx}

\setbeamertemplate{footline}
{
  \leavevmode%
  \hbox{%
  \begin{beamercolorbox}[wd=\paperwidth,ht=2.25ex,dp=1ex,right]{author in head/foot}%
    \insertframenumber{} / \inserttotalframenumber\hspace*{2ex} 
  \end{beamercolorbox}}%
  \vskip0pt%
}

\usepackage{tcolorbox}
\tcbuselibrary{minted,breakable,xparse,skins}




\providecommand{\nCr}[2]{\,^{#1}C_{#2}} % nCr
\providecommand{\nPr}[2]{\,^{#1}P_{#2}} % nPr
\providecommand{\mbf}{\mathbf}
\providecommand{\pr}[1]{\ensuremath{\Pr\left(#1\right)}}
\providecommand{\qfunc}[1]{\ensuremath{Q\left(#1\right)}}
\providecommand{\sbrak}[1]{\ensuremath{{}\left[#1\right]}}
\providecommand{\lsbrak}[1]{\ensuremath{{}\left[#1\right.}}
\providecommand{\rsbrak}[1]{\ensuremath{{}\left.#1\right]}}
\providecommand{\brak}[1]{\ensuremath{\left(#1\right)}}
\providecommand{\lbrak}[1]{\ensuremath{\left(#1\right.}}
\providecommand{\rbrak}[1]{\ensuremath{\left.#1\right)}}
\providecommand{\cbrak}[1]{\ensuremath{\left\{#1\right\}}}
\providecommand{\lcbrak}[1]{\ensuremath{\left\{#1\right.}}
\providecommand{\rcbrak}[1]{\ensuremath{\left.#1\right\}}}
\theoremstyle{remark}
\newcommand{\sgn}{\mathop{\mathrm{sgn}}}
\providecommand{\abs}[1]{\left\vert#1\right\vert}
\providecommand{\res}[1]{\Res\displaylimits_{#1}} 
\providecommand{\norm}[1]{\lVert#1\rVert}
\providecommand{\mtx}[1]{\mathbf{#1}}
\providecommand{\mean}[1]{E\left[ #1 \right]}
\providecommand{\fourier}{\overset{\mathcal{F}}{ \rightleftharpoons}}
%\providecommand{\hilbert}{\overset{\mathcal{H}}{ \rightleftharpoons}}
\providecommand{\system}{\overset{\mathcal{H}}{ \longleftrightarrow}}
	%\newcommand{\solution}[2]{\textbf{Solution:}{#1}}
%\newcommand{\solution}{\noindent \textbf{Solution: }}
\providecommand{\dec}[2]{\ensuremath{\overset{#1}{\underset{#2}{\gtrless}}}}
\newcommand{\myvec}[1]{\ensuremath{\begin{pmatrix}#1\end{pmatrix}}}
\let\vec\mathbf

\lstset{
%language=C,
frame=single, 
breaklines=true,
columns=fullflexible
}

\numberwithin{equation}{section}

\lstset{
  language=Python,
  basicstyle=\ttfamily\small,
  keywordstyle=\color{blue},
  stringstyle=\color{orange},
  numbers=left,
  numberstyle=\tiny\color{gray},
  breaklines=true,
  showstringspaces=false
}

\title{Problem 12.349}
\author{ee25btech11023-Venkata Sai}

\date{\today} 
\begin{document}

\begin{frame}
\titlepage
\end{frame}

\section*{Outline}
\begin{frame}
\tableofcontents
\end{frame}

\section{Problem}

\begin{frame}
\frametitle{Problem}
Let $T_1 , T_2 : \mathbb{R}_5 \rightarrow \mathbb{R}_3 $ be linear transformations such that rank($T_1$) = 3 and
nullity$\brak{T_2}$ = 3. Let $T_3 : \mathbb{R}_3 \rightarrow \mathbb{R}_3$ be a linear transformation such that $T_3 \circ T_1 = T_2$ .
Then rank$\brak{T_3}$ is \dots \hfill (MA 2014)
\end{frame}
%\subsection{Literature}
\section{Solution}

 
\subsection{Given}
\begin{frame}
\frametitle{Given}
According to Rank-Nullity theorem,\\
For a linear transformation $T : \mathbb{R}_m \rightarrow \mathbb{R}_n $
\begin{align}
    \text{rank}\brak{T}+\text{nullity}\brak{T}=\dim\brak{\text{domain}}
\end{align}
where $\dim\mathbb{R}_m $ is the dimension of the domain i.e vector space $\mathbb{R}_m $ \\ \newline
Given $T_2 : \mathbb{R}_5 \rightarrow \mathbb{R}_3 $ and nullity$\brak{T_2}$=3 
\begin{align}
    \text{rank}\brak{T_2}+\text{nullity}&\brak{T_2}=\dim\mathbb{R}_5 \\
    \text{rank}\brak{T_2}&+3=5 \\
    \text{rank}\brak{T_2} &= 2
\end{align}
Given $T_1 : \mathbb{R}_5 \rightarrow \mathbb{R}_3 $ and rank($T_1$)=3
\begin{align}
    &\dim\brak{\text{Co-domain}} =3 \\
    &\text{rank}\brak{T_1}=\dim\brak{\text{Co-domain}} 
\end{align}
\end{frame}
\subsection{Conclusion}
\begin{frame}
\frametitle{Conclusion}
It is onto and hence
\begin{align}
   \dim\brak{\text{Im}\brak{T_1}}&= \dim\brak{\text{Co-domain}} \\
   \dim\brak{\text{Im}\brak{T_1}}&= 3 \implies \text{Im}\brak{T_1}=\mathbb{R}_3
\end{align}
where $\text{Im}\brak{T}$ is the Image space of the linear transformation $T$ \\
Given $T_3 : \mathbb{R}_3 \rightarrow \mathbb{R}_3 $ 
\begin{align}
    T_3 \circ T_1 &= T_2\\
   \brak{ T_3 \circ T_1}\brak{\mathbb{R}_5} &= \text{Im}\brak{T_2} \\
   T_3 \brak{T_1\brak{R_5}} &= \text{Im}\brak{T_2} \\
   T_3 \brak{\text{Im}\brak{T_1}} &= \text{Im}\brak{T_2} \\
   T_3 \brak{\mathbb{R}_3} &= \text{Im}\brak{T_2} \\
   \text{Im}\brak{T_3} &= \text{Im}\brak{T_2} \\
   \implies  \text{rank}(T_3) &= \text{rank}(T_2)
\end{align}
From \brak{4}
\begin{align}
    \text{rank}(T_3)=2
\end{align}
 \end{frame}
\end{document}
