\let\negmedspace\undefined
\let\negthickspace\undefined
\documentclass[journal]{IEEEtran}
\usepackage[a5paper, margin=10mm, onecolumn]{geometry}
%\usepackage{lmodern} % Ensure lmodern is loaded for pdflatex
\usepackage{tfrupee} % Include tfrupee package

\setlength{\headheight}{1cm} % Set the height of the header box
\setlength{\headsep}{0mm}     % Set the distance between the header box and the top of the text

\usepackage{gvv-book}
\usepackage{gvv}
\usepackage{cite}
\usepackage{amsmath,amssymb,amsfonts,amsthm}
\usepackage{algorithmic}
\usepackage{graphicx}
\usepackage{textcomp}
\usepackage{xcolor}
\usepackage{txfonts}
\usepackage{listings}
\usepackage{enumitem}
\usepackage{mathtools}
\usepackage{gensymb}
\usepackage{comment}
\usepackage[breaklinks=true]{hyperref}
\usepackage{tkz-euclide} 
\usepackage{listings}
% \usepackage{gvv}                                        
\def\inputGnumericTable{}                                 
\usepackage[latin1]{inputenc}                                
\usepackage{color}                                            
\usepackage{array}                                            
\usepackage{longtable}                                       
\usepackage{calc}                                             
\usepackage{multirow}                                         
\usepackage{hhline}                                           
\usepackage{ifthen}                                           
\usepackage{lscape}
\begin{document}

\bibliographystyle{IEEEtran}

\title{4.6.11}
\author{EE25BTECH11023 - Venkata Sai}
% \maketitle
% \newpage
% \bigskip
{\let\newpage\relax\maketitle}

\renewcommand{\thefigure}{\theenumi}
\renewcommand{\thetable}{\theenumi}
\setlength{\intextsep}{10pt} % Space between text and floats


\numberwithin{align}{enumi}
\numberwithin{figure}{enumi}
\renewcommand{\thetable}{\theenumi}

\textbf{Question}:\newline
Find the equation of the line passing through the point $(1, -3, 2)$ and parallel to the line
\begin{align}
 \vec{r} = \brak{2 + \lambda}\hat{i} + \lambda\hat{j} + \brak{2\lambda - 1}\hat{k} 
\end{align}


\textbf{Solution: }
Given line is
\begin{align}
 \vec{r} = \myvec{2+\lambda\\\lambda\\2\lambda-1} \quad \vec{a}=\myvec{1\\-3\\2}
\end{align}
The vector equation of given line is given by
\begin{align}
\vec{r}=\myvec{2\\0\\-1}+\lambda\myvec{1\\1\\2}
\end{align}
The direction vectors of given line are
\begin{align}
    \vec{m}=\myvec{1\\1\\2}
\end{align}
The lines with direction vectors $\vec{m}$ and $\vec{n}$ are parallel if
\begin{align}
\vec{m}=\vec{n}  \implies  \vec{n}=\myvec{1\\1\\2}
\end{align}
The equation of a line is given by
\begin{align}
\vec{r}=\vec{a}+t\vec{n}
\end{align}
\begin{align}
\vec{r}=\myvec{1\\-3\\2}+t\myvec{1\\1\\2}
\end{align}

\begin{figure}[h!]
   \centering
   \includegraphics[width=0.7\columnwidth]{figs/fig1.png}
   \caption{}
   \label{Figure}
\end{figure}
\end{document}  
