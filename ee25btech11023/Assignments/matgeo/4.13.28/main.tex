\let\negmedspace\undefined
\let\negthickspace\undefined
\documentclass[journal]{IEEEtran}
\usepackage[a5paper, margin=10mm, onecolumn]{geometry}
%\usepackage{lmodern} % Ensure lmodern is loaded for pdflatex
\usepackage{tfrupee} % Include tfrupee package

\setlength{\headheight}{1cm} % Set the height of the header box
\setlength{\headsep}{0mm}     % Set the distance between the header box and the top of the text

\usepackage{gvv-book}
\usepackage{gvv}
\usepackage{cite}
\usepackage{amsmath,amssymb,amsfonts,amsthm}
\usepackage{algorithmic}
\usepackage{graphicx}
\usepackage{textcomp}
\usepackage{xcolor}
\usepackage{txfonts}
\usepackage{listings}
\usepackage{enumitem}
\usepackage{mathtools}
\usepackage{gensymb}
\usepackage{comment}
\usepackage[breaklinks=true]{hyperref}
\usepackage{tkz-euclide} 
\usepackage{listings}
% \usepackage{gvv}                                        
\def\inputGnumericTable{}                                 
\usepackage[latin1]{inputenc}                                
\usepackage{color}                                            
\usepackage{array}                                            
\usepackage{longtable}                                       
\usepackage{calc}                                             
\usepackage{multirow}                                         
\usepackage{hhline}                                           
\usepackage{ifthen}                                           
\usepackage{lscape}
\begin{document}

\bibliographystyle{IEEEtran}

\title{4.13.28}
\author{EE25BTECH11023 - Venkata Sai}
% \maketitle
% \newpage
% \bigskip
\maketitle \vspace{-1cm}
\renewcommand{\thefigure}{\theenumi}
\renewcommand{\thetable}{\theenumi}
\setlength{\intextsep}{10pt} % Space between text and floats

\numberwithin{align}{enumi}
\numberwithin{figure}{enumi}
\renewcommand{\thetable}{\theenumi}

\textbf{Question:}  \\
Slope of a line passing through $\vec{P}\brak{2,3}$ and intersecting the line $x+y=7$ at a distance of 4 units from $\vec{P}$, is

\textbf{Solution:}  
Given  
\begin{align}
\vec{P}=\myvec{2\\3}
\end{align}
Equation of a line through $\vec{P}$ and having slope $m$ is
\begin{align}
 \myvec{-m & 1}\myvec{x-2 \\ y-3}=0  
 \end{align}
 \begin{align}
 \myvec{-m & 1}\brak{\myvec{x\\y}-\myvec{2\\3}}=0 \implies \myvec{-m & 1}\myvec{x\\y}=\myvec{-m & 1}\myvec{2\\3}
\end{align}
\begin{align}
\myvec{-m & 1}\myvec{x\\y}=3-2m
\end{align}
\begin{align}
  x+y=7 \implies  \myvec{1 & 1}\myvec{x\\y}=7
\end{align}
\begin{align}
\myvec{-m & 1 & 3-2m \\ 1 &1 & 7} \xleftrightarrow{R_1\leftrightarrow R_2} \myvec{1 &1 & 7 \\ -m & 1 & 3-2m} \xleftrightarrow{R_2\rightarrow R_2+mR_1} 
\myvec{1 & 1& 7 \\0 & 1+m & 3+5m}
\end{align}
\begin{align}
y=\frac{3+5m}{1+m}
\end{align}
Given the point is at a distance of 4 units from point $\vec{P}$
\begin{align}
  \norm{\myvec{x\\y}-\myvec{2\\3}}=4 \implies \norm{\myvec{x-2\\y-3}}=4
\end{align}
\begin{align}
\sqrt{\brak{x-2}^2+{\brak{y-3}^2}}=4  \\
\sqrt{\brak{7-y-2}^2+{\brak{y-3}^2}}=4  
\end{align}
\begin{align}
\brak{5-y}^2+{\brak{y-3}^2}=4^2=16 \\
25+y^2-10y+y^2+9-6y=16 \\
2y^2-16y+18=0 \implies y^2-8y+9=0 \\
y^2-8y+9+16=16 \implies \brak{y-4}^2=7 
\end{align}
\begin{align}
&y-4=\pm\sqrt7 \\
y=4&-\sqrt{7}\ \brak{\text{or}}\  4+\sqrt{7}
\end{align}
\begin{align}
\frac{3+5m}{1+m} = \frac{3+5m+2-2}{1+m}= \frac{5+5m-2}{1+m}=\frac{5\brak{m+1}-2}{1+m}=5-\frac{2}{1+m}
\end{align}
\begin{align}
5-\frac{2}{1+m}= 4-\sqrt{7}\quad &\brak{\text{or}}\ \quad  5-\frac{2}{1+m}=4+\sqrt{7} \\
\frac{2}{1+m} = 1+\sqrt{7}\quad &\brak{\text{or}}\ \quad \frac{2}{1+m} =1-\sqrt{7} \\
1+m = \frac{2}{1+\sqrt{7}}\quad &\brak{\text{or}}\ \quad 1+m = \frac{2}{1-\sqrt{7}}  \\
m=\frac{2-1-\sqrt{7}}{1+\sqrt{7}}\quad &\brak{\text{or}}\ \quad m = \frac{2-1+\sqrt{7}}{1-\sqrt{7}}  \\
m=\frac{1-\sqrt{7}}{1+\sqrt{7}} \quad &\brak{\text{or}}\ \quad m=\frac{1+\sqrt{7}}{1-\sqrt{7}}
\end{align}
\begin{figure}[h!]
   \centering
   \includegraphics[width=0.7\columnwidth]{figs/fig1.png}
   \caption{}
   \label{Figure}
\end{figure}
\end{document}  