\let\negmedspace\undefined
\let\negthickspace\undefined
\documentclass[journal]{IEEEtran}
\usepackage[a5paper, margin=10mm, onecolumn]{geometry}
%\usepackage{lmodern} % Ensure lmodern is loaded for pdflatex
\usepackage{tfrupee} % Include tfrupee package

\setlength{\headheight}{1cm} % Set the height of the header box
\setlength{\headsep}{0mm}     % Set the distance between the header box and the top of the text

\usepackage{gvv-book}
\usepackage{gvv}
\usepackage{cite}
\usepackage{amsmath,amssymb,amsfonts,amsthm}
\usepackage{algorithmic}
\usepackage{graphicx}
\usepackage{textcomp}
\usepackage{xcolor}
\usepackage{txfonts}
\usepackage{listings}
\usepackage{enumitem}
\usepackage{mathtools}
\usepackage{gensymb}
\usepackage{comment}
\usepackage[breaklinks=true]{hyperref}
\usepackage{tkz-euclide} 
\usepackage{listings}
% \usepackage{gvv}                                        
\def\inputGnumericTable{}                                 
\usepackage[latin1]{inputenc}                                
\usepackage{color}                                            
\usepackage{array}                                            
\usepackage{longtable}                                       
\usepackage{calc}                                             
\usepackage{multirow}                                         
\usepackage{hhline}                                           
\usepackage{ifthen}                                           
\usepackage{lscape}
\begin{document}

\bibliographystyle{IEEEtran}

\title{4.13.28}
\author{EE25BTECH11023 - Venkata Sai}
% \maketitle
% \newpage
% \bigskip
\maketitle \vspace{-1cm}
\renewcommand{\thefigure}{\theenumi}
\renewcommand{\thetable}{\theenumi}
\setlength{\intextsep}{10pt} % Space between text and floats

\numberwithin{align}{enumi}
\numberwithin{figure}{enumi}
\renewcommand{\thetable}{\theenumi}

\textbf{Question:}  \\
Slope of a line passing through $\vec{P}\brak{2,3}$ and intersecting the line $x+y=7$ at a distance of 4 units from $\vec{P}$, is

\textbf{Solution:}  
Given  
\begin{align}
\vec{P}=\myvec{2\\3}
\end{align}
Equation of a line through $\vec{P}$ and having slope $m$ is
\begin{align}
 \myvec{-m & 1}\myvec{x-2 \\ y-3}=0  
 \end{align}
 \begin{align}
 \myvec{-m & 1}\brak{\myvec{x\\y}-\myvec{2\\3}}=0 \implies \myvec{-m & 1}\myvec{x\\y}=\myvec{-m & 1}\myvec{2\\3}
\end{align}
\begin{align}
\myvec{-m & 1}\myvec{x\\y}=3-2m
\end{align}
\begin{align}
  x+y=7 \implies  \myvec{1 & 1}\myvec{x\\y}=7
\end{align}
\begin{align}
\myvec{-m & 1 & 3-2m \\ 1 &1 & 7} \xleftrightarrow{R_1\leftrightarrow R_2} \myvec{1 &1 & 7 \\ -m & 1 & 3-2m} \xleftrightarrow{R_2\rightarrow R_2+mR_1} 
\myvec{1 & 1& 7 \\0 & 1+m & 3+5m}
\end{align}
\begin{align}
y=&\frac{3+5m}{1+m}\\
x+y=7\implies x=&7-y \implies x=7-\frac{3+5m}{1+m}
\end{align}
\begin{align}
x=\frac{7+7m-3-5m}{1+m}=\frac{4+2m}{1+m}
\end{align}
Given the point is at a distance of 4 units from point $\vec{P}$
\begin{align}
  \norm{\myvec{x\\y}-\myvec{2\\3}}=4 \implies \norm{\myvec{\frac{4+2m}{1+m}-2\\\frac{3+5m}{1+m}-3}}=4
\end{align}
\begin{align}
\norm{\myvec{\frac{4+2m-2-2m}{1+m}\\ \frac{3+5m-3-3m}{1+m}}}=\norm{\myvec{\frac{2}{1+m}\\ \frac{2m}{1+m}}}=4\\
\sqrt{\brak{\frac{2}{1+m}}^2+{\brak{\frac{2m}{1+m}}^2}}=4 
\end{align}
\begin{align}
\frac{4+4m^2}{\brak{1+m}^2}&=4^2=16 \\
4\brak{1+m^2}=16\brak{1+m^2+2m} &\implies \brak{1+m^2}=4\brak{1+m^2+2m} 
\end{align}
\begin{align}
4+4m^2+8m=1&+m^2 \implies 3m^2+8m+3=0 \\
m^2+&\frac{8m}{3}+1=0 \\
m^2+\frac{8m}{3}&+1+\brak{\frac{4}{3}}^2=\brak{\frac{4}{3}}^2 \\
\brak{m+\frac{4}{3}}^2&=\frac{16}{9}-1=\frac{7}{9} \\
m+\frac{4}{3}&=\pm\frac{\sqrt{7}}{3} \\
m=\frac{-4+\sqrt{7}}{3}\ &\text{or}\ \frac{-4-\sqrt{7}}{3}
\end{align}
According to options 
\begin{align}
    \frac{-4+\sqrt{7}}{3}=\frac{-8+2\sqrt{7}}{6}=\frac{1-\sqrt7}{1+\sqrt{7}}
\end{align}
\begin{figure}[h!]
   \centering
   \includegraphics[width=0.7\columnwidth]{figs/fig1.png}
   \caption{}
   \label{Figure}
\end{figure}
\end{document}  
