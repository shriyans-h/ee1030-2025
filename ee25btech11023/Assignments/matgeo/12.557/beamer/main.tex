\documentclass{beamer}
\usepackage[utf8]{inputenc}

\usetheme{Boadilla}
\usecolortheme{lily}
\usepackage{amsmath,amssymb,amsfonts,amsthm}
\usepackage{mathtools}
\usepackage{txfonts}
\usepackage{tkz-euclide}
\usepackage{listings}
\usepackage{multicol}
\usepackage{adjustbox}
\usepackage{array}
\usepackage{tabularx}
\usepackage{lmodern}
\usepackage{gvv}
\usepackage{circuitikz}
\usepackage{tikz}
\usepackage{graphicx}

\setbeamertemplate{footline}
{
  \leavevmode%
  \hbox{%
  \begin{beamercolorbox}[wd=\paperwidth,ht=2.25ex,dp=1ex,right]{author in head/foot}%
    \insertframenumber{} / \inserttotalframenumber\hspace*{2ex}
  \end{beamercolorbox}}%
  \vskip0pt%
}

\usepackage{tcolorbox}
\tcbuselibrary{minted,breakable,xparse,skins}




\providecommand{\nCr}[2]{\,^{#1}C_{#2}} % nCr
\providecommand{\nPr}[2]{\,^{#1}P_{#2}} % nPr
\providecommand{\mbf}{\mathbf}
\providecommand{\pr}[1]{\ensuremath{\Pr\left(#1\right)}}
\providecommand{\qfunc}[1]{\ensuremath{Q\left(#1\right)}}
\providecommand{\sbrak}[1]{\ensuremath{{}\left[#1\right]}}
\providecommand{\lsbrak}[1]{\ensuremath{{}\left[#1\right.}}
\providecommand{\rsbrak}[1]{\ensuremath{{}\left.#1\right]}}
\providecommand{\brak}[1]{\ensuremath{\left(#1\right)}}
\providecommand{\lbrak}[1]{\ensuremath{\left(#1\right.}}
\providecommand{\rbrak}[1]{\ensuremath{\left.#1\right)}}
\providecommand{\cbrak}[1]{\ensuremath{\left\{#1\right\}}}
\providecommand{\lcbrak}[1]{\ensuremath{\left\{#1\right.}}
\providecommand{\rcbrak}[1]{\ensuremath{\left.#1\right\}}}
\theoremstyle{remark}
\newcommand{\sgn}{\mathop{\mathrm{sgn}}}
\providecommand{\abs}[1]{\left\vert#1\right\vert}
\providecommand{\res}[1]{\Res\displaylimits_{#1}}
\providecommand{\norm}[1]{\lVert#1\rVert}
\providecommand{\mtx}[1]{\mathbf{#1}}
\providecommand{\mean}[1]{E\left[ #1 \right]}
\providecommand{\fourier}{\overset{\mathcal{F}}{ \rightleftharpoons}}
%\providecommand{\hilbert}{\overset{\mathcal{H}}{ \rightleftharpoons}}
\providecommand{\system}{\overset{\mathcal{H}}{ \longleftrightarrow}}
	%\newcommand{\solution}[2]{\textbf{Solution:}{#1}}
%\newcommand{\solution}{\noindent \textbf{Solution: }}
\providecommand{\dec}[2]{\ensuremath{\overset{#1}{\underset{#2}{\gtrless}}}}
\newcommand{\myvec}[1]{\ensuremath{\begin{pmatrix}#1\end{pmatrix}}}
\let\vec\mathbf

\lstset{
%language=C,
frame=single,
breaklines=true,
columns=fullflexible
}

\numberwithin{equation}{section}

\lstset{
  language=Python,
  basicstyle=\ttfamily\small,
  keywordstyle=\color{blue},
  stringstyle=\color{orange},
  numbers=left,
  numberstyle=\tiny\color{gray},
  breaklines=true,
  showstringspaces=false
}

\title{Problem 12.557}
\author{ee25btech11023-Venkata Sai}

\date{\today}
\begin{document}

\begin{frame}
\titlepage
\end{frame}

\section*{Outline}
\begin{frame}
\tableofcontents
\end{frame}

\section{Problem}

\begin{frame}
\frametitle{Problem}
Let $\vec{A}=\myvec{5&-3\\6&-4}$.Then the trace of $\vec{A}^{1000}$ equals\\
\end{frame}
%\subsection{Literature}
\section{Solution}


\subsection{Given}
\begin{frame}
\frametitle{Given}
  Given
  \begin{align}
      \vec{A}=\myvec{5&-3\\6&-4} \\
  \end{align}
  To find eigen values
  \begin{align}
      |\vec{A}-\lambda\vec{I}|=0
      \end{align}
      \begin{align}
            \mydet{\myvec{5&-3\\6&-4}-\lambda\myvec{1&0\\0&1}}=0\\
            \mydet{\myvec{5&-3\\6&-4}-\myvec{\lambda&0\\0&\lambda}}=0\\
            \mydet{\myvec{5-\lambda&-3\\6&-4-\lambda}}=0 \\
            \brak{5-\lambda}\brak{-4-\lambda}+3\brak{6}=0
  \end{align}
\end{frame}
\subsection{Finding eigen values}
\begin{frame}
\frametitle{Finding eigen values}
 \begin{align}
     \lambda^2+4\lambda-5\lambda-20+18=0 \\
            \lambda^2-\lambda-2=0 \\
            \brak{\lambda-2}\brak{\lambda+1}=0 \\
            \lambda_1=2\ \brak{\text{and}}\ \lambda_2=-1
 \end{align}
 For a given matrix $\vec{A}$
\begin{align}
     \vec{A}&=\vec{P}\vec{D}\vec{P}^{-1} \\
      \vec{A}^2&=\brak{\vec{P}\vec{D}\vec{P}^{-1}}^2 \\
      &=\vec{P}\vec{D}\vec{P}^{-1}\vec{P}\vec{D}\vec{P}^{-1}\\
      &=\vec{P}\vec{D}\vec{I}\vec{D}\vec{P}^{-1}\\
      &=\vec{P}\vec{D}^2\vec{P}^{-1}
 \end{align}
 where
 \end{frame}
 \subsection{Formula}
\begin{frame}
\frametitle{Formula}
 \begin{align}
\vec{D}=\myvec{\lambda_1&0\\0&\lambda_2}=\myvec{2&0\\0&-1}
 \end{align}
 \begin{align}
     \vec{A}^k&=\vec{P}\vec{D}^k\vec{P}^{-1}\\
\text{trace}\brak{\vec{A}^k}&=\text{trace}\brak{\vec{P}\vec{D}^k\vec{P}^{-1}} \\
&=\text{trace}\brak{\brak{\vec{P}\vec{D}^k}\vec{P}^{-1}}
     \end{align}
     Since trace$\brak{\vec{AB}}$=trace$\brak{\vec{BA}}$
     \begin{align}
\text{trace}\brak{\vec{A}^k}&=\text{trace}\brak{\brak{\vec{P}\vec{D}^k}\vec{P}^{-1}}  \\
&=\text{trace}\brak{\vec{P}^{-1}\brak{\vec{P}\vec{D}^k}}  \\
\text{trace}\brak{\vec{A}^k}&=\text{trace}\brak{\vec{I}\vec{D}^k}=\text{trace}\brak{\vec{D}^k}
     \end{align}
\end{frame}
\subsection{Conclusion}
\begin{frame}
\frametitle{Conclusion}
   \begin{align}
     \text{trace}\brak{\vec{A}^{1000}}&=\text{trace}\brak{\vec{D}^{1000}} \\
&=\text{trace}\myvec{2^{1000}&0\\0&\brak{-1}^{1000}} \\
&=2^{1000}+1
 \end{align}
 \end{frame}
 \section{C code}
  \begin{frame}[fragile]
\frametitle{C code }
\begin{lstlisting}[language=C]
void get_matrix_A(double* data) {
    data[0] = 5.0;
    data[1] = -3.0;
    data[2] = 6.0;
    data[3] = -4.0;
}

\end{lstlisting}
\end{frame}
\section{Python code}
 \begin{frame}[fragile]
\frametitle{Python code for Solving }
\begin{lstlisting}[language=Python]
import ctypes
import numpy as np

def get_trace_of_power_matrix():

    lib = ctypes.CDLL('./code.so')

    double_array_4 = ctypes.c_double * 4
    lib.get_matrix_A.argtypes = [ctypes.POINTER(ctypes.c_double)]

    out_data_c = double_array_4()
    lib.get_matrix_A(out_data_c)

    A = np.array(list(out_data_c)).reshape((2, 2))

    eigenvalues = np.linalg.eigvals(A)
\end{lstlisting}
\end{frame}

 \begin{frame}[fragile]
\frametitle{Python code for Solving }
\begin{lstlisting}[language=Python]
lambda_1, lambda_2 = eigenvalues
    trace_A_1000 = lambda_1**1000 + lambda_2**1000
    return A, (lambda_1, lambda_2), trace_A_1000

if __name__ == '__main__':
    matrix_A, eigs, final_trace = get_trace_of_power_matrix()
    l1, l2 = eigs
    print(f"\ntrace(A^1000) = ({l1.real:.0f})^1000 + ({l2.real:.0f})^1000")


\end{lstlisting}
\end{frame}

\end{document}
