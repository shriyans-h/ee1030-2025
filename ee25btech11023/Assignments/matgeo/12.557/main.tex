\let\negmedspace\undefined
\let\negthickspace\undefined
\documentclass[journal]{IEEEtran}
\usepackage[a5paper, margin=10mm, onecolumn]{geometry}
%\usepackage{lmodern} % Ensure lmodern is loaded for pdflatex
\usepackage{tfrupee} % Include tfrupee package

\setlength{\headheight}{1cm} % Set the height of the header box
\setlength{\headsep}{0mm}     % Set the distance between the header box and the top of the text

\usepackage{gvv-book}
\usepackage{gvv}
\usepackage{cite}
\usepackage{amsmath,amssymb,amsfonts,amsthm}
\usepackage{algorithmic}
\usepackage{graphicx}
\usepackage{textcomp}
\usepackage{xcolor}
\usepackage{txfonts}
\usepackage{listings}
\usepackage{enumitem}
\usepackage{mathtools}
\usepackage{gensymb}
\usepackage{comment}
\usepackage[breaklinks=true]{hyperref}
\usepackage{tkz-euclide} 
\usepackage{listings}
% \usepackage{gvv}                                        
\def\inputGnumericTable{}                                 
\usepackage[latin1]{inputenc}                                
\usepackage{color}                                            
\usepackage{array}                                            
\usepackage{longtable}                                       
\usepackage{calc}                                             
\usepackage{multirow}                                         
\usepackage{hhline}                                           
\usepackage{ifthen}                                           
\usepackage{lscape}
\begin{document}

\bibliographystyle{IEEEtran}

\title{12.557}
\author{EE25BTECH11023 - Venkata Sai}
% \maketitle
% \newpage
% \bigskip
\maketitle 
\renewcommand{\thefigure}{\theenumi}
\renewcommand{\thetable}{\theenumi}
\setlength{\intextsep}{10pt} % Space between text and floats

\numberwithin{align}{enumi}
\numberwithin{figure}{enumi}
\renewcommand{\thetable}{\theenumi}
\vspace{-1em}
\textbf{Question:}  \\
Let $\vec{A}=\myvec{5&-3\\6&-4}$.Then the trace of $\vec{A}^{1000}$ equals\\
\textbf{Solution:}  \\
  Given
  \begin{align}
      \vec{A}=\myvec{5&-3\\6&-4} \\
  \end{align}
  To find eigen values 
  \begin{align}
      |\vec{A}-\lambda\vec{I}|=0
      \end{align}
      \begin{align}
            \mydet{\myvec{5&-3\\6&-4}-\lambda\myvec{1&0\\0&1}}=0\\
            \mydet{\myvec{5&-3\\6&-4}-\myvec{\lambda&0\\0&\lambda}}=0\\
            \mydet{\myvec{5-\lambda&-3\\6&-4-\lambda}}=0 \\
            \brak{5-\lambda}\brak{-4-\lambda}+3\brak{6}=0\\
            \lambda^2+4\lambda-5\lambda-20+18=0 \\
            \lambda^2-\lambda-2=0 \\
            \brak{\lambda-2}\brak{\lambda+1}=0 \\
            \lambda_1=2\ \brak{\text{and}}\ \lambda_2=-1
  \end{align}
For a given matrix $\vec{A}$
\begin{align}
     \vec{A}&=\vec{P}\vec{D}\vec{P}^{-1} \\
      \vec{A}^2&=\brak{\vec{P}\vec{D}\vec{P}^{-1}}^2 \\
      &=\vec{P}\vec{D}\vec{P}^{-1}\vec{P}\vec{D}\vec{P}^{-1}\\
      &=\vec{P}\vec{D}\vec{I}\vec{D}\vec{P}^{-1}\\
      &=\vec{P}\vec{D}^2\vec{P}^{-1}
 \end{align}
 where 
 \begin{align}
\vec{D}=\myvec{\lambda_1&0\\0&\lambda_2}=\myvec{2&0\\0&-1}\\
 \end{align}
 \begin{align}
     \vec{A}^k&=\vec{P}\vec{D}^k\vec{P}^{-1}\\
\text{trace}\brak{\vec{A}^k}&=\text{trace}\brak{\vec{P}\vec{D}^k\vec{P}^{-1}} \\
&=\text{trace}\brak{\brak{\vec{P}\vec{D}^k}\vec{P}^{-1}} 
     \end{align}
     Since trace$\brak{\vec{AB}}$=trace$\brak{\vec{BA}}$
     \begin{align}
\text{trace}\brak{\vec{A}^k}&=\text{trace}\brak{\brak{\vec{P}\vec{D}^k}\vec{P}^{-1}}  \\
&=\text{trace}\brak{\vec{P}^{-1}\brak{\vec{P}\vec{D}^k}}  \\
\text{trace}\brak{\vec{A}^k}&=\text{trace}\brak{\vec{I}\vec{D}^k}=\text{trace}\brak{\vec{D}^k}
     \end{align}
     \begin{align}
     \text{trace}\brak{\vec{A}^{1000}}&=\text{trace}\brak{\vec{D}^{1000}} \\
&=\text{trace}\myvec{2^{1000}&0\\0&\brak{-1}^{1000}} \\
&=2^{1000}+1
 \end{align}
 \end{document}
