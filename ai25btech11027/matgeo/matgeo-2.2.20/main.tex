\let\negmedspace\undefined
\let\negthickspace\undefined
\documentclass[journal,12pt,onecolumn]{IEEEtran}
\usepackage{cite}
\usepackage{amsmath,amssymb,amsfonts,amsthm}
\usepackage{algorithmic}
\usepackage{graphicx}
\graphicspath{{./figs/}}
\usepackage{textcomp}
\usepackage{xcolor}
\usepackage{txfonts}
\usepackage{listings}
\usepackage{enumitem}
\usepackage{mathtools}
\usepackage{gensymb}
\usepackage{comment}
\usepackage{caption}
\usepackage[breaklinks=true]{hyperref}
\usepackage{tkz-euclide} 
\usepackage{listings}
\usepackage{gvv}                                        
%\def\inputGnumericTable{}                                 
\usepackage[latin1]{inputenc}     
\usepackage{xparse}
\usepackage{color}                                            
\usepackage{array}                                            
\usepackage{longtable}                                       
\usepackage{calc}                                             
\usepackage{multirow}
\usepackage{multicol}
\usepackage{hhline}                                           
\usepackage{ifthen}                                           
\usepackage{lscape}
\usepackage{tabularx}
\usepackage{array}
\usepackage{float}
%\newtheorem{theorem}{Theorem}[section]
%\newtheorem{theorem}{Theorem}[section]
%\newtheorem{problem}{Problem}
%\newtheorem{proposition}{Proposition}[section]
%\newtheorem{lemma}{Lemma}[section]
%\newtheorem{corollary}[theorem]{Corollary}
%\newtheorem{example}{Example}[section]
%\newtheorem{definition}[problem]{Definition}

\begin{document}

%\textbf{\Large 2.2.20} \\
%\textbf{\large AI25BTECH11027 - NAGA BHUVANA} \\
\title{Matgeo-2.2.20}
\author{AI25BTECH11027 - NAGA BHUVANA}
% \maketitle
% \newpage
% \bigskip
%\begin{document}
{\let\newpage\relax\maketitle}
%\renewcommand{\thefigure}{\theenumi}
%\renewcommand{\thetable}{\theenumi}
\noindent
		\textbf{Question}:\\
        If the co-ordinates of the points $\vec{A},\vec{B},\vec{C},\vec{D}$ be (1,2,3),(4,5,7),(-4,3,-6) and (2,9,2) respectively, then find the angle between lines AB and CD.\\
\textbf{Solution:}\\
Let 
\begin{align}
    \vec{A}=\myvec{1\\2\\3},\vec{B}=\myvec{4\\5\\7},\vec{C}=\myvec{-4\\3\\-6} \, \text{and} \:  \vec{D}=\myvec{2\\9\\2}
\end{align}
\begin{align}
    \vec{B-A}=\myvec{4\\5\\7}-\myvec{1\\2\\3}=\myvec{3\\3\\4}
\end{align}	
\begin{align}
    \vec{D-C}=\myvec{2\\9\\2}-\myvec{-4\\3\\-6}=\myvec{6\\6\\8}
\end{align}
Let the angle between $\vec{B-A}$ and $\vec{D-C}$ be $\theta$
\begin{align}
    \cos{\theta}=\frac{\myvec{B-A}^T \myvec{D-C}}{\|\vec{B-A}\| \|\vec{D-C}\|}
\end{align}
\begin{align}
    \cos{\theta}=\frac{\myvec{3 & 3 & 4}\myvec{6\\6\\8}}{\sqrt{34}\sqrt{136}}
\end{align}
\begin{align}
    \cos{\theta}=\frac{(3)(6)+(3)(6)+(4)(8)}{68}
\end{align}
\begin{align}
    \cos{\theta}=\frac{68}{68}
\end{align}
\begin{align}
    \cos{\theta}=1\\
    \theta=0^\circ
\end{align}
$\therefore$ The angle between lines $\vec{(B-A)}$ and $\vec{(D-C)}$ is $0^\circ$ (Collinear lines)
\begin{figure}[H]
	\centering
	\includegraphics[width=0.7\linewidth]{figs/fig1.png}
	\caption{}
	\label{fig}
\end{figure}
\end{document}



