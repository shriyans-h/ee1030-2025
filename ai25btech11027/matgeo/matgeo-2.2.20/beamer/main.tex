\documentclass{beamer}
\mode<presentation>
\usepackage{amsmath}
\usepackage{amssymb}
%\usepackage{advdate}
\usepackage{graphicx}
\graphicspath{{./figs/}}
\usepackage{adjustbox}
\usepackage{subcaption}
\usepackage{enumitem}
\usepackage{multicol}
\usepackage{mathtools}
\usepackage{listings}
\usepackage{url}
\def\UrlBreaks{\do\/\do-}
\usetheme{Boadilla}
\usecolortheme{lily}
\setbeamertemplate{footline}
{
  \leavevmode%
  \hbox{%
  \begin{beamercolorbox}[wd=\paperwidth,ht=2.25ex,dp=1ex,right]{author in head/foot}%
    \insertframenumber{} / \inserttotalframenumber\hspace*{2ex} 
  \end{beamercolorbox}}%
  \vskip0pt%
}
\setbeamertemplate{navigation symbols}{}

\providecommand{\nCr}[2]{\,^{#1}C_{#2}} % nCr
\providecommand{\nPr}[2]{\,^{#1}P_{#2}} % nPr
\providecommand{\mbf}{\mathbf}
\providecommand{\pr}[1]{\ensuremath{\Pr\left(#1\right)}}
\providecommand{\qfunc}[1]{\ensuremath{Q\left(#1\right)}}
\providecommand{\sbrak}[1]{\ensuremath{{}\left[#1\right]}}
\providecommand{\lsbrak}[1]{\ensuremath{{}\left[#1\right.}}
\providecommand{\rsbrak}[1]{\ensuremath{{}\left.#1\right]}}
\providecommand{\brak}[1]{\ensuremath{\left(#1\right)}}
\providecommand{\lbrak}[1]{\ensuremath{\left(#1\right.}}
\providecommand{\rbrak}[1]{\ensuremath{\left.#1\right)}}
\providecommand{\cbrak}[1]{\ensuremath{\left\{#1\right\}}}
\providecommand{\lcbrak}[1]{\ensuremath{\left\{#1\right.}}
\providecommand{\rcbrak}[1]{\ensuremath{\left.#1\right\}}}
\theoremstyle{remark}
\newtheorem{rem}{Remark}
\newcommand{\sgn}{\mathop{\mathrm{sgn}}}
\providecommand{\abs}[1]{\left\vert#1\right\vert}
\providecommand{\res}[1]{\Res\displaylimits_{#1}} 
\providecommand{\norm}[1]{\lVert#1\rVert}
\providecommand{\mtx}[1]{\mathbf{#1}}
\providecommand{\mean}[1]{E\left[ #1 \right]}
\providecommand{\fourier}{\overset{\mathcal{F}}{ \rightleftharpoons}}
%\providecommand{\hilbert}{\overset{\mathcal{H}}{ \rightleftharpoons}}
\providecommand{\system}[1]{\overset{\mathcal{#1}}{ \longleftrightarrow}}
%\providecommand{\system}{\overset{\mathcal{H}}{ \longleftrightarrow}}
	%\newcommand{\solution}[2]{\textbf{Solution:}{#1}}
%\newcommand{\solution}{\noindent \textbf{Solution: }}
\providecommand{\dec}[2]{\ensuremath{\overset{#1}{\underset{#2}{\gtrless}}}}
\newcommand{\myvec}[1]{\ensuremath{\begin{pmatrix}#1\end{pmatrix}}}
\let\vec\mathbf

\lstset{
%language=C,
frame=single, 
breaklines=true,
columns=fullflexible
}

\numberwithin{equation}{section}
\title{2.2.20}
\author{AI25BTECH11027 - NAGA BHUVANA}
% \maketitle
% \newpage
% \bigskip
\begin{document}
{\let\newpage\relax\maketitle}
\renewcommand{\thefigure}{\theenumi}
\renewcommand{\thetable}{\theenumi}
   \noindent
		\textbf{Question}:\\
        If the co-ordinates of the points $\vec{A},\vec{B},\vec{C},\vec{D}$ be (1,2,3),(4,5,7),(-4,3,-6) and (2,9,2) respectively, then find the angle between lines AB and CD.\\
\textbf{Solution:}\\
Let 
\begin{align}
    \vec{A}=\myvec{1\\2\\3},\vec{B}=\myvec{4\\5\\7},\vec{C}=\myvec{-4\\3\\-6} \, \text{and} \:  \vec{D}=\myvec{2\\9\\2}
\end{align}
\begin{align}
    \vec{B-A}=\myvec{4\\5\\7}-\myvec{1\\2\\3}=\myvec{3\\3\\4}
\end{align}	
\begin{align}
    \vec{D-C}=\myvec{2\\9\\2}-\myvec{-4\\3\\-6}=\myvec{6\\6\\8}
\end{align}
Let the angle between $\vec{B-A}$ and $\vec{D-C}$ be $\theta$
\begin{align}
    \cos{\theta}=\frac{\myvec{B-A}^T \myvec{D-C}}{\|\vec{B-A}\| \|\vec{D-C}\|}
\end{align}
\begin{align}
    \cos{\theta}=\frac{\myvec{3 & 3 & 4}\myvec{6\\6\\8}}{\sqrt{34}\sqrt{136}}
\end{align}
\begin{align}
    \cos{\theta}=\frac{(3)(6)+(3)(6)+(4)(8)}{68}
\end{align}
\begin{align}
    \cos{\theta}=\frac{68}{68}
\end{align}
\begin{align}
    \cos{\theta}=1\\
    \theta=0^\circ
\end{align}
$\therefore$ The angle between lines $\vec{(B-A)}$ and $\vec{(D-C)}$ is $0^\circ$ (Collinear lines)
%Graphical Representation
 \frametitle{Graphical Representation}
 \centering
 \includegraphics[width=0.6\linewidth]{figs/fig1.png}
\end{document}
