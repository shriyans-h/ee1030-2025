\let\negmedspace\undefined
\let\negthickspace\undefined
\documentclass[journal,12pt,onecolumn]{IEEEtran}
\usepackage{cite}
\usepackage{amsmath,amssymb,amsfonts,amsthm}
\usepackage{algorithmic}
\usepackage{graphicx}
\graphicspath{{./figs/}}
\usepackage{textcomp}
\usepackage{xcolor}
\usepackage{txfonts}
\usepackage{listings}
\usepackage{enumitem}
\usepackage{mathtools}
\usepackage{gensymb}
\usepackage{comment}
\usepackage{caption}
\usepackage[breaklinks=true]{hyperref}
\usepackage{tkz-euclide} 
\usepackage{listings}
\usepackage{gvv}                                        
%\def\inputGnumericTable{}                                 
\usepackage[latin1]{inputenc}     
\usepackage{xparse}
\usepackage{color}                                            
\usepackage{array}                                            
\usepackage{longtable}                                       
\usepackage{calc}                                             
\usepackage{multirow}
\usepackage{multicol}
\usepackage{hhline}                                           
\usepackage{ifthen}                                           
\usepackage{lscape}
\usepackage{tabularx}
\usepackage{array}
\usepackage{float}
%\newtheorem{theorem}{Theorem}[section]
%\newtheorem{theorem}{Theorem}[section]
%\newtheorem{problem}{Problem}
%\newtheorem{proposition}{Proposition}[section]
%\newtheorem{lemma}{Lemma}[section]
%\newtheorem{corollary}[theorem]{Corollary}
%\newtheorem{example}{Example}[section]
%\newtheorem{definition}[problem]{Definition}

\begin{document}

%\textbf{\Large 1.7.13} \\
%\textbf{\large AI25BTECH11027 - NAGA BHUVANA} \\
\title{1.7.13}
\author{AI25BTECH11027 - NAGA BHUVANA}
% \maketitle
% \newpage
% \bigskip
%\begin{document}
{\let\newpage\relax\maketitle}
%\renewcommand{\thefigure}{\theenumi}
%\renewcommand{\thetable}{\theenumi}

\textbf{Question}:\\
\noindent Find the value of p for which the points $\brak{-5,1}$ , \brak{1,p} and $\brak{4,-2}$ are collinear.\\
\textbf{solution:}\\
Let the points be
\begin{align}
\vec{A}=\myvec{-5\\1} , \vec{B}=\myvec{1\\p} and \vec{C}=\myvec{4\\2}\\
\end{align}
Given that the three points are collinear ,
That is Rank of the Augmented matrix of $\vec{B-A}$ and $\vec{C-A}$ must be 1.
\begin{align}
 \implies   rank\myvec{\vec{B-A} & \vec{C-A}}^T=1
\end{align}
\begin{align}
    \vec{B-A}=\myvec{1-(-5)\\p-1}=\myvec{6\\p-1}
\end{align}
\begin{align}
    \vec{C-A}=\myvec{4-(-5)\\-2-1}=\myvec{9\\-3}
\end{align}
Now Consider the augmented matrix $\vec{M}$\\
\begin{align}
    \vec{M}=\myvec{6 & 9 \\ p-1 & -3}^T=\myvec{6 & p-1 \\ 9 & -3}
\end{align}
By doing Row operations $R_2 \longrightarrow R_2/3$ and  $R_2 \longrightarrow 2R_2-R_1$\\
\begin{align}
    \vec{M}=\myvec{6 & p-1\\0 & -p-1}
\end{align}
As the rank($\vec{M}$)=1\\
\begin{align}
    \implies -p-1=0\\
\end{align}
\begin{align}
  \implies  \boxed{p=-1}
\end{align}
\begin{center}
$\therefore$ The value of p is $-1$
\end{center}
\begin{figure}[H]
 \centering
	\includegraphics[width=0.7\linewidth]{figs/fig1.png}
	\caption{}
	\label{fig}
\end{figure}
\end{document}
