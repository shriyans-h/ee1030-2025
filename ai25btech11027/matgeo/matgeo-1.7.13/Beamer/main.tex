\documentclass{beamer}
\mode<presentation>
\usepackage{amsmath}
\usepackage{amssymb}
%\usepackage{advdate}
\usepackage{graphicx}
\graphicspath{{./figs/}}
\usepackage{adjustbox}
\usepackage{subcaption}
\usepackage{enumitem}
\usepackage{multicol}
\usepackage{mathtools}
\usepackage{listings}
\usepackage{url}
\def\UrlBreaks{\do\/\do-}
\usetheme{Boadilla}
\usecolortheme{lily}
\setbeamertemplate{footline}
{
  \leavevmode%
  \hbox{%
  \begin{beamercolorbox}[wd=\paperwidth,ht=2.25ex,dp=1ex,right]{author in head/foot}%
    \insertframenumber{} / \inserttotalframenumber\hspace*{2ex} 
  \end{beamercolorbox}}%
  \vskip0pt%
}
\setbeamertemplate{navigation symbols}{}

\providecommand{\nCr}[2]{\,^{#1}C_{#2}} % nCr
\providecommand{\nPr}[2]{\,^{#1}P_{#2}} % nPr
\providecommand{\mbf}{\mathbf}
\providecommand{\pr}[1]{\ensuremath{\Pr\left(#1\right)}}
\providecommand{\qfunc}[1]{\ensuremath{Q\left(#1\right)}}
\providecommand{\sbrak}[1]{\ensuremath{{}\left[#1\right]}}
\providecommand{\lsbrak}[1]{\ensuremath{{}\left[#1\right.}}
\providecommand{\rsbrak}[1]{\ensuremath{{}\left.#1\right]}}
\providecommand{\brak}[1]{\ensuremath{\left(#1\right)}}
\providecommand{\lbrak}[1]{\ensuremath{\left(#1\right.}}
\providecommand{\rbrak}[1]{\ensuremath{\left.#1\right)}}
\providecommand{\cbrak}[1]{\ensuremath{\left\{#1\right\}}}
\providecommand{\lcbrak}[1]{\ensuremath{\left\{#1\right.}}
\providecommand{\rcbrak}[1]{\ensuremath{\left.#1\right\}}}
\theoremstyle{remark}
\newtheorem{rem}{Remark}
\newcommand{\sgn}{\mathop{\mathrm{sgn}}}
\providecommand{\abs}[1]{\left\vert#1\right\vert}
\providecommand{\res}[1]{\Res\displaylimits_{#1}} 
\providecommand{\norm}[1]{\lVert#1\rVert}
\providecommand{\mtx}[1]{\mathbf{#1}}
\providecommand{\mean}[1]{E\left[ #1 \right]}
\providecommand{\fourier}{\overset{\mathcal{F}}{ \rightleftharpoons}}
%\providecommand{\hilbert}{\overset{\mathcal{H}}{ \rightleftharpoons}}
\providecommand{\system}[1]{\overset{\mathcal{#1}}{ \longleftrightarrow}}
%\providecommand{\system}{\overset{\mathcal{H}}{ \longleftrightarrow}}
	%\newcommand{\solution}[2]{\textbf{Solution:}{#1}}
%\newcommand{\solution}{\noindent \textbf{Solution: }}
\providecommand{\dec}[2]{\ensuremath{\overset{#1}{\underset{#2}{\gtrless}}}}
\newcommand{\myvec}[1]{\ensuremath{\begin{pmatrix}#1\end{pmatrix}}}
\let\vec\mathbf

\lstset{
%language=C,
frame=single, 
breaklines=true,
columns=fullflexible
}

\numberwithin{equation}{section}
\title{1.7.13}
\author{AI25BTECH11027 - NAGA BHUVANA}
% \maketitle
% \newpage
% \bigskip
\begin{document}
{\let\newpage\relax\maketitle}
\renewcommand{\thefigure}{\theenumi}
\renewcommand{\thetable}{\theenumi}
\textbf{Question}:\\
\noindent Find the value of p for which the points $\brak{-5,1}$ , \brak{1,p} and $\brak{4,-2}$ are collinear.\\
\textbf{solution:}\\
Let the points be
\begin{align}
\vec{A}=\myvec{-5\\1} , \vec{B}=\myvec{1\\p} and \vec{C}=\myvec{4\\2}\\
\end{align}
Given that the three points are collinear ,
That is Rank of the Augmented matrix of $\vec{B-A}$ and $\vec{C-A}$ must be 1.
\begin{align}
 \implies   rank\myvec{\vec{B-A} & \vec{C-A}}^T=1
\end{align}
\begin{align}
    \vec{B-A}=\myvec{1-(-5)\\p-1}=\myvec{6\\p-1}
\end{align}
\begin{align}
    \vec{C-A}=\myvec{4-(-5)\\-2-1}=\myvec{9\\-3}
\end{align}
Now Consider the augmented matrix $\vec{M}$\\
\begin{align}
    \vec{M}=\myvec{6 & 9 \\ p-1 & -3}^T=\myvec{6 & p-1 \\ 9 & -3}
\end{align}
By doing Row operations $R_2 \longrightarrow R_2/3$ and  $R_2 \longrightarrow 2R_2-R_1$\\
\begin{align}
    \vec{M}=\myvec{6 & p-1\\0 & -p-1}
\end{align}
As the rank($\vec{M}$)=1\\
\begin{align}
    \implies -p-1=0\\
\end{align}
\begin{align}
  \implies  \boxed{p=-1}
\end{align}
\begin{center}
$\therefore$ The value of p is $-1$
\end{center}
% Graphical Representation
	\frametitle{Graphical Representation}
	\centering
	\includegraphics[width=0.6\linewidth]{figs/fig1.png}
\end{document}

