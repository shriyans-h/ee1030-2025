\let\negmedspace\undefined
\let\negthickspace\undefined
\documentclass[journal]{IEEEtran}
\usepackage[a5paper, margin=10mm, onecolumn]{geometry}
%\usepackage{lmodern} % Ensure lmodern is loaded for pdflatex
\usepackage{tfrupee} % Include tfrupee package

\setlength{\headheight}{1cm} % Set the height of the header box
\setlength{\headsep}{0mm}     % Set the distance between the header box and the top of the text

\usepackage{gvv-book}
\usepackage{gvv}
\usepackage{cite}
\usepackage{amsmath,amssymb,amsfonts,amsthm}
\usepackage{algorithmic}
\usepackage{graphicx}
\usepackage{textcomp}
\usepackage{xcolor}
\usepackage{txfonts}
\usepackage{listings}
\usepackage{enumitem}
\usepackage{mathtools}
\usepackage{gensymb}
\usepackage[breaklinks=true]{hyperref}
\usepackage{tkz-euclide} 
\usepackage{listings}
% \usepackage{gvv}                                        
\def\inputGnumericTable{}                                 
\usepackage[latin1]{inputenc}                                
\usepackage{color}                                            
\usepackage{array}                                            
\usepackage{longtable}                                       
\usepackage{calc}                                             
\usepackage{multirow}                                         
\usepackage{hhline}                                           
\usepackage{ifthen}                                           
\usepackage{lscape}
\usepackage{circuitikz}
\usepackage{comment}
\tikzstyle{block} = [rectangle, draw, fill=blue!20, 
text width=4em, text centered, rounded corners, minimum height=3em]
\tikzstyle{sum} = [draw, fill=blue!10, circle, minimum size=1cm, node distance=1.5cm]
\tikzstyle{input} = [coordinate]
\tikzstyle{output} = [coordinate]


\begin{document}
	
	\bibliographystyle{IEEEtran}
	\vspace{3cm}
	
	\title{5.4.36}
	\author{EE25BTECH11042-Nipun Dasari}
	\maketitle
	% \newpage
	% \bigskip
	{\let\newpage\relax\maketitle}
	
	\renewcommand{\thefigure}{\theenumi}
	\renewcommand{\thetable}{\theenumi}
	\setlength{\intextsep}{10pt} % Space between text and floats
	
	
	\numberwithin{equation}{enumi}
	\numberwithin{figure}{enumi}
	\renewcommand{\thetable}{\theenumi}
	
	\textbf{Question}:\\
	Using elementary transformations, find the inverse of the following matrix. 
	\begin{align*}
		\myvec{2&-1&-2\\0&2&-1\\3&-5&0}
	\end{align*}
	\solution \\
	Let us solve the given question theoretically and then verify the solution computationally.\\
	\\
	To solve for the inverse of a matrix, we can employ the Gauss-Jordan approach.
	\begin{align}
		\augvec{3}{3}{2&-1&-2& 1& 0&0\\ 0&2&-1& 0& 1&0\\3&-5&0& 0& 0&1}
		&\xleftrightarrow{\,R_1 \gets \frac{1}{2}R_1}
		\augvec{3}{3}{1 & -1/2 & -1 & 1/2 & 0 & 0 \\ 0 & 2 & -1 & 0 & 1 & 0 \\ 3 & -5 & 0 & 0 & 0 & 1} \\[1.5em]
		&\xleftrightarrow{\,R_3 \gets R_3 - 3R_1}
		\augvec{3}{3}{1 & -1/2 & -1 & 1/2 & 0 & 0 \\ 0 & 2 & -1 & 0 & 1 & 0 \\ 0 & -7/2 & 3 & -3/2 & 0 & 1} \\[1.5em]
		&\xleftrightarrow{\,R_2 \gets \frac{1}{2}R_2}
		\augvec{3}{3}{1 & -1/2 & -1 & 1/2 & 0 & 0 \\ 0 & 1 & -1/2 & 0 & 1/2 & 0 \\ 0 & -7/2 & 3 & -3/2 & 0 & 1} \\[1.5em]
		&\xleftrightarrow[\,R_3 \gets R_3 + \frac{7}{2}R_2]{\,R_1 \gets R_1 + \frac{1}{2}R_2}
		\augvec{3}{3}{1 & 0 & -5/4 & 1/2 & 1/4 & 0 \\ 0 & 1 & -1/2 & 0 & 1/2 & 0 \\ 0 & 0 & 5/4 & -3/2 & 7/4 & 1} \\[1.5em]
		&\xleftrightarrow{\,R_3 \gets \frac{4}{5}R_3}
		\augvec{3}{3}{1 & 0 & -5/4 & 1/2 & 1/4 & 0 \\ 0 & 1 & -1/2 & 0 & 1/2 & 0 \\ 0 & 0 & 1 & -6/5 & 7/5 & 4/5} \\[1.5em]
		&\xleftrightarrow[\,R_2 \gets R_2 + \frac{1}{2}R_3]{\,R_1 \gets R_1 + \frac{5}{4}R_3}
		\augvec{3}{3}{1 & 0 & 0 & -1 & 2 & 1 \\ 0 & 1 & 0 & -3/5 & 6/5 & 2/5 \\ 0 & 0 & 1 & -6/5 & 7/5 & 4/5}
	\end{align}
	
	\begin{align}
		\therefore \text{Inverse of the given Matrix:}\myvec{-1 & 2 & 1 \\-3/5 & 6/5 & 2/5\\-6/5 & 7/5 & 4/5}
	\end{align}
	
\end{document}