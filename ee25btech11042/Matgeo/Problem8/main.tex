\let\negmedspace\undefined
\let\negthickspace\undefined
\documentclass[journal]{IEEEtran}
\usepackage[a5paper, margin=10mm, onecolumn]{geometry}
%\usepackage{lmodern} % Ensure lmodern is loaded for pdflatex
\usepackage{tfrupee} % Include tfrupee package

\setlength{\headheight}{1cm} % Set the height of the header box
\setlength{\headsep}{0mm}     % Set the distance between the header box and the top of the text

\usepackage{gvv-book}
\usepackage{gvv}
\usepackage{cite}
\usepackage{amsmath,amssymb,amsfonts,amsthm}
\usepackage{algorithmic}
\usepackage{graphicx}
\usepackage{textcomp}
\usepackage{xcolor}
\usepackage{txfonts}
\usepackage{listings}
\usepackage{enumitem}
\usepackage{mathtools}
\usepackage{gensymb}
\usepackage[breaklinks=true]{hyperref}
\usepackage{tkz-euclide} 
\usepackage{listings}
% \usepackage{gvv}                                        
\def\inputGnumericTable{}                                 
\usepackage[latin1]{inputenc}                                
\usepackage{color}                                            
\usepackage{array}                                            
\usepackage{longtable}                                       
\usepackage{calc}                                             
\usepackage{multirow}                                         
\usepackage{hhline}                                           
\usepackage{ifthen}                                           
\usepackage{lscape}
\usepackage{circuitikz}
\usepackage{comment}
\tikzstyle{block} = [rectangle, draw, fill=blue!20, 
text width=4em, text centered, rounded corners, minimum height=3em]
\tikzstyle{sum} = [draw, fill=blue!10, circle, minimum size=1cm, node distance=1.5cm]
\tikzstyle{input} = [coordinate]
\tikzstyle{output} = [coordinate]
%--------------------------

\begin{document}
	
	\bibliographystyle{IEEEtran}
	\vspace{3cm}
	
	\title{5.2.44}
	\author{EE25BTECH11042 - Nipun Dasari}
	\maketitle
	
	\renewcommand{\thefigure}{\theenumi}
	\renewcommand{\thetable}{\theenumi}
	\setlength{\intextsep}{10pt}
	
	\numberwithin{equation}{enumi}
	\numberwithin{figure}{enumi}
	\renewcommand{\thetable}{\theenumi}
	
	\textbf{Question}:\\
	Solve the following system of rational equations
	\begin{align}
		\frac{10}{x+y}+\frac{2}{x-y} &= 4 \\
		\frac{15}{x+y}-\frac{5}{x-y} &= -2
	\end{align}
	
	\solution \\
	Introduce $a$ and $b$ as follows:
	\begin{align}
		a = \frac{1}{x+y}, \quad b = \frac{1}{x-y}
	\end{align}
	Also define the vector $\vec{a} = \begin{myvec}{a\\b}\end{myvec}$. This gives us the simplified linear system:
	\begin{align}
		\begin{myvec}{10 & 2} \end{myvec}\vec{a} &= 4 \\
		\begin{myvec}{15 & -5} \end{myvec}\vec{a} &= -2
	\end{align}
	
	The augmented matrix for this system is:
	\begin{align}
		\augvec{2}{1}{10&2&4\\15&-5&-2}
		&\xleftrightarrow{\,R_2 \gets R_2- \frac{3}{2} R_1}
		\augvec{2}{1}{10&2&4\\0&-8&-8}  \nonumber \\
		&\xleftrightarrow{\,R_1 \gets R_1+\frac{1}{4} R_2}
		\augvec{2}{1}{10&0&2\\0&-8&-8} \nonumber \\
		&\xleftrightarrow[\,R_2 \gets -\frac{1}{8} R_2]{\,R_1 \gets \frac{1}{10} R_1}
		\augvec{2}{1}{1&0&\frac{1}{5}\\0&1&1}
	\end{align}
	From the reduced matrix, we have the solution:
	\begin{align}
		\vec{a} = \begin{myvec}{\frac{1}{5}\\1} \end{myvec}
	\end{align}
	Substituting back for $x$ and $y$ gives:
	\begin{align}
		\begin{myvec}{\frac{1}{x+y} \\ \frac{1}{x-y}} \end{myvec} = \begin{myvec}{\frac{1}{5}\\1} \end{myvec}
		\implies \begin{myvec}{x+y \\ x-y} \end{myvec} = \begin{myvec}{5\\1} \end{myvec}
	\end{align}
	This is another linear system, $\begin{myvec}{1&1 \\ 1&-1} \end{myvec}\vec{x} = \begin{myvec}{5\\1} \end{myvec}$, where $\vec{x} = \begin{myvec}{x\\y}\end{myvec}$.
	We can solve for $\vec{x}$ by multiplying by the inverse of the matrix:
	\begin{align}
		\vec{x} &= \begin{myvec}{1&1 \\ 1&-1} \end{myvec}^{-1} \begin{myvec}{5\\1} \end{myvec} \\
		&= \begin{myvec}{\frac{1}{2}&\frac{1}{2} \\ \frac{1}{2}&-\frac{1}{2}} \end{myvec} \begin{myvec}{5\\1} \end{myvec} \\
		&= \begin{myvec}{\frac{1}{2}(5) + \frac{1}{2}(1) \\ \frac{1}{2}(5) - \frac{1}{2}(1)} \end{myvec} \\
		&= \begin{myvec}{\frac{6}{2} \\ \frac{4}{2}} \end{myvec} = \begin{myvec}{3 \\ 2} \end{myvec}
	\end{align}
	Thus, the solution is $\mathbf{x=3}$ and $\mathbf{y=2}$.
	
	
	\begin{figure}[H]
		\centering
		\includegraphics[width = 0.8\columnwidth]{figs/Figure_1.png}
		\caption*{}
		\label{fig1}
	\end{figure}
	\begin{figure}[H]
		\centering
		\includegraphics[width = 0.8\columnwidth]{figs/Figure_2.png}
		\caption*{}
		\label{fig2}
	\end{figure}
	
\end{document}