\let\negmedspace\undefined
\let\negthickspace\undefined
\documentclass[journal]{IEEEtran}
\usepackage[a5paper, margin=10mm, onecolumn]{geometry}
%\usepackage{lmodern} % Ensure lmodern is loaded for pdflatex
\usepackage{tfrupee} % Include tfrupee package

\setlength{\headheight}{1cm} % Set the height of the header box
\setlength{\headsep}{0mm}     % Set the distance between the header box and the top of the text

\usepackage{gvv-book}
\usepackage{gvv}
\usepackage{cite}
\usepackage{amsmath,amssymb,amsfonts,amsthm}
\usepackage{algorithmic}
\usepackage{graphicx}
\usepackage{textcomp}
\usepackage{xcolor}
\usepackage{txfonts}
\usepackage{listings}
\usepackage{enumitem}
\usepackage{mathtools}
\usepackage{gensymb}
\usepackage[breaklinks=true]{hyperref}
\usepackage{tkz-euclide} 
\usepackage{listings}
% \usepackage{gvv}                                        
\def\inputGnumericTable{}                                 
\usepackage[latin1]{inputenc}                                
\usepackage{color}                                            
\usepackage{array}                                            
\usepackage{longtable}                                       
\usepackage{calc}                                             
\usepackage{multirow}                                         
\usepackage{hhline}                                           
\usepackage{ifthen}                                           
\usepackage{lscape}
\usepackage{circuitikz}
\usepackage{comment}
\tikzstyle{block} = [rectangle, draw, fill=blue!20, 
text width=4em, text centered, rounded corners, minimum height=3em]
\tikzstyle{sum} = [draw, fill=blue!10, circle, minimum size=1cm, node distance=1.5cm]
\tikzstyle{input} = [coordinate]
\tikzstyle{output} = [coordinate]
%--------------------------

\begin{document}
	
	\bibliographystyle{IEEEtran}
	\vspace{3cm}
	
	\title{5.2.44}
	\author{EE25BTECH11042 - Nipun Dasari}
	\maketitle
	
	\renewcommand{\thefigure}{\theenumi}
	\renewcommand{\thetable}{\theenumi}
	\setlength{\intextsep}{10pt}
	
	\numberwithin{equation}{enumi}
	\numberwithin{figure}{enumi}
	\renewcommand{\thetable}{\theenumi}
	
	\textbf{Question}:\\
	Solve the following system of rational equations
	\begin{align}
		\frac{10}{x+y}+\frac{2}{x-y} &= 4 \\
		\frac{15}{x+y}-\frac{5}{x-y} &= -2
	\end{align}
	
	\solution \\
	 $\frac{a}{x+y} + \frac{a}{x-y} = C$ becomes:
	\begin{align}
		c(x^2 - y^2) - (a+b)x + (a-b)y = 0 \label{eq:hyperbola_form}
	\end{align}
	Matrix form: $\vec{x}^\top V \vec{x} + 2\vec{u}^\top \vec{x} = 0$, where:
	\begin{align}
		\vec{x} = \begin{myvec}{x \\ y}\end{myvec}, \quad
		V = \begin{myvec}{c & 0 \\ 0 & -c}\end{myvec}, \quad
		\vec{u} = \begin{myvec}{-(a+b)/2 \\ (a-b)/2}\end{myvec} \label{eq:matrix_defs}
	\end{align}
	The intersection points of the two hyperbolas lie on a Common chord, $c_1H_2-c_2H_1=0$, where $H_1=0$ and $H_2=0$ are the equations of each of hyperbolas. This results in the linear equation $\vec{n}^\top\vec{x}=0$,
	\begin{align}
		d = c_1(a_2+b_2) - c_2(a_1+b_1) \label{eq:D_coeff} \\
		e = c_2(a_1-b_1) - c_1(a_2-b_2) \label{eq:E_coeff} \\
		\text{where d and e are obtained by eliminating the quadratic terms  } \vec{n} = \begin{myvec}{d \\ e}\end{myvec}
	\end{align}
	
	The solution is the non-trivial intersection point of this common chord and either hyperbola
	\begin{align}
		% --- Start with y in terms of x
		y &= -\frac{d}{e}x \label{eq:y_formula} \\
		% --- Substitute y into the hyperbola equation
		c_1\brak{x^2 - \brak{-\frac{d}{e}x}^2} - \brak{a_1+b_1}x + \brak{a_1-b_1}\brak{-\frac{d}{e}x} &= 0 \\
		% --- Factor out the common 'x' term. This is a critical step for correctness.
		\implies x \brak{ x\brak{c_1\brak{\frac{e^2-d^2}{e^2}}} - \brak{\frac{e\brak{a_1+b_1} + d\brak{a_1-b_1}}{e}} } &= 0 \\
		% --- To find the non-trivial solution, set the inner bracket to zero and rearrange
		\therefore x\brak{c_1\brak{\frac{e^2-d^2}{e^2}}} &= \brak{\frac{e\brak{a_1+b_1}+d\brak{a_1-b_1}}{e}} \\
		% --- Solve for x to get the final formula
		\implies x &= \frac{e\brak{e\brak{a_1+b_1}+d\brak{a_1-b_1}}}{c_1\brak{e^2-d^2}} \label{eq:x_formula}
	\end{align}
	For the given system, the coefficients are:
	\begin{align}
		a_1 = 10, b_1 = 2, c_1 = 4 \quad \text{and} \quad a_2 = 15, b_2 = -5, c_2 = -2
	\end{align}
	Using \eqref{eq:D_coeff} and \eqref{eq:E_coeff}, we calculate the common chord coefficients:
	\begin{align}
		d = 4\brak{15-5} - \brak{-2}\brak{10+2} = 4\brak{10} + 2\brak{12} = 64 \\
		e = \brak{-2}\brak{10-2} - 4\brak{15--5} = -2(8) - 4(20) = -96
	\end{align}
	Substituting these into the formula for $x$ from \eqref{eq:x_formula}:
	\begin{align}
		x = \frac{\brak{-96} \brak{\brak{-96}\brak{12} + \brak{64}\brak{8}}}{4\brak{\brak{-96}^2-\brak{64}^2}} \\
		= \frac{-96  \brak{-1152 + 512}}{4\brak{9216 - 4096}} \\
		= \frac{-96\brak{-640}}{4\brak{5120}} = \frac{61440}{20480} = 3 \label{eq:x_val}
	\end{align}
	Using the value of $x$ from \eqref{eq:x_val} in the formula for $y$ \eqref{eq:y_formula}:
	\begin{align}
		y = -\frac{64}{-96}(3) = \frac{2}{3}3 = 2 \label{eq:y_val}
	\end{align}
	Thus, from \eqref{eq:x_val} and \eqref{eq:y_val}, the solution is $x=3$ and $y=2$.
	
	\begin{figure}[H]
		\centering
		\includegraphics[width = 0.6\columnwidth]{figs/Figure_1.png}
		\caption*{}
		\label{fig1}
	\end{figure}
	\begin{figure}[H]
		\centering
		\includegraphics[width = 0.6\columnwidth]{figs/Figure_2.png}
		\caption*{}
		\label{fig2}
	\end{figure}
	
\end{document}