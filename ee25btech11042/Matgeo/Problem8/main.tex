\let\negmedspace\undefined
\let\negthickspace\undefined
\documentclass[journal]{IEEEtran}
\usepackage[a5paper, margin=10mm, onecolumn]{geometry}
\usepackage{lmodern} % Ensure lmodern is loaded for pdflatex
\usepackage{tfrupee} % Include tfrupee package

\setlength{\headheight}{1cm} % Set the height of the header box
\setlength{\headsep}{0mm}     % Set the distance between the header box and the top of the text

\usepackage{gvv-book}
\usepackage{gvv}
\usepackage{cite}
\usepackage{amsmath,amssymb,amsfonts,amsthm}
\usepackage{algorithmic}
\usepackage{graphicx}
\graphicspath{{./figs/}}
\usepackage{textcomp}
\usepackage{xcolor}
\usepackage{txfonts}
\usepackage{listings}
\usepackage{enumitem}
\usepackage{mathtools}
\usepackage{gensymb}
\usepackage{comment}
\usepackage[breaklinks=true]{hyperref}
\usepackage{tkz-euclide} 
\usepackage{listings}
\usepackage{gvv}                                        
\def\inputGnumericTable{}                    
\usepackage[latin1]{inputenc}                                
\usepackage{color}                                            
\usepackage{array}                                            
\usepackage{longtable}                                       
\usepackage{calc}                            
\usepackage{multirow}                                         
\usepackage{hhline}                                          
\usepackage{ifthen}                                           
\usepackage{lscape}
\usepackage{circuitikz}

\begin{document}
	
	\bibliographystyle{IEEEtran}
	\vspace{3cm}
	
	\title{5.2.44}
	\author{EE25BTECH11042 - Nipun Dasari}
	\maketitle
	
	\renewcommand{\thefigure}{\theenumi}
	\renewcommand{\thetable}{\theenumi}
	\setlength{\intextsep}{10pt} % Space between text and floats
	
	
	\numberwithin{equation}{enumi}
	\numberwithin{figure}{enumi}
	\renewcommand{\thetable}{\theenumi}
	
	\textbf{Question}:\\
	Solve the following system of rational equations
	\begin{align}
		\frac{10}{x+y}+\frac{2}{x-y} = 4
	\end{align}
	\begin{align}
		\frac{15}{x+y}-\frac{5}{x-y} = -2
	\end{align}
	
	
	\solution \\
	Introduce $a$ and $b$ as follows:
	\begin{align}
		a = \frac{1}{x+y} \text{  } b = \frac{1}{x-y}
	\end{align}
	Also define
	\begin{align}
		\vec{a} = \begin{myvec}{a\\b} \end{myvec}
	\end{align}
	This gives us simplified equations
	\begin{align}
		\begin{myvec}{10 & 2} \end{myvec}\vec{a} = 4
	\end{align}
	
	\begin{align}
		\begin{myvec}{15 & -5} \end{myvec}\vec{a} = -2
	\end{align}
	

	Augmented matrix for the given system is
	\begin{align}
			\augvec{2}{1}{10&2&4\\15&-5&-2}
	\end{align}
	By row reductions
	\begin{align}
		
				\augvec{2}{1}{10&2&4\\15&-5&-2}
			\xleftrightarrow{\,R_2 \gets R_2- \frac{3}{2} \times R_1}
			\augvec{2}{1}{10&2&4\\0&-8&-8}  
			\xleftrightarrow{\,R_1 \gets R_1+\frac{1}{4} \times R_2}
			\augvec{2}{1}{10&0&2\\0&-8&-8}\\
			\augvec{2}{1}{10&0&2\\0&-8&-8}
			\xleftrightarrow{\,R_1 \gets \frac{1}{10} \times R_1}
			\augvec{2}{1}{1&0&\frac{1}{5}\\0&-8&-8}
			\xleftrightarrow{\,R_2 \gets \frac{1}{-8} \times R_2}
			\augvec{2}{1}{1&0&\frac{1}{5}\\0&1&1}
	\end{align}
	\begin{align}
		\vec{a} = \begin{myvec}{\frac{1}{5}\\1} \end{myvec}
	\end{align}
	Substituting value of a and b again we get
	\begin{align}
		\begin{myvec}{\frac{1}{x+y} \\ \frac{1}{x-y}} \end{myvec} = \begin{myvec}{\frac{1}{5}\\1} \end{myvec}
	\end{align}
	
	\begin{align}
		\implies \begin{myvec}{x+y \\ x-y} \end{myvec} = \begin{myvec}{5\\1} \end{myvec}
	\end{align}
	Introduce	
	\begin{align}
		\vec{x} = \begin{myvec}{x\\y} \end{myvec}
	\end{align}
	This gives us the equation
	\begin{align}
		\begin{myvec}{1&1 \\ 1&-1} \end{myvec}\vec{x} = \begin{myvec}{5\\1} \end{myvec}
	\end{align}
	
	\begin{align}
		\implies \vec{x} = 	\begin{myvec}{5\\1} \end{myvec}\begin{myvec}{1&1 \\ 1&-1} \end{myvec}^{-1}
	\end{align}
	\begin{align}
		\implies \vec{x} = 	\begin{myvec}{5\\1} \end{myvec}\begin{myvec}{\frac{1}{2}&\frac{1}{2} \\ \frac{1}{2}&-\frac{1}{2}} \end{myvec}
	\end{align}
	\begin{align}
		\implies \vec{x} = 	\begin{myvec}{\frac{5}{2}+\frac{1}{2} \\ \frac{5}{2}-\frac{1}{2}} \end{myvec}
	\end{align}
	\begin{align}
		\implies \vec{x} = 	\begin{myvec}{3 \\ 2} \end{myvec}
	\end{align}
	Thus $x=3$ and $y=2$
	
	\begin{figure}[H]
		\centering
		\includegraphics[width = 0.8\columnwidth]{figs/Figure_1.png}
		\caption*{}
		\label{fig1}
	\end{figure}
	\begin{figure}[H]
		\centering
		\includegraphics[width = 0.8\columnwidth]{figs/Figure_2.png}
		\caption*{}
		\label{fig2}
	\end{figure}
	
	
\end{document}