\documentclass{beamer}
\usepackage[utf8]{inputenc}

\usetheme{Madrid}
\usecolortheme{default}
\usepackage{amsmath,amssymb,amsfonts,amsthm}
\usepackage{txfonts}
\usepackage{tkz-euclide}
\usepackage{listings}
\usepackage{adjustbox}
\usepackage{array}
\usepackage{tabularx}
\usepackage{gvv}
\usepackage{lmodern}
\usepackage{circuitikz}
\usepackage{tikz}
\usepackage{graphicx}

\setbeamertemplate{page number in head/foot}[totalframenumber]

\usepackage{tcolorbox}
\tcbuselibrary{minted,breakable,xparse,skins}



\definecolor{bg}{gray}{0.95}
\DeclareTCBListing{mintedbox}{O{}m!O{}}{%
	breakable=true,
	listing engine=minted,
	listing only,
	minted language=#2,
	minted style=default,
	minted options={%
		linenos,
		gobble=0,
		breaklines=true,
		breakafter=,,
		fontsize=\small,
		numbersep=8pt,
		#1},
	boxsep=0pt,
	left skip=0pt,
	right skip=0pt,
	left=25pt,
	right=0pt,
	top=3pt,
	bottom=3pt,
	arc=5pt,
	leftrule=0pt,
	rightrule=0pt,
	bottomrule=2pt,
	toprule=2pt,
	colback=bg,
	colframe=orange!70,
	enhanced,
	overlay={%
		\begin{tcbclipinterior}
			\fill[orange!20!white] (frame.south west) rectangle ([xshift=20pt]frame.north west);
	\end{tcbclipinterior}},
	#3,
}
\lstset{
	language=C,
	basicstyle=\ttfamily\small,
	keywordstyle=\color{blue},
	stringstyle=\color{orange},
	commentstyle=\color{green!60!black},
	numbers=left,
	numberstyle=\tiny\color{gray},
	breaklines=true,
	showstringspaces=false,
}
%------------------------------------------------------------
%This block of code defines the information to appear in the
%Title page
\title %optional
{1.9.30}
\date{January 9, 2025}
%\subtitle{A short story}

\author % (optional)
{Nipun Dasari - EE25BTECH11042}



\begin{document}
	
	\frame{\titlepage}
	\begin{frame}{Question}
	Solve the following system of rational equations
	\begin{align}
		\frac{10}{x+y}+\frac{2}{x-y} = 4
	\end{align}
	\begin{align}
		\frac{15}{x+y}-\frac{5}{x-y} = -2
	\end{align}
	
	\end{frame}
	
	
	\begin{frame}{Theoretical Solution}
		Introduce $a$ and $b$ as follows:
	\begin{align}
		a = \frac{1}{x+y} \text{  } b = \frac{1}{x-y}
	\end{align}
	Also define
	\begin{align}
		\vec{a} = \begin{myvec}{a\\b} \end{myvec}
	\end{align}
	This gives us simplified equations
	\begin{align}
		\begin{myvec}{10 & 2} \end{myvec}\vec{a} = 4
	\end{align}
	
	\begin{align}
		\begin{myvec}{15 & -5} \end{myvec}\vec{a} = -2
	\end{align}
	
	
	Augmented matrix for the given system is
	\begin{align}
		\augvec{2}{1}{10&2&4\\15&-5&-2}
	\end{align}
	\end{frame}
	\begin{frame}{Theoretical Solution}
	 $\frac{A}{x+y} + \frac{B}{x-y} = C$ becomes:
	\begin{align}
		c(x^2 - y^2) - (a+b)x + (a-b)y = 0 \label{eq:hyperbola_form}
	\end{align}
	Matrix form: $\vec{x}^\top V \vec{x} + 2\vec{u}^\top \vec{x} = 0$, where:
	\begin{align}
		\vec{x} = \begin{myvec}{x \\ y}\end{myvec}, \quad
		V = \begin{myvec}{c & 0 \\ 0 & -c}\end{myvec}, \quad
		\vec{u} = \begin{myvec}{-(a+b)/2 \\ (a-b)/2}\end{myvec} \label{eq:matrix_defs}
	\end{align}
	The intersection points of the two hyperbolas lie on a Common chord, $c_1H_2-c_2H_1=0$, where $H_1=0$ and $H_2=0$ are the equations of each of hyperbolas. This results in the linear equation $\vec{n}^\top\vec{x}=0$,
	\begin{align}
		d = c_1(a_2+b_2) - c_2(a_1+b_1) \label{eq:D_coeff} \\
		e = c_2(a_1-b_1) - c_1(a_2-b_2) \label{eq:E_coeff} \\
		\text{where d and e are obtained by eliminating the quadratic terms  } \vec{n} = \begin{myvec}{d \\ e}\end{myvec}
	\end{align}
	\end{frame}
		\begin{frame}{Theoretical Solution}
		
	The solution is the non-trivial intersection point of this common chord and either hyperbola
	\begin{align}
		% --- Start with y in terms of x
		y = -\frac{d}{e}x \label{eq:y_formula} \\
		% --- Substitute y into the hyperbola equation
		c_1\brak{x^2 - \brak{-\frac{d}{e}x}^2} - \brak{a_1+b_1}x + \brak{a_1-b_1}\brak{-\frac{d}{e}x} = 0 \\
		% --- Factor out the common 'x' term. This is a critical step for correctness.
		\implies x \brak{ x\brak{c_1\brak{\frac{e^2-d^2}{e^2}}} - \brak{\frac{e\brak{a_1+b_1} + d\brak{a_1-b_1}}{e}} } = 0 \\
		% --- To find the non-trivial solution, set the inner bracket to zero and rearrange
		\therefore x\brak{c_1\brak{\frac{e^2-d^2}{e^2}}} = \brak{\frac{e\brak{a_1+b_1}+d\brak{a_1-b_1}}{e}} \\
		% --- Solve for x to get the final formula
		x = \frac{e\brak{e\brak{a_1+b_1}+d\brak{a_1-b_1}}}{c_1\brak{e^2-d^2}} \label{eq:x_formula}
	\end{align}
	\end{frame}
		\begin{frame}{Theoretical Solution}
			For the given system, the coefficients are:
		\begin{align}
			a_1 = 10, b_1 = 2, c_1 = 4 \quad \text{and} \quad a_2 = 15, b_2 = -5, c_2 = -2
		\end{align}
		Using \eqref{eq:D_coeff} and \eqref{eq:E_coeff}, we calculate the common chord coefficients:
		\begin{align}
			d = 4\brak{15-5} - \brak{-2}\brak{10+2} = 4\brak{10} + 2\brak{12} = 64 \\
			e = \brak{-2}\brak{10-2} - 4\brak{15--5} = -2(8) - 4(20) = -96
		\end{align}
		Substituting these into the formula for $x$ from \eqref{eq:x_formula}:
		
	\end{frame}	
	\begin{frame}{Theoretical Solution}
	\begin{align}
		x = \frac{\brak{-96} \brak{\brak{-96}\brak{12} + \brak{64}\brak{8}}}{4\brak{\brak{-96}^2-\brak{64}^2}} \\
		= \frac{-96  \brak{-1152 + 512}}{4\brak{9216 - 4096}} \\
		= \frac{-96\brak{-640}}{4\brak{5120}} = \frac{61440}{20480} = 3 \label{eq:x_val}
	\end{align}
	Using the value of $x$ from \eqref{eq:x_val} in the formula for $y$ \eqref{eq:y_formula}:
	\begin{align}
		y = -\frac{64}{-96}(3) = \frac{2}{3}3 = 2 \label{eq:y_val}
	\end{align}
	Thus, from \eqref{eq:x_val} and \eqref{eq:y_val}, the solution is $x=3$ and $y=2$.
	
	\end{frame}
	
	\begin{frame}[fragile]
		\frametitle{C Code}
		
		\begin{lstlisting}
	void get_conic_data(double* data_out) {							// Hyperbola 1: V1=[[4,0],[0,-4]], u1=[-6,4]			data_out[0] = 4.0;  data_out[1] = 0.0;
		data_out[2] = 0.0;  data_out[3] = -4.0;
		data_out[4] = -6.0; data_out[5] = 4.0;
		// Hyperbola 2: V2=[[-2,0],[0,2]], u2=[-5,10]
		data_out[6] = -2.0; data_out[7] = 0.0;
		data_out[8] = 0.0;  data_out[9] = 2.0;
		data_out[10] = -5.0; data_out[11] = 10.0;
		// Solution Point: [3, 2]
	data_out[12] = 3.0; data_out[13] = 2.0;
		}
		\end{lstlisting}
	\end{frame}
	\begin{frame}[fragile]
		\frametitle{Python Code using shared output}
		
		\begin{lstlisting}
# plot_from_so.py
import ctypes
import numpy as np
import matplotlib.pyplot as plt
# Load the shared library
lib = ctypes.CDLL('./5.2.44.so')
# Define the function signature
get_data_func = lib.get_conic_data
get_data_func.argtypes = [np.ctypeslib.ndpointer(dtype=np.double, ndim=1, flags='C_CONTIGUOUS')]
get_data_func.restype = None		
		\end{lstlisting}
	\end{frame}

	
	\begin{frame}[fragile]
		\frametitle{Python Code using shared output}
		\begin{lstlisting}
# Create a buffer and call the C function
output_array = np.zeros(14, dtype=np.double)			get_data_func(output_array)
# Unpack the data from C
V1 = output_array[0:4].reshape((2, 2))
u1 = output_array[4:6]
V2 = output_array[6:10].reshape((2, 2))
u2 = output_array[10:12]
solution_point = output_array[12:14]
# --- Plotting Code ---
x_vals = np.linspace(-10, 15, 500)
y_vals = np.linspace(-10, 15, 500)
X, Y = np.meshgrid(x_vals, y_vals)
eq1 = V1[0,0]*X**2 + V1[1,1]*Y**2 + 2*(u1[0]*X + u1[1]*Y)
eq2 = V2[0,0]*X**2 + V2[1,1]*Y**2 + 2*(u2[0]*X + u2[1]*Y)		
		\end{lstlisting}
	\end{frame}
	\begin{frame}[fragile]
		\frametitle{Python Code using shared output}
		\begin{lstlisting}		
plt.figure(figsize=(10, 10))
plt.contour(X, Y, eq1, levels=[0], colors='red')
plt.contour(X, Y, eq2, levels=[0], colors='blue')
plt.plot(x_vals, (2/3)*x_vals, 'g--', label='Common Chord')
plt.plot(solution_point[0], solution_point[1], 'ko', markersize=10, label=f'Solution from C: ({solution_point[0]}, {solution_point[1]})')
plt.title('Plot from C Shared Library Data', fontsize=16)
plt.xlabel('x-axis'); plt.ylabel('y-axis')
plt.grid(True, linestyle='--'); plt.axhline(0, color='k', lw=0.5); plt.axvline(0, color='k', lw=0.5)
plt.gca().set_aspect('equal', adjustable='box'); plt.xlim(-5, 10); plt.ylim(-5, 10)
plt.legend()
plt.savefig('so_python_plot.png')
print("Plot saved to so_python_plot.png")
plt.show()			
		\end{lstlisting}
	\end{frame}
	\begin{frame}[fragile]
		\frametitle{Python Code}
		\begin{lstlisting}
# Code to plot the solution of the system of rational equations
import numpy as np
import matplotlib.pyplot as plt
# --- Define the parameters for the two hyperbolas ---
# Hyperbola 1: 4(x^2 - y^2) - 12x + 8y = 0
V1 = np.array([[4, 0], [0, -4]])
u1 = np.array([-6, 4])
f1 = 0
# Hyperbola 2: -2(x^2 - y^2) - 10x + 20y = 0
V2 = np.array([[-2, 0], [0, 2]])
u2 = np.array([-5, 10])
f2 = 0		
		\end{lstlisting}
	\end{frame}
	\begin{frame}[fragile]
		\frametitle{Python Code}
	\begin{lstlisting}
# --- Set up the plotting grid ---
# Generate a grid of points to evaluate the equations on
x_vals = np.linspace(-10, 15, 500)
y_vals = np.linspace(-10, 15, 500)
X, Y = np.meshgrid(x_vals, y_vals)
# --- Define the hyperbola equations ---
# Equation is x^T V x + 2u^T x + f = 0
# For a point (x,y), the vector is [x, y]
# So x^T V x becomes V[0,0]*x^2 + V[1,1]*y^2
# and 2u^T x becomes 2*(u[0]*x + u[1]*y)
eq1 = V1[0,0]*X**2 + V1[1,1]*Y**2 + 2*(u1[0]*X + u1[1]*Y) + f1
eq2 = V2[0,0]*X**2 + V2[1,1]*Y**2 + 2*(u2[0]*X + u2[1]*Y) + f2
	\end{lstlisting}
	
\end{frame}
\begin{frame}[fragile]
	\frametitle{Python Code}
	\begin{lstlisting}
# --- Create the Plot ---
plt.figure(figsize=(10, 10))
# Plot the hyperbolas by finding where their equations equal zero
plt.contour(X, Y, eq1, levels=[0], colors='red', linewidths=2)
plt.contour(X, Y, eq2, levels=[0], colors='blue', linewidths=2)
# --- Plot the Common Chord and Solution ---
# The common chord is 64x - 96y = 0, which simplifies to 2x - 3y = 0
# So, y = (2/3)x
plt.plot(x_vals, (2/3)*x_vals, 'g--', label='Common Chord: $2x - 3y = 0$')
# The solution point
solution_point = np.array([3, 2])
plt.plot(solution_point[0], solution_point[1], 'ko', markersize=10, label='Solution (3, 2)')
	\end{lstlisting}
	
\end{frame}
\begin{frame}[fragile]
	\frametitle{Python Code}
	\begin{lstlisting}
plt.title('Intersection of Hyperbolas', fontsize=16)
plt.xlabel('x-axis', fontsize=12)
plt.ylabel('y-axis', fontsize=12)
plt.grid(True, which='both', linestyle='--', linewidth=0.5)
plt.axhline(0, color='black', linewidth=0.5)
plt.axvline(0, color='black', linewidth=0.5)
plt.gca().set_aspect('equal', adjustable='box')
plt.xlim(-5, 10)
plt.ylim(-5, 10)
	\end{lstlisting}
	
\end{frame}
\begin{frame}[fragile]
	\frametitle{Python Code}
	\begin{lstlisting}
# Create a legend
plt.legend(handles=[
plt.Line2D([0], [0], color='red', lw=2, label='Hyperbola 1'),
plt.Line2D([0], [0], color='blue', lw=2, label='Hyperbola 2'),
plt.Line2D([0], [0], color='g', linestyle='--', label='Common Chord'),
plt.Line2D([0], [0], marker='o', color='k', linestyle='', markersize=8, label='Solution (3, 2)')
])
# Save and show the plot
plt.savefig('hyperbola_intersection.png')
plt.show()
	\end{lstlisting}
	
\end{frame}

	
	\begin{frame}{Plot by python using shared output from c}
		\begin{figure}[H]
			\centering
			\includegraphics[width = 0.8\columnwidth]{figs/Figure_1.png}
			\caption*{}
			\label{fig1}
		\end{figure}
	\end{frame}
	\begin{frame}{Plot by python}
		\begin{figure}[H]
			\centering
			\includegraphics[width = 0.8\columnwidth]{figs/Figure_2.png}
			\caption*{}
			\label{fig2}
		\end{figure}
	\end{frame}
\end{document}