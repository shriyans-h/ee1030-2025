\documentclass{beamer}
\usepackage[utf8]{inputenc}

\usetheme{Madrid}
\usecolortheme{default}
\usepackage{amsmath,amssymb,amsfonts,amsthm}
\usepackage{txfonts}
\usepackage{tkz-euclide}
\usepackage{listings}
\usepackage{adjustbox}
\usepackage{array}
\usepackage{tabularx}
\usepackage{gvv}
\usepackage{lmodern}
\usepackage{circuitikz}
\usepackage{tikz}
\usepackage{graphicx}

\setbeamertemplate{page number in head/foot}[totalframenumber]

\usepackage{tcolorbox}
\tcbuselibrary{minted,breakable,xparse,skins}



\definecolor{bg}{gray}{0.95}
\DeclareTCBListing{mintedbox}{O{}m!O{}}{%
	breakable=true,
	listing engine=minted,
	listing only,
	minted language=#2,
	minted style=default,
	minted options={%
		linenos,
		gobble=0,
		breaklines=true,
		breakafter=,,
		fontsize=\small,
		numbersep=8pt,
		#1},
	boxsep=0pt,
	left skip=0pt,
	right skip=0pt,
	left=25pt,
	right=0pt,
	top=3pt,
	bottom=3pt,
	arc=5pt,
	leftrule=0pt,
	rightrule=0pt,
	bottomrule=2pt,
	toprule=2pt,
	colback=bg,
	colframe=orange!70,
	enhanced,
	overlay={%
		\begin{tcbclipinterior}
			\fill[orange!20!white] (frame.south west) rectangle ([xshift=20pt]frame.north west);
	\end{tcbclipinterior}},
	#3,
}
\lstset{
	language=C,
	basicstyle=\ttfamily\small,
	keywordstyle=\color{blue},
	stringstyle=\color{orange},
	commentstyle=\color{green!60!black},
	numbers=left,
	numberstyle=\tiny\color{gray},
	breaklines=true,
	showstringspaces=false,
}
%------------------------------------------------------------
%This block of code defines the information to appear in the
%Title page
\title %optional
{1.9.30}
\date{January 9, 2025}
%\subtitle{A short story}

\author % (optional)
{Nipun Dasari - EE25BTECH11042}



\begin{document}
	
	\frame{\titlepage}
	\begin{frame}{Question}
	Solve the following system of rational equations
	\begin{align}
		\frac{10}{x+y}+\frac{2}{x-y} = 4
	\end{align}
	\begin{align}
		\frac{15}{x+y}-\frac{5}{x-y} = -2
	\end{align}
	
	\end{frame}
	
	
	\begin{frame}{Theoretical Solution}
		Introduce $a$ and $b$ as follows:
	\begin{align}
		a = \frac{1}{x+y} \text{  } b = \frac{1}{x-y}
	\end{align}
	Also define
	\begin{align}
		\vec{a} = \begin{myvec}{a\\b} \end{myvec}
	\end{align}
	This gives us simplified equations
	\begin{align}
		\begin{myvec}{10 & 2} \end{myvec}\vec{a} = 4
	\end{align}
	
	\begin{align}
		\begin{myvec}{15 & -5} \end{myvec}\vec{a} = -2
	\end{align}
	
	
	Augmented matrix for the given system is
	\begin{align}
		\augvec{2}{1}{10&2&4\\15&-5&-2}
	\end{align}
	\end{frame}
	\begin{frame}{Theoretical Solution}
		By row reductions
		\begin{align}
			
			\augvec{2}{1}{10&2&4\\15&-5&-2}
			\xleftrightarrow{\,R_2 \gets R_2- \frac{3}{2} \times R_1}
			\augvec{2}{1}{10&2&4\\0&-8&-8}  
			\xleftrightarrow{\,R_1 \gets R_1+\frac{1}{4} \times R_2}
			\augvec{2}{1}{10&0&2\\0&-8&-8}\\
			\augvec{2}{1}{10&0&2\\0&-8&-8}
			\xleftrightarrow{\,R_1 \gets \frac{1}{10} \times R_1}
			\augvec{2}{1}{1&0&\frac{1}{5}\\0&-8&-8}
			\xleftrightarrow{\,R_2 \gets \frac{1}{-8} \times R_2}
			\augvec{2}{1}{1&0&\frac{1}{5}\\0&1&1}
		\end{align}
		\begin{align}
			\vec{a} = \begin{myvec}{\frac{1}{5}\\1} \end{myvec}
		\end{align}
		Substituting value of a and b again we get
		\begin{align}
			\begin{myvec}{\frac{1}{x+y} \\ \frac{1}{x-y}} \end{myvec} = \begin{myvec}{\frac{1}{5}\\1} \end{myvec}
		\end{align}
	\end{frame}
		\begin{frame}{Theoretical Solution}
		\begin{align}
			\implies \begin{myvec}{x+y \\ x-y} \end{myvec} = \begin{myvec}{5\\1} \end{myvec}
		\end{align}
		Introduce	
		\begin{align}
			\vec{x} = \begin{myvec}{x\\y} \end{myvec}
		\end{align}
		This gives us the equation
		\begin{align}
			\begin{myvec}{1&1 \\ 1&-1} \end{myvec}\vec{x} = \begin{myvec}{5\\1} \end{myvec}
		\end{align}
		
		\begin{align}
			\implies \vec{x} = 	\begin{myvec}{5\\1} \end{myvec}\begin{myvec}{1&1 \\ 1&-1} \end{myvec}^{-1}
		\end{align}
	\end{frame}
		\begin{frame}{Theoretical Solution}
		\begin{align}
			\implies \vec{x} = 	\begin{myvec}{5\\1} \end{myvec}\begin{myvec}{\frac{1}{2}&\frac{1}{2} \\ \frac{1}{2}&-\frac{1}{2}} \end{myvec}
		\end{align}
		\begin{align}
			\implies \vec{x} = 	\begin{myvec}{\frac{5}{2}+\frac{1}{2} \\ \frac{5}{2}-\frac{1}{2}} \end{myvec}
		\end{align}
		\begin{align}
			\implies \vec{x} = 	\begin{myvec}{3 \\ 2} \end{myvec}
		\end{align}
		Thus $x=3$ and $y=2$
		
	\end{frame}
	\begin{frame}[fragile]
		\frametitle{C Code- equidistant check function }
		
		\begin{lstlisting}
			#include <stdio.h>
			
			void rref_solver(double aug[2][3], double solution[2]) {
				// Normalize first row (pivot = aug[0][0])
				double pivot = aug[0][0];
				for (int j = 0; j < 3; j++) {
					aug[0][j] /= pivot;
				}
				
				// Eliminate below pivot
				double factor = aug[1][0];
				for (int j = 0; j < 3; j++) {
					aug[1][j] -= factor * aug[0][j];
				}
		\end{lstlisting}
	\end{frame}
	\begin{frame}[fragile]
		\frametitle{C Code- equidistant check function }
		
		\begin{lstlisting}
			 pivot = aug[1][1];
			for (int j = 0; j < 3; j++) {
				aug[1][j] /= pivot;
			}
			
			// Eliminate above pivot
			factor = aug[0][1];
			for (int j = 0; j < 3; j++) {
				aug[0][j] -= factor * aug[1][j];
			}
			
			// Extract solution
			solution[0] = aug[0][2]; // x
			solution[1] = aug[1][2]; // y
			
		\end{lstlisting}
	\end{frame}

	
	\begin{frame}[fragile]
		\frametitle{Python Code using shared output}
		\begin{lstlisting}
			import ctypes
			import numpy as np
			import matplotlib.pyplot as plt
			import matplotlib as mp
			# Load the shared C library
			lib = ctypes.CDLL("./5.2.44.so")
			# Define argument and return types
			lib.rref_solver.argtypes = [ctypes.c_double * 6, ctypes.c_double * 2]
			# Create augmented matrix for system:
			aug = (ctypes.c_double * 6)(1, 1, 5,   1, -1, 1)  # Flattened 2x3
			solution = (ctypes.c_double * 2)()
		\end{lstlisting}
	\end{frame}
	\begin{frame}[fragile]
		\frametitle{Python Code using shared output}
		\begin{lstlisting}		
			# Call C function
			lib.rref_solver(aug, solution)
			# Convert result to numpy vector (ensure flat)
			x_sol = np.array([solution[0], solution[1]], dtype=float).flatten()
			print("Solution vector from C:", x_sol) # This correctly prints [3. 2.]
			# plot
			x_vals = np.linspace(-2, 10, 400)
			y1 = 5 - x_vals         # Correct for x + y = 5
			y2 = x_vals - 1         # CORRECTED for x - y = 1
			plt.plot(x_vals, y1, label=r"$x+y=5$")
			plt.plot(x_vals, y2, label=r"$x-y=1$")
			plt.scatter(x_sol[0], x_sol[1], color="red", zorder=5)
			plt.text(float(x_sol[0])+0.2, float(x_sol[1]), f"({x_sol[0]:.1f}, {x_sol[1]:.1f})", color="red")
		\end{lstlisting}
	\end{frame}
	\begin{frame}[fragile]
		\frametitle{Python Code using shared output}
		\begin{lstlisting}
		plt.plot(x_vals, y1, label=r"$x+y=5$")
		plt.plot(x_vals, y2, label=r"$x-y=1$")
		plt.scatter(x_sol[0], x_sol[1], color="red", zorder=5)
		plt.text(float(x_sol[0])+0.2, float(x_sol[1]), f"({x_sol[0]:.1f}, {x_sol[1]:.1f})", color="red")
		plt.xlabel("x")
		plt.ylabel("y")
		plt.title("Graphical Solution of the Linear System")
		plt.axhline(0, color="black", linewidth=0.8)
		plt.axvline(0, color="black", linewidth=0.8)
		plt.legend()
		plt.grid(True)
		plt.savefig("Figure_1_Corrected.png")
		plt.show()
		\end{lstlisting}
	\end{frame}
	\begin{frame}[fragile]
		\frametitle{Python Code}
	\begin{lstlisting}
	import numpy as np
	import matplotlib.pyplot as plt
	import matplotlib as mp
	mp.use("TkAgg")
	A=np.array([[1,1],[1,-1]],dtype=float)
	b=np.array([5,1], dtype=float)
	x=np.linalg.solve(A,b)
	print("Solution vector for the system of equations:",x)
	\end{lstlisting}
	
\end{frame}
\begin{frame}[fragile]
	\frametitle{Python Code}
	\begin{lstlisting}
		# Making a plot
		x_vals = np.linspace(-2, 10, 400)
		# Rearranged equations to express y in terms of x
		y1 = (5 - x_vals)        # from x + 3y = 6
		y2 = (x_vals-1)    # from 2x - 3y = 12
		# Plot lines
		plt.plot(x_vals, y1, label=r"$x + y = 5$")
		plt.plot(x_vals, y2, label=r"$x - y = 1$")
		# Mark solution
		plt.scatter(x[0], x[1], color="red", zorder=5)
		plt.text(x[0]+0.2, x[1], f"({x[0]:.1f}, {x[1]:.1f})", color="red")
	\end{lstlisting}
	
\end{frame}
\begin{frame}[fragile]
	\frametitle{Python Code}
	\begin{lstlisting}
		# Formatting
		plt.xlabel("x")
		plt.ylabel("y")
		plt.title("Graphical Solution of the Linear System")
		plt.axhline(0, color='black', linewidth=0.8)
		plt.axvline(0, color='black', linewidth=0.8)
		plt.legend()
		plt.grid(True)
		plt.savefig("Figure_2")
		plt.show()
	\end{lstlisting}
	
\end{frame}

	
	\begin{frame}{Plot by python using shared output from c}
		\begin{figure}[H]
			\centering
			\includegraphics[width = 0.8\columnwidth]{figs/Figure_1.png}
			\caption*{}
			\label{fig1}
		\end{figure}
	\end{frame}
	\begin{frame}{Plot by python}
		\begin{figure}[H]
			\centering
			\includegraphics[width = 0.8\columnwidth]{figs/Figure_2.png}
			\caption*{}
			\label{fig2}
		\end{figure}
	\end{frame}
\end{document}