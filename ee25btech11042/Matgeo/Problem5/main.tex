\let\negmedspace\undefined
\let\negthickspace\undefined
\documentclass[journal]{IEEEtran}
\usepackage[a5paper, margin=10mm, onecolumn]{geometry}
\usepackage{lmodern} % Ensure lmodern is loaded for pdflatex
\usepackage{tfrupee} % Include tfrupee package

\setlength{\headheight}{1cm} % Set the height of the header box
\setlength{\headsep}{0mm}     % Set the distance between the header box and the top of the text

% --- Assuming gvv-book and gvv packages define \solution, \brak, etc. ---
\usepackage{gvv-book}
\usepackage{gvv}
\usepackage{cite}
\usepackage{amsmath,amssymb,amsfonts,amsthm}
\usepackage{algorithmic}
\usepackage{graphicx}
\graphicspath{{./figs/}}
\usepackage{textcomp}
\usepackage{xcolor}
\usepackage{txfonts}
\usepackage{listings}
\usepackage{enumitem}
\usepackage{mathtools}
\usepackage{gensymb}
\usepackage{comment}
\usepackage[breaklinks=true]{hyperref}
\usepackage{tkz-euclide} 
\usepackage{listings}
\usepackage{gvv}                                        
\def\inputGnumericTable{}                    
\usepackage[latin1]{inputenc}                                
\usepackage{color}                                            
\usepackage{array}                                            
\usepackage{longtable}                                       
\usepackage{calc}                            
\usepackage{multirow}                                         
\usepackage{hhline}                                          
\usepackage{ifthen}                                           
\usepackage{lscape}
\usepackage{circuitikz}

\begin{document}
	
	\bibliographystyle{IEEEtran}
	\vspace{3cm}
	
	\title{4.3.32}
	\author{EE25BTECH11042 - Nipun Dasari}
	\maketitle
	
	\renewcommand{\thefigure}{\theenumi}
	\renewcommand{\thetable}{\theenumi}
	\setlength{\intextsep}{10pt} % Space between text and floats
	
	
	\numberwithin{equation}{enumi}
	\numberwithin{figure}{enumi}
	\renewcommand{\thetable}{\theenumi}
	
	\textbf{Question}:\\
	Find the slope of a line which cuts off intercepts of equal length on the axes is. Solve using matrices. \\ 
	\solution \\
	
	Let the line cut the x-axis at an intercept 'a' and the y-axis at an intercept 'b'. The points where the line intersects the axes can be represented by position vectors (column matrices).
	
	The point of x-intercept is $P_1 = (a, 0)$. Its position vector is:
	\begin{align}
		\vec{p_1} = \begin{myvec}{a \\ 0} \end{myvec}
	\end{align}
	The point of y-intercept is $P_2 = (0, b)$. Its position vector is:
	\begin{align}
		\vec{p_2} = \begin{myvec}{0 \\ b} \end{myvec}
	\end{align}
	
	A direction vector for the line can be found by taking the difference between the two position vectors:
	\begin{align}
		\vec{v} = \vec{p_2} - \vec{p_1} = \begin{myvec}{0 \\ b} \end{myvec} - \begin{myvec}{a \\ 0} \end{myvec} = \begin{myvec}{-a \\ b} \end{myvec} \label{eq:dir_vec}
	\end{align}
	The direction vector $\vec{v}$ can be written as $\vec{v} = \begin{myvec}{\Delta x \\\Delta y}\end{myvec}$. The slope, $m$, is defined as the ratio of the change in y to the change in x.
	\begin{align}
		m = \frac{\Delta y}{\Delta x} = \frac{b}{-a} \label{0.4}
	\end{align}
	
	The problem states that the intercepts have equal length, which means their magnitudes are equal:
	\begin{align}
		|a| = |b|
	\end{align}
	
	\textbf{Case 1: The intercepts are equal ($b = a$)}\\
	Substituting $b=a$ into the slope equation (assuming $a \neq 0$):
	By \eqref{0.4}
	\begin{align}
		m_1 = \frac{a}{-a} = -1
	\end{align}
	
	\textbf{Case 2: The intercepts are opposite ($b = -a$)}\\
	Substituting $b=-a$ into the slope equation (assuming $a \neq 0$):
	By \eqref{0.4}
	\begin{align}
		m_2 = \frac{-a}{-a} = 1
	\end{align}
	
	Thus, using a matrix representation for the points, we find that the two possible slopes are -1 and 1.
	
	\begin{figure}[H]
		\centering
		\includegraphics[width = 0.8\columnwidth]{Figure_1.png}
		\caption*{}
		\label{fig1}
	\end{figure}
	
\end{document}