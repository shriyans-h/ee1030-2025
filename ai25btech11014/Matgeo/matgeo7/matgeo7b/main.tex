\documentclass{beamer}
\usepackage[utf8]{inputenc}

\usetheme{Madrid}
\usecolortheme{default}
\usepackage{amsmath,amssymb,amsfonts,amsthm}
\usepackage{txfonts}
\usepackage{tkz-euclide}
\usepackage{listings}
\usepackage[T1]{fontenc}
\usepackage{adjustbox}
\usepackage{array}
\usepackage{tabularx}
\usepackage{gvv}
\usepackage{lmodern}
\usepackage{circuitikz}
\usepackage{tikz}
\usepackage{graphicx}

\setbeamertemplate{page number in head/foot}[totalframenumber]

\usepackage{tcolorbox}
\tcbuselibrary{minted,breakable,xparse,skins}

\definecolor{bg}{gray}{0.95}
\DeclareTCBListing{mintedbox}{O{}m!O{}}{%
  breakable=true,
  listing engine=minted,
  listing only,
  minted language=#2,
  minted style=default,
  minted options={%
    linenos,
    gobble=0,
    breaklines=true,
    breakafter=,,,
    fontsize=\small,
    numbersep=8pt,
    #1},
  boxsep=0pt,
  left skip=0pt,
  right skip=0pt,
  left=25pt,
  right=0pt,
  top=3pt,
  bottom=3pt,
  arc=5pt,
  leftrule=0pt,
  rightrule=0pt,
  bottomrule=2pt,
  toprule=2pt,
  colback=bg,
  colframe=orange!70,
  enhanced,
  overlay={%
    \begin{tcbclipinterior}
    \fill[orange!20!white] (frame.south west) rectangle ([xshift=20pt]frame.north west);
    \end{tcbclipinterior}},
  #3,
}

\lstset{
    language=C,
    basicstyle=\ttfamily\small,
    keywordstyle=\color{blue},
    stringstyle=\color{orange},
    commentstyle=\color{green!60!black},
    numbers=left,
    numberstyle=\tiny\color{gray},
    breaklines=true,
    showstringspaces=false,
}

\title{4.4.10}
\author{AI25BTECH11014 - Gooty Suhas}

\begin{document}
\frame{\titlepage}

\begin{frame}{Question}
Point \( \vec{P} \) divides the line segment joining:
\[
\vec{A} = \myvec{2 \\ 1},\quad
\vec{B} = \myvec{k \\ 8}
\]
in the ratio:
\[
\frac{AP}{PB} = \frac{1}{3}
\]
and lies on the line:
\[
\myvec{2 & -1} \vec{P} = -1
\]
\end{frame}

\begin{frame}{Section Formula (Matrix Form)}
Let the ratio be \( m:n = 1:3 \).  
Then the section formula becomes:
\[
\vec{P} = \frac{n\vec{A} + m\vec{B}}{m + n}
= \frac{3\vec{A} + 1\vec{B}}{4}
\]
Substitute:
\[
\vec{A} = \myvec{2 \\ 1},\quad \vec{B} = \myvec{k \\ 8}
\Rightarrow
\vec{P} = \frac{1}{4} \left( 3 \myvec{2 \\ 1} + \myvec{k \\ 8} \right)
= \frac{1}{4} \myvec{6 + k \\ 11}
= \myvec{\frac{6 + k}{4} \\ \frac{11}{4}}
\]
\end{frame}




\begin{frame}{Substitute into Line Equation}
Given:
\[
\myvec{2 & -1} \vec{P} = -1
\]
Substitute:
\[
\myvec{2 & -1} \myvec{\frac{6 + k}{4} \\ \frac{11}{4}} = -1
\Rightarrow \frac{2(6 + k) - 11}{4} = -1
\]
\[
\Rightarrow \frac{12 + 2k - 11}{4} = -1
\Rightarrow \frac{1 + 2k}{4} = -1
\Rightarrow 1 + 2k = -4
\Rightarrow k = -\frac{5}{2}
\]
\end{frame}



\begin{frame}{Final Answer}
\[
\boxed{k = -\frac{5}{2}}
\]
This is the value of \( k \) such that point \( \vec{P} \) lies on the line and divides \( \vec{A}\vec{B} \) in the ratio \( 1:3 \).
\end{frame}


\begin{frame}[fragile]{Python Code }
\begin{lstlisting}[language=Python]
from sympy import symbols, Matrix, Eq, solve

k = symbols('k')
A = Matrix([2, 1])
B = Matrix([k, 8])
P = (3 * A + B) / 4
n = Matrix([[2, -1]])
eq = Eq(n @ P, -1)
sol = solve(eq, k)
print(sol[0])
\end{lstlisting}
\end{frame}



\begin{frame}[fragile]{C Code }
\begin{lstlisting}[language=C]
#include <stdio.h>

int main() {
    double A_x, A_y, B_y, m, n, rhs;

    scanf("%lf %lf %lf %lf %lf %lf", &A_x, &A_y, &B_y, &m, &n, &rhs);

    double P_y = (n * A_y + m * B_y) / (m + n);
    double k = ((rhs + P_y) * (m + n) - 2.0 * n * A_x) / (2.0 * m);

    printf("%.6f\n", k);
    return 0;
}
\end{lstlisting}
\end{frame}


\begin{frame}[fragile]{Python Code — With .so}
\begin{lstlisting}[language=Python]
import subprocess

inputs = [2.0, 1.0, 8.0, 1.0, 3.0, -1.0]
input_str = ' '.join(map(str, inputs))

result = subprocess.run(
    ["./k_solver"],
    input=input_str,
    capture_output=True,
    text=True
)

k_value = float(result.stdout.strip())
print(k_value)
\end{lstlisting}
\end{frame}






\begin{frame}{Plot}
\begin{figure}[H]
\centering
\includegraphics[width=0.78\linewidth]{Figs/Fig1.png}
\caption{Point \( \vec{P} \) on the line and dividing \( \vec{A}\vec{B} \)}
\end{figure}
\end{frame}

\end{document}




