\let\negmedspace\undefined
\let\negthickspace\undefined
\documentclass[journal]{IEEEtran}
\usepackage[a5paper, margin=10mm, onecolumn]{geometry}
\usepackage{tfrupee}

\setlength{\headheight}{1cm}
\setlength{\headsep}{0mm}
\usepackage{gvv-book}
\usepackage{gvv}
\usepackage{cite}
\usepackage{amsmath,amssymb,amsfonts,amsthm}
\usepackage{algorithmic}
\usepackage{graphicx}
\usepackage{textcomp}
\usepackage{xcolor}
\usepackage{txfonts}
\usepackage{listings}
\usepackage{enumitem}
\usepackage{mathtools}
\usepackage{gensymb}
\usepackage{comment}
\usepackage[breaklinks=true]{hyperref}
\usepackage{tkz-euclide}
\def\inputGnumericTable{}
\usepackage[latin1]{inputenc}
\usepackage{color}
\usepackage{array}
\usepackage{longtable}
\usepackage{calc}
\usepackage{multirow}
\usepackage{hhline}
\usepackage{ifthen}
\usepackage{lscape}
\usepackage{booktabs}
\usepackage{tikz}
\usetikzlibrary{arrows.meta,angles,quotes}

\begin{document}

\bibliographystyle{IEEEtran}
\vspace{3cm}

\title{4.4.10}
\author{AI25BTECH11014 - Gooty Suhas}
{\let\newpage\relax\maketitle}

\renewcommand{\thefigure}{\theenumi}
\renewcommand{\thetable}{\theenumi}
\setlength{\intextsep}{10pt}
\numberwithin{equation}{enumi}
\numberwithin{figure}{enumi}
\renewcommand{\thetable}{\theenumi}

\section*{\large\textbf{Problem}}

Point \( \vec{P} \) divides the line segment joining \( \vec{A} = \myvec{2 \\ 1} \) and \( \vec{B} = \myvec{k \\ 8} \) such that:
\begin{equation}
\frac{AP}{PB} = \frac{1}{3}
\end{equation}
and lies on the line:
\begin{equation}
\myvec{2 & -1} \vec{P} = -1
\end{equation}

\section*{\large\textbf{Solution}}

Let the ratio be \( m:n = 1:3 \).  
Then the section formula in matrix form is:
\begin{equation}
\vec{P} = \frac{n\vec{A} + m\vec{B}}{m + n}
= \frac{3\vec{A} + 1\vec{B}}{4}
\end{equation}

Substitute:
\[
\vec{A} = \myvec{2 \\ 1},\quad \vec{B} = \myvec{k \\ 8}
\Rightarrow
\vec{P} = \frac{1}{4} \left( 3 \myvec{2 \\ 1} + \myvec{k \\ 8} \right)
= \frac{1}{4} \myvec{6 + k \\ 3 + 8}
= \myvec{\frac{6 + k}{4} \\ \frac{11}{4}}
\]

Substitute into the line equation:
\begin{equation}
\myvec{2 & -1} \myvec{\frac{6 + k}{4} \\ \frac{11}{4}} = -1
\end{equation}

Simplify:
\[
\frac{2(6 + k) - 11}{4} = -1
\Rightarrow \frac{12 + 2k - 11}{4} = -1
\Rightarrow \frac{1 + 2k}{4} = -1
\Rightarrow 1 + 2k = -4
\Rightarrow k = -\frac{5}{2}
\]

\section*{\large\textbf{Final Answer}}

\begin{equation}
\boxed{k = -\frac{5}{2}}
\end{equation}

\begin{figure}[H]
\centering
\includegraphics[width=01.1\linewidth]{Figs/Fig1.png}
\caption{Point \( \vec{P} \) dividing \( \vec{A}\vec{B} \) and lying on the line}
\end{figure}

\end{document}










