\let\negmedspace\undefined
\let\negthickspace\undefined
\documentclass[journal]{IEEEtran}
\usepackage[a5paper, margin=10mm, onecolumn]{geometry}
\usepackage{tfrupee}

\setlength{\headheight}{1cm}
\setlength{\headsep}{0mm}
\usepackage{gvv-book}
\usepackage{gvv}
\usepackage{cite}
\usepackage{amsmath,amssymb,amsfonts,amsthm}
\usepackage{algorithmic}
\usepackage{graphicx}
\usepackage{textcomp}
\usepackage{xcolor}
\usepackage{txfonts}
\usepackage{listings}
\usepackage{enumitem}
\usepackage{mathtools}
\usepackage{gensymb}
\usepackage{comment}
\usepackage[breaklinks=true]{hyperref}
\usepackage{tkz-euclide}
\def\inputGnumericTable{}
\usepackage[latin1]{inputenc}
\usepackage{color}
\usepackage{array}
\usepackage{longtable}
\usepackage{calc}
\usepackage{multirow}
\usepackage{hhline}
\usepackage{ifthen}
\usepackage{lscape}
\usepackage{booktabs}
\usepackage{tikz}
\usetikzlibrary{arrows.meta,angles,quotes}

\begin{document}

\bibliographystyle{IEEEtran}
\vspace{3cm}

\title{2.10.3}
\author{AI25BTECH11014 - Gooty Suhas}
{\let\newpage\relax\maketitle}

\renewcommand{\thefigure}{\theenumi}
\renewcommand{\thetable}{\theenumi}
\setlength{\intextsep}{10pt}
\numberwithin{equation}{enumi}
\numberwithin{figure}{enumi}
\renewcommand{\thetable}{\theenumi}

\section*{\large\textbf{Problem}}

Find the unit vector perpendicular to the plane determined by the points:
\[
\vec{P} = \myvec{1 \\ -1 \\ 2}, \quad
\vec{Q} = \myvec{2 \\ 0 \\ -1}, \quad
\vec{R} = \myvec{0 \\ 2 \\ 1}
\]

\section*{\large\textbf{Solution}}

We solve using the plane equation:
\[
\vec{n}^T \vec{x} = c
\quad \text{where} \quad
c = \vec{n}^T \vec{P}
\]

Compute direction vectors:
\[
\vec{Q} - \vec{P} = \myvec{1 \\ 1 \\ -3}, \quad
\vec{R} - \vec{P} = \myvec{-1 \\ 3 \\ -1}
\]

Let \( \vec{n} \) be perpendicular to both:
\[
\vec{n}^T (\vec{Q} - \vec{P}) = 0, \quad
\vec{n}^T (\vec{R} - \vec{P}) = 0
\]

Solving this system gives:
\[
\vec{n} = \myvec{8 \\ 2 \\ 4}
\]

Then:
\[
c = \vec{n}^T \vec{P}
= 8 \cdot 1 + 2 \cdot (-1) + 4 \cdot 2 = 14
\]

So the plane equation is:
\[
\vec{n}^T \vec{x} = 14
\]

Normalize:
\[
\|\vec{n}\| = \sqrt{8^2 + 2^2 + 4^2} = \sqrt{84}
\quad \Rightarrow \quad
\hat{n} = \frac{1}{\sqrt{84}} \myvec{8 \\ 2 \\ 4}
= \myvec{\frac{8}{\sqrt{84}} \\ \frac{2}{\sqrt{84}} \\ \frac{4}{\sqrt{84}}}
\]

Then:
\[
\hat{n}^T \vec{x} = \frac{1}{\sqrt{84}} \vec{n}^T \vec{x}
= \frac{1}{\sqrt{84}} \cdot 14 = 1
\]

\section*{\large\textbf{Final Answer}}

\[
\boxed{
\hat{n} = \myvec{\frac{8}{\sqrt{84}} \\
\frac{2}{\sqrt{84}} \\
\frac{4}{\sqrt{84}}}
\quad \text{and} \quad
\hat{n}^T \vec{x} = 1
}
\]

\begin{figure}[H]
\centering
\includegraphics[width=0.8\linewidth]{Figs/fig1.png}
\caption{Plane and its unit normal}
\end{figure}

\end{document}









