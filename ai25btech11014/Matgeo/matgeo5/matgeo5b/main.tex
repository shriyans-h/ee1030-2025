\documentclass{beamer}
\usepackage[utf8]{inputenc}
\usetheme{Madrid}
\usecolortheme{default}
\usepackage{amsmath,amssymb,amsfonts,amsthm}
\usepackage{txfonts}
\usepackage{tkz-euclide}
\usepackage{listings}
\usepackage[T1]{fontenc}
\usepackage{adjustbox}
\usepackage{array}
\usepackage{tabularx}
\usepackage{gvv}
\usepackage{lmodern}
\usepackage{circuitikz}
\usepackage{tikz}
\usepackage{graphicx}
\setbeamertemplate{page number in head/foot}[totalframenumber]

% Define listing style
\lstset{
  basicstyle=\ttfamily\small,
  keywordstyle=\color{blue},
  stringstyle=\color{orange},
  commentstyle=\color{green!60!black},
  numbers=left,
  numberstyle=\tiny\color{gray},
  breaklines=true,
  showstringspaces=false,
  frame=single,
  captionpos=b
}

\title{2.10.3}
\author{AI25BTECH11014 - Gooty Suhas}
\begin{document}
\frame{\titlepage}

\begin{frame}{Question}
Find the unit vector perpendicular to the plane determined by:
\[
\vec{P} = \myvec{1 \\ -1 \\ 2},\quad
\vec{Q} = \myvec{2 \\ 0 \\ -1},\quad
\vec{R} = \myvec{0 \\ 2 \\ 1}
\]
\end{frame}

\begin{frame}{Direction Vectors}
Compute:
\[
\vec{Q} - \vec{P} = \myvec{1 \\ 1 \\ -3},\quad
\vec{R} - \vec{P} = \myvec{-1 \\ 3 \\ -1}
\]
Let \( \vec{n} \) be perpendicular to both:
\[
\vec{n}^T (\vec{Q} - \vec{P}) = 0,\quad
\vec{n}^T (\vec{R} - \vec{P}) = 0
\]
Solving gives:
\[
\vec{n} = \myvec{8 \\ 2 \\ 4}
\]
\end{frame}

\begin{frame}{Plane Equation}
We use the plane equation:
\[
\vec{n}^T \vec{x} = \vec{n}^T \vec{P}
\]
Compute:
\[
\vec{n}^T \vec{P} = 8 \cdot 1 + 2 \cdot (-1) + 4 \cdot 2 = 14
\Rightarrow \vec{n}^T \vec{x} = 14
\]
Normalize:
\[
\|\vec{n}\| = \sqrt{8^2 + 2^2 + 4^2} = \sqrt{84}
\Rightarrow \hat{n} = \frac{1}{\sqrt{84}} \myvec{8 \\ 2 \\ 4}
\]
Then:
\[
\hat{n}^T \vec{x} = \frac{1}{\sqrt{84}} \cdot 14 = 1
\]
\[
\boxed{\hat{n}^T \vec{x} = 1}
\]
\end{frame}

\begin{frame}{Final Answer}
\[
\boxed{
\hat{n} = \myvec{\frac{8}{\sqrt{84}} \\
\frac{2}{\sqrt{84}} \\
\frac{4}{\sqrt{84}}}
}
\]
This is the unit vector perpendicular to the plane
defined by \( \vec{P}, \vec{Q}, \vec{R} \)
\end{frame}
\begin{frame}[fragile]{Python Code (Part 1)}
\begin{lstlisting}[language=Python]
import numpy as np

P = np.array([1, -1, 2])
Q = np.array([2, 0, -1])
R = np.array([0, 2, 1])

A = Q - P
B = R - P

M = np.array([A, B])
U, S, Vt = np.linalg.svd(M)
\end{lstlisting}
\end{frame}

\begin{frame}[fragile]{Python Code (Part 2)}
\begin{lstlisting}[language=Python]
N = Vt[-1]  # Null space vector

unit_vector = N / np.linalg.norm(N)
print("Unit vector:", unit_vector)
\end{lstlisting}
\end{frame}


\begin{frame}[fragile]{C Code for .so File (Part 1)}
\begin{lstlisting}
#include <math.h>

void normal_vector(float* A,
                   float* B,
                   float* out) {

  float z = 1.0;

  float denom = A[0]*B[1] - B[0]*A[1];
  float x = (B[1]*A[2] - A[1]*B[2]) / denom;
\end{lstlisting}
\end{frame}

\begin{frame}[fragile]{C Code for .so File (Part 2)}
\begin{lstlisting}
  float y = (A[0]*B[2] - B[0]*A[2]) / denom;

  out[0] = x;
  out[1] = y;
  out[2] = z;

  float mag = sqrt(out[0]*out[0] +
                   out[1]*out[1] +
                   out[2]*out[2]);

  for(int i=0;i<3;i++) out[i]/=mag;
}
\end{lstlisting}
\end{frame}





\begin{frame}[fragile]{Python Code Using .so File (Part 1)}
\begin{lstlisting}[language=Python]
import ctypes
import numpy as np

lib = ctypes.CDLL('./libnormal.so')
lib.normal_vector.argtypes = [
  ctypes.POINTER(ctypes.c_float),
  ctypes.POINTER(ctypes.c_float),
  ctypes.POINTER(ctypes.c_float)
]

P = np.array([1, -1, 2], np.float32)
\end{lstlisting}
\end{frame}

\begin{frame}[fragile]{Python Code Using .so File (Part 2)}
\begin{lstlisting}[language=Python]
Q = np.array([2, 0, -1], np.float32)
R = np.array([0, 2, 1], np.float32)

A = Q - P
B = R - P
out = np.zeros(3, np.float32)

lib.normal_vector(
  A.ctypes.data_as(ctypes.POINTER(ctypes.c_float)),
  B.ctypes.data_as(ctypes.POINTER(ctypes.c_float)),
  out.ctypes.data_as(ctypes.POINTER(ctypes.c_float))
)

print("Unit vector:", out)
\end{lstlisting}
\end{frame}



















\begin{frame}{Plot}
\begin{figure}[H]
\centering
\includegraphics[width=0.78\linewidth]{Figs/fig1.png}
\caption{Plane and its unit normal}
\end{figure}
\end{frame}






\end{document}



