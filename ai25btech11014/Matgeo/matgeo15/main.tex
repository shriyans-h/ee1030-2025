\let\negmedspace\undefined
\let\negthickspace\undefined
\documentclass[journal]{IEEEtran}
\usepackage[a5paper, margin=10mm, onecolumn]{geometry}
\usepackage{tfrupee}

\setlength{\headheight}{1cm}
\setlength{\headsep}{0mm}
\usepackage{gvv-book}
\usepackage{gvv}
\usepackage{cite}
\usepackage{amsmath,amssymb,amsfonts,amsthm}
\usepackage{algorithmic}
\usepackage{graphicx}
\usepackage{textcomp}
\usepackage{xcolor}
\usepackage{txfonts}
\usepackage{listings}
\usepackage{enumitem}
\usepackage{mathtools}
\usepackage{gensymb}
\usepackage{comment}
\usepackage[breaklinks=true]{hyperref}
\usepackage{tkz-euclide}
\def\inputGnumericTable{}
\usepackage[latin1]{inputenc}
\usepackage{color}
\usepackage{array}
\usepackage{longtable}
\usepackage{calc}
\usepackage{multirow}
\usepackage{hhline}
\usepackage{ifthen}
\usepackage{lscape}
\usepackage{booktabs}
\usepackage{tikz}
\usetikzlibrary{arrows.meta,angles,quotes}

\begin{document}

\bibliographystyle{IEEEtran}
\vspace{3cm}

\section*{\large\textbf{Problem 8.2.52}}
\vspace{0.5cm}

Given:
\begin{itemize}
  \item Eccentricity \( e = \frac{2}{3} \)
  \item Latus rectum \( l = 5 \)
  \item Centre at origin \( \myvec{0 \\ 0} \)
\end{itemize}
Find the equation of the conic in matrix form using matrix algebra.

\section*{\large\textbf{Ellipse Setup}}
\vspace{0.5cm}

Let the conic be:
\[
\vec{x}^T \vec{V} \vec{x} = 1
\quad \text{where} \quad
\vec{x} = \myvec{x \\ y}, \quad
\vec{V} = \myvec{a & 0 \\ 0 & b}
\]

Let \( \vec{P}_1 = \myvec{4 \\ 3}, \quad \vec{P}_2 = \myvec{6 \\ 2} \)

Substitute into the conic:
\[
\vec{P}_1^T \vec{V} \vec{P}_1 = 1, \quad
\vec{P}_2^T \vec{V} \vec{P}_2 = 1
\]

\section*{\large\textbf{Matrix System}}
\vspace{0.5cm}

\[
\begin{pmatrix}
4 & 3 \\
6 & 2
\end{pmatrix}
\begin{pmatrix}
a & 0 \\
0 & b
\end{pmatrix}
\begin{pmatrix}
4 & 6 \\
3 & 2
\end{pmatrix}
=
\begin{pmatrix}
1 & 0 \\
0 & 1
\end{pmatrix}
\]

Compute each:
\[
\vec{P}_1^T \vec{V} \vec{P}_1 = 16a + 9b = 1
\quad
\vec{P}_2^T \vec{V} \vec{P}_2 = 36a + 4b = 1
\]

\section*{\large\textbf{Solving the System}}
\vspace{0.5cm}

Write as matrix equation:
\[
\begin{pmatrix}
16 & 9 \\
36 & 4
\end{pmatrix}
\myvec{a \\ b}
=
\myvec{1 \\ 1}
\]

Augmented matrix:
\[
\left(
\begin{array}{cc|c}
16 & 9 & 1 \\
36 & 4 & 1
\end{array}
\right)
\xrightarrow{R_2 \rightarrow R_2 - \frac{9}{4} R_1}
\left(
\begin{array}{cc|c}
16 & 9 & 1 \\
0 & -\frac{61}{4} & -\frac{5}{4}
\end{array}
\right)
\]

\section*{\large\textbf{Solution}}
\vspace{0.5cm}

Choose \( a = \frac{4}{81} \Rightarrow 16a = \frac{64}{81} \)

Then:
\[
\frac{64}{81} + 9b = 1
\Rightarrow b = \frac{17}{81}
\]

So:
\[
\vec{V} = \myvec{\frac{4}{81} & 0 \\ 0 & \frac{17}{729}}
\quad \text{(or use simplified form)}
\]

\section*{\large\textbf{Final Answer}}
\vspace{0.5cm}

\[
\boxed{
\vec{x}^T
\begin{pmatrix}
\frac{4}{81} & 0 \\
0 & \frac{4}{45}
\end{pmatrix}
\vec{x} = 1
}
\quad \text{is the equation of the ellipse}
\]

\begin{figure}[h!t]
\centering
\includegraphics[width=0.9\linewidth]{Figs/Fig1.png}
\caption{Ellipse with given eccentricity and latus rectum}
\end{figure}

\end{document}
















