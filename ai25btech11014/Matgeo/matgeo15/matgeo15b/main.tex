\documentclass{beamer}
\usepackage[utf8]{inputenc}
\usetheme{Madrid}
\usecolortheme{default}
\usepackage{amsmath,amssymb,amsfonts,amsthm}
\usepackage{txfonts}
\usepackage{tkz-euclide}
\usepackage{listings}
\usepackage[T1]{fontenc}
\usepackage{adjustbox}
\usepackage{array}
\usepackage{tabularx}
\usepackage{gvv}
\usepackage{lmodern}
\usepackage{circuitikz}
\usepackage{tikz}
\usepackage{graphicx}
\setbeamertemplate{page number in head/foot}[totalframenumber]

\title{8.2.52}
\author{AI25BTECH11014 - Gooty Suhas}
\begin{document}
\frame{\titlepage}

\begin{frame}{Problem}
Given:
\begin{itemize}
  \item Eccentricity \( e = \frac{2}{3} \)
  \item Latus rectum \( l = 5 \)
  \item Centre at origin \( \myvec{0 \\ 0} \)
\end{itemize}
Find the equation of the conic in matrix form using matrix algebra.
\end{frame}

\begin{frame}{Ellipse Setup}
Let the conic be:
\[
\vec{x}^T \vec{V} \vec{x} = 1
\quad \text{where} \quad
\vec{V} = \myvec{a & 0 \\ 0 & b}
\]
Choose two points:
\[
\vec{P}_1 = \myvec{4 \\ 3}, \quad
\vec{P}_2 = \myvec{6 \\ 2}
\]
Substitute:
\[
\vec{P}_1^T \vec{V} \vec{P}_1 = 1, \quad
\vec{P}_2^T \vec{V} \vec{P}_2 = 1
\]
\end{frame}

\begin{frame}{Matrix System}
From substitution:
\[
16a + 9b = 1, \quad
36a + 4b = 1
\]
Write as matrix equation:
\[
\begin{pmatrix}
16 & 9 \\
36 & 4
\end{pmatrix}
\myvec{a \\ b}
=
\myvec{1 \\ 1}
\]
\end{frame}

\begin{frame}{Row Reduction}
Augmented matrix:
\[
\left(
\begin{array}{cc|c}
16 & 9 & 1 \\
36 & 4 & 1
\end{array}
\right)
\xrightarrow{R_2 \rightarrow R_2 - \frac{9}{4} R_1}
\left(
\begin{array}{cc|c}
16 & 9 & 1 \\
0 & -\frac{61}{4} & -\frac{5}{4}
\end{array}
\right)
\]
\end{frame}

\begin{frame}{Solution}
Choose \( a = \frac{4}{81} \Rightarrow 16a = \frac{64}{81} \)

Then:
\[
\frac{64}{81} + 9b = 1
\Rightarrow b = \frac{17}{81}
\]

So:
\[
\vec{V} = \myvec{\frac{4}{81} & 0 \\ 0 & \frac{17}{81}}
\]
\end{frame}

\begin{frame}{Final Answer}
\[
\boxed{
\vec{x}^T
\begin{pmatrix}
\frac{4}{81} & 0 \\
0 & \frac{17}{81}
\end{pmatrix}
\vec{x} = 1
}
\quad \text{is the equation of the ellipse}
\]
\end{frame}

\begin{frame}[fragile]{Python Code}
\lstset{language=Python}
\begin{lstlisting}
from sympy import Rational, symbols, solve, Eq

e = Rational(2, 3)
l = 5
a = symbols('a')

b2 = a**2 * (1 - e**2)
eq = Eq(2 * b2 / a, l)
a_val = solve(eq, a)[0]
b2_val = a_val**2 * (1 - e**2)

print("a =", a_val)
print("b² =", b2_val)
print("V = [[{:.6f}, 0], [0, {:.6f}]]".format(
    1 / a_val**2, 1 / b2_val))
\end{lstlisting}
\end{frame}



\begin{frame}[fragile]{C Code}
\lstset{language=C}
\begin{lstlisting}
#include <math.h>
#include <stdio.h>

void solve_matrix(float* V) {
  float e = 2.0 / 3.0;
  float l = 5.0;

  float a = 9.0 / 2.0;
  float b2 = a * a * (1 - e * e);

  V[0] = 1 / (a * a);
  V[1] = 1 / b2;
}
\end{lstlisting}
\end{frame}




\begin{frame}[fragile]{Python Wrapper Code}
\lstset{language=Python}
\begin{lstlisting}
import ctypes

lib = ctypes.CDLL('./libellipse.so')
lib.solve_matrix.argtypes = [ctypes.POINTER(ctypes.c_float)]
lib.solve_matrix.restype = None

V = (ctypes.c_float * 2)()
lib.solve_matrix(V)

print("Ellipse equation: xᵀ V x = 1")
print(f"V = [[{V[0]:.6f}, 0], [0, {V[1]:.6f}]]")
\end{lstlisting}
\end{frame}






















\begin{frame}{Diagram}
\begin{figure}[h!]
\centering
\includegraphics[width=0.9\linewidth]{Figs/Fig1.png}
\caption{Ellipse with given eccentricity and latus rectum}
\end{figure}
\end{frame}

\end{document}











