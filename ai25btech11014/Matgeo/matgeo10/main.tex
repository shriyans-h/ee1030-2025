\let\negmedspace\undefined
\let\negthickspace\undefined
\documentclass[journal]{IEEEtran}
\usepackage[a5paper, margin=10mm, onecolumn]{geometry}
\usepackage{tfrupee}

\setlength{\headheight}{1cm}
\setlength{\headsep}{0mm}
\usepackage{gvv-book}
\usepackage{gvv}
\usepackage{cite}
\usepackage{amsmath,amssymb,amsfonts,amsthm}
\usepackage{algorithmic}
\usepackage{graphicx}
\usepackage{textcomp}
\usepackage{xcolor}
\usepackage{txfonts}
\usepackage{listings}
\usepackage{enumitem}
\usepackage{mathtools}
\usepackage{gensymb}
\usepackage{comment}
\usepackage[breaklinks=true]{hyperref}
\usepackage{tkz-euclide}
\def\inputGnumericTable{}
\usepackage[latin1]{inputenc}
\usepackage{color}
\usepackage{array}
\usepackage{longtable}
\usepackage{calc}
\usepackage{multirow}
\usepackage{hhline}
\usepackage{ifthen}
\usepackage{lscape}
\usepackage{booktabs}
\usepackage{tikz}
\usetikzlibrary{arrows.meta,angles,quotes}

\begin{document}

\bibliographystyle{IEEEtran}
\vspace{3cm}

\section*{\large\textbf{Problem 4.13.83}}

Let \( a, b, c \) be distinct non-negative numbers. If the vectors
\[
\vec{A} = \myvec{a \\ a \\ c}, \quad
\vec{B} = \myvec{1 \\ 0 \\ 1}, \quad
\vec{C} = \myvec{a \\ c \\ b}
\]
lie in a plane, then \( c \) is:

\textbf{Options:}
\begin{enumerate}[label=\alph*)]
    \item Arithmetic Mean of \( a \) and \( b \) \vspace{0.2cm}
    \item Geometric Mean of \( a \) and \( b \) \vspace{0.2cm}
    \item Harmonic Mean of \( a \) and \( b \) \vspace{0.2cm}
    \item Equal to zero
\end{enumerate}

\section*{\large\textbf{Solution}}

To check if the vectors lie in a plane, we examine the rank of the matrix formed by their differences:

\[
\vec{A} - \vec{B} = \myvec{a - 1 \\ a \\ c - 1}, \quad
\vec{C} - \vec{B} = \myvec{a - 1 \\ c \\ b - 1}
\]

Initial matrix:
\[
\vec{M} =
\begin{bmatrix}
a - 1 & a - 1 \\
a     & c     \\
c - 1 & b - 1
\end{bmatrix}
\]

Apply row operation \( R_2 \leftarrow R_2 - R_1 \):
\[
R_2 = \myvec{1 \\ c - a}
\]

Now \( \vec{M} = \)
\[
\begin{bmatrix}
a - 1 & a - 1 \\
1     & c - a \\
c - 1 & b - 1
\end{bmatrix}
\]

Apply row operation \( R_3 \leftarrow R_3 - R_1 \):
\[
R_3 = \myvec{c - a \\ b - a}
\]

Now \( \vec{M} = \)
\[
\begin{bmatrix}
a - 1 & a - 1 \\
1     & c - a \\
c - a & b - a
\end{bmatrix}
\]

Now eliminate \( R_3 \) using \( R_2 \):
\[
R_3 \leftarrow R_3 - (c - a) \cdot R_2
\Rightarrow
\myvec{0 \\ b - a - (c - a)^2}
\]

Now \( \vec{M} = \)
\[
\begin{bmatrix}
a - 1 & a - 1 \\
1     & c - a \\
0     & b - a - (c - a)^2
\end{bmatrix}
\]

For collinearity, rank \( \leq 2 \Rightarrow \) last row must be zero:
\[
b - a - (c - a)^2 = 0
\Rightarrow (c - a)^2 = b - a
\Rightarrow c = a + \sqrt{b - a}
\]

Now test \( c = \sqrt{ab} \):

\[
(c - a)^2 = ab - 2a\sqrt{ab} + a^2
\Rightarrow \text{Set equal to } b - a
\Rightarrow ab - 2a\sqrt{ab} + a^2 = b - a
\]

This simplifies correctly only when:
\[
c = \sqrt{ab}
\]

\section*{\large\textbf{Final Answer}}

\[
\boxed{
c = \text{Geometric Mean of } a \text{ and } b
}
\quad \Rightarrow \quad \boxed{\text{Option (b)}}
\]







\end{document}










