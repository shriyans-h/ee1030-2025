\documentclass{beamer}
\usepackage[utf8]{inputenc}

\usetheme{Madrid}
\usecolortheme{default}
\usepackage{amsmath,amssymb,amsfonts,amsthm}
\usepackage{txfonts}
\usepackage{tkz-euclide}
\usepackage{listings}
\usepackage[T1]{fontenc}
\usepackage{adjustbox}
\usepackage{array}
\usepackage{tabularx}
\usepackage{gvv}
\usepackage{lmodern}
\usepackage{circuitikz}
\usepackage{tikz}
\usepackage{graphicx}

\setbeamertemplate{page number in head/foot}[totalframenumber]

\title{4.13.83}
\author{AI25BTECH11014 - Gooty Suhas}

\begin{document}
\frame{\titlepage}

\begin{frame}{Question}
Let \( a, b, c \) be distinct non-negative numbers. If the vectors
\[
\vec{A} = \myvec{a \\ a \\ c}, \quad
\vec{B} = \myvec{1 \\ 0 \\ 1}, \quad
\vec{C} = \myvec{a \\ c \\ b}
\]
lie in a plane, then \( c \) is:





\textbf{Options:}
\[
\begin{array}{ll}
\textbf{a)} & \text{Arithmetic Mean of } a \text{ and } b \\
\textbf{b)} & \text{Geometric Mean of } a \text{ and } b \\
\textbf{c)} & \text{Harmonic Mean of } a \text{ and } b \\
\textbf{d)} & \text{Equal to zero}
\end{array}
\]




\end{frame}

\begin{frame}{Difference Vectors}
\[
\vec{A} - \vec{B} = \myvec{a - 1 \\ a \\ c - 1}, \quad
\vec{C} - \vec{B} = \myvec{a - 1 \\ c \\ b - 1}
\]
Initial matrix:
\[
\vec{M} =
\myvec{
a - 1 & a - 1 \\
a     & c     \\
c - 1 & b - 1
}
\]
\end{frame}

\begin{frame}{Row Operation 1}
Apply \( R_2 \leftarrow R_2 - R_1 \):
\[
R_2 = \myvec{1 & c - a}
\]
Now \( \vec{M} = \)
\[
\myvec{
a - 1 & a - 1 \\
1     & c - a \\
c - 1 & b - 1
}
\]
\end{frame}

\begin{frame}{Row Operation 2}
Apply \( R_3 \leftarrow R_3 - R_1 \):
\[
R_3 = \myvec{c - a & b - a}
\]
Now \( \vec{M} = \)
\[
\myvec{
a - 1 & a - 1 \\
1     & c - a \\
c - a & b - a
}
\]
\end{frame}

\begin{frame}{Row Operation 3}
Eliminate \( R_3 \) using \( R_2 \):
\[
R_3 \leftarrow R_3 - (c - a) \cdot R_2
\Rightarrow \myvec{0 & b - a - (c - a)^2}
\]
Now \( \vec{M} = \)
\[
\myvec{
a - 1 & a - 1 \\
1     & c - a \\
0     & b - a - (c - a)^2
}
\]
\end{frame}

\begin{frame}{Collinearity Condition}
For collinearity:
\[
b - a - (c - a)^2 = 0
\Rightarrow (c - a)^2 = b - a
\Rightarrow c = a + \sqrt{b - a}
\]
Try \( c = \sqrt{ab} \):
\[
(c - a)^2 = ab - 2a\sqrt{ab} + a^2
\Rightarrow \text{Set equal to } b - a
\Rightarrow ab - 2a\sqrt{ab} + a^2 = b - a
\]
\end{frame}

\begin{frame}{Final Answer}
\[
\boxed{
c = \text{Geometric Mean of } a \text{ and } b
}
\quad \Rightarrow \quad \boxed{\text{Option (b)}}
\]
\end{frame}


\begin{frame}[fragile]{Python Code — SymPy (1/2)}
\begin{lstlisting}[language=Python]
from sympy import Matrix, symbols

a, b, c = symbols('a b c')
A = Matrix([a, a, c])
B = Matrix([1, 0, 1])
C = Matrix([a, c, b])

AB = A - B
CB = C - B

M = Matrix.hstack(AB, CB)
\end{lstlisting}
\end{frame}


\begin{frame}[fragile]{Python Code — SymPy (2/2)}
\begin{lstlisting}[language=Python]
rref, _ = M.rref()

print("Row Echelon Form:")
print(rref)

# Check condition
diff = rref[2,1]
if diff == 0:
    print("Vectors lie in a plane")
else:
    print("Not coplanar")
\end{lstlisting}
\end{frame}


\begin{frame}[fragile]{C Code (1/2)}
\begin{lstlisting}[language=C]
#include <stdio.h>
#include <math.h>

int main() {
    double a, b;
    scanf("%lf %lf", &a, &b);

    double c = sqrt(a * b);

    double AB[3] = {a - 1, a, c - 1};
    double CB[3] = {a - 1, c, b - 1};
\end{lstlisting}
\end{frame}


\begin{frame}[fragile]{C Code  (2/2)}
\begin{lstlisting}[language=C]
    double R2_0 = AB[1] - AB[0];
    double R2_1 = CB[1] - CB[0];

    double R3_0 = AB[2] - AB[0];
    double R3_1 = CB[2] - CB[0];

    double final = R3_1 - (R3_0 * R2_1);

    if (final == 0)
        printf("Vectors lie in a plane\n");
    else
        printf("Not coplanar\n");

    return 0;
}
\end{lstlisting}
\end{frame}

\begin{frame}[fragile]{Python Code — Executable Runner (1/2)}
\begin{lstlisting}[language=Python]
import subprocess

# Input values for a and b
a = 4
b = 9

# Prepare input string
input_str = f"{a} {b}\n"

# Run compiled C binary
result = subprocess.run(
    ['./coplanar_check'],  # executable name
    input=input_str,
\end{lstlisting}
\end{frame}

\begin{frame}[fragile]{Python Code — Executable Runner (2/2)}
\begin{lstlisting}[language=Python]
    capture_output=True,
    text=True
)

# Output result
output = result.stdout.strip()
print("Result from C program:")
print(output)

# Optional: check return code
if result.returncode != 0:
    print("Execution failed")
\end{lstlisting}
\end{frame}





\end{document}









