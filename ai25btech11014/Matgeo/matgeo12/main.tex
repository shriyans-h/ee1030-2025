\documentclass[journal]{IEEEtran}
\usepackage[a5paper, margin=10mm, onecolumn]{geometry}
\usepackage{amsmath,amssymb}
\usepackage{txfonts}
\usepackage{graphicx}
\usepackage{gvv}
\usepackage{listings}
\usepackage{xcolor}

% Listings configuration
\lstset{
  frame=single,
  breaklines=true,
  columns=fullflexible,
  basicstyle=\ttfamily\footnotesize,
  keywordstyle=\color{blue},
  commentstyle=\color{gray},
  stringstyle=\color{orange},
  showstringspaces=false
}
\title{5.5.31 }
\author{AI25BTECH11014 — Gooty Suhas}

\begin{document}
\maketitle

\section*{Problem Statement}

Solve the following system using matrix row operations.  
Let \( \vec{M} = \myvec{\dfrac{1}{x} \\ \dfrac{1}{y} \\ \dfrac{1}{z}} \), and find its value.

\[
\begin{aligned}
2 \cdot \dfrac{1}{x} + 3 \cdot \dfrac{1}{y} + 10 \cdot \dfrac{1}{z} &= 4 \\
4 \cdot \dfrac{1}{x} + 6 \cdot \dfrac{1}{y} + 5 \cdot \dfrac{1}{z} &= 1 \\
6 \cdot \dfrac{1}{x} + 9 \cdot \dfrac{1}{y} + 20 \cdot \dfrac{1}{z} &= 2
\end{aligned}
\]

\vspace{0.4cm}
\section*{Matrix Form}

Coefficient matrix:
\[
\vec{A} =
\left[
\begin{array}{c}
\myvec{2 & 3 & 10} \\
\myvec{4 & 6 & 5} \\
\myvec{6 & 9 & 20}
\end{array}
\right],
\quad
\vec{B} = \myvec{4 \\ 1 \\ 2}
\]

Augmented matrix:
\[
\left[
\begin{array}{c}
\myvec{2 & 3 & 10 & 4} \\
\myvec{4 & 6 & 5 & 1} \\
\myvec{6 & 9 & 20 & 2}
\end{array}
\right]
\]

\vspace{0.4cm}
\section*{Row Operations}

Step 1: \( R_1 \leftarrow R_1 \div 2 \)
\[
\left[
\begin{array}{c}
\myvec{1 & \dfrac{3}{2} & 5 & 2} \\
\myvec{4 & 6 & 5 & 1} \\
\myvec{6 & 9 & 20 & 2}
\end{array}
\right]
\]

Step 2: \( R_2 \leftarrow R_2 - 4 \cdot R_1 \)
\[
\left[
\begin{array}{c}
\myvec{1 & \dfrac{3}{2} & 5 & 2} \\
\myvec{0 & 0 & -15 & -7} \\
\myvec{6 & 9 & 20 & 2}
\end{array}
\right]
\]

Step 3: \( R_3 \leftarrow R_3 - 6 \cdot R_1 \)
\[
\left[
\begin{array}{c}
\myvec{1 & \dfrac{3}{2} & 5 & 2} \\
\myvec{0 & 0 & -15 & -7} \\
\myvec{0 & 0 & -10 & -10}
\end{array}
\right]
\]

Step 4: \( R_3 \leftarrow R_3 - R_2 \)
\[
\left[
\begin{array}{c}
\myvec{1 & \dfrac{3}{2} & 5 & 2} \\
\myvec{0 & 0 & -15 & -7} \\
\myvec{0 & 0 & 5 & -3}
\end{array}
\right]
\]

Step 5: \( R_3 \leftarrow R_3 \div 5 \)
\[
\left[
\begin{array}{c}
\myvec{1 & \dfrac{3}{2} & 5 & 2} \\
\myvec{0 & 0 & -15 & -7} \\
\myvec{0 & 0 & 1 & -\dfrac{3}{5}}
\end{array}
\right]
\]

Step 6: \( R_2 \leftarrow R_2 + 15 \cdot R_3 \)
\[
\left[
\begin{array}{c}
\myvec{1 & \dfrac{3}{2} & 5 & 2} \\
\myvec{0 & 0 & 0 & \dfrac{4}{5}} \\
\myvec{0 & 0 & 1 & -\dfrac{3}{5}}
\end{array}
\right]
\]

Step 7: \( R_1 \leftarrow R_1 - 5 \cdot R_3 \)
\[
\left[
\begin{array}{c}
\myvec{1 & \dfrac{3}{2} & 0 & 5} \\
\myvec{0 & 0 & 0 & \dfrac{4}{5}} \\
\myvec{0 & 0 & 1 & -\dfrac{3}{5}}
\end{array}
\right]
\]

\vspace{0.4cm}

\section*{Final Matrix}

After performing row operations, we arrive at:

\[
\left[
\begin{array}{c}
\myvec{1 & \dfrac{3}{2} & 0 & 5} \\
\myvec{0 & 0 & 0 & \dfrac{4}{5}} \\
\myvec{0 & 0 & 1 & -\dfrac{3}{5}}
\end{array}
\right]
\]

This corresponds to:
\[
\begin{aligned}
\dfrac{1}{x} &= 5 \\
\dfrac{1}{z} &= -\dfrac{3}{5} \\
0 &= \dfrac{4}{5} \quad \text{(contradiction)}
\end{aligned}
\]

\vspace{0.3cm}
\textbf{Conclusion:} The system is inconsistent and has no solution.  
The vector \( \vec{M} = \myvec{\dfrac{1}{x} \\ \dfrac{1}{y} \\ \dfrac{1}{z}} \) is undefined.









\begin{figure}[h!]
\centering
\includegraphics[width=1.3\linewidth]{Figs/Fig1.png}
\caption{Approximate solution}
\label{fig:your_label}
\end{figure}



\end{document}








