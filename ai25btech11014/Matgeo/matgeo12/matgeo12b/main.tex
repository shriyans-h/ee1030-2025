\documentclass{beamer}
\usepackage[utf8]{inputenc}

\usetheme{Madrid}
\usecolortheme{default}
\usepackage{amsmath,amssymb,amsfonts,amsthm}
\usepackage{txfonts}
\usepackage{tkz-euclide}
\usepackage{listings}
\usepackage[T1]{fontenc}
\usepackage{adjustbox}
\usepackage{array}
\usepackage{tabularx}
\usepackage{gvv}
\usepackage{lmodern}
\usepackage{circuitikz}
\usepackage{tikz}
\usepackage{graphicx}

\setbeamertemplate{page number in head/foot}[totalframenumber]

\usepackage{tcolorbox}
\tcbuselibrary{minted,breakable,xparse,skins}

\definecolor{bg}{gray}{0.95}
\DeclareTCBListing{mintedbox}{O{}m!O{}}{%
  breakable=true,
  listing engine=minted,
  listing only,
  minted language=#2,
  minted style=default,
  minted options={%
    linenos,
    gobble=0,
    breaklines=true,
    breakafter=,,
    fontsize=\small,
    numbersep=8pt,
    #1},
  boxsep=0pt,
  left skip=0pt,
  right skip=0pt,
  left=25pt,
  right=0pt,
  top=3pt,
  bottom=3pt,
  arc=5pt,
  leftrule=0pt,
  rightrule=0pt,
  bottomrule=2pt,
  toprule=2pt,
  colback=bg,
  colframe=orange!70,
  enhanced,
  overlay={%
    \begin{tcbclipinterior}
    \fill[orange!20!white] (frame.south west) rectangle ([xshift=20pt]frame.north west);
    \end{tcbclipinterior}},
  #3,
}

\lstset{
    language=C,
    basicstyle=\ttfamily\small,
    keywordstyle=\color{blue},
    stringstyle=\color{orange},
    commentstyle=\color{green!60!black},
    numbers=left,
    numberstyle=\tiny\color{gray},
    breaklines=true,
    showstringspaces=false,
}

\title{5.5.31}
\author{AI25BTECH11014 - Suhas}

\begin{document}

\frame{\titlepage}

\begin{frame}{Question}
Solve the following system using matrix row operations.  
Let \( \vec{M} = \myvec{\dfrac{1}{x} \\ \dfrac{1}{y} \\ \dfrac{1}{z}} \), and find its value.

\[
\begin{aligned}
2 \cdot \dfrac{1}{x} + 3 \cdot \dfrac{1}{y} + 10 \cdot \dfrac{1}{z} &= 4 \\
4 \cdot \dfrac{1}{x} + 6 \cdot \dfrac{1}{y} + 5 \cdot \dfrac{1}{z} &= 1 \\
6 \cdot \dfrac{1}{x} + 9 \cdot \dfrac{1}{y} + 20 \cdot \dfrac{1}{z} &= 2
\end{aligned}
\]
\end{frame}

\begin{frame}{Matrix Form}
Augmented matrix:
\[
\left[
\begin{array}{c}
\myvec{2 & 3 & 10 & 4} \\
\myvec{4 & 6 & 5 & 1} \\
\myvec{6 & 9 & 20 & 2}
\end{array}
\right]
\]
\end{frame}

\begin{frame}{Row Operations}
Step 1: \( R_1 \leftarrow R_1 \div 2 \)
\[
\left[
\begin{array}{c}
\myvec{1 & \dfrac{3}{2} & 5 & 2} \\
\myvec{4 & 6 & 5 & 1} \\
\myvec{6 & 9 & 20 & 2}
\end{array}
\right]
\]

Step 2: \( R_2 \leftarrow R_2 - 4 \cdot R_1 \)
\[
\left[
\begin{array}{c}
\myvec{1 & \dfrac{3}{2} & 5 & 2} \\
\myvec{0 & 0 & -15 & -7} \\
\myvec{6 & 9 & 20 & 2}
\end{array}
\right]
\]
\end{frame}

\begin{frame}{Row Operations (contd.)}
Step 3: \( R_3 \leftarrow R_3 - 6 \cdot R_1 \)
\[
\left[
\begin{array}{c}
\myvec{1 & \dfrac{3}{2} & 5 & 2} \\
\myvec{0 & 0 & -15 & -7} \\
\myvec{0 & 0 & -10 & -10}
\end{array}
\right]
\]

Step 4: \( R_3 \leftarrow R_3 - R_2 \)
\[
\left[
\begin{array}{c}
\myvec{1 & \dfrac{3}{2} & 5 & 2} \\
\myvec{0 & 0 & -15 & -7} \\
\myvec{0 & 0 & 5 & -3}
\end{array}
\right]
\]
\end{frame}

\begin{frame}{Row Operations (contd.)}
Step 5: \( R_3 \leftarrow R_3 \div 5 \)
\[
\left[
\begin{array}{c}
\myvec{1 & \dfrac{3}{2} & 5 & 2} \\
\myvec{0 & 0 & -15 & -7} \\
\myvec{0 & 0 & 1 & -\dfrac{3}{5}}
\end{array}
\right]
\]

Step 6: \( R_2 \leftarrow R_2 + 15 \cdot R_3 \)
\[
\left[
\begin{array}{c}
\myvec{1 & \dfrac{3}{2} & 5 & 2} \\
\myvec{0 & 0 & 0 & \dfrac{4}{5}} \\
\myvec{0 & 0 & 1 & -\dfrac{3}{5}}
\end{array}
\right]
\]
\end{frame}

\begin{frame}{Row Operations (final)}
Step 7: \( R_1 \leftarrow R_1 - 5 \cdot R_3 \)
\[
\left[
\begin{array}{c}
\myvec{1 & \dfrac{3}{2} & 0 & 5} \\
\myvec{0 & 0 & 0 & \dfrac{4}{5}} \\
\myvec{0 & 0 & 1 & -\dfrac{3}{5}}
\end{array}
\right]
\]
\end{frame}

\begin{frame}{Conclusion}
From the final matrix:
\[
\left[
\begin{array}{c}
\myvec{1 & \dfrac{3}{2} & 0 & 5} \\
\myvec{0 & 0 & 0 & \dfrac{4}{5}} \\
\myvec{0 & 0 & 1 & -\dfrac{3}{5}}
\end{array}
\right]
\]

This corresponds to:
\[
\begin{aligned}
\dfrac{1}{x} &= 5 \\
\dfrac{1}{z} &= -\dfrac{3}{5} \\
0 &= \dfrac{4}{5} \quad \text{(contradiction)}
\end{aligned}
\]

\vspace{0.3cm}
Since the second row implies a false statement,  
\[
\boxed{\text{The system is inconsistent and has no solution.}}
\]
\end{frame}


\begin{frame}[fragile]{Python Code (Part 1)}
\begin{lstlisting}[language=Python]
import numpy as np

A = np.array([
    [2, 3, 10],
    [4, 6, 5],
    [6, 9, 20]
], dtype=np.float32)

B = np.array([4, 1, 2], dtype=np.float32)

U, residuals, rank, s = np.linalg.lstsq(A, B, rcond=None)
\end{lstlisting}
\end{frame}

\begin{frame}[fragile]{Python Code (Part 2)}
\begin{lstlisting}[language=Python]
if residuals.size > 0 and residuals[0] > 1e-6:
    print("System is inconsistent. No exact solution exists.")
else:
    u, v, w = U
    x, y, z = 1/u, 1/v, 1/w
    print(f"x = {x:.3f}, y = {y:.3f}, z = {z:.3f}")
\end{lstlisting}
\end{frame}

\begin{frame}[fragile]{Python Code Using .so (Part 3)}
\begin{lstlisting}[language=Python]
# After calling solve_system(...)
if abs(U[0]*2 + U[1]*3 + U[2]*10 - 4) > 1e-3 or \
   abs(U[0]*4 + U[1]*6 + U[2]*5 - 1) > 1e-3 or \
   abs(U[0]*6 + U[1]*9 + U[2]*20 - 2) > 1e-3:
    print("System is inconsistent. No exact solution exists.")
else:
    x, y, z = 1/U[0], 1/U[1], 1/U[2]
    print(f"x = {x}, y = {y}, z = {z}")
\end{lstlisting}
\end{frame}

\begin{frame}[fragile]{Python Code Using .so (Part 1)}
\begin{lstlisting}[language=Python]
import ctypes
import numpy as np

lib = ctypes.CDLL('./libsystem.so')
lib.solve_system.argtypes = [
    ctypes.POINTER(ctypes.c_float),
    ctypes.POINTER(ctypes.c_float),
    ctypes.POINTER(ctypes.c_float)
]

lib.solve_system.restype = None
\end{lstlisting}
\end{frame}


\begin{frame}[fragile]{Python Code Using .so (Part 2)}
\begin{lstlisting}[language=Python]
A = np.array([
    [2, 3, 10],
    [4, 6, 5],
    [6, 9, 20]
], dtype=np.float32).flatten()

B = np.array([4, 1, 2], dtype=np.float32)
U = np.zeros(3, dtype=np.float32)

lib.solve_system(
    A.ctypes.data_as(ctypes.POINTER(ctypes.c_float)),
    B.ctypes.data_as(ctypes.POINTER(ctypes.c_float)),
    U.ctypes.data_as(ctypes.POINTER(ctypes.c_float))
)
\end{lstlisting}
\end{frame}



\begin{frame}[fragile]{Python Code Using .so (Part 3)}
\begin{lstlisting}[language=Python]
def check_inconsistency(A, B, U):
    A = A.reshape(3, 3)
    residuals = A @ U - B
    return np.any(np.abs(residuals) > 1e-3)

if check_inconsistency(A, B, U):
    print("System is inconsistent. No exact solution exists.")
else:
    x, y, z = 1/U[0], 1/U[1], 1/U[2]
    print(f"x = {x:.3f}, y = {y:.3f}, z = {z:.3f}")
\end{lstlisting}
\end{frame}













\begin{frame}{Plot}
\begin{figure}[H]
    \centering
    \includegraphics[width=0.9\linewidth]{Figs/Fig1.png}
    \caption{Approximate solution}
    \label{fig:fig1}
\end{figure}
\end{frame}

\end{document}






