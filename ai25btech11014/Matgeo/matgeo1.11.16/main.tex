\let\negmedspace\undefined
\let\negthickspace\undefined
\documentclass[journal]{IEEEtran}
\usepackage[a5paper, margin=10mm, onecolumn]{geometry}
\usepackage{tfrupee}

\setlength{\headheight}{1cm}
\setlength{\headsep}{0mm}
\usepackage{gvv-book}
\usepackage{gvv}
\usepackage{cite}
\usepackage{amsmath,amssymb,amsfonts,amsthm}
\usepackage{algorithmic}
\usepackage{graphicx}
\usepackage{textcomp}
\usepackage{xcolor}
\usepackage{txfonts}
\usepackage{listings}
\usepackage{enumitem}
\usepackage{mathtools}
\usepackage{gensymb}
\usepackage{comment}
\usepackage[breaklinks=true]{hyperref}
\usepackage{tkz-euclide}
\def\inputGnumericTable{}
\usepackage[latin1]{inputenc}
\usepackage{color}
\usepackage{array}
\usepackage{longtable}
\usepackage{calc}
\usepackage{multirow}
\usepackage{hhline}
\usepackage{ifthen}
\usepackage{lscape}
\usepackage{booktabs}
\usepackage{tikz}
\usetikzlibrary{arrows.meta,angles,quotes}

\begin{document}

\bibliographystyle{IEEEtran}
\vspace{3cm}

\title{1.11.16}
\author{AI25BTECH11014 - Gooty Suhas}
{\let\newpage\relax\maketitle}

\renewcommand{\thefigure}{\theenumi}
\renewcommand{\thetable}{\theenumi}
\setlength{\intextsep}{10pt}
\numberwithin{equation}{enumi}
\numberwithin{figure}{enumi}
\renewcommand{\thetable}{\theenumi}

\section*{\large\textbf{Problem}}
\vspace{0.5cm}

The Cartesian equation of a line is  
\[
\frac{x - 1}{2} = \frac{y + 2}{2} = \frac{z - 3}{3}
\]  
Find the direction cosines of a line parallel to this line.

\section*{\large\textbf{Solution}}
\vspace{0.5cm}

Let the direction vector be named:
\begin{align}
\mathbf{D} = \myvec{2\\2\\3}
\end{align}

Compute the magnitude of \(\mathbf{D}\):
\begin{align}
\|\mathbf{D}\| &= \sqrt{2^2 + 2^2 + 3^2} \\
&= \sqrt{4 + 4 + 9} = \sqrt{17}
\end{align}

Normalize the direction vector:
\begin{align}
\mathbf{L} = \frac{1}{\sqrt{17}} \cdot \mathbf{D}
\end{align}

Let the point vector be:
\begin{align}
\mathbf{P} = \myvec{1\\-2\\3}
\end{align}

Then the line can be expressed in matrix form as:
\begin{align}
\mathbf{R} = \mathbf{P} + \lambda \cdot \mathbf{D}
\end{align}

Where \(\lambda \in \mathbb{R}\) is a scalar parameter.


\begin{figure}[h!t]
    \centering
    \includegraphics[width=1\linewidth]{figs/fig1}
    \caption{The line}
    \label{fig:fig1}
\end{figure}








\end{document}







































