\let\negmedspace\undefined
\let\negthickspace\undefined
\documentclass[journal]{IEEEtran}
\usepackage[a5paper, margin=10mm, onecolumn]{geometry}
\usepackage{tfrupee}

\setlength{\headheight}{1cm}
\setlength{\headsep}{0mm}
\usepackage{gvv-book}
\usepackage{gvv}
\usepackage{cite}
\usepackage{amsmath,amssymb,amsfonts,amsthm}
\usepackage{algorithmic}
\usepackage{graphicx}
\usepackage{textcomp}
\usepackage{xcolor}
\usepackage{txfonts}
\usepackage{listings}
\usepackage{enumitem}
\usepackage{mathtools}
\usepackage{gensymb}
\usepackage{comment}
\usepackage[breaklinks=true]{hyperref}
\usepackage{tkz-euclide}
\def\inputGnumericTable{}
\usepackage[latin1]{inputenc}
\usepackage{color}
\usepackage{array}
\usepackage{longtable}
\usepackage{calc}
\usepackage{multirow}
\usepackage{hhline}
\usepackage{ifthen}
\usepackage{lscape}
\usepackage{booktabs}
\usepackage{tikz}
\usetikzlibrary{arrows.meta,angles,quotes}

\begin{document}

\bibliographystyle{IEEEtran}
\vspace{3cm}

\section*{\large\textbf{Problem 4.12.19}}

The point \( \vec{P} = \myvec{4 \\ 1} \) undergoes the following two successive transformations:

\begin{enumerate}
    \item Reflection about the line \( y = x \)
    \item Translation through a distance of 2 units along the positive x-axis
\end{enumerate}

Find the final coordinates of the point.

\textbf{Options:}
\begin{enumerate}[label=\alph*)]
    \item \( \myvec{4 \\ 3} \) \vspace{0.2cm}
    \item \( \myvec{3 \\ 4} \) \vspace{0.2cm}
    \item \( \myvec{1 \\ 4} \) \vspace{0.2cm}
    \item \( \myvec{3.5 \\ 3.5} \)
\end{enumerate}

\section*{\large\textbf{Solution}}

Let the original point be:
\[
\vec{P} = \myvec{4 \\ 1}
\]

The reflection of \( \vec{P} \) about the line \( y = x \) is:
\[
\vec{R} = \myvec{1 \\ 4}
\]

Let the final point after translation be:
\[
\vec{Q} = \myvec{x \\ 4}
\]

To find \( x \), we use the fact that \( \vec{P}, \vec{R}, \vec{Q} \) are collinear. So the rank of the matrix formed by their differences must be 1:

\[
\vec{R} - \vec{P} = \myvec{-3 \\ 3}, \quad
\vec{Q} - \vec{R} = \myvec{x - 1 \\ 0}
\]

Form the matrix:
\[
\vec{M} = \myvec{-3 & x - 1 \\ 3 & 0}
\]

Apply row operations to reduce to echelon form:

\[
R_1 = \myvec{-3 & x - 1}, \quad
R_2 = \myvec{3 & 0}
\]

Add \( R_1 + R_2 \):
\[
R_1 \leftarrow R_1 + R_2 = \myvec{0 & x - 1}
\]

Now the matrix becomes:
\[
\myvec{3 & 0 \\ 0 & x - 1}
\]

For rank to be 1 (collinearity), second row must be zero:
\[
x - 1 = 0 \Rightarrow x = 1
\]

Then:
\[
\vec{Q} = \myvec{1 \\ 4}
\]

Now apply translation:
\[
\vec{T} = \myvec{2 \\ 0}, \quad
\vec{F} = \vec{Q} + \vec{T} = \myvec{1 \\ 4} + \myvec{2 \\ 0} = \myvec{3 \\ 4}
\]

\section*{\large\textbf{Final Answer}}

\[
\boxed{
\text{Final coordinates: } \myvec{3 \\ 4}
}
\quad \Rightarrow \quad \boxed{\text{Option (b)}}
\]

\section*{\large\textbf{Plot}}

\begin{figure}[h!]
\centering
\includegraphics[width=0.85\linewidth]{Figs/Fig1.png}
\caption{Transformation of point \( \vec{P} \) through reflection and translation}
\end{figure}

\end{document}




