\documentclass{beamer}
\usepackage[utf8]{inputenc}

\usetheme{Madrid}
\usecolortheme{default}
\usepackage{amsmath,amssymb,amsfonts,amsthm}
\usepackage{txfonts}
\usepackage{tkz-euclide}
\usepackage{listings}
\usepackage[T1]{fontenc}
\usepackage{adjustbox}
\usepackage{array}
\usepackage{tabularx}
\usepackage{gvv}
\usepackage{lmodern}
\usepackage{circuitikz}
\usepackage{tikz}
\usepackage{graphicx}

\setbeamertemplate{page number in head/foot}[totalframenumber]

\title{4.12.19}
\author{AI25BTECH11014 - Gooty Suhas}

\begin{document}
\frame{\titlepage}

\begin{frame}{Question}
The point \( \vec{P} = \myvec{4 \\ 1} \) undergoes:
\begin{enumerate}
    \item Reflection about the line \( y = x \)
    \item Translation 2 units along the positive x-axis
\end{enumerate}
Find the final coordinates of the point.

\textbf{Options:}
\begin{enumerate}[label=\alph*)]
    \item \( \myvec{4 \\ 3} \) \vspace{0.2cm}
    \item \( \myvec{3 \\ 4} \) \vspace{0.2cm}
    \item \( \myvec{1 \\ 4} \) \vspace{0.2cm}
    \item \( \myvec{3.5 \\ 3.5} \)
\end{enumerate}
\end{frame}

\begin{frame}{Reflection}
Original point:
\[
\vec{P} = \myvec{4 \\ 1}
\]
Reflection about \( y = x \):
\[
\vec{R} = \myvec{1 \\ 4}
\]
Let final point be:
\[
\vec{Q} = \myvec{x \\ 4}
\]
\end{frame}

\begin{frame}{Collinearity via Rank}
Vectors:
\[
\vec{R} - \vec{P} = \myvec{-3 \\ 3}, \quad
\vec{Q} - \vec{R} = \myvec{x - 1 \\ 0}
\]
Matrix:
\[
\vec{M} = \myvec{-3 & x - 1 \\ 3 & 0}
\]
Row operations:
\[
R_1 \leftarrow R_1 + R_2 = \myvec{0 & x - 1}
\]
Echelon form:
\[
\myvec{3 & 0 \\ 0 & x - 1}
\Rightarrow x - 1 = 0 \Rightarrow x = 1
\]
\end{frame}

\begin{frame}{Translation}
Translation vector:
\[
\vec{T} = \myvec{2 \\ 0}
\]
Final point:
\[
\vec{F} = \vec{Q} + \vec{T} = \myvec{1 \\ 4} + \myvec{2 \\ 0} = \myvec{3 \\ 4}
\]
\end{frame}

\begin{frame}{Final Answer}
\[
\boxed{
\text{Final coordinates: } \myvec{3 \\ 4}
}
\quad \Rightarrow \quad \boxed{\text{Option (b)}}
\]
\end{frame}

\begin{frame}[fragile]{Python Code — SymPy}
\begin{lstlisting}[language=Python]
from sympy import Matrix

# Original point
P = Matrix([4, 1])

# Reflection about y = x
R = Matrix([[0, 1], [1, 0]]) * P

# Translation vector
T = Matrix([2, 0])
F = R + T

print("Reflected point:", R)
print("Final point:", F)
\end{lstlisting}
\end{frame}

\begin{frame}[fragile]{C Code — Matrix Only (1/2)}
\begin{lstlisting}[language=C]
#include <stdio.h>

int main() {
    // Original point
    double P[2] = {4, 1};

    // Reflection about y = x
    double R[2];
    R[0] = P[1];
    R[1] = P[0];

    // Translation vector
    double T[2] = {2, 0};
    double F[2];
\end{lstlisting}
\end{frame}

\begin{frame}[fragile]{C Code — Matrix Only (2/2)}
\begin{lstlisting}[language=C]
    // Apply translation
    F[0] = R[0] + T[0];
    F[1] = R[1] + T[1];

    // Output results
    printf("Reflected point: (%.1f, %.1f)\n", R[0], R[1]);
    printf("Final point: (%.1f, %.1f)\n", F[0], F[1]);

    return 0;
}
\end{lstlisting}
\end{frame}

\begin{frame}[fragile]{Python Code — With .so or Executable}
\begin{lstlisting}[language=Python]
import subprocess

# Input for C program
input_str = "4 1"

result = subprocess.run(
    ['./reflect_translate'],  # compiled C binary
    input=input_str,
    capture_output=True,
    text=True
)

print(result.stdout.strip())
\end{lstlisting}
\end{frame}




























\begin{frame}{Plot}
\begin{figure}[H]
\centering
\includegraphics[width=1\linewidth]{Figs/Fig1.png}
\caption{Transformation of point \( \vec{P} \) through reflection and translation}
\end{figure}
\end{frame}

\end{document}









