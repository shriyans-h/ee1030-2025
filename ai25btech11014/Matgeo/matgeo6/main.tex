\let\negmedspace\undefined
\let\negthickspace\undefined
\documentclass[journal]{IEEEtran}
\usepackage[a5paper, margin=10mm, onecolumn]{geometry}
\usepackage{tfrupee}

\setlength{\headheight}{1cm}
\setlength{\headsep}{0mm}
\usepackage{gvv-book}
\usepackage{gvv}
\usepackage{cite}
\usepackage{amsmath,amssymb,amsfonts,amsthm}
\usepackage{graphicx}
\usepackage{mathtools}
\usepackage{tikz}
\usepackage{txfonts}
\usepackage{enumitem}
\usepackage{gensymb}
\usepackage{booktabs}

\begin{document}

\bibliographystyle{IEEEtran}
\vspace{3cm}

\title{3.2.30}
\author{AI25BTECH11014 - Gooty Suhas}
{\let\newpage\relax\maketitle}

\renewcommand{\thefigure}{\theenumi}
\renewcommand{\thetable}{\theenumi}
\setlength{\intextsep}{10pt}
\numberwithin{equation}{enumi}
\numberwithin{figure}{enumi}
\renewcommand{\thetable}{\theenumi}

\section*{\large\textbf{Problem}}
\vspace{0.5cm}

Construct a triangle \( \triangle ABC \) given:
\[
\angle B = 105^\circ, \quad \angle C = 90^\circ, \quad AB + BC + CA = 10 \text{ cm}
\]

\section*{\large\textbf{Matrix Formulation}}
\vspace{0.5cm}

Let the side lengths be:
\[
\vec{x} = \begin{bmatrix} a \\ b \\ c \end{bmatrix}
\]

Define the system:
\[
\begin{bmatrix}
1 & 1 & 1 \\
-1 & \cos C & \cos B \\
0 & \sin C & -\sin B
\end{bmatrix}
\vec{x}
=
\begin{bmatrix}
10 \\
0 \\
0
\end{bmatrix}
\]

Substitute:
\[
\cos C = 0, \quad \sin C = 1, \quad \cos B = \cos(105^\circ), \quad \sin B = \sin(105^\circ)
\]

Numerically:
\[
\cos(105^\circ) \approx -0.2588, \quad \sin(105^\circ) \approx 0.9659
\]

So the system becomes:
\[
\begin{bmatrix}
1 & 1 & 1 \\
-1 & 0 & -0.2588 \\
0 & 1 & -0.9659
\end{bmatrix}
\begin{bmatrix}
a \\
b \\
c
\end{bmatrix}
=
\begin{bmatrix}
10 \\
0 \\
0
\end{bmatrix}
\]

\section*{\large\textbf{Solution}}
\vspace{0.5cm}

Solving the matrix system:
\[
\begin{bmatrix}
a \\
b \\
c
\end{bmatrix}
=
\begin{bmatrix}
-1.52 \\
5.66 \\
5.86
\end{bmatrix}
\]

\section*{\large\textbf{Conclusion}}
\vspace{0.5cm}

Since \( a < 0 \), the triangle is not physically constructible.

\[
\text{Construction is not possible.}
\]

\end{document}















