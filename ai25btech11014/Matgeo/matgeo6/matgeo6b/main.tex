\documentclass{beamer}
\usepackage[utf8]{inputenc}

\usetheme{Madrid}
\usecolortheme{default}
\usepackage{amsmath,amssymb,amsfonts,amsthm}
\usepackage{txfonts}
\usepackage{tkz-euclide}
\usepackage{listings}
\usepackage[T1]{fontenc}
\usepackage{adjustbox}
\usepackage{array}
\usepackage{tabularx}
\usepackage{gvv}
\usepackage{lmodern}
\usepackage{circuitikz}
\usepackage{tikz}
\usepackage{graphicx}

\setbeamertemplate{page number in head/foot}[totalframenumber]

\title{3.2.30}
\author{AI25BTECH11014 - Gooty Suhas}

\begin{document}

\frame{\titlepage}

\begin{frame}{Question}
Construct a triangle \( \triangle ABC \) given:
\[
\angle B = 105^\circ, \quad \angle C = 90^\circ, \quad AB + BC + CA = 10 \text{ cm}
\]
\end{frame}

\begin{frame}{Matrix Formulation}
Let the side lengths be:
\[
\vec{x} = \begin{bmatrix} a \\ b \\ c \end{bmatrix}
\]
\[
\begin{bmatrix}
1 & 1 & 1 \\
-1 & \cos C & \cos B \\
0 & \sin C & -\sin B
\end{bmatrix}
\vec{x}
=
\begin{bmatrix}
10 \\
0 \\
0
\end{bmatrix}
\]
\end{frame}

\begin{frame}{Numerical Substitution}
Substitute:
\[
\cos C = 0, \quad \sin C = 1
\]
\[
\cos B \approx -0.2588, \quad \sin B \approx 0.9659
\]
\[
\begin{bmatrix}
1 & 1 & 1 \\
-1 & 0 & -0.2588 \\
0 & 1 & -0.9659
\end{bmatrix}
\begin{bmatrix}
a \\
b \\
c
\end{bmatrix}
=
\begin{bmatrix}
10 \\
0 \\
0
\end{bmatrix}
\]
\end{frame}

\begin{frame}{Matrix Solution}
Solving the system:
\[
\begin{bmatrix}
a \\
b \\
c
\end{bmatrix}
=
\begin{bmatrix}
-1.52 \\
5.66 \\
5.86
\end{bmatrix}
\]
\end{frame}

\begin{frame}{Conclusion}
Since side \( a \approx -1.52 \) is negative,  
the triangle is not physically constructible.
\[
\text{Construction is not possible.}
\]
\end{frame}

\begin{frame}[fragile]{Python Code (Part 1)}
\begin{lstlisting}[language=Python]
import numpy as np

A = np.array([
  [1, 1, 1],
  [-1, 0, -0.2588],
  [0, 1, -0.9659]
], dtype=np.float32)

b = np.array([10, 0, 0], dtype=np.float32)

x = np.linalg.solve(A, b)
print("Solution [a b c]:", x)
\end{lstlisting}
\end{frame}

\begin{frame}[fragile]{C Code for .so File}
\begin{lstlisting}
#include <stdio.h>

void solve_triangle(float* A,
                    float* b,
                    float* x) {

  float invA[9] = {
    -0.152, 0.332, 0.820,
     0.566, 0.566, -0.132,
     0.586, 0.102, -0.688
  };
\end{lstlisting}
\end{frame}

\begin{frame}[fragile]{C Code Continued}
\begin{lstlisting}
  for (int i = 0; i < 3; i++) {
    x[i] = 0;
    for (int j = 0; j < 3; j++) {
      x[i] += invA[3*i + j] * b[j];
    }
  }
}
\end{lstlisting}
\end{frame}

\begin{frame}[fragile]{Python Code Using .so File}
\begin{lstlisting}[language=Python]
import ctypes
import numpy as np

lib = ctypes.CDLL('./libtriangle.so')
lib.solve_triangle.argtypes = [
  ctypes.POINTER(ctypes.c_float),
  ctypes.POINTER(ctypes.c_float),
  ctypes.POINTER(ctypes.c_float)
]
\end{lstlisting}
\end{frame}

\begin{frame}[fragile]{Python Code Continued}
\begin{lstlisting}[language=Python]
A = np.array([
  [1, 1, 1],
  [-1, 0, -0.2588],
  [0, 1, -0.9659]
], dtype=np.float32)

b = np.array([10, 0, 0], dtype=np.float32)
x = np.zeros(3, dtype=np.float32)

lib.solve_triangle(
  A.ctypes.data_as(ctypes.POINTER(ctypes.c_float)),
  b.ctypes.data_as(ctypes.POINTER(ctypes.c_float)),
  x.ctypes.data_as(ctypes.POINTER(ctypes.c_float))
)

print("Solution [a b c]:", x)
\end{lstlisting}
\end{frame}

\end{document}








