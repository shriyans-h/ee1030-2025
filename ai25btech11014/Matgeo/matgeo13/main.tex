\let\negmedspace\undefined
\let\negthickspace\undefined
\documentclass[journal]{IEEEtran}
\usepackage[a5paper, margin=10mm, onecolumn]{geometry}
\usepackage{tfrupee}

\setlength{\headheight}{1cm}
\setlength{\headsep}{0mm}
\usepackage{gvv-book}
\usepackage{gvv}
\usepackage{cite}
\usepackage{amsmath,amssymb,amsfonts,amsthm}
\usepackage{algorithmic}
\usepackage{graphicx}
\usepackage{textcomp}
\usepackage{xcolor}
\usepackage{txfonts}
\usepackage{listings}
\usepackage{enumitem}
\usepackage{mathtools}
\usepackage{gensymb}
\usepackage{comment}
\usepackage[breaklinks=true]{hyperref}
\usepackage{tkz-euclide}
\def\inputGnumericTable{}
\usepackage[latin1]{inputenc}
\usepackage{color}
\usepackage{array}
\usepackage{longtable}
\usepackage{calc}
\usepackage{multirow}
\usepackage{hhline}
\usepackage{ifthen}
\usepackage{lscape}
\usepackage{booktabs}
\usepackage{tikz}
\usetikzlibrary{arrows.meta,angles,quotes}

\begin{document}

\bibliographystyle{IEEEtran}
\vspace{3cm}

\title{5.13.1}
\author{AI25BTECH11014 - Gooty Suhas}
{\let\newpage\relax\maketitle}

\renewcommand{\thefigure}{\theenumi}
\renewcommand{\thetable}{\theenumi}
\setlength{\intextsep}{10pt}
\numberwithin{equation}{enumi}
\numberwithin{figure}{enumi}
\renewcommand{\thetable}{\theenumi}

\section*{\large\textbf{Problem}}

If the system of linear equations
\begin{align}
x + 2a y + a z &= 0 \\
x + 3b y + b z &= 0 \\
x + 4c y + c z &= 0
\end{align}
has a non-zero solution, then \( a, b, c \)
\begin{enumerate}[label=\alph*)]
    \item satisfy \( a + 2b + 3c = 0 \)
    \item are in A.P.
    \item are in G.P.
    \item are in H.P.
\end{enumerate}

\section*{\large\textbf{Solution}}

We write the system in matrix form:
\[
\vec{M}_0 =
\begin{bmatrix}
1 & 2a & a \\
1 & 3b & b \\
1 & 4c & c
\end{bmatrix}
\]

Apply row operations. First, subtract Row 1 from Rows 2 and 3:
\[
\vec{M}_1 =
\begin{bmatrix}
1 & 2a & a \\
0 & 3b - 2a & b - a \\
0 & 4c - 2a & c - a
\end{bmatrix}
\]

Now subtract Row 2 from Row 3:
\[
\vec{M}_2 =
\begin{bmatrix}
1 & 2a & a \\
0 & 3b - 2a & b - a \\
0 & 4c - 3b & c - b
\end{bmatrix}
\]

For a non-zero solution to exist, the rows must be linearly dependent. So Row 3 must be a scalar multiple of Row 2:
\[
\frac{4c - 3b}{3b - 2a} = \frac{c - b}{b - a}
\]

Cross-multiplying:
\[
(4c - 3b)(b - a) = (c - b)(3b - 2a)
\]

Expanding both sides:
\begin{align}
4bc - 4ac - 3b^2 + 3ab &= 3bc - 2ac - 3b^2 + 2ab
\end{align}

Cancel \( -3b^2 \) and rearrange:
\[
4bc - 4ac + 3ab = 3bc - 2ac + 2ab
\]

Bring all terms to one side:
\[
(4bc - 3bc) + (3ab - 2ab) + (-4ac + 2ac) = 0
\Rightarrow bc + ab - 2ac = 0
\]

\section*{\large\textbf{Correct Condition}}

\[
\boxed{ab + bc = 2ac}
\]

\section*{\large\textbf{Verification of Option d}}

If \( a, b, c \) are in H.P., then \( \frac{1}{a}, \frac{1}{b}, \frac{1}{c} \) are in A.P.:
\[
\frac{2}{b} = \frac{1}{a} + \frac{1}{c}
\]

Multiply both sides by \( abc \):
\[
2ac = bc + ab
\Rightarrow ab + bc = 2ac
\]

Which matches our derived condition:
\[
\boxed{ab + bc = 2ac}
\]

Thus, option d) is correct.

\section*{\large\textbf{Final Answer}}
\[
\boxed{\text{Option d) } a, b, c \text{ are in H.P.}}
\]

\end{document}









