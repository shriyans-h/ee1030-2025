\documentclass{beamer}
\usepackage[utf8]{inputenc}
\usetheme{Madrid}
\usecolortheme{default}
\usepackage{amsmath,amssymb,amsfonts,amsthm}
\usepackage{txfonts}
\usepackage{tkz-euclide}
\usepackage{listings}
\usepackage[T1]{fontenc}
\usepackage{adjustbox}
\usepackage{array}
\usepackage{tabularx}
\usepackage{gvv}
\usepackage{lmodern}
\usepackage{circuitikz}
\usepackage{tikz}
\usepackage{graphicx}
\setbeamertemplate{page number in head/foot}[totalframenumber]

\title{5.13.1}
\author{AI25BTECH11014 - Gooty Suhas}
\begin{document}
\frame{\titlepage}





\begin{frame}{Problem}
If the system of equations
\[
\begin{aligned}
x + 2a y + a z &= 0 \\
x + 3b y + b z &= 0 \\
x + 4c y + c z &= 0
\end{aligned}
\]
has a non-zero solution, then what is the relation among \( a, b, c \)?

\vspace{0.5em}
\textbf{Options:}
\begin{itemize}
  \item[a)] \( a + 2b + 3c = 0 \)
  \item[b)] \( a, b, c \) are in arithmetic progression (A.P.)
  \item[c)] \( a, b, c \) are in geometric progression (G.P.)
  \item[d)] \( a, b, c \) are in harmonic progression (H.P.)
\end{itemize}
\end{frame}






\begin{frame}{Matrix Form}
We write the system as:
\[
\vec{M}_0 =
\begin{bmatrix}
1 & 2a & a \\
1 & 3b & b \\
1 & 4c & c
\end{bmatrix}
\]
Since the system has a non-zero solution, the rows of \( \vec{M}_0 \) must be linearly dependent.
\end{frame}

\begin{frame}{Row Operations}
Subtract Row 1 from Rows 2 and 3:
\[
\vec{M}_1 =
\begin{bmatrix}
1 & 2a & a \\
0 & 3b - 2a & b - a \\
0 & 4c - 2a & c - a
\end{bmatrix}
\]
Now subtract Row 2 from Row 3:
\[
\vec{M}_2 =
\begin{bmatrix}
1 & 2a & a \\
0 & 3b - 2a & b - a \\
0 & 4c - 3b & c - b
\end{bmatrix}
\]
\end{frame}

\begin{frame}{Linear Dependence Condition}
For linear dependence:
\[
\frac{4c - 3b}{3b - 2a} = \frac{c - b}{b - a}
\]
Cross-multiplying:
\[
(4c - 3b)(b - a) = (c - b)(3b - 2a)
\]
\end{frame}

\begin{frame}{Algebraic Expansion}
Expand both sides:
\[
\begin{aligned}
\text{LHS} &= 4bc - 4ac - 3b^2 + 3ab \\
\text{RHS} &= 3bc - 2ac - 3b^2 + 2ab
\end{aligned}
\]
Subtract RHS from LHS:
\[
(4bc - 3bc) + (3ab - 2ab) + (-4ac + 2ac) = 0
\Rightarrow \boxed{ab + bc = 2ac}
\]
\end{frame}

\begin{frame}{Verification of Option d}
If \( a, b, c \) are in H.P., then \( \frac{1}{a}, \frac{1}{b}, \frac{1}{c} \) are in A.P.:
\[
\frac{2}{b} = \frac{1}{a} + \frac{1}{c}
\]
Multiply both sides by \( abc \):
\[
2ac = bc + ab
\Rightarrow \boxed{ab + bc = 2ac}
\]
This matches our derived condition.
\end{frame}

\begin{frame}{Final Answer}
\[
\boxed{ab + bc = 2ac}
\quad \Rightarrow \quad
\boxed{\text{Option d) } a, b, c \text{ are in H.P.}}
\]
\end{frame}

\begin{frame}[fragile]{independent\_verify.py}
\begin{lstlisting}[language=Python]
from sympy import symbols, Eq, simplify

a, b, c = symbols('a b c')

# Harmonic progression condition: 2/b = 1/a + 1/c
hp = Eq(2/b, 1/a + 1/c)

# Multiply both sides by abc
lhs = simplify(2*a*c)
rhs = simplify(a*b + b*c)

print("ab + bc =", rhs)
print("2ac =", lhs)
print("Condition holds:", lhs == rhs)
\end{lstlisting}
\end{frame}


\begin{frame}[fragile]{verify\_condition.c}
\begin{lstlisting}[language=C]
#include <math.h>

void verify_condition(float a, float b, float c, float* result) {
  float lhs = a*b + b*c;
  float rhs = 2*a*c;

  if (fabs(lhs - rhs) < 1e-6)
    *result = 1.0;
  else
    *result = 0.0;
}
\end{lstlisting}
\end{frame}

\begin{frame}[fragile]{call\_verify.py (Part 1)}
\begin{lstlisting}[language=Python]
import ctypes

lib = ctypes.CDLL('./libverify.so')

lib.verify_condition.argtypes = [
    ctypes.c_float, ctypes.c_float,
    ctypes.c_float,
    ctypes.POINTER(ctypes.c_float)
]
lib.verify_condition.restype = None
\end{lstlisting}
\end{frame}


\begin{frame}[fragile]{call\_verify.py (Part 2)}
\begin{lstlisting}[language=Python]
a = ctypes.c_float(1.0)
b = ctypes.c_float(2.0)
c = ctypes.c_float(0.5)
result = ctypes.c_float()

lib.verify_condition(a, b, c,
                     ctypes.byref(result))

if result.value == 1.0:
    print("Verified: ab + bc = 2ac")
else:
    print("Condition fails")
\end{lstlisting}
\end{frame}










\end{document}






