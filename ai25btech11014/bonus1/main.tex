\let\negmedspace\undefined
\let\negthickspace\undefined
\documentclass[journal]{IEEEtran}
\usepackage[a5paper, margin=10mm, onecolumn]{geometry}
\usepackage{tfrupee}

\setlength{\headheight}{1cm}
\setlength{\headsep}{0mm}
\usepackage{gvv-book}
\usepackage{gvv}
\usepackage{cite}
\usepackage{amsmath,amssymb,amsfonts,amsthm}
\usepackage{algorithmic}
\usepackage{graphicx}
\usepackage{textcomp}
\usepackage{xcolor}
\usepackage{txfonts}
\usepackage{listings}
\usepackage{enumitem}
\usepackage{mathtools}
\usepackage{gensymb}
\usepackage{comment}
\usepackage[breaklinks=true]{hyperref}
\usepackage{tkz-euclide}
\def\inputGnumericTable{}
\usepackage[latin1]{inputenc}
\usepackage{color}
\usepackage{array}
\usepackage{longtable}
\usepackage{calc}
\usepackage{multirow}
\usepackage{hhline}
\usepackage{ifthen}
\usepackage{lscape}
\usepackage{booktabs}
\usepackage{tikz}
\usetikzlibrary{arrows.meta,angles,quotes}

\begin{document}

\bibliographystyle{IEEEtran}
\vspace{3cm}

\title{Bonus question}
\author{AI25BTECH11014 - Gooty Suhas}
{\let\newpage\relax\maketitle}

\renewcommand{\thefigure}{\theenumi}
\renewcommand{\thetable}{\theenumi}
\setlength{\intextsep}{10pt}
\numberwithin{equation}{enumi}
\numberwithin{figure}{enumi}
\renewcommand{\thetable}{\theenumi}

\section*{\large\textbf{Problem}}
\vspace{0.5cm}

Prove that if \( s_1s_2 > 0 \), the points lie on the same side of the line, and if \( s_1s_2 < 0 \), they lie on opposite sides.

\section*{\large\textbf{Solution}}
\vspace{0.5cm}

Let the points be:
\[
\vec{A} = \myvec{a_1 \\ a_2}, \quad
\vec{B} = \myvec{b_1 \\ b_2}
\]

Let the line be defined by:
\[
\vec{n}^T \vec{x} + c = 0
\]
where \( \vec{n} = \myvec{n_1 \\ n_2} \) is the normal vector.

\subsection*{Step 1: Signed Distance from Line}

The signed distances of points \( A \) and \( B \) from the line are:
\[
s_1 = \vec{n}^T \vec{A} + c, \quad
s_2 = \vec{n}^T \vec{B} + c
\]

\subsection*{Step 2: Point Dividing Line Segment}

Let point \( P \) divide \( AB \) in the ratio \( m:n \). Then:
\[
\vec{P} = \frac{n\vec{A} + m\vec{B}}{m+n}
\]

Substitute \( \vec{P} \) into the line equation:
\[
\vec{n}^T \left( \frac{n\vec{A} + m\vec{B}}{m+n} \right) + c = 0
\]

Multiply both sides by \( m+n \):
\[
n(\vec{n}^T \vec{A}) + m(\vec{n}^T \vec{B}) + c(m+n) = 0
\]

Group terms:
\[
n(s_1) + m(s_2) = 0
\Rightarrow \frac{m}{n} = -\frac{s_1}{s_2}
\]

\subsection*{Step 3: Ratio Interpretation}

Hence, the ratio in which the line divides the segment \( AB \) is:
\[
m:n = -s_1 : s_2
\]

\subsection*{Step 4: Side Condition Proof}

\begin{itemize}
    \item If \( s_1 \cdot s_2 > 0 \), then both signed distances have the same sign. So, points \( A \) and \( B \) lie on the \textbf{same side} of the line.
    \item If \( s_1 \cdot s_2 < 0 \), then the signs are opposite. So, points \( A \) and \( B \) lie on \textbf{different sides}.
\end{itemize}

\[
\boxed{
\text{If } s_1s_2 > 0 \Rightarrow \text{Same side}, \quad
\text{If } s_1s_2 < 0 \Rightarrow \text{Opposite sides}
}
\]

\end{document}