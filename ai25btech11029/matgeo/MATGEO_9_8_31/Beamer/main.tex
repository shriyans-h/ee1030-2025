\documentclass{beamer}
\usepackage[utf8]{inputenc}

\usetheme{Madrid}
\usecolortheme{default}
\usepackage{amsmath,amssymb,amsfonts,amsthm}
\usepackage{txfonts}
\usepackage{tkz-euclide}
\usepackage{listings}
\usepackage{adjustbox}
\usepackage{array}
\usepackage{tabularx}
\usepackage{gvv}
\usepackage{lmodern}
\usepackage{circuitikz}
\usepackage{tikz}
\usepackage{graphicx}
\usepackage{multicol}

\setbeamertemplate{page number in head/foot}[totalframenumber]

\usepackage{tcolorbox}
\tcbuselibrary{minted,breakable,xparse,skins}

\definecolor{bg}{gray}{0.95}
\DeclareTCBListing{mintedbox}{O{}m!O{}}{%
  breakable=true,
  listing engine=minted,
  listing only,
  minted language=#2,
  minted style=default,
  minted options={%
    linenos,
    gobble=0,
    breaklines=true,
    breakafter=,,
    fontsize=\small,
    numbersep=8pt,
    #1},
  boxsep=0pt,
  left skip=0pt,
  right skip=0pt,
  left=25pt,
  right=0pt,
  top=3pt,
  bottom=3pt,
  arc=5pt,
  leftrule=0pt,
  rightrule=0pt,
  bottomrule=2pt,
  toprule=2pt,
  colback=bg,
  colframe=orange!70,
  enhanced,
  overlay={%
    \begin{tcbclipinterior}
    \fill[orange!20!white] (frame.south west) rectangle ([xshift=20pt]frame.north west);
    \end{tcbclipinterior}},
  #3,
}
\lstset{
    language=C,
    basicstyle=\ttfamily\small,
    keywordstyle=\color{blue},
    stringstyle=\color{orange},
    commentstyle=\color{green!60!black},
    numbers=left,
    numberstyle=\tiny\color{gray},
    breaklines=true,
    showstringspaces=false,
}

\title 
{9.8.31}
\date{}

\author
{SAMYAK GONDANE - AI25BTECH11029}

\begin{document}

\frame{\titlepage}

\begin{frame}{Question}
Consider a circle with its centre lying on focus of the parabola $y^2 = 2px$ such that it touches the directrix of the parabola. Then a point of intersection of the circle and the parabola is

\begin{multicols}{2}
\begin{enumerate}
    \item $(\frac{p}{2}, p)$ or $(\frac{p}{2}, -p)$
    \item $(\frac{p}{2}, -\frac{p}{2})$
    \item $(-\frac{p}{2}, p)$
    \item $(-\frac{p}{2}, -\frac{p}{2})$
\end{enumerate}
\end{multicols}

\end{frame}

\begin{frame}{Solution}


\textbf{General Conic Form}

Any conic can be represented as:


\begin{align}
\vec{x}^T A \vec{x} + \vec{b}^T \vec{x} + c = 0
\end{align}
\end{frame}


\begin{frame}{Solution}
\textbf{Parabola: $y^2 = 2px$}

Rewriting:


\begin{align}
y^2 - 2px = 0
\end{align}


Matrix representation:

\begin{align}
\vec{A}_p = \myvec{0 & 0 \\ 0 & 1}, \quad
\vec{b}_p = \myvec{-2p \\ 0}, \quad
c_p = 0
\end{align}



So the parabola becomes:


\begin{align}
\vec{x}^T A_p \vec{x} + \vec{b}_p^T \vec{x} = 0
\end{align}
\end{frame}


\begin{frame}{Solution}
\textbf{Circle: Center at $(\frac{p}{2}, 0)$, Radius $p$}

Circle equation:


\begin{align}
(x - \frac{p}{2})^2 + y^2 = p^2
\Rightarrow x^2 - px + \frac{p^2}{4} + y^2 = p^2
\Rightarrow x^2 + y^2 - px - \frac{3p^2}{4} = 0
\end{align}



Matrix representation:


\begin{align}
\vec{A}_c = \myvec{1 & 0 \\ 0 & 1}, \quad
\vec{b}_c = \myvec{-p \\ 0}, \quad
c_c = -\frac{3p^2}{4}
\end{align}



So the circle becomes:


\begin{align}
\vec{x}^T A_c \vec{x} + \vec{b}_c^T \vec{x} + c_c = 0
\end{align}
\end{frame}


\begin{frame}{Solution}
\textbf{Solving the System}

From the parabola:


\begin{align}
y^2 = 2px
\end{align}



Substitute into the circle:


\begin{align}
x^2 + y^2 - px - \frac{3p^2}{4} = 0
\Rightarrow x^2 + 2px - px - \frac{3p^2}{4} = 0
\Rightarrow x^2 + px - \frac{3p^2}{4} = 0
\end{align}
\end{frame}

\begin{frame}{Solution}
Solve the quadratic:


\begin{align}
x = \frac{-p \pm \sqrt{p^2 + 4 \cdot \frac{3p^2}{4}}}{2}
= \frac{-p \pm \sqrt{4p^2}}{2}
= \frac{-p \pm 2p}{2}
\Rightarrow x = \frac{p}{2},\ -\frac{3p}{2}
\end{align}



Now find $y$ using $y^2 = 2px$:

For $x = \frac{p}{2}$:


\begin{align}
y^2 = p^2 \Rightarrow y = \pm p
\end{align}
\end{frame}


\begin{frame}{Solution}
\textbf{Final Answer}

Intersection points:


\begin{align}
(\frac{p}{2}, p), \quad (\frac{p}{2}, -p)
\end{align}



\textbf{Correct Option: (a)}
\end{frame}

\begin{frame}{Plot}
    \begin{figure}
        \centering
        \includegraphics[width=0.8\linewidth]{./figs/Figure_1.png}
        \caption{Caption}
        \label{fig:placeholder}
    \end{figure}
\end{frame}
\end{document}