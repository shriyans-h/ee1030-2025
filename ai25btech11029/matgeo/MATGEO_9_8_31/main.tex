\documentclass{article}
\usepackage{gvv-book}
\usepackage{gvv}
\usepackage{amsmath}
\usepackage{amsfonts}
\usepackage{tikz}
\usepackage{setspace}
\usepackage{gensymb}
\usepackage[cmex10]{amsmath}
\usepackage{amsthm}
\usepackage{mathrsfs}
\usepackage{txfonts}
\usepackage{stfloats}
\usepackage{bm}
\usepackage{cite}
\usepackage{cases}
\usepackage{subfig}
\usepackage{longtable}
\usepackage{multirow}
\usepackage{enumitem}
\usepackage{mathtools}
\usepackage{tikz}
\usepackage{circuitikz}
\usepackage{verbatim}
\usepackage[breaklinks=true]{hyperref}
\usepackage{tkz-euclide}
\usepackage{listings}
\usepackage{color}    
\usepackage{array}    
\usepackage{longtable}
\usepackage{calc}     
\usepackage{multirow} 
\usepackage{hhline}   
\usepackage{ifthen}   
\usepackage{lscape}     
\usepackage{chngcntr}
\usepackage{graphicx}
\usepackage{float}
\usepackage{multicol}
\usepackage[a4paper, left = 1.5cm, right = 1.5cm]{geometry}

\begin{document}

\begin{center}
\large
    \textbf{Samyak Gondane-AI25BTECH11029}
\end{center}
\date{}

\section*{Question}
Consider a circle with its centre lying on focus of the parabola $y^2 = 2px$ such that it touches the directrix of the parabola. Then a point of intersection of the circle and the parabola is

\begin{multicols}{2}
\begin{enumerate}
    \item $(\frac{p}{2}, p)$ or $(\frac{p}{2}, -p)$
    \item $(\frac{p}{2}, -\frac{p}{2})$
    \item $(-\frac{p}{2}, p)$
    \item $(-\frac{p}{2}, -\frac{p}{2})$
\end{enumerate}
\end{multicols}

\section*{Solution}

\subsection*{General Conic Form}

Any conic can be represented as:


\begin{align}
\vec{x}^T A \vec{x} + \vec{b}^T \vec{x} + c = 0
\end{align}



\subsection*{Parabola: $y^2 = 2px$}

Rewriting:


\begin{align}
y^2 - 2px = 0
\end{align}



Matrix representation:


\begin{align}
\vec{A}_p = \myvec{0 & 0 \\ 0 & 1}, \quad
\vec{b}_p = \myvec{-2p \\ 0}, \quad
c_p = 0
\end{align}



So the parabola becomes:


\begin{align}
\vec{x}^T A_p \vec{x} + \vec{b}_p^T \vec{x} = 0
\end{align}



\subsection*{Circle: Center at $(\frac{p}{2}, 0)$, Radius $p$}

Circle equation:


\begin{align}
(x - \frac{p}{2})^2 + y^2 = p^2
\Rightarrow x^2 - px + \frac{p^2}{4} + y^2 = p^2
\Rightarrow x^2 + y^2 - px - \frac{3p^2}{4} = 0
\end{align}



Matrix representation:


\begin{align}
\vec{A}_c = \myvec{1 & 0 \\ 0 & 1}, \quad
\vec{b}_c = \myvec{-p \\ 0}, \quad
c_c = -\frac{3p^2}{4}
\end{align}



So the circle becomes:


\begin{align}
\vec{x}^T A_c \vec{x} + \vec{b}_c^T \vec{x} + c_c = 0
\end{align}



\subsection*{Solving the System}

From the parabola:

\begin{align}
y^2 = 2px
\end{align}


Substitute into the circle:

\begin{align}
x^2 + y^2 - px - \frac{3p^2}{4} = 0
\Rightarrow x^2 + 2px - px - \frac{3p^2}{4} = 0
\Rightarrow x^2 + px - \frac{3p^2}{4} = 0
\end{align}



Solve the quadratic:


\begin{align}
x = \frac{-p \pm \sqrt{p^2 + 4 \cdot \frac{3p^2}{4}}}{2}
= \frac{-p \pm \sqrt{4p^2}}{2}
= \frac{-p \pm 2p}{2}
\Rightarrow x = \frac{p}{2},\ -\frac{3p}{2}
\end{align}



Now find $y$ using $y^2 = 2px$:

For $x = \frac{p}{2}$:


\begin{align}
y^2 = p^2 \Rightarrow y = \pm p
\end{align}



\subsection*{Final Answer}

Intersection points:


\begin{align}
(\frac{p}{2}, p), \quad (\frac{p}{2}, -p)
\end{align}


\textbf{Correct Option: (a)}

\begin{figure}[H]
    \centering
    \includegraphics[width=0.8\linewidth]{./figs/Figure_1.png}
    \caption{Caption}
    \label{fig:placeholder}
\end{figure}





\end{document}