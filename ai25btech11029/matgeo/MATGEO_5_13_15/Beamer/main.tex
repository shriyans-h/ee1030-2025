\documentclass{beamer}
\usepackage[utf8]{inputenc}

\usetheme{Madrid}
\usecolortheme{default}
\usepackage{amsmath,amssymb,amsfonts,amsthm}
\usepackage{txfonts}
\usepackage{tkz-euclide}
\usepackage{listings}
\usepackage{adjustbox}
\usepackage{array}
\usepackage{tabularx}
\usepackage{gvv}
\usepackage{lmodern}
\usepackage{circuitikz}
\usepackage{tikz}
\usepackage{graphicx}
\usepackage{multicol}

\setbeamertemplate{page number in head/foot}[totalframenumber]

\usepackage{tcolorbox}
\tcbuselibrary{minted,breakable,xparse,skins}

\definecolor{bg}{gray}{0.95}
\DeclareTCBListing{mintedbox}{O{}m!O{}}{%
  breakable=true,
  listing engine=minted,
  listing only,
  minted language=#2,
  minted style=default,
  minted options={%
    linenos,
    gobble=0,
    breaklines=true,
    breakafter=,,
    fontsize=\small,
    numbersep=8pt,
    #1},
  boxsep=0pt,
  left skip=0pt,
  right skip=0pt,
  left=25pt,
  right=0pt,
  top=3pt,
  bottom=3pt,
  arc=5pt,
  leftrule=0pt,
  rightrule=0pt,
  bottomrule=2pt,
  toprule=2pt,
  colback=bg,
  colframe=orange!70,
  enhanced,
  overlay={%
    \begin{tcbclipinterior}
    \fill[orange!20!white] (frame.south west) rectangle ([xshift=20pt]frame.north west);
    \end{tcbclipinterior}},
  #3,
}
\lstset{
    language=C,
    basicstyle=\ttfamily\small,
    keywordstyle=\color{blue},
    stringstyle=\color{orange},
    commentstyle=\color{green!60!black},
    numbers=left,
    numberstyle=\tiny\color{gray},
    breaklines=true,
    showstringspaces=false,
}

\title 
{5.13.15}
\date{}

\author
{SAMYAK GONDANE - AI25BTECH11029}

\begin{document}

\frame{\titlepage}

\begin{frame}{Question}
The area of a triangle formed by vertices $\vec{O}, \vec{A}$ and $\vec{B}$, where $\vec{OA} = \hat{i} + 2\hat{j} + 3\hat{k}$ and $\vec{OB} = -3\hat{i} - 2\hat{j} + \hat{k}$ is
\end{frame}


\begin{frame}{Solution}

\textbf{Given}:

\begin{align}
\textbf{A} = \myvec{ 5a & -b \\ 3 & 2 }
\end{align}


\begin{align}
Adj(\textbf{A}) = \myvec{ 2 & b \\ -3 & 5a }
\end{align}
\end{frame}


\begin{frame}{Solution}
\textbf{Compute $AA^T$}

First, compute the transpose:

\begin{align}
A^\top = \myvec{ 5a & 3 \\ -b & 2 }
\end{align}

Now multiply:

\begin{align}
AA^\top = \myvec{ 5a & -b \\ 3 & 2 } \myvec{ 5a & 3 \\ -b & 2 }
\end{align}
\end{frame}


\begin{frame}{Solution}
Compute each entry:

\begin{align}
AA^\top = \myvec{
(5a)^2 + (-b)^2 & 5a \cdot 3 + (-b) \cdot 2 \\
3 \cdot 5a + 2 \cdot (-b) & 3^2 + 2^2
}
= \myvec{
25a^2 + b^2 & 15a - 2b \\
15a - 2b & 13
}
\end{align}

\subsection*{Equate $AA^T = \textbf{A}Adj\textbf{A}$}


\begin{align}
\myvec{
25a^2 + b^2 & 15a - 2b \\
15a - 2b & 13
}
=
\myvec{
2 & b \\
-3 & 5a
}
\end{align}\\


\end{frame}


\begin{frame}{Solution}
Compare bottom-right entries:
\begin{align}
13 = 5a \Rightarrow a = \frac{13}{5}
\end{align}

Compare top-right entries:
\begin{align}
15a - 2b = b \Rightarrow 15a = 3b \Rightarrow b = 13
\end{align}
\end{frame}

\begin{frame}{Solution}
\textbf{Final Step: Compute $5a + b$}

\begin{align}
5a + b = 5 \times \frac{13}{5} + 13 = 26 = \boxed{26}
\end{align}

\end{frame}

\end{document}
