\documentclass{article}
\usepackage{gvv-book}
\usepackage{gvv}
\usepackage{amsmath}
\usepackage{amsfonts}
\usepackage{tikz}
\usepackage{setspace}
\usepackage{gensymb}
\usepackage[cmex10]{amsmath}
\usepackage{amsthm}
\usepackage{mathrsfs}
\usepackage{txfonts}
\usepackage{stfloats}
\usepackage{bm}
\usepackage{cite}
\usepackage{cases}
\usepackage{subfig}
\usepackage{longtable}
\usepackage{multirow}
\usepackage{enumitem}
\usepackage{mathtools}
\usepackage{tikz}
\usepackage{circuitikz}
\usepackage{verbatim}
\usepackage[breaklinks=true]{hyperref}
\usepackage{tkz-euclide}
\usepackage{listings}
\usepackage{color}    
\usepackage{array}    
\usepackage{longtable}
\usepackage{calc}     
\usepackage{multirow} 
\usepackage{hhline}   
\usepackage{ifthen}   
\usepackage{lscape}     
\usepackage{chngcntr}
\usepackage{graphicx}
\usepackage{float}
\usepackage{multicol}
\usepackage[a4paper, left = 1.5cm, right = 1.5cm]{geometry}

\begin{document}

\begin{center}
\large
    \textbf{Samyak Gondane-AI25BTECH11029}
\end{center}
\date{}

\section*{Question}
If $\textbf{A} = \myvec{5a & -b \\ 3 & 2}$ and \textbf{A}adj(\textbf{A}) = \textbf{A}$\textbf{A}^T$, then $5a + b$ is equal to


\section*{Solution}

Given:

\begin{align}
A = \myvec{ 5a & -b \\ 3 & 2 }
\end{align}


\begin{align}
\text{Adj}(A) = \myvec{ 2 & b \\ -3 & 5a }
\end{align}



\subsection*{Compute $AA^\top$}

First, compute the transpose:


\begin{align}
A^\top = \myvec{ 5a & 3 \\ -b & 2 }
\end{align}



Now multiply:


\begin{align}
AA^\top = \myvec{ 5a & -b \\ 3 & 2 } \myvec{ 5a & 3 \\ -b & 2 }
\end{align}



Compute each entry:


\begin{align}
AA^\top = \myvec{
(5a)^2 + (-b)^2 & 5a \cdot 3 + (-b) \cdot 2 \\
3 \cdot 5a + 2 \cdot (-b) & 3^2 + 2^2
}
= \myvec{
25a^2 + b^2 & 15a - 2b \\
15a - 2b & 13
}
\end{align}



\subsection*{Equate $AA^T = \textbf{A}Adj\textbf{A}$}

\begin{align}
\myvec{
25a^2 + b^2 & 15a - 2b \\
15a - 2b & 13
}
=
\myvec{
2 & b \\
-3 & 5a
}
\end{align}\\

Compare bottom-right entries:
\begin{align}
13 = 5a \Rightarrow a = \frac{13}{5}
\end{align}

Compare top-right entries:
\begin{align}
15a - 2b = b \Rightarrow 15a = 3b \Rightarrow b = 13
\end{align}


\subsection*{Final Step: Compute $5a + b$}

\begin{align}
5a + b = 5 \times \frac{13}{5} + 13 = 26 = \boxed{26}
\end{align}


\end{document}
