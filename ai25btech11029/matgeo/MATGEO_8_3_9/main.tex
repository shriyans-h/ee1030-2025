\documentclass{article}
\usepackage{gvv-book}
\usepackage{gvv}
\usepackage{amsmath}
\usepackage{amsfonts}
\usepackage{tikz}
\usepackage{setspace}
\usepackage{gensymb}
\usepackage[cmex10]{amsmath}
\usepackage{amsthm}
\usepackage{mathrsfs}
\usepackage{txfonts}
\usepackage{stfloats}
\usepackage{bm}
\usepackage{cite}
\usepackage{cases}
\usepackage{subfig}
\usepackage{longtable}
\usepackage{multirow}
\usepackage{enumitem}
\usepackage{mathtools}
\usepackage{tikz}
\usepackage{circuitikz}
\usepackage{verbatim}
\usepackage[breaklinks=true]{hyperref}
\usepackage{tkz-euclide}
\usepackage{listings}
\usepackage{color}    
\usepackage{array}    
\usepackage{longtable}
\usepackage{calc}     
\usepackage{multirow} 
\usepackage{hhline}   
\usepackage{ifthen}   
\usepackage{lscape}     
\usepackage{chngcntr}
\usepackage{graphicx}
\usepackage{float}
\usepackage{multicol}
\usepackage[a4paper, left = 1.5cm, right = 1.5cm]{geometry}

\begin{document}

\begin{center}
\large
    \textbf{Samyak Gondane-AI25BTECH11029}
\end{center}
\date{}

\section*{Question}
If the latus rectum of an ellipse is equal to half of minor axis, then find its eccentric

\section*{Solution}


\subsection*{Matrix Representation of an Ellipse}

The general quadratic form of a centered ellipse is:


\begin{align}
\mathbf{x}^T A \mathbf{x} = 1
\quad \text{where } A = 
\begin{bmatrix}
\frac{1}{a^2} & 0 \\
0 & \frac{1}{b^2}
\end{bmatrix}
\end{align}



Here, $a$ and $b$ are the semi-major and semi-minor axes respectively.

\subsection*{Geometric Condition}

The latus rectum $L$ of an ellipse is given by:


\begin{align}
L = \frac{2b^2}{a}
\end{align}



Given:
\begin{align}
L = \frac{1}{2} \cdot 2b = b
\quad \Rightarrow \quad \frac{2b^2}{a} = b
\quad \Rightarrow \quad 2b = a
\end{align}

Thus, we have:

\begin{align}
a = 2b
\end{align}

\subsection*{Eccentricity Calculation}

Eccentricity $e$ of an ellipse is:


\begin{align}
e = \sqrt{1 - \frac{b^2}{a^2}}
\end{align}

Substituting $a = 2b$:

\begin{align}
e = \sqrt{1 - \frac{b^2}{(2b)^2}} = \sqrt{1 - \frac{1}{4}} = \sqrt{\frac{3}{4}} = \frac{\sqrt{3}}{2}
\end{align}



\subsection*{Final Answer}



\begin{align}
\boxed{e = \frac{\sqrt{3}}{2}}
\end{align}


\begin{figure}[H]
    \centering
    \includegraphics[width=0.7\linewidth]{./figs/Figure_1.png}
    \caption{}
    \label{fig:fig1}
\end{figure}

\end{document}