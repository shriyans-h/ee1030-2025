\documentclass{article}
\usepackage{gvv-book}
\usepackage{gvv}
\usepackage{amsmath}
\usepackage{amsfonts}
\usepackage{tikz}
\usepackage{setspace}
\usepackage{gensymb}
\usepackage[cmex10]{amsmath}
\usepackage{amsthm}
\usepackage{mathrsfs}
\usepackage{txfonts}
\usepackage{stfloats}
\usepackage{bm}
\usepackage{cite}
\usepackage{cases}
\usepackage{subfig}
\usepackage{longtable}
\usepackage{multirow}
\usepackage{enumitem}
\usepackage{mathtools}
\usepackage{tikz}
\usepackage{circuitikz}
\usepackage{verbatim}
\usepackage[breaklinks=true]{hyperref}
\usepackage{tkz-euclide}
\usepackage{listings}
\usepackage{color}    
\usepackage{array}    
\usepackage{longtable}
\usepackage{calc}     
\usepackage{multirow} 
\usepackage{hhline}   
\usepackage{ifthen}   
\usepackage{lscape}     
\usepackage{chngcntr}
\usepackage{graphicx}
\usepackage{float}
\usepackage{multicol}
\usepackage[a4paper, left = 1.5cm, right = 1.5cm]{geometry}

\begin{document}

\begin{center}
\large
    \textbf{Samyak Gondane-AI25BTECH11029}
\end{center}
\date{}

\section*{Question}
Find the equation of the circle passing through $(0, 0)$ and making intercepts $a$ and $b$ on the coordinate axes.


\section*{Solution}

We use the general matrix form of a conic:


\begin{align}
\vec{x}^T \vec{V} \vec{x} + 2\vec{u}^T \vec{x} + f = 0
\end{align}


where:


\begin{align}
\vec{x} = \myvec{x \\ y}, \quad
\vec{V} = \myvec{1 & 0 \\ 0 & 1}, \quad
\vec{u} = \myvec{u_1 \\ u_2}, \quad
f \in \mathbb{R}
\end{align}



Substitute the three points:

\begin{itemize}
  \item At $(0, 0)$: $f = 0$
  \item At $(a, 0)$: $a^2 + 2u_1 a = 0 \Rightarrow u_1 = -\frac{a}{2}$
  \item At $(0, b)$: $b^2 + 2u_2 b = 0 \Rightarrow u_2 = -\frac{b}{2}$
\end{itemize}

So,


\begin{align}
\vec{u} = \myvec{-\frac{a}{2} \\ -\frac{b}{2}}, \quad f = 0
\end{align}



Substitute back into the general form:


\begin{align}
\vec{x}^T \myvec{1 & 0 \\ 0 & 1} \vec{x}
+ 2 \myvec{-\frac{a}{2} & -\frac{b}{2}} \vec{x}
= 0
\end{align}



Simplify:


\begin{align}
x^2 + y^2 - ax - by = 0
\end{align}

\begin{figure}[H]
    \centering
    \includegraphics[width=0.7\linewidth]{./figs/Figure_1.png}
    \caption{}
    \label{fig:fig1}
\end{figure}

\end{document}
