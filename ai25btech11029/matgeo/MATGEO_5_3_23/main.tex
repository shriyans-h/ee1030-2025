\documentclass{article}
\usepackage{gvv-book}
\usepackage{gvv}
\usepackage{amsmath}
\usepackage{amsfonts}
\usepackage{tikz}
\usepackage{setspace}
\usepackage{gensymb}
\usepackage[cmex10]{amsmath}
\usepackage{amsthm}
\usepackage{mathrsfs}
\usepackage{txfonts}
\usepackage{stfloats}
\usepackage{bm}
\usepackage{cite}
\usepackage{cases}
\usepackage{subfig}
\usepackage{longtable}
\usepackage{multirow}
\usepackage{enumitem}
\usepackage{mathtools}
\usepackage{tikz}
\usepackage{circuitikz}
\usepackage{verbatim}
\usepackage[breaklinks=true]{hyperref}
\usepackage{tkz-euclide}
\usepackage{listings}
\usepackage{color}    
\usepackage{array}    
\usepackage{longtable}
\usepackage{calc}     
\usepackage{multirow} 
\usepackage{hhline}   
\usepackage{ifthen}   
\usepackage{lscape}     
\usepackage{chngcntr}
\usepackage{graphicx}
\usepackage{float}
\usepackage{multicol}
\usepackage[a4paper, left = 1.5cm, right = 1.5cm]{geometry}

\begin{document}

\begin{center}
\large
    \textbf{Samyak Gondane-AI25BTECH11029}
\end{center}
\date{}

\section*{Question}
For what value of $k$, will the following pain of equations have infinitly many solutions\\

$2x + 3y = 7$ and $(k + 2)x - 3(1 - k)y = 5k + 1$

\section*{Solution}

\begin{align}
2x + 3y &= 7 \quad \text{(1)} \\
(k + 2)x - 3(1 - k)y &= 5k + 1 \quad \text{(2)}
\end{align}



\subsubsection*{Matrix Representation}
Write the system as an augmented matrix:
\begin{align}
\myvec{
2 & 3 & \vert &7 \\
k+2 & -3(1-k) & \vert &5k + 1
} 
= 
\myvec{
2 & 3 & \vert &7 \\
k+2 & -3 + 3k & \vert & 5k + 1
}
\end{align}


\subsection*{Condition for Infinitely Many Solutions}
For infinitely many solutions, the second row must be a scalar multiple of the first row. Let the scalar be $\lambda$. Then:

\begin{align}
k + 2 &= 2\lambda \quad \text{(i)} \\
-3 + 3k &= 3\lambda \quad \text{(ii)} \\
5k + 1 &= 7\lambda \quad \text{(iii)}
\end{align}

From (i), solve for $\lambda$:

\begin{align}
\lambda = \frac{k + 2}{2}
\end{align}

Substitute into (ii):

\begin{align}
-3 + 3k = 3 (\frac{k + 2}{2})\\
-3 + 3k = \frac{3k + 6}{2}\\
-6 + 6k = 3k + 6\\
3k = 12\\
k = 4
\end{align}


\subsection*{Verification}
Check with equation (iii):

\begin{align}
\lambda = \frac{4 + 2}{2} = 3 \\
7\lambda = 21 \\
5k + 1 = 5(4) + 1 = 21
\end{align}


\subsection*{Final Answer}

\begin{align}
\boxed{k = 4}
\end{align}

\end{document}