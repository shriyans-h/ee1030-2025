\documentclass{beamer}
\usepackage[utf8]{inputenc}

\usetheme{Madrid}
\usecolortheme{default}
\usepackage{amsmath,amssymb,amsfonts,amsthm}
\usepackage{txfonts}
\usepackage{tkz-euclide}
\usepackage{listings}
\usepackage{adjustbox}
\usepackage{array}
\usepackage{tabularx}
\usepackage{gvv}
\usepackage{lmodern}
\usepackage{circuitikz}
\usepackage{tikz}
\usepackage{graphicx}
\usepackage{multicol}

\setbeamertemplate{page number in head/foot}[totalframenumber]

\usepackage{tcolorbox}
\tcbuselibrary{minted,breakable,xparse,skins}

\definecolor{bg}{gray}{0.95}
\DeclareTCBListing{mintedbox}{O{}m!O{}}{%
  breakable=true,
  listing engine=minted,
  listing only,
  minted language=#2,
  minted style=default,
  minted options={%
    linenos,
    gobble=0,
    breaklines=true,
    breakafter=,,
    fontsize=\small,
    numbersep=8pt,
    #1},
  boxsep=0pt,
  left skip=0pt,
  right skip=0pt,
  left=25pt,
  right=0pt,
  top=3pt,
  bottom=3pt,
  arc=5pt,
  leftrule=0pt,
  rightrule=0pt,
  bottomrule=2pt,
  toprule=2pt,
  colback=bg,
  colframe=orange!70,
  enhanced,
  overlay={%
    \begin{tcbclipinterior}
    \fill[orange!20!white] (frame.south west) rectangle ([xshift=20pt]frame.north west);
    \end{tcbclipinterior}},
  #3,
}
\lstset{
    language=C,
    basicstyle=\ttfamily\small,
    keywordstyle=\color{blue},
    stringstyle=\color{orange},
    commentstyle=\color{green!60!black},
    numbers=left,
    numberstyle=\tiny\color{gray},
    breaklines=true,
    showstringspaces=false,
}

\title 
{5.3.23}
\date{}

\author
{SAMYAK GONDANE - AI25BTECH11029}

\begin{document}

\frame{\titlepage}

\begin{frame}{Question}
For what value of $k$, will the following pair of equations have infinitly many solutions\\

$2x + 3y = 7$ and $(k + 2)x - 3(1 - k)y = 5k + 1$
\end{frame}

\begin{frame}{Solution}

\begin{align}
2x + 3y &= 7 \quad \text{(1)} \\
(k + 2)x - 3(1 - k)y &= 5k + 1 \quad \text{(2)}
\end{align}

\textbf{Matrix Representation}\\
Write the system as an augmented matrix:
\begin{align}
\myvec{
2 & 3 & \vert &7 \\
k+2 & -3(1-k) & \vert &5k + 1
} 
= 
\myvec{
2 & 3 & \vert &7 \\
k+2 & -3 + 3k & \vert & 5k + 1
}
\end{align}
\end{frame}

\begin{frame}{Solution}
\textbf{Condition for Infinitely Many Solutions}\\
For infinitely many solutions, the second row must be a scalar multiple of the first row. Let the scalar be $\lambda$. Then:

\begin{align}
k + 2 &= 2\lambda \quad \text{(i)} \\
-3 + 3k &= 3\lambda \quad \text{(ii)} \\
5k + 1 &= 7\lambda \quad \text{(iii)}
\end{align}
\end{frame}

\begin{frame}{Solution}
From (i), solve for $\lambda$:

\begin{align}
\lambda = \frac{k + 2}{2}
\end{align}

Substitute into (ii):

\begin{align}
-3 + 3k = 3 (\frac{k + 2}{2})\\
-3 + 3k = \frac{3k + 6}{2}\\
-6 + 6k = 3k + 6\\
3k = 12\\
k = 4
\end{align}
\end{frame}

\begin{frame}{Solution}
\textbf{Verification}\\
Check with equation (iii):

\begin{align}
\lambda = \frac{4 + 2}{2} = 3 \\
7\lambda = 21 \\
5k + 1 = 5(4) + 1 = 21
\end{align}

\textbf{Final Answer}

\begin{align}
\boxed{k = 4}
\end{align}
\end{frame}

\end{document}