\documentclass{beamer}
\usepackage[utf8]{inputenc}

\usetheme{Madrid}
\usecolortheme{default}
\usepackage{amsmath,amssymb,amsfonts,amsthm}
\usepackage{txfonts}
\usepackage{tkz-euclide}
\usepackage{listings}
\usepackage{adjustbox}
\usepackage{array}
\usepackage{tabularx}
\usepackage{gvv}
\usepackage{lmodern}
\usepackage{circuitikz}
\usepackage{tikz}
\usepackage{graphicx}
\usepackage{multicol}

\setbeamertemplate{page number in head/foot}[totalframenumber]

\usepackage{tcolorbox}
\tcbuselibrary{minted,breakable,xparse,skins}

\definecolor{bg}{gray}{0.95}
\DeclareTCBListing{mintedbox}{O{}m!O{}}{%
  breakable=true,
  listing engine=minted,
  listing only,
  minted language=#2,
  minted style=default,
  minted options={%
    linenos,
    gobble=0,
    breaklines=true,
    breakafter=,,
    fontsize=\small,
    numbersep=8pt,
    #1},
  boxsep=0pt,
  left skip=0pt,
  right skip=0pt,
  left=25pt,
  right=0pt,
  top=3pt,
  bottom=3pt,
  arc=5pt,
  leftrule=0pt,
  rightrule=0pt,
  bottomrule=2pt,
  toprule=2pt,
  colback=bg,
  colframe=orange!70,
  enhanced,
  overlay={%
    \begin{tcbclipinterior}
    \fill[orange!20!white] (frame.south west) rectangle ([xshift=20pt]frame.north west);
    \end{tcbclipinterior}},
  #3,
}
\lstset{
    language=C,
    basicstyle=\ttfamily\small,
    keywordstyle=\color{blue},
    stringstyle=\color{orange},
    commentstyle=\color{green!60!black},
    numbers=left,
    numberstyle=\tiny\color{gray},
    breaklines=true,
    showstringspaces=false,
}

\title 
{2.7.1}
\date{}

\author
{SAMYAK GONDANE - AI25BTECH11029}

\begin{document}

\frame{\titlepage}

\begin{frame}{Question}
The area of a triangle formed by vertices $\vec{O}, \vec{A}$ and $\vec{B}$, where $\vec{OA} = \hat{i} + 2\hat{j} + 3\hat{k}$ and $\vec{OB} = -3\hat{i} - 2\hat{j} + \hat{k}$ is
\end{frame}

\begin{frame}{Solution}
Let \vec{O} be the origin

Represent the vectors in matrix form:
\begin{align}
\vec{A} = \myvec{1 \\ 2 \\ 3}, \quad
\vec{B} = \myvec{-3 \\ 1 \\ 1}
\end{align}

The area of triangle $OAB$ is given by:
\begin{align}
\text{Area} = \frac{1}{2} \norm{\vec{A} \times \vec{B}}
\end{align}

Compute the cross product using determinant:
\begin{align}
\vec{A} \times \vec{B} =
\myvec{
(2)(1) - (3)(1) \\
(3)(-3) - (1)(1) \\
(1)(1) - (2)(-3)
}
=
\myvec{
2 - 3 \\
-9 - 1 \\
1 + 6
}
=
\myvec{
-1 \\
-10 \\
7
}
\end{align}

\end{frame}

\begin{frame}{Solution}

Simplifying:
\begin{align}
\Rightarrow \vec{A} \times \vec{B} = \myvec{-1 \\ -10 \\ 7}
\end{align}

Magnitude of the cross product:
\begin{align}
\norm{\vec{A} \times \vec{B}} = \sqrt{(-1)^2 + (-10)^2 + 7^2} = \sqrt{1 + 100 + 49} = \sqrt{150}
\end{align}

Final area:
\begin{align}
\text{Area} = \frac{1}{2} \sqrt{150}
\end{align}

\end{frame}

\begin{frame}{Plot}
    \begin{figure}
        \centering
        \includegraphics[width=0.79\linewidth]{figs/Figure_1.png}
        \caption{}
        \label{fig:fig1}
    \end{figure}
\end{frame}

\end{document}
