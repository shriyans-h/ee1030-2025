 \let\negmedspace\undefined
\let\negthickspace\undefined
\documentclass[journal]{IEEEtran}
\usepackage[a5paper, margin=10mm, onecolumn]{geometry}
%\usepackage{lmodern} % Ensure lmodern is loaded for pdflatex
\usepackage{tfrupee} % Include tfrupee package

\setlength{\headheight}{1cm} % Set the height of the header box
\setlength{\headsep}{0mm}     % Set the distance between the header box and the top of the text
\usepackage{gvv-book}
\usepackage{gvv}
\usepackage{cite}
\usepackage{amsmath,amssymb,amsfonts,amsthm}
\usepackage{algorithmic}
\usepackage{graphicx}
\usepackage{textcomp}
\usepackage{xcolor}
\usepackage{txfonts}
\usepackage{listings}
\usepackage{enumitem}
\usepackage{mathtools}
\usepackage{gensymb}
\usepackage{comment}
\usepackage[breaklinks=true]{hyperref}
\usepackage{tkz-euclide} 
\usepackage{listings}
% \usepackage{gvv}                                        
\def\inputGnumericTable{}                                 
\usepackage[latin1]{inputenc}                                
\usepackage{color}                                            
\usepackage{array}                                            
\usepackage{longtable}                                       
\usepackage{calc}                                             
\usepackage{multirow}                                         
\usepackage{hhline}                                           
\usepackage{ifthen}                                           
\usepackage{lscape}



\usepackage{amsmath,amssymb}
\usepackage{booktabs}
\usepackage{tikz}
\usetikzlibrary{arrows.meta,angles,quotes}





\begin{document}

\bibliographystyle{IEEEtran}
\vspace{3cm}

\title{5.6.7}
\author{AI25BTECH11021 - Abhiram Reddy N}
% \maketitle
% \newpage
% \bigskip
{\let\newpage\relax\maketitle}

\renewcommand{\thefigure}{\theenumi}
\renewcommand{\thetable}{\theenumi}
\setlength{\intextsep}{10pt} % Space between text and floats


\numberwithin{equation}{enumi}
\numberwithin{figure}{enumi}
\renewcommand{\thetable}{\theenumi}


\section*{Question}
Let
\[
A=\begin{pmatrix}
2 & -1 & 1\\[4pt]
-1 & 2 & -1\\[4pt]
1 & -1 & 2
\end{pmatrix}.
\]
Verify that
\begin{equation}\label{eq:CH}
A^{3}-6A^{2}+9A-4I=0,
\end{equation}
by using the Cayley--Hamilton theorem, and hence find \(A^{-1}\). When computing \(A^{-1}\) show the relation to the adjugate (transpose of cofactors).

\subsection*{Step 1: Characteristic polynomial}
Compute the characteristic polynomial of \(A\):
\begin{align}
\chi_A(\lambda)&=\det(\lambda I - A)\nonumber\\
&=\det\begin{pmatrix}
\lambda-2 & 1 & -1\\[4pt]
1 & \lambda-2 & 1\\[4pt]
-1 & 1 & \lambda-2
\end{pmatrix}.\label{eq:char-det}
\end{align}
Expanding (or by direct computation) one obtains the factorisation
\begin{equation}\label{eq:char-factor}
\chi_A(\lambda)=(\lambda-4)(\lambda-1)^{2}
=\lambda^{3}-6\lambda^{2}+9\lambda-4.
\end{equation}

\subsection*{Step 2: Cayley--Hamilton theorem}
By the Cayley--Hamilton theorem the matrix \(A\) satisfies its own characteristic polynomial, i.e.
\begin{equation}\label{eq:CH-matrix}
A^{3}-6A^{2}+9A-4I=0,
\end{equation}
which is the desired identity \eqref{eq:CH}.

\subsection*{Step 3: Express \(A^{-1}\) from the polynomial}
Assuming \(A\) is invertible (we will check the determinant shortly), multiply \eqref{eq:CH-matrix} on the right by \(A^{-1}\) to obtain
\begin{equation}\label{eq:pre-inv}
A^{2}-6A+9I-4A^{-1}=0.
\end{equation}
Rearrange \eqref{eq:pre-inv} to solve for \(A^{-1}\):
\begin{equation}\label{eq:inv-formula}
A^{-1}=\tfrac{1}{4}\bigl(A^{2}-6A+9I\bigr).
\end{equation}



\subsection*{Step 4: Determinant and adjugate (transpose of cofactors)}
From the characteristic polynomial \eqref{eq:char-factor} we read off the eigenvalues \(4,1,1\). Thus
\begin{equation}\label{eq:det}
\det(A)=4,
\end{equation}
so \(A\) is invertible. The adjugate matrix \(\operatorname{adj}(A)\) satisfies
\[
\operatorname{adj}(A)=\det(A)\,A^{-1}=4A^{-1}.
\]
From \eqref{eq:inv-explicit} we get
\begin{equation}\label{eq:adj}
\operatorname{adj}(A)=
\begin{pmatrix}
3 & 1 & -1\\[4pt]
1 & 3 & 1\\[4pt]
-1 & 1 & 3
\end{pmatrix}.
\end{equation}
The adjugate is the transpose of the cofactor matrix; in this case \(\operatorname{adj}(A)\) is symmetric, so taking the transpose of the cofactor matrix yields the same matrix \eqref{eq:adj}. Finally,
\begin{equation}\label{eq:inv-final}
A^{-1}=\frac{1}{\det(A)}\operatorname{adj}(A)=\frac{1}{4}
\begin{pmatrix}
3 & 1 & -1\\[4pt]
1 & 3 & 1\\[4pt]
-1 & 1 & 3
\end{pmatrix},
\end{equation}
which matches \eqref{eq:inv-explicit}.


\bigskip
\noindent\textbf{Answer.} The Cayley--Hamilton identity \eqref{eq:CH} holds, and
\[
\boxed{\,A^{-1}=\frac{1}{4}\bigl(A^{2}-6A+9I\bigr)
=\begin{pmatrix}
\frac{3}{4} & \frac{1}{4} & -\frac{1}{4}\\[6pt]
\frac{1}{4} & \frac{3}{4} & \frac{1}{4}\\[6pt]
-\frac{1}{4} & \frac{1}{4} & \frac{3}{4}
\end{pmatrix}\,}
\]
which may equivalently be obtained from \(A^{-1}=\dfrac{1}{\det(A)}\operatorname{adj}(A)\) with \(\det(A)=4\) and \(\operatorname{adj}(A)\) given in \eqref{eq:adj} (the adjugate is the transpose of the cofactor matrix).



\end{document}