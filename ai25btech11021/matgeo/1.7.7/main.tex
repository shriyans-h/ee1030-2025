 \let\negmedspace\undefined
\let\negthickspace\undefined
\documentclass[journal]{IEEEtran}
\usepackage[a5paper, margin=10mm, onecolumn]{geometry}
%\usepackage{lmodern} % Ensure lmodern is loaded for pdflatex
\usepackage{tfrupee} % Include tfrupee package

\setlength{\headheight}{1cm} % Set the height of the header box
\setlength{\headsep}{0mm}     % Set the distance between the header box and the top of the text
\usepackage{gvv-book}
\usepackage{gvv}
\usepackage{cite}
\usepackage{amsmath,amssymb,amsfonts,amsthm}
\usepackage{algorithmic}
\usepackage{graphicx}
\usepackage{textcomp}
\usepackage{xcolor}
\usepackage{txfonts}
\usepackage{listings}
\usepackage{enumitem}
\usepackage{mathtools}
\usepackage{gensymb}
\usepackage{comment}
\usepackage[breaklinks=true]{hyperref}
\usepackage{tkz-euclide} 
\usepackage{listings}
% \usepackage{gvv}                                        
\def\inputGnumericTable{}                                 
\usepackage[latin1]{inputenc}                                
\usepackage{color}                                            
\usepackage{array}                                            
\usepackage{longtable}                                       
\usepackage{calc}                                             
\usepackage{multirow}                                         
\usepackage{hhline}                                           
\usepackage{ifthen}                                           
\usepackage{lscape}



\usepackage{amsmath,amssymb}
\usepackage{booktabs}
\usepackage{tikz}
\usetikzlibrary{arrows.meta,angles,quotes}





\begin{document}

\bibliographystyle{IEEEtran}
\vspace{3cm}

\title{1.2.29}
\author{AI25BTECH11021 - Abhiram Reddy N}
% \maketitle
% \newpage
% \bigskip
{\let\newpage\relax\maketitle}

\renewcommand{\thefigure}{\theenumi}
\renewcommand{\thetable}{\theenumi}
\setlength{\intextsep}{10pt} % Space between text and floats


\numberwithin{equation}{enumi}
\numberwithin{figure}{enumi}
\renewcommand{\thetable}{\theenumi}


\textbf{Question}:\\For what value of p are the points (2, 1), (p, -1), and (-1, 3) collinear?


\textbf{Solution}

The points \( A, B, C \) are collinear if the vectors \( \overrightarrow{AB} \) and \( \overrightarrow{AC} \) are linearly dependent. This means one is a scalar multiple of the other.

We form the vectors:
\[
\overrightarrow{AB} = (p - 2, -1 - 1) = (p - 2, -2)
\]
\[
\overrightarrow{AC} = (-1 - 2, 3 - 1) = (-3, 2)
\]

Create the matrix with these vectors as rows:
\[
M = \begin{bmatrix}
p - 2 & -2 \\
-3 & 2
\end{bmatrix}
\]

Perform row operations to put the matrix into echelon form.

\[
R_1 = [p-2 \quad -2]
\]

Eliminate the first element of the second row:
\[
R_2 \to R_2 + \frac{3}{p-2} R_1
\]
\[
R_2 = [-3, 2] + \frac{3}{p-2}[p-2, -2] = [-3 + 3, \quad 2 - \frac{6}{p-2}] = [0, \quad 2 - \frac{6}{p-2}]
\]

For the vectors to be linearly dependent, the second row must be zero:
\[
2 - \frac{6}{p-2} = 0
\]

Multiply both sides by \( p - 2 \):
\[
2(p - 2) - 6 = 0
\]
\[
2p - 4 - 6 = 0
\]
\[
2p - 10 = 0
\]
\[
\boxed{p = 5}
\]

---

\section*{Verification by plotting}

The points become:
\[
A = (2, 1), \quad B = (5, -1), \quad C = (-1, 3)
\]

These points lie on the same straight line.









\begin{figure}[htbp]
\centering
\includegraphics[width=0.8\columnwidth]{figs/python image.png} 
\caption{Relative wind vector $\mathbf{R}$ obtained as $\mathbf{W}-\mathbf{V}$}
\label{fig:wind}
\end{figure}
















\end{document}