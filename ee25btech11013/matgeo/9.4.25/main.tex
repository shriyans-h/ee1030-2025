\let\negmedspace\undefined
\let\negthickspace\undefined
\documentclass[journal]{IEEEtran}
\usepackage[a5paper, margin=10mm, onecolumn]{geometry}
%\usepackage{lmodern} % Ensure lmodern is loaded for pdflatex
\usepackage{tfrupee} % Include tfrupee package

\setlength{\headheight}{1cm} % Set the height of the header box
\setlength{\headsep}{0mm}     % Set the distance between the header box and the top of the text

\usepackage{gvv-book}
\usepackage{comment}
\usepackage{gvv}
\usepackage{cite}
\usepackage{amsmath,amssymb,amsfonts,amsthm}
\usepackage{algorithmic}
\usepackage{graphicx}
\usepackage{textcomp}
\usepackage{xcolor}
%\usepackage{txfonts}
\usepackage{listings}
\usepackage{enumitem}
\usepackage{mathtools}
\usepackage{gensymb}
\usepackage{comment}
\usepackage[breaklinks=true]{hyperref}
\usepackage{tkz-euclide} 
\usepackage{listings}
% \usepackage{gvv}                                        
\def\inputGnumericTable{}                                 
\usepackage[latin1]{inputenc}                                
\usepackage{color}                                            
\usepackage{array}                                            
\usepackage{longtable}                                       
\usepackage{calc}                                             
\usepackage{multirow}                                         
\usepackage{hhline}                                           
\usepackage{ifthen}                                           
\usepackage{lscape}
\usepackage{circuitikz}
\tikzstyle{block} = [rectangle, draw, fill=blue!20, 
    text width=4em, text centered, rounded corners, minimum height=3em]
\tikzstyle{sum} = [draw, fill=blue!10, circle, minimum size=1cm, node distance=1.5cm]
\tikzstyle{input} = [coordinate]
\tikzstyle{output} = [coordinate]


\begin{document}

\bibliographystyle{IEEEtran}
\vspace{3cm}

\title{9.4.25}
\author{EE25BTECH11013 - Bhargav}
\maketitle
    {\let\newpage\relax\maketitle}

\renewcommand{\thefigure}{\theenumi}
\renewcommand{\thetable}{\theenumi}
\setlength{\intextsep}{10pt} % Space between text and floats

\numberwithin{equation}{enumi}
\numberwithin{figure}{enumi}
\renewcommand{\thetable}{\theenumi}

\textbf{Question}: \\
Find the roots of the quadratic equation graphically.
\begin{align}
5x^2 - 6x - 2 = 0
\end{align}
\solution \\

\begin{align}
y = 5x^2 - 6x - 2 = 0
\end{align}

This equation can be represented as the conic
\begin{align}
\vec{x^T}\vec{V}\vec{x} + 2\vec{u^T}\vec{x} + f = 0
\end{align}
\begin{align}
\vec{V} = \myvec{5 & 0 \\ 0 & 0}, \vec{u} = \myvec{-3 \\ 0}, f = -2
\end{align}

To find the roots, we find the points of intersection of the conic with the x-axis.
\begin{align}
\vec{x} = \vec{h} + k_i\vec{m}    
\end{align}
\begin{align}
\vec{h}=\myvec{0 \\ 0}, \vec{m} = \myvec{1 \\ 0}
\end{align}

The value of $k_i$ can be found out by solving the line and conic equation

\begin{align}
(\vec{h} + k_i \vec{m})^{\top} \vec{V} (\vec{h} + k_i \vec{m}) + 2\vec{u}^{\top} (\vec{h} + k_i \vec{m}) + f &= 0 \\
\implies k_i^{2} \vec{m}^{\top}\vec{V}\vec{m} + 2k_i \vec{m}^{\top} (\vec{V}\vec{h} + \vec{u}) + \vec{h}^{\top}\vec{V}\vec{h} + 2\vec{u}^{\top}\vec{h} + f &= 0 \\
\text{or, } k_i^{2} \vec{m}^{\top}\vec{V}\vec{m} + 2k_i \vec{m}^{\top} (\vec{V}\vec{h} + \vec{u}) + g(\vec{h}) &= 0
\end{align}

Solving the above quadratic gives the equation
\begin{align}
k_i = \frac{1}{\vec{m}^{\top}\vec{V}\vec{m}}
\brak{
    -\vec{m}^{\top} (\vec{V}\vec{h} + \vec{u})
    \;\pm\;
    \sqrt{ \sbrak{\vec{m}^{\top}(\vec{V}\vec{h} + \vec{u})}^2
    - g(\vec{h}) \, (\vec{m}^{\top}\vec{V}\vec{m}) }
    }
\end{align}

Substituting the values in the above equation gives
\begin{align}
\therefore k_i = \frac{3}{5} \;\pm\; \frac{\sqrt{19}}{5}
\end{align}

\begin{align}
\implies k_1 = \frac{3}{5} + \frac{\sqrt{19}}{5}, k_2 = \frac{3}{5} - \frac{\sqrt{19}}{5}
\end{align}

\begin{align}
\therefore \vec{x} = \vec{h} + k_i\vec{m} = \myvec{\frac{3}{5} + \frac{\sqrt{19}}{5} \\ 0} ,  \myvec{\frac{3}{5} - \frac{\sqrt{19}}{5} \\ 0}
\end{align}

\begin{figure}[h!]
    \centering
    \includegraphics[height=0.5\textheight, keepaspectratio]{figs/Figure_1.png}
    \label{figure_1}
\end{figure}

\end{document}


