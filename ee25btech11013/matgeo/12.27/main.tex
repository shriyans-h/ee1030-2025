\let\negmedspace\undefined
\let\negthickspace\undefined
\documentclass[journal]{IEEEtran}
\usepackage[a5paper, margin=10mm, onecolumn]{geometry}
%\usepackage{lmodern} % Ensure lmodern is loaded for pdflatex
\usepackage{tfrupee} % Include tfrupee package

\setlength{\headheight}{1cm} % Set the height of the header box
\setlength{\headsep}{0mm}     % Set the distance between the header box and the top of the text

\usepackage{gvv-book}
\usepackage{comment}
\usepackage{gvv}
\usepackage{cite}
\usepackage{amsmath,amssymb,amsfonts,amsthm}
\usepackage{algorithmic}
\usepackage{graphicx}
\usepackage{textcomp}
\usepackage{xcolor}
%\usepackage{txfonts}
\usepackage{listings}
\usepackage{enumitem}
\usepackage{mathtools}
\usepackage{gensymb}
\usepackage{comment}
\usepackage[breaklinks=true]{hyperref}
\usepackage{tkz-euclide} 
\usepackage{listings}
% \usepackage{gvv}                                        
\def\inputGnumericTable{}                                 
\usepackage[latin1]{inputenc}                                
\usepackage{color}                                            
\usepackage{array}                                            
\usepackage{longtable}                                       
\usepackage{calc}                                             
\usepackage{multirow}                                         
\usepackage{hhline}                                           
\usepackage{ifthen}                                           
\usepackage{lscape}
\usepackage{circuitikz}
\tikzstyle{block} = [rectangle, draw, fill=blue!20, 
    text width=4em, text centered, rounded corners, minimum height=3em]
\tikzstyle{sum} = [draw, fill=blue!10, circle, minimum size=1cm, node distance=1.5cm]
\tikzstyle{input} = [coordinate]
\tikzstyle{output} = [coordinate]


\begin{document}

\bibliographystyle{IEEEtran}
\vspace{3cm}

\title{12.27}
\author{EE25BTECH11013 - Bhargav}
\maketitle
    {\let\newpage\relax\maketitle}

\renewcommand{\thefigure}{\theenumi}
\renewcommand{\thetable}{\theenumi}
\setlength{\intextsep}{10pt} % Space between text and floats

\numberwithin{equation}{enumi}
\numberwithin{figure}{enumi}
\renewcommand{\thetable}{\theenumi}

\textbf{Question}: \\
1200 men and 500 women can build a bridge in 2 weeks. 900 men and 250 women will take 3 weeks to build the same bridge. How many men will be needed to build the bridge in one week? \\
\solution \\
Let one man complete work in x weeks and one woman complete work in y weeks\\
In one week a man can complete $\frac{1}{x}$ work and woman can complete $\frac{1}{y}$
\begin{align}
\frac{1200}{x} + \frac{500}{y} = \frac{1}{2}
\end{align}
\begin{align}
\frac{900}{x} + \frac{250}{y} = \frac{1}{3}
\end{align}

\begin{align}
\myvec{1200 & 500 \\ 900 & 250}\myvec{\frac{1}{x} \\ \frac{1}{y}} = \myvec{\frac{1}{2} \\ \frac{1}{3}}
\end{align}
This can be converted into an augmented matrix and can be solved by Gaussian elimination:
\begin{align}
\augvec{2}{1}{1200 & 500 & \frac{1}{2} \\ 900 & 250 & \frac{1}{3}} \xleftrightarrow[R_2 \leftarrow R_2/125]{R_2 \leftarrow R_2 - 3R_1/4} \augvec{2}{1}{1200 & 500 & \frac{1}{2} \\ 0 & 1 & \frac{1}{3000}}
\end{align}
\begin{align}
\xleftarrow[R_1 \leftarrow R_1 - 500R_2]{R_1 \leftarrow R_1/1200} \augvec{2}{1}{1 & 0 & \frac{1}{3600} \\ 0 & 1 & \frac{1}{3000}}
\end{align}

\begin{align}
\myvec{\frac{1}{x} \\ \frac{1}{y}} = \myvec{\frac{1}{3600} \\ \frac{1}{3000}}
\end{align}

A man can finish the work in 3600 weeks, a woman can finish the work in 3000 weeks
Therefore 3600 men are required for completing the task in 1 week
\end{document}
