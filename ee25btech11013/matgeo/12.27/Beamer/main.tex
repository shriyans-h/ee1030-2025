\documentclass{beamer}
\usepackage[utf8]{inputenc}

\usetheme{Madrid}
\usecolortheme{default}
\usepackage{amsmath,amssymb,amsfonts,amsthm}
\usepackage{mathtools}
\usepackage{txfonts}
\usepackage{tkz-euclide}
\usepackage{listings}
\usepackage{adjustbox}
\usepackage{tfrupee}
\usepackage{array}
\usepackage{gensymb}
\usepackage{tabularx}
\usepackage{gvv}
\usepackage{lmodern}
\usepackage{circuitikz}
\usepackage{tikz}
\lstset{literate={·}{{$\cdot$}}1 {λ}{{$\lambda$}}1 {→}{{$\to$}}1}
\usepackage{graphicx}

\setbeamertemplate{page number in head/foot}[totalframenumber]

\usepackage{tcolorbox}
\tcbuselibrary{minted,breakable,xparse,skins}

\definecolor{bg}{gray}{0.95}
\DeclareTCBListing{mintedbox}{O{}m!O{}}{%
  breakable=true,
  listing engine=minted,
  listing only,
  minted language=#2,
  minted style=default,
  minted options={%
    linenos,
    gobble=0,
    breaklines=true,
    breakafter=,,
    fontsize=\small,
    numbersep=8pt,
    #1},
  boxsep=0pt,
  left skip=0pt,
  right skip=0pt,
  left=25pt,
  right=0pt,
  top=3pt,
  bottom=3pt,
  arc=5pt,
  leftrule=0pt,
  rightrule=0pt,
  bottomrule=2pt,
  toprule=2pt,
  colback=bg,
  colframe=orange!70,
  enhanced,
  overlay={%
    \begin{tcbclipinterior}
    \fill[orange!20!white] (frame.south west) rectangle ([xshift=20pt]frame.north west);
    \end{tcbclipinterior}},
  #3,
}
\lstset{
    language=C,
    basicstyle=\ttfamily\small,
    keywordstyle=\color{blue},
    stringstyle=\color{orange},
    commentstyle=\color{green!60!black},
    numbers=left,
    numberstyle=\tiny\color{gray},
    breaklines=true,
    showstringspaces=false,
}

\title{12.27}
\date{September 29, 2025}
\author{Bhargav - EE25BTECH11013}

\begin{document}

\frame{\titlepage}

\begin{frame}{Question}
\textbf{Question}: \\
1200 men and 500 women can build a bridge in 2 weeks. 900 men and 250 women will take 3 weeks to build the same bridge. How many men will be needed to build the bridge in one week? \\
\end{frame}
\begin{frame}{Solution}
Let one man complete work in x weeks and one woman complete work in y weeks\\
In one week a man can complete $\frac{1}{x}$ work and woman can complete $\frac{1}{y}$
\begin{align}
\frac{1200}{x} + \frac{500}{y} = \frac{1}{2}
\end{align}
\begin{align}
\frac{900}{x} + \frac{250}{y} = \frac{1}{3}
\end{align}

\begin{align}
\myvec{1200 & 500 \\ 900 & 250}\myvec{\frac{1}{x} \\ \frac{1}{y}} = \myvec{\frac{1}{2} \\ \frac{1}{3}}
\end{align}

\end{frame}

\begin{frame}{Solution}

This can be converted into an augmented matrix and can be solved by Gaussian elimination:
\begin{align}
\augvec{2}{1}{1200 & 500 & \frac{1}{2} \\ 900 & 250 & \frac{1}{3}} \xleftrightarrow[R_2 \leftarrow R_2/125]{R_2 \leftarrow R_2 - 3R_1/4} \augvec{2}{1}{1200 & 500 & \frac{1}{2} \\ 0 & 1 & \frac{1}{3000}}
\end{align}
\begin{align}
\xleftarrow[R_1 \leftarrow R_1 - 500R_2]{R_1 \leftarrow R_1/1200} \augvec{2}{1}{1 & 0 & \frac{1}{3600} \\ 0 & 1 & \frac{1}{3000}}
\end{align}

\begin{align}
\myvec{\frac{1}{x} \\ \frac{1}{y}} = \myvec{\frac{1}{3600} \\ \frac{1}{3000}}
\end{align}

A man can finish the work in 3600 weeks, a woman can finish the work in 3000 weeks
\end{frame}



\begin{frame}[fragile]
    \frametitle{C Code}
    \begin{lstlisting}
void mat_vec_mult(double* a, double* x, double* result) {
    result[0] = a[0] * x[0] + a[1] * x[1];
    result[1] = a[2] * x[0] + a[3] * x[1];
}


    \end{lstlisting}
\end{frame}
\begin{frame}[fragile]
    \frametitle{Python + C Code}
    \begin{lstlisting}
import numpy as np
import ctypes
lib_path = "./libcode.so" 
c_lib = ctypes.CDLL(lib_path)
c_lib.mat_vec_mult.argtypes = [
    ctypes.POINTER(ctypes.c_double),
    ctypes.POINTER(ctypes.c_double),
    ctypes.POINTER(ctypes.c_double)
]
c_lib.mat_vec_mult.restype = None

a = np.array([[1200, 500], [900, 250]], dtype=np.float64)
b = np.array([[1/2], [1/3]], dtype=np.float64)

x = np.linalg.solve(a, b)
result_from_c = np.zeros_like(b)
a_ptr = a.ctypes.data_as(ctypes.POINTER(ctypes.c_double))


    \end{lstlisting}
\end{frame}
\begin{frame}[fragile]
    \frametitle{Python + C Code}
    \begin{lstlisting}
x_ptr = x.ctypes.data_as(ctypes.POINTER(ctypes.c_double))
result_ptr = result_from_c.ctypes.data_as(ctypes.POINTER(ctypes.c_double))

c_lib.mat_vec_mult(a_ptr, x_ptr, result_ptr)

print("Numpy result:")
print(1/x[0])
print(1/x[1])
print("\nResult of A*x from C code (for verification):")
print(result_from_c[0])
print(result_from_c[1])
if np.allclose(result_from_c, b):
    print("\nVerification successful: The C result matches 'b'.")




    \end{lstlisting}
\end{frame}
\begin{frame}[fragile]
    \frametitle{Python Code}
    \begin{lstlisting}
import numpy as np
import matplotlib.pyplot as plt

a = np.array([[1200, 500], [900, 250]])
b = np.array([[1/2], [1/3]])
x = np.linalg.solve(a, b)
print("Man can finish the task in ", 1/x[0], " weeks")
print("Woman can finish the task in ", 1/x[1], " weeks")



    \end{lstlisting}
\end{frame}


\end{document}

