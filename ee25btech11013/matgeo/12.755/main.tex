\let\negmedspace\undefined
\let\negthickspace\undefined
\documentclass[journal]{IEEEtran}
\usepackage[a5paper, margin=10mm, onecolumn]{geometry}
%\usepackage{lmodern} % Ensure lmodern is loaded for pdflatex
\usepackage{tfrupee} % Include tfrupee package

\setlength{\headheight}{1cm} % Set the height of the header box
\setlength{\headsep}{0mm}     % Set the distance between the header box and the top of the text

\usepackage{gvv-book}
\usepackage{comment}
\usepackage{gvv}
\usepackage{cite}
\usepackage{amsmath,amssymb,amsfonts,amsthm}
\usepackage{algorithmic}
\usepackage{graphicx}
\usepackage{textcomp}
\usepackage{xcolor}
%\usepackage{txfonts}
\usepackage{listings}
\usepackage{enumitem}
\usepackage{mathtools}
\usepackage{gensymb}
\usepackage{comment}
\usepackage[breaklinks=true]{hyperref}
\usepackage{tkz-euclide} 
\usepackage{listings}
% \usepackage{gvv}                                        
\def\inputGnumericTable{}                                 
\usepackage[latin1]{inputenc}                                
\usepackage{color}                                            
\usepackage{array}                                            
\usepackage{longtable}                                       
\usepackage{calc}                                             
\usepackage{multirow}                                         
\usepackage{hhline}                                           
\usepackage{ifthen}                                           
\usepackage{lscape}
\usepackage{circuitikz}
\tikzstyle{block} = [rectangle, draw, fill=blue!20, 
    text width=4em, text centered, rounded corners, minimum height=3em]
\tikzstyle{sum} = [draw, fill=blue!10, circle, minimum size=1cm, node distance=1.5cm]
\tikzstyle{input} = [coordinate]
\tikzstyle{output} = [coordinate]


\begin{document}

\bibliographystyle{IEEEtran}
\vspace{3cm}

\title{12.755}
\author{EE25BTECH11013 - Bhargav}
\maketitle
    {\let\newpage\relax\maketitle}

\renewcommand{\thefigure}{\theenumi}
\renewcommand{\thetable}{\theenumi}
\setlength{\intextsep}{10pt} % Space between text and floats

\numberwithin{equation}{enumi}
\numberwithin{figure}{enumi}
\renewcommand{\thetable}{\theenumi}

\textbf{Question}: \\
Which one of the following vectors is an eigenvector corresponding to the eigenvalue $\lambda = 1$ for the matrix $\vec{A} = \myvec{1 & -1 & 0 \\  1 & -1 & 1 \\ -1 & 0 & 1}$ is

\solution \\

The eigenvalue of matrix $\vec{A}$ can be found out by ( where $\lambda$ is the eigenvalue, $\vec{x}$ is the eigenvector, $\vec{I}$ is the identity matrix)
\begin{align}
\vec{A}\vec{x} = \lambda \vec{x} \implies \brak{\vec{A}-\lambda\vec{I}}\vec{x} = \vec{0}
\end{align}
\begin{align}
\brak{\vec{A}-\vec{I}}\vec{x} = \vec{0}
\end{align}
\begin{align}
\implies \myvec{0 & -1 & 0 \\ 1 & -2 & 1 \\ -1 & 0 & 0}\vec{x} = \vec{0}
\end{align}
This can be solved by representing it as an augmented matrix and using row elimination
\begin{align}
\augvec{3}{1}{0 & -1 & 0 & 0 \\ 1 & -2 & 1 & 0 \\ -1 & 0 & 0 & 0} \xleftrightarrow{R_1 \leftarrow R_1 + R_2} \augvec{3}{1}{1 & -3 & 1 & 0 \\ 1 & -2 & 1 & 0 \\ -1 & 0 & 0 & 0} \xleftrightarrow[R_3 \leftarrow R_3 + R_1]{R_2 \leftarrow R_2 - R_1}
\end{align}
\begin{align}
\augvec{3}{1}{1 & -3 & 1 & 0 \\ 0 & 1 & 0 & 0 \\ 0 & -3 & 1 & 0}\xleftrightarrow[R_3 \leftarrow R_3 + 3R_2]{R_1 \leftarrow R_1 + 3R_2}\augvec{3}{1}{1 & 0 & 0 & 0 \\ 0 & 1 & 0 & 0 \\ 0 & 0 & 1 & 0}
\end{align}

Thus we get $\vec{x} = \myvec{0 \\ 0 \\ 0}$ which is the eigenvector of the matrix $\vec{A}$ corresponding to the eigenvalue $\lambda = 1$

This can be further verified by the intersection of planes
\begin{align}
-y = 0
\end{align}
\begin{align}
x-2y+z=0
\end{align}
\begin{align}
-x = 0
\end{align}

The intersection point is $\myvec{0 \\ 0 \\ 0}$
    \begin{figure}[H]
        \centering
        \includegraphics[height=0.5\textheight, keepaspectratio]{figs/Figure_1.png}
        \label{figure_1}
    \end{figure}
\end{document}

