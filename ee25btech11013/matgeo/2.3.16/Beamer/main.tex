\documentclass{beamer}
\usepackage[utf8]{inputenc}

\usetheme{Madrid}
\usecolortheme{default}
\usepackage{amsmath,amssymb,amsfonts,amsthm}
\usepackage{txfonts}
\usepackage{tkz-euclide}
\usepackage{listings}
\usepackage{adjustbox}
\usepackage{array}
\usepackage{tabularx}
\usepackage{gvv}
\usepackage{lmodern}
\usepackage{circuitikz}
\usepackage{tikz}
\usepackage{graphicx}

\setbeamertemplate{page number in head/foot}[totalframumber]

\title{2.3.16}

\author{EE25BTECH11013 - Bhargav}
\date{September 1, 2025}

\begin{document}

\frame{\titlepage}


\begin{frame}{Question}
    If $\vec{p}$ is a unit vector and $(\vec{x}-\vec{p})\cdot(\vec{x}+\vec{p})=80$, then find $\norm{\vec{x}}$.
\end{frame}


\begin{frame}{Solution}
    We are given the equation:
    \begin{align}
        (\vec{x}-\vec{p})^\top \cdot(\vec{x}+\vec{p}) = 80
    \end{align}
    
    \vfill
    
    Expand the product:
    \begin{align}
        \vec{x}^\top \vec{x} + \vec{x}^\top \vec{p} - \vec{p}^\top \vec{x} - \vec{p}^\top \vec{p}&= 80
    \end{align}
    
    \vfill
    
    Since the product is commutative ($\vec{x}^\top \cdot\vec{p} = \vec{p}^\top \cdot\vec{x}$), the middle terms cancel out:
    \begin{align}
         \vec{x}^\top \cdot\vec{x} - \vec{p}^\top \cdot\vec{p} &= 80
    \end{align}
\end{frame}


\begin{frame}{Final Calculation}
    Since $\vec{v}^\top \cdot\vec{v} = \norm{\vec{v}}^2$.
    \begin{align}
        \norm{\vec{x}}^2 - \norm{\vec{p}}^2 = 80
    \end{align}
    
    \vfill
    
    We are given that $\vec{p}$ is a \textbf{unit vector}, so its magnitude is 1.
    \begin{align}
        \norm{\vec{p}} = 1 \implies \norm{\vec{p}}^2 = 1
    \end{align}
    
    \vfill
    
    Substituting this value into the equation:
    \begin{align}
        \norm{\vec{x}}^2 - 1 &= 80 \\
        \norm{\vec{x}}^2 &= 81 \\
        \norm{\vec{x}} &= 9
    \end{align}
    
    \vfill
    
    \textbf{Therefore, the magnitude of vector $\vec{x}$ is 9.}
\end{frame}


\begin{frame}{Verification Example}
    The theoretical solution can be verified by example.
    
    \vfill
    
    Assume that $\vec{p}$ is the unit vector $\myvec{1 \\ 0}$.
    
    \vfill

    Then from the code we get a possible vector $\vec{x}$ would be $\myvec{9 \\ 0}$.
    
    \vfill
    
    The magnitude of the $\vec{x}$ is verified to be $9$.
    
    
\end{frame}

\begin{frame}[fragile]
    \frametitle{C Code}
    \begin{lstlisting}
#include <stdio.h>
#include <math.h>
void find_magnitude(double *x, double *x_norm) {
    double p[2] = {1.0, 0.0};
    double given_value = 80.0;
    double p_norm_sq = p[0]*p[0] + p[1]*p[1];
    double x_norm_sq = given_value + p_norm_sq;
    *x_norm = sqrt(x_norm_sq);
    x[0] = *x_norm;
    x[1] = 0.0;
}

    \end{lstlisting}
\end{frame}

\begin{frame}[fragile]
    \frametitle{Python + C Code}
    \begin{lstlisting}
import ctypes
import numpy as np
lib = ctypes.CDLL("./libmagnitude.so")
lib.find_magnitude.argtypes = [np.ctypeslib.ndpointer(dtype=np.float64, ndim=1, flags="C"), ctypes.POINTER(ctypes.c_double)]
lib.find_magnitude.restype = None
x = np.zeros(2, dtype=np.float64)
x_norm = ctypes.c_double()
lib.find_magnitude(x, ctypes.byref(x_norm))
print("Result from C:")
print("x =", x)
print("||x|| =", x_norm.value)
p = np.array([1.0, 0.0])
lhs = np.dot(x - p, x + p)
print("(x - p)^T (x + p) =", lhs)
print("||x|| =", np.linalg.norm(x))
    \end{lstlisting}
\end{frame}

\begin{frame}[fragile]
    \frametitle{Python Code}
    \begin{lstlisting}
import numpy as np
p = np.array([1, 0])
given_value = 80
p_norm_sq = np.dot(p, p)
x_norm_sq = given_value + p_norm_sq
x_norm = np.sqrt(x_norm_sq)
x = np.array([x_norm, 0])
lhs = np.dot(x - p, x + p)
print("||p||^2 =", p_norm_sq)
print("||x|| =", x_norm)
print("Example x =", x)
print("Verification (x - p)^T (x + p) =", lhs)

    \end{lstlisting}
\end{frame}

\end{document}