\let\negmedspace\undefined
\let\negthickspace\undefined
\documentclass[journal]{IEEEtran}
\usepackage[a5paper, margin=10mm, onecolumn]{geometry}
%\usepackage{lmodern} % Ensure lmodern is loaded for pdflatex
\usepackage{tfrupee} % Include tfrupee package

\setlength{\headheight}{1cm} % Set the height of the header box
\setlength{\headsep}{0mm}     % Set the distance between the header box and the top of the text

\usepackage{gvv-book}
\usepackage{comment}
\usepackage{gvv}
\usepackage{cite}
\usepackage{amsmath,amssymb,amsfonts,amsthm}
\DeclareMathOperator{\Tr}{Tr}
\usepackage{algorithmic}
\usepackage{graphicx}
\usepackage{textcomp}
\usepackage{xcolor}
%\usepackage{txfonts}
\usepackage{listings}
\usepackage{enumitem}
\usepackage{mathtools}
\usepackage{gensymb}
\usepackage{comment}
\usepackage[breaklinks=true]{hyperref}
\usepackage{tkz-euclide} 
\usepackage{listings}
% \usepackage{gvv}                                        
\def\inputGnumericTable{}                                 
\usepackage[latin1]{inputenc}                                
\usepackage{color}                                            
\usepackage{array}                                            
\usepackage{longtable}                                       
\usepackage{calc}                                             
\usepackage{multirow}                                         
\usepackage{hhline}                                           
\usepackage{ifthen}                                           
\usepackage{lscape}
\usepackage{circuitikz}
\tikzstyle{block} = [rectangle, draw, fill=blue!20, 
    text width=4em, text centered, rounded corners, minimum height=3em]
\tikzstyle{sum} = [draw, fill=blue!10, circle, minimum size=1cm, node distance=1.5cm]
\tikzstyle{input} = [coordinate]
\tikzstyle{output} = [coordinate]


\begin{document}

\bibliographystyle{IEEEtran}
\vspace{3cm}

\title{12.339}
\author{EE25BTECH11013 - Bhargav}
\maketitle
    {\let\newpage\relax\maketitle}

\renewcommand{\thefigure}{\theenumi}
\renewcommand{\thetable}{\theenumi}
\setlength{\intextsep}{10pt} % Space between text and floats

\numberwithin{equation}{enumi}
\numberwithin{figure}{enumi}
\renewcommand{\thetable}{\theenumi}

\textbf{Question}: \\
If
\begin{align}
\vec{A} = \myvec{3 & -3 \\ -3 & 4}
\end{align}
then 
\begin{align}
\det{\brak{-\vec{A}^2 + 7\vec{A} - 3\vec{I}}}
\end{align}
is \\ \\
\solution \\

The characteristic equation of matrix $\vec{A}$ is
\begin{align}
f(\lambda) = \abs{\vec{A} - \lambda\vec{I}} = 0
\end{align}
\begin{align}
\abs{\myvec{3-\lambda & -3 \\ -3 & 4-\lambda}} = 0
\end{align}
\begin{align}
\implies \brak{3-\lambda}\brak{4-\lambda} - 9 = 0
\end{align}
\begin{align}
-\lambda^2 + 7\lambda -3 = 0
\end{align}

According to Cayley-Hamilton Theorem:
\begin{align}
f(\lambda) = f(\vec{A})
\end{align}
\begin{align}
\therefore -\vec{A}^2 + 7\vec{A} - 3\vec{I} = 0
\end{align}
\begin{align}
\therefore \abs{-\vec{A}^2 + 7\vec{A} - 3\vec{I}} = 0
\end{align}
\end{document}