\documentclass{beamer}
\usepackage[utf8]{inputenc}

\usetheme{Madrid}
\usecolortheme{default}
\usepackage{amsmath,amssymb,amsfonts,amsthm}
\usepackage{txfonts}
\usepackage{tkz-euclide}
\usepackage{listings}
\usepackage{adjustbox}
\usepackage{array}
\usepackage{tabularx}
\usepackage{gvv}
\usepackage{lmodern}
\usepackage{circuitikz}
\usepackage{tikz}
\usepackage{graphicx}

\setbeamertemplate{page number in head/foot}[totalframumber]

\title{2.10.42}
\author{EE25BTECH11013 - Bhargav}
\date{September 5, 2025}

\begin{document}

\frame{\titlepage}

\begin{frame}{Question}
If $\vec{a}, \vec{b}$ and $\vec{c}$ are unit coplanar vectors, evaluate the scalar triple product:

$\sbrak{2\vec{a} - \vec{b} \;\quad 2\vec{b} - \vec{c} \;\quad 2\vec{c} - \vec{a}}$

\vfill

\end{frame}

\begin{frame}{Solution}
\textbf{Given Condition:} Since the vectors are coplanar, their scalar triple product is zero.
\begin{align}
\vec{B} = \myvec{2\vec{a} - \vec{b} & 2\vec{b} - \vec{c} & 2\vec{c} - \vec{a}} = 
\myvec{
\vec{a} & \vec{b} & \vec{c}
}
\myvec{
2 & 0 & -1 \\
-1 & 2 & 0 \\
0 & -1 & 2
}
\end{align}

\end{frame}

\begin{frame}{Solution}
\begin{align}
\therefore \det(\vec{B}) = \det(\vec{A})\cdot \det(\vec{M})
\end{align}
Since $\vec{a}, \vec{b}, \vec{c}$ are coplanar,
\begin{align}
\sbrak{\vec{a} \;\quad \vec{b} \;\quad \vec{c}} = 0.
\end{align}


\begin{align}
\Rightarrow \sbrak{2\vec{a} - \vec{b}\;\quad 2\vec{b} - \vec{c}\;\quad 2\vec{c} - \vec{a}} = \det(\vec{M}) \cdot \sbrak{\vec{a}\;\quad \vec{b}\;\quad \vec{c}} = 0.
\end{align}

\end{frame}


\begin{frame}{Verification}
This can be verified by taking an example of 3 coplanar unit vectors.\\
\begin{align}
\vec{a} = \myvec{1 \\ 0 \\ 0}
\end{align}

\begin{align}
\vec{b} = \myvec{0 \\ 1 \\ 0}
\end{align}

\begin{align}
\vec{c} = \myvec{0.6 \\ 0.8 \\ 0}
\end{align}

\begin{align}
\vec{X} = \sbrak{2\vec{a}-\vec{b} \;\quad 2\vec{b}-c \;\quad 2\vec{c}-\vec{a}} 
\end{align}

From the code, it is clear that the value of $\vec{X}$ is 0   
\end{frame}
\begin{frame}[fragile]
    \frametitle{C Code}
    \begin{lstlisting}
#include <stdio.h>

typedef struct {
    double x, y, z;
} Vector;


Vector createVector(double x, double y, double z) {
    Vector v = {x, y, z};
    return v;
}

Vector subtract(Vector u, Vector v) {
    return createVector(u.x - v.x, u.y - v.y, u.z - v.z);
}

    \end{lstlisting}
\end{frame}

\begin{frame}[fragile]
    \frametitle{C Code}
    \begin{lstlisting}
Vector scale(Vector u, double k) {
    return createVector(k*u.x, k*u.y, k*u.z);
}

Vector cross(Vector u, Vector v) {
    return createVector(
        u.y*v.z - u.z*v.y,
        u.z*v.x - u.x*v.z,
        u.x*v.y - u.y*v.x
    );
}

double dot(Vector u, Vector v) {
    return u.x*v.x + u.y*v.y + u.z*v.z;
}

    \end{lstlisting}
\end{frame}

\begin{frame}[fragile]
    \frametitle{C Code}
    \begin{lstlisting}
double triple(Vector u, Vector v, Vector w) {
    return dot(u, cross(v, w));
}

Vector twominus(Vector a, Vector b) {
    return subtract(scale(a, 2), b); 
}


double computeX(Vector a, Vector b, Vector c) {
    Vector v1 = twominus(a, b);
    Vector v2 = twominus(b, c);
    Vector v3 = twominus(c, a);
    return triple(v1, v2, v3);
}


    \end{lstlisting}
\end{frame}

\begin{frame}[fragile]
    \frametitle{C Code}
    \begin{lstlisting}
__attribute__((visibility("default"))) 
double computeX_py(double ax, double ay, double az,
                   double bx, double by, double bz,
                   double cx, double cy, double cz) {
    Vector a = createVector(ax, ay, az);
    Vector b = createVector(bx, by, bz);
    Vector c = createVector(cx, cy, cz);
    return computeX(a, b, c);
}

    \end{lstlisting}
\end{frame}

\begin{frame}[fragile]
    \frametitle{Python + C Code}
    \begin{lstlisting}
import ctypes
lib = ctypes.CDLL("./libscalartp.so")
lib.computeX_py.argtypes = [ctypes.c_double, ctypes.c_double, ctypes.c_double,
                            ctypes.c_double, ctypes.c_double, ctypes.c_double,
                            ctypes.c_double, ctypes.c_double, ctypes.c_double]
lib.computeX_py.restype = ctypes.c_double
a = (1.0, 0.0, 0.0)
b = (0.0, 1.0, 0.0)
c = (0.6, 0.8, 0.0)
X = lib.computeX_py(*a, *b, *c)
print("X =", X)
    \end{lstlisting}
\end{frame}

\begin{frame}[fragile]
    \frametitle{Python Code}
    \begin{lstlisting}
import numpy as np


a = np.array([1, 0, 0])
b = np.array([0,1,0])
c = np.array([0.6, 0.8, 0])

x = np.dot(2*a-b, np.cross(2*b-c,2*c-a))
print("Value of x: ", x)


    \end{lstlisting}
\end{frame}

\end{document}