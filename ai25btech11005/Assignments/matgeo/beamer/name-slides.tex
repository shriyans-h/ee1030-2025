\documentclass{beamer}
\usepackage[utf8]{inputenc}

\usetheme{Madrid}
\usecolortheme{default}
\usepackage{amsmath,amssymb,amsfonts,amsthm}
\usepackage{txfonts}
\usepackage{tkz-euclide}
\usepackage{listings}
\usepackage{adjustbox}
\usepackage{array}
\usepackage{tabularx}
\usepackage{gvv}
\usepackage{lmodern}
\usepackage{circuitikz}
\usepackage{tikz}
\usepackage{graphicx}

\setbeamertemplate{page number in head/foot}[totalframenumber]

\usepackage{tcolorbox}
\tcbuselibrary{minted,breakable,xparse,skins}



\definecolor{bg}{gray}{0.95}
\DeclareTCBListing{mintedbox}{O{}m!O{}}{%
  breakable=true,
  listing engine=minted,
  listing only,
  minted language=#2,
  minted style=default,
  minted options={%
    linenos,
    gobble=0,
    breaklines=true,
    breakafter=,,
    fontsize=\small,
    numbersep=8pt,
    #1},
  boxsep=0pt,
  left skip=0pt,
  right skip=0pt,
  left=25pt,
  right=0pt,
  top=3pt,
  bottom=3pt,
  arc=5pt,
  leftrule=0pt,
  rightrule=0pt,
  bottomrule=2pt,
  toprule=2pt,
  colback=bg,
  colframe=orange!70,
  enhanced,
  overlay={%
    \begin{tcbclipinterior}
    \fill[orange!20!white] (frame.south west) rectangle ([xshift=20pt]frame.north west);
    \end{tcbclipinterior}},
  #3,
}
\lstset{
    language=C,
    basicstyle=\ttfamily\small,
    keywordstyle=\color{blue},
    stringstyle=\color{orange},
    commentstyle=\color{green!60!black},
    numbers=left,
    numberstyle=\tiny\color{gray},
    breaklines=true,
    showstringspaces=false,
}
%------------------------------------------------------------
%This block of code defines the information to appear in the
%Title page
\title %optional
{1.2.11}
\date{August 21,2025}
%\subtitle{A short story}

\author % (optional)
{Megha Shyam-AI25BTECH11005}



\begin{document}


\frame{\titlepage}
\begin{frame}{Question}
 Find the slope of lines
\begin{enumerate}
    \item Passing through the points $(3, -2)$ and $(-1, 4)$
    \item Passing through the points $(3, -2)$ and $(7, -2)$
    \item Passing through the points $(3, -2)$ and $(3, 4)$
    \item Making inclination of $60^\circ$ with the positive direction of $x$-axis
\end{enumerate}
\end{frame}



\begin{frame}{Theoretical Solution}

We will use direction ratios. For two points \(P(x_1,y_1)\) and \(Q(x_2,y_2)\), a direction vector (column matrix) is
\[
\vec{d}=\begin{pmatrix}x_2-x_1\\[4pt]y_2-y_1\end{pmatrix}=\begin{pmatrix}l\\[4pt]m\end{pmatrix},
\]
so the direction ratios are \((l,m)\) and the slope is 
\[
\frac{m}{l}\quad (l\neq0).
\]

\begin{enumerate}
\item \(P(3,-2),\;Q(-1,4)\).
\[
\vec{d}=\begin{pmatrix}-1-3\\[4pt]4-(-2)\end{pmatrix}=\begin{pmatrix}-4\\[4pt]6\end{pmatrix}.
\]
Direction ratios \((l,m)=(-4,6)\). Thus the slope is
\[
m=\frac{6}{-4}=-\frac{3}{2}.
\]
\end{enumerate}
\end{frame}
\begin{frame}{Theoretical Solution}
\begin{enumerate}
    

\item \(P(3,-2),\;Q(7,-2)\).
\[
\vec{d}=\begin{pmatrix}7-3\\[4pt]-2-(-2)\end{pmatrix}=\begin{pmatrix}4\\[4pt]0\end{pmatrix}.
\]
Direction ratios \((l,m)=(4,0)\). Slope \(=\dfrac{0}{4}=0\). (horizontal line)

\item \(P(3,-2),\;Q(3,4)\).
\[
\vec{d}=\begin{pmatrix}3-3\\[4pt]4-(-2)\end{pmatrix}=\begin{pmatrix}0\\[4pt]6\end{pmatrix}.
\]
Direction ratios \((l,m)=(0,6)\). Here \(l=0\), so the slope is undefined (vertical line).
\end{enumerate}
\end{frame}

\begin{frame}{Theoretical Solution}
\begin{enumerate}
    

\item Line making inclination \(\theta=60^\circ\) with positive \(x\)-axis.

A unit direction vector for angle \(\theta\) is 
\(\begin{pmatrix}\cos\theta\\[4pt]\sin\theta\end{pmatrix}\).  
Thus direction ratios may be taken as
\[
\begin{pmatrix}l\\[4pt]m\end{pmatrix}=\begin{pmatrix}\cos60^\circ\\[4pt]\sin60^\circ\end{pmatrix}=\begin{pmatrix}\tfrac{1}{2}\\[4pt]\tfrac{\sqrt3}{2}\end{pmatrix},
\]
so the slope is
\[
m=\frac{\sin60^\circ}{\cos60^\circ}=\tan60^\circ=\sqrt{3}.
\]
\end{enumerate}


\end{frame}

\begin{frame}{Equation}
We will use direction ratios. For two points \(P(x_1,y_1)\) and \(Q(x_2,y_2)\), a direction vector (column matrix) is
\[
\vec{d}=\begin{pmatrix}x_2-x_1\\[4pt]y_2-y_1\end{pmatrix}=\begin{pmatrix}l\\[4pt]m\end{pmatrix},
\]
so the direction ratios are \((l,m)\) and the slope is 
\[
\frac{m}{l}\quad (l\neq0).
\]
\end{frame}




\begin{frame}{Plot}
 \begin{figure}[H]
     \centering
     \includegraphics[width=0.5\linewidth]{figs/fig1.png}
     \caption{fig1}
     \label{fig:placeholder}
 \end{figure}
     
    
\end{frame}


\end{document}