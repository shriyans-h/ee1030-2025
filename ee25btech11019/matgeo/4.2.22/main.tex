\let\negmedspace\undefined
\let\negthickspace\undefined
\documentclass[journal]{IEEEtran}
\usepackage[a5paper, margin=10mm, onecolumn]{geometry}
%\usepackage{lmodern} % Ensure lmodern is loaded for pdflatex
\usepackage{tfrupee} % Include tfrupee package

\setlength{\headheight}{1cm} % Set the height of the header box
\setlength{\headsep}{0mm}     % Set the distance between the header box and the top of the text

\usepackage{gvv-book}
%\usepackage{gvv}
\usepackage{cite}
\usepackage{amsmath,amssymb,amsfonts,amsthm}
\usepackage{algorithmic}
\usepackage{graphicx}
\usepackage{textcomp}
\usepackage{xcolor}
\usepackage{txfonts}
\usepackage{listings}
\usepackage{enumitem}
\usepackage{mathtools}
\usepackage{gensymb}
\usepackage{comment}
\usepackage[breaklinks=true]{hyperref}
\usepackage{tkz-euclide} 
\usepackage{listings}
\usepackage{gvv}                                        
\def\inputGnumericTable{}                                 
\usepackage[latin1]{inputenc}                                
\usepackage{color}                                            
\usepackage{array}                                            
\usepackage{longtable}                                       
\usepackage{calc}                                             
\usepackage{multirow}                                         
\usepackage{hhline}                                           
\usepackage{ifthen}                                           
\usepackage{lscape}
\begin{document}

\bibliographystyle{IEEEtran}

\title{4.2.22}
\author{EE25BTECH11019 - Darji Vivek M.}
{\let\newpage\relax\maketitle}

\renewcommand{\thefigure}{\theenumi}
\renewcommand{\thetable}{\theenumi}
\setlength{\intextsep}{10pt}
\numberwithin{figure}{enumi}
\renewcommand{\thetable}{\theenumi}
\textbf{Question}:\\
Show that the two lines
\[
a_1 x + b_1 y + c_1 = 0,\qquad a_2 x + b_2 y + c_2 = 0
\]
with \(b_1 b_2\neq 0\) are parallel iff \(\dfrac{a_1}{b_1}=\dfrac{a_2}{b_2}\).
\\
\solution

\begin{align}
&\text{Form the $2\times2$ coefficient matrix of normals: } 
\vec{M}=\myvec{a_1 & a_2\\ b_1 & b_2}.
\end{align}

Assume \(\dfrac{a_1}{b_1}=\dfrac{a_2}{b_2}\). Then there exists \(k\in\mathbb{R}\) such that
\begin{align}
a_2 = k\,a_1,\qquad b_2 = k\,b_1.
\end{align}

Write the rows of \textbf{M} as row vectors:
\begin{align}
\text{Row}_1 &= (a_1,\;a_2) = (a_1,\;k a_1) = a_1(1,\;k),\\
\text{Row}_2 &= (b_1,\;b_2) = (b_1,\;k b_1) = b_1(1,\;k).
\end{align}

Perform the row operation \( \text{Row}_2 \leftarrow \text{Row}_2 - \dfrac{b_1}{a_1}\,\text{Row}_1\) (assuming \(a_1\neq0\); if \(a_1=0\) use a symmetric argument swapping roles). Because Row\(_2\) is \( \dfrac{b_1}{a_1}\) times Row\(_1\), this operation yields the zero row:
\begin{align}
\text{Row}_2 &\mapsto \text{Row}_2 - \frac{b_1}{a_1}\text{Row}_1 = (0,0).
\end{align}

Thus the row-echelon form of \textbf{M} has exactly one nonzero row, so
\begin{align}
\operatorname{rank}(\vec{M})=1.
\end{align}

Rank \(1\) means the two column vectors (or equivalently the two normal vectors) are linearly dependent - i.e. collinear - hence the associated lines have the same slope and are parallel.

Conversely, if \(\operatorname{rank}(\vec{M})=1\) then the two rows (or columns) are proportional, which gives \(a_2 = k a_1\) and \(b_2 = k b_1\) for some \(k\), and therefore \(\dfrac{a_1}{b_1}=\dfrac{a_2}{b_2}\).

\begin{figure}[H]
\centering
\includegraphics[width=0.75\columnwidth]{figs/5.png}
\caption{\centering plot}
\label{fig:placeholder_125}
\end{figure}
\end{document}
