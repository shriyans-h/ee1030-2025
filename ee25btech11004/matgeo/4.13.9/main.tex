\documentclass[12pt]{article}
\usepackage{graphicx} % Required for inserting images
\usepackage{enumitem}
\usepackage{mathtools}
\usepackage{amsmath}
\usepackage{gvv-book}
\usepackage{gvv}

\title{\textbf{4.13.9}}
\author{\textbf{EE25BTECH11004 - Aditya Appana}}
\date{September 20, 2025}

\begin{document}

\maketitle

\section*{Question}
Let the algebraic sum of the perpendicular distances from the points (2,0),(0,2) and
(1,1) to a variable straight line be zero; then the line passes through a fixed point
whose coordinates are \rule{1.5cm}{0.15mm}.

\section*{Solution}

The normal form of a line is:
\begin{align}
\vec{n}^Tx = c
\end{align}\\
The perpendicular distance of a point from a line is:
\begin{align}
\frac{|\vec{n}^Tx - c|}{||n||}
\end{align}\\
It is given that the algebraic sum of the perpendicular distances of three points (2,0), (0,2), and (1,1) to a line $\vec{n}^Tx = c$ is 0. Therefore:

\begin{align}
\frac{\vec{n}^Tx_1 - c}{||n||} + \frac{\vec{n}^Tx_2 - c}{||n||} + \frac{\vec{n}^Tx_3 - c}{||n||} = 0
\end{align}\\
Substituting the points:

\begin{align}
\frac{\vec{n}^T\myvec{2\\0} - c}{||n||} + \frac{\vec{n}^T\myvec{0\\2} - c}{||n||} + \frac{\vec{n}^T\myvec{1\\1} - c}{||n||} = 0 \\
\frac{\vec{n}^T\myvec{3\\3} - 3c}{||n||} = 0 \\
\frac{\vec{n}^T\myvec{1\\1} - c}{||n||} = 0 \\
\vec{n}^T\myvec{1\\1} = c
\end{align}\\
Therefore the line passes through the fixed point (1,1).

\begin{figure}[H]
    \centering
    \includegraphics[width=0.7\columnwidth]{Figs/527.png}
    \caption{Plot}
    \label{fig:placeholder}
\end{figure}



\end{document}