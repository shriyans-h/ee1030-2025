\documentclass{beamer}
\mode<presentation>
\usepackage{amsmath}
\usepackage{amssymb}
%\usepackage{advdate}
\usepackage{adjustbox}
\usepackage{subcaption}
\usepackage{enumitem}
\usepackage{multicol}
\usepackage{mathtools}
\usepackage{listings}
\usepackage{url}
\def\UrlBreaks{\do\/\do-}
\usetheme{Boadilla}
\usecolortheme{lily}
\setbeamertemplate{footline}
{
  \leavevmode%
  \hbox{%
  \begin{beamercolorbox}[wd=\paperwidth,ht=2.25ex,dp=1ex,right]{author in head/foot}%
    \insertframenumber{} / \inserttotalframenumber\hspace*{2ex} 
  \end{beamercolorbox}}%
  \vskip0pt%
}
\setbeamertemplate{navigation symbols}{}

\providecommand{\nCr}[2]{\,^{#1}C_{#2}} % nCr
\providecommand{\nPr}[2]{\,^{#1}P_{#2}} % nPr
\providecommand{\mbf}{\mathbf}
\providecommand{\pr}[1]{\ensuremath{\Pr\left(#1\right)}}
\providecommand{\qfunc}[1]{\ensuremath{Q\left(#1\right)}}
\providecommand{\sbrak}[1]{\ensuremath{{}\left[#1\right]}}
\providecommand{\lsbrak}[1]{\ensuremath{{}\left[#1\right.}}
\providecommand{\rsbrak}[1]{\ensuremath{{}\left.#1\right]}}
\providecommand{\brak}[1]{\ensuremath{\left(#1\right)}}
\providecommand{\lbrak}[1]{\ensuremath{\left(#1\right.}}
\providecommand{\rbrak}[1]{\ensuremath{\left.#1\right)}}
\providecommand{\cbrak}[1]{\ensuremath{\left\{#1\right\}}}
\providecommand{\lcbrak}[1]{\ensuremath{\left\{#1\right.}}
\providecommand{\rcbrak}[1]{\ensuremath{\left.#1\right\}}}
\theoremstyle{remark}
\newtheorem{rem}{Remark}
\newcommand{\sgn}{\mathop{\mathrm{sgn}}}
\providecommand{\abs}[1]{\left\vert#1\right\vert}
\providecommand{\res}[1]{\Res\displaylimits_{#1}} 
\providecommand{\norm}[1]{\lVert#1\rVert}
\providecommand{\mtx}[1]{\mathbf{#1}}
\providecommand{\mean}[1]{E\left[ #1 \right]}
\providecommand{\fourier}{\overset{\mathcal{F}}{ \rightleftharpoons}}
%\providecommand{\hilbert}{\overset{\mathcal{H}}{ \rightleftharpoons}}
\providecommand{\system}[1]{\overset{\mathcal{#1}}{ \longleftrightarrow}}
%\providecommand{\system}{\overset{\mathcal{H}}{ \longleftrightarrow}}
	%\newcommand{\solution}[2]{\textbf{Solution:}{#1}}
%\newcommand{\solution}{\noindent \textbf{Solution: }}
\providecommand{\dec}[2]{\ensuremath{\overset{#1}{\underset{#2}{\gtrless}}}}
\newcommand{\myvec}[1]{\ensuremath{\begin{pmatrix}#1\end{pmatrix}}}
\newcommand{\mydet}[1]{\ensuremath{\begin{vmatrix}#1\end{vmatrix}}}
\let\vec\mathbf

\lstset{
%language=C,
frame=single, 
breaklines=true,
columns=fullflexible
}

\numberwithin{equation}{section}
\title{4.7.62}
\author{AI25BTECH11001 - ABHISEK MOHAPATRA}
% \maketitle
% \newpage
% \bigskip
\begin{document}
{\let\newpage\relax\maketitle}
\renewcommand{\thefigure}{\theenumi}
\renewcommand{\thetable}{\theenumi}


	 	\textbf{Question}:
If $\myvec{a &a^2 &1+ a3\\ b & b^2 & 1 + b^3\\c & c^2 & 1 + c^3} = 0$ and the vectors $\vec{A}=\myvec{1 & a & a^2}$, $\vec{B}=\myvec{1 & b & b^2}$, $\vec{C}=\myvec{1 & c & c^2}$ are co-planar, then the product abc = \rule{1cm}{0.15mm}.
		

		\textbf{Solution:} 
Let equation of the plane be $\vec{n}^\top\vec{x}=0$.

so, 
\begin{align}
\vec{n}^\top\vec{A}=0,
\vec{n}^\top\vec{B}=0,
\vec{n}^\top\vec{C}=0
\end{align}
so ,
\begin{align}
		\myvec{\vec{A}&\vec{B}&\vec{C}}^\top\vec{n}=0,
\end{align}
so for a unique plane to exist the rank of the matrix at left must be 3.Or, 
\begin{align}
		det\myvec{\vec{A}&\vec{B}&\vec{C}} \neq 0
\end{align}
\begin{align}
		\Rightarrow \mydet{1&a &a^2 \\1& b & b^2 \\1& c & c^2}	= 0
\end{align}

solving the given determinant,
\begin{align}
\mydet{a &a^2 &1+ a^3\\ b & b^2 & 1 + b^3\\c & c^2 & 1 + c^3} = 0
\end{align}
\begin{align}
\mydet{a &a^2 & a^3\\ b & b^2 & b^3\\c & c^2 & c^3} + \mydet{a &a^2 & 1\\ b & b^2 & 1\\c & c^2 & 1} = 0
\end{align}
\begin{align}
		\mydet{a &a^2 & a^3\\ b & b^2 & b^3\\c & c^2 & c^3} + \mydet{1&a &a^2 \\1& b & b^2 \\1& c & c^2} = 0
\end{align}
\begin{align}
		 abc\mydet{1&a &a^2 \\1& b & b^2 \\1& c & c^2}  + \mydet{1&a &a^2 \\1& b & b^2 \\1& c & c^2} = 0
\end{align}
\begin{align}
		\brak{abc+1}\mydet{1&a &a^2 \\1& b & b^2 \\1& c & c^2}	= 0
\end{align}
so,
\begin{align}
abc+1 = 0\Rightarrow
abc = -1
\end{align}


\end{document}
