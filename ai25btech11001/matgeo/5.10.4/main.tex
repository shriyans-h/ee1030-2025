\let\negmedspace\undefined
\let\negthickspace\undefined
\documentclass[journal,12pt,onecolumn]{IEEEtran}
\usepackage{cite}
\usepackage{amsmath,amssymb,amsfonts,amsthm}
\usepackage{algorithmic}
\usepackage{graphicx}
\graphicspath{{./figs/}}
\usepackage{textcomp}
\usepackage{xcolor}
\usepackage{txfonts}
\usepackage{listings}
\usepackage{enumitem}
\usepackage{mathtools}
\usepackage{gensymb}
\usepackage{comment}
\usepackage{caption}
\usepackage[breaklinks=true]{hyperref}
\usepackage{tkz-euclide} 
\usepackage{listings}
\usepackage{gvv}                                        
%\def\inputGnumericTable{}                                 
\usepackage[latin1]{inputenc}     
\usepackage{xparse}
\usepackage{color}                                            
\usepackage{array}                                            
\usepackage{longtable}                                       
\usepackage{calc}                                             
\usepackage{multirow}
\usepackage{multicol}
\usepackage{hhline}                                           
\usepackage{ifthen}                                           
\usepackage{lscape}
\usepackage{tabularx}
\usepackage{array}
\usepackage{float}

\begin{document}

\title{5.10.4}
\author{AI25BTECH11001 - ABHISEK MOHAPATRA}
{\let\newpage\relax\maketitle}
	
	 	\textbf{Question}:
		Write the balanced chemical equations for the following reaction.
		\begin{align}
				BaCl_2+K_2SO_4 \rightarrow BaSO_4+KCl
		\end{align}
		\textbf{Solution:}
		Let the balanced version of given equation be
		\begin{align}
				x_1BaCl_2+x_2K_2SO_4 \rightarrow x_3BaSO_4+x_4KCl
		\end{align}

		which results in the following equations:


		\begin{align}
				\brak{x_1-x_3}Ba = 0
		\end{align}
		\begin{align}
				\brak{2x_1-x_4}Cl = 0
		\end{align}
		\begin{align}
				\brak{2x_2-x_4}K = 0
		\end{align}
		\begin{align}
				\brak{x_2-x_3}S = 0
		\end{align}
		\begin{align}
				\brak{4x_2 - 4x_3}O = 0
		\end{align}
		which can be expresseed as
		\begin{align}
				x_1+ 0x_2 +(-1)x_3 +x_4= 0
		\end{align}
		\begin{align}
				2x_1+ 0x_2 +0x_3 +(-1)x_4= 0
		\end{align}
		\begin{align}
				0x_1+ 2x_2 +0x_3 +(-1)x_4= 0
		\end{align}
		\begin{align}
				0x_1+ x_2 +(-1)x_3 +0x_4= 0
		\end{align}
resulting in the matrix equation

		\begin{align}
				\myvec{1 & 0 &-1&0\\2&0&0&-1\\0&2&0&-1\\0&1&-1&0}\vec{X} = 0
		\end{align}
		which can be reduced as follows
		\begin{align}
				\myvec{
						1&0&-1&0\\
						2&0&0&-1\\
						0&2&0&-1\\
						0&1&-1&0
				}\xleftrightarrow[]{R_2\leftarrow R_2-R_1} 
				\myvec{
						1&0&-1&0\\
						0&0&2&-1\\
						0&2&0&-1\\
						0&1&-1&0}
		\end{align}
		\begin{align}
				\xleftrightarrow[]{R_3\leftrightarrow R_2} 
				\myvec{
						1&0&-1&0\\
						0&2&0&-1\\
						0&0&2&-1\\
						0&1&-1&0
				}\xleftrightarrow[]{R_4\leftarrow R_4+\frac{1}{2}R_2-\frac{1}{2}R_1} 
				\myvec{
						1&0&-1&0\\
						0&2&0&-1\\
						0&0&2&-1\\
						0&0&0&0}
		\end{align}
		\begin{align}
				\xleftrightarrow[]{R_1\leftrightarrow R_1+\frac{1}{2}R_3} 
				\myvec{
						1&0&0&-\frac{1}{2}\\
						0&2&0&-1\\
						0&0&2&-1\\
						0&0&0&0}
				\xleftrightarrow[R_3\leftrightarrow \frac{1}{2}R_3]{R_2\leftrightarrow \frac{1}{2}R_2} 
				\myvec{
						1&0&0&-\frac{1}{2}\\
						0&1&0&-\frac{1}{2}\\
						0&0&1&-\frac{1}{2}\\
						0&0&0&0}
		\end{align}
		Thus,
		\begin{align}
				x_1 = \frac{1}{2}x_4,x_2 = \frac{1}{2}x_4,x_3 = \frac{1}{2}x_4
		\end{align}
		\begin{align}
				\Rightarrow \vec{X} = x_4\myvec{\frac{1}{2}\\\frac{1}{2}\\\frac{1}{2}\\1} = \myvec{1\\1\\1\\2}
		\end{align}
		by substituting $x_4$ = 2. Hence, The equation finally becomes
		\begin{align}
				BaCl_2+K_2SO_4 \rightarrow BaSO_4+2KCl
		\end{align}

				


\end{document}



