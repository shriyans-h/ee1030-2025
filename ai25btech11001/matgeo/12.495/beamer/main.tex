
\documentclass{beamer}
\mode<presentation>
\usepackage{amsmath}
\usepackage{amssymb}
%\usepackage{advdate}
\usepackage{graphicx}
\graphicspath{{../figs/}}
\usepackage{adjustbox}
\usepackage{subcaption}
\usepackage{enumitem}
\usepackage{multicol}
\usepackage{mathtools}
\usepackage{listings}
\usepackage{url}
\def\UrlBreaks{\do\/\do-}
\usetheme{Boadilla}
\usecolortheme{lily}
\setbeamertemplate{footline}
{
  \leavevmode%
  \hbox{%
  \begin{beamercolorbox}[wd=\paperwidth,ht=2.25ex,dp=1ex,right]{author in head/foot}%
    \insertframenumber{} / \inserttotalframenumber\hspace*{2ex} 
  \end{beamercolorbox}}%
  \vskip0pt%
}
\setbeamertemplate{navigation symbols}{}
\let\solution\relax
\usepackage{gvv}
\lstset{
%language=C,
frame=single, 
breaklines=true,
columns=fullflexible
}

\numberwithin{equation}{section}
\title{12.495}
\author{AI25BTECH11001 - ABHISEK MOHAPATRA}
\begin{document}
{\let\newpage\relax\maketitle}
\renewcommand{\thefigure}{\theenumi}
\renewcommand{\thetable}{\theenumi}

	 	\textbf{Question}:
The matrix $\vec{P}$ is the inverse of a matrix $\vec{Q}$. If $\vec{I}$ denotes the identity matrix, which one of the following options is correct?

a) $\vec{P}\vec{Q} = \vec{I}$ but $\vec{Q}\vec{P} \neq \vec{I}$

b) $\vec{Q}\vec{P} = \vec{I}$ but $\vec{P}\vec{Q} \neq \vec{I}$

c) $\vec{P}\vec{Q} = \vec{I}$ and $\vec{Q}\vec{P} = \vec{I}$

d) $\vec{P}\vec{Q} - \vec{Q}\vec{P} = \vec{I}$

		\textbf{Solution:}
Let $\vec{P}$ is inverse of a matrix $\vec{Q}$ and $\vec{P}\vec{Q} = \vec{I}$

Let there exist $\vec{C}$ such that $\vec{Q}\vec{C} = \vec{I}$

\begin{align}
		\vec{Q}\vec{C} = \vec{I}
\end{align}
\begin{align}
		\Rightarrow \vec{P}\vec{Q}\vec{C} = \vec{P} \Rightarrow \vec{C} = \vec{P}
\end{align}

so $\vec{P}$$\vec{Q}$ = $\vec{Q}$$\vec{P}$ = $\vec{I}$.
So, option (c) is correct.




\end{document}




