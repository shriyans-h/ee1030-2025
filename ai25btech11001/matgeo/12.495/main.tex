\let\negmedspace\undefined
\let\negthickspace\undefined
\documentclass[journal,12pt,onecolumn]{IEEEtran}
\usepackage{cite}
\usepackage{amsmath,amssymb,amsfonts,amsthm}
\usepackage{algorithmic}
\usepackage{graphicx}
\graphicspath{{./figs/}}
\usepackage{textcomp}
\usepackage{xcolor}
\usepackage{txfonts}
\usepackage{listings}
\usepackage{enumitem}
\usepackage{mathtools}
\usepackage{gensymb}
\usepackage{comment}
\usepackage{caption}
\usepackage[breaklinks=true]{hyperref}
\usepackage{tkz-euclide} 
\usepackage{listings}
\usepackage{gvv}                                        
%\def\inputGnumericTable{}                                 
\usepackage[latin1]{inputenc}     
\usepackage{xparse}
\usepackage{color}                                            
\usepackage{array}                                            
\usepackage{longtable}                                       
\usepackage{calc}                                             
\usepackage{multirow}
\usepackage{multicol}
\usepackage{hhline}                                           
\usepackage{ifthen}                                           
\usepackage{lscape}
\usepackage{tabularx}
\usepackage{array}
\usepackage{float}

\begin{document}

\title{12.495}
\author{AI25BTECH11001 - ABHISEK MOHAPATRA}
{\let\newpage\relax\maketitle}
	
	 	\textbf{Question}:
The matrix $\vec{P}$ is the inverse of a matrix $\vec{Q}$. If $\vec{I}$ denotes the identity matrix, which one of the following options is correct?

a) $\vec{P}\vec{Q} = \vec{I}$ but $\vec{Q}\vec{P} \neq \vec{I}$

b) $\vec{Q}\vec{P} = \vec{I}$ but $\vec{P}\vec{Q} \neq \vec{I}$

c) $\vec{P}\vec{Q} = \vec{I}$ and $\vec{Q}\vec{P} = \vec{I}$

d) $\vec{P}\vec{Q} - \vec{Q}\vec{P} = \vec{I}$

		\textbf{Solution:}
Let $\vec{P}$ is inverse of a matrix $\vec{Q}$ and $\vec{P}\vec{Q} = \vec{I}$

Let there exist $\vec{C}$ such that $\vec{Q}\vec{C} = \vec{I}$

\begin{align}
		\vec{Q}\vec{C} = \vec{I}
\end{align}
\begin{align}
		\Rightarrow \vec{P}\vec{Q}\vec{C} = \vec{P} \Rightarrow \vec{C} = \vec{P}
\end{align}

so $\vec{P}$$\vec{Q}$ = $\vec{Q}$$\vec{P}$ = $\vec{I}$.
So, option (c) is correct.




\end{document}



