\let\negmedspace\undefined
\let\negthickspace\undefined
\documentclass[journal,12pt,onecolumn]{IEEEtran}
\usepackage{cite}
\usepackage{amsmath,amssymb,amsfonts,amsthm}
\usepackage{algorithmic}
\usepackage{graphicx}
\graphicspath{{./figs/}}
\usepackage{textcomp}
\usepackage{xcolor}
\usepackage{txfonts}
\usepackage{listings}
\usepackage{enumitem}
\usepackage{mathtools}
\usepackage{gensymb}
\usepackage{comment}
\usepackage{caption}
\usepackage[breaklinks=true]{hyperref}
\usepackage{tkz-euclide} 
\usepackage{listings}
\usepackage{gvv}                                        
%\def\inputGnumericTable{}                                 
\usepackage[latin1]{inputenc}     
\usepackage{xparse}
\usepackage{color}                                            
\usepackage{array}                                            
\usepackage{longtable}                                       
\usepackage{calc}                                             
\usepackage{multirow}
\usepackage{multicol}
\usepackage{hhline}                                           
\usepackage{ifthen}                                           
\usepackage{lscape}
\usepackage{tabularx}
\usepackage{array}
\usepackage{float}
%\newtheorem{theorem}{Theorem}[section]
%\newtheorem{theorem}{Theorem}[section]
%\newtheorem{problem}{Problem}
%\newtheorem{proposition}{Proposition}[section]
%\newtheorem{lemma}{Lemma}[section]
%\newtheorem{corollary}[theorem]{Corollary}
%\newtheorem{example}{Example}[section]
%\newtheorem{definition}[problem]{Definition}

\begin{document}

%\textbf{\Large 1.2.1} \\
%\textbf{\large AI25BTECH11001 - Abhisek Mohapatra} \\
\title{5.5.18}
\author{AI25BTECH11001 - ABHISEK MOHAPATRA}
% \maketitle
% \newpage
% \bigskip
%\begin{document}
{\let\newpage\relax\maketitle}
%\renewcommand{\thefigure}{\theenumi}
%\renewcommand{\thetable}{\theenumi}
	 	\textbf{Question}:
Find the inverse of the following matrix, using elementary transformations
\begin{align*}
		\myvec{2&3&1\\2&4&1\\3&7&2}
\end{align*}
		

		\textbf{Solution:}
		Given:
		\begin{align}
				\vec{A}\vec{A}^{-1} = \vec{I}
		\end{align}
		\begin{align}
				\myvec{2&3&1\\2&4&1\\3&7&2}\vec{A}^{-1} = \myvec{1&0&0\\0&1&0\\0&0&1}
		\end{align}
		Augumented Matrix:
		\begin{align}
				\augvec{3}{3}{
						2&3&1&1&0&0\\
						2&4&1&0&1&0\\
						3&7&2&0&0&1}
		\end{align}
		\begin{align}
				\xrightarrow[]{R_2\rightarrow R_2-R_1}\augvec{3}{3}{
						2&3&1&1&0&0\\
						0&1&0&-1&1&0\\
						3&7&2&0&0&1}
		\end{align}
		\begin{align}
				\xrightarrow[]{R_3\rightarrow R_3-\frac{3}{2}R_1}\augvec{3}{3}{
						2&3&1&1&0&0\\
						0&1&0&-1&1&0\\
						0&\frac{5}{2}&\frac{1}{2}&-\frac{3}{2}&0&1}
		\end{align}
		\begin{align}
				\xrightarrow[]{R_3\rightarrow R_3-\frac{5}{2}R_2}\augvec{3}{3}{
						2&3&1&1&0&0\\
						0&1&0&-1&1&0\\
						0&0&\frac{1}{2}&1&-\frac{5}{2}&1}
		\end{align}
		\begin{align}
				\xrightarrow[]{R_1\rightarrow R_1-3R_2-2R_3}\augvec{3}{3}{
						2&0&0&2&2&-2\\
						0&1&0&-1&1&0\\
						0&0&\frac{1}{2}&1&-\frac{5}{2}&1}
		\end{align}
		\begin{align}
				\xrightarrow[]{R_1\rightarrow R_1-3R_2-2R_3}\augvec{3}{3}{
						1&0&0&1&1&-1\\
						0&1&0&-1&1&0\\
						0&0&1&2&-5&2}
		\end{align}So,
		\begin{align}
				\vec{A}^{-1} = \myvec{1&1&-1\\-1&1&0\\2&-5&2}
		\end{align}

\end{document}

