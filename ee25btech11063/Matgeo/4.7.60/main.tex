\let\negmedspace\undefined
\let\negthickspace\undefined
\documentclass[journal]{IEEEtran}
\usepackage[a4paper, margin=10mm, onecolumn]{geometry}
%\usepackage{lmodern} % Ensure lmodern is loaded for pdflatex
\usepackage{tfrupee} % Include tfrupee package

\setlength{\headheight}{1cm} % Set the height of the header box
\setlength{\headsep}{0mm}  % Set the distance between the header box and the top of the text

\usepackage{gvv-book}
\usepackage{gvv}
\usepackage{cite}
\usepackage{amsmath,amssymb,amsfonts,amsthm}
\usepackage{algorithmic}
\usepackage{graphicx}
\usepackage{float}
\usepackage{textcomp}
\usepackage{xcolor}
\usepackage{txfonts}
\usepackage{listings}
\usepackage{enumitem}
\usepackage{mathtools}
\usepackage{gensymb}
\usepackage{comment}
\usepackage[breaklinks=true]{hyperref}
\usepackage{tkz-euclide} 
\usepackage{listings}
% \usepackage{gvv}                                        
\def\inputGnumericTable{}                                 
\usepackage[latin1]{inputenc}                                
\usepackage{color}                                            
\usepackage{array}                                            
\usepackage{longtable}                                       
\usepackage{calc}                                             
\usepackage{multirow}                                         
\usepackage{hhline}                                           
\usepackage{ifthen}                                           
\usepackage{lscape}
\usepackage{tikz}
\usetikzlibrary{patterns}

\begin{document}

\bibliographystyle{IEEEtran}
\vspace{3cm}

\title{4.7.60}
\author{ee25btech11063-vejith}

\maketitle
% \maketitle
% \newpage
% \bigskip
{\let\newpage\relax\maketitle}
\renewcommand{\thefigure}{\theenumi}
\renewcommand{\thetable}{\theenumi}
\setlength{\intextsep}{10pt} % Space between text and floats
\textbf{Question}:\\
Reduce the equation $\sqrt{3}$x+y-8=0 into normal form.Find the values of p and $\omega$.\\
\textbf{Solution}:\\
Given line equation is
\begin{align}
    \sqrt{3}x+y-8=0
\end{align} which can be written as 
\begin{align}
    \Vec{n}^T\Vec{x}=c\\
    \implies \brak{\sqrt{3}\hspace{0.5cm}1}\myvec{x\\y}=8
\end{align}
\begin{align}
    \Vec{n}=\myvec{\sqrt{3}\\1} \text{ and } \Vec{x}=\myvec{x\\y} \text{ and c}=8
\end{align}
Length (norm) of  $\Vec{n}$ is given as
\begin{align}
    \norm{\Vec{n}}=\sqrt{\Vec{n}^T\Vec{n}}=2.
\end{align}
The unit normal is given by
\begin{align}
    \hat{\Vec{n}}=\frac{\Vec{n}}{\norm{\Vec{n}}}=\myvec{\sqrt{3}/2 \\ 1/2}
\end{align}
Divide the line equation by $\norm{\Vec{n}}$ to get the normal form
\begin{align}
    \implies \Vec{n}^T\Vec{x}=4.\\
    \implies \brak{\frac{\sqrt{3}}{2} \hspace{0.5cm} \frac{1}{2}}\myvec{x\\y}=4.
    \end{align}
The standard form of line in normal form is given by 
\begin{align}
 \brak{\cos{\omega}\hspace{0.5cm}\sin{\omega}}\myvec{x\\y}=\text{p}.   
\end{align}
   On comparing equations (8) and (9) we get
   \begin{align}
       \text{p}=4 \text{ and }\omega=\frac{\pi}{6}
   \end{align}
   \begin{figure}[H]
    \centering
    \includegraphics[width=0.66\columnwidth]{figs/01.png}
    \label{fig-1}
\end{figure}
\end{document}
