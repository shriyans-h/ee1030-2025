\documentclass{beamer}
\mode<presentation>
\usepackage{amsmath,amssymb,mathtools}
\usepackage{textcomp}
\usepackage{gensymb}
\usepackage{adjustbox}
\usepackage{subcaption}
\usepackage{enumitem}
\usepackage{multicol}
\usepackage{listings}
\usepackage{url}
\usepackage{graphicx} % <-- needed for images
\def\UrlBreaks{\do\/\do-}

\usetheme{Boadilla}
\usecolortheme{lily}
\setbeamertemplate{footline}{
  \leavevmode%
  \hbox{%
  \begin{beamercolorbox}[wd=\paperwidth,ht=2ex,dp=1ex,right]{author in head/foot}%
    \insertframenumber{} / \inserttotalframenumber\hspace*{2ex}
  \end{beamercolorbox}}%
  \vskip0pt%
}
\setbeamertemplate{navigation symbols}{}

\lstset{
  frame=single,
  breaklines=true,
  columns=fullflexible,
  basicstyle=\ttfamily\tiny   % tiny font so code fits
}

\numberwithin{equation}{section}

% ---- your macros ----
\providecommand{\nCr}[2]{\,^{#1}C_{#2}}
\providecommand{\nPr}[2]{\,^{#1}P_{#2}}
\providecommand{\mbf}{\mathbf}
\providecommand{\pr}[1]{\ensuremath{\Pr\left(#1\right)}}
\providecommand{\qfunc}[1]{\ensuremath{Q\left(#1\right)}}
\providecommand{\sbrak}[1]{\ensuremath{{}\left[#1\right]}}
\providecommand{\lsbrak}[1]{\ensuremath{{}\left[#1\right.}}
\providecommand{\rsbrak}[1]{\ensuremath{\left.#1\right]}}
\providecommand{\brak}[1]{\ensuremath{\left(#1\right)}}
\providecommand{\lbrak}[1]{\ensuremath{\left(#1\right.}}
\providecommand{\rbrak}[1]{\ensuremath{\left.#1\right)}}
\providecommand{\cbrak}[1]{\ensuremath{\left\{#1\right\}}}
\providecommand{\lcbrak}[1]{\ensuremath{\left\{#1\right.}}
\providecommand{\rcbrak}[1]{\ensuremath{\left.#1\right\}}}
\theoremstyle{remark}
\newtheorem{rem}{Remark}
\newcommand{\sgn}{\mathop{\mathrm{sgn}}}
\providecommand{\abs}[1]{\left\vert#1\right\vert}
\providecommand{\res}[1]{\Res\displaylimits_{#1}}
\providecommand{\norm}[1]{\lVert#1\rVert}
\providecommand{\mtx}[1]{\mathbf{#1}}
\providecommand{\mean}[1]{E\left[ #1 \right]}
\providecommand{\fourier}{\overset{\mathcal{F}}{ \rightleftharpoons}}
\providecommand{\system}{\overset{\mathcal{H}}{ \longleftrightarrow}}
\providecommand{\dec}[2]{\ensuremath{\overset{#1}{\underset{#2}{\gtrless}}}}
\newcommand{\myvec}[1]{\ensuremath{\begin{pmatrix}#1\end{pmatrix}}}
\let\vec\mathbf

\title{Matgeo Presentation - Problem 4.7.60}
\author{ee25btech11063 - Vejith}

\begin{document}


\frame{\titlepage}
\begin{frame}{Question}
Reduce the equation $\sqrt{3}$x+y-8=0 into normal form.Find the values of p and $\omega$.
\end{frame}

\begin{frame}{Solution}
    Given line equation is
\begin{align}
    \sqrt{3}x+y-8=0
\end{align} which can be written as 
\begin{align}
    \Vec{n}^T\Vec{x}=c\\
    \implies \brak{\sqrt{3}\hspace{0.5cm}1}\myvec{x\\y}=8
\end{align}
\begin{align}
    \Vec{n}=\myvec{\sqrt{3}\\1} \text{ and } \Vec{x}=\myvec{x\\y} \text{ and c}=8
\end{align}
Length (norm) of  $\Vec{n}$ is given as
\begin{align}
    \norm{\Vec{n}}=\sqrt{\Vec{n}^T\Vec{n}}=2.
\end{align}
The unit normal is given by
\begin{align}
    \hat{\Vec{n}}=\frac{\Vec{n}}{\norm{\Vec{n}}}=\myvec{\sqrt{3}/2 \\ 1/2}
\end{align}
\end{frame}
\begin{frame}{Solution}
    Divide the line equation by $\norm{\Vec{n}}$ to get the normal form
\begin{align}
    \implies \Vec{n}^T\Vec{x}=4.\\
    \implies \brak{\frac{\sqrt{3}}{2} \hspace{0.5cm} \frac{1}{2}}\myvec{x\\y}=4.
    \end{align}
The standard form of line in normal form is given by 
\begin{align}
 \brak{\cos{\omega}\hspace{0.5cm}\sin{\omega}}\myvec{x\\y}=\text{p}.   
\end{align}
   On comparing equations (8) and (9) we get
   \begin{align}
       \text{p}=4 \text{ and }\omega=\frac{\pi}{6}
   \end{align}
\end{frame}

\begin{frame}{Plot}
    \begin{figure}[H]
    \centering
    \includegraphics[width=0.9\columnwidth]{figs/01.png}
    \label{fig-1}
\end{figure}
\end{frame}
% --------- CODE APPENDIX ---------
\section*{Appendix: Code}

% C program
\begin{frame}[fragile]{C Code: triangle.c}
\begin{lstlisting}[language=C]
#include <stdio.h>
#include <math.h>

int main() {
    FILE *fp;
    fp = fopen("norm.dat", "w");
    if (fp == NULL) {
        printf("Error opening file!\n");
        return 1;
    }

    // Line equation: sqrt(3)x + y - 8 = 0
    // Normal vector n = [sqrt(3), 1]
    double n[2] = {sqrt(3), 1.0};
    
    // Compute norm of n
    double norm = sqrt(n[0]*n[0] + n[1]*n[1]);

    // Unit normal (n cap)
    double n_cap[2];
    n_cap[0] = n[0] / norm;
    n_cap[1] = n[1] / norm;

    // Compute p = constant / norm
    double p = 8.0 / norm;

    // Compute angle omega = atan2(sin, cos)
    double omega = atan2(n_cap[1], n_cap[0]);  // in radians

    // Write results into file
    fprintf(fp, "Normal vector n = [%.4f, %.4f]\n", n[0], n[1]);
    fprintf(fp, "Norm of n = %.4f\n", norm);
    fprintf(fp, "Unit normal n_cap = [%.4f, %.4f]\n", n_cap[0], n_cap[1]);
   \end{lstlisting}
\end{frame}

% --------- CODE APPENDIX ---------
\section*{Appendix: Code}

% C program
\begin{frame}[fragile]{C Code: triangle.c}
\begin{lstlisting}[language=C]
 fprintf(fp, "Normal form: (%.4f)x + (%.4f)y = %.4f\n", n_cap[0], n_cap[1], p);
 fprintf(fp, "p = %.4f\n", p);
    fprintf(fp, "omega (radians) = %.4f\n", omega);
    fprintf(fp, "omega (degrees) = %.2f\n", omega * 180.0 / M_PI);

    fclose(fp);

    printf("Results written to norm.dat\n");
    return 0;
}
\end{lstlisting}
\end{frame}

\begin{frame}[fragile]{Python: plot.py}
\begin{lstlisting}[language=Python]
 import numpy as np
import matplotlib.pyplot as plt

# Line: sqrt(3)x + y - 8 = 0
# => y = -sqrt(3)x + 8
x = np.linspace(-2, 8, 400)
y = -np.sqrt(3) * x + 8

# Plot line
plt.plot(x, y, 'b', label=r'$\sqrt{3}x + y - 8 = 0$')

# Plot normal vector at the foot of perpendicular (p=4, omega=30°)
p = 4
omega = np.pi / 6  # 30 degrees
# Point on the line at perpendicular distance p from origin
x0 = p * np.cos(omega)
y0 = p * np.sin(omega)

# Draw perpendicular from origin
plt.plot([0, x0], [0, y0], 'r--', label='Normal (p=4, ω=30°)')
plt.scatter([x0], [y0], color='k')  # mark foot of perpendicular
plt.scatter([0], [0], color='g', label='Origin')

# Labels, legend, grid
plt.xlabel("x-axis")
plt.ylabel("y-axis")
plt.title("Line in Normal Form")
plt.legend()
plt.grid(True)
plt.axis("equal")
# Save figure
plt.savefig("line_normal_form.png", dpi=300)
plt.close()
\end{lstlisting}
\end{frame} 
\end{document}
