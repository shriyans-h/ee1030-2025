\let\negmedspace\undefined
\let\negthickspace\undefined
\documentclass[journal]{IEEEtran}
\usepackage[a4paper, margin=10mm, onecolumn]{geometry}
%\usepackage{lmodern} % Ensure lmodern is loaded for pdflatex
\usepackage{tfrupee} % Include tfrupee package

\setlength{\headheight}{1cm} % Set the height of the header box
\setlength{\headsep}{0mm}  % Set the distance between the header box and the top of the text

\usepackage{gvv-book}
\usepackage{gvv}
\usepackage{cite}
\usepackage{amsmath,amssymb,amsfonts,amsthm}
\usepackage{algorithmic}
\usepackage{graphicx}
\usepackage{float}
\usepackage{textcomp}
\usepackage{xcolor}
\usepackage{txfonts}
\usepackage{listings}
\usepackage{enumitem}
\usepackage{mathtools}
\usepackage{gensymb}
\usepackage{comment}
\usepackage[breaklinks=true]{hyperref}
\usepackage{tkz-euclide} 
\usepackage{listings}
% \usepackage{gvv}                                        
\def\inputGnumericTable{}                                 
\usepackage[latin1]{inputenc}                                
\usepackage{color}                                            
\usepackage{array}                                            
\usepackage{longtable}                                       
\usepackage{calc}                                             
\usepackage{multirow}                                         
\usepackage{hhline}                                           
\usepackage{ifthen}                                           
\usepackage{lscape}
\usepackage{tikz}
\usetikzlibrary{patterns}

\begin{document}

\bibliographystyle{IEEEtran}
\vspace{3cm}

\title{3.2.4}
\author{ee25btech11063-vejith}

\maketitle
% \maketitle
% \newpage
% \bigskip
{\let\newpage\relax\maketitle}
\renewcommand{\thefigure}{\theenumi}
\renewcommand{\thetable}{\theenumi}
\setlength{\intextsep}{10pt} % Space between text and floats
\textbf{Question}:\\
Construct the triangle BD$^{\prime}$C$^{\prime}$ similar to $\triangle$BDC with scale factor $\frac{4}{3}$.Draw the line segment D$^{\prime}$A$^{\prime}$. parallel to DA where A$^p$ prime lies on extended side BA.Is A$^{\prime}$BC$^{\prime}$D$^{\prime}$ a parallelogram?\\ 
\textbf{solution}
\begin{table}[h!]    
  \centering
  \begin{tabular}[12pt]{ |c| c|}
    \hline
      \textbf{Point} & \textbf{Name}\\ 
      \hline
      \myvec{0\\0} & Point A\\
      \hline
      \myvec{4\\0} & Point B\\
      \hline
      \myvec{4\\3} & Point C\\
      \hline
      \myvec{0\\3} &  Point D\\
      \hline
      \myvec{-4/3\\4} & Point D$^{\prime}$\\
      \hline
      \myvec{4\\4} & Point C$^{\prime}$\\
      \hline
      \myvec{-4/3\\0} & Point A$^{\prime}$\\
        \hline
\end{tabular}
  \caption{Variables Used}
  \label{}
\end{table}\\
consider $\triangle$BDC.constructs a $\triangle$BD$^{\prime}C^{\prime}$
 with scale factor $\frac{4}{3}$.\\
This means 
\begin{align}
  \triangle BD^{\prime}C^{\prime} \sim \triangle BDC.\\
\frac{BD^{\prime}}{BD} \;=\; \frac{BC^{\prime}}{BC} \;=\; \frac{D^{\prime}C^{\prime}}{DC} \;=\; \frac{4}{3}.
\end{align}

So D$^{\prime}$ lies on extension of BD and C$^{\prime}$.\\
\textbf{Construct A$^{\prime}$}\\
Draw D$^{\prime}$A$^{\prime}$ $\parallel$ DA with A$^{\prime}$ on extension of BA.\\ \\
\textbf{Check the parallelogram property}\\
1.By construction    D$^{\prime}$A$^{\prime}$ $\parallel$ DA.\\
But since DA $\parallel$ C$^{\prime}$B(by similarity of triangles),we get:
\begin{align}
    D{^\prime}A^{\prime} \parallel BC^{\prime}.
\end{align}
2.A$^{\prime}$ lies on extended BA,we have :
\begin{align}
    A^{\prime}B \parallel D^{\prime}C^{\prime}.
\end{align}
Thus:
\begin{align}
    A^{\prime}B \parallel D^{\prime}C^{\prime}.\\
D{^\prime}A^{\prime} \parallel BC^{\prime}.
\end{align}
so,opposite sides are parallel.\\
$\implies$A$^{\prime}$BC$^{\prime}$D$^{\prime}$ is a parallelogram
\\ \\ \\ \\ \\ \\ \\ \\ \\ \\ \\ \\ \\ \\ \\ 
\begin{figure}[H]
    \centering
    \includegraphics[width=1.0\columnwidth]{figs/01.png}
    \label{fig-1}
\end{figure}



\end{document}


