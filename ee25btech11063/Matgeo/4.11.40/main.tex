\let\negmedspace\undefined
\let\negthickspace\undefined
\documentclass[journal]{IEEEtran}
\usepackage[a4paper, margin=10mm, onecolumn]{geometry}
%\usepackage{lmodern} % Ensure lmodern is loaded for pdflatex
\usepackage{tfrupee} % Include tfrupee package

\setlength{\headheight}{1cm} % Set the height of the header box
\setlength{\headsep}{0mm}  % Set the distance between the header box and the top of the text

\usepackage{gvv-book}
\usepackage{gvv}
\usepackage{cite}
\usepackage{amsmath,amssymb,amsfonts,amsthm}
\usepackage{algorithmic}
\usepackage{graphicx}
\usepackage{float}
\usepackage{textcomp}
\usepackage{xcolor}
\usepackage{txfonts}
\usepackage{listings}
\usepackage{enumitem}
\usepackage{mathtools}
\usepackage{gensymb}
\usepackage{comment}
\usepackage[breaklinks=true]{hyperref}
\usepackage{tkz-euclide} 
\usepackage{listings}
% \usepackage{gvv}                                        
\def\inputGnumericTable{}                                 
\usepackage[latin1]{inputenc}                                
\usepackage{color}                                            
\usepackage{array}                                            
\usepackage{longtable}                                       
\usepackage{calc}                                             
\usepackage{multirow}                                         
\usepackage{hhline}                                           
\usepackage{ifthen}                                           
\usepackage{lscape}
\usepackage{tikz}
\usetikzlibrary{patterns}

\begin{document}

\bibliographystyle{IEEEtran}
\vspace{3cm}

\title{4.11.40}
\author{ee25btech11063-vejith}

\maketitle
% \maketitle
% \newpage
% \bigskip
{\let\newpage\relax\maketitle}
\renewcommand{\thefigure}{\theenumi}
\renewcommand{\thetable}{\theenumi}
\setlength{\intextsep}{10pt} % Space between text and floats
\textbf{Question}\\
Find the area of the region bounded by line y=3x+2, the X axis and the ordinates x=-2 and x=1.\\
\textbf{Solution:}\\
let
\begin{align}
    \vec{A}=\myvec{-2\\0}\\
    \vec{C}=\myvec{1\\0}\\
    \end{align}
let $\vec{D}$ \text{ and } $\vec{E}$ be the vectors on the line corresponding to x= -2 and x=1\\

Given line equation is 
\begin{align}
    -3x+y=2
\end{align}
 which can be expressed as
 \begin{align}
     \Vec{n}^T\Vec{x}=c\\
     \implies \brak{-3\hspace{0.5cm}1}\myvec{x\\y}=2
    \end{align}
 \begin{align}
    \Vec{n}=\myvec{-3\\1} \text{ and } \Vec{x}=\myvec{x\\y} \text{ and c}=2
\end{align}
let us find the vector $\Vec{D}$
\begin{align}
    \brak{-3\hspace{0.5cm}1}\myvec{-2\\y}=2 \hspace{0.5cm} \implies y=-4\\
    \implies \Vec{D}= \myvec{-2\\-4}
\end{align}
as y$<$0 we should find the $\Vec{B}$ where the line meets the x axis 
\begin{align}
    \brak{-3\hspace{0.5cm}1}\myvec{x\\0}=2 \hspace{0.5cm} \implies3x=-2\\
    \implies \Vec{B}= \myvec{\frac{-2}{3}\\0}
    \end{align}
    let us find the vector $\Vec{E}$
    \begin{align}
    \brak{-3\hspace{0.5cm}1}\myvec{1\\y}=2 \hspace{0.5cm} \implies y=5\\
    \implies \Vec{E}= \myvec{1\\5}
\end{align}
The area to be computed is area of $\triangle$EBC+area of $\triangle$ABD
\begin{align}
    ar(\triangle ABD)=\frac{1}{2}\norm{(A-B)\times(A-D)}\\
    =\frac{1}{2}\norm{\myvec{\frac{4}{3}}\times\myvec{0\\4}}=\frac{8}{3}\\
     ar(\triangle EBC)=\frac{1}{2}\norm{(E-c)\times(B-c)}\\
    =\frac{1}{2}\norm{\myvec{0\\5}\times\myvec{-5/3\\0}}=\frac{25}{6}\\
    \implies \text{area of the region is }=\frac{8}{3}+\frac{25}{6}=\frac{41}{6}
\end{align}
\\ \\ \\ \\ \\ \\
\begin{figure}[H]
    \centering
    \includegraphics[width=0.96\columnwidth]{figs/01.png}
    \label{fig-1}
\end{figure}
\end{document}
