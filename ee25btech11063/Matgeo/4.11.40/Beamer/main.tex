\documentclass{beamer}
\mode<presentation>
\usepackage{amsmath,amssymb,mathtools}
\usepackage{textcomp}
\usepackage{gensymb}
\usepackage{adjustbox}
\usepackage{subcaption}
\usepackage{enumitem}
\usepackage{multicol}
\usepackage{listings}
\usepackage{url}
\usepackage{graphicx} % <-- needed for images
\def\UrlBreaks{\do\/\do-}

\usetheme{Boadilla}
\usecolortheme{lily}
\setbeamertemplate{footline}{
  \leavevmode%
  \hbox{%
  \begin{beamercolorbox}[wd=\paperwidth,ht=2ex,dp=1ex,right]{author in head/foot}%
    \insertframenumber{} / \inserttotalframenumber\hspace*{2ex}
  \end{beamercolorbox}}%
  \vskip0pt%
}
\setbeamertemplate{navigation symbols}{}

\lstset{
  frame=single,
  breaklines=true,
  columns=fullflexible,
  basicstyle=\ttfamily\tiny   % tiny font so code fits
}

\numberwithin{equation}{section}

% ---- your macros ----
\providecommand{\nCr}[2]{\,^{#1}C_{#2}}
\providecommand{\nPr}[2]{\,^{#1}P_{#2}}
\providecommand{\mbf}{\mathbf}
\providecommand{\pr}[1]{\ensuremath{\Pr\left(#1\right)}}
\providecommand{\qfunc}[1]{\ensuremath{Q\left(#1\right)}}
\providecommand{\sbrak}[1]{\ensuremath{{}\left[#1\right]}}
\providecommand{\lsbrak}[1]{\ensuremath{{}\left[#1\right.}}
\providecommand{\rsbrak}[1]{\ensuremath{\left.#1\right]}}
\providecommand{\brak}[1]{\ensuremath{\left(#1\right)}}
\providecommand{\lbrak}[1]{\ensuremath{\left(#1\right.}}
\providecommand{\rbrak}[1]{\ensuremath{\left.#1\right)}}
\providecommand{\cbrak}[1]{\ensuremath{\left\{#1\right\}}}
\providecommand{\lcbrak}[1]{\ensuremath{\left\{#1\right.}}
\providecommand{\rcbrak}[1]{\ensuremath{\left.#1\right\}}}
\theoremstyle{remark}
\newtheorem{rem}{Remark}
\newcommand{\sgn}{\mathop{\mathrm{sgn}}}
\providecommand{\abs}[1]{\left\vert#1\right\vert}
\providecommand{\res}[1]{\Res\displaylimits_{#1}}
\providecommand{\norm}[1]{\lVert#1\rVert}
\providecommand{\mtx}[1]{\mathbf{#1}}
\providecommand{\mean}[1]{E\left[ #1 \right]}
\providecommand{\fourier}{\overset{\mathcal{F}}{ \rightleftharpoons}}
\providecommand{\system}{\overset{\mathcal{H}}{ \longleftrightarrow}}
\providecommand{\dec}[2]{\ensuremath{\overset{#1}{\underset{#2}{\gtrless}}}}
\newcommand{\myvec}[1]{\ensuremath{\begin{pmatrix}#1\end{pmatrix}}}
\let\vec\mathbf

\title{Matgeo Presentation - Problem 4.11.40}
\author{ee25btech11063 - Vejith}

\begin{document}


\frame{\titlepage}
\begin{frame}{Question}
Find the area of the region bounded by line y=3x+2, the X axis and the ordinates x=-2 and x=1.
\end{frame}

\begin{frame}{Solution}
   let
\begin{align}
    \vec{A}=\myvec{-2\\0}\\
    \vec{C}=\myvec{1\\0}\\
    \end{align}
let $\vec{D}$ \text{ and } $\vec{E}$ be the vectors on the line corresponding to x= -2 and x=1\\

Given line equation is 
\begin{align}
    -3x+y=2
\end{align}
 which can be expressed as
 \begin{align}
     \Vec{n}^T\Vec{x}=c\\
     \implies \brak{-3\hspace{0.5cm}1}\myvec{x\\y}=2\\
     \Vec{n}=\myvec{-3\\1} \text{ and } \Vec{x}=\myvec{x\\y} \text{ and c}=2
\end{align}
\end{frame}

\begin{frame}{Solution}
    let us find the vector $\Vec{D}$
\begin{align}
    \brak{-3\hspace{0.5cm}1}\myvec{-2\\y}=2 \hspace{0.5cm} \implies y=-4\\
    \implies \Vec{D}= \myvec{-2\\-4}
\end{align}
as y$<$0 we should find the $\Vec{B}$ where the line meets the x axis 
\begin{align}
    \brak{-3\hspace{0.5cm}1}\myvec{x\\0}=2 \hspace{0.5cm} \implies3x=-2\\
    \implies \Vec{B}= \myvec{\frac{-2}{3}\\0}
    \end{align}
    let us find the vector $\Vec{E}$
    \begin{align}
    \brak{-3\hspace{0.5cm}1}\myvec{1\\y}=2 \hspace{0.5cm} \implies y=5\\
    \implies \Vec{E}= \myvec{1\\5}
\end{align}
\end{frame}

\begin{frame}{Conclusion}
    The area to be computed is area of $\triangle$EBC+area of $\triangle$ABD
\begin{align}
    ar(\triangle ABD)=\frac{1}{2}\norm{(A-B)\times(A-D)}\\
    =\frac{1}{2}\norm{\myvec{\frac{4}{3}}\times\myvec{0\\4}}=\frac{8}{3}\\
     ar(\triangle EBC)=\frac{1}{2}\norm{(E-c)\times(B-c)}\\
    =\frac{1}{2}\norm{\myvec{0\\5}\times\myvec{-5/3\\0}}=\frac{25}{6}\\
    \implies \text{area of the region is }=\frac{8}{3}+\frac{25}{6}=\frac{41}{6}
\end{align} 
\end{frame}

\begin{frame}{Plot}
    \begin{figure}[H]
    \centering
    \includegraphics[width=0.9\columnwidth]{figs/01.png}
    \label{fig-1}
\end{figure}
\end{frame}

% --------- CODE APPENDIX ---------
\section*{Appendix: Code}

% C program
\begin{frame}[fragile]{C Code: area.c}
\begin{lstlisting}[language=C]
#include <stdio.h>
#include <stdlib.h>
#include <math.h>   // for fabs()

int main() {
    FILE *fp;
    double area;

    // Compute area = ∫ from -2 to 1 |3x+2| dx
    // Split into two parts: [-2, -2/3] and [-2/3, 1]

    double x0 = -2.0/3.0;

    // Antiderivative F(x) = (3/2)x^2 + 2x
    double F_at_x0 = (1.5 * x0 * x0) + (2.0 * x0);
    double F_at_neg2 = (1.5 * (-2) * (-2)) + (2.0 * (-2));
    double F_at_1 = (1.5 * 1 * 1) + (2.0 * 1);

    // Left interval [-2, -2/3] -> integrand is negative
    double left = -(F_at_x0 - F_at_neg2);

    // Right interval [-2/3, 1] -> integrand is positive
    double right = (F_at_1 - F_at_x0);

    area = left + right; // total geometric area

    // Open file area.dat
    fp = fopen("area.dat", "w");
    if (fp == NULL) {
        printf("Error opening file!\n");
        return 1;
    }
\end{lstlisting}
\end{frame}

% --------- CODE APPENDIX ---------
\section*{Appendix: Code}

% C program
\begin{frame}[fragile]{C Code: area.c}
\begin{lstlisting}[language=C]
    // Write output
    fprintf(fp, "Line equation in matrix form: [3  -1] [x y]^T = -2\n");
    fprintf(fp, "Area of the region = %.6f (exact = 41/6)\n", area);

    fclose(fp);

    printf("Area successfully written to area.dat\n");
    return 0;
}
\end{lstlisting}
\end{frame}

\begin{frame}[fragile]{Python: plot.py}
\begin{lstlisting}[language=Python]
 import matplotlib.pyplot as plt
import numpy as np

# Define line
x = np.linspace(-2, 1, 400)
y = 3*x + 2

# Vertices
A = (-2, 0)
B = (-2/3, 0)   # x-intercept
C = (1, 0)
D = (-2, -4)
E = (1, 5)

# Plot line
plt.plot(x, y, 'b-', label="y = 3x + 2")

# Plot x-axis
plt.axhline(0, color='black', linewidth=1)

# Shade the bounded area (polygon A-B-C-E-D-A)
polygon_x = [A[0], B[0], C[0], E[0], D[0], A[0]]
polygon_y = [A[1], B[1], C[1], E[1], D[1], A[1]]
plt.fill(polygon_x, polygon_y, color='orange', alpha=0.6, label="Bounded Area")

# Mark and label vertices (B shown symbolically)
vertices = {
    "A(-2,0)": A,
    "B(-2/3,0)": B,   # keep symbolic form
    "C(1,0)": C,
    "D(-2,-4)": D,
    "E(1,5)": E
}
\end{lstlisting}
\end{frame} 

\begin{frame}[fragile]{Python: plot.py}
\begin{lstlisting}[language=Python]
for name, (px, py) in vertices.items():
    plt.plot(px, py, 'ro')
    plt.text(px+0.05, py+0.2, name, fontsize=9)

# Annotate area
plt.text(-0.5, 1.5, "Area = 41/6", fontsize=12, color="red", weight="bold")

# Labels and formatting
plt.xlabel("x-axis")
plt.ylabel("y-axis")
plt.title("Bounded Region between y=3x+2 and x-axis (x∈[-2,1])")
plt.legend()
plt.grid(True)

# Save figure
plt.savefig("bounded_area_vertices.png", dpi=300)
plt.show()
\end{lstlisting}
\end{frame} 
\end{document}
