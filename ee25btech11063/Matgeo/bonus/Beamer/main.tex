\documentclass{beamer}
\mode<presentation>
\usepackage{amsmath,amssymb,mathtools}
\usepackage{textcomp}
\usepackage{gensymb}
\usepackage{adjustbox}
\usepackage{subcaption}
\usepackage{enumitem}
\usepackage{multicol}
\usepackage{listings}
\usepackage{url}
\usepackage{graphicx} % <-- needed for images
\def\UrlBreaks{\do\/\do-}

\usetheme{Boadilla}
\usecolortheme{lily}
\setbeamertemplate{footline}{
  \leavevmode%
  \hbox{%
  \begin{beamercolorbox}[wd=\paperwidth,ht=2ex,dp=1ex,right]{author in head/foot}%
    \insertframenumber{} / \inserttotalframenumber\hspace*{2ex}
  \end{beamercolorbox}}%
  \vskip0pt%
}
\setbeamertemplate{navigation symbols}{}

\lstset{
  frame=single,
  breaklines=true,
  columns=fullflexible,
  basicstyle=\ttfamily\tiny   % tiny font so code fits
}

\numberwithin{equation}{section}

% ---- your macros ----
\providecommand{\nCr}[2]{\,^{#1}C_{#2}}
\providecommand{\nPr}[2]{\,^{#1}P_{#2}}
\providecommand{\mbf}{\mathbf}
\providecommand{\pr}[1]{\ensuremath{\Pr\left(#1\right)}}
\providecommand{\qfunc}[1]{\ensuremath{Q\left(#1\right)}}
\providecommand{\sbrak}[1]{\ensuremath{{}\left[#1\right]}}
\providecommand{\lsbrak}[1]{\ensuremath{{}\left[#1\right.}}
\providecommand{\rsbrak}[1]{\ensuremath{\left.#1\right]}}
\providecommand{\brak}[1]{\ensuremath{\left(#1\right)}}
\providecommand{\lbrak}[1]{\ensuremath{\left(#1\right.}}
\providecommand{\rbrak}[1]{\ensuremath{\left.#1\right)}}
\providecommand{\cbrak}[1]{\ensuremath{\left\{#1\right\}}}
\providecommand{\lcbrak}[1]{\ensuremath{\left\{#1\right.}}
\providecommand{\rcbrak}[1]{\ensuremath{\left.#1\right\}}}
\theoremstyle{remark}
\newtheorem{rem}{Remark}
\newcommand{\sgn}{\mathop{\mathrm{sgn}}}
\providecommand{\abs}[1]{\left\vert#1\right\vert}
\providecommand{\res}[1]{\Res\displaylimits_{#1}}
\providecommand{\norm}[1]{\lVert#1\rVert}
\providecommand{\mtx}[1]{\mathbf{#1}}
\providecommand{\mean}[1]{E\left[ #1 \right]}
\providecommand{\fourier}{\overset{\mathcal{F}}{ \rightleftharpoons}}
\providecommand{\system}{\overset{\mathcal{H}}{ \longleftrightarrow}}
\providecommand{\dec}[2]{\ensuremath{\overset{#1}{\underset{#2}{\gtrless}}}}
\newcommand{\myvec}[1]{\ensuremath{\begin{pmatrix}#1\end{pmatrix}}}
\let\vec\mathbf

\title{Matgeo Presentation - Bonus Problem}
\author{ee25btech11063 - Vejith}

\begin{document}


\frame{\titlepage}
\begin{frame}{Question}
Given 3 vectors $\Vec{A}$,$\Vec{B}$,$\Vec{C}$ are coplanar then show det($\Vec{M}$) =0 where\\ $\Vec{M}$=($\Vec{A}$  $\Vec{B}$  $\Vec{C}$)
\end{frame}

\begin{frame}{Solution}
   Equation of plane through 3 coplanar points is 
\begin{align}
    \Vec{n}^T\Vec{x}=0\\
    \implies \Vec{n}^T\Vec{A}= \Vec{n}^T\Vec{B} = \Vec{n}^T\Vec{C}=0\\
    \Vec{M}=(\Vec{A} \hspace{0.5cm} \Vec{B} \hspace{0.5cm} \Vec{C})\\
    \implies \Vec{n}^T\Vec{M}=(\Vec{n}^T\Vec{A} \hspace{0.5cm}  \Vec{n}^T\Vec{B} \hspace{0.5cm} \Vec{n}^T\Vec{C})\\
    \implies \Vec{n}^T\Vec{M}=(0\hspace{0.5cm} 0 \hspace{0.5cm} 0)\\
    \implies \Vec{n}^T\Vec{M}=\Vec{0}
    \end{align}
From (0.6) it means $\Vec{M}$ has a non trivial vector in it$'s$ null space
\begin{align}
    \implies rank(\Vec{M})<3.
\end{align}

For a 3$\times$ 3  square matrix like $\Vec{M}$ if det($\Vec{M}$)$\neq$0 means $\Vec{M}$ is invertible which means $\Vec{M}$ is a full rank matrix\\ 
$\implies$ rank($\Vec{M}$)$=$3.\brak{\text{if det($\Vec{M}$)$\neq$0}}
\end{frame}


\begin{frame}{Solution}
From (0.7)  rank($\Vec{M}$)$<$3 \\
$\implies$ $\Vec{M}$ is not invertible\\
$\implies$ det($\Vec{M}$)$=$0 \\
    \textbf{proof 2}:\\
 3 vectors $\Vec{A}$,$\Vec{B}$,$\Vec{C}$ are coplanar means they are linearly dependent.\\
 let$'$s assume
 \begin{align}
     \Vec{C}=\alpha \Vec{A}+\beta \Vec{B}.\\
     \text{det}(\Vec{M})=\text{det}((\Vec{A} \hspace{0.5cm} \Vec{B} \hspace{0.5cm} \Vec{C})\\
 =\text{det}((\Vec{A} \hspace{0.5cm} \Vec{B} \hspace{0.5cm} \alpha \Vec{A}+\beta \Vec{B})\\
 =\alpha \text{det}((\Vec{A} \hspace{0.5cm} \Vec{B} \hspace{0.5cm} \Vec{A}) + \beta \text{det}((\Vec{A} \hspace{0.5cm} \Vec{B} \hspace{0.5cm} \Vec{B})=0\\
 \implies \text{det}(\Vec{M})=0
 \end{align}
\end{frame}
\end{document}
