\documentclass{beamer}
\mode<presentation>
\usepackage{amsmath,amssymb,mathtools}
\usepackage{textcomp}
\usepackage{gensymb}
\usepackage{adjustbox}
\usepackage{subcaption}
\usepackage{enumitem}
\usepackage{multicol}
\usepackage{listings}
\usepackage{url}
\usepackage{graphicx} % <-- needed for images
\def\UrlBreaks{\do\/\do-}

\usetheme{Boadilla}
\usecolortheme{lily}
\setbeamertemplate{footline}{
  \leavevmode%
  \hbox{%
  \begin{beamercolorbox}[wd=\paperwidth,ht=2ex,dp=1ex,right]{author in head/foot}%
    \insertframenumber{} / \inserttotalframenumber\hspace*{2ex}
  \end{beamercolorbox}}%
  \vskip0pt%
}
\setbeamertemplate{navigation symbols}{}

\lstset{
  frame=single,
  breaklines=true,
  columns=fullflexible,
  basicstyle=\ttfamily\tiny   % tiny font so code fits
}

\numberwithin{equation}{section}

% ---- your macros ----
\providecommand{\nCr}[2]{\,^{#1}C_{#2}}
\providecommand{\nPr}[2]{\,^{#1}P_{#2}}
\providecommand{\mbf}{\mathbf}
\providecommand{\pr}[1]{\ensuremath{\Pr\left(#1\right)}}
\providecommand{\qfunc}[1]{\ensuremath{Q\left(#1\right)}}
\providecommand{\sbrak}[1]{\ensuremath{{}\left[#1\right]}}
\providecommand{\lsbrak}[1]{\ensuremath{{}\left[#1\right.}}
\providecommand{\rsbrak}[1]{\ensuremath{\left.#1\right]}}
\providecommand{\brak}[1]{\ensuremath{\left(#1\right)}}
\providecommand{\lbrak}[1]{\ensuremath{\left(#1\right.}}
\providecommand{\rbrak}[1]{\ensuremath{\left.#1\right)}}
\providecommand{\cbrak}[1]{\ensuremath{\left\{#1\right\}}}
\providecommand{\lcbrak}[1]{\ensuremath{\left\{#1\right.}}
\providecommand{\rcbrak}[1]{\ensuremath{\left.#1\right\}}}
\theoremstyle{remark}
\newtheorem{rem}{Remark}
\newcommand{\sgn}{\mathop{\mathrm{sgn}}}
\providecommand{\abs}[1]{\left\vert#1\right\vert}
\providecommand{\res}[1]{\Res\displaylimits_{#1}}
\providecommand{\norm}[1]{\lVert#1\rVert}
\providecommand{\mtx}[1]{\mathbf{#1}}
\providecommand{\mean}[1]{E\left[ #1 \right]}
\providecommand{\fourier}{\overset{\mathcal{F}}{ \rightleftharpoons}}
\providecommand{\system}{\overset{\mathcal{H}}{ \longleftrightarrow}}
\providecommand{\dec}[2]{\ensuremath{\overset{#1}{\underset{#2}{\gtrless}}}}
\newcommand{\myvec}[1]{\ensuremath{\begin{pmatrix}#1\end{pmatrix}}}
\let\vec\mathbf

\title{Matgeo Presentation - Problem 12.388}
\author{ee25btech11063 - Vejith}

\begin{document}


\frame{\titlepage}
\begin{frame}{Question}
For the matrix $\Vec{A}$=$\begin{pmatrix}
    5 & 3\\
    1 & 3
\end{pmatrix}$,ONE of the normalized eigenvectors is given as \brak{\text{ME } 2012}\\
        a) \myvec{\frac{3}{2}\\ \frac{1}{2}}\\
        b) \myvec{\frac{1}{\sqrt{2}}\\ \frac{-1}{\sqrt{2}}}\\
        c) \myvec{\frac{3}{\sqrt{10}}\\ \frac{1}{{\sqrt{10}}}}\\
        d) \myvec{\frac{1}{\sqrt{5}}\\ \frac{2}{\sqrt{5}}}\\
\end{frame}

\begin{frame}{Solution}
    Given 
\begin{align}
    \vec{A}=\begin{pmatrix}
    5 & 3\\
    1 & 3
\end{pmatrix}
\end{align}
For matrix $\Vec{A}$ the characterstic polynomial is given by 
\begin{align}
    |\vec{A}-\lambda \vec{I}|=0
\end{align}
\begin{align}
    \text{char}\vec{A}=\left|
\begin{array}{cc}
5-\lambda & 3 \\[6pt]
1 & 3  -\lambda
\end{array}
\right|=0\\
\implies (5- \lambda)( 3- \lambda)-3=0\\
\implies \lambda ^2-8\lambda +12=0\\
\implies (\lambda -2)(\lambda -6)=0
\end{align}
Thus,the eigen values are given by  
\begin{align}
    \lambda_1=6 \text{ and } \lambda_2=2
\end{align}
\end{frame}

\begin{frame}{Solution}
For $\lambda_1$, the augmented matrix formed from the eigenvalue-eigenvector equation is 
\begin{align}
    \begin{pmatrix}
        -1 & 3\\
        1 & -3
    \end{pmatrix}  &\xleftrightarrow{R_2 \leftarrow R_2 + R_1} \begin{pmatrix}
        -1 & 3\\
        0 & 0
    \end{pmatrix}
\end{align}
Hence,the normalized eigenvector is
\begin{align}
    \vec{v_1}=\frac{1}{\sqrt{10}}\myvec{3\\1}
\end{align}

For $\lambda_2$, the augmented matrix formed from the eigenvalue-eigenvector equation is 
\begin{align}
    \begin{pmatrix}
        3 & 3\\
        1 & 1
    \end{pmatrix}  &\xleftrightarrow{R_2 \leftarrow R_2 - \frac{1}{3} \times R_1} \begin{pmatrix}
        3 & 3\\
        0 & 0
    \end{pmatrix}
\end{align}
Hence,the normalized eigenvector is
\begin{align}
    \vec{v_2}=\frac{1}{\sqrt{2}}\myvec{1\\-1}
\end{align}
\end{frame}

\begin{frame}{Conclusion}
The normalized eigen vectors are 
\begin{align}
    \vec{v_1}=\myvec{\frac{3}{\sqrt{10}}\\ \frac{1}{\sqrt{10}}} \text{ and }\vec{v_2}=\myvec{\frac{1}{\sqrt{2}}\\ \frac{-1}{\sqrt{2}}}
\end{align}
\end{frame}

% --------- CODE APPENDIX ---------
\section*{Appendix: Code}

% C program
\begin{frame}[fragile]{C Code: eigen.c}
\begin{lstlisting}[language=C]
#include <stdio.h>
#include <math.h>

int main() {
    FILE *fp;
    fp = fopen("eigen.dat", "w");
    if (fp == NULL) {
        printf("Error opening file!\n");
        return 1;
    }
    // Eigenvalues are already known: λ1=6, λ2=2
    // We’ll compute the corresponding eigenvectors.

    // Eigenvector for λ1 = 6  --> [3, 1]
    double v1x = 3, v1y = 1;
    double norm1 = sqrt(v1x*v1x + v1y*v1y);
    v1x /= norm1;
    v1y /= norm1;

    // Eigenvector for λ2 = 2  --> [1, -1]
    double v2x = 1, v2y = -1;
    double norm2 = sqrt(v2x*v2x + v2y*v2y);
    v2x /= norm2;
    v2y /= norm2;
    fprintf(fp, "Normalized Eigenvectors:\n");
    fprintf(fp, "For λ1 = 6 : ( %.6f , %.6f )\n", v1x, v1y);
    fprintf(fp, "For λ2 = 2 : ( %.6f , %.6f )\n", v2x, v2y);

    fclose(fp);

    printf("Eigenvectors written to eigen.dat\n");
    return 0;
}


\end{lstlisting}
\end{frame}

% Python plotting
\begin{frame}[fragile]{Python: solution.py}
\begin{lstlisting}[language=Python]
import numpy as np
from sympy import Matrix

# Define the matrix
A = np.array([[5, 3],
              [1, 3]], dtype=float)

# --- Using NumPy ---
# Eigen decomposition
eigenvalues, eigenvectors = np.linalg.eig(A)

print("Using NumPy:")
for i in range(len(eigenvalues)):
    vec = eigenvectors[:, i]
    # Normalize vector
    norm_vec = vec / np.linalg.norm(vec)
    print(f"Eigenvalue: {eigenvalues[i]:.0f}  --> Normalized eigenvector: {norm_vec}")

# --- Using SymPy (symbolic check) ---
M = Matrix([[5, 3],
            [1, 3]])

eigs = M.eigenvects()

print("\nUsing SymPy:")
for val, mult, vecs in eigs:
    for v in vecs:
        # Normalize with sympy
        v_normalized = v.normalized()
        print(f"Eigenvalue: {val}  --> Normalized eigenvector: {v_normalized}")


\end{lstlisting}
\end{frame}
\end{document}
