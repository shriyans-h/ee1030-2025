\let\negmedspace\undefined
\let\negthickspace\undefined
\documentclass[journal]{IEEEtran}
\usepackage[a5paper, margin=10mm, onecolumn]{geometry}
\usepackage{tfrupee} % Include tfrupee package

\setlength{\headheight}{1cm} % Set the height of the header box
\setlength{\headsep}{0mm}     % Set the distance between the header box and the top of the text

\usepackage{gvv-book}
\usepackage{gvv}
\usepackage{cite}
\usepackage{amsmath,amssymb,amsfonts,amsthm}
\usepackage{algorithmic}
\usepackage{graphicx}
\usepackage{textcomp}
\usepackage{xcolor}
\usepackage{txfonts}
\usepackage{listings}
\usepackage{enumitem}
\usepackage{mathtools}
\usepackage{gensymb}
\usepackage{comment}
\usepackage[breaklinks=true]{hyperref}
\usepackage{tkz-euclide} 
\usepackage{listings}
\def\inputGnumericTable{}                            
\usepackage[latin1]{inputenc}                            
\usepackage{color}                                  
\usepackage{array}                                  
\usepackage{longtable}                               
\usepackage{calc}                                   
\usepackage{multirow}                               
\usepackage{hhline}                                 
\usepackage{ifthen}                                 
\usepackage{lscape}
\begin{document}

\bibliographystyle{IEEEtran}
\vspace{3cm}

\title{2.7.15}
\author{EE25BTECH11003 - Adharvan Kshathriya Bommagani}
{\newpage\maketitle}

\renewcommand{\thefigure}{\theenumi}
\renewcommand{\thetable}{\theenumi}
\setlength{\intextsep}{10pt} % Space between text and floats
\textbf{Question}:\\
Find the volume of a parallelepiped whose edges are given by $-3\hat{i} + 7\hat{j} + 5\hat{k}$, $-5\hat{i} + 7\hat{j} - 3\hat{k}$ and $7\hat{i} - 5\hat{j} - 3\hat{k}$.
  
\bigskip

\textbf{Solution}:\\

Let $\vec{a}$, $\vec{b}$ and $\vec{c}$ be three vectors representing the edges of the given parallelepiped.
\begin{align}
    \vec{a} = \myvec{ - 3 \\ 7 \\ 5 },  
    \vec{b} = \myvec{ - 5 \\ 7 \\ - 3 },
     \vec{c} = \myvec{ 7 \\ -5 \\ -3 }
\end{align}

The volume is given by $V = \sqrt{\det(G)}$, where $G$ is the Gram matrix formed by the dot products of the vectors. Based on calculations, the Gram matrix is:
$$
G = \myvec{
83 & 49 & -71 \\
49 & 83 & -61 \\
-71 & -61 & 83
}
$$

We use \textbf{row reduction} to convert $G$ into an upper triangular matrix $U$.  
The determinant is unchanged by adding a multiple of one row to another.\\

\textbf{Step 1:}
\begin{align*}
\myvec{
83 & 49 & -71 \\
49 & 83 & -61 \\
-71 & -61 & 83
}
\xrightarrow{\;R_2 \to R_2 - \frac{49}{83}R_1\;}
\myvec{
83 & 49 & -71 \\
0 & \tfrac{4488}{83} & \tfrac{-1584}{83} \\
-71 & -61 & 83
}
\end{align*}


\textbf{Step 2:}
\begin{align*}
\myvec{
83 & 49 & -71 \\
0 & \tfrac{4488}{83} & \tfrac{-1584}{83} \\
-71 & -61 & 83
}
\xrightarrow{\;R_3 \to R_3 + \frac{71}{83}R_1\;}
\myvec{
83 & 49 & -71 \\
0 & \tfrac{4488}{83} & \tfrac{-1584}{83} \\
0 & \tfrac{-1584}{83} & \tfrac{1848}{83}
}
\end{align*}


\textbf{Step 3:}
\begin{align*}
\myvec{
83 & 49 & -71 \\
0 & \tfrac{4488}{83} & \tfrac{-1584}{83} \\
0 & \tfrac{-1584}{83} & \tfrac{1848}{83}
}
\xrightarrow{\;R_3 \to R_3 + \tfrac{6}{17}R_2\;}
\myvec{
83 & 49 & -71 \\
0 & \tfrac{4488}{83} & \tfrac{-1584}{83} \\
0 & 0 & \tfrac{264}{17}
}
\end{align*}
\newpage
The matrix is now upper triangular. The determinant of $G$ is the product of the diagonal entries of $U$.
\begin{align*}
\det(G) &= 83 \times \frac{4488}{83} \times \frac{264}{17} \\
&= 4488 \times \frac{264}{17} \\
&= (17 \times 264) \times \frac{264}{17} \\
&= 264 \times 264 = \mathbf{69696}
\end{align*}
\\
The volume is the square root of the determinant.
\begin{align*}
\text{Volume} &= \sqrt{\det(G)} = \sqrt{69696} \\
\text{Volume} &= \mathbf{264} \ \text{cubic units}
\end{align*}
\\


Therefore, the volume of the parallelepiped is 264 cubic units.\\


\textbf{Parallelopiped Defined by Vectors $\vec{a}$, $\vec{b}$ and $\vec{c}$ :}
\begin{figure}[h!]
    \centering
    \includegraphics[width=0.9\columnwidth]{figs/fig1.png}
    \caption{Figure for 2.7.15}
    \label{fig1}
\end{figure}



\end{document}
