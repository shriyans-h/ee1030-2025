\documentclass{beamer}
\usepackage[utf8]{inputenc}

\usetheme{Madrid}
\usecolortheme{default}
\usepackage{amsmath,amssymb,amsfonts,amsthm}
\usepackage{txfonts}
\usepackage{tkz-euclide}
\usepackage{listings}
\usepackage{adjustbox}
\usepackage{array}
\usepackage{tabularx}
\usepackage{gvv}
\usepackage{lmodern}
\usepackage{circuitikz}
\usepackage{tikz}
\usepackage{graphicx}
\usepackage{gensymb} % For using \degree symbol

\setbeamertemplate{page number in head/foot}[totalframenumber]

% Title Information
\title{2.7.15}
\date{September 29, 2025}
\author{ADHARVAN KSHATHRIYA BOMMAGANI - EE25BTECH11003}

\begin{document}

% Title Slide
\frame{\titlepage}

% Question Slide
\begin{frame}{Question}
Find the volume of a parallelepiped whose edges are given by:
\begin{align*}
\vec{a} = -3\hat{i} + 7\hat{j} + 5\hat{k}, \quad
\vec{b} = -5\hat{i} + 7\hat{j} - 3\hat{k}, \quad
\vec{c} = 7\hat{i} - 5\hat{j} - 3\hat{k}.
\end{align*}
\end{frame}

% Step 1: Represent vectors and Gram matrix
\begin{frame}{Theoretical Solution}
Let the three vectors representing the edges be:
\begin{align*}
\vec{a} = \myvec{-3 \\ 7 \\ 5}, \quad
\vec{b} = \myvec{-5 \\ 7 \\ -3}, \quad
\vec{c} = \myvec{7 \\ -5 \\ -3}.
\end{align*}

The volume is given by:
\begin{align*}
V = \sqrt{\det(G)},
\end{align*}
where \(G\) is the Gram matrix formed by dot products of the vectors.

The Gram matrix is:
\begin{align*}
G = 
\myvec{
83 & 49 & -71 \\
49 & 83 & -61 \\
-71 & -61 & 83
}
\end{align*}
\end{frame}

% Step 2: Row reduction - Step 1
\begin{frame}{Theoretical Solution}
We use row reduction to make \(G\) upper triangular.

Perform the operations:
\begin{align*}
\myvec{
83 & 49 & -71 \\
49 & 83 & -61 \\
-71 & -61 & 83
}
\xrightarrow{\textstyle R_2 \to R_2 - \frac{49}{83}R_1}
\myvec{
83 & 49 & -71 \\
0 & \tfrac{4488}{83} & \tfrac{-1584}{83} \\
-71 & -61 & 83
}
\end{align*}

\end{frame}
\begin{frame}{Theoretical Solution}

\begin{align*}
\myvec{
83 & 49 & -71 \\
0 & \tfrac{4488}{83} & \tfrac{-1584}{83} \\
-71 & -61 & 83
}
\xrightarrow{\textstyle R_3 \to R_3 + \frac{71}{83}R_1}
\myvec{
83 & 49 & -71 \\
0 & \tfrac{4488}{83} & \tfrac{-1584}{83} \\
0 & \tfrac{-1584}{83} & \tfrac{1848}{83}
}
\end{align*}


\end{frame}


% Step 3: Row reduction - Step 2
\begin{frame}{Theoretical Solution}

\begin{align*}
\myvec{
83 & 49 & -71 \\
0 & \tfrac{4488}{83} & \tfrac{-1584}{83} \\
0 & \tfrac{-1584}{83} & \tfrac{1848}{83}
}
\xrightarrow{\textstyle R_3 \to R_3 + \frac{6}{17}R_2}
\myvec{
83 & 49 & -71 \\
0 & \tfrac{4488}{83} & \tfrac{-1584}{83} \\
0 & 0 & \tfrac{264}{17}
}
\end{align*}

\end{frame}

% Step 4: Determinant calculation
\begin{frame}{Theoretical Solution}
The matrix is now upper triangular, so the determinant is the product of the diagonal entries:
\begin{align*}
\det(G) = 83 \times \frac{4488}{83} \times \frac{264}{17}.
\end{align*}

Simplifying step-by-step:
\begin{align*}
\det(G) = 4488 \times \frac{264}{17} = (17 \times 264) \times \frac{264}{17} = 264 \times 264 = 69696
\end{align*}

The volume is:
\begin{align*}
V = \sqrt{\det(G)} = \sqrt{69696} = \mathbf{264} \ \text{cubic units.}
\end{align*}
\end{frame}

% Step 5: Final figure
\begin{frame}{Plot}
\centering
\textbf{Parallelepiped defined by } $\vec{a}, \vec{b}, \vec{c}$
\begin{figure}[H]
    \centering
    \includegraphics[width=0.9\columnwidth]{figs/fig1.png}
    \caption{3D Representation of the Parallelepiped}
\end{figure}
\end{frame}

\end{document}
