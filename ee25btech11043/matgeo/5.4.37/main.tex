\documentclass[journal]{IEEEtran}
\usepackage[a5paper, margin=10mm, onecolumn]{geometry}
\usepackage{lmodern}
\usepackage{tfrupee}
\setlength{\headheight}{1cm}
\setlength{\headsep}{0mm}

\usepackage{gvv-book}
\usepackage{gvv}
\usepackage{cite}
\usepackage{amsmath,amssymb,amsfonts,amsthm}
\usepackage{algorithmic}
\usepackage{graphicx}
\usepackage{textcomp}
\usepackage{xcolor}
\usepackage{txfonts}
\usepackage{listings}
\usepackage{enumitem}
\usepackage{mathtools}
\usepackage{gensymb}
\usepackage{comment}
\usepackage[breaklinks=true]{hyperref}
\usepackage{tkz-euclide}
\usepackage{listings}
\def\inputGnumericTable{}
\usepackage[latin1]{inputenc}
\usepackage{color}
\usepackage{array}
\usepackage{longtable}
\usepackage{calc}
\usepackage{multirow}
\usepackage{hhline}
\usepackage{ifthen}
\usepackage{lscape}
\usepackage{xparse}

\bibliographystyle{IEEEtran}

\title{5.4.37}
\author{EE25BTECH11043 - Nishid Khandagre} 

\begin{document}
\maketitle

\renewcommand{\thefigure}{\theenumi}
\renewcommand{\thetable}{\theenumi}

\numberwithin{equation}{enumi}
\numberwithin{figure}{enumi}

\textbf{Question}:
Using elementary transformations, find the inverse of the following matrix


\textbf{Solution: }
Let the given matrix be $\vec{A}$:
\begin{align}
A = \myvec{
1 & 1 & -2 \\
2 & 1 & -3 \\
5 & 4 & -9}
\end{align}
To find $\vec{A}^{-1}$, we augment the matrix $\vec{A}$ with the identity matrix $\vec{I}$:
\begin{align}
\myvec{
        1 & 1 & -2 & \vrule & 1 & 0 & 0 \\
        2 & 1 & -3 & \vrule & 0 & 1 & 0 \\
        5 & 4 & -9 & \vrule & 0 & 0 & 1
    }
\end{align}
Apply elementary row operations:
\begin{align*}
R_2 &\rightarrow R_2 - 2R_1 \\
R_3 &\rightarrow R_3 - 5R_1 \\
\end{align*}
\begin{align}
\myvec{
1 & 1 & -2 & \vrule & 1 & 0 & 0 \\
0 & -1 & 1 & \vrule & -2 & 1 & 0 \\
0 & -1 & 1 & \vrule & -5 & 0 & 1
}
\end{align}
Then
\begin{align*}
R_3 &\rightarrow R_3 - R_2 \\
\end{align*}
\begin{align}
\myvec{
1 & 1 & -2 & \vrule & 1 & 0 & 0 \\
0 & -1 & 1 & \vrule & -2 & 1 & 0 \\
0 & 0 & 0 &  \vrule & -3 & -1 & 1
}
\end{align}

Since in the left block the last row is all zeros $\myvec{0 & 0 & 0}$. So the left block cannot be\\ converted into Identity matrix.\\\\
Also the left block has rank$<$3, So the left block is singular.\\ \\
Therefore inverse of matrix does not exist.

\end{document}
