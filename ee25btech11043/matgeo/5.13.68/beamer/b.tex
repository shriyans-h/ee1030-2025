\documentclass{beamer}
\usepackage[utf8]{inputenc}

\usetheme{Madrid}
\usecolortheme{default}
\usepackage{amsmath,amssymb,amsfonts,amsthm}
\usepackage{txfonts}
\usepackage{tkz-euclide}
\usepackage{listings}
\usepackage{adjustbox}
\usepackage{array}
\usepackage{tabularx}
\usepackage{gvv}
\usepackage{lmodern}
\usepackage{circuitikz}
\usepackage{tikz}
\usepackage{graphicx}

\setbeamertemplate{page number in head/foot}[totalframenumber]

\usepackage{tcolorbox}
\tcbuselibrary{minted,breakable,xparse,skins}

\definecolor{bg}{gray}{0.95}
\DeclareTCBListing{mintedbox}{O{}m!O{}}{%
breakable=true,
listing engine=minted,
listing only,
minted language=#2,
minted style=default,
minted options={%
linenos,
gobble=0,
breaklines=true,
breakafter=,,
fontsize=\small,
numbersep=8pt,
#1},
boxsep=0pt,
left skip=0pt,
right skip=0pt,
left=25pt,
right=0pt,
top=3pt,
bottom=3pt,
arc=5pt,
leftrule=0pt,
rightrule=0pt,
bottomrule=2pt,
toprule=2pt,
colback=bg,
colframe=orange!70,
enhanced,
overlay={%
\begin{tcbclipinterior}
\fill[orange!20!white] (frame.south west) rectangle ([xshift=20pt]frame.north west);
\end{tcbclipinterior}},
#3,
}
\lstset{
language=C,
basicstyle=\ttfamily\small,
keywordstyle=\color{blue},
stringstyle=\color{orange},
commentstyle=\color{green!60!black},
numbers=left,
numberstyle=\tiny\color{gray},
breaklines=true,
showstringspaces=false,
}

\title
{5.13.68}
\date{October 3 , 2025}
\author
{EE25BTECH11043 - Nishid Khandagre}

\begin{document}

\frame{\titlepage}

\begin{frame}{Question}
For what value of $k$ do the following system of equations possess a non trivial solution over the set of rationals Q?
\begin{align*}
x + ky + 3z &= 0 \\
3x + ky - 2z &= 0 \\
2x + 3y - 4z &= 0
\end{align*}
For that value of $k$, find all the solutions of the system.
\end{frame}

\begin{frame}{Theoretical Solution}
The given system of equations can be written in augmented matrix form as:
\begin{align}
\myvec{
1 & k & 3 & \vrule & 0 \\
3 & k & -2 & \vrule & 0 \\
2 & 3 & -4 & \vrule & 0
}
\end{align}
Apply Gaussian elimination to find the row echelon form.
First, perform row operations: $R_2 \rightarrow R_2 - 3R_1$ and $R_3 \rightarrow R_3 - 2R_1$.
\begin{align}
\myvec{
1 & k & 3 & \vrule & 0 \\
0 & k - 3k & -2 - 9 & \vrule & 0 \\
0 & 3 - 2k & -4 - 6 & \vrule & 0
}
\end{align}
This simplifies to:
\begin{align}
\myvec{
1 & k & 3 & \vrule & 0 \\
0 & -2k & -11 & \vrule & 0 \\
0 & 3 - 2k & -10 & \vrule & 0
}
\end{align}
\end{frame}

\begin{frame}{Theoretical Solution}
For a non-trivial solution, the rank of the coefficient matrix must be less than 3.
If $-2k = 0$, then $k=0$. In this case, the matrix becomes:
\begin{align}
\myvec{
1 & 0 & 3 & \vrule & 0 \\
0 & 0 & -11 & \vrule & 0 \\
0 & 3 & -10 & \vrule & 0
}
\end{align}
This matrix has rank 3, which would lead to only the trivial solution. So $k \neq 0$.\\
If $k \neq 0$, we can proceed
$R_2$: $R_2 \rightarrow -R_2$:
\begin{align}
\myvec{
1 & k & 3 & \vrule & 0 \\
0 & 2k & 11 & \vrule & 0 \\
0 & 3 - 2k & -10 & \vrule & 0
}
\end{align}
\end{frame}

\begin{frame}{Theoretical Solution}
$R_3 \rightarrow 2k R_3 - (3-2k) R_2$:
\begin{align}
\myvec{
1 & k & 3 & \vrule & 0 \\
0 & 2k & 11 & \vrule & 0 \\
0 & 0 & 2k - 33 & \vrule & 0
}
\end{align}

For a non-trivial solution, the rank of matrix must be less than 3,therefore
\begin{align}
2k - 33 &= 0 \\
2k &= 33 \\
k &= \frac{33}{2}
\end{align}
\end{frame}

\begin{frame}{Theoretical Solution}
Now, find the solutions for $k = \frac{33}{2}$. The augmented matrix is:
\begin{align}
\myvec{
1 & 33/2 & 3 & \vrule & 0 \\
3 & 33/2 & -2 & \vrule & 0 \\
2 & 3 & -4 & \vrule & 0
}
\end{align}
Perform $R_2 \rightarrow R_2 - 3R_1$:
\begin{align}
\myvec{
1 & 33/2 & 3 & \vrule & 0 \\
0 & -33 & -11 & \vrule & 0 \\
2 & 3 & -4 & \vrule & 0
}
\end{align}
Perform $R_3 \rightarrow R_3 - 2R_1$:
\begin{align}
\myvec{
1 & 33/2 & 3 & \vrule & 0 \\
0 & -33 & -11 & \vrule & 0 \\
0 & -30 & -10 & \vrule & 0
}
\end{align}
\end{frame}

\begin{frame}{Theoretical Solution}
$R_2 \rightarrow R_2 / (-11)$ and $R_3 \rightarrow R_3 / (-10)$:
\begin{align}
\myvec{
1 & 33/2 & 3 & \vrule & 0 \\
0 & 3 & 1 & \vrule & 0 \\
0 & 3 & 1 & \vrule & 0
}
\end{align}
Perform $R_3 \rightarrow R_3 - R_2$:
\begin{align}
\myvec{
1 & 33/2 & 3 & \vrule & 0 \\
0 & 3 & 1 & \vrule & 0 \\
0 & 0 & 0 & \vrule & 0
}
\end{align}
The rank of the coefficient matrix is 2, which is less than 3, so there are non-trivial solutions.\\
From the second row: $3y + z = 0 \Rightarrow z = -3y$.\\
From the first row: $x + \frac{33}{2}y + 3z = 0$.
Substitute $z = -3y$:
$x = -\frac{15}{2}y$.
\end{frame}

\begin{frame}{Theoretical Solution}
Let $y = 2t$ 
Then $x = -\frac{15}{2}(2t) = -15t$.
And $z = -3(2t) = -6t$



The solutions are of the form:
$\myvec{x \\ y \\ z} = t \myvec{-15 \\ 2 \\ -6}$ for any $t \in \mathbb{Q}$.
\end{frame}

\begin{frame}[fragile]
\frametitle{C Code}
\begin{lstlisting}
#include <stdio.h>

// Function to calculate the determinant of the 3x3 matrix
// | 1  k   3 |
// | 3  k  -2 |
// | 2  3  -4 |
double calculateDeterminant(double k_val) {
    double det = 0.0;
    det = (1.0 * (k_val * -4.0 - (-2.0 * 3.0))) -
          (k_val * (3.0 * -4.0 - (-2.0 * 2.0))) +
          (3.0 * (3.0 * 3.0 - k_val * 2.0));
    return det;
}
\end{lstlisting}
\end{frame}

\begin{frame}[fragile]
\frametitle{C Code}
\begin{lstlisting}
// Function to solve the system for a given k when det = 0
// It will express x, y in terms of z
// This function assumes a non-trivial solution exists (det = 0)
// For k = 33/2
// So, the solutions are of the form (5/2 * z, -1/3 * z, z) for any rational z.
void solveSystem(double k_val, double* x_coeff_z, double* y_coeff_z) {
    if (k_val == 33.0 / 2.0) { // Check if k is the correct value
        *x_coeff_z = 5.0 / 2.0;
        *y_coeff_z = -1.0 / 3.0;
    } else {
        // Handle cases where k is not the value that makes det=0,
        *x_coeff_z = 0.0;
        *y_coeff_z = 0.0;
    }
}
\end{lstlisting}
 \end{frame}

\begin{frame}[fragile]
\frametitle{Python Code through shared output}
\begin{lstlisting}
import ctypes
import numpy as np

# Load the shared library
lib_code = ctypes.CDLL("./code13.so")

# --- Part 1: Find k for non-trivial solution ---

# Define argument types and return type for calculateDeterminant
lib_code.calculateDeterminant.argtypes = [ctypes.c_double]
lib_code.calculateDeterminant.restype = ctypes.c_double

# Find k by iterating and checking the determinant
test_k_values = [16.0, 16.4, 16.5, 16.6, 17.0]
k_for_nontrivial = None
\end{lstlisting}
 \end{frame}

\begin{frame}[fragile]
\frametitle{Python Code through shared output}
\begin{lstlisting}
print("Testing determinant for various k values:")
for k_val in test_k_values:
    det = lib_code.calculateDeterminant(k_val)
    print(f"  For k = {k_val:.2f}, Determinant = {det:.2f}")
    if abs(det) < 1e-9: # A small epsilon for floating point comparison
        k_for_nontrivial = k_val
        break # Found k

# If not found directly, we know k = 33/2 = 16.5 analytically.
if k_for_nontrivial is None:
    k_for_nontrivial = 33.0 / 2.0
    print(f"\nAnalytically determined k for non-trivial solution: {k_for_nontrivial:.2f}")

print(f"\nValue of k for which the system possesses a non-trivial solution: k = {k_for_nontrivial:.2f}")
\end{lstlisting}
 \end{frame}

\begin{frame}[fragile]
\frametitle{Python Code through shared output}
\begin{lstlisting}
# --- Part 2: Find all solutions for that value of k ---
# Define argument types and return type for solveSystem
lib_code.solveSystem.argtypes = [
    ctypes.c_double,
    ctypes.POINTER(ctypes.c_double), # x_coeff_z
    ctypes.POINTER(ctypes.c_double)  # y_coeff_z
]
lib_code.solveSystem.restype = None

# Create ctypes doubles to hold the coefficients
x_coeff_z_result = ctypes.c_double()
y_coeff_z_result = ctypes.c_double()
# Call the C function to get the coefficients
lib_code.solveSystem(
    k_for_nontrivial,
    ctypes.byref(x_coeff_z_result),
    ctypes.byref(y_coeff_z_result)
)
\end{lstlisting}
 \end{frame}

\begin{frame}[fragile]
\frametitle{Python Code through shared output}
\begin{lstlisting}
x_coeff = x_coeff_z_result.value
y_coeff = y_coeff_z_result.value

print(f"\nFor k = {k_for_nontrivial:.2f}, the solutions are of the form:")
print(f"x = {x_coeff:.3f} * z")
print(f"y = {y_coeff:.3f} * z")
print(f"z = z (where z can be any rational number)")
print("\nThis means the solution set is a subspace spanned by the vector:")
print(f"Solution vector: ({x_coeff:.3f}, {y_coeff:.3f}, 1)")
# Example non-trivial solution (let z = 6, to get integer values for x and y based on the derived coeffs)
if k_for_nontrivial == 33.0 / 2.0:
    print("\nExample non-trivial solution (let z = 6 to get integers):")
    example_z = 6
    example_x = x_coeff * example_z
    example_y = y_coeff * example_z
\end{lstlisting}
 \end{frame}

\begin{frame}[fragile]
\frametitle{Python Code through shared output}
\begin{lstlisting}
    print(f"  Solution: ({example_x:.0f}, {example_y:.0f}, {example_z:.0f})")
    print("\nLet's verify this solution with the original equations (with k=33/2):")
    print(f"  1({example_x}) + {k_for_nontrivial}({example_y}) + 3({example_z}) = {1*example_x + k_for_nontrivial*example_y + 3*example_z}")
    print(f"  3({example_x}) + {k_for_nontrivial}({example_y}) - 2({example_z}) = {3*example_x + k_for_nontrivial*example_y - 2*example_z}")
    print(f"  2({example_x}) + 3({example_y}) - 4({example_z}) = {2*example_x + 3*example_y - 4*example_z}")
\end{lstlisting}
 \end{frame}

\begin{frame}[fragile]
\frametitle{Python Code: Direct}
\begin{lstlisting}
import numpy as np
import numpy.linalg as LA
import matplotlib.pyplot as plt

def calculate_determinant(k_val):
    """
    Calculates the determinant of the coefficient matrix for a given k.
    Matrix:
    | 1  k   3 |
    | 3  k  -2 |
    | 2  3  -4 |
    """
    det = (1 * (k_val * -4 - (-2 * 3))) - \
          (k_val * (3 * -4 - (-2 * 2))) + \
          (3 * (3 * 3 - k_val * 2))
    return det
\end{lstlisting}
 \end{frame}

\begin{frame}[fragile]
\frametitle{Python Code: Direct}
\begin{lstlisting}
def solve_system_coefficients(k_val):
    """
    Solves the system for x and y in terms of z when det(A) = 0.
    This uses the analytical derivation. For a general solver,
    one would implement Gaussian elimination or a similar method on the matrix.
    Assumes k_val is such that a non-trivial solution exists.
    """
    if k_val == 33.0 / 2.0: # Check if k is the correct value
        x_coeff_z = 5.0 / 2.0
        y_coeff_z = -1.0 / 3.0
        return x_coeff_z, y_coeff_z
    else:
        # If k is not the value for non-trivial solutions,
        # For homogeneous system, if det != 0, only trivial solution.
        return 0.0, 0.0
\end{lstlisting}
 \end{frame}

\begin{frame}[fragile]
\frametitle{Python Code: Direct}
\begin{lstlisting}
# --- Part 1: Find k for non-trivial solution ---
# Analytically determined k
k_for_nontrivial = 33.0 / 2.0

# Verify the determinant for this k
det_at_k = calculate_determinant(k_for_nontrivial)
print(f"For k = {k_for_nontrivial:.2f}, the determinant is: {det_at_k:.6f} (should be close to zero)")
if abs(det_at_k) < 1e-9:
    print("Determinant is effectively zero, confirming k for non-trivial solution.")
else:
    print("Warning: Determinant is not zero for this k. Check calculations.")
    exit()

print(f"\nValue of k for which the system possesses a non-trivial solution: k = {k_for_nontrivial:.2f}")
\end{lstlisting}
 \end{frame}

\begin{frame}[fragile]
\frametitle{Python Code: Direct}
\begin{lstlisting}
# --- Part 2: Find all solutions for that value of k ---
x_coeff, y_coeff = solve_system_coefficients(k_for_nontrivial)

print(f"\nFor k = {k_for_nontrivial:.2f}, the solutions are of the form:")
print(f"x = {x_coeff:.6f} * z")
print(f"y = {y_coeff:.6f} * z")
print(f"z = z (where z can be any rational number)")
print("\nThis means the solution set is a subspace spanned by the vector:")
print(f"Solution basis vector: ({x_coeff:.6f}, {y_coeff:.6f}, 1)")

# Example non-trivial solution (let z = 6, to get integer values for x and y based on the derived coeffs)
print("\nExample non-trivial solution (let z = 6):")
example_z = 6
example_x = x_coeff * example_z
example_y = y_coeff * example_z
\end{lstlisting}
 \end{frame}

\begin{frame}[fragile]
\frametitle{Python Code: Direct}
\begin{lstlisting}
print(f"  If z = {example_z}, then x = {example_x:.0f}, y = {example_y:.0f}")
print(f"  Solution: ({example_x:.0f}, {example_y:.0f}, {example_z:.0f})")
print("\nLet's verify this solution with the original equations (with k=33/2):")
k_val_for_verify = 33.0 / 2.0
eq1_result = (1 * example_x) + (k_val_for_verify * example_y) + (3 * example_z)
eq2_result = (3 * example_x) + (k_val_for_verify * example_y) - (2 * example_z)
eq3_result = (2 * example_x) + (3 * example_y) - (4 * example_z)

print(f"  x + ky + 3z = {eq1_result:.6f} (should be 0)")
print(f"  3x + ky - 2z = {eq2_result:.6f} (should be 0)")
print(f"  2x + 3y - 4z = {eq3_result:.6f} (should be 0)")
\end{lstlisting}
 \end{frame}

\end{document}