\documentclass[journal]{IEEEtran}
\usepackage[a5paper, margin=10mm, onecolumn]{geometry}
\usepackage{lmodern}
\usepackage{tfrupee}
\setlength{\headheight}{1cm}
\setlength{\headsep}{0mm}

\usepackage{gvv-book}
\usepackage{gvv}
\usepackage{cite}
\usepackage{amsmath,amssymb,amsfonts,amsthm}
\usepackage{algorithmic}
\usepackage{graphicx}
\usepackage{textcomp}
\usepackage{xcolor}
\usepackage{txfonts}
\usepackage{listings}
\usepackage{enumitem}
\usepackage{mathtools}
\usepackage{gensymb}
\usepackage{comment}
\usepackage[breaklinks=true]{hyperref}
\usepackage{tkz-euclide}
\usepackage{listings}
\def\inputGnumericTable{}
\usepackage[latin1]{inputenc}
\colorlet{punct}{red!60!black}
\definecolor{background}{HTML}{EEEEEE}
\definecolor{delim}{RGB}{20,105,176}
\colorlet{sqbracket}{delim}
\colorlet{comment}{green!50!black}
\usepackage{array}
\usepackage{longtable}
\usepackage{calc}
\usepackage{multirow}
\usepackage{hhline}
\usepackage{ifthen}
\usepackage{lscape}
\usepackage{xparse}

\bibliographystyle{IEEEtran}

\title{Bonus Question}
\author{EE25BTECH11043 - Nishid Khandagre} % Replace with your name

\begin{document}
\maketitle

\renewcommand{\thefigure}{\theenumi}
\renewcommand{\thetable}{\theenumi}

\numberwithin{align}{enumi}
\numberwithin{figure}{enumi}

\textbf{Question}:
Let $\vec{a}, \vec{b}, \vec{c}$ be unit vectors such that $\vec{a} + \vec{b} + \vec{c} = \vec{0}$. Which of the following are correct?
\begin{enumerate}
    \item $\vec{a} \times \vec{b} = \vec{b} \times \vec{c} = \vec{c} \times \vec{a} = \vec{0}$
    \item $\vec{a} \times \vec{b} = \vec{b} \times \vec{c} = \vec{c} \times \vec{a} \neq \vec{0}$
    \item $\vec{a} \times \vec{b} = \vec{b} \times \vec{c} = \vec{a} \times \vec{c} \neq \vec{0}$
    \item $\vec{a} \times \vec{b}, \vec{b} \times \vec{c}, \vec{c} \times \vec{a}$ are mutually perpendicular.
\end{enumerate}

\textbf{Solution: }
Given that $\vec{a}, \vec{b}, \vec{c}$ are unit vectors.
\begin{align}
    \vec{a}^T \vec{a} = \vec{b}^T \vec{b} = \vec{c}^T \vec{c} = 1
\end{align}
\begin{align}
    \vec{a} + \vec{b} + \vec{c} &= \vec{0}
\end{align}


 Given 
 \begin{align}
    \myvec{\vec{a} & \vec{b} & \vec{c}}\myvec{1\\1\\1}&=\vec{0}
\end{align}
\begin{align}
\vec{A}&=\myvec{\vec{a} & \vec{b} & \vec{c}}
\end{align}
\begin{align}
\vec{A}\myvec{1\\1\\1}&=\vec{0}
\end{align}
\begin{align}
\vec{A}^T\vec{A}\myvec{1\\1\\1}&=\vec{0} 
\end{align}
\begin{align}
\vec{G}\myvec{1\\1\\1}&=\vec{0}
\end{align}

Gram matrix $\vec{G}$:
\begin{align}
    \vec{G} = \myvec{ \vec{a}^T \vec{a} & \vec{a}^T \vec{b} & \vec{a}^T \vec{c} \\ \vec{b}^T \vec{a} & \vec{b}^T \vec{b} & \vec{b}^T \vec{c} \\ \vec{c}^T \vec{a} & \vec{c}^T \vec{b} & \vec{c}^T \vec{c} } &= \myvec{ 1 & \vec{a}^T \vec{b} & \vec{a}^T \vec{c} \\ \vec{b}^T \vec{a} & 1 & \vec{b}^T \vec{c} \\ \vec{c}^T \vec{a} & \vec{c}^T \vec{b} & 1 }
\end{align}

\begin{align}
    \vec{G} \myvec{1 \\ 1 \\ 1} = \myvec{1 + \vec{a}^T \vec{b} + \vec{a}^T \vec{c} \\ \vec{b}^T \vec{a} + 1 + \vec{b}^T \vec{c} \\ \vec{c}^T \vec{a} + \vec{c}^T \vec{b} + 1 } = \myvec{0 \\ 0 \\ 0}
\end{align}


\begin{align}
    1 + \vec{a}^T \vec{b} + \vec{a}^T \vec{c} &= 0 \label{eq:scalar1} \\
    \vec{b}^T \vec{a} + 1 + \vec{b}^T \vec{c} &= 0 \label{eq:scalar2} \\
    \vec{c}^T \vec{a} + \vec{c}^T \vec{b} + 1 &= 0 \label{eq:scalar3}
\end{align}

from this we get 
\begin{align}
    \vec{a}^T \vec{b} = \vec{b}^T \vec{c} = \vec{c}^T \vec{a} = k
\end{align}



Now
\begin{align}
    \myvec{\vec{a} + \vec{b} + \vec{c}}&=\vec{0} \\
\end{align}
\begin{align}
    \myvec{\vec{a} + \vec{b} + \vec{c}}^T \myvec{\vec{a} + \vec{b} + \vec{c}} = \vec{0}^T \vec{0} = 0
\end{align}

\begin{align}
    \vec{a}^T \vec{a} + \vec{b}^T \vec{b} + \vec{c}^T \vec{c} + 2\myvec{\vec{a}^T \vec{b} + \vec{b}^T \vec{c} + \vec{c}^T \vec{a}} = 0
\end{align}


Substitute $\vec{a}^T \vec{a} = \vec{b}^T \vec{b} = \vec{c}^T \vec{c} = 1$


\begin{align}
    1 + 1 + 1 + 2\myvec{\vec{a}^T \vec{b} + \vec{b}^T \vec{c} + \vec{c}^T \vec{a}} &= 0
\end{align}
\begin{align}
    3 + 2\myvec{\vec{a}^T \vec{b} + \vec{b}^T \vec{c} + \vec{c}^T \vec{a}} &= 0
\end{align}
Hence
\begin{align} 
    \vec{a}^T \vec{b} + \vec{b}^T \vec{c} + \vec{c}^T \vec{a} &= -\frac{3}{2}
\end{align}
\begin{align}
    k + k + k &= -\frac{3}{2} \\ 3k &= -\frac{3}{2} \\ k &= -\frac{1}{2}
\end{align}
Thus,
\begin{align}
    \vec{a}^T \vec{b} = \vec{b}^T \vec{c} = \vec{c}^T \vec{a} = -\frac{1}{2}
\end{align}


Given 
\begin{align}
\vec{a} + \vec{b} + \vec{c} = \vec{0} \\
\vec{a} \times (\vec{a} + \vec{b} + \vec{c}) = \vec{a} \times \vec{0}
\end{align}
\begin{align}
    \vec{a} \times \vec{a} + \vec{a} \times \vec{b} + \vec{a} \times \vec{c} = \vec{0}
\end{align}
Since $\vec{a} \times \vec{a} = \vec{0}$:
\begin{align}
    \vec{a} \times \vec{b} + \vec{a} \times \vec{c} = \vec{0} \\ \vec{a} \times \vec{b} = -\vec{a} \times \vec{c} \\ \vec{a} \times \vec{b} = \vec{c} \times \vec{a}
\end{align}
\begin{align}
\vec{b} \times (\vec{a} + \vec{b} + \vec{c}) = \vec{b} \times \vec{a} + \vec{b} \times \vec{b} + \vec{b} \times \vec{c} = \vec{0}
\end{align}
Since $\vec{b} \times \vec{b} = \vec{0}$:
\begin{align}
    \vec{b} \times \vec{a} + \vec{b} \times \vec{c} = \vec{0} \\ -\vec{a} \times \vec{b} + \vec{b} \times \vec{c} = \vec{0} \\ \vec{b} \times \vec{c} = \vec{a} \times \vec{b}
\end{align}

Combining these, :
\begin{align}
    \vec{a} \times \vec{b} = \vec{b} \times \vec{c} = \vec{c} \times \vec{a}
\end{align}

Now
\begin{align}
    \norm{\vec{a} \times \vec{b}}^2 = \norm{\vec{a}}^2 \norm{\vec{b}}^2 - (\vec{a}^T \vec{b})^2
\end{align}
Substitute the values:
\begin{align}
    \norm{\vec{a} \times \vec{b}}^2 = (1)^2 (1)^2 - \left(-\frac{1}{2}\right)^2 = 1 - \frac{1}{4} = \frac{3}{4}
\end{align}
Therefore,
\begin{align}
    \norm{\vec{a} \times \vec{b}} = \sqrt{\frac{3}{4}} = \frac{\sqrt{3}}{2}
\end{align}
Since $\norm{\vec{a} \times \vec{b}} = \frac{\sqrt{3}}{2} \neq 0$, therefore $\vec{a} \times \vec{b} \neq \vec{0}$.
Thus,
\begin{align}
    \vec{a} \times \vec{b} = \vec{b} \times \vec{c} = \vec{c} \times \vec{a} \neq \vec{0}
\end{align}


\end{document}