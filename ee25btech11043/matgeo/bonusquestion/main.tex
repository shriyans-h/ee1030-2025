\documentclass[journal]{IEEEtran}
\usepackage[a5paper, margin=10mm, onecolumn]{geometry}
\usepackage{lmodern} 
\usepackage{tfrupee} 
\setlength{\headheight}{1cm}
\setlength{\headsep}{0mm}   

\usepackage{gvv-book}
\usepackage{gvv}
\usepackage{cite}
\usepackage{amsmath,amssymb,amsfonts,amsthm}
\usepackage{algorithmic}
\usepackage{graphicx}
\usepackage{textcomp}
\usepackage{xcolor}
\usepackage{txfonts}
\usepackage{listings}
\usepackage{enumitem}
\usepackage{mathtools}
\usepackage{gensymb}
\usepackage{comment}
\usepackage[breaklinks=true]{hyperref}
\usepackage{tkz-euclide} 
\usepackage{listings}                             
\def\inputGnumericTable{}                                 
\usepackage[latin1]{inputenc}                                
\usepackage{color}                                            
\usepackage{array}                                            
\usepackage{longtable}                                       
\usepackage{calc}                                             
\usepackage{multirow}                                         
\usepackage{hhline}                                           
\usepackage{ifthen}                                           
\usepackage{lscape}
\usepackage{xparse}

\bibliographystyle{IEEEtran}

\title{Bonus question}
\author{EE25BTECH11043 - Nishid Khandagre} % Replace with your name

\begin{document}
\maketitle

\renewcommand{\thefigure}{\theenumi}
\renewcommand{\thetable}{\theenumi}

\numberwithin{equation}{enumi}
\numberwithin{figure}{enumi} 

\textbf{Question}:
If vectors $\vec{A}, \vec{B}, \vec{C}$ are coplanar, then the matrix $(\vec{A} \ \vec{B} \ \vec{C})$ is singular.
\\

\textbf{Solution: }

Given $\vec{A}, \vec{B}, \vec{C}$ are coplanar.\\

$\therefore \vec{A}, \vec{B}, \vec{C}$ lie in the same plane passing through the origin.


Equation of a plane passing through the origin:
\begin{align}
lx + my + nz = 0
\end{align}


\begin{align}
\myvec{l & m & n} \myvec{x \\ y \\ z} = 0
\end{align}

There must be a non-zero normal vector for this plane.

Let $\vec{n} = \myvec{l \\ m \\ n}$ be a non-zero normal vector to this plane.

Then the equation of the plane is
\begin{align}
\vec{n}^T \myvec{x \\ y \\ z} = 0
\end{align}

Since $\vec{A}, \vec{B}, \vec{C}$ lie in this plane:


\begin{align}
\vec{n}^T \vec{A} &= 0 \\
\vec{n}^T \vec{B} &= 0 \\
\vec{n}^T \vec{C} &= 0
\end{align}


\begin{align}
\vec{n}^T \myvec{ \vec{A} & \vec{B} & \vec{C} } &= \vec{0}
\end{align}



Let $\vec{M} = (\vec{A} \ \vec{B} \ \vec{C})$.
\begin{align}
\vec{n}^T \vec{M} &= \vec{0}
\end{align}

It means rows or columns of matrix $\vec{M}$ is linearly dependent.
Hence,
\begin{align}
\det(\vec{M}) = 0
\end{align}

Therefore, matrix $\vec{M}$ is singular.

Therefore, if $\vec{A}$,$\vec{B}$,$\vec{C}$ are coplanar, then the matrix \myvec{ \vec{A} & \vec{B} & \vec{C} } is singular.

\end{document}