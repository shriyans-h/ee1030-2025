\let\negmedspace\undefined
\let\negthickspace\undefined
\documentclass[journal]{IEEEtran}
\usepackage[a5paper, margin=10mm, onecolumn]{geometry}
\usepackage{lmodern} % Ensure lmodern is loaded for pdflatex
\usepackage{tfrupee} % Include tfrupee package

\setlength{\headheight}{1cm} % Set the height of the header box
\setlength{\headsep}{0mm}     % Set the distance between the header box and the top of the text

\usepackage{gvv-book}
\usepackage{gvv}
\usepackage{cite}
\usepackage{amsmath,amssymb,amsfonts,amsthm}
\usepackage{algorithmic}
\usepackage{graphicx}
\usepackage{textcomp}
\usepackage{xcolor}
\usepackage{txfonts}
\usepackage{listings}
\usepackage{enumitem}
\usepackage{mathtools}
\usepackage{gensymb}
\usepackage{comment}
\usepackage[breaklinks=true]{hyperref}
\usepackage{tkz-euclide} 
\usepackage{listings}
 \usepackage{gvv}                                        
\def\inputGnumericTable{}                                 
\usepackage[latin1]{inputenc}                                
\usepackage{color}                                            
\usepackage{array}                                            
\usepackage{longtable}                                       
\usepackage{calc}                                             
\usepackage{multirow}                                         
\usepackage{hhline}                                           
\usepackage{ifthen}                                           
\usepackage{lscape}
\begin{document}

\bibliographystyle{IEEEtran}


\title{2.4.16}
\author{EE25BTECH11021 - Dhanush Sagar
}
% \maketitle
% \newpage
% \bigskip
{\let\newpage\relax\maketitle}

\renewcommand{\thefigure}{\theenumi}
\renewcommand{\thetable}{\theenumi}
\setlength{\intextsep}{10pt} % Space between text and floats


\numberwithin{equation}{enumi}
\numberwithin{figure}{enumi}
\renewcommand{\thetable}{\theenumi}






We consider the vectors
\begin{align*}
\vec{A} = \myvec{0\\7\\-10}, \qquad
\vec{B} = \myvec{1\\6\\-6}, \qquad
\vec{C} = \myvec{4\\9\\-6}.
\end{align*}

\section*{ PROOF OF: $\vec{A},\vec{B},\vec{C}$ are not collinear (rank method)}
Form the difference vectors $\vec{B}-\vec{A}$ and $\vec{C}-\vec{A}$.

\begin{align}
\vec{B}-\vec{A} &= \myvec{1-0\\[4pt]6-7\\[4pt]-6-(-10)}
\end{align}

\begin{align}
\vec{B}-\vec{A} &= \myvec{1\\-1\\4}
\end{align}

\begin{align}
\vec{C}-\vec{A} &= \myvec{4-0\\[4pt]9-7\\[4pt]-6-(-10)}
\end{align}

\begin{align}
\vec{C}-\vec{A} &= \myvec{4\\2\\4}
\end{align}

Place these as columns in the $3\times 2$ matrix $M$.

\begin{align}
M &= \begin{pmatrix} \vec{B}-\vec{A} & \vec{C}-\vec{A} \end{pmatrix}
\end{align}

\begin{align}
M &= \begin{pmatrix}
1 & 4\\[4pt]
-1 & 2\\[4pt]
4 & 4
\end{pmatrix}
\end{align}

Compute the $2\times2$ minor using rows $1$ and $2$.

\begin{align}
\det\begin{pmatrix}1 & 4\\[4pt] -1 & 2\end{pmatrix}
&= 1\cdot 2 - 4\cdot(-1)
\end{align}

\begin{align}
\det\begin{pmatrix}1 & 4\\[4pt] -1 & 2\end{pmatrix}
&= 2 + 4 = 6 \neq 0
\end{align}

Hence $\operatorname{rank}(M)=2$, so $\vec{B}-\vec{A}$ and $\vec{C}-\vec{A}$ are linearly independent.  
Therefore $\vec{A},\vec{B},\vec{C}$ are not collinear and determine a triangle.

\section*{a) verification for isosceles triangles}
\begin{align}
\vec{AB} &= \vec{B}-\vec{A} = \myvec{1\\-1\\4},
\end{align}

\begin{align}
\vec{BC} &= \vec{C}-\vec{B} = \myvec{4-1\\[4pt]9-6\\[4pt]-6-(-6)}
\end{align}

\begin{align}
\vec{BC} &= \myvec{3\\3\\0},
\end{align}

\begin{align}
\vec{CA} &= \vec{A}-\vec{C} = \myvec{0-4\\[4pt]7-9\\[4pt]-10-(-6)}
\end{align}

\begin{align}
\vec{CA} &= \myvec{-4\\-2\\-4}.
\end{align}


\begin{align}
\|\vec{AB}\|^2 &= (\vec{B}-\vec{A})^T(\vec{B}-\vec{A})
\end{align}

\begin{align}
\|\vec{AB}\|^2 &= 1^2 + (-1)^2 + 4^2
\end{align}

\begin{align}
\|\vec{AB}\|^2 &= 18
\end{align}

\begin{align}
\|\vec{BC}\|^2 &= (\vec{C}-\vec{B})^T(\vec{C}-\vec{B})
\end{align}

\begin{align}
\|\vec{BC}\|^2 &= 3^2 + 3^2 + 0^2
\end{align}

\begin{align}
\|\vec{BC}\|^2 &= 18
\end{align}

\begin{align}
\|\vec{CA}\|^2 &= (\vec{A}-\vec{C})^T(\vec{A}-\vec{C})
\end{align}

\begin{align}
\|\vec{CA}\|^2 &= (-4)^2 + (-2)^2 + (-4)^2
\end{align}

\begin{align}
\|\vec{CA}\|^2 &= 36
\end{align}



\begin{align}
\|\vec{AB}\| &= \|\vec{BC}\| = 3\sqrt{2}, \\
\|\vec{CA}\| &= 6
\end{align}

Therefore the non-collinear vectors $\vec{A},\vec{B},\vec{C}$ determine a triangle, and since two sides are equal, that triangle is \textbf{isosceles} (with equal sides $\vec{AB}$ and $\vec{BC}$).

\section*{B)veification for right angled triangle (matrix / inner-product test)}
To show the triangle is right-angled, compute the inner product of two adjacent side vectors $\vec{AB}$ and $\vec{BC}$.

\begin{align}
(\vec{AB})^T(\vec{BC}) &= \myvec{1&-1&4}\myvec{3\\3\\0}
\end{align}

\begin{align}
(\vec{AB})^T(\vec{BC}) &= 1\cdot 3 + (-1)\cdot 3 + 4\cdot 0
\end{align}

\begin{align}
(\vec{AB})^T(\vec{BC}) &= 3 - 3 + 0 = 0.
\end{align}

Since the inner product is zero, $\vec{AB}\perp\vec{BC}$ and therefore the angle $\angle ABC$ is a right angle; the triangle is \textbf{right-angled at $B$}.



\noindent\textbf{Final statement:} The non-collinear vectors $\vec{A},\vec{B},\vec{C}$ determine a triangle which is both \textbf{isosceles} (with $\|\vec{AB}\|=\|\vec{BC}\|$) and \textbf{right-angled} (with $\vec{AB}\perp\vec{BC}$); hence the triangle is a \emph{right isosceles} triangle with the right angle at vertex $B$.

\begin{figure}[H]
    \centering
    \includegraphics[width=0.9\columnwidth]{figs/fig1.png}
    \caption{}
    \label{fig:}
\end{figure}

\end{document}

