\documentclass{beamer}
\mode<presentation>
\usepackage{amsmath,amssymb,mathtools}
\usepackage{textcomp}
\usepackage{gensymb}
\usepackage{adjustbox}
\usepackage{subcaption}
\usepackage{enumitem}
\usepackage{multicol}
\usepackage{listings}
\usepackage{url}
\usepackage{graphicx} % <-- needed for images
\def\UrlBreaks{\do\/\do-}

\usetheme{Boadilla}
\usecolortheme{lily}
\setbeamertemplate{footline}{
  \leavevmode%
  \hbox{%
  \begin{beamercolorbox}[wd=\paperwidth,ht=2ex,dp=1ex,right]{author in head/foot}%
    \insertframenumber{} / \inserttotalframenumber\hspace*{2ex}
  \end{beamercolorbox}}%
  \vskip0pt%
}
\setbeamertemplate{navigation symbols}{}

\lstset{
  frame=single,
  breaklines=true,
  columns=fullflexible,
  basicstyle=\ttfamily\tiny   % tiny font so code fits
}

\numberwithin{equation}{section}

% ---- your macros ----
\providecommand{\nCr}[2]{\,^{#1}C_{#2}}
\providecommand{\nPr}[2]{\,^{#1}P_{#2}}
\providecommand{\mbf}{\mathbf}
\providecommand{\pr}[1]{\ensuremath{\Pr\left(#1\right)}}
\providecommand{\qfunc}[1]{\ensuremath{Q\left(#1\right)}}
\providecommand{\sbrak}[1]{\ensuremath{{}\left[#1\right]}}
\providecommand{\lsbrak}[1]{\ensuremath{{}\left[#1\right.}}
\providecommand{\rsbrak}[1]{\ensuremath{\left.#1\right]}}
\providecommand{\brak}[1]{\ensuremath{\left(#1\right)}}
\providecommand{\lbrak}[1]{\ensuremath{\left(#1\right.}}
\providecommand{\rbrak}[1]{\ensuremath{\left.#1\right)}}
\providecommand{\cbrak}[1]{\ensuremath{\left\{#1\right\}}}
\providecommand{\lcbrak}[1]{\ensuremath{\left\{#1\right.}}
\providecommand{\rcbrak}[1]{\ensuremath{\left.#1\right\}}}
\theoremstyle{remark}
\newtheorem{rem}{Remark}
\newcommand{\sgn}{\mathop{\mathrm{sgn}}}
\providecommand{\abs}[1]{\left\vert#1\right\vert}
\providecommand{\res}[1]{\Res\displaylimits_{#1}}
\providecommand{\norm}[1]{\lVert#1\rVert}
\providecommand{\mtx}[1]{\mathbf{#1}}
\providecommand{\mean}[1]{E\left[ #1 \right]}
\providecommand{\fourier}{\overset{\mathcal{F}}{ \rightleftharpoons}}
\providecommand{\system}{\overset{\mathcal{H}}{ \longleftrightarrow}}
\providecommand{\dec}[2]{\ensuremath{\overset{#1}{\underset{#2}{\gtrless}}}}
\newcommand{\myvec}[1]{\ensuremath{\begin{pmatrix}#1\end{pmatrix}}}
\let\vec\mathbf
% ---------------------

\title{Matgeo Presentation - Problem 2.4.16}
\author{ee25btech11021 - Dhanush sagar}

\begin{document}
	

		




%---------------- Title Page ----------------
\begin{frame}
  \titlepage
\end{frame}

%---------------- Problem Statement ----------------
\begin{frame}{Problem Statement}
  \begin{itemize}
    \item Given two points
    \begin{align*}
    \vec{A} = \myvec{0\\7\\-10}, \qquad
\vec{B} = \myvec{1\\6\\-6}, \qquad
\vec{C} = \myvec{4\\9\\-6}.
    \end{align*}
    \item(a) prove the given points forms isosceles triangle
    \item (b) prove the given points forms right angled triangle
   
  \end{itemize}
\end{frame}

%---------------- Mathematical Formula ----------------
\begin{frame}{solution}
 

\textbf{solution :}
We consider the vectors
\begin{align*}
\vec{A} = \myvec{0\\7\\-10}, \qquad
\vec{B} = \myvec{1\\6\\-6}, \qquad
\vec{C} = \myvec{4\\9\\-6}.
\end{align*}

\section*{ PROOF OF: $\vec{A},\vec{B},\vec{C}$ are not collinear (rank method)}
Form the difference vectors $\vec{B}-\vec{A}$ and $\vec{C}-\vec{A}$.

\begin{align}
\vec{B}-\vec{A} &= \myvec{1-0\\[4pt]6-7\\[4pt]-6-(-10)}
\end{align}

\begin{align}
\vec{B}-\vec{A} &= \myvec{1\\-1\\4}
\end{align}

\end{frame}
\begin{frame}{solution}

\begin{align}
\vec{C}-\vec{A} &= \myvec{4-0\\[4pt]9-7\\[4pt]-6-(-10)}
\end{align}

\begin{align}
\vec{C}-\vec{A} &= \myvec{4\\2\\4}
\end{align}

Place these as columns in the $3\times 2$ matrix $M$.

\begin{align}
M &= \begin{pmatrix} \vec{B}-\vec{A} & \vec{C}-\vec{A} \end{pmatrix}
\end{align}

\begin{align}
M &= \begin{pmatrix}
1 & 4\\[4pt]
-1 & 2\\[4pt]
4 & 4
\end{pmatrix}
\end{align}

\end{frame}
\begin{frame}{solution}

Compute the $2\times2$ minor using rows $1$ and $2$.

\begin{align}
\det\begin{pmatrix}1 & 4\\[4pt] -1 & 2\end{pmatrix}
&= 1\cdot 2 - 4\cdot(-1)
\end{align}

\begin{align}
\det\begin{pmatrix}1 & 4\\[4pt] -1 & 2\end{pmatrix}
&= 2 + 4 = 6 \neq 0
\end{align}

Hence $\operatorname{rank}(M)=2$, so $\vec{B}-\vec{A}$ and $\vec{C}-\vec{A}$ are linearly independent.  
Therefore $\vec{A},\vec{B},\vec{C}$ are not collinear and determine a triangle.

\section*{a) verification for isosceles triangles}
\begin{align}
\vec{AB} &= \vec{B}-\vec{A} = \myvec{1\\-1\\4},
\end{align}

\end{frame}
\begin{frame}{solution A }

\begin{align}
\vec{BC} &= \vec{C}-\vec{B} = \myvec{4-1\\[4pt]9-6\\[4pt]-6-(-6)}
\end{align}

\begin{align}
\vec{BC} &= \myvec{3\\3\\0},
\end{align}

\begin{align}
\vec{CA} &= \vec{A}-\vec{C} = \myvec{0-4\\[4pt]7-9\\[4pt]-10-(-6)}
\end{align}

\begin{align}
\vec{CA} &= \myvec{-4\\-2\\-4}.
\end{align}

\end{frame}
\begin{frame}{solution}

\begin{align}
\|\vec{AB}\|^2 &= (\vec{B}-\vec{A})^T(\vec{B}-\vec{A})
\end{align}

\begin{align}
\|\vec{AB}\|^2 &= 1^2 + (-1)^2 + 4^2
\end{align}

\begin{align}
\|\vec{AB}\|^2 &= 18
\end{align}

\begin{align}
\|\vec{BC}\|^2 &= (\vec{C}-\vec{B})^T(\vec{C}-\vec{B})
\end{align}

\begin{align}
\|\vec{BC}\|^2 &= 3^2 + 3^2 + 0^2
\end{align}

\begin{align}
\|\vec{BC}\|^2 &= 18
\end{align}

\end{frame}
\begin{frame}{solution}

\begin{align}
\|\vec{CA}\|^2 &= (\vec{A}-\vec{C})^T(\vec{A}-\vec{C})
\end{align}

\begin{align}
\|\vec{CA}\|^2 &= (-4)^2 + (-2)^2 + (-4)^2
\end{align}

\begin{align}
\|\vec{CA}\|^2 &= 36
\end{align}



\begin{align}
\|\vec{AB}\| &= \|\vec{BC}\| = 3\sqrt{2}, \\
\|\vec{CA}\| &= 6
\end{align}

Therefore the non-collinear vectors $\vec{A},\vec{B},\vec{C}$ determine a triangle, and since two sides are equal, that triangle is \textbf{isosceles} (with equal sides $\vec{AB}$ and $\vec{BC}$).

\end{frame}
\begin{frame}{solution (B)}

\section* {B)veification for right angled triangle (matrix / inner-product test)}
To show the triangle is right-angled, compute the inner product of two adjacent side vectors $\vec{AB}$ and $\vec{BC}$.
\

    


\begin{align}
(\vec{AB})^T(\vec{BC}) &= \myvec{1&-1&4}\myvec{3\\3\\0}
\end{align}

\begin{align}
(\vec{AB})^T(\vec{BC}) &= 1\cdot 3 + (-1)\cdot 3 + 4\cdot 0
\end{align}

\begin{align}
(\vec{AB})^T(\vec{BC}) &= 3 - 3 + 0 = 0.
\end{align}

Since the inner product is zero, $\vec{AB}\perp\vec{BC}$ and therefore the angle $\angle ABC$ is a right angle; the triangle is \textbf{right-angled at $B$}.



\noindent\textbf{Final statement:} The non-collinear vectors $\vec{A},\vec{B},\vec{C}$ determine a triangle which is both \textbf{isosceles} (with $\|\vec{AB}\|=\|\vec{BC}\|$) and \textbf{right-angled} (with $\vec{AB}\perp\vec{BC}$); hence the triangle is a \emph{right isosceles} triangle with the right angle at vertex $B$.
\end{frame}
%---------------- C Source Code ----------------
\begin{frame}[fragile]{C Source Code:gen point.c}
\begin{verbatim}
#include <stdio.h>
// Function to write points into a file
void generate_points(const char *filename) {
    FILE *fp = fopen(filename, "w");
    if (fp == NULL) {
        printf("Error opening file!\n");
        return;
    }// Points A, B, C
    double A[3] = {0, 7, -10};
    double B[3] = {1, 6, -6};
    double C[3] = {4, 9, -6};
    fprintf(fp, "%lf %lf %lf\n", A[0], A[1], A[2]);
    fprintf(fp, "%lf %lf %lf\n", B[0], B[1], B[2]);
    fprintf(fp, "%lf %lf %lf\n", C[0], C[1], C[2]);
    fclose(fp);
}
\end{verbatim}
\end{frame}

%---------------- Python solve.py ----------------
\begin{frame}[fragile]{Python Script: solve triangle.py}
\begin{verbatim}
import ctypes
import numpy as np

# Load the shared object
lib = ctypes.CDLL("./gen_points.so")

# Call the C function to generate points.dat
lib.generate_points(b"points.dat")

# Load points from file
points = np.loadtxt("points.dat")
A, B, C = points

# Function to compute squared distance
def dist2(P, Q):
    return np.sum((P - Q) ** 2)
    \end{verbatim}
\end{frame}

\begin{frame}[fragile]{Python Script: solve triangle.py}
\begin{verbatim}
# Squared lengths
AB2 = dist2(A, B)
BC2 = dist2(B, C)
CA2 = dist2(C, A)

print("Squared lengths:")
print("AB^2 =", AB2, " BC^2 =", BC2, " CA^2 =", CA2)

# Check isosceles (two sides equal)
isosceles = (AB2 == BC2) or (BC2 == CA2) or (CA2 == AB2)
print("Isosceles Triangle:", isosceles)

# Check right angle (Pythagoras theorem)
right_angle = (AB2 + BC2 == CA2) or (BC2 + CA2 == AB2) or (CA2 + AB2 == BC2)
print("Right Angled Triangle:", right_angle)
\end{verbatim}
\end{frame}

%---------------- Python plot.py ----------------
\begin{frame}[fragile]{Python Script: plot triangle.py}
\begin{verbatim}
import sys
sys.path.insert(0, '/home/dhanush-sagar/matgeo/codes/CoordGeo')
import numpy as np
import matplotlib.pyplot as plt

# Local imports
from line.funcs import *
from triangle.funcs import *
from conics.funcs import circ_gen

# Load points
points = np.loadtxt("points.dat")
A, B, C = points
# Plot triangle
tri_coords = np.vstack((A, B, C, A))  # close loop
fig = plt.figure()
ax = fig.add_subplot(111, projection='3d')
\end{verbatim}
\end{frame}

\begin{frame}[fragile]{Python Script: plot triangle.py}
\begin{verbatim}

ax.plot(tri_coords[:,0], tri_coords[:,1], tri_coords[:,2], 'b-o', label='Triangle ABC')

# Mark points
ax.text(A[0], A[1], A[2], "A", color='red')
ax.text(B[0], B[1], B[2], "B", color='red')
ax.text(C[0], C[1], C[2], "C", color='red')

ax.set_xlabel('X-axis')
ax.set_ylabel('Y-axis')
ax.set_zlabel('Z-axis')
ax.legend()
plt.savefig("triangle_plot.png")
plt.show()

\end{verbatim}
\end{frame}

%---------------- Result Plot ----------------
\begin{frame}{Result Plot}
 \begin{figure}[H]
     \centering
     \includegraphics[width=0.8\columnwidth]{figs/fig1.png}
     \caption*{}
     \label{fig:fig1}
 \end{figure}
  
\end{frame}

\end{document}
