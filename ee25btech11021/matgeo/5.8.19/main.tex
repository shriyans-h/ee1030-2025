\let\negmedspace\undefined
\let\negthickspace\undefined
\documentclass[journal]{IEEEtran}
\usepackage[a5paper, margin=10mm, onecolumn]{geometry}
%\usepackage{lmodern} % Ensure lmodern is loaded for pdflatex
\usepackage{tfrupee} % Include tfrupee package

\setlength{\headheight}{1cm} % Set the height of the header box
\setlength{\headsep}{0mm}     % Set the distance between the header box and the top of the text

\usepackage{gvv-book}
\usepackage{gvv}
\usepackage{cite}
\usepackage{amsmath,amssymb,amsfonts,amsthm}
\usepackage{algorithmic}
\usepackage{graphicx}
\usepackage{textcomp}
\usepackage{xcolor}
\usepackage{txfonts}
\usepackage{listings}
\usepackage{enumitem}
\usepackage{mathtools}
\usepackage{gensymb}
\usepackage{comment}
\usepackage[breaklinks=true]{hyperref}
\usepackage{tkz-euclide} 
\usepackage{listings}
                                         
\def\inputGnumericTable{}                                 
\usepackage[latin1]{inputenc}                                
\usepackage{color}                                            
\usepackage{array}                                            
\usepackage{longtable}                                       
\usepackage{calc}                                             
\usepackage{multirow}                                         
\usepackage{hhline}                                           
\usepackage{ifthen}                                           
\usepackage{lscape}
\begin{document}


\bibliographystyle{IEEEtran}

\title{5.8.19}
\author{EE25BTECH11021 - Dhanush sagar}
% \maketitle
% \newpage
% \bigskip
\maketitle \vspace{-1cm}
\renewcommand{\thefigure}{\theenumi}
\renewcommand{\thetable}{\theenumi}
\setlength{\intextsep}{10pt} % Space between text and floats

\
\numberwithin{figure}{enumi}
\renewcommand{\thetable}{\theenumi}

\textbf{Question:}  \\
If we add 1 to the numerator and subtract 1 from the denominator, a fraction reduces
to 1. It becomes 1/2 if we only add 1 to the denominator. What is the fraction?


\textbf{Solution:}
% Represent the unknown fraction as a homogeneous vector
Given
Let the unknown fraction be represented as:
\begin{align}
\vec{v} &= \myvec{x \\ y \\ 1}
\end{align}
so that the fraction equals:
\begin{align}
\frac{x}{y}.
\end{align}

% Transformation: add 1 to numerator and subtract 1 from denominator
case 1 :Adding 1 to the numerator and subtracting 1 from the denominator:
\begin{align}
\vec{T_1} &= \myvec{1 & 0 & 1 \\ 0 & 1 & -1 \\ 0 & 0 & 1} \\
\vec{T_1} \vec{v} &= \myvec{x+1 \\ y-1 \\ 1}
\end{align}



case 2 :Adding 1 to the denominator:
\begin{align}
\vec{\vec{T_2}} &= \myvec{1 & 0 & 0 \\ 0 & 1 & 1 \\ 0 & 0 & 1} \\
\vec{\vec{T_2}} \vec{v} &= \myvec{x \\ y+1 \\ 1}
\end{align}

% Fraction condition as linear functional
Condition for a fraction $\frac{a}{b}=k$:
\begin{align}
\myvec{1 & -k & 0} \myvec{a \\ b \\ 1} &= 0
\end{align}

% Apply condition for T1*v = 1
Applying the first condition ($\vec{T_1}*\vec{v}$ = 1):
\begin{align}
\vec{r_1} &= \myvec{1 & -1 & 0} \vec{T_1} \\
    &= \myvec{1 & -1 & 2} \\
\vec{r_1} \vec{v} &= 0
\end{align}


% Apply condition for T2*v = 1/2
Applying the second condition ($\vec{T_2}*\vec{v}$ = 1/2):
\begin{align}
\vec{r_2} &= \myvec{2 & -1 & 0} \vec{\vec{T_2}} \\
    &= \myvec{2 & -1 & -1} \\
\vec{r_2} \vec{v} &= 0
\end{align}

% Stack equations into system
System of equations in matrix form:
\begin{align}
\vec{M} \vec{v} &= 0 \\
\vec{M} &= \myvec{1 & -1 & 2 \\ 2 & -1 & -1}
\end{align}

% Partition matrix



% Solve with Gaussian elimination
Partitioning M into A and c, and vector v into u:
\begin{align}
\vec{M} &= \myvec{\vec{A} & \vec{c}} \\
\vec{A} &= \myvec{1 & -1 \\ 2 & -1}, \quad \vec{c} = \myvec{2 \\ -1} \\
\vec{v} &= \myvec{\vec{u} \\ 1} \\
\vec{A} \vec{u} + \vec{c} &= 0 \implies \vec{A} \vec{u} = -\vec{c}
\end{align}

% Solve with Gaussian elimination using \augvec


Form the augmented matrix:
\begin{align}
\left[\myvec{1 & -1 \\ 2 & -1} \;\middle|\; \myvec{-2 \\ 1}\right]
\end{align}

% Step 2: Eliminate the entry below the pivot (column 1)
Eliminate below the pivot using 
\(r_2 \leftarrow r_2 - 2 r_1\):
\begin{align}
\left[\myvec{1 & -1 \\ 0 & 1} \;\middle|\; \myvec{-2 \\ 5}\right]
\end{align}


% Step 3: Eliminate the entry above the pivot (column 2)
Eliminate above the pivot using 
\(r_1 \leftarrow r_1 + r_2\):
\begin{align}
\left[\myvec{1 & 0 \\ 0 & 1} \;\middle|\; \myvec{3 \\ 5}\right]
\end{align}



% Solution for u
Reading off the solution for u:
\begin{align}
\vec{u} &= \myvec{3 \\ 5}
\end{align}

% Final fraction
Hence the homogeneous vector and fraction:
\begin{align}
\vec{v} &= \myvec{3 \\ 5 \\ 1} \\
\frac{x}{y} &= \frac{3}{5}
\end{align}


\begin{figure}[H]
    \centering
    \includegraphics[width=0.5\columnwidth]{figs/fig1.png}
    \caption{}
    \label{fig:placeholder}
\end{figure}
\end{document}