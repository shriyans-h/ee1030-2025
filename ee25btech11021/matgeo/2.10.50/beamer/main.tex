\documentclass{beamer}

\mode<presentation>
\usepackage{amsmath,amssymb,mathtools}
\usepackage{textcomp}
\usepackage{gensymb}
\usepackage{adjustbox}
\usepackage{subcaption}
\usepackage{enumitem}
\usepackage[utf8]{inputenc}
\usepackage{amssymb}
\usepackage{newunicodechar}
\usepackage{enumitem}
\setlist{nosep} % optional: removes vertical gaps
\setlist[enumerate]{label=\arabic*)} % custom numbering if you want


\newunicodechar{✅}{\checkmark}
\newunicodechar{❌}{\texttimes}
\usepackage{multicol}
\usepackage{listings}
\usepackage{url}
\usepackage{graphicx} % <-- needed for images
\def\UrlBreaks{\do\/\do-}

\usetheme{Boadilla}
\usecolortheme{lily}
\setbeamertemplate{footline}{
  \leavevmode%
  \hbox{%
  \begin{beamercolorbox}[wd=\paperwidth,ht=2ex,dp=1ex,right]{author in head/foot}%
    \insertframenumber{} / \inserttotalframenumber\hspace*{2ex}
  \end{beamercolorbox}}%
  \vskip0pt%
}
\setbeamertemplate{navigation symbols}{}

\lstset{
  frame=single,
  breaklines=true,
  columns=fullflexible,
  basicstyle=\ttfamily\tiny   % tiny font so code fits
}

\numberwithin{equation}{section}

% ---- your macros ----
\providecommand{\nCr}[2]{\,^{#1}{#2}}
\providecommand{\nPr}[2]{\,^{#1}P_{#2}}
\providecommand{\mbf}{\mathbf}
\providecommand{\pr}[1]{\ensuremath{\Pr\left(#1\right)}}
\providecommand{\qfunc}[1]{\ensuremath{Q\left(#1\right)}}
\providecommand{\sbrak}[1]{\ensuremath{{}\left[#1\right]}}
\providecommand{\lsbrak}[1]{\ensuremath{{}\left[#1\right.}}
\providecommand{\rsbrak}[1]{\ensuremath{\left.#1\right]}}
\providecommand{\brak}[1]{\ensuremath{\left(#1\right)}}
\providecommand{\lbrak}[1]{\ensuremath{\left(#1\right.}}
\providecommand{\rbrak}[1]{\ensuremath{\left.#1\right)}}
\providecommand{\cbrak}[1]{\ensuremath{\left\{#1\right\}}}
\providecommand{\lcbrak}[1]{\ensuremath{\left\{#1\right.}}
\providecommand{\rcbrak}[1]{\ensuremath{\left.#1\right\}}}
\theoremstyle{remark}
\newtheorem{rem}{Remark}
\newcommand{\sgn}{\mathop{\mathrm{sgn}}}
\providecommand{\abs}[1]{\left\vert#1\right\vert}
\providecommand{\res}[1]{\Res\displaylimits_{#1}}
\providecommand{\norm}[1]{\lVert#1\rVert}
\providecommand{\mtx}[1]{\mathbf{#1}}
\providecommand{\mean}[1]{E\left[ #1 \right]}
\providecommand{\fourier}{\overset{\mathcal{F}}{ \rightleftharpoons}}
\providecommand{\system}{\overset{\mathcal{H}}{ \longleftrightarrow}}
\providecommand{\dec}[2]{\ensuremath{\overset{#1}{\underset{#2}{\gtrless}}}}
\newcommand{\myvec}[1]{\ensuremath{\begin{pmatrix}#1\end{pmatrix}}}
\let\vec\mathbf
% ---------------------

\title{Matgeo Presentation - Problem 2.7.33}
\author{ee25btech11021 - Dhanush sagar}

\begin{document}
	

		




%---------------- Title Page ----------------
\begin{frame}
  \titlepage
\end{frame}

%---------------- Problem Statement ----------------
\begin{frame}{Problem Statement}

 A variable plane at a distance of one unit from the origin cuts the coordinate axes at $A, B$ and $C$.  


If the centroid $D(x,y,z)$ of triangle $ABC$ satisfies the relation  
\begin{align*}
\frac{1}{x^{2}} + \frac{1}{y^{2}} + \frac{1}{z^{2}} = k,
\end{align*}
then the value of $k$ is :  

\begin{multicols}{2}
\begin{enumerate}
   \item $3$
    \item $1$
    \item $\tfrac{1}{3}$
    \item $9$
\end{enumerate}
\end{multicols}

\end{frame}

%---------------- Mathematical Formula ----------------
\begin{frame}{solution}
Write the plane in vector form as
\begin{align}
\vec{n}^\top \vec{x} &= 1,
\end{align}
so the constant on the right is \(1\) (no scalar \(c\) appears).

The plane meets the coordinate axes at
\begin{align}
\vec{A} &= \myvec{a\\0\\0}, \qquad
\vec{B} = \myvec{0\\b\\0}, \qquad
\vec{C} = \myvec{0\\0\\c}.
\end{align}

Define
\begin{align}
\vec{e} &= \myvec{1\\1\\1}, \qquad 
\vec{M} = \myvec{a&0&0\\[4pt]0&b&0\\[4pt]0&0&c}.
\end{align}
\end{frame}
\begin{frame}{solution}
Since $\vec{A},\vec{B},\vec{C}$ lie on the plane, we have
\begin{align}
\vec{n}^\top \vec{A} &= 1, \\[4pt]
\vec{n}^\top \vec{B} &= 1, \\[4pt]
\vec{n}^\top \vec{C} &= 1.
\end{align}

These combine to the single matrix equation
\begin{align}
\vec{n}^\top \vec{M} &= \vec{e}^\top,
\end{align}
and transposing gives
\begin{align}
\vec{M}^\top \vec{n} &= \vec{e}.
\end{align}

Because \(\vec{M}\) is diagonal, invertible, and equals its transpose, we obtain
\begin{align}
\vec{n} &= \vec{M}^{-1}\vec{e}.
\end{align}
The perpendicular distance \(d\) of the plane \(\vec{n}^\top \vec{x}=1\) from the origin is
\end{frame}
\begin{frame}{solution}
\begin{align}
d &= \frac{|1|}{\|\vec{n}\|}
= \frac{1}{\|\vec{M}^{-1}\vec{e}\|}.
\end{align}
Hence the quadratic-form relation
\begin{align}
\vec{e}^\top \vec{M}^{-2} \vec{e} &= \frac{1}{d^2}.
\end{align}

The centroid of the triangle \(ABC\) is
\begin{align}
\vec{D} &= \frac{\vec{A}+\vec{B}+\vec{C}}{3}
= \tfrac{1}{3}\vec{M}\vec{e},
\end{align}
so the coordinates of the centroid are
\begin{align}
x &= \tfrac{a}{3}, \quad y=\tfrac{b}{3}, \quad z=\tfrac{c}{3},
\quad\text{and}\quad \vec{D}=\myvec{x\\y\\z}.
\end{align}

Compute the desired sum:
\begin{align}
\frac{1}{x^2}+\frac{1}{y^2}+\frac{1}{z^2}
&= \frac{1}{(a/3)^2}+\frac{1}{(b/3)^2}+\frac{1}{(c/3)^2} \\
&= 9\!\left(\frac{1}{a^2}+\frac{1}{b^2}+\frac{1}{c^2}\right).
\end{align}
\end{frame}
\begin{frame}{solution}
To express this in matrix form note that
\begin{align}
\vec{M}^{-2} &= \myvec{\tfrac{1}{a^2}&0&0\\[4pt]0&\tfrac{1}{b^2}&0\\[4pt]0&0&\tfrac{1}{c^2}}, \\[6pt]
\vec{M}^{-2}\vec{e} &= \myvec{\tfrac{1}{a^2}\\[4pt]\tfrac{1}{b^2}\\[4pt]\tfrac{1}{c^2}}, \\[6pt]
\vec{e}^\top \vec{M}^{-2}\vec{e} &= \frac{1}{a^2}+\frac{1}{b^2}+\frac{1}{c^2}.
\end{align}
Therefore
\begin{align}
\frac{1}{x^2}+\frac{1}{y^2}+\frac{1}{z^2}
&= 9\,\vec{e}^\top \vec{M}^{-2}\vec{e}.
\end{align}
\end{frame}
\begin{frame}{solution}
Combining with the distance relation gives the compact formula
\begin{align}
\frac{1}{x^2}+\frac{1}{y^2}+\frac{1}{z^2} &= \frac{9}{d^2}.
\end{align}

For the given problem \(d=1\), so
\begin{align}
\frac{1}{x^2}+\frac{1}{y^2}+\frac{1}{z^2} &= 9,
\end{align}
and thus
\[
\boxed{k=9}.
\]

\end{frame}

%---------------- C Source Code ----------------
\begin{frame}[fragile]{C Source Code:plane points.c}
\begin{verbatim}
#include <stdio.h>

typedef struct {
    double x, y, z;
} Point;
// Generate points array for A, B, C
void generate_plane_points(double a, double b, double c, double *points_array) {
    points_array[0] = a; points_array[1] = 0; points_array[2] = 0; // A
    points_array[3] = 0; points_array[4] = b; points_array[5] = 0; // B
    points_array[6] = 0; points_array[7] = 0; points_array[8] = c; // C
}
// Compute k
double compute_plane_k(double a, double b, double c) {
    double x = a/3.0, y = b/3.0, z = c/3.0;
    return 1.0/(x*x) + 1.0/(y*y) + 1.0/(z*z);
}


\end{verbatim}
\end{frame}

%---------------- Python solve.py ----------------
\begin{frame}[fragile]{Python Script:  solve plane.py}
\begin{verbatim}
import ctypes
import sympy as sp
import numpy as np
# --- 1. Load C library ---
lib = ctypes.CDLL("./plane_points.so")
lib.generate_plane_points.argtypes = [ctypes.c_double, ctypes.c_double, ctypes.c_double, ctypes.POINTER(ctypes.c_double)]
lib.generate_plane_points.restype = None
# --- 2. Symbolic variables ---
a, b, c = sp.symbols('a b c', positive=True)
# --- 3. Compute centroid symbolically ---
A = sp.Matrix([a, 0, 0])
B = sp.Matrix([0, b, 0])
C = sp.Matrix([0, 0, c])
D = (A + B + C)/3
# --- 4. Compute k symbolically ---
k = 1/D[0]**2 + 1/D[1]**2 + 1/D[2]**2
# --- 5. Apply plane condition: 1/a^2 + 1/b^2 + 1/c^2 = 1 ---
\end{verbatim}
\end{frame}
\begin{frame}[fragile]{Python Script:  solve plane.py}
\begin{verbatim}
plane_condition = 1/a**2 + 1/b**2 + 1/c**2
k_final = k.subs(plane_condition, 1)
print("Centroid D =", D)
print("k in terms of a,b,c =", k)
print("Using plane condition 1/a^2 + 1/b^2 + 1/c^2 = 1")
print("Final k =", k_final)
# --- 6. Optional: Call C program to generate points numerically ---
# Here we must provide numeric values to C, just as placeholders
points_array = (ctypes.c_double * 9)()
lib.generate_plane_points(1.0, 1.0, 1.0, points_array)
A_num = np.array(points_array[0:3])
B_num = np.array(points_array[3:6])
C_num = np.array(points_array[6:9])
print("\nPoints from C code (numeric placeholders):")
print("A =", A_num)
print("B =", B_num)
print("C =", C_num)
\end{verbatim}
\end{frame}

%---------------- Python plot.py ----------------

\begin{frame}[fragile]{Python Script: plot plane.py}
\begin{verbatim}
import matplotlib.pyplot as plt
import numpy as np
import ctypes
# Load C library
lib = ctypes.CDLL("./plane_points.so")
lib.generate_plane_points.argtypes = [ctypes.c_double, ctypes.c_double, ctypes.c_double, ctypes.POINTER(ctypes.c_double)]
lib.generate_plane_points.restype = None
# --- Choose numeric a, b, c satisfying 1/a^2 + 1/b^2 + 1/c^2 = 1 ---
a = b = c = np.sqrt(3)  # This ensures plane distance = 1
# Generate points
points_array = (ctypes.c_double * 9)()
lib.generate_plane_points(a, b, c, points_array)
A = np.array(points_array[0:3])
B = np.array(points_array[3:6])
C = np.array(points_array[6:9])
D = (A + B + C)/3  # centroid
# Plane distance from origin
\end{verbatim}
\end{frame}
\begin{frame}[fragile]{Python Script: plot plane.py}
\begin{verbatim}
distance = 1 / np.sqrt(1/a**2 + 1/b**2 + 1/c**2)
# Plot triangle, centroid, and plane
fig = plt.figure()
ax = fig.add_subplot(111, projection='3d')
for P, color, label in zip([A, B, C, D], ['red','green','blue','black'], ['A','B','C','Centroid D']):
    ax.scatter(*P, color=color, label=label)
# Triangle edges
for P1, P2 in [(A, B), (B, C), (C, A)]:
    ax.plot([P1[0], P2[0]], [P1[1], P2[1]], [P1[2], P2[2]], 'gray')
# Plane surface
xx, yy = np.meshgrid(np.linspace(0, a, 10), np.linspace(0, b, 10))
zz = c*(1 - xx/a - yy/b)
ax.plot_surface(xx, yy, zz, alpha=0.3, color='cyan')
ax.set_xlabel('X'); ax.set_ylabel('Y'); ax.set_zlabel('Z')
ax.set_title(f'Triangle ABC, Centroid D, Plane\nDistance from origin = {distance:.2f}')
ax.legend()
plt.savefig('triangle_centroid_plane_distance.png')  plt.show()

\end{verbatim}
\end{frame}

%---------------- Result Plot ----------------
\begin{frame}{Result Plot}
 \begin{figure}[H]
     \centering
     \includegraphics[width=0.8\columnwidth]{figs/fig1.png}
     \caption*{}
     \label{fig:fig1}
 \end{figure}
 
\end{frame}

\end{document}
