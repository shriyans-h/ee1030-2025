\let\negmedspace\undefined
\let\negthickspace\undefined
\documentclass[journal]{IEEEtran}
\usepackage[a5paper, margin=10mm, onecolumn]{geometry}
\usepackage{lmodern} % Ensure lmodern is loaded for pdflatex
\usepackage{tfrupee} % Include tfrupee package

\setlength{\headheight}{1cm} % Set the height of the header box
\setlength{\headsep}{0mm}     % Set the distance between the header box and the top of the text

\usepackage{gvv-book}
\usepackage{gvv}
\usepackage{cite}
\usepackage{amsmath,amssymb,amsfonts,amsthm}
\usepackage{algorithmic}
\usepackage{graphicx}
\usepackage{textcomp}
\usepackage{xcolor}
\usepackage{txfonts}
\usepackage{listings}
\usepackage{enumitem}
\usepackage{mathtools}
\usepackage{gensymb}
\usepackage{comment}
\usepackage[breaklinks=true]{hyperref}
\usepackage{tkz-euclide} 
\usepackage{listings}
 \usepackage{gvv}                                        
\def\inputGnumericTable{}                                 
\usepackage[latin1]{inputenc}                                
\usepackage{color}                                            
\usepackage{array}                                            
\usepackage{longtable}                                       
\usepackage{calc}                                             
\usepackage{multirow}                                         
\usepackage{hhline}                                           
\usepackage{ifthen}                                           
\usepackage{lscape}
\begin{document}

\bibliographystyle{IEEEtran}


\title{2.10.50}
\author{EE25BTECH11021 - Dhanush Sagar
}
% \maketitle
% \newpage
% \bigskip
{\let\newpage\relax\maketitle}

\renewcommand{\thefigure}{\theenumi}
\renewcommand{\thetable}{\theenumi}
\setlength{\intextsep}{10pt} % Space between text and floats


\numberwithin{equation}{enumi}
\numberwithin{figure}{enumi}
\renewcommand{\thetable}{\theenumi}


\textbf{Question} \\
A variable plane at a distance of one unit from the origin cuts the coordinate axes at $A, B$ and $C$.  


If the centroid $D(x,y,z)$ of triangle $ABC$ satisfies the relation  
\begin{align*}
\frac{1}{x^{2}} + \frac{1}{y^{2}} + \frac{1}{z^{2}} = k,
\end{align*}
then the value of $k$ is :  

\begin{multicols}{2}
\begin{enumerate}
   \item $3$
    \item $1$
    \item $\tfrac{1}{3}$
    \item $9$
\end{enumerate}
\end{multicols}

\textbf{Solution} \\
The plane is written in vector form as
\begin{align}
\vec{n}^\top \vec{x} &= c,
\end{align}
where $\vec{n}\in\mathbb{R}^3$ is the normal vector.
The plane cuts the coordinate axes at
\begin{align}
\vec{A} &= \myvec{a\\0\\0}, \qquad
\vec{B} = \myvec{0\\b\\0}, \qquad
\vec{C} = \myvec{0\\0\\c}.
\end{align}
Define
\begin{align}
\vec{e} &= \myvec{1\\1\\1}, \qquad 
\vec{M} = \myvec{a&0&0\\[4pt]0&b&0\\[4pt]0&0&c}.
\end{align}
Since $\vec{A},\vec{B},\vec{C}$ lie on the plane,
\begin{align}
\vec{n}^\top \vec{M} &= c\, \vec{e}^\top.
\end{align}
Taking transpose,
\begin{align}
\vec{M}^\top \vec{n} &= c\, \vec{e}.
\end{align}
Since $\vec{M}$ is diagonal,
\begin{align}
\vec{n} &= c\, \vec{M}^{-1} \vec{e}.
\end{align}
The perpendicular distance of the plane from the origin is
\begin{align}
d &= \frac{|c|}{\|\vec{n}\|}
= \frac{|c|}{|c|\,\|\vec{M}^{-1}\vec{e}\|}
= \frac{1}{\|\vec{M}^{-1}\vec{e}\|},
\end{align}
hence
\begin{align}
\vec{e}^\top \vec{M}^{-2} \vec{e} &= \frac{1}{d^2}.
\end{align}
The centroid of $\triangle ABC$ is
\begin{align}
\vec{D} &= \frac{\vec{A}+\vec{B}+\vec{C}}{3}
= \tfrac{1}{3}\vec{M}\vec{e}.
\end{align}
Thus the centroid coordinates are
\begin{align}
x &= \tfrac{a}{3}, \quad y = \tfrac{b}{3}, \quad z = \tfrac{c}{3},
\quad\text{so}\quad
\vec{D} = \myvec{x\\y\\z}.
\end{align}
Now we compute
\begin{align}
\frac{1}{x^2}+\frac{1}{y^2}+\frac{1}{z^2}
&= 9\left(\frac{1}{a^2}+\frac{1}{b^2}+\frac{1}{c^2}\right).
\end{align}
To connect with the matrix form, observe that
\begin{align}
\vec{M}^{-2} &= \myvec{\tfrac{1}{a^2}&0&0\\[4pt]0&\tfrac{1}{b^2}&0\\[4pt]0&0&\tfrac{1}{c^2}}, \\[6pt]
\vec{M}^{-2}\vec{e} &= \myvec{\tfrac{1}{a^2}\\[4pt]\tfrac{1}{b^2}\\[4pt]\tfrac{1}{c^2}}, \\[6pt]
\vec{e}^\top \vec{M}^{-2}\vec{e} &= \frac{1}{a^2}+\frac{1}{b^2}+\frac{1}{c^2}.
\end{align}
Therefore
\begin{align}
\frac{1}{x^2}+\frac{1}{y^2}+\frac{1}{z^2}
&= 9\,\vec{e}^\top \vec{M}^{-2} \vec{e}.
\end{align}
Using the distance relation,
\begin{align}
\frac{1}{x^2}+\frac{1}{y^2}+\frac{1}{z^2} &= \frac{9}{d^2}.
\end{align}
For $d=1$,
\begin{align}
\frac{1}{x^2}+\frac{1}{y^2}+\frac{1}{z^2} &= 9,
\end{align}
so
\[
\boxed{k=9}.
\]
\begin{figure}[H]
    \centering
    \includegraphics[width=0.7\linewidth]{figs/fig1.png}
    \caption{}
    \label{fig:fig1}
\end{figure}







\end{document}

\end{document}




