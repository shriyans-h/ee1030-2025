\let\negmedspace\undefined
\let\negthickspace\undefined
\documentclass[journal]{IEEEtran}
\usepackage[a5paper, margin=10mm, onecolumn]{geometry}
\usepackage{lmodern} % Ensure lmodern is loaded for pdflatex
\usepackage{tfrupee} % Include tfrupee package

\setlength{\headheight}{1cm} % Set the height of the header box
\setlength{\headsep}{0mm}     % Set the distance between the header box and the top of the text

\usepackage{gvv-book}
\usepackage{gvv}
\usepackage{cite}
\usepackage{amsmath,amssymb,amsfonts,amsthm}
\usepackage{algorithmic}
\usepackage{graphicx}
\usepackage{textcomp}
\usepackage{xcolor}
\usepackage{txfonts}
\usepackage{listings}
\usepackage{enumitem}
\usepackage{mathtools}
\usepackage{gensymb}
\usepackage{comment}
\usepackage[breaklinks=true]{hyperref}
\usepackage{tkz-euclide} 
\usepackage{listings}
 \usepackage{gvv}                                        
\def\inputGnumericTable{}                                 
\usepackage[latin1]{inputenc}                                
\usepackage{color}                                            
\usepackage{array}                                            
\usepackage{longtable}                                       
\usepackage{calc}                                             
\usepackage{multirow}                                         
\usepackage{hhline}                                           
\usepackage{ifthen}                                           
\usepackage{lscape}
\begin{document}

\bibliographystyle{IEEEtran}


\title{2.10.50}
\author{EE25BTECH11021 - Dhanush Sagar
}
% \maketitle
% \newpage
% \bigskip
{\let\newpage\relax\maketitle}

\renewcommand{\thefigure}{\theenumi}
\renewcommand{\thetable}{\theenumi}
\setlength{\intextsep}{10pt} % Space between text and floats


\numberwithin{equation}{enumi}
\numberwithin{figure}{enumi}
\renewcommand{\thetable}{\theenumi}


\textbf{Question} \\
 A variable plane at a distance of one unit from the origin cuts the coordinate axes at $A, B$ and $C$.  


If the centroid $D(x,y,z)$ of triangle $ABC$ satisfies the relation  
\begin{align*}
\frac{1}{x^{2}} + \frac{1}{y^{2}} + \frac{1}{z^{2}} = k,
\end{align*}
then the value of $k$ is :  

\begin{multicols}{2}
\begin{enumerate}
    \item $3$
    \item $1$
    \item $\tfrac{1}{3}$
    \item $9$
\end{enumerate}
\end{multicols}

\textbf{Solution} \\

Let the plane meet the coordinate axes at 
\begin{align*}
\vec{A} = \myvec{a\\0\\0}, 
\vec{B} = \myvec{0\\b\\0}, 
\vec{C} = \myvec{0\\0\\c}.
\end{align*}
Define
\begin{align*}
    M = \mathrm{diag}(a,b,c), 
\quad 
e = \myvec{1 \\ 1 \\ 1}.
\end{align*}

The normal vector of the plane is given by
\begin{align}
n &= M^{-1} e 
= \myvec{ \tfrac{1}{a} \\[6pt] \tfrac{1}{b} \\[6pt] \tfrac{1}{c} }
\end{align}

The squared norm of the normal vector is
\begin{align}
\|n\|^2 &= e^T M^{-2} e 
= \frac{1}{a^2} + \frac{1}{b^2} + \frac{1}{c^2}
\end{align}

The perpendicular distance of the plane from the origin is
\begin{align}
d &= \frac{|1|}{\|n\|} 
= \frac{1}{\sqrt{e^T M^{-2} e}}
\end{align}

Thus, the relation between $a,b,c$ and $d$ is
\begin{align}
e^T M^{-2} e &= \frac{1}{d^2}
\end{align}

The centroid of the triangle $ABC$ is
\begin{align}
D &= \frac{1}{3} \myvec{ a \\ b \\ c } 
= \frac{1}{3} M e
\end{align}

Hence, the coordinates of the centroid are
\begin{align}
x &= \tfrac{a}{3}, \quad y = \tfrac{b}{3}, \quad z = \tfrac{c}{3}
\end{align}

Now, the required expression is
\begin{align}
\frac{1}{x^2} + \frac{1}{y^2} + \frac{1}{z^2} 
&= \frac{1}{(a/3)^2} + \frac{1}{(b/3)^2} + \frac{1}{(c/3)^2}
\end{align}

Simplifying, we get
\begin{align}
\frac{1}{x^2} + \frac{1}{y^2} + \frac{1}{z^2}
&= 9\left(\frac{1}{a^2} + \frac{1}{b^2} + \frac{1}{c^2}\right)
\end{align}

In matrix form, this becomes
\begin{align}
\frac{1}{x^2} + \frac{1}{y^2} + \frac{1}{z^2}
&= 9\, e^T M^{-2} e
\end{align}

Using the relation obtained earlier,
\begin{align}
\frac{1}{x^2} + \frac{1}{y^2} + \frac{1}{z^2}
&= \frac{9}{d^2}
\end{align}

For $d=1$, we finally obtain
\begin{align}
\frac{1}{x^2} + \frac{1}{y^2} + \frac{1}{z^2} &= 9
\end{align}

\[
\boxed{k=9}
\]
\begin{figure}[H]
    \centering
    \includegraphics[width=0.55\linewidth]{figs/fig1.png}
    \caption{}
    \label{fig:fig1}
\end{figure}
\end{document}




