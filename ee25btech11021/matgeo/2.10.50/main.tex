\let\negmedspace\undefined
\let\negthickspace\undefined
\documentclass[journal]{IEEEtran}
\usepackage[a5paper, margin=10mm, onecolumn]{geometry}
\usepackage{lmodern} % Ensure lmodern is loaded for pdflatex
\usepackage{tfrupee} % Include tfrupee package

\setlength{\headheight}{1cm} % Set the height of the header box
\setlength{\headsep}{0mm}     % Set the distance between the header box and the top of the text

\usepackage{gvv-book}
\usepackage{gvv}
\usepackage{cite}
\usepackage{amsmath,amssymb,amsfonts,amsthm}
\usepackage{algorithmic}
\usepackage{graphicx}
\usepackage{textcomp}
\usepackage{xcolor}
\usepackage{txfonts}
\usepackage{listings}
\usepackage{enumitem}
\usepackage{mathtools}
\usepackage{gensymb}
\usepackage{comment}
\usepackage[breaklinks=true]{hyperref}
\usepackage{tkz-euclide} 
\usepackage{listings}
 \usepackage{gvv}                                        
\def\inputGnumericTable{}                                 
\usepackage[latin1]{inputenc}                                
\usepackage{color}                                            
\usepackage{array}                                            
\usepackage{longtable}                                       
\usepackage{calc}                                             
\usepackage{multirow}                                         
\usepackage{hhline}                                           
\usepackage{ifthen}                                           
\usepackage{lscape}
\begin{document}

\bibliographystyle{IEEEtran}


\title{2.10.50}
\author{EE25BTECH11021 - Dhanush Sagar
}
% \maketitle
% \newpage
% \bigskip
{\let\newpage\relax\maketitle}

\renewcommand{\thefigure}{\theenumi}
\renewcommand{\thetable}{\theenumi}
\setlength{\intextsep}{10pt} % Space between text and floats


\numberwithin{equation}{enumi}
\numberwithin{figure}{enumi}
\renewcommand{\thetable}{\theenumi}


\textbf{Question} \\
A variable plane at a distance of one unit from the origin cuts the coordinate axes at $A, B$ and $C$.  


If the centroid $D(x,y,z)$ of triangle $ABC$ satisfies the relation  
\begin{align*}
\frac{1}{x^{2}} + \frac{1}{y^{2}} + \frac{1}{z^{2}} = k,
\end{align*}
then the value of $k$ is :  

\begin{multicols}{2}
\begin{enumerate}
   \item $3$
    \item $1$
    \item $\tfrac{1}{3}$
    \item $9$
\end{enumerate}
\end{multicols}

\textbf{Solution} \\
We write the plane in vector form as
\begin{align}
\vec{n}^{\top}\vec{x} &= 1,
\end{align}
and introduce the axis intercepts
\begin{align}
\vec{A} &= \myvec{a\\[4pt]0\\[4pt]0}, & 
\vec{B} &= \myvec{0\\[4pt]b\\[4pt]0}, &
\vec{C} &= \myvec{0\\[4pt]0\\[4pt]c}.
\end{align}

Define
\begin{align}
\vec{e} &= \myvec{1\\[4pt]1\\[4pt]1}, &
\mathbf{M} &= \myvec{a&0&0\\[4pt]0&b&0\\[4pt]0&0&c}.
\end{align}
(Thus the columns of $\mathbf{M}$ are $\vec{A},\vec{B},\vec{C}$.)

Since $\vec{A},\vec{B},\vec{C}$ lie on the plane we have
\begin{align}
\vec{n}^{\top}\vec{A} &= 1, & 
\vec{n}^{\top}\vec{B} &= 1, & 
\vec{n}^{\top}\vec{C} &= 1.
\end{align}
These three equations combine into the single matrix relation
\begin{align}
\vec{n}^{\top}\mathbf{M} &= \vec{e}^{\top}.
\end{align}
Transposing yields
\begin{align}
\mathbf{M}^{\top}\vec{n} &= \vec{e}.
\end{align}
Because $\mathbf{M}$ is diagonal (hence $\mathbf{M}=\mathbf{M}^{\top}$) and invertible,
\begin{align}
\vec{n} &= \mathbf{M}^{-1}\vec{e}.
\end{align}

The centroid $\vec{D}$ of triangle $ABC$ is
\begin{align}
\vec{D} &= \frac{\vec{A}+\vec{B}+\vec{C}}{3}
= \tfrac{1}{3}\mathbf{M}\vec{e}.
\end{align}
Introduce the diagonal matrix of centroid coordinates
\begin{align}
\mathbf{X} &= \operatorname{diag}(x,y,z),
\end{align}
and substitute the given relation
\begin{align}
\vec{D} &= \mathbf{X}\vec{e}.
\end{align}
Comparing with $\vec{D}=\tfrac{1}{3}\mathbf{M}\vec{e}$ we obtain the matrix identity
\begin{align}
\mathbf{M} &= 3\mathbf{X}.
\end{align}

Using $\vec{n}=\mathbf{M}^{-1}\vec{e}$ and $\mathbf{M}=3\mathbf{X}$ we get
\begin{align}
\vec{n} &= (3\mathbf{X})^{-1}\vec{e}
      = \tfrac{1}{3}\mathbf{X}^{-1}\vec{e}.
\end{align}

The perpendicular distance \(d\) from the origin to the plane $\vec{n}^{\top}\vec{x}=1$ is
\begin{align}
d &= \frac{1}{\|\vec{n}\|}
   = \frac{1}{\sqrt{\vec{n}^{\top}\vec{n}}}.
\end{align}
Compute $\vec{n}^{\top}\vec{n}$ in purely matrix form:
\begin{align}
\vec{n}^{\top}\vec{n}
&= \Big(\tfrac{1}{3}\mathbf{X}^{-1}\vec{e}\Big)^{\top}
   \Big(\tfrac{1}{3}\mathbf{X}^{-1}\vec{e}\Big) \\
&= \tfrac{1}{9}\,\vec{e}^{\top}\mathbf{X}^{-2}\vec{e}.
\end{align}
Hence
\begin{align}
\frac{1}{d^2} = \vec{n}^{\top}\vec{n} = \tfrac{1}{9}\,\vec{e}^{\top}\mathbf{X}^{-2}\vec{e}.
\end{align}

\subsection*{Proof of expansion}

Since $\mathbf{X}=\operatorname{diag}(x,y,z)$, we have
\begin{align}
\mathbf{X}^{-2} &= \operatorname{diag}\!\left(\tfrac{1}{x^2},\tfrac{1}{y^2},\tfrac{1}{z^2}\right).
\end{align}
Thus
\begin{align}
\vec{e}^{\top}\mathbf{X}^{-2}\vec{e}
&= \myvec{1&1&1}
   \myvec{
     \tfrac{1}{x^2} & 0 & 0 \\
     0 & \tfrac{1}{y^2} & 0 \\
     0 & 0 & \tfrac{1}{z^2}
   }
   \myvec{1\\1\\1} \\
&= \tfrac{1}{x^2}+\tfrac{1}{y^2}+\tfrac{1}{z^2}.
\end{align}

\subsection*{Final result}

Therefore,
\begin{align}
\frac{1}{x^2}+\frac{1}{y^2}+\frac{1}{z^2} &= \frac{9}{d^2}.
\end{align}
For the given problem $d=1$,
\begin{align}
\frac{1}{x^2}+\frac{1}{y^2}+\frac{1}{z^2} &= 9,
\end{align}
so the required constant is
\[
\boxed{k=9}.
\]

\begin{figure}[H]
    \centering
    \includegraphics[width=0.8\linewidth]{figs/fig1.png}
    \caption{}
    \label{fig:fig1}
\end{figure}


\end{document}






