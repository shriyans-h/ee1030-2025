\let\negmedspace\undefined
\let\negthickspace\undefined
\documentclass[journal]{IEEEtran}
\usepackage[a5paper, margin=10mm, onecolumn]{geometry}
\usepackage{lmodern} % Ensure lmodern is loaded for pdflatex
\usepackage{tfrupee} % Include tfrupee package

\setlength{\headheight}{1cm} % Set the height of the header box
\setlength{\headsep}{0mm}     % Set the distance between the header box and the top of the text

\usepackage{gvv-book}
\usepackage{gvv}
\usepackage{cite}
\usepackage{amsmath,amssymb,amsfonts,amsthm}
\usepackage{algorithmic}
\usepackage{graphicx}
\usepackage{textcomp}
\usepackage{xcolor}
\usepackage{txfonts}
\usepackage{listings}
\usepackage{enumitem}
\usepackage{mathtools}
\usepackage{gensymb}
\usepackage{comment}
\usepackage[breaklinks=true]{hyperref}
\usepackage{tkz-euclide} 
\usepackage{listings}
 \usepackage{gvv}                                        
\def\inputGnumericTable{}                                 
\usepackage[latin1]{inputenc}                                
\usepackage{color}                                            
\usepackage{array}                                            
\usepackage{longtable}                                       
\usepackage{calc}                                             
\usepackage{multirow}                                         
\usepackage{hhline}                                           
\usepackage{ifthen}                                           
\usepackage{lscape}
\begin{document}

\bibliographystyle{IEEEtran}


\title{2.7.33}
\author{EE25BTECH11021 - Dhanush Sagar
}
% \maketitle
% \newpage
% \bigskip
{\let\newpage\relax\maketitle}

\renewcommand{\thefigure}{\theenumi}
\renewcommand{\thetable}{\theenumi}
\setlength{\intextsep}{10pt} % Space between text and floats


\numberwithin{equation}{enumi}
\numberwithin{figure}{enumi}
\renewcommand{\thetable}{\theenumi}


\textbf{Question:} \\
Find the value of $p$ if
\[
(2\hat{i} + 6\hat{j} + 27\hat{k}) \times (\hat{i} + 3\hat{j} + p\hat{k}) = 0.
\]



\textbf{Solution:} \\
The given vectors are
\begin{align}
\vec{A} &= \myvec{2 \\ 6 \\ 27}, 
& \vec{B} &= \myvec{1 \\ 3 \\ p}.
\end{align}

Construct the matrix
\begin{align}
M &= \myvec{2 & 6 & 27 \\ 1 & 3 & p}.
\end{align}

If $\vec{A} \times \vec{B} = 0$, then $\vec{A}$ and $\vec{B}$ are linearly dependent. 
Thus,
\begin{align}
\operatorname{rank}(M) < 2.
\end{align}

For a $2 \times 3$ matrix, this happens exactly when all $2 \times 2$ minors vanish.

First minor:
\begin{align}
\det\myvec{2 & 6 \\ 1 & 3} &= 6 - 6 = 0.
\end{align}

Second minor:
\begin{align}
\det\myvec{2 & 27 \\ 1 & p} &= 2p - 27.
\end{align}

Third minor:
\begin{align}
\det\myvec{6 & 27 \\ 3 & p} &= 6p - 81.
\end{align}

For $\operatorname{rank}(M)<2$, all three determinants must vanish. 
The first is already zero. From the second,
\begin{align}
2p - 27 &= 0  \Rightarrow  p = \tfrac{27}{2}.
\end{align}

From the third,
\begin{align}
6p - 81 &= 0  \Rightarrow  p = \tfrac{27}{2}.
\end{align}

Thus, the required value is
\begin{align}
\boxed{p = \tfrac{27}{2}}
\end{align}
\begin{figure}[H]
    \centering
    \includegraphics[width=0.5\linewidth]{figs/fig1.png}
    \caption{}
    \label{fig:fig1}
\end{figure}
\end{document}