\let\negmedspace\undefined
\let\negthickspace\undefined
\documentclass[journal]{IEEEtran}
\usepackage[a5paper, margin=10mm, onecolumn]{geometry}
\usepackage{lmodern} % Ensure lmodern is loaded for pdflatex
\usepackage{tfrupee} % Include tfrupee package

\setlength{\headheight}{1cm} % Set the height of the header box
\setlength{\headsep}{0mm}     % Set the distance between the header box and the top of the text

\usepackage{gvv-book}
\usepackage{gvv}
\usepackage{cite}
\usepackage{amsmath,amssymb,amsfonts,amsthm}
\usepackage{algorithmic}
\usepackage{graphicx}
\usepackage{textcomp}
\usepackage{xcolor}
\usepackage{txfonts}
\usepackage{listings}
\usepackage{enumitem}
\usepackage{mathtools}
\usepackage{gensymb}
\usepackage{comment}
\usepackage[breaklinks=true]{hyperref}
\usepackage{tkz-euclide} 
\usepackage{listings}
 \usepackage{gvv}                                        
\def\inputGnumericTable{}                                 
\usepackage[latin1]{inputenc}                                
\usepackage{color}                                            
\usepackage{array}                                            
\usepackage{longtable}                                       
\usepackage{calc}                                             
\usepackage{multirow}                                         
\usepackage{hhline}                                           
\usepackage{ifthen}                                           
\usepackage{lscape}
\begin{document}

\bibliographystyle{IEEEtran}


\title{2.7.33}
\author{EE25BTECH11021 - Dhanush Sagar
}
% \maketitle
% \newpage
% \bigskip
{\let\newpage\relax\maketitle}

\renewcommand{\thefigure}{\theenumi}
\renewcommand{\thetable}{\theenumi}
\setlength{\intextsep}{10pt} % Space between text and floats


\numberwithin{equation}{enumi}
\numberwithin{figure}{enumi}
\renewcommand{\thetable}{\theenumi}


\textbf{Question:} \\
Find the value of $p$ if
\[
(2\hat{i} + 6\hat{j} + 27\hat{k}) \times (\hat{i} + 3\hat{j} + p\hat{k}) = 0.
\]

\textbf{Solution:} \\
The given vectors are
\begin{align}
\vec{A} = \myvec{2 \\ 6 \\ 27}, 
\vec{B} = \myvec{1 \\ 3 \\ p}.
\end{align}

Since
\begin{align}
\vec{A} \times \vec{B} = 0,
\end{align}
the vectors $\vec{A}$ and $\vec{B}$ are linearly dependent. Therefore, there exists a scalar $t$ such that
\begin{align}
\vec{B} &= t \vec{A}.
\end{align}

Substituting the components,
\begin{align}
\myvec{1 \\ 3 \\ p} &= t \myvec{2 \\ 6 \\ 27}.
\end{align}

From the first coordinate,
\begin{align}
1 &= 2t  \Rightarrow  t = \tfrac{1}{2}.
\end{align}

From the second coordinate,
\begin{align}
3 &= 6t  \Rightarrow  t = \tfrac{1}{2},
\end{align}
which is consistent with the first.

Finally, from the third coordinate,
\begin{align}
p &= 27t = 27 \cdot \tfrac{1}{2} = \tfrac{27}{2}.
\end{align}

\textbf{Final Answer:}
\begin{align}
\boxed{p = \tfrac{27}{2}}
\end{align}
\begin{figure}[H]
    \centering
    \includegraphics[width=0.5\linewidth]{figs/fig1.png}
    \caption{}
    \label{fig:fig1}
\end{figure}
\end{document}