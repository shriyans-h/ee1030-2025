\documentclass{beamer}
\mode<presentation>
\usepackage{amsmath,amssymb,mathtools}
\usepackage{textcomp}
\usepackage{gensymb}
\usepackage{adjustbox}
\usepackage{subcaption}
\usepackage{enumitem}
\usepackage[utf8]{inputenc}
\usepackage{amssymb}
\usepackage{newunicodechar}

\newunicodechar{✅}{\checkmark}
\newunicodechar{❌}{\texttimes}
\usepackage{multicol}
\usepackage{listings}
\usepackage{url}
\usepackage{graphicx} % <-- needed for images
\def\UrlBreaks{\do\/\do-}

\usetheme{Boadilla}
\usecolortheme{lily}
\setbeamertemplate{footline}{
  \leavevmode%
  \hbox{%
  \begin{beamercolorbox}[wd=\paperwidth,ht=2ex,dp=1ex,right]{author in head/foot}%
    \insertframenumber{} / \inserttotalframenumber\hspace*{2ex}
  \end{beamercolorbox}}%
  \vskip0pt%
}
\setbeamertemplate{navigation symbols}{}

\lstset{
  frame=single,
  breaklines=true,
  columns=fullflexible,
  basicstyle=\ttfamily\tiny   % tiny font so code fits
}

\numberwithin{equation}{section}

% ---- your macros ----
\providecommand{\nCr}[2]{\,^{#1}{#2}}
\providecommand{\nPr}[2]{\,^{#1}P_{#2}}
\providecommand{\mbf}{\mathbf}
\providecommand{\pr}[1]{\ensuremath{\Pr\left(#1\right)}}
\providecommand{\qfunc}[1]{\ensuremath{Q\left(#1\right)}}
\providecommand{\sbrak}[1]{\ensuremath{{}\left[#1\right]}}
\providecommand{\lsbrak}[1]{\ensuremath{{}\left[#1\right.}}
\providecommand{\rsbrak}[1]{\ensuremath{\left.#1\right]}}
\providecommand{\brak}[1]{\ensuremath{\left(#1\right)}}
\providecommand{\lbrak}[1]{\ensuremath{\left(#1\right.}}
\providecommand{\rbrak}[1]{\ensuremath{\left.#1\right)}}
\providecommand{\cbrak}[1]{\ensuremath{\left\{#1\right\}}}
\providecommand{\lcbrak}[1]{\ensuremath{\left\{#1\right.}}
\providecommand{\rcbrak}[1]{\ensuremath{\left.#1\right\}}}
\theoremstyle{remark}
\newtheorem{rem}{Remark}
\newcommand{\sgn}{\mathop{\mathrm{sgn}}}
\providecommand{\abs}[1]{\left\vert#1\right\vert}
\providecommand{\res}[1]{\Res\displaylimits_{#1}}
\providecommand{\norm}[1]{\lVert#1\rVert}
\providecommand{\mtx}[1]{\mathbf{#1}}
\providecommand{\mean}[1]{E\left[ #1 \right]}
\providecommand{\fourier}{\overset{\mathcal{F}}{ \rightleftharpoons}}
\providecommand{\system}{\overset{\mathcal{H}}{ \longleftrightarrow}}
\providecommand{\dec}[2]{\ensuremath{\overset{#1}{\underset{#2}{\gtrless}}}}
\newcommand{\myvec}[1]{\ensuremath{\begin{pmatrix}#1\end{pmatrix}}}
\let\vec\mathbf
% ---------------------

\title{Matgeo Presentation - Problem 2.7.33}
\author{ee25btech11021 - Dhanush sagar}

\begin{document}
	

		




%---------------- Title Page ----------------
\begin{frame}
  \titlepage
\end{frame}

%---------------- Problem Statement ----------------
\begin{frame}{Problem Statement}
   Find the value of $p$ if
\[
(2\hat{i} + 6\hat{j} + 27\hat{k}) \times (\hat{i} + 3\hat{j} + p\hat{k}) = 0.
\]
\end{frame}

%---------------- Mathematical Formula ----------------
\begin{frame}{solution}
 \textbf{Solution:} \\
The given vectors are
\begin{align}
\vec{A} &= \myvec{2 \\ 6 \\ 27}, 
& \vec{B} &= \myvec{1 \\ 3 \\ p}.
\end{align}

Construct the matrix
\begin{align}
M &= \myvec{2 & 6 & 27 \\ 1 & 3 & p}.
\end{align}

If $\vec{A} \times \vec{B} = 0$, then $\vec{A}$ and $\vec{B}$ are linearly dependent. 
Thus,
\begin{align}
\operatorname{rank}(M) < 2.
\end{align}

For a $2 \times 3$ matrix, this happens exactly when all $2 \times 2$ minors vanish.

First minor:
\begin{align}
\det\myvec{2 & 6 \\ 1 & 3} &= 6 - 6 = 0.
\end{align}
\end{frame}
\begin{frame}{solution}
Second minor:
\begin{align}
\det\myvec{2 & 27 \\ 1 & p} &= 2p - 27.
\end{align}

Third minor:
\begin{align}
\det\myvec{6 & 27 \\ 3 & p} &= 6p - 81.
\end{align}

For $\operatorname{rank}(M)<2$, all three determinants must vanish. 
The first is already zero. From the second,
\begin{align}
2p - 27 &= 0  \Rightarrow  p = \tfrac{27}{2}.
\end{align}

From the third,
\begin{align}
6p - 81 &= 0  \Rightarrow  p = \tfrac{27}{2}.
\end{align}

Thus, the required value is
\begin{align}
\boxed{p = \tfrac{27}{2}}
\end{align}



\end{frame}

%---------------- C Source Code ----------------
\begin{frame}[fragile]{C Source Code: cross.c}
\begin{verbatim}
#include <stdio.h>
void cross_product(double a[3], double b[3], double result[3]) {
    result[0] = a[1]*b[2] - a[2]*b[1];
    result[1] = a[2]*b[0] - a[0]*b[2];
    result[2] = a[0]*b[1] - a[1]*b[0];
}
double find_p() {
    double a[3] = {2, 6, 27};
    double p = 27.0 / 2.0;
    double b[3] = {1, 3, p};
    double res[3];
    cross_product(a, b, res);
   if(res[0] == 0 && res[1] == 0 && res[2] == 0) {
        return p;
    }
    return -1;
}

\end{verbatim}
\end{frame}

%---------------- Python solve.py ----------------
\begin{frame}[fragile]{Python Script: vector solve.py}
\begin{verbatim}
import ctypes
import numpy as np
# Load shared library
lib = ctypes.CDLL("./cross.so")
lib.find_p.restype = ctypes.c_double
# Call C function
p = lib.find_p()
print("Computed value of p:", p)
# Verify using numpy
A = np.array([2, 6, 27])
B = np.array([1, 3, p])
cross_prod = np.cross(A, B)
print("Cross product A × B =", cross_prod)
if np.allclose(cross_prod, [0, 0, 0]):
    print("✅ A and B are parallel. Solution verified.")
else:
    print("❌ Something went wrong.")
\end{verbatim}
\end{frame}

%---------------- Python plot.py ----------------

\begin{frame}[fragile]{Python Script: plot vector.py}
\begin{verbatim}
import numpy as np
import matplotlib.pyplot as plt
# Vectors
A = np.array([2, 6, 27])
p = 27/2
B = np.array([1, 3, p])
# Plot
fig = plt.figure()
ax = fig.add_subplot(111, projection='3d')
ax.quiver(0, 0, 0, A[0], A[1], A[2], color='r', label=f'A = {A}')
ax.quiver(0, 0, 0, B[0], B[1], B[2], color='b', label=f'B = {B}')
ax.set_xlabel('X')
ax.set_ylabel('Y')
ax.set_zlabel('Z')
ax.legend()
plt.title("Vectors A and B (parallel)")
plt.savefig("vectors.png")              plt.show()

\end{verbatim}
\end{frame}

%---------------- Result Plot ----------------
\begin{frame}{Result Plot}
 \begin{figure}[H]
     \centering
     \includegraphics[width=0.5\columnwidth]{figs/fig1.png}
     \caption*{}
     \label{fig:fig1}
 \end{figure}
 
\end{frame}

\end{document}
