\documentclass{beamer}

\usepackage[utf8]{inputenc}
\usepackage{lmodern} 
\usepackage{listings}
\usepackage{xcolor} 
\usepackage{graphicx}
\usepackage{multicol}
\usepackage{amsmath,amssymb,amsfonts}

\definecolor{myblue}{RGB}{48, 63, 159}
\setbeamercolor{palette primary}{bg=myblue, fg=white}
\setbeamercolor{structure}{fg=myblue}
\setbeamercolor{frametitle}{bg=myblue, fg=white}
\setbeamercolor{title}{bg=myblue, fg=white}
\setbeamercolor{footlinecolor}{bg=myblue, fg=white}

\defbeamertemplate*{title page}{mytemplate}{
	\vfill
	\begin{center}
		\begin{beamercolorbox}[wd=0.8\paperwidth, center, rounded=true, shadow=true]{title}
			\usebeamerfont{title}\inserttitle\par
		\end{beamercolorbox}
		\vspace{2cm} 
		\usebeamerfont{author}\insertauthor
		\vspace{1cm} 
		\usebeamerfont{date}\insertdate
	\end{center}
	\vfill
}

\defbeamertemplate*{frametitle}{mytemplate}{
	\begin{beamercolorbox}[wd=\paperwidth, ht=2.5ex, dp=1.5ex, left]{frametitle}
		\hspace{1em}\usebeamerfont{frametitle}\insertframetitle
	\end{beamercolorbox}
}

\setbeamertemplate{footline}{
	\begin{beamercolorbox}[wd=\paperwidth, ht=2.25ex, dp=1ex]{footlinecolor}
		\hspace{1em}\usebeamerfont{author in footline}\insertshortauthor
		\hfill
		\usebeamerfont{title in footline}\insertshorttitle
		\hfill
		\usebeamerfont{date in footline}\insertdate \hspace{1em} \insertframenumber/\inserttotalframenumber \hspace{0.5em}
	\end{beamercolorbox}
}

\setbeamerfont{author in footline}{size=\tiny}
\setbeamerfont{title in footline}{size=\tiny}
\setbeamerfont{date in footline}{size=\tiny}

\newcommand{\myvec}[1]{\ensuremath{\begin{pmatrix}#1\end{pmatrix}}}
\providecommand{\brak}[1]{\ensuremath{\left(#1\right)}}

\title{12.482}
\author{Shriyansh Chawda-EE25BTECH11052}

\begin{document}
	
	\setbeamertemplate{footline}{} 
	\frame{\titlepage}
	
	\begin{frame}{Question}
			Consider the matrix $\mathbf{M} = \myvec{5 & 3\\3 & 5}$. The normalized eigen-vector corresponding to the smallest eigen-value of the matrix $\mathbf{M}$ is:\\
		\hfill{(PE 2016)}
		\begin{multicols}{4} 
			\begin{enumerate}
				\item $\begin{pmatrix} \frac{\sqrt{3}}{2} \\ \frac{1}{2} \end{pmatrix}$
				\item $\begin{pmatrix} \frac{\sqrt{3}}{2} \\ -\frac{1}{2} \end{pmatrix}$
				\item $\begin{pmatrix} \frac{1}{\sqrt{2}} \\ -\frac{1}{\sqrt{2}} \end{pmatrix}$
				\item $\begin{pmatrix} \frac{1}{\sqrt{2}} \\ \frac{1}{\sqrt{2}} \end{pmatrix}$
			\end{enumerate}
		\end{multicols} 
	\end{frame}
	
	\begin{frame}{Solution}
	Given matrix:
\begin{align}
	\mathbf{M} = \myvec{5 & 3\\3 & 5}
\end{align}
The characteristic equation is:
\begin{align}
	\det(\mathbf{M} - \lambda\mathbf{I}) &= 0\\
	\det\myvec{5-\lambda & 3\\3 & 5-\lambda} &= 0\\
	(5-\lambda)(5-\lambda) - (3)(3) &= 0\\
	(5-\lambda)^2 - 9 &= 0\\
	\lambda^2 - 10\lambda + 16 &= 0
\end{align}
Using the quadratic formula:

	\end{frame}
	
	\begin{frame}{Solution}
		\begin{align}
			\lambda &= \frac{10 \pm \sqrt{100 - 64}}{2} = \frac{10 \pm \sqrt{36}}{2} = \frac{10 \pm 6}{2}\\
			\lambda_1 &=  8 \quad \text{(largest eigenvalue)}\\
			\lambda_2 &=  2 \quad \text{(smallest eigenvalue)}
		\end{align}
			For $\lambda = 2$, solve $(\mathbf{M} - 2\mathbf{I})\mathbf{v} = \mathbf{0}$:
		\begin{align}
			\myvec{5-2 & 3\\3 & 5-2}\myvec{v_1\\v_2} &= \myvec{0\\0}\\
			\myvec{3 & 3\\3 & 3}\myvec{v_1\\v_2} &= \myvec{0\\0}\\
			3v_1 + 3v_2 &= 0\\
			v_1 + v_2 &= 0 \implies v_1 = -v_2
		\end{align}
	\end{frame}

\begin{frame}{Solution}
	Let $v_2 = t$, then $v_1 = -t$.\\
The general eigenvector is:
\begin{align}
	\mathbf{v} = t\myvec{-1\\1} = t\myvec{1\\-1} \cdot (-1)
\end{align}
We can choose:
\begin{align}
	\mathbf{v} = \myvec{1\\-1}
\end{align}

\subsection*{Step 3: Normalize the Eigenvector}

The normalized eigenvector is:
\begin{align}
	\hat{\mathbf{v}} &= \frac{\mathbf{v}}{\|\mathbf{v}\|}\\
	\|\mathbf{v}\| &= \sqrt{\mathbf{v}^\top \mathbf{v}}= \sqrt{2}\\
	\hat{\mathbf{v}} &= \frac{1}{\sqrt{2}}\myvec{1\\-1} = \myvec{\frac{1}{\sqrt{2}} \\ -\frac{1}{\sqrt{2}}}
\end{align}

\end{frame}
$$\boxed{\text{The correct answer is (c) } \myvec{\frac{1}{\sqrt{2}} \\ -\frac{1}{\sqrt{2}}}}$$	

\end{document}