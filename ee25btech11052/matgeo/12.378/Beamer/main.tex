\documentclass{beamer}

\usepackage[utf8]{inputenc}
\usepackage{lmodern} 
\usepackage{listings}
\usepackage{xcolor} 
\usepackage{graphicx}
\usepackage{multicol}
\usepackage{amsmath,amssymb,amsfonts}

\definecolor{myblue}{RGB}{48, 63, 159}
\setbeamercolor{palette primary}{bg=myblue, fg=white}
\setbeamercolor{structure}{fg=myblue}
\setbeamercolor{frametitle}{bg=myblue, fg=white}
\setbeamercolor{title}{bg=myblue, fg=white}
\setbeamercolor{footlinecolor}{bg=myblue, fg=white}

\defbeamertemplate*{title page}{mytemplate}{
	\vfill
	\begin{center}
		\begin{beamercolorbox}[wd=0.8\paperwidth, center, rounded=true, shadow=true]{title}
			\usebeamerfont{title}\inserttitle\par
		\end{beamercolorbox}
		\vspace{2cm} 
		\usebeamerfont{author}\insertauthor
		\vspace{1cm} 
		\usebeamerfont{date}\insertdate
	\end{center}
	\vfill
}

\defbeamertemplate*{frametitle}{mytemplate}{
	\begin{beamercolorbox}[wd=\paperwidth, ht=2.5ex, dp=1.5ex, left]{frametitle}
		\hspace{1em}\usebeamerfont{frametitle}\insertframetitle
	\end{beamercolorbox}
}

\setbeamertemplate{footline}{
	\begin{beamercolorbox}[wd=\paperwidth, ht=2.25ex, dp=1ex]{footlinecolor}
		\hspace{1em}\usebeamerfont{author in footline}\insertshortauthor
		\hfill
		\usebeamerfont{title in footline}\insertshorttitle
		\hfill
		\usebeamerfont{date in footline}\insertdate \hspace{1em} \insertframenumber/\inserttotalframenumber \hspace{0.5em}
	\end{beamercolorbox}
}

\setbeamerfont{author in footline}{size=\tiny}
\setbeamerfont{title in footline}{size=\tiny}
\setbeamerfont{date in footline}{size=\tiny}

\newcommand{\myvec}[1]{\ensuremath{\begin{pmatrix}#1\end{pmatrix}}}
\providecommand{\brak}[1]{\ensuremath{\left(#1\right)}}

\title{12.482}
\author{Shriyansh Chawda-EE25BTECH11052}

\begin{document}
	
	\setbeamertemplate{footline}{} 
	\frame{\titlepage}
	
	\begin{frame}{Question}
	The eigenvalues of the matrix 
\[
\myvec{6 & 1 \\ -2 & 3}
\]
are

\begin{multicols}{4}
	\begin{enumerate}
		\item[(a)] (3, 6)
		\item[(b)] (1, -2)
		\item[(c)] (5, 4)
		\item[(d)] (1, 6)
	\end{enumerate}
\end{multicols}

	\end{frame}
	
	\begin{frame}{Solution}
	According to the Cayley-Hamilton theorem,  $\det(\mathbf{A} - \lambda\mathbf{I}) = 0$.
\begin{align}
	\det(\mathbf{A} - \lambda\mathbf{I}) &= \det\myvec{6-\lambda & 1 \\ -2 & 3-\lambda} \\
	&= (6-\lambda)(3-\lambda) - (1)(-2) \\
	&= 18 - 6\lambda - 3\lambda + \lambda^2 + 2 \\
	&= \lambda^2 - 9\lambda + 20 = 0\\
	&=(\lambda - 5)(\lambda - 4) = 0
\end{align}
Thus, the eigenvalues are $\lambda_1 = 5$ and $\lambda_2 = 4$. This corresponds to option (c).

	\end{frame}
	


\end{document}