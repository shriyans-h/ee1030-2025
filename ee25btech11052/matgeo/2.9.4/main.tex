\let\negmedspace\undefined
\let\negthickspace\undefined
\documentclass[journal]{IEEEtran}
\usepackage[a5paper, margin=10mm, onecolumn]{geometry}

\usepackage{tfrupee} 

\setlength{\headheight}{1cm} 
\setlength{\headsep}{0mm}      

\usepackage{gvv-book}
\usepackage{gvv}
\usepackage{cite}
\usepackage{amsmath,amssymb,amsfonts,amsthm}
\usepackage{algorithmic}
\usepackage{graphicx}
\usepackage{textcomp}
\usepackage{xcolor}
\usepackage{txfonts}
\usepackage{listings}
\usepackage{enumitem}
\usepackage{mathtools}
\usepackage{gensymb}
\usepackage{comment}
\usepackage[breaklinks=true]{hyperref}
\usepackage{tkz-euclide} 
\usepackage{listings}

\def\inputGnumericTable{}                                 
\usepackage[latin1]{inputenc}                                
\usepackage{color}                                            
\usepackage{array}                                            
\usepackage{longtable}                                       
\usepackage{calc}                                             
\usepackage{multirow}                                         
\usepackage{hhline}                                           
\usepackage{ifthen}                                           
\usepackage{lscape}
\begin{document}
	
	\bibliographystyle{IEEEtran}
	\vspace{3cm}
	
	\title{2.9.4}
	\author{EE25BTECH11052 - Shriyansh Kalpesh Chawda}


	{\let\newpage\relax\maketitle}
	
	\renewcommand{\thefigure}{\theenumi}
	\renewcommand{\thetable}{\theenumi}
	\setlength{\intextsep}{10pt} 
	
	\numberwithin{equation}{enumi}
	\numberwithin{figure}{enumi}
	\renewcommand{\thetable}{\theenumi}
	\textbf{Question}:\\
	\\
	If  $\vec{a} = \hat{i} + \hat{j} + \hat{k}, \quad \vec{a} \cdot \vec{b} = 1, \quad \text{and} \quad \vec{a} \times \vec{b} = \hat{j} - \hat{k},$ then find $|\vec{b}|$.
	\hfill (12, 2022)
	\\
	\solution\\
	We are given the vectors in component form:
	\begin{align}
	 \vec{a} = \myvec{1 \\ 1 \\ 1}.\\
	 \vec{a} \times \vec{b} = \myvec{0 \\ 1 \\ -1}.\\
	 \vec{b} = \myvec{b_1 \\ b_2 \\ b_3}
	 \end{align}
	From the dot product: \\
	\begin{align}
	\vec{a}^\top \vec{b} = 1 \implies
	\begin{pmatrix} 1 & 1 & 1 \end{pmatrix} \myvec{b_1 \\ b_2 \\ b_3} = 1\\
	b_1 + b_2 + b_3 = 1 
	\end{align}
	From the cross product: \\
\begin{align}     
	\vec{a} \times \vec{b} = \begin{vmatrix} \hat{i} & \hat{j} & \hat{k} \\ 1 & 1 & 1 \\ b_1 & b_2 & b_3 \end{vmatrix} = \brak{b_3 - b_2}\hat{i} + \brak{b_1 - b_3}\hat{j} + \brak{b_2 - b_1}\hat{k}
\end{align}
Comparing Equation (0.2) and (0.6)
	\begin{align}
		b_3 - b_2 &= 0   \\
		b_1 - b_3 &= 1 
	\end{align}
Substituting values in (0.5):
		\begin{align}
			\brak{1 + b_3} + \brak{b_3} + b_3 &= 1 \\
			1 + 3b_3 &= 1 \\
			3b_3 &= 0 \implies b_3 = 0
		\end{align}
So, now for $b_2 \text{ and } b_1$
		\begin{align}
			b_2 &= b_3 = 0 \\
			b_1 &= 1 + b_3 = 1 + 0 = 1
		\end{align}

	So,  $\vec{b}$ = $\myvec{1 \\ 0 \\ 0}$,\\
	To find magnitude,
	\begin{align}
		\vec{b}^\top \vec{b} = 1 \\
		\myvec{ 1 & 0 & 0 } \myvec{1 \\ 0 \\ 0} = 1
	\end{align}
The magnitude of vector $\vec{b}$ is \textbf{1}.
	
\end{document}