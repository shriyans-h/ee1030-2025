\documentclass{beamer}

\usepackage[utf8]{inputenc}
\usepackage{lmodern} 
\usepackage[utf8]{inputenc}
\usepackage{lmodern} 
\usepackage{listings}
\usepackage{xcolor} 
\usepackage{graphicx}
\definecolor{myblue}{RGB}{48, 63, 159}
\setbeamercolor{palette primary}{bg=myblue, fg=white}
\setbeamercolor{structure}{fg=myblue}
\setbeamercolor{frametitle}{bg=myblue, fg=white}
\setbeamercolor{title}{bg=myblue, fg=white}
\setbeamercolor{footlinecolor}{bg=myblue, fg=white}


\defbeamertemplate*{title page}{mytemplate}{
	\vfill
	\begin{center}
		
		\begin{beamercolorbox}[wd=0.8\paperwidth, center, rounded=true, shadow=true]{title}
			\usebeamerfont{title}\inserttitle\par
		\end{beamercolorbox}
		\vspace{2cm} 
		
		\usebeamerfont{author}\insertauthor
		\vspace{1cm} 
		\usebeamerfont{date}\insertdate
	\end{center}
	\vfill
}


\defbeamertemplate*{frametitle}{mytemplate}{
	\begin{beamercolorbox}[wd=\paperwidth, ht=2.5ex, dp=1.5ex, left]{frametitle}
		\hspace{1em}\usebeamerfont{frametitle}\insertframetitle
	\end{beamercolorbox}
}


\setbeamertemplate{footline}{
	\begin{beamercolorbox}[wd=\paperwidth, ht=2.25ex, dp=1ex]{footlinecolor}
		\hspace{1em}\usebeamerfont{author in footline}\insertshortauthor
		\hfill
		\usebeamerfont{title in footline}\insertshorttitle
		\hfill
		\usebeamerfont{date in footline}\insertdate \hspace{1em} \insertframenumber/\inserttotalframenumber \hspace{0.5em}
	\end{beamercolorbox}
}


\setbeamerfont{author in footline}{size=\tiny}
\setbeamerfont{title in footline}{size=\tiny}
\setbeamerfont{date in footline}{size=\tiny}

\newcommand{\myvec}[1]{\ensuremath{\begin{pmatrix}#1\end{pmatrix}}}
\providecommand{\brak}[1]{\ensuremath{\left(#1\right)}}


\title{2.9.4}
\author{Shriyansh Chawda-EE25BTECH11052}
\date{August 23, 2025}



\begin{document}
	

		\setbeamertemplate{footline}{} 
		\frame{\titlepage}
	
	
	

	\begin{frame}{Question} 
If   $\vec{a} = \hat{i} + \hat{j} + \hat{k}, \quad \vec{a} \cdot \vec{b} = 1, \quad \text{and} \quad \vec{a} \times \vec{b} = \hat{j} - \hat{k},$ then find \textbf{$|\vec{b}|$}.
\hfill (12, 2022)
	\end{frame}
	
\begin{frame}{Solution}
	We are given the vectors in component form:
\begin{align}
	\vec{a} = \myvec{1 \\ 1 \\ 1}.\\
	\vec{a} \times \vec{b} = \myvec{0 \\ 1 \\ -1}.\\
	\vec{b} = \myvec{b_1 \\ b_2 \\ b_3}
\end{align}
From the dot product: \\

\end{frame}

\begin{frame}{Solution}
	\begin{align}
		\vec{a}^\top \vec{b} = 1 \implies
		\begin{pmatrix} 1 & 1 & 1 \end{pmatrix} \myvec{b_1 \\ b_2 \\ b_3} = 1\\
		b_1 + b_2 + b_3 = 1 
	\end{align}
	From the cross product: \\
	
\begin{align}     
	\vec{a} \times \vec{b} = \begin{vmatrix} \hat{i} & \hat{j} & \hat{k} \\ 1 & 1 & 1 \\ b_1 & b_2 & b_3 \end{vmatrix} = \brak{b_3 - b_2}\hat{i} + \brak{b_1 - b_3}\hat{j} + \brak{b_2 - b_1}\hat{k}
\end{align}
Comparing Equation (0.2) and (0.6)
\begin{align}
	b_3 - b_2 &= 0   \\
	b_1 - b_3 &= 1 
\end{align}
Substituting values in (0.5):


\end{frame}




\begin{frame}{Solution}
\begin{align}
	\brak{1 + b_3} + \brak{b_3} + b_3 &= 1 \\
	1 + 3b_3 &= 1 \\
	3b_3 &= 0 \implies b_3 = 0
\end{align}
So, now for $b_2 \text{ and } b_1$
\begin{align}
	b_2 &= b_3 = 0 \\
	b_1 &= 1 + b_3 = 1 + 0 = 1
\end{align}
So,  $\vec{b}$ = $\myvec{1 \\ 0 \\ 0}$,\\
To find magnitude,
\end{frame}
\begin{frame}{Solution}
\begin{align}
	\vec{b}^\top \vec{b} = 1 \\
	\myvec{ 1 & 0 & 0 } \myvec{1 \\ 0 \\ 0} = 1
\end{align}
The magnitude of vector $\vec{b}$ is \textbf{1}.
\end{frame}


\end{document}