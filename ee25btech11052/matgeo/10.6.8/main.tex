\documentclass[article]{IEEEtran}
\usepackage[a5paper, margin=10mm, onecolumn]{geometry}

\usepackage{tfrupee} 
\setlength{\headheight}{1cm} 
\setlength{\headsep}{0mm}       
\usepackage{multicol}
\usepackage{gvv-book}
\usepackage{gvv}
\usepackage{cite}
\usepackage{amsmath,amssymb,amsfonts,amsthm}
\usepackage{algorithmic}
\usepackage{graphicx}
\usepackage{textcomp}
\usepackage{xcolor}
\usepackage{txfonts}
\usepackage{listings}
\usepackage{enumitem}
\usepackage{mathtools}
\usepackage{gensymb}
\usepackage{comment}
\usepackage[breaklinks=true]{hyperref}
\usepackage{tkz-euclide} 
\usepackage{listings}

\def\inputGnumericTable{}                                 
\usepackage[latin1]{inputenc}                                
\usepackage{color}                       

\usepackage{array}                                            
\usepackage{longtable}                                   

\usepackage{calc}                                             
\usepackage{multirow}                                         
\usepackage{hhline}          

\usepackage{ifthen}                                           
\usepackage{lscape}
\begin{document}
	\title{10.6.8}
	\author{EE25BTECH11052 - Shriyansh Kalpesh Chawda}
	\maketitle
	\textbf{Question}\\
	Construct a pair of tangents to a circle of radius 4cm from a point P lying outside the circle at 
	a distance of 6cm from the centre.
	\hfill{(10, 2023)}\\
	\textbf{Solution}\\
	Let the center of the circle be at origin,The equation is $x^2 + y^2 = 16$ and Point P (at distance 6 from center along x-axis) 
	\begin{align}
		O = \myvec{0\\0}\\
		P = \myvec{6\\0} \\
		\vec{x}^\top \vec{V} \vec{x} + 2\vec{u}^\top \vec{x} + f = 0
	\end{align}
	where
	\begin{equation}
		\vec{V} = \myvec{1 & 0\\0 & 1}, \quad \vec{u} = \myvec{0\\0}, \quad f = -16
	\end{equation}
	The center and radius are:
	\begin{equation}
		\vec{c} = -\vec{u} = \myvec{0\\0}, \quad
		r = \sqrt{\norm{\vec{u}}^2 - f} = \sqrt{0 + 16} = 4
	\end{equation}
The equation of a tangent line at a point of contact $\vec{q}$ on the circle is given by:
	\begin{equation}
		\brak{\vec{V}\vec{q} + \vec{u}}^\top \vec{x} + \vec{u}^\top \vec{q} + f = 0
	\end{equation}
	For this tangent to pass through the external point $P = \myvec{6\\0}$, the equation must hold when $\vec{x}=P$. Since $\vec{V}=\vec{I}$ and $\vec{u}=\vec{0}$:
	\begin{equation}
		\brak{\vec{I}\vec{q}}^\top P + f = 0
	\end{equation}
	\begin{equation}
		\vec{q}^\top P - 16 = 0
	\end{equation}
	
	Let $\vec{q} = \myvec{q_1\\q_2}$ be a point of contact. It must satisfy two conditions:
	\begin{enumerate}
		\item[(a)] $\vec{q}$ lies on the circle: $q_1^2 + q_2^2 = 16$
		\item[(b)] The tangent at $\vec{q}$ passes through $P$: $\vec{q}^\top P = 16$
	\end{enumerate}
From condition (a) \& (b):
	\begin{align}
		\myvec{q_1 & q_2}\myvec{6\\0} = 16 \implies 6q_1 = 16 \implies q_1 = \frac{8}{3}\\
		\brak{\frac{8}{3}}^2 + q_2^2 = 16 \implies \frac{64}{9} + q_2^2 = 16 \implies q_2^2 = \frac{80}{9}\\
		q_2 = \pm\frac{4\sqrt{5}}{3}
	\end{align}
The two points of contact are:
	\begin{equation}
		\vec{q}_1 = \myvec{\frac{8}{3}\\\frac{4\sqrt{5}}{3}}, \quad \vec{q}_2 = \myvec{\frac{8}{3}\\-\frac{4\sqrt{5}}{3}}
	\end{equation}
	The equation of the tangent at a point $\vec{q}$ is $\brak{\vec{V}\vec{q} + \vec{u}}^\top \vec{x} + \vec{u}^\top \vec{q} + f = 0$.\\
	\textbf{Tangent 1} at $\vec{q}_1 $:
	\begin{align}
		\vec{V}\vec{q}_1 &= \myvec{\frac{8}{3}\\\frac{4\sqrt{5}}{3}}\\
		\myvec{\frac{8}{3} & \frac{4\sqrt{5}}{3}}\myvec{x\\y} - 16 &= 0\\
		\frac{8}{3}x + \frac{4\sqrt{5}}{3}y &= 16\\
		2x + \sqrt{5}y &= 12
	\end{align}
	\textbf{Tangent 2} at $\vec{q}_2 $:
	\begin{align}
		\myvec{\frac{8}{3} & -\frac{4\sqrt{5}}{3}}\myvec{x\\y} - 16 &= 0\\
		2x - \sqrt{5}y &= 12
	\end{align}
The equations of tangents are:
	\begin{align}
		\boxed{2x + \sqrt{5}y = 12 \quad \text{and} \quad 2x - \sqrt{5}y = 12}
	\end{align}
\begin{figure}[H]
	\centering
	\includegraphics[width=1.1\linewidth]{figs/tangents_plot}
\end{figure}
	
	
\end{document}