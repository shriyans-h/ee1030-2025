\documentclass[article]{IEEEtran}
\usepackage[a5paper, margin=10mm, onecolumn]{geometry}

\usepackage{tfrupee} 
\setlength{\headheight}{1cm} 
\setlength{\headsep}{0mm}       
\usepackage{multicol}
\usepackage{gvv-book}
\usepackage{gvv}
\usepackage{cite}
\usepackage{amsmath,amssymb,amsfonts,amsthm}
\usepackage{algorithmic}
\usepackage{graphicx}
\usepackage{textcomp}
\usepackage{xcolor}
\usepackage{txfonts}
\usepackage{listings}
\usepackage{enumitem}
\usepackage{mathtools}
\usepackage{gensymb}
\usepackage{comment}
\usepackage[breaklinks=true]{hyperref}
\usepackage{tkz-euclide} 
\usepackage{listings}

\def\inputGnumericTable{}                                 
\usepackage[latin1]{inputenc}                                
\usepackage{color}                       

\usepackage{array}                                            
\usepackage{longtable}                                   

\usepackage{calc}                                             
\usepackage{multirow}                                         
\usepackage{hhline}          

\usepackage{ifthen}                                           
\usepackage{lscape}
\begin{document}
	\title{10.6.8}
	\author{EE25BTECH11052 - Shriyansh Kalpesh Chawda}
	\maketitle
	\textbf{Question}\\
	Construct a pair of tangents to a circle of radius 4cm from a point P lying outside the circle at 
	a distance of 6cm from the centre.
	\hfill{(10, 2023)}\\
	
	\textbf{Solution}\\
	Let the center of the circle be at origin and point P be at distance 6 from center along x-axis.
	\begin{align}
		\vec{O} &= \myvec{0\\0}\\
		\vec{h} &= \myvec{6\\0}
	\end{align}
	
	The equation of the circle $x^2 + y^2 = 16$ can be written in the general conic form (8.1.2.1):
	\begin{align}
		\vec{x}^\top \vec{V}\vec{x} + 2\vec{u}^\top\vec{x} + f &= 0
	\end{align}
	
	The parameters of the circle are:
	\begin{align}
		\vec{V} &= \vec{I} = \myvec{1 & 0\\0 & 1}\\
		\vec{u} &= \myvec{0\\0}\\
		f &= -16
	\end{align}
	
	The center and radius of the circle are:
	\begin{align}
		\vec{c} &= -\vec{u} = \myvec{0\\0}\\
		r &= \sqrt{\norm{\vec{u}}^2 - f} = \sqrt{0 - (-16)} = 4
	\end{align}
	
	\textbf{Finding Direction Vectors using Eigenvalue Decomposition:}\\
	A point $\vec{h}$ lies on a tangent to the conic if it satisfies formula (10.1.10.1):
	\begin{align}
		\vec{m}^\top\brak{\brak{\vec{V}\vec{h} + \vec{u}}\brak{\vec{V}\vec{h} + \vec{u}}^\top - \vec{V}g(\vec{h})}\vec{m} = 0
	\end{align}
	
	First, calculate $g(\vec{h})$:
	\begin{align}
		g(\vec{h}) &= \vec{h}^\top\vec{V}\vec{h} + 2\vec{u}^\top\vec{h} + f\\
		&= \myvec{6 & 0}\myvec{1 & 0\\0 & 1}\myvec{6\\0} + 0 - 16\\
		&= 36 - 16 = 20
	\end{align}
	
	Calculate $\vec{V}\vec{h} + \vec{u}$:
	\begin{align}
		\vec{V}\vec{h} + \vec{u} &= \myvec{1 & 0\\0 & 1}\myvec{6\\0} + \myvec{0\\0} = \myvec{6\\0}
	\end{align}
	
	Define matrix $\vec{Q}$:
	\begin{align}
		\vec{Q} &= \brak{\vec{V}\vec{h} + \vec{u}}\brak{\vec{V}\vec{h} + \vec{u}}^\top - \vec{V}g(\vec{h})\\
		&= \myvec{6\\0}\myvec{6 & 0} - 20\myvec{1 & 0\\0 & 1}\\
		&= \myvec{36 & 0\\0 & 0} - \myvec{20 & 0\\0 & 20}\\
		&= \myvec{16 & 0\\0 & -20}
	\end{align}
	
	\textbf{Eigenvalue Decomposition of Q:}\\
	The matrix $\vec{Q}$ is diagonal, so the eigenvalues are:
	\begin{align}
		\lambda_1 &= 16\\
		\lambda_2 &= -20
	\end{align}
	
	The eigenvector matrix is:
	\begin{align}
		\vec{P} = \vec{I} = \myvec{1 & 0\\0 & 1}
	\end{align}
	
	From the condition $\vec{m}^\top\vec{Q}\vec{m} = 0$:
	\begin{align}
		16m_1^2 - 20m_2^2 &= 0\\
		\frac{m_1^2}{m_2^2} &= \frac{20}{16} = \frac{5}{4}\\
		\frac{m_1}{m_2} &= \pm\frac{\sqrt{5}}{2}
	\end{align}
	
	The two direction vectors are:
	\begin{align}
		\vec{m}_1 &= \myvec{\sqrt{5}\\2}\\
		\vec{m}_2 &= \myvec{\sqrt{5}\\-2}
	\end{align}
	
	The corresponding normal vectors (perpendicular to direction vectors):
	\begin{align}
		\vec{n}_1 &= \myvec{-2\\\sqrt{5}} = -\myvec{2\\-\sqrt{5}}\\
		\vec{n}_2 &= \myvec{2\\\sqrt{5}}
	\end{align}
	
	Choosing proper signs:
	\begin{align}
		\vec{n}_1 &= \myvec{2\\-\sqrt{5}}\\
		\vec{n}_2 &= \myvec{2\\\sqrt{5}}
	\end{align}
	
	\textbf{Finding Points of Contact using Formula (10.1.8.1):}\\
	For a circle, the points of contact are given by:
	\begin{align}
		\vec{q}_{ij} = \pm r\frac{\vec{n}_j}{\norm{\vec{n}_j}} - \vec{u}, \quad i,j = 1,2
	\end{align}
	
	For $\vec{n}_1 = \myvec{2\\-\sqrt{5}}$:
	\begin{align}
		\norm{\vec{n}_1} &= \sqrt{4 + 5} = 3\\
		\vec{q}_1 &= r\frac{\vec{n}_1}{\norm{\vec{n}_1}} - \vec{u}\\
		&= 4 \cdot \frac{1}{3}\myvec{2\\-\sqrt{5}} - \myvec{0\\0}\\
		&= \myvec{\frac{8}{3}\\-\frac{4\sqrt{5}}{3}}
	\end{align}
	
	For $\vec{n}_2 = \myvec{2\\\sqrt{5}}$:
	\begin{align}
		\norm{\vec{n}_2} &= 3\\
		\vec{q}_2 &= 4 \cdot \frac{1}{3}\myvec{2\\\sqrt{5}} - \myvec{0\\0}\\
		&= \myvec{\frac{8}{3}\\\frac{4\sqrt{5}}{3}}
	\end{align}
	
	Therefore, the points of contact are:
	\begin{align}
		\vec{q}_1 &= \myvec{\frac{8}{3}\\-\frac{4\sqrt{5}}{3}}\\
		\vec{q}_2 &= \myvec{\frac{8}{3}\\\frac{4\sqrt{5}}{3}}
	\end{align}
	
	\textbf{Verification:}\\
	Check if points lie on circle:
	\begin{align}
		\norm{\vec{q}_1}^2 &= \brak{\frac{8}{3}}^2 + \brak{-\frac{4\sqrt{5}}{3}}^2\\
		&= \frac{64}{9} + \frac{80}{9} = \frac{144}{9} = 16 \quad \checkmark
	\end{align}
	
	Check if tangent from $\vec{h}$:
	\begin{align}
		\brak{\vec{h} - \vec{q}_1}^\top\vec{q}_1 &= \myvec{6 - \frac{8}{3} & 0 + \frac{4\sqrt{5}}{3}}\myvec{\frac{8}{3}\\-\frac{4\sqrt{5}}{3}}\\
		&= \frac{10}{3} \cdot \frac{8}{3} + \frac{4\sqrt{5}}{3} \cdot \brak{-\frac{4\sqrt{5}}{3}}\\
		&= \frac{80}{9} - \frac{80}{9} = 0 \quad \checkmark
	\end{align}
	
	\textbf{Equations of Tangents using Formula (10.1.2.1):}\\
	The equation of tangent at point $\vec{q}$ is:
	\begin{align}
		\brak{\vec{V}\vec{q} + \vec{u}}^\top\vec{x} + \vec{u}^\top\vec{q} + f = 0
	\end{align}
	
	For our circle with $\vec{V} = \vec{I}$ and $\vec{u} = \vec{0}$:
	\begin{align}
		\vec{q}^\top\vec{x} + f = 0\\
		\vec{q}^\top\vec{x} = 16
	\end{align}
	
	\textbf{Tangent 1} at $\vec{q}_1 = \myvec{\frac{8}{3}\\-\frac{4\sqrt{5}}{3}}$:
	\begin{align}
		\myvec{\frac{8}{3} & -\frac{4\sqrt{5}}{3}}\myvec{x\\y} &= 16\\
		\frac{8}{3}x - \frac{4\sqrt{5}}{3}y &= 16\\
		8x - 4\sqrt{5}y &= 48\\
		2x - \sqrt{5}y &= 12
	\end{align}
	
	\textbf{Tangent 2} at $\vec{q}_2 = \myvec{\frac{8}{3}\\\frac{4\sqrt{5}}{3}}$:
	\begin{align}
		\myvec{\frac{8}{3} & \frac{4\sqrt{5}}{3}}\myvec{x\\y} &= 16\\
		2x + \sqrt{5}y &= 12
	\end{align}
	
	The equations of the pair of tangents are:
	\begin{align}
		\boxed{2x - \sqrt{5}y = 12 \quad \text{and} \quad 2x + \sqrt{5}y = 12}
	\end{align}
	
	\textbf{Verification - Tangents pass through P:}
	\begin{align}
		2(6) - \sqrt{5}(0) &= 12 \quad \checkmark\\
		2(6) + \sqrt{5}(0) &= 12 \quad \checkmark
	\end{align}
	
	\begin{figure}[H]
		\centering
		\includegraphics[width=1.1\linewidth]{figs/tangents_plot}
	\end{figure}
	
\end{document}