\documentclass[article]{IEEEtran}
\usepackage[a5paper, margin=10mm, onecolumn]{geometry}

\usepackage{tfrupee} 
\setlength{\headheight}{1cm} 
\setlength{\headsep}{0mm}       
\usepackage{multicol}
\usepackage{gvv-book}
\usepackage{gvv}
\usepackage{cite}
\usepackage{amsmath,amssymb,amsfonts,amsthm}
\usepackage{algorithmic}
\usepackage{graphicx}
\usepackage{textcomp}
\usepackage{xcolor}
\usepackage{txfonts}
\usepackage{listings}
\usepackage{enumitem}
\usepackage{mathtools}
\usepackage{gensymb}
\usepackage{comment}
\usepackage[breaklinks=true]{hyperref}
\usepackage{tkz-euclide} 
\usepackage{listings}

\def\inputGnumericTable{}                                 
\usepackage[latin1]{inputenc}                                
\usepackage{color}                       

\usepackage{array}                                            
\usepackage{longtable}                                   

\usepackage{calc}                                             
\usepackage{multirow}                                         
\usepackage{hhline}          

\usepackage{ifthen}                                           
\usepackage{lscape}
\begin{document}
	\title{10.6.8}
	\author{EE25BTECH11052 - Shriyansh Kalpesh Chawda}
	\maketitle
	\textbf{Question}\\
	Construct a pair of tangents to a circle of radius 4cm from a point P lying outside the circle at 
	a distance of 6cm from the centre.
	\hfill{(10, 2023)}\\
	\textbf{Solution}\\
	Let the center of the circle be at origin. The equation is $x^2 + y^2 = 16$ and Point P is at distance 6 from center along x-axis.
	\begin{align}
		O &= \myvec{0\\0}\\
		P &= \myvec{6\\0} \\
		\vec{x}^\top \vec{V} \vec{x} + 2\vec{u}^\top \vec{x} + f &= 0
	\end{align}
	where
	\begin{equation}
		\vec{V} = \myvec{1 & 0\\0 & 1}, \quad \vec{u} = \myvec{0\\0}, \quad f = -16
	\end{equation}
\textbf{Eigenvalue Decomposition of $\vec{V}$}:\\
	The eigenvalue equation is:
	\begin{equation}
		\vec{V}\vec{p} = \lambda\vec{p}
	\end{equation}
The characteristic equation is:
	\begin{align}
		\det(\vec{V} - \lambda\vec{I}) &= 0\\
		\det\myvec{1-\lambda & 0\\0 & 1-\lambda} &= 0\\
		(1-\lambda)^2 &= 0\\
		\lambda_1 = \lambda_2 &= 1
	\end{align}
For $\lambda_1 = 1$, the eigenvector $\vec{p}_1$:
	\begin{align}
		(\vec{V} - \lambda_1\vec{I})\vec{p}_1 &= 0\\
		\myvec{0 & 0\\0 & 0}\myvec{p_{11}\\p_{12}} &= \myvec{0\\0}
	\end{align}
Choose $\vec{p}_1 = \myvec{1\\0}$ (normalized)
For $\lambda_2 = 1$, the eigenvector $\vec{p}_2$:
	\begin{equation}
		 \vec{p}_2 = \myvec{0\\1} \text{ (orthogonal to } \vec{p}_1\text{)}
	\end{equation}
The eigenvector matrix (orthogonal) is:
	\begin{equation}
		\vec{P} = \myvec{1 & 0\\0 & 1} = \vec{I}
	\end{equation}
	\textbf{Now using Spectral Decomposition:}\\
	\begin{align}
		\vec{V} &= \vec{P}\vec{D}\vec{P}^\top\\
		&= \myvec{1 & 0\\0 & 1}\myvec{1 & 0\\0 & 1}\myvec{1 & 0\\0 & 1}\\
		&= \vec{I}
	\end{align}
	where $\vec{D} = \myvec{\lambda_1 & 0\\0 & \lambda_2} = \myvec{1 & 0\\0 & 1}$
	
	\textbf{Step 3: Principal Axes Representation}\\
	The conic in principal axes coordinates: Since $\vec{P} = \vec{I}$ and $\vec{u} = \vec{0}$, the transformation is trivial.
	\begin{equation}
		\text{Let } \vec{y} = \vec{P}^\top(\vec{x} - \vec{c}), \text{ where } \vec{c} = -\vec{V}^{-1}\vec{u} = \myvec{0\\0}
	\end{equation}
	
	In principal axes:
	\begin{equation}
		\lambda_1 y_1^2 + \lambda_2 y_2^2 = -f \implies y_1^2 + y_2^2 = 16
	\end{equation}
	
	This confirms the circle has semi-axes along eigenvector directions with radii:
	\begin{equation}
		a = b = \sqrt{\frac{-f}{\lambda_1}} = \sqrt{\frac{16}{1}} = 4
	\end{equation}
	
	\textbf{Step 4: Finding Contact Points using Eigenvector Framework}\\
	Transform point $P$ to principal coordinates:
	\begin{equation}
		\vec{y}_P = \vec{P}^\top(P - \vec{c}) = \vec{I}\myvec{6\\0} = \myvec{6\\0}
	\end{equation}
	
	For tangent from external point, contact point $\vec{q}$ satisfies:
	\begin{enumerate}
		\item[(a)] $\vec{q}$ lies on circle: $\vec{q}^\top\vec{V}\vec{q} + f = 0$
		\item[(b)] Tangent passes through $P$: $(\vec{V}\vec{q})^\top P + f = 0$
	\end{enumerate}
	
	From condition (b) with $\vec{V} = \vec{I}$:
	\begin{align}
		\vec{q}^\top P + f &= 0\\
		\myvec{q_1 & q_2}\myvec{6\\0} &= 16\\
		6q_1 &= 16 \implies q_1 = \frac{8}{3}
	\end{align}
	
	From condition (a):
	\begin{align}
		q_1^2 + q_2^2 &= 16\\
		\brak{\frac{8}{3}}^2 + q_2^2 &= 16\\
		\frac{64}{9} + q_2^2 &= 16\\
		q_2^2 &= \frac{80}{9}\\
		q_2 &= \pm\frac{4\sqrt{5}}{3}
	\end{align}
Expressing contact points using eigenvectors:
	\begin{align}
		\vec{q}_1 &= \frac{8}{3}\vec{p}_1 + \frac{4\sqrt{5}}{3}\vec{p}_2 = \frac{8}{3}\myvec{1\\0} + \frac{4\sqrt{5}}{3}\myvec{0\\1} = \myvec{\frac{8}{3}\\\frac{4\sqrt{5}}{3}}\\
		\vec{q}_2 &= \frac{8}{3}\vec{p}_1 - \frac{4\sqrt{5}}{3}\vec{p}_2 = \frac{8}{3}\myvec{1\\0} - \frac{4\sqrt{5}}{3}\myvec{0\\1} = \myvec{\frac{8}{3}\\-\frac{4\sqrt{5}}{3}}
	\end{align}
	The tangent at $\vec{q}$ is: $(\vec{V}\vec{q})^\top \vec{x} + f = 0$
	
	\textbf{Tangent 1} at $\vec{q}_1$:
	\begin{align}
		\vec{V}\vec{q}_1 &= \myvec{\frac{8}{3}\\\frac{4\sqrt{5}}{3}}\\
		\myvec{\frac{8}{3} & \frac{4\sqrt{5}}{3}}\myvec{x\\y} - 16 &= 0\\
		\frac{8}{3}x + \frac{4\sqrt{5}}{3}y &= 16\\
		2x + \sqrt{5}y &= 12
	\end{align}
	
	\textbf{Tangent 2} at $\vec{q}_2$:
	\begin{align}
		\myvec{\frac{8}{3} & -\frac{4\sqrt{5}}{3}}\myvec{x\\y} - 16 &= 0\\
		2x - \sqrt{5}y &= 12
	\end{align}
	
	The equations of tangents are:
	\begin{align}
		\boxed{2x + \sqrt{5}y = 12 \quad \text{and} \quad 2x - \sqrt{5}y = 12}
	\end{align}
	
	\begin{figure}[H]
		\centering
		\includegraphics[width=1.1\linewidth]{figs/tangents_plot}
	\end{figure}
	
	
\end{document}