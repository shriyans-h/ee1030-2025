\documentclass{beamer}
\usepackage{tfrupee} 
\usepackage[utf8]{inputenc}
\usepackage{lmodern} 
\usepackage{listings}
\usepackage{xcolor} 
\usepackage{graphicx}
\usepackage{amsmath,amssymb,amsfonts}

\definecolor{myblue}{RGB}{48, 63, 159}
\setbeamercolor{palette primary}{bg=myblue, fg=white}
\setbeamercolor{structure}{fg=myblue}
\setbeamercolor{frametitle}{bg=myblue, fg=white}
\setbeamercolor{title}{bg=myblue, fg=white}
\setbeamercolor{footlinecolor}{bg=myblue, fg=white}

\defbeamertemplate*{title page}{mytemplate}{
	\vfill
	\begin{center}
		\begin{beamercolorbox}[wd=0.8\paperwidth, center, rounded=true, shadow=true]{title}
			\usebeamerfont{title}\inserttitle\par
		\end{beamercolorbox}
		\vspace{2cm} 
		\usebeamerfont{author}\insertauthor
		\vspace{1cm} 
		\usebeamerfont{date}\insertdate
	\end{center}
	\vfill
}

\defbeamertemplate*{frametitle}{mytemplate}{
	\begin{beamercolorbox}[wd=\paperwidth, ht=2.5ex, dp=1.5ex, left]{frametitle}
		\hspace{1em}\usebeamerfont{frametitle}\insertframetitle
	\end{beamercolorbox}
}

\setbeamertemplate{footline}{
	\begin{beamercolorbox}[wd=\paperwidth, ht=2.25ex, dp=1ex]{footlinecolor}
		\hspace{1em}\usebeamerfont{author in footline}\insertshortauthor
		\hfill
		\usebeamerfont{title in footline}\insertshorttitle
		\hfill
		\usebeamerfont{date in footline}\insertdate \hspace{1em} \insertframenumber/\inserttotalframenumber \hspace{0.5em}
	\end{beamercolorbox}
}

\setbeamerfont{author in footline}{size=\tiny}
\setbeamerfont{title in footline}{size=\tiny}
\setbeamerfont{date in footline}{size=\tiny}

\newcommand{\myvec}[1]{\ensuremath{\begin{pmatrix}#1\end{pmatrix}}}
\providecommand{\brak}[1]{\ensuremath{\left(#1\right)}}

\title{10.6.8 - Eigenvector Method}
\author{Shriyansh Chawda-EE25BTECH11052}

\begin{document}
	
	\setbeamertemplate{footline}{} 
	\frame{\titlepage}
	
	\begin{frame}{Question} 
		Construct a pair of tangents to a circle of radius 4cm from a point P lying outside the circle at 
		a distance of 6cm from the centre.
		\hfill{(10, 2023)}\\
		\vspace{1em}
		\textbf{Method: Using Eigenvector Decomposition}
	\end{frame}
	
	\begin{frame}{Problem Setup}
		Let the center of the circle be at origin. The equation is $x^2 + y^2 = 16$ and Point P is at distance 6 from center along x-axis.
		\begin{align}
			O &= \myvec{0\\0}\\
			P &= \myvec{6\\0} \\
			\vec{x}^\top \vec{V} \vec{x} + 2\vec{u}^\top \vec{x} + f &= 0
		\end{align}
		where
		\begin{equation}
			\vec{V} = \myvec{1 & 0\\0 & 1}, \quad \vec{u} = \myvec{0\\0}, \quad f = -16
		\end{equation}
	\end{frame}
	
	\begin{frame}{Step 1: Eigenvalue Decomposition}
		The eigenvalue equation is:
		\begin{equation}
			\vec{V}\vec{p} = \lambda\vec{p}
		\end{equation}
		The characteristic equation:
		\begin{align}
			\det(\vec{V} - \lambda\vec{I}) &= 0\\
			\det\myvec{1-\lambda & 0\\0 & 1-\lambda} &= 0\\
			(1-\lambda)^2 &= 0
		\end{align}
		\textbf{Eigenvalues:} $\lambda_1 = \lambda_2 = 1$
	\end{frame}
	
	\begin{frame}{Step 2: Finding Eigenvectors}
		For $\lambda_1 = 1$:
		\begin{align}
			(\vec{V} - \lambda_1\vec{I})\vec{p}_1 &= 0\\
			\myvec{0 & 0\\0 & 0}\myvec{p_{11}\\p_{12}} &= \myvec{0\\0}
		\end{align}
		Choose normalized eigenvector: $\vec{p}_1 = \myvec{1\\0}$
		
		\vspace{1em}
		For $\lambda_2 = 1$, choose orthogonal eigenvector:
		\begin{equation}
			\vec{p}_2 = \myvec{0\\1}
		\end{equation}
	\end{frame}
	
	\begin{frame}{Step 3: Eigenvector Matrix}
		The orthogonal eigenvector matrix is:
		\begin{equation}
			\vec{P} = \myvec{\vec{p}_1 & \vec{p}_2} = \myvec{1 & 0\\0 & 1} = \vec{I}
		\end{equation}
		
		\textbf{Spectral Decomposition:}
		\begin{align}
			\vec{V} &= \vec{P}\vec{D}\vec{P}^\top\\
			&= \myvec{1 & 0\\0 & 1}\myvec{1 & 0\\0 & 1}\myvec{1 & 0\\0 & 1} = \vec{I}
		\end{align}
		where $\vec{D} = \myvec{\lambda_1 & 0\\0 & \lambda_2} = \myvec{1 & 0\\0 & 1}$
	\end{frame}
	
	\begin{frame}{Step 4: Principal Axes Transformation}
		Transform to principal coordinates:
		\begin{equation}
			\vec{y} = \vec{P}^\top(\vec{x} - \vec{c})
		\end{equation}
		where $\vec{c} = -\vec{V}^{-1}\vec{u} = \myvec{0\\0}$
		
		\vspace{1em}
		In principal axes, the conic equation becomes:
		\begin{equation}
			\lambda_1 y_1^2 + \lambda_2 y_2^2 = -f
		\end{equation}
		\begin{equation}
			y_1^2 + y_2^2 = 16
		\end{equation}
	\end{frame}
	
	\begin{frame}{Step 5: Semi-axes from Eigenvalues}
		The radius along each eigenvector direction:
		\begin{equation}
			a = b = \sqrt{\frac{-f}{\lambda_1}} = \sqrt{\frac{16}{1}} = 4
		\end{equation}
		
		This confirms:
		\begin{itemize}
			\item Circle is symmetric in all directions
			\item Eigenvectors $\vec{p}_1$ and $\vec{p}_2$ form principal axes
			\item Radius = 4 cm along both axes
		\end{itemize}
	\end{frame}
	
	\begin{frame}{Step 6: Transform Point P}
		Transform external point $P$ to principal coordinates:
		\begin{equation}
			\vec{y}_P = \vec{P}^\top(P - \vec{c}) = \vec{I}\myvec{6\\0} = \myvec{6\\0}
		\end{equation}
		
		\vspace{1em}
		For tangent from external point, contact point $\vec{q}$ must satisfy:
		\begin{enumerate}
			\item[(a)] $\vec{q}$ lies on circle: $\vec{q}^\top\vec{V}\vec{q} + f = 0$
			\item[(b)] Tangent passes through $P$: $(\vec{V}\vec{q})^\top P + f = 0$
		\end{enumerate}
	\end{frame}
	
	\begin{frame}{Step 7: Finding Contact Points}
		From condition (b) with $\vec{V} = \vec{I}$:
		\begin{align}
			\vec{q}^\top P + f &= 0\\
			\myvec{q_1 & q_2}\myvec{6\\0} &= 16\\
			6q_1 &= 16 \implies q_1 = \frac{8}{3}
		\end{align}
		
		From condition (a):
		\begin{align}
			q_1^2 + q_2^2 &= 16\\
			\brak{\frac{8}{3}}^2 + q_2^2 &= 16\\
			q_2^2 &= \frac{80}{9} \implies q_2 = \pm\frac{4\sqrt{5}}{3}
		\end{align}
	\end{frame}
	
	\begin{frame}{Step 8: Express in Eigenvector Basis}
		The contact points expressed as linear combinations of eigenvectors:
		\begin{align}
			\vec{q}_1 &= \frac{8}{3}\vec{p}_1 + \frac{4\sqrt{5}}{3}\vec{p}_2\\
			&= \frac{8}{3}\myvec{1\\0} + \frac{4\sqrt{5}}{3}\myvec{0\\1}\\
			&= \myvec{\frac{8}{3}\\\frac{4\sqrt{5}}{3}}
		\end{align}
		
		\begin{align}
			\vec{q}_2 &= \frac{8}{3}\vec{p}_1 - \frac{4\sqrt{5}}{3}\vec{p}_2\\
			&= \myvec{\frac{8}{3}\\-\frac{4\sqrt{5}}{3}}
		\end{align}
	\end{frame}
	
	\begin{frame}{Step 9: Tangent Equations}
		The tangent at $\vec{q}$ is: $(\vec{V}\vec{q})^\top \vec{x} + f = 0$
		
		\vspace{1em}
		\textbf{Tangent 1} at $\vec{q}_1$:
		\begin{align}
			\vec{V}\vec{q}_1 &= \myvec{\frac{8}{3}\\\frac{4\sqrt{5}}{3}}\\
			\myvec{\frac{8}{3} & \frac{4\sqrt{5}}{3}}\myvec{x\\y} - 16 &= 0\\
			\frac{8}{3}x + \frac{4\sqrt{5}}{3}y &= 16\\
			2x + \sqrt{5}y &= 12
		\end{align}
	\end{frame}
	
	\begin{frame}{Step 10: Second Tangent}
		\textbf{Tangent 2} at $\vec{q}_2$:
		\begin{align}
			\myvec{\frac{8}{3} & -\frac{4\sqrt{5}}{3}}\myvec{x\\y} - 16 &= 0\\
			2x - \sqrt{5}y &= 12
		\end{align}
		
		\vspace{1em}
		\textbf{Final Answer:}
		\begin{equation}
			\boxed{2x + \sqrt{5}y = 12 \quad \text{and} \quad 2x - \sqrt{5}y = 12}
		\end{equation}
	\end{frame}
	
	\begin{frame}{Summary of Eigenvector Method}
		\begin{enumerate}
			\item Found eigenvalues: $\lambda_1 = \lambda_2 = 1$
			\item Computed eigenvectors: $\vec{p}_1 = \myvec{1\\0}$, $\vec{p}_2 = \myvec{0\\1}$
			\item Spectral decomposition: $\vec{V} = \vec{P}\vec{D}\vec{P}^\top$
			\item Transformed to principal axes
			\item Calculated semi-axes using: $a = \sqrt{-f/\lambda_1}$
			\item Found contact points in eigenvector basis
			\item Derived tangent equations
		\end{enumerate}
		
		\vspace{0.5em}
		\textbf{Key Insight:} Eigenvectors define the natural coordinate system for the conic!
	\end{frame}
	
	\begin{frame}{Plot}
		\begin{figure}[H]
			\centering
			\includegraphics[width=0.85\linewidth]{figs/tangents_plot}
			\caption{Circle with tangents from external point P}
			\label{fig:tangentsplot}
		\end{figure}
	\end{frame}
	
\end{document}