\let\negmedspace\undefined
\let\negthickspace\undefined
\documentclass[journal]{IEEEtran}
\usepackage[a5paper, margin=10mm, onecolumn]{geometry}
%\usepackage{lmodern} % Ensure lmodern is loaded for pdflatex
\usepackage{tfrupee} % Include tfrupee package

\setlength{\headheight}{1cm} % Set the height of the header box
\setlength{\headsep}{0mm}     % Set the distance between the header box and the top of the text

\usepackage{gvv-book}
\usepackage{gvv}
\usepackage{cite}
\usepackage{amsmath,amssymb,amsfonts,amsthm}
\usepackage{algorithmic}
\usepackage{graphicx}
\usepackage{textcomp}
\usepackage{xcolor}
\usepackage{txfonts}
\usepackage{listings}
\usepackage{enumitem}
\usepackage{mathtools}
\usepackage{gensymb}
\usepackage{comment}
\usepackage[breaklinks=true]{hyperref}
\usepackage{tkz-euclide} 
\usepackage{listings}
% \usepackage{gvv}                                        
\def\inputGnumericTable{}                                 
\usepackage[latin1]{inputenc}                                
\usepackage{color}                                            
\usepackage{array}                                            
\usepackage{longtable}                                       
\usepackage{calc}                                             
\usepackage{multirow}                                         
\usepackage{hhline}                                           
\usepackage{ifthen}                                           
\usepackage{lscape}
\begin{document}
\bibliographystyle{IEEEtran}
\vspace{3cm}

\title{6.2.6}
\author{EE25BTECH11060 - V.Namaswi}
% \maketitle
% \newpage
% \bigskip
{\let\newpage\relax\maketitle}
\renewcommand{\thefigure}{\theenumi}
\renewcommand{\thetable}{\theenumi}
\setlength{\intextsep}{10pt} % Space between text and floats
\textbf{Question}\\
Find matrix X such that\\
\begin{align}
   X \begin{pmatrix}
        1 & 2 & 3\\
        4 & 5 & 6
    \end{pmatrix}= \begin{pmatrix}
        -7 & -8 & -9 \\ 2 & 4 & 6
    \end{pmatrix}
\end{align}\\
\textbf{Solution}\\
As X is a 2x2 matrix,\\
First solving for Row 1\\
Formation of Argumented Matrix 
\begin{align}
    \augvec{2}{1}{1 & 4 & -7 \\ 2 & 5 & -8 \\ 3 & 6 & -9}
\end{align}
Replace $R_2 \to R_2 - 2R_1$
\begin{align}
    \augvec{2}{1}{1 & 4 & -7 \\ 0 & -3 &  6 \\ 3 & 6 & -9}
\end{align}
Replace $R_3 \to R_3 - 3R_1$
\begin{align}
    \augvec{2}{1}{1 & 4 & -7 \\ 0 & -3 &  6 \\ 0 & -6 & 12}
\end{align}
Replace $R_3 \to R_3 - 2R_2$
\begin{align}
    \augvec{2}{1}{1 & 4 & -7 \\ 0 & -3 &  6 \\ 0 & 0 & 0}
\end{align}
So,Row 1 
\begin{align}
    \begin{pmatrix}
        1 & -2
    \end{pmatrix}
\end{align}
Solving for Row 2 \\
Formation of Argumented Matrix
\begin{align}
      \augvec{2}{1}{1 & 4 & -7 \\ 2 & 5 & -8 \\ 3 & 6 & -9}
\end{align}
Replace $R_3 \to R_3-R_2$
\begin{align}
\augvec{2}{1}{1 & 4 & -7 \\ 2 & 5 & -8 \\ 1 & 1 & -1}
\end{align}
Replace $R_2 \to R_2-(R_1+R_3)$
\begin{align}
    \augvec{2}{1}{1 & 4 & -7 \\ 0 & 0 & 0 \\ 1 & 1 & -1}
\end{align}
Replace $R_3 \to R_3-R_1$
\begin{align}
    \augvec{2}{1}{1 & 4 & -7 \\ 0 & 0 & 0 \\ 0 & -3 & 6}
\end{align}
So,Row 2
\begin{align}
    \begin{pmatrix}
        1 & -2
    \end{pmatrix}
\end{align}
Hence $\Vec{X}$
\begin{align}
    =\begin{pmatrix}
        1 & -2 \\ 1 & -2
\end{pmatrix}
\end{align}
    \centering
    \includegraphics[width=\columnwidth, height=0.8\textheight, keepaspectratio]{figs/Figure_13.png}  
\end{document}
