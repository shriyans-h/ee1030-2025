\documentclass{beamer}
\usepackage[utf8]{inputenc}

\usetheme{Madrid}
\usecolortheme{default}
\usepackage{amsmath,amssymb,amsfonts,amsthm}
\usepackage{txfonts}
\usepackage{tkz-euclide}
\usepackage{listings}
\usepackage{adjustbox}
\usepackage{array}
\usepackage{tabularx}
\usepackage{gvv}
\usepackage{lmodern}
\usepackage{circuitikz}
\usepackage{tikz}
\usepackage{graphicx}

\setbeamertemplate{page number in head/foot}[totalframenumber]

\usepackage{tcolorbox}
\tcbuselibrary{minted,breakable,xparse,skins}



\definecolor{bg}{gray}{0.95}
\DeclareTCBListing{mintedbox}{O{}m!O{}}{%
  breakable=true,
  listing engine=minted,
  listing only,
  minted language=#2,
  minted style=default,
  minted options={%
    linenos,
    gobble=0,
    breaklines=true,
    breakafter=,,
    fontsize=\small,
    numbersep=8pt,
    #1},
  boxsep=0pt,
  left skip=0pt,
  right skip=0pt,
  left=25pt,
  right=0pt,
  top=3pt,
  bottom=3pt,
  arc=5pt,
  leftrule=0pt,
  rightrule=0pt,
  bottomrule=2pt,
  toprule=2pt,
  colback=bg,
  colframe=orange!70,
  enhanced,
  overlay={%
    \begin{tcbclipinterior}
    \fill[orange!20!white] (frame.south west) rectangle ([xshift=20pt]frame.north west);
    \end{tcbclipinterior}},
  #3,
}
\lstset{
    language=C,
    basicstyle=\ttfamily\small,
    keywordstyle=\color{blue},
    stringstyle=\color{orange},
    commentstyle=\color{green!60!black},
    numbers=left,
    numberstyle=\tiny\color{gray},
    breaklines=true,
    showstringspaces=false,
}
\begin{document}

\title 
{3.2.1}
\date{september 10,2025}


\author 
{Namaswi-EE25BTECH11060}
\frame{\titlepage}
\begin{frame}{Question}
 Draw a triangle ABC in which AB=4cm,BC=6cm and AC=9cm.
\end{frame}
\begin{frame}{solution}
 According to given data lets assume,\\
\begin{align*}
    A=\begin{pmatrix}0\\0\end{pmatrix}\qquad 
B=\begin{pmatrix}4\\0\end{pmatrix}\qquad 
C=\begin{pmatrix}x\\y\end{pmatrix}   
\end{align*}

 
\end{frame}

\begin{frame}{solution}
  \begin{align}
    ||C-A||=9\\
    C^\top C=81\\
    ||C-B||=6\\
    (C-B)^\top (C-B)=36\\
    C^\top C-2 B^\top C +B^\top B=36\\
    as,B^\top B =16 \; ; C^\top C=81\\
    2B^\top C =61\\
    \begin{pmatrix}
        8 & 0
    \end{pmatrix}C=61\\
    Augmented\;Matrix \implies
    \begin{pmatrix}
        8 & 0 \;| \;61
    \end{pmatrix}\\
    \implies
    \begin{pmatrix}
        1 & 0 \;|\; 61/8
    \end{pmatrix}\\
    x=61/8
\end{align}
\end{frame}
\begin{frame}{Solution}
\begin{align}
     as,C^\top C =81\\
    x^2+y^2=81\\
    y=\sqrt{\frac{1463}{64}}\\
    C=\begin{pmatrix}
        7.625\\ \pm4.781
    \end{pmatrix}
\end{align}
Refer fig
\end{frame}

\begin{frame}[fragile]
    \frametitle{C Code }

    \begin{lstlisting}

 #include <stdio.h>
#include <math.h>

int main() {
    // Given distances (according to your solution)
    double AC = 9.0, BC = 6.0;
    double AB = 4.0;

    // Fix A and B
    double Ax = 0.0, Ay = 0.0;
    double Bx = 4.0, By = 0.0;

    // Step 1: Find x using the relation (from your derivation)
    // 2 * B^T * C = AC^2 + B^T B - BC^2
    double x = (AC*AC + (Bx*Bx + By*By) - BC*BC) / (2 * Bx);

 \end{lstlisting}
\end{frame}

\begin{frame}[fragile]
    \frametitle{C Code}
    \begin{lstlisting}
     // Step 2: Use AC^2 = x^2 + y^2 to solve for y
    double y_square = AC*AC - x*x;

    if (y_square < 0) {
        printf("No real solution (triangle inequality violated).\n");
        return 0;
    }

    double y1 = sqrt(y_square);
    double y2 = -sqrt(y_square);

    \end{lstlisting}
\end{frame}

\begin{frame}[fragile]
    \frametitle{C Code}
    \begin{lstlisting}
  // Print results
    printf("Coordinates of A: (%.3f, %.3f)\n", Ax, Ay);
    printf("Coordinates of B: (%.3f, %.3f)\n", Bx, By);
    printf("Possible coordinates of C:\n");
    printf("C1 = (%.3f, %.3f)\n", x, y1);
    printf("C2 = (%.3f, %.3f)\n", x, y2);

    return 0;
}
    \end{lstlisting}
\end{frame}

\begin{frame}[fragile]
    \frametitle{Python Code}
    \begin{lstlisting}
 import matplotlib.pyplot as plt

# Define vertices
A = (0, 0)
B = (4, 0)
C = (7.625, 4.781)

# Collect coordinates (close the triangle by repeating A at the end)
x_coords = [A[0], B[0], C[0], A[0]]
y_coords = [A[1], B[1], C[1], A[1]]

# Plot the triangle
plt.figure(figsize=(6,6))
plt.plot(x_coords, y_coords, 'b-', linewidth=2)      # Triangle edges
plt.fill(x_coords, y_coords, 'skyblue', alpha=0.3)   # Fill inside


    \end{lstlisting}
\end{frame}

\begin{frame}[fragile]
    \frametitle{Python Code}
    \begin{lstlisting}
 # Plot vertices
plt.scatter(*A, color='red', s=60)
plt.scatter(*B, color='green', s=60)
plt.scatter(*C, color='purple', s=60)

# Label points
plt.text(A[0]-0.3, A[1]-0.3, 'A', fontsize=12, fontweight='bold')
plt.text(B[0]+0.2, B[1]-0.3, 'B', fontsize=12, fontweight='bold')
plt.text(C[0]+0.2, C[1]+0.2, 'C', fontsize=12, fontweight='bold')

# Formatting
plt.axhline(0, color='gray', linewidth=0.5)
plt.axvline(0, color='gray', linewidth=0.5)
plt.gca().set_aspect('equal', adjustable='box')
plt.title("Triangle ABC")
plt.grid(True)
plt.show()



    \end{lstlisting}
\end{frame}
\begin{frame}[fragile]
    \frametitle{C and Python Code}
    \begin{lstlisting}
 import ctypes
import os

# Load shared object
lib = ctypes.CDLL(os.path.abspath("./triangle.so"))

# Define argument and return types
lib.find_triangle.argtypes = [ctypes.c_double, ctypes.c_double, ctypes.c_double,
                              ctypes.POINTER(ctypes.c_double),
                              ctypes.POINTER(ctypes.c_double),
                              ctypes.POINTER(ctypes.c_double)]

    \end{lstlisting}
\end{frame}
 \begin{frame}[fragile]
    \frametitle{C and Python Code}
    \begin{lstlisting}
 # Function wrapper
def find_triangle(AB, AC, BC):
    x = ctypes.c_double()
    y1 = ctypes.c_double()
    y2 = ctypes.c_double()
    
    lib.find_triangle(AB, AC, BC,
                      ctypes.byref(x), ctypes.byref(y1), ctypes.byref(y2))
    
    return x.value, y1.value, y2.value


    \end{lstlisting}
\end{frame}
 \begin{frame}[fragile]
    \frametitle{C and Python Code}
    \begin{lstlisting}
 
# Example usage
if __name__ == "__main__":
    AB, AC, BC = 4.0, 9.0, 6.0   # Example
    x, y1, y2 = find_triangle(AB, AC, BC)
    
    print(f"A = (0,0)")
    print(f"B = ({AB},0)")
    print(f"C1 = ({x:.3f},{y1:.3f})")
    print(f"C2 = ({x:.3f},{y2:.3f})")


    \end{lstlisting}
\end{frame}
  


\begin{frame}{Plot}
    \centering
    \includegraphics[width=\columnwidth, height=0.8\textheight, keepaspectratio]{Figure_5.png}     
\end{frame}
\end{document}