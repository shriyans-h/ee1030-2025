\let\negmedspace\undefined
\let\negthickspace\undefined
\documentclass[journal]{IEEEtran}
\usepackage[a5paper, margin=10mm, onecolumn]{geometry}
%\usepackage{lmodern} % Ensure lmodern is loaded for pdflatex
\usepackage{tfrupee} % Include tfrupee package

\setlength{\headheight}{1cm} % Set the height of the header box
\setlength{\headsep}{0mm}     % Set the distance between the header box and the top of the text

\usepackage{gvv-book}
\usepackage{gvv}
\usepackage{cite}
\usepackage{amsmath,amssymb,amsfonts,amsthm}
\usepackage{algorithmic}
\usepackage{graphicx}
\usepackage{textcomp}
\usepackage{xcolor}
\usepackage{txfonts}
\usepackage{listings}
\usepackage{enumitem}
\usepackage{mathtools}
\usepackage{gensymb}
\usepackage{comment}
\usepackage[breaklinks=true]{hyperref}
\usepackage{tkz-euclide} 
\usepackage{listings}
% \usepackage{gvv}                                        
\def\inputGnumericTable{}                                 
\usepackage[latin1]{inputenc}                                
\usepackage{color}                                            
\usepackage{array}                                            
\usepackage{longtable}                                       
\usepackage{calc}                                             
\usepackage{multirow}                                         
\usepackage{hhline}                                           
\usepackage{ifthen}                                           
\usepackage{lscape}
\begin{document}

\bibliographystyle{IEEEtran}
\vspace{3cm}

\title{4.3.50}
\author{EE25BTECH11060 - V.Namaswi}
% \maketitle
% \newpage
% \bigskip
{\let\newpage\relax\maketitle}

\renewcommand{\thefigure}{\theenumi}
\renewcommand{\thetable}{\theenumi}
\setlength{\intextsep}{10pt} % Space between text and floats
\textbf{Question}\\ Find the equation of the lines which makes intercepts -3 and 2 on the x and y axes respectively.\\
\textbf{Solution}\\
Let $\brak{-3,0}$  and  $\brak{0,2}$  be the intercept points\\
\begin{align}
\Vec{m}=\begin{pmatrix}
    -3 \\ 0
\end{pmatrix}-\begin{pmatrix}
    0  \\  2
\end{pmatrix}\\
\Vec{m}=\begin{pmatrix}
    1 \\ \frac{2}{3}
\end{pmatrix}\\
\Vec{n}=\begin{pmatrix}
    \frac{-2}{3} 
\end{pmatrix}
\end{align}

Equation \; of \;line \;is\; given by $n^\top(x-h)=0$
\begin{align}
\begin{pmatrix}
    \frac{-2}{3} & 1
\end{pmatrix}\begin{pmatrix}
    x - \begin{pmatrix}
        0 \\ 2
    \end{pmatrix}
\end{pmatrix}=0\\
\begin{pmatrix}
    \frac{-2}{3} & 1
\end{pmatrix}x=2
\end{align}
Refer fig
 \centering
\includegraphics[width=\columnwidth, height=0.8\textheight, keepaspectratio]{figs/Figure_6.png}     

 

\end{document}