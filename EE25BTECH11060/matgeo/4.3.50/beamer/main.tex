\documentclass{beamer}
\usepackage[utf8]{inputenc}

\usetheme{Madrid}
\usecolortheme{default}
\usepackage{amsmath,amssymb,amsfonts,amsthm}
\usepackage{txfonts}
\usepackage{tkz-euclide}
\usepackage{listings}
\usepackage{adjustbox}
\usepackage{array}
\usepackage{tabularx}
\usepackage{gvv}
\usepackage{lmodern}
\usepackage{circuitikz}
\usepackage{tikz}
\usepackage{graphicx}

\setbeamertemplate{page number in head/foot}[totalframenumber]

\usepackage{tcolorbox}
\tcbuselibrary{minted,breakable,xparse,skins}



\definecolor{bg}{gray}{0.95}
\DeclareTCBListing{mintedbox}{O{}m!O{}}{%
  breakable=true,
  listing engine=minted,
  listing only,
  minted language=#2,
  minted style=default,
  minted options={%
    linenos,
    gobble=0,
    breaklines=true,
    breakafter=,,
    fontsize=\small,
    numbersep=8pt,
    #1},
  boxsep=0pt,
  left skip=0pt,
  right skip=0pt,
  left=25pt,
  right=0pt,
  top=3pt,
  bottom=3pt,
  arc=5pt,
  leftrule=0pt,
  rightrule=0pt,
  bottomrule=2pt,
  toprule=2pt,
  colback=bg,
  colframe=orange!70,
  enhanced,
  overlay={%
    \begin{tcbclipinterior}
    \fill[orange!20!white] (frame.south west) rectangle ([xshift=20pt]frame.north west);
    \end{tcbclipinterior}},
  #3,
}
\lstset{
    language=C,
    basicstyle=\ttfamily\small,
    keywordstyle=\color{blue},
    stringstyle=\color{orange},
    commentstyle=\color{green!60!black},
    numbers=left,
    numberstyle=\tiny\color{gray},
    breaklines=true,
    showstringspaces=false,
}
\begin{document}

\title 
{4.3.50}
\date{september 14,2025}


\author 
{Namaswi-EE25BTECH11060}
\frame{\titlepage}
\begin{frame}{Question}
Find the equation of the lines which makes intercepts -3 and 2 on the x and y axes respectively
\end{frame}
\begin{frame}{solution}
 Let $\brak{-3,0}$  and  $\brak{0,2}$  be the intercept points\\
\begin{align}
\Vec{m}=\begin{pmatrix}
    -3 \\ 0
\end{pmatrix}-\begin{pmatrix}
    0  \\  2
\end{pmatrix}\\
\Vec{m}=\begin{pmatrix}
    1 \\ \frac{2}{3}
\end{pmatrix}\\
\Vec{n}=\begin{pmatrix}
    \frac{-2}{3} 
\end{pmatrix}
\end{align}
\end{frame}

\begin{frame}{solution}
 Equation \; of \;line \;is\; given by $n^\top(x-h)=0$
\begin{align}
\begin{pmatrix}
    \frac{-2}{3} & 1
\end{pmatrix}\begin{pmatrix}
    x - \begin{pmatrix}
        0 \\ 2
    \end{pmatrix}
\end{pmatrix}=0\\
\begin{pmatrix}
    \frac{-2}{3} & 1
\end{pmatrix}x=2
\end{align}
\end{frame}
 

\begin{frame}[fragile]
    \frametitle{C Code}
    \begin{lstlisting}
#include <stdio.h>
int main() {
    // Intercept points
    double A[2] = {-3, 0};   // x-intercept (-3,0)
    double B[2] = {0, 2};    // y-intercept (0,2)

    // Direction vector m = A - B
    double m[2];
    m[0] = A[0] - B[0];
    m[1] = A[1] - B[1];

    printf("Direction vector m = (%.2f, %.2f)\n", m[0], m[1]);

    // Normal vector n (perpendicular to m)
    double n[2];
    n[0] = -m[1];   // -y
    n[1] = m[0];    // x
\end{lstlisting}
\end{frame}
\begin{frame}[fragile]
    \frametitle{C Code }
    \begin{lstlisting}
 printf("Normal vector n = (%.2f, %.2f)\n", n[0], n[1]);

    // Point h (we take y-intercept B as reference point)
    double h[2] = {0, 2};

    // Equation: n^T * (x - h) = 0
    // Expanding: n^T * x = n^T * h
    double c = n[0]*h[0] + n[1]*h[1];

    printf("Equation of line: %.2fx + %.2fy = %.2f\n", n[0], n[1], c);

    return 0;
}
\end{lstlisting}
\end{frame}

 

\begin{frame}[fragile]
\frametitle{Python Code}
\begin{lstlisting}
  import matplotlib.pyplot as plt
import numpy as np

# Line equation: -2x + 3y = 6
# Solve for y: y = (2x + 6)/3

# Define x values for plotting
x = np.linspace(-10, 10, 400)
y = (2 * x + 6) / 3

# Find intercepts
# X-intercept: set y = 0 → -2x = 6 → x = -3 → A = (-3, 0)
# Y-intercept: set x = 0 → 3y = 6 → y = 2 → B = (0, 2)
A = (-3, 0)
B = (0, 2)
\end{lstlisting}
\end{frame}

\begin{frame}[fragile]
\frametitle{Python Code}
\begin{lstlisting}
  # Plot the line
plt.plot(x, y, label='Line: -2x + 3y = 6', color='blue')

# Mark the intercepts
plt.scatter(*A, color='red', zorder=5)
plt.scatter(*B, color='green', zorder=5)

# Annotate the points
plt.text(A[0]-1, A[1]-0.5, f'A {A}', color='red', fontsize=12)
plt.text(B[0]+0.2, B[1]+0.2, f'B {B}', color='green', fontsize=12)

# Axes lines
plt.axhline(0, color='black', linewidth=1)
plt.axvline(0, color='black', linewidth=1)

\end{lstlisting}
\end{frame}
\begin{frame}[fragile]
    \frametitle{Python Code}
    \begin{lstlisting}
# Graph settings
plt.title('Graph of the Line -2x + 3y = 6')
plt.xlabel('x-axis')
plt.ylabel('y-axis')
plt.grid(True)
plt.legend()
plt.axis('equal')
plt.xlim(-10, 10)
plt.ylim(-10, 10)

# Show the plot
plt.show()

\end{lstlisting}
\end{frame}
\begin{frame}[fragile]
\frametitle{C and Python Code}
\begin{lstlisting}
 import ctypes
import os

# Load the shared object file
lib_path = os.path.abspath("liblineeq.so")
lib = ctypes.CDLL(lib_path)

# Define the function's argument types
lib.line_from_intercepts.argtypes = [ctypes.c_double, ctypes.c_double]

# Optional: Define the return type (void function, so None)
lib.line_from_intercepts.restype = None

\end{lstlisting}
\end{frame}
\begin{frame}[fragile]
\frametitle{C and Python Code}
\begin{lstlisting}
 # Example intercepts
x_intercept = -3.0
y_intercept = 2.0

print("Calling C function from Python with:")
print(f"  X-intercept = {x_intercept}")
print(f"  Y-intercept = {y_intercept}\n")

# Call the function
lib.line_from_intercepts(x_intercept, y_intercept)
\end{lstlisting}
\end{frame}

\begin{frame}{Plot}
    \centering
    \includegraphics[width=\columnwidth, height=0.8\textheight, keepaspectratio]{Figure_6.png}     
\end{frame}
\end{document}