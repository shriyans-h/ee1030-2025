\documentclass[journal]{IEEEtran}
\usepackage[a5paper, margin=10mm, onecolumn]{geometry}
\usepackage{lmodern}

\setlength{\headheight}{1cm}
\setlength{\headsep}{0mm}

\usepackage{gvv-book}
\usepackage{gvv}
\usepackage{cite}
\usepackage{amsmath,amssymb,amsfonts,amsthm}
\usepackage{graphicx}
\graphicspath{{./figs/}}
\usepackage{xcolor}
\usepackage{txfonts}
\usepackage{enumitem}
\usepackage{mathtools}
\usepackage{hyperref}
\usepackage{tikz}
\usepackage{tkz-euclide}

\begin{document}

\bibliographystyle{IEEEtran}
\vspace{3cm}

\title{4.13.41}
\author{EE25BTECH11036 - M Chanakya Srinivas}
\maketitle

\renewcommand{\thetable}{\theenumi}
\setlength{\intextsep}{10pt}
\renewcommand\theequation{\arabic{equation}}


\section*{Problem}
Find the area of the parallelogram formed by the lines
\[
y = m x, \quad y = m x + 1, \quad y = n x, \quad y = n x + 1
\]
.

\section*{Solution }

\subsection*{Step 1: Represent lines in parametric vector form}

\begin{align}
\vec{r}_1 &= \kappa_1 \myvec{1\\m}, & 
\vec{r}_2 &= \myvec{0\\1} + \kappa_2 \myvec{1\\m}, \\
\vec{r}_3 &= \mu_1 \myvec{1\\n}, & 
\vec{r}_4 &= \myvec{0\\1} + \mu_2 \myvec{1\\n}.
\end{align}

Here, \(\vec{r}_1, \vec{r}_2\) represent the pair of lines with slope \(m\), and
\(\vec{r}_3, \vec{r}_4\) represent the pair of lines with slope \(n\).

\subsection*{Step 2: Compute vertices of parallelogram using intersection}

Intersection of \(\vec{r}_1\) and \(\vec{r}_3\):
\[
\vec{P} : \kappa_1 \myvec{1\\m} = \mu_1 \myvec{1\\n} \quad \Rightarrow \quad \vec{P} = \myvec{0\\0}
\]

Intersection of \(\vec{r}_2\) and \(\vec{r}_3\):
\[
\vec{Q} : \myvec{0\\1} + \kappa_2 \myvec{1\\m} = \mu_1 \myvec{1\\n} \quad \Rightarrow 
\myvec{ \kappa_2 \\ 1 + m \kappa_2 } = 
\myvec{ \mu_1 \\ n \mu_1 }
\]

Solve augmented matrix form :
\[
\underbrace{\myvec{ 1 & -1 \\ m & -n }}_{M} 
\underbrace{\myvec{\kappa_2 \\ \mu_1}}_{\vec{z}} = 
\underbrace{\myvec{0 \\ -1}}_{\vec{b}}
\]

\subsection*{Step 3: Row-reduction / consistency}

\begin{align*}
R_2 &\to R_2 - m R_1: & (m - n) \mu_1 &= m - 1 \quad \Rightarrow \quad \mu_1 = \frac{1}{m-n} \\
R_1 &\to R_1: & \kappa_2 - \mu_1 &= 0 \quad \Rightarrow \quad \kappa_2 = \frac{1}{m-n}
\end{align*}

Hence
\[
\vec{Q} = \myvec{-1/(m-n)\\-m/(m-n)}
\]

Similarly, intersections give the other vertices
\[
\vec{R} = \myvec{1/(m-n)\\m/(m-n)}, \quad \vec{S} = \myvec{0\\1}
\]

\subsection*{Step 4: Area using vector cross product (Chapter 4 vector style)}

\begin{align}
\vec{PQ} &= \vec{Q} - \vec{P} = \myvec{-1/(m-n)\\-m/(m-n)}, \\
\vec{PR} &= \vec{R} - \vec{P} = \myvec{1/(m-n)\\ m/(m-n)}, \\
\text{Area} &= \big\| \vec{PQ} \times \vec{PR} \big\| \\
&= \left| \left(-\frac{1}{m-n}\right)\left(\frac{m}{m-n}\right) - \left(-\frac{m}{m-n}\right) \left(\frac{1}{m-n}\right) \right| \\
&= \frac{1}{|m-n|}
\end{align}

\subsection*{Answer}
\[
\boxed{\text{Area of the parallelogram} = \frac{1}{|m-n|}}
\]
\begin{figure}[h]
    \centering
    \includegraphics[width=0.9\columnwidth]{figs/Figure_81.png}
    \caption{}
    \label{fig:placeholder}
\end{figure}
\begin{figure}
    \centering
    \includegraphics[width=0.9\columnwidth]{figs/fig82.png}
    \caption{}
    \label{fig:placeholder}
\end{figure}
\end{document}