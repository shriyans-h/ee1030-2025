\documentclass[journal]{IEEEtran}
\usepackage[a5paper, margin=10mm, onecolumn]{geometry}
\usepackage{lmodern}

\setlength{\headheight}{1cm}
\setlength{\headsep}{0mm}

\usepackage{gvv-book}
\usepackage{gvv}
\usepackage{cite}
\usepackage{amsmath,amssymb,amsfonts,amsthm}
\usepackage{graphicx}
\graphicspath{{./figs/}}
\usepackage{xcolor}
\usepackage{txfonts}
\usepackage{enumitem}
\usepackage{mathtools}
\usepackage{hyperref}
\usepackage{tikz}
\usepackage{tkz-euclide}

\begin{document}

\bibliographystyle{IEEEtran}
\vspace{3cm}

\title{4.3.25}
\author{EE25BTECH11036 - M Chanakya Srinivas}
\maketitle

\renewcommand{\thetable}{\theenumi}
\setlength{\intextsep}{10pt}
\renewcommand\theequation{\arabic{equation}}


\section*{Problem}
Find the ratio in which the $YZ$ plane divides the line segment joining the points 
\[
\vec{A} = \myvec{-2\\4\\7}, \quad 
\vec{B} = \myvec{3\\-5\\8}.
\]

\section*{Solution}

The line joining $\vec{A}$ and $\vec{B}$ can be written in parametric form as
\begin{align}
\vec{R} &= \vec{A} + \lambda\brak{\vec{B}-\vec{A}} \label{eq:line}
\end{align}

Substitute $\vec{A}$ and $\vec{B}$:
\begin{align}
\vec{R} 
&= \myvec{-2\\4\\7} + \lambda \brak{\myvec{3\\-5\\8}-\myvec{-2\\4\\7}} \\
&= \myvec{-2\\4\\7} + \lambda \myvec{5\\-9\\1} \label{eq:expand}
\end{align}

So
\begin{align}
\vec{R} = \myvec{-2+5\lambda \\ 4-9\lambda \\ 7+\lambda}
\end{align}

The $YZ$-plane has the equation 
\begin{align}
\myvec{1 & 0 & 0}\vec{R} = 0 \label{eq:yzplane}
\end{align}

Substituting $\vec{R}$:
\begin{align}
\myvec{1 & 0 & 0}\myvec{-2+5\lambda \\ 4-9\lambda \\ 7+\lambda} &= 0 \\
-2+5\lambda &= 0 \\
\lambda &= \tfrac{2}{5} \label{eq:lambda}
\end{align}

Thus, the point of intersection is
\begin{align}
\vec{P} &= \vec{A} + \frac{2}{5}\brak{\vec{B}-\vec{A}} \\
&= \frac{3}{5}\vec{A} + \frac{2}{5}\vec{B}
\end{align}

Therefore, the ratio in which the $YZ$-plane divides $AB$ is
\begin{align}
AP : PB = 2 : 3
\end{align}

\section*{Answer}
The $YZ$-plane divides the line segment internally in the ratio $\boxed{2:3}$.
\begin{figure}[h]
    \centering
    \includegraphics[width=0.9\columnwidth]{figs/fig_61.png}
    \caption{}
    \label{fig:placeholder}
\end{figure}
\begin{figure}
    \centering
    \includegraphics[width=0.9\columnwidth]{figs/Fig -62.png}
    \caption{}
    \label{fig:placeholder}
\end{figure}
\end{document}