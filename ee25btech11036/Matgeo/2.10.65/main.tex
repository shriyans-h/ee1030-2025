\documentclass[journal]{IEEEtran}
\usepackage[a5paper, margin=10mm, onecolumn]{geometry}
\usepackage{lmodern}

\setlength{\headheight}{1cm}
\setlength{\headsep}{0mm}

\usepackage{gvv-book}
\usepackage{gvv}
\usepackage{cite}
\usepackage{amsmath,amssymb,amsfonts,amsthm}
\usepackage{graphicx}
\graphicspath{{./figs/}}
\usepackage{xcolor}
\usepackage{txfonts}
\usepackage{enumitem}
\usepackage{mathtools}
\usepackage{hyperref}
\usepackage{tikz}
\usepackage{tkz-euclide}

\begin{document}

\bibliographystyle{IEEEtran}
\vspace{3cm}

\title{2.10.65}
\author{EE25BTECH11036 - M Chanakya Srinivas}
\maketitle

\renewcommand{\thetable}{\theenumi}
\setlength{\intextsep}{10pt}
\renewcommand\theequation{\arabic{equation}}



\title{Solution to Parallelogram Intersection Problem}
\author{}
\date{}
\maketitle

\section*{Problem Statement}
Let $OACB$ be a parallelogram with $O$ at the origin and $OC$ a diagonal. Let $D$ be the midpoint of $OA$. Using vector methods, prove that $BD$ and $CO$ intersect in the same ratio. Determine this ratio.

\section*{Solution}

Let the position vectors of the vertices be:

\begin{align}
\vec{O} &= \myvec{0\\0}, \\
\vec{A} &= \myvec{a_1\\a_2}, \\
\vec{B} &= \myvec{a_1 + b_1\\a_2 + b_2}, \\
\vec{C} &= \myvec{b_1\\b_2}.
\end{align}

Since $D$ is the midpoint of $OA$, we have:

\begin{align}
\vec{D} = \frac{\vec{O} + \vec{A}}{2} = \frac{1}{2} \myvec{a_1\\a_2}
\end{align}

---

\textbf{Step 1: Represent the lines in vector form}

Line $BD$:

\begin{align}
\vec{R_1} &= \vec{B} + \lambda (\vec{D} - \vec{B}) \\
&= \myvec{a_1 + b_1\\a_2 + b_2} + \lambda \left(\frac{1}{2} \myvec{a_1\\a_2} - \myvec{a_1 + b_1\\a_2 + b_2}\right) \\
&= \myvec{a_1 + b_1\\a_2 + b_2} - \lambda \left(\frac{1}{2} \myvec{a_1\\a_2} + \myvec{b_1\\b_2}\right)
\end{align}

Line $CO$:

\begin{align}
\vec{R_2} &= \vec{C} + \mu (\vec{O} - \vec{C}) \\
&= \myvec{b_1\\b_2} + \mu \left(\myvec{0\\0} - \myvec{b_1\\b_2}\right) \\
&= (1 - \mu) \myvec{b_1\\b_2}
\end{align}

---

\textbf{Step 2: Find the intersection by equating lines}

\begin{align}
\myvec{a_1 + b_1\\a_2 + b_2} - \lambda \left(\frac{1}{2} \myvec{a_1\\a_2} + \myvec{b_1\\b_2}\right) = (1 - \mu) \myvec{b_1\\b_2}
\end{align}

Equating coefficients:

\begin{align}
\text{For } \vec{a}: & \quad 1 - \frac{\lambda}{2} = 0 \quad \Rightarrow \quad \lambda = 2 \\
\text{For } \vec{b}: & \quad 1 - \lambda = 1 - \mu \quad \Rightarrow \quad \mu = 2
\end{align}

---

\textbf{Step 3: Interpret the ratio}

- On $BD$, $\lambda = 2$ implies the intersection divides $BD$ in the ratio $2:1$.  
- On $CO$, $\mu = 2$ implies the intersection divides $CO$ in the ratio $2:1$.  

\begin{align}
\boxed{\text{The lines $BD$ and $CO$ intersect in the ratio } 2:1.}
\end{align}




\begin{figure}
    \centering
    \includegraphics[width=0.9\columnwidth]{figs/fig51.png}
    \caption{}
    \label{fig:placeholder}
\end{figure}

\begin{figure}
    \centering
    \includegraphics[width=0.9\columnwidth]{figs/fig52.png}
    \caption{}
    \label{fig:placeholder}
\end{figure}
\end{document}