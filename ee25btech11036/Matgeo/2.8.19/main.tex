\documentclass[journal]{IEEEtran}
\usepackage[a5paper, margin=10mm, onecolumn]{geometry}
\usepackage{lmodern}

\setlength{\headheight}{1cm}
\setlength{\headsep}{0mm}

\usepackage{gvv-book}
\usepackage{gvv}
\usepackage{cite}
\usepackage{amsmath,amssymb,amsfonts,amsthm}
\usepackage{graphicx}
\graphicspath{{./figs/}}
\usepackage{xcolor}
\usepackage{txfonts}
\usepackage{enumitem}
\usepackage{mathtools}
\usepackage{hyperref}
\usepackage{tikz}
\usepackage{tkz-euclide}

\begin{document}

\bibliographystyle{IEEEtran}
\vspace{3cm}

\title{2.8.19}
\author{EE25BTECH11036 - M Chanakya Srinivas}
\maketitle

\renewcommand{\thetable}{\theenumi}
\setlength{\intextsep}{10pt}
\renewcommand\theequation{\arabic{equation}}


\title{Solution to Problem 2.8.19}
\author{}
\date{}
\maketitle

\section*{Problem Statement}
\section*{Problem}
If 
\[
\vec{r} \cdot \vec{a} = 0, \quad 
\vec{r} \cdot \vec{b} = 0, \quad 
\vec{r} \cdot \vec{c} = 0
\]
for some non-zero vector $\vec{r}$, then find the value of 
\[
\vec{a} \cdot (\vec{b} \times \vec{c}).
\]

\section*{Solution}

Since
\begin{align}
\vec{r} \cdot \vec{a} &= 0, \label{eq:1} \\
\vec{r} \cdot \vec{b} &= 0, \label{eq:2} \\
\vec{r} \cdot \vec{c} &= 0, \label{eq:3}
\end{align}
we conclude that the vectors $\vec{a}, \vec{b}, \vec{c}$ all lie in the subspace orthogonal to $\vec{r}$:
\begin{align}
\vec{a}, \vec{b}, \vec{c} \in \text{span}\{\vec{r}\}^{\perp}. \label{eq:4}
\end{align}

Since $\vec{r} \neq \vec{0}$, the orthogonal subspace $\text{span}\{\vec{r}\}^{\perp}$ is **at most 2-dimensional**:
\begin{align}
\dim(\text{span}\{\vec{r}\}^{\perp}) = 2. \label{eq:5}
\end{align}

Thus, any three vectors lying in this at most 2-dimensional subspace are **linearly dependent**:
\begin{align}
\vec{c} = \lambda_1 \vec{a} + \lambda_2 \vec{b} \quad \text{for some scalars } \lambda_1, \lambda_2. \label{eq:6}
\end{align}

The **scalar triple product** of linearly dependent vectors is zero:
\begin{align}
\vec{a} \cdot (\vec{b} \times \vec{c})
&= \det 
\myvec{
a_1 & a_2 & a_3 \\
b_1 & b_2 & b_3 \\
c_1 & c_2 & c_3
} \label{eq:7} \\
&= 0. \label{eq:8}
\end{align}

\section*{Final Answer}
\begin{align}
\boxed{\vec{a} \cdot (\vec{b} \times \vec{c}) = 0}
\end{align}
\begin{figure}
    \centering
    \includegraphics[width=0.9\columnwidth]{figs/fig41.png}
    \caption{}
    \label{fig:placeholder}
\end{figure}



\begin{figure}
    \centering
    \includegraphics[width=0.9\columnwidth]{figs/fig42.png}
    \caption{}
    \label{fig:placeholder}
\end{figure}
\end{document}