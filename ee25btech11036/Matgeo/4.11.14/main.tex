\documentclass[journal]{IEEEtran}
\usepackage[a5paper, margin=10mm, onecolumn]{geometry}
\usepackage{lmodern}

\setlength{\headheight}{1cm}
\setlength{\headsep}{0mm}

\usepackage{gvv-book}
\usepackage{gvv}
\usepackage{cite}
\usepackage{amsmath,amssymb,amsfonts,amsthm}
\usepackage{graphicx}
\graphicspath{{./figs/}}
\usepackage{xcolor}
\usepackage{txfonts}
\usepackage{enumitem}
\usepackage{mathtools}
\usepackage{hyperref}
\usepackage{tikz}
\usepackage{tkz-euclide}

\begin{document}

\bibliographystyle{IEEEtran}
\vspace{3cm}

\title{4.11.14}
\author{EE25BTECH11036 - M Chanakya Srinivas}
\maketitle

\renewcommand{\thetable}{\theenumi}
\setlength{\intextsep}{10pt}
\renewcommand\theequation{\arabic{equation}}


\section*{Question}
Find the value of \(\lambda\) for which the following lines are perpendicular to each other.
Hence determine whether the lines intersect or not.
\begin{align}
\frac{x-5}{5\lambda+2} &= \frac{2-y}{5} = \frac{1-z}{-1}, \label{eq:l1}\\
\frac{x}{1} &= \frac{y+\tfrac{1}{2}}{2\lambda} = \frac{z-1}{3}. \label{eq:l2}
\end{align}

\section*{Solution }

\subsection*{Step 1: Write lines in vector form}
Choose points and direction vectors:
\begin{align}
\vec A_1 &= \myvec{5\\2\\1}, & \vec m_1 &= \myvec{5\lambda+2\\-5\\1}, \label{A1m1}\\
\vec A_2 &= \myvec{0\\-1/2\\1}, & \vec m_2 &= \myvec{1\\2\lambda\\3}. \label{A2m2}
\end{align}
Then the lines are
\begin{align}
\vec r_1 &= \vec A_1 + \kappa_1 \vec m_1, \\
\vec r_2 &= \vec A_2 + \kappa_2 \vec m_2.
\end{align}

\subsection*{Step 2: Perpendicularity condition}
Lines are perpendicular if
\[
\vec m_1^{\top}\vec m_2 = 0.
\]
Compute the dot product:
\begin{align}
\vec m_1^{\top}\vec m_2
&= \myvec{5\lambda+2 & -5 & 1} \myvec{1\\2\lambda\\3} \\
&= (5\lambda+2)(1) + (-5)(2\lambda) + (1)(3) \\
&= 5\lambda + 2 - 10\lambda + 3 \\
&= -5\lambda + 5.
\end{align}
Hence
\[
-5\lambda + 5 = 0 \quad \Longrightarrow \quad \boxed{\lambda=1}.
\]

\subsection*{Step 3: Intersection condition}
Lines intersect if
\[
\vec r_1 = \vec r_2 \quad \Longrightarrow \quad \kappa_1 \vec m_1 - \kappa_2 \vec m_2 = \vec A_2 - \vec A_1.
\]
Define
\begin{align}
M(\lambda) &= \myvec{ \vec m_1 & -\vec m_2 } 
= \myvec{5\lambda+2 & -1 \\ -5 & -2\lambda \\ 1 & -3}, \\
\vec z &= \myvec{\kappa_1\\\kappa_2}, \qquad 
\vec b = \vec A_2 - \vec A_1 = \myvec{-5\\-5/2\\0}.
\end{align}
Then
\[
M(\lambda) \vec z = \vec b.
\]

\subsection*{Step 4: Form augmented matrix and do row reduction}
\[
\left[\begin{array}{cc|c}
5\lambda+2 & -1 & -5 \\
-5 & -2\lambda & -5/2 \\
1 & -3 & 0
\end{array}\right]
\]
Use row 3 to eliminate \(\kappa_1\):
\[
R_3: 1\cdot \kappa_1 - 3\cdot \kappa_2 = 0 \quad \Rightarrow \quad \kappa_1 = 3\kappa_2.
\]

Substitute \(\kappa_1 = 3\kappa_2\) into row 1 and row 2:
\begin{align}
\text{Row 1: } & (5\lambda+2)(3\kappa_2) - 1\cdot\kappa_2 = -5 
\quad \Rightarrow \quad (15\lambda + 5)\kappa_2 = -5, \\
\text{Row 2: } & -5(3\kappa_2) - 2\lambda \kappa_2 = -5/2 
\quad \Rightarrow \quad (-15-2\lambda)\kappa_2 = -5/2.
\end{align}

\subsection*{Step 5: Solve for consistency}
The system is consistent if
\[
\frac{-5}{15\lambda+5} = \frac{-5/2}{-15-2\lambda} \quad \Longrightarrow \quad -\frac{1}{3\lambda+1} = \frac{5}{30+4\lambda}.
\]
Solve:
\begin{align}
-(30+4\lambda) &= 5(3\lambda+1) \\
-30-4\lambda &= 15\lambda + 5 \\
19\lambda &= -35 \\
\boxed{\lambda} &= -\frac{35}{19}.
\end{align}

\subsection*{Step 6: conclusions}
\begin{itemize}
\item Perpendicularity occurs at \(\lambda=1\). At this value, rank\([M(1)\mid \vec b] > \text{rank}(M(1))\), so lines are \emph{skew}.
\item Intersection occurs at \(\lambda=-35/19\). At this value, rank\([M(\lambda)\mid \vec b] = \text{rank}(M(\lambda))\), so lines intersect (but are not perpendicular).
\end{itemize}
\begin{figure}[h]
    \centering
    \includegraphics[width=0.9\columnwidth]{figs/fig71.png}
    \caption{}
    \label{fig:placeholder}
\end{figure}
\begin{figure}
    \centering
    \includegraphics[width=0.9\columnwidth]{figs/fig72.png}
    \caption{}
    \label{fig:placeholder}
\end{figure}
\end{document}