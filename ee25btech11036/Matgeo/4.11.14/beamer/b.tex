\documentclass{beamer}
\usepackage[utf8]{inputenc}

\usetheme{Madrid}
\usecolortheme{default}
\usepackage{amsmath,amssymb,amsfonts,amsthm}
\usepackage{txfonts}
\usepackage{tkz-euclide}
\usepackage{listings}
\usepackage{adjustbox}
\usepackage{array}
\usepackage{tabularx}
\usepackage{gvv}
\usepackage{lmodern}
\usepackage{circuitikz}
\usepackage{tikz}
\usepackage{graphicx}

\setbeamertemplate{page number in head/foot}[totalframenumber]

\usepackage{tcolorbox}
\tcbuselibrary{minted,breakable,xparse,skins}



\definecolor{bg}{gray}{0.95}
\DeclareTCBListing{mintedbox}{O{}m!O{}}{%
	breakable=true,
	listing engine=minted,
	listing only,
	minted language=#2,
	minted style=default,
	minted options={%
		linenos,
		gobble=0,
		breaklines=true,
		breakafter=,,
		fontsize=\small,
		numbersep=8pt,
		#1},
	boxsep=0pt,
	left skip=0pt,
	right skip=0pt,
	left=25pt,
	right=0pt,
	top=3pt,
	bottom=3pt,
	arc=5pt,
	leftrule=0pt,
	rightrule=0pt,
	bottomrule=2pt,
	toprule=2pt,
	colback=bg,
	colframe=orange!70,
	enhanced,
	overlay={%
		\begin{tcbclipinterior}
			\fill[orange!20!white] (frame.south west) rectangle ([xshift=20pt]frame.north west);
	\end{tcbclipinterior}},
	#3,
}
\lstset{
	language=C,
	basicstyle=\ttfamily\small,
	keywordstyle=\color{blue},
	stringstyle=\color{orange},
	commentstyle=\color{green!60!black},
	numbers=left,
	numberstyle=\tiny\color{gray},
	breaklines=true,
	showstringspaces=false,
}
%------------------------------------------------------------
%This block of code defines the information to appear in the
%Title page
\title %optional
{4.11.14}
\date{}
%\subtitle{A short story}

\author % (optional)
{M Chanakya Srinivas- EE25BTECH11036}




\begin{document}


\frame{\titlepage}


\begin{frame}{Problem Statement}
Find the value of \(\lambda\) for which the following lines are perpendicular.\\
Determine whether the lines intersect or not.
\begin{align}
\frac{x-5}{5\lambda+2} &= \frac{2-y}{5} = \frac{1-z}{-1}, \\
\frac{x}{1} &= \frac{y+\tfrac{1}{2}}{2\lambda} = \frac{z-1}{3}.
\end{align}
\end{frame}

\begin{frame}{Step 1: Vector form of lines}
Choose points and direction vectors:
\begin{align}
\vec A_1 &= \myvec{5\\2\\1}, & \vec m_1 &= \myvec{5\lambda+2\\-5\\1},\\
\vec A_2 &= \myvec{0\\-1/2\\1}, & \vec m_2 &= \myvec{1\\2\lambda\\3}.
\end{align}

Lines in parametric form:
\begin{align}
\vec r_1 &= \vec A_1 + \kappa_1 \vec m_1, \\
\vec r_2 &= \vec A_2 + \kappa_2 \vec m_2.
\end{align}
\end{frame}

\begin{frame}{Step 2: Perpendicularity condition}
\[
\vec m_1^{\top} \vec m_2 = 0
\]

Compute dot product:
\begin{align}
\vec m_1^{\top} \vec m_2
&= \myvec{5\lambda+2 & -5 & 1} \myvec{1\\2\lambda\\3} \\
&= 5\lambda+2 - 10\lambda +3 \\
&= -5\lambda +5
\end{align}
\[
-5\lambda+5=0 \quad \Rightarrow \quad \boxed{\lambda=1}
\]
\end{frame}

\begin{frame}{Step 3: Intersection condition}
Lines intersect if
\[
\vec r_1 = \vec r_2 \quad \Rightarrow \quad \kappa_1 \vec m_1 - \kappa_2 \vec m_2 = \vec A_2 - \vec A_1
\]

Define
\begin{align}
M(\lambda) &= \begin{bmatrix}\vec m_1 & -\vec m_2\end{bmatrix} 
= \myvec{5\lambda+2 & -1\\ -5 & -2\lambda\\ 1 & -3},\\
\vec z &= \myvec{\kappa_1\\\kappa_2}, \quad
\vec b = \vec A_2 - \vec A_1 = \myvec{-5\\-5/2\\0}.
\end{align}
Then
\[
M(\lambda)\vec z = \vec b
\]
\end{frame}

\begin{frame}{Step 4: Augmented matrix and row reduction}
Augmented matrix:
\[
\left[\begin{array}{cc|c}
5\lambda+2 & -1 & -5 \\
-5 & -2\lambda & -5/2 \\
1 & -3 & 0
\end{array}\right]
\]

Use row 3 to eliminate \(\kappa_1\):
\[
1\cdot \kappa_1 - 3\cdot \kappa_2 = 0 \quad \Rightarrow \quad \kappa_1 = 3 \kappa_2
\]

Substitute \(\kappa_1 = 3 \kappa_2\):
\begin{align}
(15\lambda+5)\kappa_2 &= -5,\\
(-15-2\lambda)\kappa_2 &= -5/2
\end{align}
\end{frame}

\begin{frame}{Step 5: Consistency condition}
System consistent if:
\[
\frac{-5}{15\lambda+5} = \frac{-5/2}{-15-2\lambda} 
\quad \Rightarrow \quad -\frac{1}{3\lambda+1} = \frac{5}{30+4\lambda}
\]

Solve:
\begin{align}
-(30+4\lambda) &= 5(3\lambda+1) \\
-30-4\lambda &= 15\lambda + 5 \\
19\lambda &= -35 \\
\boxed{\lambda} &= -\frac{35}{19}
\end{align}
\end{frame}

\begin{frame}{Step 6: Conclusions (Chapter 4 style)}
\begin{itemize}
\item Perpendicularity occurs at \(\lambda=1\). At this value, rank\([M(1)\mid \vec b] > \text{rank}(M(1))\), so lines are \emph{skew}.
\item Intersection occurs at \(\lambda=-35/19\). At this value, rank\([M(\lambda)\mid \vec b] = \text{rank}(M(\lambda))\), so lines intersect (but are not perpendicular).
\end{itemize}
\end{frame}

\begin{frame}[fragile]{C code}
\begin{lstlisting}
    #include <math.h>
// Return lambda for which the two lines are perpendicular.
double perpendicular_lambda(void) {
    return 1.0;
}
// Return lambda for which the two lines intersect.
double intersection_lambda(void) {
    return -35.0/19.0;
}
// For a given lambda, check whether the two lines intersect.
// Writes s (parameter of line2) and t (parameter of line1) to the output pointers.
// Returns 1 if intersection exists (within tolerance), otherwise 0.
int lines_intersection_params(double lambda, double *s_out, double *t_out) {
    const double EPS = 1e-9;
    double denom1 = 1.0 - 3.0*(5.0*lambda + 2.0);
    if (fabs(denom1) < EPS) return 0;
     \end{lstlisting}
\end{frame}
    \begin{frame}[fragile]{C code}
\begin{lstlisting}
    double s1 = 5.0/denom1;
    double denom2 = 30.0 + 4.0*lambda;
    if (fabs(denom2) < EPS) return 0;
    double s2 = 5.0/denom2;
    if (fabs(s1 - s2) > 1e-6) return 0;
    double s = 0.5*(s1 + s2);
    double t = 3.0*s;
    if (s_out) *s_out = s;
    if (t_out) *t_out = t;
    return 1;
}
 \end{lstlisting}
\end{frame}
    \begin{frame}[fragile]{Python code through shared output}
\begin{lstlisting}
import ctypes
from fractions import Fraction
import numpy as np
import matplotlib.pyplot as plt
from mpl_toolkits.mplot3d import Axes3D

# Load shared library
lib = ctypes.CDLL('./liblines.so')
lib.perpendicular_lambda.restype = ctypes.c_double
lib.intersection_lambda.restype = ctypes.c_double
lib.lines_intersection_params.restype = ctypes.c_int
lib.lines_intersection_params.argtypes = [
    ctypes.c_double,
    ctypes.POINTER(ctypes.c_double),
    ctypes.POINTER(ctypes.c_double)
]
\end{lstlisting}
\end{frame}
  \begin{frame}[fragile]{Python code through shared output}
\begin{lstlisting}
# 1. Get lambdas
lam_perp = lib.perpendicular_lambda()
lam_inter = lib.intersection_lambda()

print(f"Lambda for perpendicular lines: {lam_perp} = {Fraction(lam_perp).limit_denominator()}")
print(f"Lambda for intersection: {lam_inter} = {Fraction(lam_inter).limit_denominator()}")

# 2. Check intersection at perpendicular lambda
s1 = ctypes.c_double()
t1 = ctypes.c_double()
intersects_perp = lib.lines_intersection_params(lam_perp, ctypes.byref(s1), ctypes.byref(t1))

# 3. Check intersection at intersection lambda
s2 = ctypes.c_double()
t2 = ctypes.c_double()
intersects = lib.lines_intersection_params(lam_inter, ctypes.byref(s2), ctypes.byref(t2))
\end{lstlisting}
\end{frame}
  \begin{frame}[fragile]{Python code through shared output}
\begin{lstlisting}
if intersects_perp:
    print("Lines intersect when perpendicular.")
else:
    print("❌ Lines do NOT intersect when they are perpendicular.")

if intersects:
    print("✅ Lines intersect when λ =", lam_inter)
    print("Intersection parameters: s =", s2.value, ", t =", t2.value)
    inter_point = np.array([
        s2.value,
        -0.5 + 2*lam_inter*s2.value,
        1 + 3*s2.value
    ])
    print("Intersection point:", inter_point)
else:
    inter_point = None
    print("❌ Lines do NOT intersect for intersection λ.")
\end{lstlisting}
\end{frame}
  \begin{frame}[fragile]{Python code through shared output}
\begin{lstlisting}
# 4. Plot
t_vals = np.linspace(-2, 2, 100)
x1 = 5 + (5*lam_inter + 2)*t_vals
y1 = 2 - 5*t_vals
z1 = 1 + t_vals

s_vals = np.linspace(-2, 2, 100)
x2 = s_vals
y2 = -0.5 + 2*lam_inter*s_vals
z2 = 1 + 3*s_vals

fig = plt.figure(figsize=(10, 8))
ax = fig.add_subplot(111, projection='3d')

# Plot lines
ax.plot(x1, y1, z1, label="Line 1", color='blue', linewidth=2)
ax.plot(x2, y2, z2, label="Line 2", color='red', linewidth=2)
\end{lstlisting}
\end{frame}
  \begin{frame}[fragile]{Python code through shared output}
\begin{lstlisting}
# Plot intersection
if inter_point is not None:
    ax.scatter(*inter_point, color='green', s=80, edgecolors='black', label="Intersection Point")
    ax.text(*inter_point + 0.3, 
            f"({inter_point[0]:.2f}, {inter_point[1]:.2f}, {inter_point[2]:.2f})",
            fontsize=10, color='green')
\end{lstlisting}
\end{frame}
  \begin{frame}[fragile]{Python code through shared output}
\begin{lstlisting}
# Labels and legend
ax.set_xlabel("X Axis", fontsize=12)
ax.set_ylabel("Y Axis", fontsize=12)
ax.set_zlabel("Z Axis", fontsize=12)
ax.set_title("3D Plot: Intersection of Two Lines", fontsize=14)
ax.legend()
ax.grid(True)
ax.view_init(elev=25, azim=135)

plt.tight_layout()
plt.show()

 \end{lstlisting}
\end{frame}
  \begin{frame}[fragile]{Only Python code}
\begin{lstlisting}
import numpy as np
import matplotlib.pyplot as plt

# Local imports (assuming these scripts are available and in PYTHONPATH)
from line.funcs import *
from triangle.funcs import *
from conics.funcs import circ_gen

# Given intersection parameter lambda (from your C-types code or calculation)
lam = -35/19  # approx -1.8421

# Calculate intersection parameters s and t for the lines
# The two parametric lines from your problem:
 \end{lstlisting}
\end{frame}
  \begin{frame}[fragile]{Only Python code}
\begin{lstlisting}
# Line 1:
# x = 5 + (5*lam + 2)*t
# y = 2 - 5*t
# z = 1 + t

# Line 2:
# x = s
# y = -0.5 + 2*lam*s
# z = 1 + 3*s

# We want to find s, t such that both lines intersect

# From parametric equality:
# 5 + (5*lam + 2)*t = s      ...(1)
# 2 - 5*t = -0.5 + 2*lam*s   ...(2)
# 1 + t = 1 + 3*s            ...(3)

# Solve (3):
# t = 3*s
 \end{lstlisting}
\end{frame}
  \begin{frame}[fragile]{Only Python code}
\begin{lstlisting}
# Substitute into (1):
# 5 + (5*lam + 2)*3*s = s
# 5 + 3*(5*lam + 2)*s = s
# 5 = s - 3*(5*lam + 2)*s = s*(1 - 3*(5*lam + 2))
# s = 5 / (1 - 3*(5*lam + 2))

denom = 1 - 3*(5*lam + 2)
s = 5 / denom

# Then t = 3*s
t = 3*s

# Calculate intersection point from line 2 (or line 1)
inter_point = np.array([
    s,
    -0.5 + 2*lam*s,
    1 + 3*s
])
 \end{lstlisting}
\end{frame}
  \begin{frame}[fragile]{Only Python code}
\begin{lstlisting}
print(f"Intersection parameters: s = {s:.5f}, t = {t:.5f}")
print(f"Intersection Point: ({inter_point[0]:.5f}, {inter_point[1]:.5f}, {inter_point[2]:.5f})")

# Define parametric lines for plotting near intersection point

def line1(t_vals):
    x = 5 + (5*lam + 2)*t_vals
    y = 2 - 5*t_vals
    z = 1 + t_vals
    return x, y, z

def line2(s_vals):
    x = s_vals
    y = -0.5 + 2*lam*s_vals
    z = 1 + 3*s_vals
    return x, y, z
 \end{lstlisting}
\end{frame}
  \begin{frame}[fragile]{Only Python code}
\begin{lstlisting}
# Plot ranges close to intersection
t_vals = np.linspace(t - 1, t + 1, 100)
s_vals = np.linspace(s - 1, s + 1, 100)

x1, y1, z1 = line1(t_vals)
x2, y2, z2 = line2(s_vals)

# Plotting
fig = plt.figure(figsize=(10, 8))
ax = fig.add_subplot(111, projection='3d')

ax.plot(x1, y1, z1, label='Line 1', color='blue', linewidth=2)
ax.plot(x2, y2, z2, label='Line 2', color='red', linewidth=2)

# Mark intersection
ax.scatter(*inter_point, color='green', s=80, label='Intersection Point')
 \end{lstlisting}
\end{frame}
  \begin{frame}[fragile]{Only Python code}
\begin{lstlisting}
# Annotate intersection point
ax.text(inter_point[0], inter_point[1], inter_point[2],
        f"A\n({inter_point[0]:.2f}, {inter_point[1]:.2f}, {inter_point[2]:.2f})",
        color='green', fontsize=12, ha='center', va='bottom')

# Axis labels and title
ax.set_xlabel("X axis", fontsize=12)
ax.set_ylabel("Y axis", fontsize=12)
ax.set_zlabel("Z axis", fontsize=12)
ax.set_title("3D Plot of Two Lines and their Intersection", fontsize=14)

ax.legend()
ax.grid(True)
ax.view_init(elev=20, azim=45)

plt.tight_layout()
plt.show()

 \end{lstlisting}
\end{frame}
\begin{frame}{PLOTS}
    \begin{figure}
        \centering
        \includegraphics[width=0.9\columnwidth]{figs/fig71.png}
        \caption{}
        \label{fig:placeholder}
    \end{figure}
\end{frame}
\begin{frame}{PLOTS}
    \begin{figure}
        \centering
        \includegraphics[width=0.9\columnwidth]{figs/fig72.png}
        \caption{}
        \label{fig:placeholder}
    \end{figure}
\end{frame}
\end{document}