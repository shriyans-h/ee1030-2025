\documentclass{beamer}
\usepackage[utf8]{inputenc}

\usetheme{Madrid}
\usecolortheme{default}
\usepackage{amsmath,amssymb,amsfonts,amsthm}
\usepackage{txfonts}
\usepackage{tkz-euclide}
\usepackage{listings}
\usepackage{adjustbox}
\usepackage{array}
\usepackage{tabularx}
\usepackage{gvv}
\usepackage{lmodern}
\usepackage{circuitikz}
\usepackage{tikz}
\usepackage{graphicx}

\setbeamertemplate{page number in head/foot}[totalframenumber]

\usepackage{tcolorbox}
\tcbuselibrary{minted,breakable,xparse,skins}



\definecolor{bg}{gray}{0.95}
\DeclareTCBListing{mintedbox}{O{}m!O{}}{%
	breakable=true,
	listing engine=minted,
	listing only,
	minted language=#2,
	minted style=default,
	minted options={%
		linenos,
		gobble=0,
		breaklines=true,
		breakafter=,,
		fontsize=\small,
		numbersep=8pt,
		#1},
	boxsep=0pt,
	left skip=0pt,
	right skip=0pt,
	left=25pt,
	right=0pt,
	top=3pt,
	bottom=3pt,
	arc=5pt,
	leftrule=0pt,
	rightrule=0pt,
	bottomrule=2pt,
	toprule=2pt,
	colback=bg,
	colframe=orange!70,
	enhanced,
	overlay={%
		\begin{tcbclipinterior}
			\fill[orange!20!white] (frame.south west) rectangle ([xshift=20pt]frame.north west);
	\end{tcbclipinterior}},
	#3,
}
\lstset{
	language=C,
	basicstyle=\ttfamily\small,
	keywordstyle=\color{blue},
	stringstyle=\color{orange},
	commentstyle=\color{green!60!black},
	numbers=left,
	numberstyle=\tiny\color{gray},
	breaklines=true,
	showstringspaces=false,
}
%------------------------------------------------------------
%This block of code defines the information to appear in the
%Title page
\title %optional
{5.13.62}
\date{}
%\subtitle{A short story}

\author % (optional)
{M Chanakya Srinivas- EE25BTECH11036}




\begin{document}


\frame{\titlepage}



%---------------------------

\begin{frame}{Problem Statement}
\textbf{Question:} \\[0.5em]
How many \( 3 \times 3 \) matrices \( M \), with entries from the set \( \{0,1,2\} \), satisfy:
\begin{align}
\operatorname{tr}(M^\top M) = 5
\end{align}
\end{frame}

%---------------------------

\begin{frame}{Matrix Structure}
Let the matrix \( M \in \mathbb{R}^{3 \times 3} \) be:
\begin{align}
M = \vec{(c_1 \quad c_2 \quad c_3)}
\end{align}
where each column vector is:
\begin{align}
\vec{c_j} = \myvec{c_{1j} \\ c_{2j} \\ c_{3j}}, \quad j = 1,2,3
\end{align}
Each \( c_{ij} \in \{0,1,2\} \), so total possible vectors:
\begin{align}
3^3 = 27
\end{align}
\end{frame}

%---------------------------

\begin{frame}{Computing \( M^\top M \)}
The product:
\begin{align}
M^\top M = 
\myvec{
\vec{c_1}^\top \vec{c_1} & \vec{c_1}^\top \vec{c_2} & \vec{c_1}^\top \vec{c_3} \\
\vec{c_2}^\top \vec{c_1} & \vec{c_2}^\top \vec{c_2} & \vec{c_2}^\top \vec{c_3} \\
\vec{c_3}^\top \vec{c_1} & \vec{c_3}^\top \vec{c_2} & \vec{c_3}^\top \vec{c_3}
}
\end{align}

Trace of the matrix is:
\begin{align}
\operatorname{tr}(M^\top M) &= \vec{c_1}^\top \vec{c_1} + \vec{c_2}^\top \vec{c_2} + \vec{c_3}^\top \vec{c_3} \\
&= \|\vec{c_1}\|^2 + \|\vec{c_2}\|^2 + \|\vec{c_3}\|^2
\end{align}
\end{frame}

%---------------------------

\begin{frame}{Norm Constraint}
Let:
\begin{align}
n_1 = \|\vec{c_1}\|^2, \quad n_2 = \|\vec{c_2}\|^2, \quad n_3 = \|\vec{c_3}\|^2
\end{align}
We want:
\begin{align}
n_1 + n_2 + n_3 = 5
\end{align}
\end{frame}

%---------------------------

\begin{frame}{Norm Counts for Vectors}
Define \( N(n) \) as number of vectors \( \vec{v} \in \{0,1,2\}^3 \) with squared norm \( n \). Then:

\begin{align}
N(0) &= 1 \quad \text{(only } \myvec{0\\0\\0}) \\
N(1) &= 3 \quad \text{(one entry 1, others 0)} \\
N(2) &= 3 \quad \text{(one entry 2, others 0)} \\
N(3) &= 6 \\
N(4) &= 3 \\
N(5) &= 6
\end{align}
\end{frame}

%---------------------------

\begin{frame}{Counting Valid Matrices}
We count all ordered triples \( (\vec{c_1}, \vec{c_2}, \vec{c_3}) \) such that:
\begin{align}
n_1 + n_2 + n_3 = 5
\end{align}

Total count is:
\begin{align}
\text{Total} = \sum_{\substack{n_1 + n_2 + n_3 = 5 \\ 0 \leq n_i \leq 5}} N(n_1) \cdot N(n_2) \cdot N(n_3)
\end{align}

Evaluating this sum gives:
\begin{align}
\boxed{198}
\end{align}
\end{frame}

%---------------------------

\begin{frame}{Final Answer}
\begin{align}
\boxed{\text{Number of matrices } M = 198}
\end{align}
\end{frame}

\begin{frame}[fragile]{C code}
\begin{lstlisting}

#include <stdint.h>
int get_count(void) {
    int count = 0;
    // There are 9 entries; treat them as base-3 digits 0,1,2
    for (int mask = 0; mask < 19683; ++mask) { // 3^9 = 19683
        int tmp = mask;
        int sumsq = 0;
        for (int i = 0; i < 9; ++i) {
            int digit = tmp % 3; // 0,1,2
            tmp /= 3;
            sumsq += digit*digit;
            if (sumsq > 5) break; // small optimization
        }
        if (sumsq == 5) ++count;
    }
    return count;
}
\end{lstlisting}
\end{frame}
\begin{frame}[fragile]{Python code through shared output}
\begin{lstlisting}
import ctypes
import os

# Adjust path if needed
libpath = os.path.abspath("./libcount.so")
lib = ctypes.CDLL(libpath)

lib.get_count.restype = ctypes.c_int

result = lib.get_count()
print("Number of 3x3 matrices with trace(M^T M)=5:", result)
\end{lstlisting}
\end{frame}
\begin{frame}[fragile]{Only Python code}
\begin{lstlisting}
import numpy as np
from itertools import product

def count_matrices():
    # Step 1: compute squared norm multiplicities for column vectors in {0,1,2}^3
    cols = np.array(list(product((0,1,2), repeat=3)))
    norms = np.sum(cols**2, axis=1)
    unique, counts = np.unique(norms, return_counts=True)
    N = dict(zip(unique, counts))  # N[n] = multiplicity of norm n
    \end{lstlisting}
\end{frame}
\begin{frame}[fragile]{Only Python code}
\begin{lstlisting}
    # Step 2: sum over all triples of norms that add to 5
    total = 0
    for n1, c1 in N.items():
        for n2, c2 in N.items():
            for n3, c3 in N.items():
                if n1 + n2 + n3 == 5:
                    total += c1 * c2 * c3
    return total, N

if __name__ == "__main__":
    total, N = count_matrices()
    print("Single-column squared norm multiplicities N(n):")
    for n in sorted(N):
        print(f"  n={n}: {N[n]} ways")
    print("\nTotal matrices =", total)
\end{lstlisting}
\end{frame}
\end{document}
