\let\negmedspace\undefined
\let\negthickspace\undefined
\documentclass[journal]{IEEEtran}
\usepackage[a5paper, margin=10mm, onecolumn]{geometry}
%\usepackage{lmodern} 
\usepackage{tfrupee} 

\setlength{\headheight}{1cm} 
\setlength{\headsep}{0mm}     

\usepackage{gvv-book}
\usepackage{gvv}
\usepackage{cite}
\usepackage{amsmath,amssymb,amsfonts,amsthm}
\usepackage{algorithmic}
\usepackage{graphicx}
\usepackage{textcomp}
\usepackage{xcolor}
\usepackage{txfonts}
\usepackage{listings}
\usepackage{enumitem}
\usepackage{mathtools}
\usepackage{gensymb}
\usepackage{comment}
\usepackage[breaklinks=true]{hyperref}
\usepackage{tkz-euclide} 
\usepackage{listings}                                        
\def\inputGnumericTable{}                                 
\usepackage[latin1]{inputenc}                                
\usepackage{color}                                            
\usepackage{array}                                            
\usepackage{longtable}                                       
\usepackage{calc}                                             
\usepackage{multirow}                                         
\usepackage{hhline}                                           
\usepackage{ifthen}                                           
\usepackage{lscape}

\begin{document}

\bibliographystyle{IEEEtran}
\vspace{3cm}

\title{12.497}
\author{AI25BTECH11003 - Bhavesh Gaikwad}
{\let\newpage\relax\maketitle}

\renewcommand{\thefigure}{\theenumi}
\renewcommand{\thetable}{\theenumi}
\setlength{\intextsep}{10pt} 

\numberwithin{equation}{enumi}
\numberwithin{figure}{enumi}
\renewcommand{\thetable}{\theenumi}


\textbf{Question}: If the rank of a (5$\times$6) matrix $\vec{Q}$ is 4, then which one of the following statements is correct?

\hfill{(EE 2008)}

\begin{itemize}
    \item[a)]$\vec{Q}$ will have four linearly independent rows and four linearly independent columns.
    \item[b)]$\vec{Q}$ will have four linearly independent rows and five linearly independent columns.
    \item[c)]$\vec{Q}\vec{Q}^\top$ will be invertible.
    \item[d)]$\vec{Q}^\top\vec{Q}$ will be invertible 
\end{itemize}

\bigskip
 
\textbf{Solution:}\\

\textbf{Primary Analysis:}\\
Since rank($\vec{Q}$)=4 $\Rightarrow \; \therefore \vec{Q}$ will have four linearly independent rows and four linearly independent columns.\\

Option-A:\\
Correct Option by Primary Analysis itself.\\

Option-B:\\
Incorrect Option by Primary Analysis itself.\\

Option C:\\
$\vec{Q}\vec{Q}^\top$ is a $5 \times 5$ matrix.\\
Since $\text{rank}(\vec{Q}^\top) = \text{rank}(\vec{Q}) = 4$,\\
By the Gram matrix rank theorem, $\text{rank}(\vec{A}\vec{A}^\top) = \text{rank}(\vec{A})$ for any matrix $\vec{A}$.\\
Applying this theorem,
\begin{align}
\text{rank}(\vec{Q}\vec{Q}^\top) = \text{rank}(\vec{Q}) = 4
\end{align}
Since $\vec{Q}\vec{Q}^\top$ is a $5 \times 5$ matrix with rank $4 < 5$, it is not full rank and therefore s$\det{(\vec{Q}\vec{Q}^\top)}$ = 0. A square matrix is invertible if and only if it has full rank. Therefore, $\vec{Q}\vec{Q}^\top$ is NOT invertible.

Thus, Incorrect Option.\\


Option D:\\
$\vec{Q}^\top\vec{Q}$ is a $6 \times 6$ matrix.\\
Since $\text{rank}(\vec{Q}^\top) = \text{rank}(\vec{Q}) = 4$,\\
By the Gram matrix rank theorem, $\text{rank}(\vec{A}^\top\vec{A}) = \text{rank}(\vec{A})$ for any matrix $\vec{A}$.\\
Applying this theorem,
\begin{align}
\text{rank}(\vec{Q}^\top\vec{Q}) = \text{rank}(\vec{Q}) = 4
\end{align}
Since $\vec{Q}^\top\vec{Q}$ is a $6 \times 6$ matrix with rank $4 < 6$, it is not full rank and therefore $\det{(\vec{Q}^\top\vec{Q})}$ = 0. A square matrix is invertible if and only if it has full rank. Therefore, $\vec{Q}^\top\vec{Q}$ is NOT invertible.

Thus, Incorrect Option.\\

\begin{align*}
    \boxed{\text{Only Option-A is Correct.}}
\end{align*}

\end{document}  
