\documentclass{beamer}
\usepackage[utf8]{inputenc}

\usetheme{Madrid}
\usecolortheme{default}
\usepackage{amsmath,amssymb,amsfonts,amsthm}
\usepackage{txfonts}
\usepackage{tkz-euclide}
\usepackage{listings}
\usepackage{adjustbox}
\usepackage{array}
\usepackage{tabularx}
\usepackage{gvv}
\usepackage{lmodern}
\usepackage{circuitikz}
\usepackage{tikz}
\usepackage{graphicx}

\setbeamertemplate{page number in head/foot}[totalframenumber]

\usepackage{tcolorbox}
\tcbuselibrary{minted,breakable,xparse,skins}



\definecolor{bg}{gray}{0.95}
\DeclareTCBListing{mintedbox}{O{}m!O{}}{%
  breakable=true,
  listing engine=minted,
  listing only,
  minted language=#2,
  minted style=default,
  minted options={%
    linenos,
    gobble=0,
    breaklines=true,
    breakafter=,,
    fontsize=\small,
    numbersep=8pt,
    #1},
  boxsep=0pt,
  left skip=0pt,
  right skip=0pt,
  left=25pt,
  right=0pt,
  top=3pt,
  bottom=3pt,
  arc=5pt,
  leftrule=0pt,
  rightrule=0pt,
  bottomrule=2pt,
  toprule=2pt,
  colback=bg,
  colframe=orange!70,
  enhanced,
  overlay={%
    \begin{tcbclipinterior}
    \fill[orange!20!white] (frame.south west) rectangle ([xshift=20pt]frame.north west);
    \end{tcbclipinterior}},
  #3,
}
\lstset{
    language=C,
    basicstyle=\ttfamily\small,
    keywordstyle=\color{blue},
    stringstyle=\color{orange},
    commentstyle=\color{green!60!black},
    numbers=left,
    numberstyle=\tiny\color{gray},
    breaklines=true,
    showstringspaces=false,
}
%------------------------------------------------------------

\title
{4.7.64}
\date{September 4,2025}
\author 
{AI25BTECH11003 - Bhavesh Gaikwad}



\begin{document}


\frame{\titlepage}
\begin{frame}{Question}
\centering
Find the distance between the point $\vec{P}$(6, 5, 9) and the plane determined by the points $\vec{A}$(3, -1, 2), $\vec{B}$(5, 2, 4) and $\vec{C}$(-1, -1, 6).
\end{frame}


\begin{frame}[fragile]
    \frametitle{Theoretical Solution}
Given:

\begin{align}
P = \myvec{6\\5\\9}, \,
A = \myvec{3\\-1\\2}, \,
B = \myvec{5\\2\\4}, \,
C = \myvec{-1\\-1\\6}.
\end{align}

First, form two direction vectors on the plane using the given points.
\begin{align} LET \, \,
\vec{u} &= B - A = \myvec{2\\3\\2}, &
\vec{v} &= C - A = \myvec{-4\\0\\4}.
\end{align}

Let $\vec{n} = \myvec{n_1 \\ n_2 \\ n_3}$ be the perpendicular vector to the Plane.\\
Therefore, the equation of the Plane is $\vec{n}^\top \vec{x}=1$\\
Let the equation of the Plane be $\myvec{n_1 & n_2 & n_3}\vec{x}=1$\\
\end{frame}

\begin{frame}[fragile]
\frametitle{Theoretical Solution}

Finding $\vec{n}$ which is orthogonal to both $\vec{u}$ and $\vec{v}$ by solving the homogeneous system:
\begin{equation}
\myvec{2 & 3 & 2\\ -4 & 0 & 4}\myvec{n_1\\n_2\\n_3}=\myvec{0\\0}.
\end{equation}

Row-reduce and solve for a convenient integer solution.
\begin{align}
\myvec{2 & 3 & 2\\ -4 & 0 & 4}
&\xrightarrow{R_2\leftarrow R_2+2R_1}
\myvec{2 & 3 & 2\\ 0 & 6 & 8},\\
6n_2+8n_3&=0 \;\Rightarrow\; 3n_2+4n_3=0, \quad
n_3=3,\; n_2=-4,\; 2n_1+3n_2+2n_3=0 \Rightarrow n_1=3,\\
\therefore \vec{n} &= \frac{1}{19}\myvec{3\\-4\\3}.
\end{align}

Writing the plane as $\vec{n}^\top \vec{x} = 1$
\end{frame}


\begin{frame}[fragile]
    \frametitle{Theoretical Solution}

Finally, applying the point-to-plane distance formula and simplify.

\begin{align}
d &= \frac{\abs{\vec{n}^\top P - 1}}{\norm{\vec{n}}}\\
&= \frac{\abs{3\cdot 6+(-4)\cdot 5+3\cdot 9-19}}{\sqrt{3^2+(-4)^2+3^2}}\\
&= \frac{\abs{25-19}}{\sqrt{34}}\\
&= \frac{6}{\sqrt{34}}=\frac{3\sqrt{34}}{17}.
\end{align}

\begin{align}
    \boxed{\text{The Distance between the Plane and } \vec{P} \, is \, \dfrac{3\sqrt{34}}{17} \, units.}
\end{align}
\end{frame}

\begin{frame}[fragile]
    \frametitle{Theoretical Solution}
\end{frame}

\begin{frame}[fragile]
    \frametitle{C Code}
    \begin{lstlisting}
#include <stdio.h>
#include <math.h>

/* 3D point and vector types */
typedef struct { double x, y, z; } Point3D;
typedef struct { double x, y, z; } Vector3D;

/* Subtract two points (p1 - p2) -> vector */
Vector3D subtract_points(Point3D p1, Point3D p2) {
    Vector3D r;
    r.x = p1.x - p2.x;
    r.y = p1.y - p2.y;
    r.z = p1.z - p2.z;
    return r;
}
    \end{lstlisting}
\end{frame}

\begin{frame}[fragile]
    \frametitle{C Code}
    \begin{lstlisting}

/* Cross product v1 x v2 */
Vector3D cross_product(Vector3D v1, Vector3D v2) {
    Vector3D r;
    r.x = v1.y * v2.z - v1.z * v2.y;
    r.y = v1.z * v2.x - v1.x * v2.z;
    r.z = v1.x * v2.y - v1.y * v2.x;
    return r;
}

/* Vector magnitude */
double magnitude(Vector3D v) {
    return sqrt(v.x * v.x + v.y * v.y + v.z * v.z);
}
    \end{lstlisting}
\end{frame}

\begin{frame}[fragile]
    \frametitle{C Code}
    \begin{lstlisting}
/* Plane normal from three points: n = (B - A) x (C - A) */
Vector3D find_plane_normal(Point3D A, Point3D B, Point3D C) {
    Vector3D AB = subtract_points(B, A);
    Vector3D AC = subtract_points(C, A);
    return cross_product(AB, AC);
}

/* Plane constant k in: n.x*x + n.y*y + n.z*z = k, using a point on plane */
double find_plane_constant(Point3D A, Vector3D n) {
    return n.x * A.x + n.y * A.y + n.z * A.z;
}
    \end{lstlisting}
\end{frame}

\begin{frame}[fragile]
    \frametitle{C Code}
    \begin{lstlisting}
/* Foot of perpendicular from P to plane with normal n and constant k */
Point3D find_foot_of_perpendicular(Point3D P, Vector3D n, double k) {
    /* Line: L(t) = P + t n; impose n * L(t) = k to solve for t */
    double num = k - (n.x * P.x + n.y * P.y + n.z * P.z);
    double den = n.x * n.x + n.y * n.y + n.z * n.z;
    double t = num / den;

    Point3D Q;
    Q.x = P.x + t * n.x;
    Q.y = P.y + t * n.y;
    Q.z = P.z + t * n.z;
    return Q;
}

    \end{lstlisting}
\end{frame}

\begin{frame}[fragile]
    \frametitle{C Code}
    \begin{lstlisting}

/* Distance between two points */
double distance_between_points(Point3D p1, Point3D p2) {
    Vector3D d = subtract_points(p1, p2);
    return magnitude(d);
}

/* Save points to points.dat in text format */
void save_points_to_file(Point3D A, Point3D B, Point3D C, Point3D P, Point3D Q) {
    FILE *fp = fopen("points.dat", "w");
    if (!fp) return;
    fprintf(fp, "A %.10f %.10f %.10f\n", A.x, A.y, A.z);
    fprintf(fp, "B %.10f %.10f %.10f\n", B.x, B.y, B.z);
    fprintf(fp, "C %.10f %.10f %.10f\n", C.x, C.y, C.z);
    fprintf(fp, "P %.10f %.10f %.10f\n", P.x, P.y, P.z);
    fprintf(fp, "Q %.10f %.10f %.10f\n", Q.x, Q.y, Q.z);
    fclose(fp);
}
    \end{lstlisting}
\end{frame}

\begin{frame}[fragile]
    \frametitle{C Code}
    \begin{lstlisting}
/* Solve distance from P to plane ABC, return distance, write Q via pointer, and save A,B,C,P,Q to points.dat */
double solve_point_to_plane_distance(Point3D A, Point3D B, Point3D C, Point3D P, Point3D *Q_out) {
    Vector3D n = find_plane_normal(A, B, C);
    double k = find_plane_constant(A, n);
    Point3D Q = find_foot_of_perpendicular(P, n, k);
    if (Q_out) *Q_out = Q;
    save_points_to_file(A, B, C, P, Q);
    return distance_between_points(P, Q);
}
    \end{lstlisting}
\end{frame}

\begin{frame}[fragile]
    \frametitle{C Code}
    \begin{lstlisting}
/* Convenience function with the exact problem data from the PDF:
   A(3,-1,2), B(5,2,4), C(-1,-1,6), P(6,5,9).
   Calls the solver, writes points.dat, and returns the distance. */
double generate_points_and_save(void) {
    Point3D A = { 3.0, -1.0, 2.0 };
    Point3D B = { 5.0,  2.0, 4.0 };
    Point3D C = {-1.0, -1.0, 6.0 };
    Point3D P = { 6.0,  5.0, 9.0 };
    Point3D Q;
    return solve_point_to_plane_distance(A, B, C, P, &Q);
}
    \end{lstlisting}
\end{frame}

\begin{frame}[fragile]
    \frametitle{Python Code}
    \begin{lstlisting}
import ctypes
import numpy as np
import matplotlib.pyplot as plt
from mpl_toolkits.mplot3d import Axes3D
import os

# Load the shared library
lib = ctypes.CDLL('./plane.so')

# Define the Point3D structure for ctypes
class Point3D(ctypes.Structure):
    _fields_ = [("x", ctypes.c_double),
                ("y", ctypes.c_double), 
                ("z", ctypes.c_double)]
    \end{lstlisting}
\end{frame}

\begin{frame}[fragile]
    \frametitle{Python Code}
    \begin{lstlisting}
class Vector3D(ctypes.Structure):
    _fields_ = [("x", ctypes.c_double),
                ("y", ctypes.c_double),
                ("z", ctypes.c_double)]

# Define function signatures
lib.solve_point_to_plane_distance.argtypes = [Point3D, Point3D, Point3D, Point3D, ctypes.POINTER(Point3D)]
lib.solve_point_to_plane_distance.restype = ctypes.c_double

def read_points_from_file(filename):
    """Read points from the data file"""
    points = {}
    \end{lstlisting}
\end{frame}

\begin{frame}[fragile]
    \frametitle{Python Code}
    \begin{lstlisting}
if not os.path.exists(filename):
        print(f"File {filename} not found. Running C program first...")
        # If points.dat doesn't exist, we need to run the C program
        os.system('make main')

    with open(filename, 'r') as f:
        for line in f:
            parts = line.strip().split()
            if len(parts) == 4:
                label = parts[0]
                x, y, z = map(float, parts[1:])
                points[label] = (x, y, z)

    return points
    \end{lstlisting}
\end{frame}

\begin{frame}[fragile]
    \frametitle{Python Code}
    \begin{lstlisting}
def create_visualization(points):
    """Create 3D visualization of points and plane"""
    fig = plt.figure(figsize=(12, 10))
    ax = fig.add_subplot(111, projection='3d')

    # Extract point coordinates
    A = np.array(points['A'])
    B = np.array(points['B'])
    C = np.array(points['C'])
    P = np.array(points['P'])
    Q = np.array(points['Q'])
    \end{lstlisting}
\end{frame}

\begin{frame}[fragile]
    \frametitle{Python Code}
    \begin{lstlisting}
    # Plot the points
    ax.scatter(*A, color='red', s=100, label='A(3,-1,2)')
    ax.scatter(*B, color='green', s=100, label='B(5,2,4)')
    ax.scatter(*C, color='blue', s=100, label='C(-1,-1,6)')
    ax.scatter(*P, color='orange', s=150, label='P(6,5,9)', marker='^')
    ax.scatter(*Q, color='purple', s=120, label='Q (foot of perpendicular)', marker='s')

    # Add text labels for points
    ax.text(A[0], A[1], A[2], '  A', fontsize=12)
    ax.text(B[0], B[1], B[2], '  B', fontsize=12)
    ax.text(C[0], C[1], C[2], '  C', fontsize=12)
    ax.text(P[0], P[1], P[2], '  P', fontsize=12)
    ax.text(Q[0], Q[1], Q[2], '  Q', fontsize=12)
    \end{lstlisting}
\end{frame}

\begin{frame}[fragile]
    \frametitle{Python Code}
    \begin{lstlisting}
# Draw line from P to Q (perpendicular distance)
    ax.plot([P[0], Q[0]], [P[1], Q[1]], [P[2], Q[2]], 'k--', linewidth=2, label='Perpendicular distance')
    
    # Create a grid around the points
    all_points = np.array([A, B, C, P, Q])
    x_min, x_max = all_points[:, 0].min() - 2, all_points[:, 0].max() + 2
    y_min, y_max = all_points[:, 1].min() - 2, all_points[:, 1].max() + 2

    xx, yy = np.meshgrid(np.linspace(x_min, x_max, 20), np.linspace(y_min, y_max, 20))
    zz = (19 - 3*xx + 4*yy) / 3

    # Plot the plane
    ax.plot_surface(xx, yy, zz, alpha=0.3, color='lightblue')
    \end{lstlisting}
\end{frame}

\begin{frame}[fragile]
    \frametitle{Python Code}
    \begin{lstlisting}
# Draw triangle ABC on the plane to show the plane boundary
    triangle_x = [A[0], B[0], C[0], A[0]]
    triangle_y = [A[1], B[1], C[1], A[1]]
    triangle_z = [A[2], B[2], C[2], A[2]]
    ax.plot(triangle_x, triangle_y, triangle_z, 'r-', linewidth=2, label='Triangle ABC')

    # Set labels and title
    ax.set_xlabel('X')
    ax.set_ylabel('Y')
    ax.set_zlabel('Z')
    ax.set_title('4.7.64')
    ax.legend()

    # Set equal aspect ratio
    ax.set_box_aspect([1,1,1])
    \end{lstlisting}
\end{frame}

\begin{frame}[fragile]
    \frametitle{Python Code}
    \begin{lstlisting}
# Save the figure
    plt.tight_layout()
    plt.savefig('fig1.png', dpi=300, bbox_inches='tight')
    plt.close()

    print("Visualization saved as fig1.png")

def main():
    """Main function"""
    # Define the points from the problem
    A = Point3D(3.0, -1.0, 2.0)
    B = Point3D(5.0, 2.0, 4.0)
    C = Point3D(-1.0, -1.0, 6.0)
    P = Point3D(6.0, 5.0, 9.0)
    Q = Point3D()
    \end{lstlisting}
\end{frame}

\begin{frame}[fragile]
    \frametitle{Python Code}
    \begin{lstlisting}
 print("Using shared library to solve the problem...")
    print("Points:")
    print(f"A = ({A.x}, {A.y}, {A.z})")
    print(f"B = ({B.x}, {B.y}, {B.z})")
    print(f"C = ({C.x}, {C.y}, {C.z})")
    print(f"P = ({P.x}, {P.y}, {P.z})")

    try:
        # Call the C function
        distance = lib.solve_point_to_plane_distance(A, B, C, P, ctypes.byref(Q))
        print(f"\nCalculated distance: {distance:.6f} units")
        print(f"Foot of perpendicular Q: ({Q.x:.6f}, {Q.y:.6f}, {Q.z:.6f})")

        # Read points from file and create visualization
        points_dict = read_points_from_file('points.dat')
        print(f"\nPoints read from file: {points_dict}")
    \end{lstlisting}
\end{frame}

\begin{frame}[fragile]
    \frametitle{Python Code}
    \begin{lstlisting}
create_visualization(points_dict)

    except Exception as e:
        print(f"Error: {e}")
        print("Make sure to compile the shared library first with: make plane.so")

if __name__ == "__main__":
    main()
    \end{lstlisting}
\end{frame}

\begin{frame}{Plane}
   \centering
    \includegraphics[width=\columnwidth, height=0.8\textheight, keepaspectratio]{figs/fig1.png}
    \label{fig:Beamer/figs/fig1.png}
\end{frame}


\end{document}