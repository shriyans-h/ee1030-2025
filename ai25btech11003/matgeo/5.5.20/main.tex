\let\negmedspace\undefined
\let\negthickspace\undefined
\documentclass[journal]{IEEEtran}
\usepackage[a5paper, margin=10mm, onecolumn]{geometry}
%\usepackage{lmodern} 
\usepackage{tfrupee} 

\setlength{\headheight}{1cm} 
\setlength{\headsep}{0mm}     

\usepackage{gvv-book}
\usepackage{gvv}
\usepackage{cite}
\usepackage{amsmath,amssymb,amsfonts,amsthm}
\usepackage{algorithmic}
\usepackage{graphicx}
\usepackage{textcomp}
\usepackage{xcolor}
\usepackage{txfonts}
\usepackage{listings}
\usepackage{enumitem}
\usepackage{mathtools}
\usepackage{gensymb}
\usepackage{comment}
\usepackage[breaklinks=true]{hyperref}
\usepackage{tkz-euclide} 
\usepackage{listings}                                        
\def\inputGnumericTable{}                                 
\usepackage[latin1]{inputenc}                                
\usepackage{color}                                            
\usepackage{array}                                            
\usepackage{longtable}                                       
\usepackage{calc}                                             
\usepackage{multirow}                                         
\usepackage{hhline}                                           
\usepackage{ifthen}                                           
\usepackage{lscape}

\begin{document}

\bibliographystyle{IEEEtran}
\vspace{3cm}

\title{5.5.20}
\author{AI25BTECH11003 - Bhavesh Gaikwad}
{\let\newpage\relax\maketitle}

\renewcommand{\thefigure}{\theenumi}
\renewcommand{\thetable}{\theenumi}
\setlength{\intextsep}{10pt} 


\numberwithin{equation}{enumi}
\numberwithin{figure}{enumi}
\renewcommand{\thetable}{\theenumi}


\textbf{Question}: 
Using elementary row transformations, find the inverse of the matrix
$$\myvec{3 & 0 & -1 \\ 2 & 3 & 0 \\ 0 & 4 & 1}$$

\textbf{Solution:}\\
Let $\vec{A} = \myvec{3 & 0 & -1 \\ 2 & 3 & 0 \\ 0 & 4 & 1}$\\

Augment the matrix $\vec{A}$ with the identity
\begin{align}
[\vec{A} \, | \, \vec{I}] =
\left(
\begin{array}{ccc|ccc}
3 & 0 & -1 & 1 & 0 & 0 \\
2 & 3 & 0 & 0 & 1 & 0 \\
0 & 4 & 1 & 0 & 0 & 1 \\
\end{array}
\right)
\end{align}

Row Transformation-1: $R_1 \rightarrow \frac{R_1}{3}$
\begin{align}
\left(
\begin{array}{ccc|ccc}
1 & 0 & -\frac{1}{3} & \frac{1}{3} & 0 & 0 \\
2 & 3 & 0 & 0 & 1 & 0 \\
0 & 4 & 1 & 0 & 0 & 1 \\
\end{array}
\right)
\end{align}

Row Transformation-2: $R_2 \rightarrow R_2 - 2R_1$
\begin{align}
\left(
\begin{array}{ccc|ccc}
1 & 0 & -\frac{1}{3} & \frac{1}{3} & 0 & 0 \\
0 & 3 & \frac{2}{3} & -\frac{2}{3} & 1 & 0 \\
0 & 4 & 1 & 0 & 0 & 1 \\
\end{array}
\right)
\end{align}

Row Transformation-3: $R_2 \rightarrow \frac{R_2}{3}$
\begin{align}
\left(
\begin{array}{ccc|ccc}
1 & 0 & -\frac{1}{3} & \frac{1}{3} & 0 & 0 \\
0 & 1 & \frac{2}{9} & -\frac{2}{9} & \frac{1}{3} & 0 \\
0 & 4 & 1 & 0 & 0 & 1 \\
\end{array}
\right)
\end{align}

Row Transformations 4 and 5: Replace $R_1 \rightarrow R_1 + \frac{1}{3}R_2$ And $R_3 \rightarrow R_3 - 4R_2$
\begin{align}
\left(
\begin{array}{ccc|ccc}
1 & 0 & -\frac{7}{27} & \frac{7}{27} & \frac{1}{9} & 0 \\
0 & 1 & \frac{2}{9} & -\frac{2}{9} & \frac{1}{3} & 0 \\
0 & 0 & 1 & 8 & -4 & 9 \\
\end{array}
\right)
\end{align}

Row Transformation-6: $R_3 \rightarrow 9R_3$
\begin{align}
\left(
\begin{array}{ccc|ccc}
1 & 0 & -\frac{7}{27} & \frac{7}{27} & \frac{1}{9} & 0 \\
0 & 1 & \frac{2}{9} & -\frac{2}{9} & \frac{1}{3} & 0 \\
0 & 0 & 1 & 8 & -4 & 9 \\
\end{array}
\right)
\end{align}

Row Transformations 7 and 8: $R_1 \rightarrow R_1 + \frac{7}{27}R_3 $ And $R_2 \rightarrow R_2 - \frac{2}{9}R_3 $
\begin{align}
\left(
\begin{array}{ccc|ccc}
1 & 0 & 0 & \frac{7}{3} & -\frac{25}{27} & \frac{7}{3} \\
0 & 1 & 0 & -2 & \frac{11}{9} & -2 \\
0 & 0 & 1 & 8 & -4 & 9 \\
\end{array}
\right)
\end{align}

The Inverse Matrix of $\vec{A}$:
\begin{align}
\vec{A}^{-1} = \myvec{ \dfrac{7}{3} & -\dfrac{25}{27} & \dfrac{7}{3} \\[1ex]
-2 & \dfrac{11}{9} & -2 \\[1ex]
8 & -4 & 9 }
\end{align}


\end{document}  
