\documentclass{beamer}
\usepackage[utf8]{inputenc}

\usetheme{Madrid}
\usecolortheme{default}
\usepackage{amsmath,amssymb,amsfonts,amsthm}
\usepackage{txfonts}
\usepackage{tkz-euclide}
\usepackage{listings}
\usepackage{adjustbox}
\usepackage{array}
\usepackage{tabularx}
\usepackage{gvv}
\usepackage{lmodern}
\usepackage{circuitikz}
\usepackage{tikz}
\usepackage{graphicx}

\setbeamertemplate{page number in head/foot}[totalframenumber]

\usepackage{tcolorbox}
\tcbuselibrary{minted,breakable,xparse,skins}



\definecolor{bg}{gray}{0.95}
\DeclareTCBListing{mintedbox}{O{}m!O{}}{%
  breakable=true,
  listing engine=minted,
  listing only,
  minted language=#2,
  minted style=default,
  minted options={%
    linenos,
    gobble=0,
    breaklines=true,
    breakafter=,,
    fontsize=\small,
    numbersep=8pt,
    #1},
  boxsep=0pt,
  left skip=0pt,
  right skip=0pt,
  left=25pt,
  right=0pt,
  top=3pt,
  bottom=3pt,
  arc=5pt,
  leftrule=0pt,
  rightrule=0pt,
  bottomrule=2pt,
  toprule=2pt,
  colback=bg,
  colframe=orange!70,
  enhanced,
  overlay={%
    \begin{tcbclipinterior}
    \fill[orange!20!white] (frame.south west) rectangle ([xshift=20pt]frame.north west);
    \end{tcbclipinterior}},
  #3,
}
\lstset{
    language=C,
    basicstyle=\ttfamily\small,
    keywordstyle=\color{blue},
    stringstyle=\color{orange},
    commentstyle=\color{green!60!black},
    numbers=left,
    numberstyle=\tiny\color{gray},
    breaklines=true,
    showstringspaces=false,
}
%------------------------------------------------------------

\title
{3.2.19}
\date{September 2,2025}
\author 
{AI25BTECH11003 - Bhavesh Gaikwad}



\begin{document}


\frame{\titlepage}
\begin{frame}{Question}
Two sides of a triangle are of lengths 5cm and 1.5cm. The length of the third side of the triangle cannot be\\
a) 3.6 cm\\
b) 4.1 cm\\
c) 3.8 cm\\
d) 3.4 cm\\
\end{frame}


\begin{frame}[fragile]
    \frametitle{Theoretical Solution}
    Let a=5 cm, b=1.5 cm, and c be the third side. For each option we test:

\begin{align}
1.\quad &a + b > c,\\
2.\quad &a + c > b,\\
3.\quad &b + c > a.
\end{align}

If all three hold, the triangle exists; otherwise it does not.

\bigskip

Option (A): c = 3.6 cm

\begin{align}
5 + 1.5 &> 3.6 \quad\Rightarrow\quad 6.5 > 3.6\quad\checkmark,\\
5 + 3.6 &> 1.5 \quad\Rightarrow\quad 8.6 > 1.5\quad\checkmark,\\
1.5 + 3.6 &> 5 \quad\Rightarrow\quad 5.1 > 5\quad\checkmark.
\end{align}

All conditions satisfied $\;\Rightarrow$ triangle exists.



\end{frame}

\begin{frame}[fragile]
    \frametitle{Theoretical Solution}
\begin{figure}[htbp]
\centering
\includegraphics[width=\columnwidth]{figs/fig1.png}
\caption{Triangle}
\label{fig:figs/fig1.png}
\end{figure}
\end{frame}

\begin{frame}[fragile]
    \frametitle{Theoretical Solution}
Option (B): c = 4.1 cm

\begin{align}
5 + 1.5 &> 4.1 \quad\Rightarrow\quad 6.5 > 4.1\quad\checkmark,\\
5 + 4.1 &> 1.5 \quad\Rightarrow\quad 9.1 > 1.5\quad\checkmark,\\
1.5 + 4.1 &> 5 \quad\Rightarrow\quad 5.6 > 5\quad\checkmark.
\end{align}

All conditions satisfied $\;\Rightarrow$ triangle exists.

\begin{figure}[htbp]
\centering
\includegraphics[width=0.5\columnwidth]{figs/fig2.png}
\caption{Triangle}
\label{fig:figs/fig2.png}
\end{figure}

\bigskip

\newpage

\end{frame}

\begin{frame}[fragile]
    \frametitle{Theoretical Solution}
 
Option (C): c = 3.8 cm

\begin{align}
5 + 1.5 &> 3.8 \quad\Rightarrow\quad 6.5 > 3.8\quad\checkmark,\\
5 + 3.8 &> 1.5 \quad\Rightarrow\quad 8.8 > 1.5\quad\checkmark,\\
1.5 + 3.8 &> 5 \quad\Rightarrow\quad 5.3 > 5\quad\checkmark.
\end{align}

All conditions satisfied $\;\Rightarrow$ triangle exists.

\begin{figure}[htbp]
\centering
\includegraphics[width=0.3\columnwidth]{figs/fig1.png}
\caption{Triangle}
\label{fig:figs/fig3.png}
\end{figure}

\bigskip

\end{frame}

\begin{frame}[fragile]
    \frametitle{Theoretical Solution}
 Option (D): c = 3.4 cm

\begin{align}
5 + 1.5 &> 3.4 \quad\Rightarrow\quad 6.5 > 3.4\quad\checkmark,\\
5 + 3.4 &> 1.5 \quad\Rightarrow\quad 8.4 > 1.5\quad\checkmark,\\
1.5 + 3.4 &> 5 \quad\Rightarrow\quad 4.9 > 5\quad\times.
\end{align}

Condition 3 fails $\;\Rightarrow$ triangle does \emph{not} exist.
\begin{figure}[htbp]
\centering
\includegraphics[width=0.4\columnwidth]{figs/fig4.png}
\caption{Triangle}
\label{fig:figs/fig4.png}
\end{figure}

$\boxed{\therefore \, \text{Option D is Incorrect.}}$

\end{frame}



\begin{frame}[fragile]
    \frametitle{Python Code}
    \begin{lstlisting}
import matplotlib.pyplot as plt
import numpy as np

def draw_segment_with_label(P, Q, label, style='k-', text_offset=(0.0, 0.0)):
    """Draw segment PQ and place a rotated length label near its midpoint."""
    P = np.array(P, dtype=float)
    Q = np.array(Q, dtype=float)
    plt.plot([P[0], Q[0]], [P[1], Q[1]], style, linewidth=2)

    mid = (P + Q) / 2.0
    dx, dy = Q - P
    angle_deg = np.degrees(np.arctan2(dy, dx))
    plt.text(
        mid[0] + text_offset[0],
        mid[1] + text_offset[1],
        label,
    \end{lstlisting}
\end{frame}

\begin{frame}[fragile]
    \frametitle{Python Code}
    \begin{lstlisting}
         ha='center',
        va='center',
        rotation=angle_deg,
        rotation_mode='anchor',
        fontsize=11,
        bbox=dict(boxstyle='round,pad=0.2', fc='white', ec='none', alpha=0.8)
    )

def plot_triangle_with_lengths(a, b, c, filename, incomplete=False):
    """
    Draw triangle with sides: AB=a, AC=b, BC=c.
    If incomplete=True, depict an impossible case by showing dashed partial sides from A and B.
    """
   
    \end{lstlisting}
\end{frame}

\begin{frame}[fragile]
    \frametitle{Python Code}
    \begin{lstlisting}
  # Fix base AB on x-axis
    A = np.array([0.0, 0.0])
    B = np.array([a,   0.0])

    plt.figure(figsize=(4.2, 4.0))

    # Always draw AB
    draw_segment_with_label(A, B, f"{a:.1f} cm", style='k-', text_offset=(0, -0.15))

    if not incomplete:
        # Compute C using law of cosines around A
        # cos(theta) = (a^2 + b^2 - c^2) / (2ab)
        cos_theta = (a*a + b*b - c*c) / (2.0*a*b)
        cos_theta = np.clip(cos_theta, -1.0, 1.0)
        theta = np.arccos(cos_theta)
        C = np.array([b*np.cos(theta), b*np.sin(theta)])

    \end{lstlisting}
\end{frame}

\begin{frame}[fragile]
    \frametitle{Python Code}
    \begin{lstlisting}
        # Draw AC and BC
        draw_segment_with_label(A, C, f"{b:.1f} cm", style='k-')
        draw_segment_with_label(B, C, f"{c:.1f} cm", style='k-')

        # Mark points
        plt.scatter([A[0], B[0], C[0]], [A[1], B[1], C[1]], c='k', s=18)
        plt.text(A[0], A[1]-0.28, 'A', ha='center')
        plt.text(B[0], B[1]-0.28, 'B', ha='center')
        plt.text(C[0], C[1]+0.18, 'C', ha='center')
    \end{lstlisting}
\end{frame}

\begin{frame}[fragile]
    \frametitle{Python Code}
    \begin{lstlisting}
 
    else:
        # Depict an impossible triangle: show partial (dashed) edges toward a "would-be" C
        # Choose a visual point above AB just for indicating direction
        Cg = np.array([a*0.30, b*0.90])  # purely for illustration

        # Partial dashed from A toward Cg
        A_dash = A + 0.55*(Cg - A)
        draw_segment_with_label(A, A_dash, f"{b:.1f} cm", style='k--')

        # Partial dashed from B toward Cg
        B_dash = B + 0.55*(Cg - B)
        draw_segment_with_label(B, B_dash, f"{c:.1f} cm", style='k--')

    \end{lstlisting}
\end{frame}

\begin{frame}[fragile]
    \frametitle{Python Code}
    \begin{lstlisting}
 
        # Mark points
        plt.scatter([A[0], B[0], A_dash[0], B_dash[0]], [A[1], B[1], A_dash[1], B_dash[1]], c='k', s=18)
        plt.text(A[0], A[1]-0.28, 'A', ha='center')
        plt.text(B[0], B[1]-0.28, 'B', ha='center')
        plt.text((A_dash[0]+B_dash[0])/2, (A_dash[1]+B_dash[1])/2 + 0.18, 'C ?', ha='center', color='crimson')

        # Add a note for impossibility
        plt.text(a*0.5, -0.75, 'Triangle cannot close', ha='center', color='crimson', fontsize=10)
    \end{lstlisting}
\end{frame}

\begin{frame}[fragile]
    \frametitle{Python Code}
    \begin{lstlisting}
  # Final touches
    plt.axis('equal')
    plt.axis('off')
    plt.tight_layout()
    plt.savefig(filename, dpi=180, bbox_inches='tight')
    plt.close()

# Given sides
a = 5.0   # AB
b = 1.5   # AC

# Options for BC
options = [3.6, 4.1, 3.8, 3.4]

# Generate four figures with side-length labels
for i, c in enumerate(options, start=1):
    out = f"fig{i}.png"
    plot_triangle_with_lengths(a, b, c, out, incomplete=(i == 4)) 
    \end{lstlisting}
\end{frame}

\begin{frame}{Vector Representation}
   \centering
    \includegraphics[width=\columnwidth, height=0.8\textheight, keepaspectratio]{figs/fig1.png}
    \label{fig:Beamer/figs/fig1.png}
\end{frame}


\end{document}