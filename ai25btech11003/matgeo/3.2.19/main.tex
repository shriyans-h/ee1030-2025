\let\negmedspace\undefined
\let\negthickspace\undefined
\documentclass[journal]{IEEEtran}
\usepackage[a5paper, margin=10mm, onecolumn]{geometry}
%\usepackage{lmodern} 
\usepackage{tfrupee} 

\setlength{\headheight}{1cm} 
\setlength{\headsep}{0mm}     

\usepackage{gvv-book}
\usepackage{gvv}
\usepackage{cite}
\usepackage{amsmath,amssymb,amsfonts,amsthm}
\usepackage{algorithmic}
\usepackage{graphicx}
\usepackage{textcomp}
\usepackage{xcolor}
\usepackage{txfonts}
\usepackage{listings}
\usepackage{enumitem}
\usepackage{mathtools}
\usepackage{gensymb}
\usepackage{comment}
\usepackage[breaklinks=true]{hyperref}
\usepackage{tkz-euclide} 
\usepackage{listings}                                        
\def\inputGnumericTable{}                                 
\usepackage[latin1]{inputenc}                                
\usepackage{color}                                            
\usepackage{array}                                            
\usepackage{longtable}                                       
\usepackage{calc}                                             
\usepackage{multirow}                                         
\usepackage{hhline}                                           
\usepackage{ifthen}                                           
\usepackage{lscape}

\begin{document}

\bibliographystyle{IEEEtran}
\vspace{3cm}

\title{3.2.19}
\author{AI25BTECH11003 - Bhavesh Gaikwad}
{\let\newpage\relax\maketitle}

\renewcommand{\thefigure}{\theenumi}
\renewcommand{\thetable}{\theenumi}
\setlength{\intextsep}{10pt} 


\numberwithin{equation}{enumi}
\numberwithin{figure}{enumi}
\renewcommand{\thetable}{\theenumi}


\textbf{Question}: Two sides of a triangle are of lengths 5cm and 1.5cm. The length of the third side of the triangle cannot be\\
a) 3.6 cm\\
b) 4.1 cm\\
c) 3.8 cm\\
d) 3.4 cm\\

\textbf{Solution:}\\

Let the vector along side AB be $\vec{a}$ \\
Let the vector along side BC be $\vec{b}$ \\
Let the vector along side AC be $\vec{c}$ \\
Let the angle between $\vec{a} \, and \, \vec{b}$ be $\theta$.\\

Given: 
\begin{equation}
    \norm{\vec{a}} = 5, \, \, \norm{\vec{b}} = 1.5
\end{equation}\\

By Triangle Law of Vector Addition,
\begin{equation}
    \vec{a} + \vec{b} = \vec{c}
\end{equation}

\begin{equation}
\vec{c}^T\vec{c} = (\vec{a}+\vec{b})^T(\vec{a}+\vec{b})    
\end{equation}

\begin{equation}
    \vec{c}^T\vec{c} = (\vec{a}^T\vec{a}) + (\vec{b}^T\vec{b}) + (\vec{a}^T\vec{b}) + (\vec{b}^T\vec{a})
\end{equation}

We know that,
\begin{equation}
    \vec{a}^T\vec{a}= \norm{\vec{a}}^2 = 25, \, \,
    \vec{b}^T\vec{b} = \norm{\vec{b}}^2 = 2.25, \, \,
    \vec{c}^T\vec{c}=\norm{\vec{c}}^2, \, \,
   \vec{a}^T\vec{b}=\vec{b}^T\vec{a}=\norm{\vec{a}}\norm{\vec{b}}\cos{(\theta)}
\end{equation}

From Equation 0.4 and 0.5, 
\begin{equation}
    \norm{\vec{c}}^2 = 27.25 + 15\cos{(\theta)}
\end{equation}

Since '$\theta$' is the angle between two vectors, Therefore 
\begin{equation}
\theta \, \epsilon \, (0,\pi)    
\end{equation}

\newpage

The maximum value of $\norm{\vec{c}}^2$ will occur when $\cos{(\theta)}=1$ OR $\theta = 0$
\begin{equation}
    \text{Therefore the maximum value of} \norm{\vec{c}}^2 \, is \, 42.25. \\
    \Rightarrow \, \text{The maximum value of} \norm{\vec{c}} is \, 6.5.
\end{equation}\\

The minimum value of $\norm{\vec{c}}^2$ will occur when $\cos{(\theta)}=-1$ OR $\theta = \pi$
\begin{equation}
    \text{Therefore the minimum value of} \norm{\vec{c}}^2 \, is \, 12.25. \\
    \Rightarrow \, \text{The minimum value of} \norm{\vec{c}} is \, 3.5.
\end{equation}

\begin{equation}
    \therefore \text{The Range of} \, \norm{\vec{c}} \, \text{for triangle to exist: } \norm{\vec{c}} \, \epsilon \, (3.5, 6.5).
\end{equation}

From option D of the Question, $\norm{\vec{c}}$ = 3.4 cm\\
But by Equation 0.10, $\norm{\vec{c}} \neq 3.4$, Since $\norm{\vec{c}} > $  3.5

\begin{align}
    \boxed{\text{Option D of the Question is Incorrect and} \norm{\vec{c}} \neq 3.4 \, cm}
\end{align}

\begin{figure}[htbp]
    \centering
    \includegraphics[width=\columnwidth]{figs/fig1.png}
    \caption{Vector Representation}
    \label{fig:fig/fig1.png}
\end{figure}
\end{document}  
