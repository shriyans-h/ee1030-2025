\let\negmedspace\undefined
\let\negthickspace\undefined
\documentclass[journal]{IEEEtran}
\usepackage[a5paper, margin=10mm, onecolumn]{geometry}
%\usepackage{lmodern} 
\usepackage{tfrupee} 

\setlength{\headheight}{1cm} 
\setlength{\headsep}{0mm}     

\usepackage{gvv-book}
\usepackage{gvv}
\usepackage{cite}
\usepackage{amsmath,amssymb,amsfonts,amsthm}
\usepackage{algorithmic}
\usepackage{graphicx}
\usepackage{textcomp}
\usepackage{xcolor}
\usepackage{txfonts}
\usepackage{listings}
\usepackage{enumitem}
\usepackage{mathtools}
\usepackage{gensymb}
\usepackage{comment}
\usepackage[breaklinks=true]{hyperref}
\usepackage{tkz-euclide} 
\usepackage{listings}                                        
\def\inputGnumericTable{}                                 
\usepackage[latin1]{inputenc}                                
\usepackage{color}                                            
\usepackage{array}                                            
\usepackage{longtable}                                       
\usepackage{calc}                                             
\usepackage{multirow}                                         
\usepackage{hhline}                                           
\usepackage{ifthen}                                           
\usepackage{lscape}

\begin{document}

\bibliographystyle{IEEEtran}
\vspace{3cm}

\title{9.8.5}
\author{AI25BTECH11003 - Bhavesh Gaikwad}
{\let\newpage\relax\maketitle}

\renewcommand{\thefigure}{\theenumi}
\renewcommand{\thetable}{\theenumi}
\setlength{\intextsep}{10pt} 


\numberwithin{equation}{enumi}
\numberwithin{figure}{enumi}
\renewcommand{\thetable}{\theenumi}


\textbf{Question}: Let $\vec{S}$ be the focus of the parabola $y^2 = 8x$ and let PQ be the common chord of the circle $x^2 + y^2 - 2x - 4y = 0$ and the given parabola. The area of the triangle PQS is\\\\

\textbf{Solution:}\\
  $\text{Given: }$\\
$\quad \text{Circle: }x^2 + y^2 - 2x - 4y = 0$\\
$\text{Parabola: }y^2 = 8x$\\\\


Parameters of the Circle:
\begin{equation}
\vec{V}_1 = \myvec{1 & 0 \\ 0 & 1}, \, \vec{u}_1 = \myvec{-1 \\ -2}, \, f_1 = 0   
\end{equation}

Parameters of the Parabola:
\begin{equation}
\vec{V}_2 = \myvec{0 & 0 \\ 0 & 1}, \, \vec{u}_2 = \myvec{-4 \\ 0}, \, f_2 = 0, \, \vec{S} = \myvec{2e \\ 0 } = \myvec{2 \\ 0}   
\end{equation}

Points of Intersection of Circle and Parabola can be given as:
\begin{equation}
\vec{X}^\top(\vec{V}_1+\mu\vec{V}_2)\vec{X} + 2(\vec{u}_1+\mu\vec{u}_2)^\top\vec{X} + (f_1 + \mu f_2)    
\end{equation}

\begin{equation}
\vec{X}^\top\myvec{1 & 0 \\ 0 & 1+\mu}\vec{X} - 2\myvec{1+4\mu & 2}\vec{X} = 0    
\end{equation}

From Equation 0.4, We get
\begin{equation}
    \vec{X}=\myvec{1+4\mu \\ \dfrac{2}{1+\mu}}
\end{equation}

Putting Value of $\vec{X}$ in Equation 0.4, We get points of intersection as:
\begin{equation}
    \vec{X}_1 = \myvec{0 \\ 0} \qquad \& \qquad \vec{X}_2 = \myvec{2 \\ 4}
\end{equation}

Therefore,
Let $\vec{P} = \myvec{0 \\ 0}$ and $\vec{Q} = \myvec{2 \\ 4}$\\\\

\newpage

The Area of Triangle PQS is:
\begin{equation}
    Area(\triangle PQS) = \dfrac{1}{2}\norm{\vec{SP} \times \vec{QP}} = 4
\end{equation}

\begin{align*}
    \boxed{\text{The Area of $\triangle PQS$ is 4 sq.units.}}
\end{align*}


\begin{figure}[htbp]
    \centering
    \includegraphics[width=0.9\columnwidth]{figs/fig1.png}
    \caption{Intersection of Two Conics and Triangle PQS}
    \label{fig:figs/fig1.png}
\end{figure}

\end{document}  
