\documentclass{beamer}
\let\vec\mathbf
\mode<presentation>
\usepackage{amsmath}
\usepackage{amssymb}
%\usepackage{advdate}
\usepackage{adjustbox}
%\usepackage{subcaption}
\usepackage{enumitem}
\usepackage{multicol}
\usepackage{mathtools}
\usepackage{listings}
\usepackage{url}
\usetheme{Boadilla}
\usecolortheme{lily}
\setbeamertemplate{footline}
{
  \leavevmode%
  \hbox{%
  \begin{beamercolorbox}[wd=\paperwidth,ht=2.25ex,dp=1ex,right]{author in head/foot}%
    \insertframenumber{} / \inserttotalframenumber\hspace*{2ex} 
  \end{beamercolorbox}}%
  \vskip0pt%
}
\setbeamertemplate{navigation symbols}{}
\providecommand{\nCr}[2]{\,^{#1}C_{#2}} % nCr
\providecommand{\nPr}[2]{\,^{#1}P_{#2}} % nPr
\providecommand{\mbf}{\mathbf}
\providecommand{\pr}[1]{\ensuremath{\Pr\left(#1\right)}}
\providecommand{\qfunc}[1]{\ensuremath{Q\left(#1\right)}}
\providecommand{\sbrak}[1]{\ensuremath{{}\left[#1\right]}}
\providecommand{\lsbrak}[1]{\ensuremath{{}\left[#1\right.}}
\providecommand{\rsbrak}[1]{\ensuremath{{}\left.#1\right]}}
\providecommand{\brak}[1]{\ensuremath{\left(#1\right)}}
\providecommand{\lbrak}[1]{\ensuremath{\left(#1\right.}}
\providecommand{\rbrak}[1]{\ensuremath{\left.#1\right)}}
\providecommand{\cbrak}[1]{\ensuremath{\left\{#1\right\}}}
\providecommand{\lcbrak}[1]{\ensuremath{\left\{#1\right.}}
\providecommand{\rcbrak}[1]{\ensuremath{\left.#1\right\}}}
\theoremstyle{remark}
\newtheorem{rem}{Remark}
\newcommand{\sgn}{\mathop{\mathrm{sgn}}}

\providecommand{\res}[1]{\Res\displaylimits_{#1}} 
\providecommand{\norm}[1]{\left\lVert#1\right\rVert}
\providecommand{\mtx}[1]{\mathbf{#1}}
\providecommand{\abs}[1]{\left\vert#1\right\vert}
\providecommand{\fourier}{\overset{\mathcal{F}}{ \rightleftharpoons}}
%\providecommand{\hilbert}{\overset{\mathcal{H}}{ \rightleftharpoons}}
\providecommand{\system}{\overset{\mathcal{H}}{ \longleftrightarrow}}
	%\newcommand{\solution}[2]{\textbf{Solution:}{#1}}
%\newcommand{\solution}{\noindent \textbf{Solution: }}align
\providecommand{\dec}[2]{\ensuremath{\overset{#1}{\underset{#2}{\gtrless}}}}
\newcommand{\myvec}[1]{\ensuremath{\begin{pmatrix}#1\end{pmatrix}}}

\title{Matrices in Geometry - 2.6.66}
\author{EE25BTECH11037  Divyansh}
\date{Sept, 2025}

\begin{document}

\maketitle


\section{Problem}
\begin{frame}
\frametitle{Problem Statement}
If vectors $\vec{a},\ \vec{b}, \ \vec{c}$ are coplanar, show that
\begin{align*}
\abs{\myvec{
\vec{a} & \vec{b} & \vec{c}\\
\vec{a}\cdot \vec{a} & \vec{a}\cdot \vec{b} & \vec{a}\cdot \vec{c} \\
\vec{b}\cdot \vec{a} & \vec{b}\cdot \vec{b} & \vec{b}\cdot \vec{c} 
}} = 0
\end{align*}
\end{frame}

\section{Solution}
\begin{frame}{Solution}
   
We have to show that 
\begin{align}
\Delta=\abs{\vec{A}} = \abs{\myvec{
\vec{a} & \vec{b} & \vec{c}\\
\vec{a}\cdot \vec{a} & \vec{a}\cdot \vec{b} & \vec{a}\cdot \vec{c} \\
\vec{b}\cdot \vec{a} & \vec{b}\cdot \vec{b} & \vec{b}\cdot \vec{c} 
}} =0
\end{align}
Since $\vec{a}, \ \vec{b}, \ \vec{c}$ are coplanar, $\vec{a}, \ \vec{b}, \ \vec{c}$ are linearly dependent. Therefore, we can express $\vec{c}$ in terms of $\vec{a}$ and $\vec{b}$ as:
\begin{align}
    \vec{c}=m\vec{a}+n\vec{b}, \text{where m, n are scalars.}
\end{align}
\end{frame}

\begin{frame}{Solution}
\begin{align}
\text{Since, }\abs{\vec{A}}=\abs{\vec{A}^{\top}} \implies \Delta=\abs{\vec{A}^{\top}}\\
    \abs{\vec{A}^{\top}}=\abs{\myvec{
                        \vec{a} & \vec{a}\cdot \vec{a} & \vec{b}\cdot \vec{a}\\
                        \vec{b}& \vec{a}\cdot \vec{b} & \vec{b}\cdot \vec{b}\\
                        \vec{c}& \vec{a}\cdot \vec{c} & \vec{b}\cdot \vec{c} 
                  }} \overset{R_3 \rightarrow R_3 - m R_1 - n R_2}{\longrightarrow}\\
    \abs{\myvec{
                \vec{a} & \vec{a}\cdot \vec{a} & \vec{b}\cdot \vec{a}\\
                \vec{b}& \vec{a}\cdot \vec{b} & \vec{b}\cdot \vec{b}\\
                \vec{c}-m\vec{a}-n\vec{b}& \vec{a}\cdot \brak{\vec{c}-m\vec{a}-n\vec{b}} & \vec{b}\cdot \brak{\vec{c}-m\vec{a}-n\vec{b}}
            }}
\end{align}
\end{frame}
\begin{frame}{Solution}
\text{Since, $\vec{c}=m\vec{a}+n\vec{b}$, $\vec{A}$ becomes }
\begin{align}
    \abs{\vec{A}^{\top}}=\abs{\myvec{
                        \vec{a} & \vec{a}\cdot \vec{a} & \vec{b}\cdot \vec{a}\\
                        \vec{b}& \vec{a}\cdot \vec{b} & \vec{b}\cdot \vec{b}\\
                        0& 0& 0 
                  }}\\
    \implies \Delta = \abs{\vec{A}^{\top}}=0 \text{, as one of the rows is zero}
\end{align} 
Hence, Proved.
\end{frame}


\end{document}
