    \documentclass{beamer}
    \usepackage{siunitx}
    \usepackage{tfrupee}
    \let\vec\mathbf
    \mode<presentation>
    \usepackage{amsmath}
    \usepackage{amssymb}
    %\usepackage{advdate}
    \usepackage{adjustbox}
    %\usepackage{subcaption}
    \usepackage{enumitem}
    \usepackage{multicol}
    \usepackage{mathtools}
    \usepackage{listings}
    \usepackage{url}
    \usetheme{Boadilla}
    \usecolortheme{lily}
    \setbeamertemplate{footline}
    {
      \leavevmode%
      \hbox{%
      \begin{beamercolorbox}[wd=\paperwidth,ht=2.25ex,dp=1ex,right]{author in head/foot}%
        \insertframenumber{} / \inserttotalframenumber\hspace*{2ex} 
      \end{beamercolorbox}}%
      \vskip0pt%
    }
    \setbeamertemplate{navigation symbols}{}
    \providecommand{\nCr}[2]{\,^{#1}C_{#2}} % nCr
    \providecommand{\nPr}[2]{\,^{#1}P_{#2}} % nPr
    \providecommand{\mbf}{\mathbf}
    \providecommand{\pr}[1]{\ensuremath{\Pr\left(#1\right)}}
    \providecommand{\qfunc}[1]{\ensuremath{Q\left(#1\right)}}
    \providecommand{\sbrak}[1]{\ensuremath{{}\left[#1\right]}}
    \providecommand{\lsbrak}[1]{\ensuremath{{}\left[#1\right.}}
    \providecommand{\rsbrak}[1]{\ensuremath{{}\left.#1\right]}}
    \providecommand{\brak}[1]{\ensuremath{\left(#1\right)}}
    \providecommand{\lbrak}[1]{\ensuremath{\left(#1\right.}}
    \providecommand{\rbrak}[1]{\ensuremath{\left.#1\right)}}
    \providecommand{\cbrak}[1]{\ensuremath{\left\{#1\right\}}}
    \providecommand{\lcbrak}[1]{\ensuremath{\left\{#1\right.}}
    \providecommand{\rcbrak}[1]{\ensuremath{\left.#1\right\}}}
    \theoremstyle{remark}
    \newtheorem{rem}{Remark}
    \newcommand{\sgn}{\mathop{\mathrm{sgn}}}
    
    \providecommand{\res}[1]{\Res\displaylimits_{#1}} 
    \providecommand{\norm}[1]{\left\lVert#1\right\rVert}
    \providecommand{\mtx}[1]{\mathbf{#1}}
    \providecommand{\abs}[1]{\left\vert#1\right\vert}
    \providecommand{\fourier}{\overset{\mathcal{F}}{ \rightleftharpoons}}
    %\providecommand{\hilbert}{\overset{\mathcal{H}}{ \rightleftharpoons}}
    \providecommand{\system}{\overset{\mathcal{H}}{ \longleftrightarrow}}
    	%\newcommand{\solution}[2]{\textbf{Solution:}{#1}}
    %\newcommand{\solution}{\noindent \textbf{Solution: }}align
    \providecommand{\dec}[2]{\ensuremath{\overset{#1}{\underset{#2}{\gtrless}}}}
    \newcommand{\myvec}[1]{\ensuremath{\begin{pmatrix}#1\end{pmatrix}}}
    
    \title{Matrices in Geometry - 12.51}
    \author{EE25BTECH11037  Divyansh}
    \date{Sept, 2025}
    
    \begin{document}
    
    \maketitle
    
    
    \section{Problem}
    \begin{frame}
    \frametitle{Problem Statement}
    Let the eigenvalues of a square matrix $\vec{A}$ of order two be 1 and 2. The corresponding eigenvectors are $\myvec{0.6 \\ 0.8}$ and $\myvec{0.8 \\ -0.6}$, respectively. Then, the element $\vec{A}\brak{2, 2}$ is

    \begin{enumerate}[label=\alph*)]
        \item -0.48
        \item 0.48
        \item 1.36
        \item 1.64
    \end{enumerate}
  \end{frame}
    
    \section{Solution}
    \begin{frame}{Solution}
    The eigenvalues of $\vec{A}$ are $\lambda_1=1$ and $\lambda_2=2$.\\
Let the given eigenvectors be
\begin{align}
    \vec{v_1}=\myvec{0.6 \\ 0.8} \ , \ \vec{v_2}=\myvec{0.8 \\ -0.6}
\end{align}
Let
\begin{align}
    \vec{P}=\myvec{\vec{v_1} & \vec{v_2}}=\myvec{0.6 & 0.8 \\ 0.8 & -0.6}
\end{align}
The given eigen vectors $\vec{v_1}$ and $\vec{v_2}$ are orthonormal, that is, they are unit vectors and their scalar product is zero. 
\begin{align}
    \therefore \vec{P}^{\top} = \vec{P}^{-1}
\end{align}
    \end{frame}
    
    \begin{frame}{Solution}
    Using spectral decomposition, we can find the matrix $\vec{A}$.
\begin{align}
    \vec{A}=\vec{P}\vec{D}\vec{P}^{\top} \ , \ \vec{D}=\myvec{\lambda_1 & 0 \\ 0 & \lambda_2}=\myvec{1 & 0 \\ 0 &2}
\end{align}
\begin{align}
    \vec{A}=\myvec{0.6 & 0.8 \\ 0.8 & -0.6}\myvec{1 & 0 \\ 0 &2}\myvec{0.6 & 0.8 \\ 0.8 & -0.6}\\
    \implies \vec{A}=\myvec{1.64 & -0.48 \\ -0.48 & 1.36}
\end{align}
The element $\vec{A}\brak{2, 2} = 1.36$ which is option c)
    \end{frame}

    
    \end{document}
