\documentclass[journal,12pt,onecolumn]{IEEEtran}
\usepackage{cite}
\usepackage{caption}
\usepackage{graphicx}
\usepackage{amsmath,amssymb,amsfonts,amsthm}
\usepackage{algorithmic}
\usepackage{graphicx}
\usepackage{textcomp}
\usepackage{xcolor}
\usepackage{tfrupee}
\usepackage{txfonts}
\usepackage{listings}
\usepackage{enumitem}
\usepackage{mathtools}
\usepackage{gensymb}
\usepackage{comment}
\usepackage[breaklinks=true]{hyperref}
\usepackage{tkz-euclide} 
\usepackage{listings}
\usepackage{gvv}
%\def\inputGnumericTable{}
\usepackage[latin1]{inputenc} 
\usetikzlibrary{arrows.meta, positioning}
\usepackage{xparse}
\usepackage{color}                                            
\usepackage{array}                                            
\usepackage{longtable}                                       
\usepackage{calc}                                             
\usepackage{multirow}
\usepackage{multicol}
\usepackage{hhline}                                           
\usepackage{ifthen}                                           
\usepackage{lscape}
\usepackage{tabularx}
\usepackage{array}
\usepackage{float}
\usepackage{marvosym}
\usepackage{float}
%\newcommand{\define}{\stackrel{\triangle}{=}}
\theoremstyle{remark}
\usepackage{circuitikz}
\captionsetup{justification=centering}
\usepackage{tikz}

\title{Matrices in Geometry 10.5.5}
\author{EE25BTECH11037 - Divyansh}
\begin{document}
\vspace{3cm}
\maketitle
{\let\newpage\relax\maketitle}
\textbf{Question: }
Construct a tangent to a circle of radius $4cm$ from a point on the concentric circle of radius $6cm$ and measure its length. Also verify the measurement by actual calculation.
\vspace{2mm}


\textbf{Solution:}
Consider two concentric circles of radii $4cm$ and $6cm$, respectively. Let the center be $\vec{O}$.
\begin{align}
    \vec{C_1}: \ \norm{\vec{x} - \vec{O}} = 4 \\
    \vec{C_2}: \ \norm{\vec{x} - \vec{O}} = 6 
\end{align}
Let $\vec{P}$ be a point on the $\vec{C_2}$. From point $\vec{P}$ a tangent is drawn to the $\vec{C_1}$ that intersects $\vec{C_1}$ at $\vec{T}$
\begin{align}
    \brak{\vec{P} - \vec{T}}^{\top} \brak{\vec{T} - \vec{O}} = 0 \ \ \ \brak{\because \text{$\vec{P}-\vec{T}$ is a tangent to $C_1$}}
\end{align}
Thus, $\triangle \vec{P}\vec{T}\vec{O}$ is a right-angled triangle. Using Pythagorean theorem, 
\begin{align}
    \norm{\vec{P} - \vec{T}}^2 + \norm{\vec{T} - \vec{O}}^2 = \norm{\vec{P} - \vec{O}}^2 \\ 
    \norm{\vec{T} - \vec{O}}=4 \ ,  \ \norm{\vec{P} - \vec{O}}=6\\
    \norm{\vec{P} - \vec{T}}^2 = 36-16 = 20 \implies \norm{\vec{P} - \vec{T}}=2\sqrt{5}
\end{align}
Thus, the length of the tangent is $2\sqrt{5} \ cm $

Let us show this in graph using center $\vec{O}= \myvec{0\\0}$
\begin{figure}
    \centering
    \includegraphics[width=1\columnwidth]{figs/1.png}
    \caption{Graph for 10.5.5}
    \label{fig:placeholder}
\end{figure}
\end{document}


