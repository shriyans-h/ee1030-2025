    \documentclass{beamer}
    \usepackage{siunitx}
    \usepackage{tfrupee}
    \let\vec\mathbf
    \mode<presentation>
    \usepackage{amsmath}
    \usepackage{amssymb}
    %\usepackage{advdate}
    \usepackage{adjustbox}
    %\usepackage{subcaption}
    \usepackage{enumitem}
    \usepackage{multicol}
    \usepackage{mathtools}
    \usepackage{listings}
    \usepackage{url}
    \usetheme{Boadilla}
    \usecolortheme{lily}
    \setbeamertemplate{footline}
    {
      \leavevmode%
      \hbox{%
      \begin{beamercolorbox}[wd=\paperwidth,ht=2.25ex,dp=1ex,right]{author in head/foot}%
        \insertframenumber{} / \inserttotalframenumber\hspace*{2ex} 
      \end{beamercolorbox}}%
      \vskip0pt%
    }
    \setbeamertemplate{navigation symbols}{}
    \providecommand{\nCr}[2]{\,^{#1}C_{#2}} % nCr
    \providecommand{\nPr}[2]{\,^{#1}P_{#2}} % nPr
    \providecommand{\mbf}{\mathbf}
    \providecommand{\pr}[1]{\ensuremath{\Pr\left(#1\right)}}
    \providecommand{\qfunc}[1]{\ensuremath{Q\left(#1\right)}}
    \providecommand{\sbrak}[1]{\ensuremath{{}\left[#1\right]}}
    \providecommand{\lsbrak}[1]{\ensuremath{{}\left[#1\right.}}
    \providecommand{\rsbrak}[1]{\ensuremath{{}\left.#1\right]}}
    \providecommand{\brak}[1]{\ensuremath{\left(#1\right)}}
    \providecommand{\lbrak}[1]{\ensuremath{\left(#1\right.}}
    \providecommand{\rbrak}[1]{\ensuremath{\left.#1\right)}}
    \providecommand{\cbrak}[1]{\ensuremath{\left\{#1\right\}}}
    \providecommand{\lcbrak}[1]{\ensuremath{\left\{#1\right.}}
    \providecommand{\rcbrak}[1]{\ensuremath{\left.#1\right\}}}
    \theoremstyle{remark}
    \newtheorem{rem}{Remark}
    \newcommand{\sgn}{\mathop{\mathrm{sgn}}}
    
    \providecommand{\res}[1]{\Res\displaylimits_{#1}} 
    \providecommand{\norm}[1]{\left\lVert#1\right\rVert}
    \providecommand{\mtx}[1]{\mathbf{#1}}
    \providecommand{\abs}[1]{\left\vert#1\right\vert}
    \providecommand{\fourier}{\overset{\mathcal{F}}{ \rightleftharpoons}}
    %\providecommand{\hilbert}{\overset{\mathcal{H}}{ \rightleftharpoons}}
    \providecommand{\system}{\overset{\mathcal{H}}{ \longleftrightarrow}}
    	%\newcommand{\solution}[2]{\textbf{Solution:}{#1}}
    %\newcommand{\solution}{\noindent \textbf{Solution: }}align
    \providecommand{\dec}[2]{\ensuremath{\overset{#1}{\underset{#2}{\gtrless}}}}
    \newcommand{\myvec}[1]{\ensuremath{\begin{pmatrix}#1\end{pmatrix}}}
    
    \title{Matrices in Geometry - 10.5.5}
    \author{EE25BTECH11037  Divyansh}
    \date{Sept, 2025}
    
    \begin{document}
    
    \maketitle
    
    
    \section{Problem}
    \begin{frame}
    \frametitle{Problem Statement}
    Construct a tangent to a circle of radius $4cm$ from a point on the concentric circle of radius $6cm$ and measure its length. Also verify the measurement by actual calculation.
    \end{frame}
    
    \section{Solution}
    \begin{frame}{Solution}
    Consider two concentric circles of radii $4cm$ and $6cm$, respectively. Let the center be $\vec{O}$.
    \begin{align}
        \vec{C_1}: \ \norm{\vec{x} - \vec{O}} = 4 \\
        \vec{C_2}: \ \norm{\vec{x} - \vec{O}} = 6 
    \end{align}
    Let $\vec{P}$ be a point on the $\vec{C_2}$. From point $\vec{P}$ a tangent is drawn to the $\vec{C_1}$ that intersects $\vec{C_1}$ at $\vec{T}$
    \begin{align}
        \brak{\vec{P} - \vec{T}}^{\top} \brak{\vec{T} - \vec{O}} = 0 \ \ \ \brak{\because \text{$\vec{P}-\vec{T}$ is a tangent to $C_1$}}
    \end{align}
    \end{frame}
    
    \begin{frame}{Solution}
    Thus, $\triangle \vec{P}\vec{T}\vec{O}$ is a right-angled triangle. Using Pythagorean theorem, 
    \begin{align}
        \norm{\vec{P} - \vec{T}}^2 + \norm{\vec{T} - \vec{O}}^2 = \norm{\vec{P} - \vec{O}}^2 \\ 
        \norm{\vec{T} - \vec{O}}=4 \ ,  \ \norm{\vec{P} - \vec{O}}=6\\
        \norm{\vec{P} - \vec{T}}^2 = 36-16 = 20 \implies \norm{\vec{P} - \vec{T}}=2\sqrt{5}
    \end{align}
    Thus, the length of the tangent is $2\sqrt{5} \ cm $
    \end{frame}
    
    \begin{frame}{Solution}
    Let us show this in graph using center $\vec{O}= \myvec{0\\0}$
    \begin{figure}
    \centering
    \includegraphics[width=0.5\columnwidth]{figs/1.png}
    \caption{Graph for 10.5.5}
    \label{fig:placeholder}
\end{figure}
    \end{frame}
    
    
    \end{document}
