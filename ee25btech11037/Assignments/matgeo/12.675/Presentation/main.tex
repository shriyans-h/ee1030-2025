\documentclass{beamer}
\usepackage{siunitx}
\usepackage{tfrupee}
\renewcommand{\vec}[1]{\mathbf{#1}}
\mode<presentation>
\usepackage{amsmath}
\usepackage{amssymb}
%\usepackage{advdate}
\usepackage{adjustbox}
%\usepackage{subcaption}
\usepackage{multicol}
\usepackage{mathtools}
\usepackage{listings}
\usepackage{url}
\usetheme{Boadilla}
\usecolortheme{lily}
\setbeamertemplate{footline}
{
  \leavevmode%
  \hbox{%
  \begin{beamercolorbox}[wd=\paperwidth,ht=2.25ex,dp=1ex,right]{author in head/foot}%
    \insertframenumber{} / \inserttotalframenumber\hspace*{2ex} 
  \end{beamercolorbox}}%
  \vskip0pt%
}
\setbeamertemplate{navigation symbols}{}
\providecommand{\nCr}[2]{\,^{#1}C_{#2}} % nCr
\providecommand{\nPr}[2]{\,^{#1}P_{#2}} % nPr
\providecommand{\mbf}{\mathbf}
\providecommand{\pr}[1]{\ensuremath{\Pr\left(#1\right)}}
\providecommand{\qfunc}[1]{\ensuremath{Q\left(#1\right)}}
\providecommand{\sbrak}[1]{\ensuremath{{}\left[#1\right]}}
\providecommand{\lsbrak}[1]{\ensuremath{{}\left[#1\right.}}
\providecommand{\rsbrak}[1]{\ensuremath{{}\left.#1\right]}}
\providecommand{\brak}[1]{\ensuremath{\left(#1\right)}}
\providecommand{\lbrak}[1]{\ensuremath{\left(#1\right.}}
\providecommand{\rbrak}[1]{\ensuremath{\left.#1\right)}}
\providecommand{\cbrak}[1]{\ensuremath{\left\{#1\right\}}}
\providecommand{\lcbrak}[1]{\ensuremath{\left\{#1\right.}}
\providecommand{\rcbrak}[1]{\ensuremath{\left.#1\right\}}}
\theoremstyle{remark}
\newtheorem{rem}{Remark}
\newcommand{\sgn}{\mathop{\mathrm{sgn}}}
\usepackage{enumitem}
\providecommand{\res}[1]{\Res\displaylimits_{#1}} 
\providecommand{\norm}[1]{\left\lVert#1\right\rVert}
\providecommand{\mtx}[1]{\mathbf{#1}}
\providecommand{\abs}[1]{\left\vert#1\right\vert}
\providecommand{\fourier}{\overset{\mathcal{F}}{ \rightleftharpoons}}
%\providecommand{\hilbert}{\overset{\mathcal{H}}{ \rightleftharpoons}}
\providecommand{\system}{\overset{\mathcal{H}}{ \longleftrightarrow}}
	%\newcommand{\solution}[2]{\textbf{Solution:}{#1}}
%\newcommand{\solution}{\noindent \textbf{Solution: }}align
\providecommand{\dec}[2]{\ensuremath{\overset{#1}{\underset{#2}{\gtrless}}}}
\newcommand{\myvec}[1]{\ensuremath{\begin{pmatrix}#1\end{pmatrix}}}

\title{Matrices in Geometry - 12.675}
\author{EE25BTECH11037  Divyansh}
\date{Sept, 2025}

\begin{document}

\maketitle


\section{Problem}
\begin{frame}
\frametitle{Problem Statement}
The ratio of the product of eigenvalues to the sum of the eigenvalues of the given matrix 
\begin{align*}
    \myvec{3&1&2\\2&-3&-1\\1&2&1}
\end{align*}
\end{frame}

\section{Solution}
\begin{frame}{Solution}
Let 
\begin{align}
    \vec{A}=\myvec{3&1&2\\2&-3&-1\\1&2&1}
\end{align}
The eigenvalues are the values of $\lambda$ that satisfy $\abs{ \vec{A} - \lambda\vec{I}} = 0$
\begin{align}
    \implies \abs{\myvec{3-\lambda&1&2\\2&-3-\lambda&-1\\1&2&1-\lambda}}=0\\
     \brak{3-\lambda}\brak{ \brak{-3-\lambda}\brak{1-\lambda} +2} - 1\brak{2-2\lambda + 1} + 2\brak{4 + 3 + \lambda}=0
     \end{align}
\end{frame}
\begin{frame}{Solution}
  \begin{align}
    \implies \lambda^3 - \lambda^2 - 11\lambda - 8 = 0
\end{align}
Let the eigenvalues be $\lambda_1, \lambda_2  , \lambda_3$, then
\begin{align}
    \lambda_1+ \lambda_2  + \lambda_3 = -\frac{-1}{1} =1\\
    \lambda_1 \lambda_2   \lambda_3 = -\frac{-8}{1}
\end{align}
Thus the ratio of product of eigenvalues to sum of eigenvalues of $\vec{A}$ is $r$
\begin{align}
    r=\frac{8}{1} = 8
\end{align}  
\end{frame}


\end{document}
