\documentclass[journal,12pt,onecolumn]{IEEEtran}
\usepackage{cite}
\usepackage{caption}
\usepackage{graphicx}
\usepackage{amsmath,amssymb,amsfonts,amsthm}
\usepackage{algorithmic}
\usepackage{graphicx}
\usepackage{textcomp}
\usepackage{xcolor}
\usepackage{txfonts}
\usepackage{listings}
\usepackage{enumitem}
\usepackage{mathtools}
\usepackage{gensymb}
\usepackage{comment}
\usepackage[breaklinks=true]{hyperref}
\usepackage{tkz-euclide} 
\usepackage{listings}
\usepackage{gvv}
%\def\inputGnumericTable{}
\usepackage[latin1]{inputenc} 
\usetikzlibrary{arrows.meta, positioning}
\usepackage{xparse}
\usepackage{color}                                            
\usepackage{array}                                            
\usepackage{longtable}                                       
\usepackage{calc}                                             
\usepackage{multirow}
\usepackage{multicol}
\usepackage{hhline}                                           
\usepackage{ifthen}                                           
\usepackage{lscape}
\usepackage{tabularx}
\usepackage{array}
\usepackage{float}

\usepackage{float}
%\newcommand{\define}{\stackrel{\triangle}{=}}
\theoremstyle{remark}
\usepackage{circuitikz}
\captionsetup{justification=centering}
\usepackage{tikz}

\title{Matrices in Geometry 5.4.32}
\author{EE25BTECH11037 - Divyansh}
\begin{document}
\vspace{3cm}
\maketitle
{\let\newpage\relax\maketitle}
\textbf{Question: }
Find inverse of the matrix 
\begin{align*}
    \vec{A}=\myvec{1 & 0 & 1 \\ 0 & 1 & 2 \\ 0 & 0 & 4}
\end{align*}

\vspace{2mm}

\textbf{Solution:}
\\
Let $\vec{B}$ be the inverse of $\vec{A}$, then
\begin{align}
    \vec{A}\vec{B}=\vec{I}
\end{align}
forming the augmented matrix,
\begin{align}
    \myvec{1 & 0 & 1 & \vrule &1 & 0 & 0\\ 0 & 1 & 2 & \vrule & 0 & 1 & 0\\ 0 & 0 & 4 &\vrule & 0 & 0&1}\overset{R_3 \rightarrow R_3 /4 }{\longleftrightarrow} 
    \myvec{1 & 0 & 1 & \vrule &1 & 0 & 0\\ 0 & 1 & 2 & \vrule & 0 & 1 & 0\\ 0 & 0 & 1 &\vrule & 0 & 0&\frac{1}{4}}\overset{R_2 \rightarrow R_2 - 2R_3 }{\longleftrightarrow}\\
    \myvec{1 & 0 & 1 & \vrule &1 & 0 & 0\\ 0 & 1 & 0 & \vrule & 0 & 1 & \frac{-1}{2}\\ 0 & 0 & 1 &\vrule & 0 & 0&\frac{1}{4}}
    \overset{R_1 \rightarrow R_1 - R_3 }{\longleftrightarrow}
    \myvec{1 & 0 & 0 & \vrule &1 & 0 & \frac{-1}{4}\\ 0 & 1 & 0 & \vrule & 0 & 1 & \frac{-1}{2}\\ 0 & 0 & 1 &\vrule & 0 & 0&\frac{1}{4}}
\end{align}

Thus, 
\begin{align}
    \vec{B}=\vec{A}^{-1} =\myvec{1 & 0 & \frac{-1}{4} \\ 1 & 0 & \frac{-1}{2} \\1 & 0 & \frac{1}{4}}
\end{align}

\end{document}
