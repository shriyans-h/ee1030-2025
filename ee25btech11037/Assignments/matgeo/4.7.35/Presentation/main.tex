\documentclass{beamer}
\let\vec\mathbf
\mode<presentation>
\usepackage{amsmath}
\usepackage{amssymb}
%\usepackage{advdate}
\usepackage{adjustbox}
%\usepackage{subcaption}
\usepackage{enumitem}
\usepackage{multicol}
\usepackage{mathtools}
\usepackage{listings}
\usepackage{url}
\usetheme{Boadilla}
\usecolortheme{lily}
\setbeamertemplate{footline}
{
  \leavevmode%
  \hbox{%
  \begin{beamercolorbox}[wd=\paperwidth,ht=2.25ex,dp=1ex,right]{author in head/foot}%
    \insertframenumber{} / \inserttotalframenumber\hspace*{2ex} 
  \end{beamercolorbox}}%
  \vskip0pt%
}
\setbeamertemplate{navigation symbols}{}
\providecommand{\nCr}[2]{\,^{#1}C_{#2}} % nCr
\providecommand{\nPr}[2]{\,^{#1}P_{#2}} % nPr
\providecommand{\mbf}{\mathbf}
\providecommand{\pr}[1]{\ensuremath{\Pr\left(#1\right)}}
\providecommand{\qfunc}[1]{\ensuremath{Q\left(#1\right)}}
\providecommand{\sbrak}[1]{\ensuremath{{}\left[#1\right]}}
\providecommand{\lsbrak}[1]{\ensuremath{{}\left[#1\right.}}
\providecommand{\rsbrak}[1]{\ensuremath{{}\left.#1\right]}}
\providecommand{\brak}[1]{\ensuremath{\left(#1\right)}}
\providecommand{\lbrak}[1]{\ensuremath{\left(#1\right.}}
\providecommand{\rbrak}[1]{\ensuremath{\left.#1\right)}}
\providecommand{\cbrak}[1]{\ensuremath{\left\{#1\right\}}}
\providecommand{\lcbrak}[1]{\ensuremath{\left\{#1\right.}}
\providecommand{\rcbrak}[1]{\ensuremath{\left.#1\right\}}}
\theoremstyle{remark}
\newtheorem{rem}{Remark}
\newcommand{\sgn}{\mathop{\mathrm{sgn}}}

\providecommand{\res}[1]{\Res\displaylimits_{#1}} 
\providecommand{\norm}[1]{\left\lVert#1\right\rVert}
\providecommand{\mtx}[1]{\mathbf{#1}}
\providecommand{\abs}[1]{\left\vert#1\right\vert}
\providecommand{\fourier}{\overset{\mathcal{F}}{ \rightleftharpoons}}
%\providecommand{\hilbert}{\overset{\mathcal{H}}{ \rightleftharpoons}}
\providecommand{\system}{\overset{\mathcal{H}}{ \longleftrightarrow}}
	%\newcommand{\solution}[2]{\textbf{Solution:}{#1}}
%\newcommand{\solution}{\noindent \textbf{Solution: }}align
\providecommand{\dec}[2]{\ensuremath{\overset{#1}{\underset{#2}{\gtrless}}}}
\newcommand{\myvec}[1]{\ensuremath{\begin{pmatrix}#1\end{pmatrix}}}

\title{Matrices in Geometry - 4.7.35}
\author{EE25BTECH11037  Divyansh}
\date{Sept, 2025}

\begin{document}

\maketitle


\section{Problem}
\begin{frame}
\frametitle{Problem Statement}
If the line drawn from the point \brak{-2,-1,-3} meets a plane at right angle at the point \brak{1,-3,3}, find the equation of the plane.
\end{frame}

\section{Solution}
\begin{frame}{Solution}
   
We have two points $\vec{A}=\myvec{-2\\ -1 \\ -3}$ and $\vec{B}=\myvec{1\\  -3\\ 3}$\\
We have to find the equation for the plane that passes through $\vec{B}$ and is perpendicular to the line that joins $\vec{A}$ and $\vec{B}$.\\
For that we first need the normal vector $\vec{n}$ to this plane, which will be:
\begin{align}
    \vec{n}=\vec{Q}-\vec{P} \implies \vec{n} =\myvec{3 \\ -2 \\ 6}
\end{align}
\end{frame}

\begin{frame}{Solution}
Therefore, the equation of this plane is given by 
\begin{align}
    \vec{n}^{\top}\vec{x}=d
\end{align}
Since the point $\vec{B}$ lies on this plane, it should satisfy this equation.
\begin{align}
    \vec{n}^{\top}\vec{B}=d \implies \myvec{3 & -2 & 6}\myvec{1 \\ -3 \\ 3}=d \implies d=27
\end{align}
Therefore, the equation of this plane is 
\begin{align}
    \vec{n}^{\top}\vec{x}=27 \implies \myvec{3 & -2 & 6}\myvec{x \\ y \\ z}=27 \implies 3x - 2y +6z=27
\end{align}
\end{frame}
\begin{frame}{Solution}
\begin{figure}[H]
    \centering
    \includegraphics[width=0.8\columnwidth]{figs/1.png}
    \caption{Figure for 4.7.35}
    \label{fig:placeholder}
\end{figure}
\end{frame}


\end{document}
