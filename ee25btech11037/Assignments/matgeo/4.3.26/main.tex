\documentclass[journal,12pt,onecolumn]{IEEEtran}
\usepackage{cite}
\usepackage{caption}
\usepackage{graphicx}
\usepackage{amsmath,amssymb,amsfonts,amsthm}
\usepackage{algorithmic}
\usepackage{graphicx}
\usepackage{textcomp}
\usepackage{xcolor}
\usepackage{txfonts}
\usepackage{listings}
\usepackage{enumitem}
\usepackage{mathtools}
\usepackage{gensymb}
\usepackage{comment}
\usepackage[breaklinks=true]{hyperref}
\usepackage{tkz-euclide} 
\usepackage{listings}
\usepackage{gvv}
%\def\inputGnumericTable{}
\usepackage[latin1]{inputenc} 
\usetikzlibrary{arrows.meta, positioning}
\usepackage{xparse}
\usepackage{color}                                            
\usepackage{array}                                            
\usepackage{longtable}                                       
\usepackage{calc}                                             
\usepackage{multirow}
\usepackage{multicol}
\usepackage{hhline}                                           
\usepackage{ifthen}                                           
\usepackage{lscape}
\usepackage{tabularx}
\usepackage{array}
\usepackage{float}

\usepackage{float}
%\newcommand{\define}{\stackrel{\triangle}{=}}
\theoremstyle{remark}
\usepackage{circuitikz}
\captionsetup{justification=centering}
\usepackage{tikz}

\title{Matrices in Geometry 4.3.26}
\author{EE25BTECH11037 - Divyansh}
\begin{document}
\vspace{3cm}
\maketitle
{\let\newpage\relax\maketitle}
\textbf{Question: }
Find the ratio in which the line segment joining the points $\vec{A}=\brak{4,8,10}$ and $\vec{B}=\brak{6, 10, -8}$ is divided bv the YZ plane.
\vspace{2mm}

\textbf{Solution:}
 \vspace{1mm}
We have two points $\vec{A}=\myvec{4\\ 8 \\  10}$ and $\vec{B}=\myvec{6\\  10\\ -8}$\\
Let $\vec{P}$ be the point on the Y-Z plane. Since it is collinear to $\vec{A}$ and $\vec{B}$, \\
Since $\vec{P}$ lies on Y-Z plane, $\vec{P}=\myvec{0 \\ P_y \\ P_z }$.\\ 
From $\brak{4.1.2.5}$
\begin{align}
    \myvec{\vec{B} & \vec{A} & \vec{P}}^{\top} \vec{n}=\myvec{1 \\ 1 \\1}\\
    \myvec{6 &10&-8 \\ 4 & 8& 10 \\ 0& P_y & P_z}\vec{n}=\myvec{1 \\ 1 \\1} \implies \myvec{6&10&-8&\vrule&1\\4&8&10&\vrule&1\\0&P_y&P_z&\vrule&1}
    \overset{R_1 \rightarrow \frac{\brak{R_1 - R_2}}{2}}{\longrightarrow} \myvec{1&1&-9&\vrule&0\\4&8&10&\vrule&1\\0&P_y&P_z&\vrule&1}\\
    \overset{R_2 \rightarrow R_2 - 4R_1}{\longrightarrow} \myvec{1&1&-9&\vrule&0\\0&4&46&\vrule&1\\0&P_y&P_z&\vrule&1}\overset{R_3 \rightarrow R_3 - R_2}{\longrightarrow} \myvec{1&1&-9&\vrule&0\\0&4&46&\vrule&1\\0&P_y-4&P_z-46&\vrule&0}
\end{align}
Since $\vec{P}$, $\vec{A}$ and $\vec{B}$ are collinear, the rank of this matrix must be less than or equal to 2. Therefore, the third row should be a zero row, and therefore, $\vec{P}=\myvec{0 \\ 4 \\ 46}$\\
Let $\vec{P}$ divide $\vec{A}$ and $\vec{B}$ in the ratio $k:1$\\
Using the formula \brak{1.1.5.2}
\begin{align}
    k=\dfrac{\brak{\vec{A}-\vec{P}}^{\top} \brak{\vec{P}-\vec{B}}}{\norm{\vec{P}-\vec{B}}^2}\\
    \implies k=\dfrac{\myvec{4 & 4 & -36} \myvec{-6 \\ -6 \\ 54}}{\norm{\myvec{-6 \\ -6 \\ 54}}}\\
    \implies k= \dfrac{-24 -24 -1944}{36 + 36 + 2916} = \dfrac{-1992}{2988}\\
    \implies k= \frac{-2}{3}
\end{align}
Hence, the Y-Z plane divides the line segment that joins the points $\vec{A}$ and $\vec{B}$ in the external ratio $2:3$.
\begin{figure}[H]
        \centering
        \includegraphics[width=1\columnwidth]{figs/1.png}
        \caption{Graph for 4.3.26}
        \label{fig:placeholder}
    \end{figure}
\end{document}