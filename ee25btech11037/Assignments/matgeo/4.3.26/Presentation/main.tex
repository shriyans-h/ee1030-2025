\documentclass{beamer}
\let\vec\mathbf
\mode<presentation>
\usepackage{amsmath}
\usepackage{amssymb}
%\usepackage{advdate}
\usepackage{adjustbox}
%\usepackage{subcaption}
\usepackage{enumitem}
\usepackage{multicol}
\usepackage{mathtools}
\usepackage{listings}
\usepackage{url}
\usetheme{Boadilla}
\usecolortheme{lily}
\setbeamertemplate{footline}
{
  \leavevmode%
  \hbox{%
  \begin{beamercolorbox}[wd=\paperwidth,ht=2.25ex,dp=1ex,right]{author in head/foot}%
    \insertframenumber{} / \inserttotalframenumber\hspace*{2ex} 
  \end{beamercolorbox}}%
  \vskip0pt%
}
\setbeamertemplate{navigation symbols}{}
\providecommand{\nCr}[2]{\,^{#1}C_{#2}} % nCr
\providecommand{\nPr}[2]{\,^{#1}P_{#2}} % nPr
\providecommand{\mbf}{\mathbf}
\providecommand{\pr}[1]{\ensuremath{\Pr\left(#1\right)}}
\providecommand{\qfunc}[1]{\ensuremath{Q\left(#1\right)}}
\providecommand{\sbrak}[1]{\ensuremath{{}\left[#1\right]}}
\providecommand{\lsbrak}[1]{\ensuremath{{}\left[#1\right.}}
\providecommand{\rsbrak}[1]{\ensuremath{{}\left.#1\right]}}
\providecommand{\brak}[1]{\ensuremath{\left(#1\right)}}
\providecommand{\lbrak}[1]{\ensuremath{\left(#1\right.}}
\providecommand{\rbrak}[1]{\ensuremath{\left.#1\right)}}
\providecommand{\cbrak}[1]{\ensuremath{\left\{#1\right\}}}
\providecommand{\lcbrak}[1]{\ensuremath{\left\{#1\right.}}
\providecommand{\rcbrak}[1]{\ensuremath{\left.#1\right\}}}
\theoremstyle{remark}
\newtheorem{rem}{Remark}
\newcommand{\sgn}{\mathop{\mathrm{sgn}}}

\providecommand{\res}[1]{\Res\displaylimits_{#1}} 
\providecommand{\norm}[1]{\left\lVert#1\right\rVert}
\providecommand{\mtx}[1]{\mathbf{#1}}
\providecommand{\abs}[1]{\left\vert#1\right\vert}
\providecommand{\fourier}{\overset{\mathcal{F}}{ \rightleftharpoons}}
%\providecommand{\hilbert}{\overset{\mathcal{H}}{ \rightleftharpoons}}
\providecommand{\system}{\overset{\mathcal{H}}{ \longleftrightarrow}}
	%\newcommand{\solution}[2]{\textbf{Solution:}{#1}}
%\newcommand{\solution}{\noindent \textbf{Solution: }}align
\providecommand{\dec}[2]{\ensuremath{\overset{#1}{\underset{#2}{\gtrless}}}}
\newcommand{\myvec}[1]{\ensuremath{\begin{pmatrix}#1\end{pmatrix}}}

\title{Matrices in Geometry - 4.3.26}
\author{EE25BTECH11037  Divyansh}
\date{Sept, 2025}

\begin{document}

\maketitle


\section{Problem}
\begin{frame}
\frametitle{Problem Statement}
Find the ratio in which the line segment joining the points $\vec{A}=\brak{4,8,10}$ and $\vec{B}=\brak{6, 10, -8}$ is divided bv the YZ plane.
\end{frame}

\section{Solution}
\begin{frame}{Solution}
   
We have two points $\vec{A}=\myvec{4\\ 8 \\  10}$ and $\vec{B}=\myvec{6\\  10\\ -8}$\\
Let $\vec{P}$ be the point on the Y-Z plane. Since it is collinear to $\vec{A}$ and $\vec{B}$, \\
Since $\vec{P}$ lies on Y-Z plane, $\vec{P}=\myvec{0 \\ P_y \\ P_z }$.\\ 
From $\brak{4.1.2.5}$
\begin{align}
    \myvec{\vec{B} & \vec{A} & \vec{P}}^{\top} \vec{n}=\myvec{1 \\ 1 \\1}\\
    \myvec{6 &10&-8 \\ 4 & 8& 10 \\ 0& P_y & P_z}\vec{n}=\myvec{1 \\ 1 \\1} \implies \myvec{6&10&-8&\vrule&1\\4&8&10&\vrule&1\\0&P_y&P_z&\vrule&1}
\end{align}
\end{frame}

\begin{frame}{Solution}
\begin{align}
    \overset{R_1 \rightarrow \frac{\brak{R_1 - R_2}}{2}}{\longrightarrow} \myvec{1&1&-9&\vrule&0\\4&8&10&\vrule&1\\0&P_y&P_z&\vrule&1}
    \overset{R_2 \rightarrow R_2 - 4R_1}{\longrightarrow} \\ \myvec{1&1&-9&\vrule&0\\0&4&46&\vrule&1\\0&P_y&P_z&\vrule&1}\overset{R_3 \rightarrow R_3 - R_2}{\longrightarrow} \myvec{1&1&-9&\vrule&0\\0&4&46&\vrule&1\\0&P_y-4&P_z-46&\vrule&0}
\end{align}
Since $\vec{P}$, $\vec{A}$ and $\vec{B}$ are collinear, the rank of this matrix must be less than or equal to 2.. Therefore, the third row should be a zero row and therefore, $\vec{P}=\myvec{0 \\ 4 \\ 46}$
\end{frame}
\begin{frame}{Solution}
Let $\vec{P}$ divide $\vec{A}$ and $\vec{B}$ in the ratio $k:1$\\
Using the formula \brak{1.1.5.2}
\begin{align}
    k=\dfrac{\brak{\vec{A}-\vec{P}}^{\top} \brak{\vec{P}-\vec{B}}}{\norm{\vec{P}-\vec{B}}^2}\\
    \implies k=\dfrac{\myvec{4 & 4 & -36} \myvec{-6 \\ -6 \\ 54}}{\norm{\myvec{-6 \\ -6 \\ 54}}}\\
    \implies k= \dfrac{-24 -24 -1944}{36 + 36 + 2916} = \dfrac{-1992}{2988}\\
    \implies k= \frac{-2}{3}
\end{align}
Hence, the Y-Z plane divides the line segment that joins the points $\vec{A}$ and $\vec{B}$ in the external ratio $2:3$.
\end{frame}
\begin{frame}{Solution}
    \begin{figure}[H]
        \centering
        \includegraphics[width=0.9\columnwidth]{figs/1.png}
        \caption{Graph for 4.3.26}
        \label{fig:placeholder}
    \end{figure}
\end{frame}

\end{document}