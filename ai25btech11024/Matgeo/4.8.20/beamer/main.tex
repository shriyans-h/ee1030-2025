\documentclass{beamer}
\mode<presentation>
\usepackage{amsmath}
\usepackage{amssymb}
%\usepackage{advdate}
\usepackage{graphicx}
\usepackage{adjustbox}
\usepackage{subcaption}
\usepackage{enumitem}
\usepackage{multicol}
\usepackage{mathtools}
\usepackage{listings}
\usepackage{url}
\def\UrlBreaks{\do\/\do-}
\usetheme{Boadilla}
\usecolortheme{lily}
\setbeamertemplate{footline}
{
  \leavevmode%
  \hbox{%
  \begin{beamercolorbox}[wd=\paperwidth,ht=2.25ex,dp=1ex,right]{author in head/foot}%
    \insertframenumber{} / \inserttotalframenumber\hspace*{2ex} 
  \end{beamercolorbox}}%
  \vskip0pt%
}
\setbeamertemplate{navigation symbols}{}

\providecommand{\nCr}[2]{\,^{#1}C_{#2}} % nCr
\providecommand{\nPr}[2]{\,^{#1}P_{#2}} % nPr
\providecommand{\mbf}{\mathbf}
\providecommand{\pr}[1]{\ensuremath{\Pr\left(#1\right)}}
\providecommand{\qfunc}[1]{\ensuremath{Q\left(#1\right)}}
\providecommand{\sbrak}[1]{\ensuremath{{}\left[#1\right]}}
\providecommand{\lsbrak}[1]{\ensuremath{{}\left[#1\right.}}
\providecommand{\rsbrak}[1]{\ensuremath{{}\left.#1\right]}}
\providecommand{\brak}[1]{\ensuremath{\left(#1\right)}}
\providecommand{\lbrak}[1]{\ensuremath{\left(#1\right.}}
\providecommand{\rbrak}[1]{\ensuremath{\left.#1\right)}}
\providecommand{\cbrak}[1]{\ensuremath{\left\{#1\right\}}}
\providecommand{\lcbrak}[1]{\ensuremath{\left\{#1\right.}}
\providecommand{\rcbrak}[1]{\ensuremath{\left.#1\right\}}}
\theoremstyle{remark}
\newtheorem{rem}{Remark}
\newcommand{\sgn}{\mathop{\mathrm{sgn}}}
\providecommand{\abs}[1]{$\left\vert#1\right\vert$}
\providecommand{\res}[1]{\Res\displaylimits_{#1}} 
\providecommand{\norm}[1]{\lVert#1\rVert}
\providecommand{\mtx}[1]{\mathbf{#1}}
\providecommand{\mean}[1]{E$\left[ #1 \right]$}
\providecommand{\fourier}{\overset{\mathcal{F}}{ \rightleftharpoons}}
%\providecommand{\hilbert}{\overset{\mathcal{H}}{ \rightleftharpoons}}
\providecommand{\system}[1]{\overset{\mathcal{#1}}{ \longleftrightarrow}}
%\providecommand{\system}{\overset{\mathcal{H}}{ \longleftrightarrow}}
	%\newcommand{\solution}[2]{\textbf{Solution:}{#1}}
%\newcommand{\solution}{\noindent \textbf{Solution: }}
\providecommand{\dec}[2]{\ensuremath{\overset{#1}{\underset{#2}{\gtrless}}}}
\newcommand{\myvec}[1]{\ensuremath{\begin{pmatrix}#1\end{pmatrix}}}
\let\vec\mathbf

\lstset{
%language=C,
frame=single, 
breaklines=true,
columns=fullflexible
}

\numberwithin{equation}{section}

\title{4.4.20}
\author{AI25BTECH11024 - Pratyush Panda}
\begin{document}
\maketitle

\begin{frame}
\textbf{Question: } \\
Find the distance between the point $\brak{2,3,4}$ measured along the line $\frac{x-4}{3}=\frac{y+5}{6}=\frac{z+1}{2}$ from the plane $3x+2y+2z+5=0$
\end{frame}

\begin{frame}
\textbf{Solution: } \\
Let the vector $\Vec{A}$ be $\myvec{2 \\ 3 \\ 4}$, and the direction vector of the line $\Vec{b}=\myvec{3 \\ 6 \\ 2}$. \\
The equation of the plane can be written as;
\begin{align}
\Vec{n}^T\Vec{X}=1 \hspace{1cm} where, \, \Vec{n}=\myvec{3 \\ 2 \\ 2}
\end{align}

The perpendicular distance between the point and the plane $\brak{x}$ can be written as;
\begin{align}
x=\frac{\Vec{n}^T\Vec{A}}{||\Vec{n}||}=\frac{20}{\sqrt{17}}
\end{align}
\end{frame}

\begin{frame}
Now, the distance along the given line can be written as $\frac{x}{\cos\theta}$. \\
Where $\cos\theta$ is the angle between $\Vec{b}$ \brak{\text{direction vector of the line}} and $\Vec{n}$ \brak{\text{normal vector of the plane}}. \\

Thus $\cos\theta$ can be written as;
\begin{align}
\cos\theta = \frac{\Vec{n}^T\Vec{b}}{||\Vec{n}||.||\Vec{b}||}=\frac{25}{7.\sqrt{17}}
\end{align}

Thus, the final distance along the line can be written as;
\begin{align}
d=||\Vec{b}||.\frac{\Vec{n}^T\Vec{A}}{\Vec{n}^T\Vec{b}}=\frac{28}{5}
\end{align}

Thus, the distance between the point $\brak{2,3,4}$ measured along the line $\frac{x-4}{3}=\frac{y+5}{6}=\frac{z+1}{2}$ from the plane $3x+2y+2z+5=0$ is $5.6$
\end{frame}

\begin{frame}
\begin{figure}[H]
\centering
\includegraphics[width=0.8\columnwidth]{figs/img.png}
\caption*{}
\end{figure}
\end{frame}

\end{document}
