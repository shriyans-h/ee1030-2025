\let\negmedspace\undefined
\let\negthickspace\undefined
\documentclass[journal]{IEEEtran}
\usepackage[a5paper, margin=10mm, onecolumn]{geometry}
%\usepackage{lmodern} % Ensure lmodern is loaded for pdflatex
\usepackage{tfrupee} % Include tfrupee package

\setlength{\headheight}{1cm} % Set the height of the header box
\setlength{\headsep}{0mm}     % Set the distance between the header box and the top of the text

\usepackage{gvv-book}
\usepackage{gvv}
\usepackage{cite}
\usepackage{amsmath,amssymb,amsfonts,amsthm}
\usepackage{algorithmic}
\usepackage{graphicx}
\usepackage{textcomp}
\usepackage{xcolor}
\usepackage{txfonts}
\usepackage{listings}
\usepackage{enumitem}
\usepackage{mathtools}
\usepackage{gensymb}
\usepackage{comment}
\usepackage[breaklinks=true]{hyperref}
\usepackage{tkz-euclide} 
\usepackage{listings}
% \usepackage{gvv}                               

\def\inputGnumericTable{}                      
\usepackage[latin1]{inputenc}                    
\usepackage{color}                              
\usepackage{array}                             
\usepackage{longtable}                          
\usepackage{calc}                               
\usepackage{multirow}                           
\usepackage{hhline}                            
\usepackage{ifthen}                          
\usepackage{lscape}
\begin{document}

\bibliographystyle{IEEEtran}
\vspace{3cm}

\title{4.8.20}
\author{AI25BTECH11024 - Pratyush Panda
}
\maketitle
% \newpage
% \bigskip
{\let\newpage\relax\maketitle}

\renewcommand{\thefigure}{\theenumi}
\renewcommand{\thetable}{\theenumi}
\setlength{\intextsep}{10pt} % Space between text and floats


\numberwithin{equation}{enumi}
\numberwithin{figure}{enumi}
\renewcommand{\thetable}{\theenumi}

\textbf{Question: } \\
Find the distance between the point $\brak{2,3,4}$ measured along the line $\frac{x-4}{3}=\frac{y+5}{6}=\frac{z+1}{2}$ from the plane $3x+2y+2z+5=0$
\vspace{0.7cm}

\textbf{Solution: } \\
Let the vector $\Vec{A}$ be $\myvec{2 \\ 3 \\ 4}$, and the direction vector of the line $\Vec{b}=\myvec{3 \\ 6 \\ 2}$. \\
The equation of the plane can be written as;
\begin{align}
\Vec{n}^T\Vec{X}=1 \hspace{1cm} where, \, \Vec{n}=\myvec{3 \\ 2 \\ 2}
\end{align}

The equation of the line passing through $\Vec{A}$ and with the direction vector $\Vec{b}$ is;
\begin{align}   
\Vec{x}=\Vec{A}+\lambda\Vec{b}=\myvec{2 \\ 3 \\ 4}+\lambda\myvec{3 \\ 6 \\ 2}
\end{align}

The point on the plane lying on this line can be found out by substituting the parametric point in the equation of the plane and find out the value of $\lambda$. \\

After solving for $\lambda$ we get $\lambda=-1$. Thus, the point is $\Vec{B}$ would be $\myvec{-1 \\ -3 \\ 2}$. \\

Thus, the final distance along the line can be written as;
\begin{align}
d=\Vec{A}^T.\Vec{B}=7
\end{align}

Thus, the distance between the point $\brak{2,3,4}$ measured along the line $\frac{x-4}{3}=\frac{y+5}{6}=\frac{z+1}{2}$ from the plane $3x+2y+2z+5=0$ is $7$

\begin{figure}[H]
\centering
\includegraphics[width=0.8\columnwidth]{figs/img.png}
\caption*{}
\end{figure}

\end{document}
