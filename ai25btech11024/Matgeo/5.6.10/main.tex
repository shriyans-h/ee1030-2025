\let\negmedspace\undefined
\let\negthickspace\undefined
\documentclass[journal]{IEEEtran}
\usepackage[a5paper, margin=10mm, onecolumn]{geometry}
%\usepackage{lmodern} % Ensure lmodern is loaded for pdflatex
\usepackage{tfrupee} % Include tfrupee package

\setlength{\headheight}{1cm} % Set the height of the header box
\setlength{\headsep}{0mm}     % Set the distance between the header box and the top of the text

\usepackage{gvv-book}
\usepackage{gvv}
\usepackage{cite}
\usepackage{amsmath,amssymb,amsfonts,amsthm}
\usepackage{algorithmic}
\usepackage{graphicx}
\usepackage{textcomp}
\usepackage{xcolor}
\usepackage{txfonts}
\usepackage{listings}
\usepackage{enumitem}
\usepackage{mathtools}
\usepackage{gensymb}
\usepackage{comment}
\usepackage[breaklinks=true]{hyperref}
\usepackage{tkz-euclide} 
\usepackage{listings}
% \usepackage{gvv}                               

\def\inputGnumericTable{}                      
\usepackage[latin1]{inputenc}                    
\usepackage{color}                              
\usepackage{array}                             
\usepackage{longtable}                          
\usepackage{calc}                               
\usepackage{multirow}                           
\usepackage{hhline}                            
\usepackage{ifthen}                          
\usepackage{lscape}
\begin{document}

\bibliographystyle{IEEEtran}
\vspace{3cm}

\title{5.6.10}
\author{AI25BTECH11024 - Pratyush Panda
}
\maketitle
% \newpage
% \bigskip
{\let\newpage\relax\maketitle}

\renewcommand{\thefigure}{\theenumi}
\renewcommand{\thetable}{\theenumi}
\setlength{\intextsep}{10pt} % Space between text and floats


\numberwithin{equation}{enumi}
\numberwithin{figure}{enumi}
\renewcommand{\thetable}{\theenumi}

\textbf{Question: } \\
If $\Vec{A}=\myvec{3 & 1 \\ -1 & 2}$, show that $\Vec{A}^2-5\Vec{A}+7\Vec{I}=0$.
\vspace{0.7cm}

\textbf{Solution: } \\
Given matrix $\Vec{A}=\myvec{3 & 1 \\ -1 & 2}$.\\

We can write the character equation for the matrix by,
\begin{align}
f\brak{\lambda}=|\Vec{A}-\lambda\Vec{I}| \\
f\brak{\lambda}=\vline\myvec{3-\lambda & 1 \\ -1 & 2-\lambda}\vline\\
f\brak{\lambda}=\lambda^2-5\lambda+7
\end{align}

Since we know by the Caley Hamilton theorem, the matrix itself satisfies its own characteristic equation. Thus, we get;
\begin{align}
\Vec{A}^2-5\Vec{A}+7\Vec{I}=0
\end{align}
Hence proved.

\end{document}