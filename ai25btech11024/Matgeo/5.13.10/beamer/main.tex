\documentclass{beamer}
\mode<presentation>
\usepackage{amsmath}
\usepackage{amssymb}
%\usepackage{advdate}
\usepackage{graphicx}
\usepackage{adjustbox}
\usepackage{subcaption}
\usepackage{enumitem}
\usepackage{multicol}
\usepackage{mathtools}
\usepackage{listings}
\usepackage{url}
\def\UrlBreaks{\do\/\do-}
\usetheme{Boadilla}
\usecolortheme{lily}
\setbeamertemplate{footline}
{
  \leavevmode%
  \hbox{%
  \begin{beamercolorbox}[wd=\paperwidth,ht=2.25ex,dp=1ex,right]{author in head/foot}%
    \insertframenumber{} / \inserttotalframenumber\hspace*{2ex} 
  \end{beamercolorbox}}%
  \vskip0pt%
}
\setbeamertemplate{navigation symbols}{}

\providecommand{\nCr}[2]{\,^{#1}C_{#2}} % nCr
\providecommand{\nPr}[2]{\,^{#1}P_{#2}} % nPr
\providecommand{\mbf}{\mathbf}
\providecommand{\pr}[1]{\ensuremath{\Pr\left(#1\right)}}
\providecommand{\qfunc}[1]{\ensuremath{Q\left(#1\right)}}
\providecommand{\sbrak}[1]{\ensuremath{{}\left[#1\right]}}
\providecommand{\lsbrak}[1]{\ensuremath{{}\left[#1\right.}}
\providecommand{\rsbrak}[1]{\ensuremath{{}\left.#1\right]}}
\providecommand{\brak}[1]{\ensuremath{\left(#1\right)}}
\providecommand{\lbrak}[1]{\ensuremath{\left(#1\right.}}
\providecommand{\rbrak}[1]{\ensuremath{\left.#1\right)}}
\providecommand{\cbrak}[1]{\ensuremath{\left\{#1\right\}}}
\providecommand{\lcbrak}[1]{\ensuremath{\left\{#1\right.}}
\providecommand{\rcbrak}[1]{\ensuremath{\left.#1\right\}}}
\theoremstyle{remark}
\newtheorem{rem}{Remark}
\newcommand{\sgn}{\mathop{\mathrm{sgn}}}
\providecommand{\abs}[1]{$\left\vert#1\right\vert$}
\providecommand{\res}[1]{\Res\displaylimits_{#1}} 
\providecommand{\norm}[1]{\lVert#1\rVert}
\providecommand{\mtx}[1]{\mathbf{#1}}
\providecommand{\mean}[1]{E$\left[ #1 \right]$}
\providecommand{\fourier}{\overset{\mathcal{F}}{ \rightleftharpoons}}
%\providecommand{\hilbert}{\overset{\mathcal{H}}{ \rightleftharpoons}}
\providecommand{\system}[1]{\overset{\mathcal{#1}}{ \longleftrightarrow}}
%\providecommand{\system}{\overset{\mathcal{H}}{ \longleftrightarrow}}
	%\newcommand{\solution}[2]{\textbf{Solution:}{#1}}
%\newcommand{\solution}{\noindent \textbf{Solution: }}
\providecommand{\dec}[2]{\ensuremath{\overset{#1}{\underset{#2}{\gtrless}}}}
\newcommand{\myvec}[1]{\ensuremath{\begin{pmatrix}#1\end{pmatrix}}}
\let\vec\mathbf

\lstset{
%language=C,
frame=single, 
breaklines=true,
columns=fullflexible
}

\numberwithin{equation}{section}

\title{5.13.10}
\author{AI25BTECH11024 - Pratyush Panda}
\begin{document}
\maketitle

\begin{frame}
\textbf{Question: } \\
if
\begin{align}
P = \myvec{1 & \alpha & 3 \\ 1 & 3 & 3 \\ 2 & 4 & 4}
\end{align}
is the adjoint of a $3 \times 3$ matrix $\Vec{A}$ and $|\Vec{A}|=4$, then $\alpha$ is equal to
\begin{enumerate}
\item[a] 4
\item[b] 11
\item[c] 5
\item[d] 0
\end{enumerate}
\end{frame}

\begin{frame}
\textbf{Solution: } \\
Given \\
\begin{align}
P = \myvec{1 & \alpha & 3 \\ 1 & 3 & 3 \\ 2 & 4 & 4}
\end{align}
is the adjoint of a $3 \times 3$ matrix $\Vec{A}$ and $|A| = 4$.

We know that,
\begin{align}
adj(\Vec{A}) = |\Vec{A}| \Vec{A}^{-1}.
\end{align}

Hence,
\begin{align}
\Vec{P} = 4\Vec{A}^{-1} \quad \Rightarrow \quad \Vec{A} = 4\Vec{P}^{-1}.
\end{align}

Taking determinants on both sides,
\begin{align}
|\Vec{A}| = |4\Vec{P}^{-1}| = 4^3 |\Vec{P}^{-1}| = 64 \cdot \frac{1}{|\Vec{P}|}.
\end{align}
\end{frame}

\begin{frame}
Since $|\Vec{A}| = 4$,
\begin{align}
\frac{64}{\det(P)} = 4 \quad \Rightarrow \quad |\Vec{P}| = 16.
\end{align}

Now compute $|\Vec{P}|$:
\begin{align}
|\Vec{P}| = \myvec{1 & \alpha & 3 \\ 1 & 3 & 3 \\ 2 & 4 & 4}
\end{align}

Simplifying,
\begin{align}
|\Vec{P}| = 1(12 - 12) - \alpha(4 - 6) + 3(4 - 6) \\
|\Vec{P}| = 0 + 2\alpha - 6 = 2(\alpha - 3).
\end{align}

Equating this with $|\Vec{P}| = 16$,
\begin{align}
2(\alpha - 3) = 16 \quad \Rightarrow \quad \alpha - 3 = 8 \quad \Rightarrow \alpha = 11.
\end{align}
\end{frame}

\end{document}