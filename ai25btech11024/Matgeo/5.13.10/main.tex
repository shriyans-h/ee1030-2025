\let\negmedspace\undefined
\let\negthickspace\undefined
\documentclass[journal]{IEEEtran}
\usepackage[a5paper, margin=10mm, onecolumn]{geometry}
%\usepackage{lmodern} % Ensure lmodern is loaded for pdflatex
\usepackage{tfrupee} % Include tfrupee package

\setlength{\headheight}{1cm} % Set the height of the header box
\setlength{\headsep}{0mm}     % Set the distance between the header box and the top of the text

\usepackage{gvv-book}
\usepackage{gvv}
\usepackage{cite}
\usepackage{amsmath,amssymb,amsfonts,amsthm}
\usepackage{algorithmic}
\usepackage{graphicx}
\usepackage{textcomp}
\usepackage{xcolor}
\usepackage{txfonts}
\usepackage{listings}
\usepackage{enumitem}
\usepackage{mathtools}
\usepackage{gensymb}
\usepackage{comment}
\usepackage[breaklinks=true]{hyperref}
\usepackage{tkz-euclide} 
\usepackage{listings}
% \usepackage{gvv}                               

\def\inputGnumericTable{}                      
\usepackage[latin1]{inputenc}                    
\usepackage{color}                              
\usepackage{array}                             
\usepackage{longtable}                          
\usepackage{calc}                               
\usepackage{multirow}                           
\usepackage{hhline}                            
\usepackage{ifthen}                          
\usepackage{lscape}
\begin{document}

\bibliographystyle{IEEEtran}
\vspace{3cm}

\title{5.13.10}
\author{AI25BTECH11024 - Pratyush Panda
}
\maketitle
% \newpage
% \bigskip
{\let\newpage\relax\maketitle}

\renewcommand{\thefigure}{\theenumi}
\renewcommand{\thetable}{\theenumi}
\setlength{\intextsep}{10pt} % Space between text and floats


\numberwithin{equation}{enumi}
\numberwithin{figure}{enumi}
\renewcommand{\thetable}{\theenumi}

\textbf{Question: } \\
if
\begin{align}
P = \myvec{1 & \alpha & 3 \\ 1 & 3 & 3 \\ 2 & 4 & 4}
\end{align}
is the adjoint of a $3 \times 3$ matrix $\Vec{A}$ and $|\Vec{A}|=4$, then $\alpha$ is equal to
\begin{enumerate}
\item 4
\item 11
\item 5
\item 0
\end{enumerate}
\vspace{0.7cm}

\textbf{Solution: } \\
Given \\
\begin{align}
P = \myvec{1 & \alpha & 3 \\ 1 & 3 & 3 \\ 2 & 4 & 4}
\end{align}
is the adjoint of a $3 \times 3$ matrix $\Vec{A}$ and $|A| = 4$.

We know that,
\begin{align}
adj(\Vec{A}) = |\Vec{A}| \Vec{A}^{-1}.
\end{align}

Hence,
\begin{align}
\Vec{P} = 4\Vec{A}^{-1} \quad \Rightarrow \quad \Vec{A} = 4\Vec{P}^{-1}.
\end{align}

Taking determinants on both sides,
\begin{align}
|\Vec{A}| = |4\Vec{P}^{-1}| = 4^3 |\Vec{P}^{-1}| = 64 \cdot \frac{1}{|\Vec{P}|}.
\end{align}

Since $|\Vec{A}| = 4$,
\begin{align}
\frac{64}{\det(P)} = 4 \quad \Rightarrow \quad |\Vec{P}| = 16.
\end{align}

Now compute $|\Vec{P}|$:
\begin{align}
|\Vec{P}| = \myvec{1 & \alpha & 3 \\ 1 & 3 & 3 \\ 2 & 4 & 4}
\end{align}

Simplifying,
\begin{align}
|\Vec{P}| = 1(12 - 12) - \alpha(4 - 6) + 3(4 - 6) \\
|\Vec{P}| = 0 + 2\alpha - 6 = 2(\alpha - 3).
\end{align}

Equating this with $|\Vec{P}| = 16$,
\begin{align}
2(\alpha - 3) = 16 \quad \Rightarrow \quad \alpha - 3 = 8 \quad \Rightarrow \alpha = 11.
\end{align}

\end{document}