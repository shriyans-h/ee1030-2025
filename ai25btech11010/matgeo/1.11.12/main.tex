\let\negmedspace\undefined
\let\negthickspace\undefined
\documentclass[journal]{IEEEtran}
\usepackage[a5paper, margin=10mm, onecolumn]{geometry}
\usepackage{tfrupee}

\setlength{\headheight}{1cm}
\setlength{\headsep}{0mm}

\usepackage{gvv-book}
\usepackage{gvv}
\usepackage{cite}
\usepackage{amsmath,amssymb,amsfonts,amsthm}
\usepackage{algorithmic}
\usepackage{graphicx}
\usepackage{textcomp}
\usepackage{xcolor}
\usepackage{txfonts}
\usepackage{listings}
\usepackage{enumitem}
\usepackage{mathtools}
\usepackage{gensymb}
\usepackage{comment}
\usepackage[breaklinks=true]{hyperref}
\usepackage{tkz-euclide} 
\usepackage{listings}

\graphicspath{{./figs/}}

\begin{document}
\title{1.6.25}
\author{AI25BTECH11010 - Dhanush Kumar}
\maketitle
\renewcommand{\thefigure}{\theenumi}
\renewcommand{\thetable}{\theenumi}

\noindent

If the sum of two unit vectors is a unit vector, prove that the magnitude of their difference is \(\sqrt{3}\).

\bigskip
\noindent\textbf{Solution:} \\

Let
\begin{align}
\vec{u}, \vec{v} \in \mathbb{R}^n, \quad
\|\vec{u}\|=1,\;\|\vec{v}\|=1.
\end{align}

Form the matrix
\begin{align}
M = \myvec{\vec{u} & \vec{v}},
\end{align}
whose Gram matrix is
\begin{align}
G &= M^TM \\
  &= \myvec{ \vec{u}^T\vec{u} & \vec{u}^T\vec{v} \\ \vec{v}^T\vec{u} & \vec{v}^T\vec{v} } \\
  &= \myvec{1 & \rho \\ \rho & 1},
\end{align}
where \(\rho = \vec{u}^T\vec{v}\).

Now,
\begin{align}
\|\vec{u}+\vec{v}\|^2 
&= \myvec{1 & 1} \, G \, \myvec{1 \\ 1} \\
&= \myvec{1 & 1}\myvec{1 & \rho \\ \rho & 1}\myvec{1 \\ 1} \\
&= 2 + 2\rho.
\end{align}

Since \(\vec{u}+\vec{v}\) is a unit vector,
\begin{align}
2 + 2\rho = 1 \;\;\Rightarrow\;\; \rho = -\tfrac{1}{2}.
\end{align}

Next,
\begin{align}
\|\vec{u}-\vec{v}\|^2 
&= \myvec{1 & -1} \, G \, \myvec{1 \\ -1} \\
&= \myvec{1 & -1}\myvec{1 & \rho \\ \rho & 1}\myvec{1 \\ -1} \\
&= 2 - 2\rho.
\end{align}

Substituting \(\rho = -\tfrac{1}{2}\),
\begin{align}
\|\vec{u}-\vec{v}\|^2 &= 2 - 2\left(-\tfrac{1}{2}\right) \\
&= 3.
\end{align}

Hence,
\begin{align}
\|\vec{u}-\vec{v}\| &= \sqrt{3}.
\end{align}

\noindent\(\therefore\) The required result is proved.
\begin{figure}[H]
  \centering
   \includegraphics[width=0.7\linewidth]{../figs/vectors_plot.png}
   \caption{}
  \label{stemplot}
\end{figure}


\end{document}

