
\documentclass{beamer}
\usepackage[utf8]{inputenc}

\usetheme{Madrid}
\usecolortheme{default}
\usepackage{amsmath,amssymb,amsfonts,amsthm}
\usepackage{txfonts}
\usepackage{tkz-euclide}
\usepackage{listings}
\usepackage{adjustbox}
\usepackage{array}
\usepackage{tabularx}
\usepackage{gvv}
\usepackage{lmodern}
\usepackage{circuitikz}
\usepackage{tikz}
\usepackage{graphicx}
\usepackage{mathtools}
\setbeamertemplate{page number in head/foot}[totalframenumber]

\usepackage{tcolorbox}
\tcbuselibrary{minted,breakable,xparse,skins}



\definecolor{bg}{gray}{0.95}
\DeclareTCBListing{mintedbox}{O{}m!O{}}{%
  breakable=true,
  listing engine=minted,
  listing only,
  minted language=#2,
  minted style=default,
  minted options={%
    linenos,
    gobble=0,
    breaklines=true,
    breakafter=,,
    fontsize=\small,
    numbersep=8pt,
    #1},
  boxsep=0pt,
  left skip=0pt,
  right skip=0pt,
  left=25pt,
  right=0pt,
  top=3pt,
  bottom=3pt,
  arc=5pt,
  leftrule=0pt,
  rightrule=0pt,
  bottomrule=2pt,
  toprule=2pt,
  colback=bg,
  colframe=orange!70,
  enhanced,
  overlay={%
    \begin{tcbclipinterior}
    \fill[orange!20!white] (frame.south west) rectangle ([xshift=20pt]frame.north west);
    \end{tcbclipinterior}},
  #3,
}
\lstset{
    language=C,
    basicstyle=\ttfamily\small,
    keywordstyle=\color{blue},
    stringstyle=\color{orange},
    commentstyle=\color{green!60!black},
    numbers=left,
    numberstyle=\tiny\color{gray},
    breaklines=true,
    showstringspaces=false,
}

\title{2.6.18}
\author{Dhanush Kumar A - AI25BTECH11010}
\date{September 9, 2025}

\begin{document}

\frame{\titlepage}

\begin{frame}{Question}
Find the area of the region bounded by the triangle whose vertices are $(-1, 0)$, $(1, 3)$ and $(3, 2)$.
\end{frame}

\begin{frame}{Variables Used}
\begin{table}[H]    
  \centering
  \begin{tabular}[12pt]{ |c| c|}
    \hline
    \textbf{Name} & \textbf{Point}\\ 
    \hline
	Point A &\myvec{h \\ k}\\
    \hline 
 Point B &\myvec{x1 \\ y1}\\
    \hline
	  Point R &\myvec{x2 \\ y2}\\
    \hline
    
    \end{tabular}
 % ensure path is correct
  \caption{Variables Used}
  \label{tab:variables}
\end{table}
\end{frame}




\begin{frame}{Solution}
\begin{align}
\text{Area of triangle ABC} &= \frac{1}{2} \, \big| (\vec{A}-\vec{B}) \times (\vec{A}-\vec{C}) \big| \\[1mm]
\vec{A}-\vec{B} &= \myvec{-1 \\ 0} - \myvec{1 \\ 3} = \myvec{-2 \\ -3} \\[1mm]
\vec{A}-\vec{C} &= \myvec{-1 \\ 0} - \myvec{3 \\ 2} = \myvec{-4 \\ -2} \\[1mm]
(\vec{A}-\vec{B}) \times (\vec{A}-\vec{C}) &= (-2)(-2) - (-3)(-4) = 4 - 12 = -8 \\[1mm]
\text{Area} &= \frac{1}{2} \, | -8 | = 4
\end{align}
\noindent\textbf{Thus, the area of the triangle is 4.}
\end{frame}

\begin{frame}[fragile]                            
\frametitle{Python code - Calculating the area of triangle}                
\begin{lstlisting}
import numpy as np
import matplotlib.pyplot as plt
import os

# Define the vertices of the triangle
A = np.array([-1, 0])
B = np.array([1, 3])
C = np.array([3, 2])

# Calculate area using cross product formula
area = 0.5 * np.abs(np.cross(B - A, C - A))
print(f"Area of the triangle: {area}")
\end{lstlisting}
\end{frame}

\begin{frame}[fragile]                            
\frametitle{Python code - Plotting the triangle}                
\begin{lstlisting}
# Prepare triangle points for plotting
triangle = np.array([A, B, C, A])  # repeat first point to close the triangle

# Plot the triangle
plt.plot(triangle[:, 0], triangle[:, 1], 'b-o', label='Triangle')
plt.fill(triangle[:, 0], triangle[:, 1], 'skyblue', alpha=0.3)
plt.text(A[0], A[1], 'A', fontsize=12, color='red')
plt.text(B[0], B[1], 'B', fontsize=12, color='red')
plt.text(C[0], C[1], 'C', fontsize=12, color='red')
plt.xlabel('X-axis')
plt.ylabel('Y-axis')
plt.title('Triangle Plot')
plt.grid(True)
plt.legend()

# Save the figure
plt.savefig('../figs/triangle_plot.png')
plt.show()
\end{lstlisting}
\end{frame}
\begin{frame}{Plot-Using  Python}
    \centering
    \includegraphics[width=\columnwidth, height=0.8\textheight, keepaspectratio]{../figs/triangle_plot.png}     
\end{frame}


\begin{frame}[fragile]                            
\frametitle{C code - To calculate the area of triangle and Save points}                
\begin{lstlisting}
#include <stdio.h>
#include <stdlib.h>
#include <math.h>

// Structure to store a 2D point
typedef struct {
    double x;
    double y;
} Point;

// Function to calculate the area of a triangle using 2D determinant formula
double triangle_area(Point A, Point B, Point C) {
    return 0.5 * fabs(A.x*(B.y - C.y) + B.x*(C.y - A.y) + C.x*(A.y - B.y));
}
\end{lstlisting}
\end{frame}
\begin{frame}[fragile]                            
\frametitle{C code - To calculate the area of triangle and Save points}                
\begin{lstlisting}

// Function to save points and area to a file
void save_points_and_area(const char *filename, Point A, Point B, Point C, double area) {
    FILE *fp = fopen(filename, "w");
    if (fp == NULL) {
        printf("Error opening file!\n");
        exit(1);
    }
    fprintf(fp, "Triangle Vertices:\n");
    fprintf(fp, "A: %.2lf %.2lf\n", A.x, A.y);
    fprintf(fp, "B: %.2lf %.2lf\n", B.x, B.y);
    fprintf(fp, "C: %.2lf %.2lf\n", C.x, C.y);
    fprintf(fp, "Area of the triangle: %.2lf\n", area);
    fclose(fp);
}
\end{lstlisting}
\end{frame}
\begin{frame}[fragile]                            
\frametitle{C code - To calculate the area of triangle and Save points}                
\begin{lstlisting}

int main() {
    // Triangle vertices
    Point A = {-1, 0};
    Point B = {1, 3};
    Point C = {3, 2};

    // Calculate area
    double area = triangle_area(A, B, C);

    // Print points and area
    printf("Triangle Vertices:\n");
    printf("A: (%.2lf, %.2lf)\n", A.x, A.y);
    printf("B: (%.2lf, %.2lf)\n", B.x, B.y);
    printf("C: (%.2lf, %.2lf)\n", C.x, C.y);
    printf("Area of the triangle: %.2lf\n", area);
\end{lstlisting}
\end{frame}
\begin{frame}[fragile]                            
\frametitle{C code - To calculate the area of triangle and Save points}                
\begin{lstlisting}

    // Save points and area to file
    save_points_and_area("points.dat", A, B, C, area);
    printf("Triangle points and area saved in points.dat\n");

    return 0;
    }
\end{lstlisting}
\end{frame}

	
\begin{frame}[fragile]                              
	\frametitle{Python code -Ploting the points using c function} 
	\begin{lstlisting}
import os
import matplotlib.pyplot as plt

# Run the C program
# On Windows use: os.system("triangle.exe")
os.system("./triangle")  # Linux/Mac

# Read points and area from points.dat
points = {}
area=0
	\end{lstlisting}
\end{frame}

	
\begin{frame}[fragile]                              
	\frametitle{Python code -Ploting the points using c function} 
	\begin{lstlisting}


with open("points.dat", "r") as file:
    for line in file:
        line = line.strip()
        if line.startswith("A:"):
            x, y = map(float, line[2:].split())
            points['A'] = (x, y)
        elif line.startswith("B:"):
            x, y = map(float, line[2:].split())
            points['B'] = (x, y)
        elif line.startswith("C:"):
            x, y = map(float, line[2:].split())
            points['C'] = (x, y)
        elif line.startswith("Area"):
     print(f"Read Area from C program: {area}")
print("Triangle Points:", points)
\end{lstlisting}
\end{frame}

	
\begin{frame}[fragile]                              
	\frametitle{Python code -Ploting the points using c function} 
	\begin{lstlisting}

# Prepare triangle points for plotting
triangle_coords = [points['A'], points['B'], points['C'], points['A']]  # close the triangle
x_vals, y_vals = zip(*triangle_coords)

# Create figs folder if it doesn't exist
os.makedirs('figs', exist_ok=True)

# Plot
plt.plot(x_vals, y_vals, 'b-o', label='Triangle')
plt.fill(x_vals, y_vals, 'skyblue', alpha=0.3)

# Label points
for label, coord in points.items():
    plt.text(coord[0], coord[1], label, fontsize=12, color='red')
\end{lstlisting}
\end{frame}

	
\begin{frame}[fragile]                              
	\frametitle{Python code -Ploting the points using c function} 
	\begin{lstlisting}

plt.title(f'Triangle Plot (Area = {area})')
plt.xlabel('X-axis')
plt.ylabel('Y-axis')
plt.grid(True)
plt.legend()

# Save the plot in figs folder
plt.savefig('../figs/triangle_plot1.png', dpi=300)
print("Triangle plot saved in figs/triangle_plot.png")
plt.show()


\end{lstlisting}                               
\end{frame}

\begin{frame}{Plot-Using  Python and C}
    \centering
    \includegraphics[width=\columnwidth, height=0.8\textheight, keepaspectratio]{../figs/triangle_plot1.png}     
\end{frame}

	


\end{document}
