\documentclass{beamer}
\usepackage[utf8]{inputenc}

\usetheme{Madrid}
\usecolortheme{default}
\usepackage{amsmath,amssymb,amsfonts,amsthm}
\usepackage{txfonts}
\usepackage{tkz-euclide}
\usepackage{listings}
\usepackage{adjustbox}
\usepackage{array}
\usepackage{tabularx}
\usepackage{gvv}
\usepackage{lmodern}
\usepackage{circuitikz}
\usepackage{tikz}
\usepackage{graphicx}
\usepackage{mathtools}
\setbeamertemplate{page number in head/foot}[totalframenumber]

\usepackage{tcolorbox}
\tcbuselibrary{minted,breakable,xparse,skins}



\definecolor{bg}{gray}{0.95}
\DeclareTCBListing{mintedbox}{O{}m!O{}}{%
  breakable=true,
  listing engine=minted,
  listing only,
  minted language=#2,
  minted style=default,
  minted options={%
    linenos,
    gobble=0,
    breaklines=true,
    breakafter=,,
    fontsize=\small,
    numbersep=8pt,
    #1},
  boxsep=0pt,
  left skip=0pt,
  right skip=0pt,
  left=25pt,
  right=0pt,
  top=3pt,
  bottom=3pt,
  arc=5pt,
  leftrule=0pt,
  rightrule=0pt,
  bottomrule=2pt,
  toprule=2pt,
  colback=bg,
  colframe=orange!70,
  enhanced,
  overlay={%
    \begin{tcbclipinterior}
    \fill[orange!20!white] (frame.south west) rectangle ([xshift=20pt]frame.north west);
    \end{tcbclipinterior}},
  #3,
}
\lstset{
    language=C,
    basicstyle=\ttfamily\small,
    keywordstyle=\color{blue},
    stringstyle=\color{orange},
    commentstyle=\color{green!60!black},
    numbers=left,
    numberstyle=\tiny\color{gray},
    breaklines=true,
    showstringspaces=false,
}

\title 
{1.6.25}
\date{September 9, 2025}


\author 
{Dhanush Kumar A - AI25BTECH11010}



\begin{document}
\frame{\titlepage}
\begin{frame}{Question}
 
Three points \(P(h, k)\), \(Q(x_1, y_1)\) and \(R(x_2, y_2)\) lie on a line. Show that 
\((h-x_1)(y_2-y_1) = (k-y_1)(x_2-x_1)\).

\end{frame}
\begin{frame}{Variables used}
\begin{table}[H]    
  \centering
  \begin{tabular}[12pt]{ |c| c|}
    \hline
    \textbf{Name} & \textbf{Point}\\ 
    \hline
	Point A &\myvec{h \\ k}\\
    \hline 
 Point B &\myvec{x1 \\ y1}\\
    \hline
	  Point R &\myvec{x2 \\ y2}\\
    \hline
    
    \end{tabular}

  \caption{Variables Used}
  \label{tab:1.6.25}
\end{table}
\end{frame}
\begin{frame}{Solution}
\begin{align}
\vec{P}-\vec{Q} &= \myvec{h-x_1 \\ k-y_1} \\
\vec{R}-\vec{Q} &= \myvec{x_2-x_1 \\ y_2-y_1}
\end{align}

Now form the matrix:
\begin{align}
	\vec{M} &=\myvec{\vec{P}-\vec{Q} & \vec{R}-\vec{Q}} &=\myvec{h-x_1 & x_2-x_1 \\ k-y_1 & y_2-y_1}
\end{align}

Apply row reduction:
\begin{align}
\myvec{h-x_1 & x_2-x_1 \\ k-y_1 & y_2-y_1}
&\xrightarrow{\;R_2 \leftarrow R_2 - \tfrac{k-y_1}{h-x_1}R_1\;}
\myvec{h-x_1 & x_2-x_1 \\ 0 & (y_2-y_1) - \tfrac{k-y_1}{h-x_1}(x_2-x_1)}
\end{align}
\end{frame}
	\begin{frame}{Solution}
Since P,Qand R lie on line the rank of matrix M is 1\\
For rank \(=1\), the second entry in the last row must vanish:
\begin{align}
(y_2-y_1)(h-x_1) - (k-y_1)(x_2-x_1) &= 0
\end{align}

Thus,
\begin{align}
(h-x_1)(y_2-y_1) &= (k-y_1)(x_2-x_1).
\end{align}

Hence proved.

\end{frame}
\end{document}


