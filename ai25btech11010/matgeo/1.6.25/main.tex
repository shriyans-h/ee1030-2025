\let\negmedspace\undefined
\let\negthickspace\undefined
\documentclass[journal]{IEEEtran}
\usepackage[a5paper, margin=10mm, onecolumn]{geometry}
\usepackage{tfrupee}

\setlength{\headheight}{1cm}
\setlength{\headsep}{0mm}

\usepackage{gvv-book}
\usepackage{gvv}
\usepackage{cite}
\usepackage{amsmath,amssymb,amsfonts,amsthm}
\usepackage{algorithmic}
\usepackage{graphicx}
\usepackage{textcomp}
\usepackage{xcolor}
\usepackage{txfonts}
\usepackage{listings}
\usepackage{enumitem}
\usepackage{mathtools}
\usepackage{gensymb}
\usepackage{comment}
\usepackage[breaklinks=true]{hyperref}
\usepackage{tkz-euclide} 
\usepackage{listings}

\graphicspath{{./figs/}}

\begin{document}
\title{1.6.25}
\author{AI25BTECH11010 - Dhanush Kumar}
\maketitle
\renewcommand{\thefigure}{\theenumi}
\renewcommand{\thetable}{\theenumi}

\noindent
\textbf{Question:} \\
Three points \(P(h, k)\), \(Q(x_1, y_1)\) and \(R(x_2, y_2)\) lie on a line. Show that 
\((h-x_1)(y_2-y_1) = (k-y_1)(x_2-x_1)\).

\bigskip
\textbf{Solution:} \\
Let
\begin{align}
\vec{P} &= \myvec{h \\ k}, &
\vec{Q} &= \myvec{x_1 \\ y_1}, &
\vec{R} &= \myvec{x_2 \\ y_2}.
\end{align}

\begin{align}
\vec{P}-\vec{Q} &= \myvec{h-x_1 \\ k-y_1} \\
\vec{R}-\vec{Q} &= \myvec{x_2-x_1 \\ y_2-y_1}
\end{align}

Now form the matrix:
\begin{align}
	\vec{M} &=\myvec{\vec{P}-\vec{Q} & \vec{R}-\vec{Q}} &=\myvec{h-x_1 & x_2-x_1 \\ k-y_1 & y_2-y_1}
\end{align}

Apply row reduction:
\begin{align}
\myvec{h-x_1 & x_2-x_1 \\ k-y_1 & y_2-y_1}
&\xrightarrow{\;R_2 \leftarrow R_2 - \tfrac{k-y_1}{h-x_1}R_1\;}
\myvec{h-x_1 & x_2-x_1 \\ 0 & (y_2-y_1) - \tfrac{k-y_1}{h-x_1}(x_2-x_1)}
\end{align}
Since P,Qand R lie on line the rank of matrix M is 1\\
For rank \(=1\), the second entry in the last row must vanish:
\begin{align}
(y_2-y_1)(h-x_1) - (k-y_1)(x_2-x_1) &= 0
\end{align}

Thus,
\begin{align}
(h-x_1)(y_2-y_1) &= (k-y_1)(x_2-x_1).
\end{align}

Hence proved.

\end{document}

