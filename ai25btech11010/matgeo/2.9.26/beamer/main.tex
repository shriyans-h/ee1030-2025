\documentclass{beamer}
\usepackage[utf8]{inputenc}

\usetheme{Madrid}
\usecolortheme{default}
\usepackage{amsmath,amssymb,amsfonts,amsthm}
\usepackage{txfonts}
\usepackage{tkz-euclide}
\usepackage{listings}
\usepackage{adjustbox}
\usepackage{array}
\usepackage{tabularx}
\usepackage{gvv}
\usepackage{lmodern}
\usepackage{circuitikz}
\usepackage{tikz}
\usepackage{graphicx}
\usepackage{mathtools}
\setbeamertemplate{page number in head/foot}[totalframenumber]

\usepackage{tcolorbox}
\tcbuselibrary{minted,breakable,xparse,skins}



\definecolor{bg}{gray}{0.95}
\DeclareTCBListing{mintedbox}{O{}m!O{}}{%
  breakable=true,
  listing engine=minted,
  listing only,
  minted language=#2,
  minted style=default,
  minted options={%
    linenos,
    gobble=0,
    breaklines=true,
    breakafter=,,
    fontsize=\small,
    numbersep=8pt,
    #1},
  boxsep=0pt,
  left skip=0pt,
  right skip=0pt,
  left=25pt,
  right=0pt,
  top=3pt,
  bottom=3pt,
  arc=5pt,
  leftrule=0pt,
  rightrule=0pt,
  bottomrule=2pt,
  toprule=2pt,
  colback=bg,
  colframe=orange!70,
  enhanced,
  overlay={%
    \begin{tcbclipinterior}
    \fill[orange!20!white] (frame.south west) rectangle ([xshift=20pt]frame.north west);
    \end{tcbclipinterior}},
  #3,
}
\lstset{
    language=C,
    basicstyle=\ttfamily\small,
    keywordstyle=\color{blue},
    stringstyle=\color{orange},
    commentstyle=\color{green!60!black},
    numbers=left,
    numberstyle=\tiny\color{gray},
    breaklines=true,
    showstringspaces=false,
}

\title{2.9.26}
\author{Dhanush Kumar A - AI25BTECH11010}
\date{September 10, 2025}

\begin{document}

\frame{\titlepage}

\begin{frame}{Question}
If $f(\alpha) = \myvec{\cos\alpha & -\sin\alpha & 0 \\ \sin\alpha & \cos\alpha & 0 \\ 0 & 0 & 1}$,  
prove that $f(\alpha)f(-\beta) = f(\alpha-\beta)$.

\end{frame}
\begin{frame}{Solution}
We have
\begin{align}
f(\theta) 
&= \myvec{\cos\theta & -\sin\theta & 0 \\ \sin\theta & \cos\theta & 0 \\ 0 & 0 & 1}, \\[1em]
f(\alpha) 
	&= \myvec{\cos\alpha & -\sin\alpha & 0 \\ \sin\alpha & \cos\alpha & 0 \\ 0 & 0 & 1}\\[1em]
f(-\beta) 
&= \myvec{\cos\beta & \sin\beta & 0 \\ -\sin\beta & \cos\beta & 0 \\ 0 & 0 & 1}, \\[1em]
f(\alpha)f(-\beta) 
&= \myvec{\cos\alpha & -\sin\alpha & 0 \\ \sin\alpha & \cos\alpha & 0 \\ 0 & 0 & 1}
   \myvec{\cos\beta & \sin\beta & 0 \\ -\sin\beta & \cos\beta & 0 \\ 0 & 0 & 1}, \\[1em]
\end{align}
\end{frame}
\begin{frame}{Solution}
	\begin{align}
&= \myvec{\cos\alpha\cos\beta + \sin\alpha\sin\beta & \cos\alpha\sin\beta - \sin\alpha\cos\beta & 0 \\[2mm]
          \sin\alpha\cos\beta - \cos\alpha\sin\beta & \sin\alpha\sin\beta + \cos\alpha\cos\beta & 0 \\[1mm]
          0 & 0 & 1}, \\[1em]
&= \myvec{\cos(\alpha-\beta) & -\sin(\alpha-\beta) & 0 \\ \sin(\alpha-\beta) & \cos(\alpha-\beta) & 0 \\ 0 & 0 & 1}, \\[1em]
&= f(\alpha-\beta).
\end{align}

Thus proved.
\end{frame}
\begin{frame}[fragile]                            
\frametitle{Python code - Verify the result}                
\begin{lstlisting}
import numpy as np

def f(theta):
    return np.array([
        [np.cos(theta), -np.sin(theta), 0],
        [np.sin(theta),  np.cos(theta), 0],
        [0, 0, 1]
    ])

# Take input
alpha = float(input("Enter alpha (in radians): "))
beta = float(input("Enter beta (in radians): "))

# Compute both sides
lhs = f(alpha) @ f(-beta)
rhs = f(alpha - beta)
\end{lstlisting}
\end{frame}

\begin{frame}[fragile]    \frametitle{Python code - Verify the result}                
\begin{lstlisting}

   
# Check equality (within tolerance, since floats may not be exact)
if np.allclose(lhs, rhs, atol=1e-9):
    print("Verified: f(alpha) f(-beta) = f(alpha - beta)")
else:
    print(" Not equal")
    print("LHS =\n", lhs)
    print("RHS =\n", rhs)
\end{lstlisting}
\end{frame}

\begin{frame}[fragile]                            
\frametitle{Output of Python code}                
\begin{lstlisting}
Enter alpha (in radians): 4
Enter beta (in radians): 5
Verified: f(alpha) f(-beta) = f(alpha - beta)


\end{lstlisting}
\end{frame}

\begin{frame}[fragile]                            
\frametitle{C code - Verify the result}                
\begin{lstlisting}
#include <stdio.h>
#include <math.h>
#include <stdbool.h>

#define SIZE 3
#define EPS 1e-9  // tolerance for floating-point comparison

// Function to build matrix f(theta)
void f(double theta, double M[SIZE][SIZE]) {
    M[0][0] = cos(theta);  M[0][1] = -sin(theta); M[0][2] = 0;
    M[1][0] = sin(theta);  M[1][1] =  cos(theta); M[1][2] = 0;
    M[2][0] = 0;           M[2][1] = 0;           M[2][2] = 1;
}
\end{lstlisting}
\end{frame}

\begin{frame}[fragile]                            
\frametitle{C code - Verify the result}                
\begin{lstlisting}

// Multiply two 3x3 matrices
void multiply(double A[SIZE][SIZE], double B[SIZE][SIZE], double C[SIZE][SIZE]) {
    for (int i = 0; i < SIZE; i++) {
        for (int j = 0; j < SIZE; j++) {
            C[i][j] = 0;
            for (int k = 0; k < SIZE; k++) {
                C[i][j] += A[i][k] * B[k][j];
            }
        }
    }
}
\end{lstlisting}
\end{frame}

\begin{frame}[fragile]                            
\frametitle{C code - Verify the result}                
\begin{lstlisting}

// Check if two matrices are approximately equal
bool equal(double A[SIZE][SIZE], double B[SIZE][SIZE]) {
    for (int i = 0; i < SIZE; i++) {
        for (int j = 0; j < SIZE; j++) {
            if (fabs(A[i][j] - B[i][j]) > EPS)
                return false;
        }
    }
    return true;
}
\end{lstlisting}
\end{frame}

\begin{frame}[fragile]                            
\frametitle{C code - Verify the result}                
\begin{lstlisting}

// Print a 3x3 matrix
void printMatrix(double M[SIZE][SIZE]) {
    for (int i = 0; i < SIZE; i++) {
        for (int j = 0; j < SIZE; j++) {
            printf("%10.6f ", M[i][j]);
        }
        printf("\n");
    }
}
\end{lstlisting}
\end{frame}

\begin{frame}[fragile]                            
\frametitle{C code - Verify the result}                
\begin{lstlisting}

int main() {
    double alpha, beta;
    printf("Enter alpha (in radians): ");
    scanf("%lf", &alpha);
    printf("Enter beta (in radians): ");
    scanf("%lf", &beta);

    double F_alpha[SIZE][SIZE], F_minus_beta[SIZE][SIZE], F_alpha_minus_beta[SIZE][SIZE];
    double lhs[SIZE][SIZE], rhs[SIZE][SIZE];
\end{lstlisting}
\end{frame}

\begin{frame}[fragile]                            
\frametitle{C code - Verify the result}                
\begin{lstlisting}

    // Build matrices
    f(alpha, F_alpha);
    f(-beta, F_minus_beta);
    f(alpha - beta, F_alpha_minus_beta);

    // Compute lhs = f(alpha) * f(-beta)
    multiply(F_alpha, F_minus_beta, lhs);

    // rhs = f(alpha - beta)
    for (int i = 0; i < SIZE; i++)
        for (int j = 0; j < SIZE; j++)
            rhs[i][j] = F_alpha_minus_beta[i][j];
\end{lstlisting}
\end{frame}

\begin{frame}[fragile]                            
\frametitle{C code - Verify the result}                
\begin{lstlisting}

    // Compare
    if (equal(lhs, rhs)) {
        printf("\n Verified: f(alpha) f(-beta) = f(alpha - beta)\n");
    } else {
        printf("\nNot equal!\n");
        printf("\nLHS =\n"); printMatrix(lhs);
        printf("\nRHS =\n"); printMatrix(rhs);
    }

    return 0;
}

\end{lstlisting}
\end{frame}

\begin{frame}[fragile]                            
\frametitle{Output of C code}                
\begin{lstlisting}
Enter alpha (in radians): 2
Enter beta (in radians): 1.23

 Verified: f(alpha) f(-beta) = f(alpha - beta)

\end{lstlisting}
\end{frame}

\end{document}

