\documentclass{beamer}
\mode<presentation>
\usepackage{amsmath,amssymb,mathtools}
\usepackage{textcomp}
\usepackage{gensymb}
\usepackage{adjustbox}
\usepackage{subcaption}
\usepackage{enumitem}
\usepackage{multicol}
\usepackage{listings}
\usepackage{url}
\usepackage{graphicx} % <-- needed for images
\def\UrlBreaks{\do\/\do-}

\usetheme{Boadilla}
\usecolortheme{lily}
\setbeamertemplate{footline}{
  \leavevmode%
  \hbox{%
  \begin{beamercolorbox}[wd=\paperwidth,ht=2ex,dp=1ex,right]{author in head/foot}%
    \insertframenumber{} / \inserttotalframenumber\hspace*{2ex}
  \end{beamercolorbox}}%
  \vskip0pt%
}
\setbeamertemplate{navigation symbols}{}

\lstset{
  frame=single,
  breaklines=true,
  columns=fullflexible,
  basicstyle=\ttfamily\tiny   % tiny font so code fits
}

\numberwithin{equation}{section}

% ---- your macros ----
\providecommand{\nCr}[2]{\,^{#1}C_{#2}}
\providecommand{\nPr}[2]{\,^{#1}P_{#2}}
\providecommand{\mbf}{\mathbf}
\providecommand{\pr}[1]{\ensuremath{\Pr\left(#1\right)}}
\providecommand{\qfunc}[1]{\ensuremath{Q\left(#1\right)}}
\providecommand{\sbrak}[1]{\ensuremath{{}\left[#1\right]}}
\providecommand{\lsbrak}[1]{\ensuremath{{}\left[#1\right.}}
\providecommand{\rsbrak}[1]{\ensuremath{\left.#1\right]}}
\providecommand{\brak}[1]{\ensuremath{\left(#1\right)}}
\providecommand{\lbrak}[1]{\ensuremath{\left(#1\right.}}
\providecommand{\rbrak}[1]{\ensuremath{\left.#1\right)}}
\providecommand{\cbrak}[1]{\ensuremath{\left\{#1\right\}}}
\providecommand{\lcbrak}[1]{\ensuremath{\left\{#1\right.}}
\providecommand{\rcbrak}[1]{\ensuremath{\left.#1\right\}}}
\theoremstyle{remark}
\newtheorem{rem}{Remark}
\newcommand{\sgn}{\mathop{\mathrm{sgn}}}
\providecommand{\abs}[1]{\left\vert#1\right\vert}
\providecommand{\res}[1]{\Res\displaylimits_{#1}}
\providecommand{\norm}[1]{\lVert#1\rVert}
\providecommand{\mtx}[1]{\mathbf{#1}}
\providecommand{\mean}[1]{E\left[ #1 \right]}
\providecommand{\fourier}{\overset{\mathcal{F}}{ \rightleftharpoons}}
\providecommand{\system}{\overset{\mathcal{H}}{ \longleftrightarrow}}
\providecommand{\dec}[2]{\ensuremath{\overset{#1}{\underset{#2}{\gtrless}}}}
\newcommand{\myvec}[1]{\ensuremath{\begin{pmatrix}#1\end{pmatrix}}}
\newcommand{\mydet}[1]{\ensuremath{\begin{vmatrix}#1\end{vmatrix}}}

\newenvironment{amatrix}[1]{%
  \left(\begin{array}{@{}*{#1}{c}|*{#1}{c}@{}}
}{%
  \end{array}\right)
}

\newcommand{\myaugvec}[2]{\ensuremath{\begin{amatrix}{#1}#2\end{amatrix}}}
\let\vec\mathbf
% ---------------------

\title{Matgeo Presentation - Problem 10.6.11}
\author{ee25btech11056 - Suraj.N}

\begin{document}

\begin{frame}
  \titlepage
\end{frame}

\begin{frame}{Problem Statement}

Draw a circle of radius $4\,$cm. Draw two tangents to the circle inclined at an angle of $60^\circ$ to each other.

\end{frame}

\begin{frame}{Data}

\begin{table}[h!]
  \centering
  \begin{tabular}[12pt]{ |c| c|}
    \hline
    \textbf{Name} & \textbf{Point}\\ 
    \hline
	Point A &\myvec{h \\ k}\\
    \hline 
 Point B &\myvec{x1 \\ y1}\\
    \hline
	  Point R &\myvec{x2 \\ y2}\\
    \hline
    
    \end{tabular}

  \caption*{Table : Circle}
  \label{10.6.11}
\end{table}

\end{frame}

\begin{frame}{Solution}

The parameters of the circle with center $\vec{0}$ are :

\begin{align}
  \vec{V} &= \vec{I} & \vec{u} &= \vec{0} & f &= -16
\end{align}

Let the point from which tangent is being drawn be $\vec{p}$ .\\

Let the point of contact be $\vec{q}$ and 

\begin{align}
\vec{q}^\top\vec{q} = 16 
\end{align}

From the condition of tangency we get 

\begin{align}
  \vec{q}^\top(\vec{q}-\vec{p}) &= 0\\
  \vec{p}^\top\vec{q} &= \vec{q}^\top\vec{q}\\
  \vec{p}^\top\vec{q} &= 16 \label{eq:pq} 
\end{align}

If the angle between the tangents is 60{\degree} then the angle betweent the normals at the points of contact is 120{\degree}.\\

\end{frame}

\begin{frame}{Solution}

Therefore,

\begin{align}
  \cos(\tfrac{120\degree}{2}) &= \frac{\vec{p}^\top\vec{q}}{\norm{\vec{p}}\norm{\vec{q}}}\\
  \norm{\vec{p}} &= 8\\
  \vec{p}^\top\vec{p} - 64 &= 0
\end{align}

Therefore the locus of point $\vec{p}$ is a circle with center $\vec{0}$ and radius 8 cm.\\

Consider point $\vec{P} = \myvec{8\\0}$ from which tangents are drawn.\\

Let the slope of tangent be m and the tangent equationn is given as :

\begin{align}
  \vec{n}^\top\vec{x} &= \vec{n}^\top\vec{P} & \vec{n} &= \myvec{-m\\1}
\end{align}

\end{frame}

\begin{frame}{Solution}

The length of perpendicular from the center of the circle to the tangent is equal to the radius and is given by :

\begin{align}
  4 &= \frac{|\vec{n}^\top\vec{0} - \vec{n}^\top\vec{P}|}{\norm{\vec{n}}}\\
  |\vec{n}^\top\vec{P}| &= 4\norm{\vec{n}}\\
  |-8m| &= 4\sqrt{m^2 + 1}\\
  m &= \pm \frac{1}{\sqrt{3}}
\end{align}

The normal vectors for the tangents are given as :

\begin{align}
  \vec{n_1} &= \myvec{-\tfrac{1}{\sqrt{3}}\\1} & \vec{n_2} &= \myvec{\tfrac{1}{\sqrt{3}}\\1} 
\end{align}

\end{frame}

\begin{frame}{Solution}

The points of contacts are given as :

\begin{align}
  \vec{q_i} &= \pm r \frac{\vec{n_i}}{\norm{\vec{n_i}}}
\end{align}

From \eqref{eq:pq} , $\vec{P}^\top\vec{q}=16$ , so the points of contact are :

\begin{align}
  \vec{q_1} &= \myvec{2\\2\sqrt{3}} & \vec{q_2} &= \myvec{2\\-2\sqrt{3}}
\end{align}

\end{frame}

\begin{frame}{Plot}

\begin{figure}[h!]
  \centering
  \includegraphics[width=0.6\columnwidth]{figs/circle_tangents.png} 
   \caption*{Fig : Circle and Tangents}
  \label{Fig1}
\end{figure}

\end{frame}

\end{document}


