\let\negmedspace\undefined
\let\negthickspace\undefined
\documentclass[journal,12pt,onecolumn]{IEEEtran}
\usepackage{cite}
\usepackage{amsmath,amssymb,amsfonts,amsthm}
\usepackage{algorithmic}
\usepackage{graphicx}
\graphicspath{{./figs/}}
\usepackage{textcomp}
\usepackage{xcolor}
\usepackage{txfonts}
\usepackage{listings}
\usepackage{enumitem}
\usepackage{mathtools}
\usepackage{gensymb}
\usepackage{comment}
\usepackage{caption}
\usepackage[breaklinks=true]{hyperref}
\usepackage{tkz-euclide} 
\usepackage{listings}
\usepackage{gvv}                                        
%\def\inputGnumericTable{}                                 
\usepackage[latin1]{inputenc}     
\usepackage{xparse}
\usepackage{color}                                            
\usepackage{array}
\usepackage{longtable}                                       
\usepackage{calc}                                             
\usepackage{multirow}
\usepackage{multicol}
\usepackage{hhline}                                           
\usepackage{ifthen}                                           
\usepackage{lscape}
\usepackage{tabularx}
\usepackage{array}
\usepackage{float}
\newtheorem{theorem}{Theorem}[section]
\newtheorem{problem}{Problem}
\newtheorem{proposition}{Proposition}[section]
\newtheorem{lemma}{Lemma}[section]
\newtheorem{corollary}[theorem]{Corollary}
\newtheorem{example}{Example}[section]
\newtheorem{definition}[problem]{Definition}
\newcommand{\BEQA}{\begin{eqnarray}}
\newcommand{\EEQA}{\end{eqnarray}}
\newcommand{\define}{\stackrel{\triangle}{=}}
\theoremstyle{remark}
\newtheorem{rem}{Remark}

\begin{document}

\title{10.6.11}
\author{ee25btech11056 - Suraj.N}
\maketitle
\renewcommand{\thefigure}{\theenumi}
\renewcommand{\thetable}{\theenumi}

\begin{document}

\textbf{Question :} Draw a circle of radius $4\,$cm. Draw two tangents to the circle inclined at an angle of $60^\circ$ to each other.


\textbf{Solution :}

\begin{table}[h!]
  \centering
  \begin{tabular}[12pt]{ |c| c|}
    \hline
    \textbf{Name} & \textbf{Point}\\ 
    \hline
	Point A &\myvec{h \\ k}\\
    \hline 
 Point B &\myvec{x1 \\ y1}\\
    \hline
	  Point R &\myvec{x2 \\ y2}\\
    \hline
    
    \end{tabular}

  \caption*{Table : Circle}
  \label{10.6.11}
\end{table}

The parameters of the circle with center $\vec{0}$ are :

\begin{align}
  \vec{V} &= \vec{I} & \vec{u} &= \vec{0} & f &= -16
\end{align}

Let the point from which tangent is being drawn be $\vec{p}$ .\\

Let the point of contact be $\vec{q}$ and 

\begin{align}
\vec{q}^\top\vec{q} = 16 
\end{align}

From the condition of tangency we get 

\begin{align}
  \vec{q}^\top(\vec{q}-\vec{p}) &= 0\\
  \vec{p}^\top\vec{q} &= \vec{q}^\top\vec{q}\\
  \vec{p}^\top\vec{q} &= 16 \label{eq:pq} 
\end{align}

If the angle between the tangents is 60{\degree} then the angle betweent the normals at the points of contact is 120{\degree}.\\

Therefore,

\begin{align}
  \cos(\tfrac{120\degree}{2}) &= \frac{\vec{p}^\top\vec{q}}{\norm{\vec{p}}\norm{\vec{q}}}\\
  \norm{\vec{p}} &= 8\\
  \vec{p}^\top\vec{p} - 64 &= 0
\end{align}

Therefore the locus of point $\vec{p}$ is a circle with center $\vec{0}$ and radius 8 cm.\\

Consider point $\vec{P} = \myvec{8\\0}$ from which tangents are drawn.\\

Let the slope of tangent be m and the tangent equationn is given as :

\begin{align}
  \vec{n}^\top\vec{x} &= \vec{n}^\top\vec{P} & \vec{n} &= \myvec{-m\\1}
\end{align}

\pagebreak

The length of perpendicular from the center of the circle to the tangent is equal to the radius and is given by :

\begin{align}
  4 &= \frac{|\vec{n}^\top\vec{0} - \vec{n}^\top\vec{P}|}{\norm{\vec{n}}}\\
  |\vec{n}^\top\vec{P}| &= 4\norm{\vec{n}}\\
  |-8m| &= 4\sqrt{m^2 + 1}\\
  m &= \pm \frac{1}{\sqrt{3}}
\end{align}

The normal vectors for the tangents are given as :

\begin{align}
  \vec{n_1} &= \myvec{-\tfrac{1}{\sqrt{3}}\\1} & \vec{n_2} &= \myvec{\tfrac{1}{\sqrt{3}}\\1} 
\end{align}

The points of contacts are given as :

\begin{align}
  \vec{q_i} &= \pm r \frac{\vec{n_i}}{\norm{\vec{n_i}}}
\end{align}

From \eqref{eq:pq} , $\vec{P}^\top\vec{q}=16$ , so the points of contact are :

\begin{align}
  \vec{q_1} &= \myvec{2\\2\sqrt{3}} & \vec{q_2} &= \myvec{2\\-2\sqrt{3}}
\end{align}

\pagebreak

\begin{figure}[h!]
  \centering
  \includegraphics[width=0.7\columnwidth]{figs/circle_tangents.png} 
   \caption*{Fig : Circle and Tangents}
  \label{Fig1}
\end{figure}

\end{document}
