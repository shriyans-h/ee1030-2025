\let\negmedspace\undefined
\let\negthickspace\undefined
\documentclass[journal,12pt,onecolumn]{IEEEtran}
\usepackage{cite}
\usepackage{amsmath,amssymb,amsfonts,amsthm}
\usepackage{algorithmic}
\usepackage{graphicx}
\graphicspath{{./figs/}}
\usepackage{textcomp}
\usepackage{xcolor}
\usepackage{txfonts}
\usepackage{listings}
\usepackage{enumitem}
\usepackage{mathtools}
\usepackage{gensymb}
\usepackage{comment}
\usepackage{caption}
\usepackage[breaklinks=true]{hyperref}
\usepackage{tkz-euclide} 
\usepackage{listings}
\usepackage{gvv}                                        
%\def\inputGnumericTable{}                                 
\usepackage[latin1]{inputenc}     
\usepackage{xparse}
\usepackage{color}                                            
\usepackage{array}
\usepackage{longtable}                                       
\usepackage{calc}                                             
\usepackage{multirow}
\usepackage{multicol}
\usepackage{hhline}                                           
\usepackage{ifthen}                                           
\usepackage{lscape}
\usepackage{tabularx}
\usepackage{array}
\usepackage{float}
\newtheorem{theorem}{Theorem}[section]
\newtheorem{problem}{Problem}
\newtheorem{proposition}{Proposition}[section]
\newtheorem{lemma}{Lemma}[section]
\newtheorem{corollary}[theorem]{Corollary}
\newtheorem{example}{Example}[section]
\newtheorem{definition}[problem]{Definition}
\newcommand{\BEQA}{\begin{eqnarray}}
\newcommand{\EEQA}{\end{eqnarray}}
\newcommand{\define}{\stackrel{\triangle}{=}}
\theoremstyle{remark}
\newtheorem{rem}{Remark}

\begin{document}

\title{4.11.33}
\author{ee25btech11056 - Suraj.N}
\maketitle
\renewcommand{\thefigure}{\theenumi}
\renewcommand{\thetable}{\theenumi}

\begin{document}

\textbf{Question :} Find the length of the intercept cut off by the plane $2x+y-z=5$ on the $X$ axis.

\textbf{Solution :}

\begin{table}[h!]
  \centering
  \begin{tabular}[12pt]{ |c| c|}
    \hline
    \textbf{Name} & \textbf{Point}\\ 
    \hline
	Point A &\myvec{h \\ k}\\
    \hline 
 Point B &\myvec{x1 \\ y1}\\
    \hline
	  Point R &\myvec{x2 \\ y2}\\
    \hline
    
    \end{tabular}

  \caption*{Table : Plane}
  \label{4.11.33}
\end{table}

The equation of plane is 
\begin{align}
  \vec{n}^\top\vec{x} &= c\\ 
  \myvec{2 & 1 & -1}\vec{x} &= 5
\end{align}

The $X$-intercept of the plane is of the form
\begin{align}
\vec{P} &= \myvec{a\\0\\0}.
\end{align}

$\vec{P}$ lies on the plane,
\begin{align}
\myvec{2 & 1 & -1}\vec{P} &= 5 \\
\myvec{2 & 1 & -1}\myvec{a\\0\\0} &= 5 \\
2a + 0 + 0 &= 5 \\
2a &= 5 \\
a &= \frac{5}{2}
\end{align}

Thus, the intercept point is
\begin{align}
\vec{P} &= \myvec{\frac{5}{2}\\0\\0}
\end{align}



\textbf{Answer:} The intercept length is $\frac{5}{2}$.

\pagebreak

\begin{figure}[h!]
  \centering
  \includegraphics[width=1\columnwidth]{figs/intercept.png} 
   \caption*{Fig : Plane}
  \label{Fig1}
\end{figure}

\end{document}


