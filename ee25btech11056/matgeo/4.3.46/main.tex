\let\negmedspace\undefined
\let\negthickspace\undefined
\documentclass[journal,12pt,onecolumn]{IEEEtran}
\usepackage{cite}
\usepackage{amsmath,amssymb,amsfonts,amsthm}
\usepackage{algorithmic}
\usepackage{graphicx}
\graphicspath{{./figs/}}
\usepackage{textcomp}
\usepackage{xcolor}
\usepackage{txfonts}
\usepackage{listings}
\usepackage{enumitem}
\usepackage{mathtools}
\usepackage{gensymb}
\usepackage{comment}
\usepackage{caption}
\usepackage[breaklinks=true]{hyperref}
\usepackage{tkz-euclide} 
\usepackage{listings}
\usepackage{gvv}                                        
%\def\inputGnumericTable{}                                 
\usepackage[latin1]{inputenc}     
\usepackage{xparse}
\usepackage{color}                                            
\usepackage{array}
\usepackage{longtable}                                       
\usepackage{calc}                                             
\usepackage{multirow}
\usepackage{multicol}
\usepackage{hhline}                                           
\usepackage{ifthen}                                           
\usepackage{lscape}
\usepackage{tabularx}
\usepackage{array}
\usepackage{float}
\newtheorem{theorem}{Theorem}[section]
\newtheorem{problem}{Problem}
\newtheorem{proposition}{Proposition}[section]
\newtheorem{lemma}{Lemma}[section]
\newtheorem{corollary}[theorem]{Corollary}
\newtheorem{example}{Example}[section]
\newtheorem{definition}[problem]{Definition}
\newcommand{\BEQA}{\begin{eqnarray}}
\newcommand{\EEQA}{\end{eqnarray}}
\newcommand{\define}{\stackrel{\triangle}{=}}
\theoremstyle{remark}
\newtheorem{rem}{Remark}

\begin{document}

\title{4.3.46}
\author{ee25btech11056 - Suraj.N}
\maketitle
\renewcommand{\thefigure}{\theenumi}
\renewcommand{\thetable}{\theenumi}

\begin{document}

\textbf{Question :} Find the coordinates of the point where the line through $(3,-4,-5)$ and $(2,-3,1)$ crosses the plane $2x+y+z=7$.

\textbf{Solution :} 

\begin{table}[h!]
  \centering
  \begin{tabular}[12pt]{ |c| c|}
    \hline
    \textbf{Name} & \textbf{Point}\\ 
    \hline
	Point A &\myvec{h \\ k}\\
    \hline 
 Point B &\myvec{x1 \\ y1}\\
    \hline
	  Point R &\myvec{x2 \\ y2}\\
    \hline
    
    \end{tabular}

  \caption*{Table : Line and Plane}
  \label{4.3.46}
\end{table}

Let the point of intersection be $\vec{I}$.  

The line is written as
\begin{align}
\vec{x} &= \vec{h} + k\,\vec{m} \\
\vec{h} &= \myvec{2\\-3\\1} \quad
\vec{m} = \myvec{-1\\1\\6}
\end{align}

So,
\begin{align}
  \vec{I} &= \vec{h} + k\,\vec{m}\\
\vec{I} &= \myvec{2\\-3\\1} + k\myvec{-1\\1\\6} = \myvec{2-k\\k-3\\1+6k}
\end{align}

The plane equation is
\begin{align}
\vec{n}^\top \vec{x} &= c \\
\vec{n} &= \myvec{2\\1\\1}, \quad c=7
\end{align}

Substitute $\vec{I}$ into the plane:
\begin{align}
\vec{n}^\top \vec{I} &= c \\
\myvec{2 & 1 & 1}\myvec{2-k\\k-3\\1+6k} &= 7
\end{align}

\begin{align}
  4 - 2k + k -3 + 1 +6k &= 7\\
  k &= 1
\end{align}

\pagebreak

Substitute $k=1$ back:
\begin{align}
\vec{I} &= \vec{h}+1\cdot \vec{m} \\
\vec{I} &= \myvec{2\\-3\\1} + \myvec{-1\\1\\6} = \myvec{2-1\\-3+1\\1+6} = \myvec{1\\-2\\7}.
\end{align}

\textbf{Answer:}
\begin{align}
\vec{I} = \myvec{1\\-2\\7}
\end{align}

\pagebreak

\begin{figure}[h!]
  \centering
  \includegraphics[width=0.9\columnwidth]{figs/intersection.png} 
   \caption*{Fig : Line and Plane}
  \label{Fig1}
\end{figure}


\end{document}

