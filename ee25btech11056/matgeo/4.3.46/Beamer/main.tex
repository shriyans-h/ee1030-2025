\documentclass{beamer}
\mode<presentation>
\usepackage{amsmath,amssymb,mathtools}
\usepackage{textcomp}
\usepackage{gensymb}
\usepackage{adjustbox}
\usepackage{subcaption}
\usepackage{enumitem}
\usepackage{multicol}
\usepackage{listings}
\usepackage{url}
\usepackage{graphicx} % <-- needed for images
\def\UrlBreaks{\do\/\do-}

\usetheme{Boadilla}
\usecolortheme{lily}
\setbeamertemplate{footline}{
  \leavevmode%
  \hbox{%
  \begin{beamercolorbox}[wd=\paperwidth,ht=2ex,dp=1ex,right]{author in head/foot}%
    \insertframenumber{} / \inserttotalframenumber\hspace*{2ex}
  \end{beamercolorbox}}%
  \vskip0pt%
}
\setbeamertemplate{navigation symbols}{}

\lstset{
  frame=single,
  breaklines=true,
  columns=fullflexible,
  basicstyle=\ttfamily\tiny   % tiny font so code fits
}

\numberwithin{equation}{section}

% ---- your macros ----
\providecommand{\nCr}[2]{\,^{#1}C_{#2}}
\providecommand{\nPr}[2]{\,^{#1}P_{#2}}
\providecommand{\mbf}{\mathbf}
\providecommand{\pr}[1]{\ensuremath{\Pr\left(#1\right)}}
\providecommand{\qfunc}[1]{\ensuremath{Q\left(#1\right)}}
\providecommand{\sbrak}[1]{\ensuremath{{}\left[#1\right]}}
\providecommand{\lsbrak}[1]{\ensuremath{{}\left[#1\right.}}
\providecommand{\rsbrak}[1]{\ensuremath{\left.#1\right]}}
\providecommand{\brak}[1]{\ensuremath{\left(#1\right)}}
\providecommand{\lbrak}[1]{\ensuremath{\left(#1\right.}}
\providecommand{\rbrak}[1]{\ensuremath{\left.#1\right)}}
\providecommand{\cbrak}[1]{\ensuremath{\left\{#1\right\}}}
\providecommand{\lcbrak}[1]{\ensuremath{\left\{#1\right.}}
\providecommand{\rcbrak}[1]{\ensuremath{\left.#1\right\}}}
\theoremstyle{remark}
\newtheorem{rem}{Remark}
\newcommand{\sgn}{\mathop{\mathrm{sgn}}}
\providecommand{\abs}[1]{\left\vert#1\right\vert}
\providecommand{\res}[1]{\Res\displaylimits_{#1}}
\providecommand{\norm}[1]{\lVert#1\rVert}
\providecommand{\mtx}[1]{\mathbf{#1}}
\providecommand{\mean}[1]{E\left[ #1 \right]}
\providecommand{\fourier}{\overset{\mathcal{F}}{ \rightleftharpoons}}
\providecommand{\system}{\overset{\mathcal{H}}{ \longleftrightarrow}}
\providecommand{\dec}[2]{\ensuremath{\overset{#1}{\underset{#2}{\gtrless}}}}
\newcommand{\myvec}[1]{\ensuremath{\begin{pmatrix}#1\end{pmatrix}}}
\newcommand{\mydet}[1]{\ensuremath{\begin{vmatrix}#1\end{vmatrix}}}
\let\vec\mathbf
% ---------------------

\title{Matgeo Presentation - Problem 4.3.46}
\author{ee25btech11056 - Suraj.N}

\begin{document}

\begin{frame}
  \titlepage
\end{frame}

\begin{frame}{Problem Statement}

Find the coordinates of the point where the line through $(3,-4,-5)$ and $(2,-3,1)$ crosses the plane $2x+y+z=7$.

\end{frame}

\begin{frame}{Data}

\begin{table}[h!]
  \centering
  \begin{tabular}[12pt]{ |c| c|}
    \hline
    \textbf{Name} & \textbf{Point}\\ 
    \hline
	Point A &\myvec{h \\ k}\\
    \hline 
 Point B &\myvec{x1 \\ y1}\\
    \hline
	  Point R &\myvec{x2 \\ y2}\\
    \hline
    
    \end{tabular}

  \caption*{Table : Line and Plane}
  \label{4.3.46}
\end{table}

\end{frame}

\begin{frame}{Solution}

Let the point of intersection be $\vec{P}$.  

The line is written as
\begin{align}
\vec{x} &= \vec{h} + k\,\vec{m} \\
\vec{h} &= \myvec{2\\-3\\1} \quad
\vec{m} = \myvec{-1\\1\\6}
\end{align}

So,
\begin{align}
\vec{P} &= \vec{h} + k\,\vec{m}
\end{align}

\end{frame}

\begin{frame}

The point also lies on the plane , so

\begin{align}
  \vec{n}^\top\vec{P} &= c\\
  \vec{n}^\top(\vec{h} + k\vec{m}) &= c\\
  \vec{n}^\top\vec{h} + k\vec{n}^\top\vec{m} &= c\\
  k = \frac{c -\vec{n}^\top\vec{h}}{\vec{n}^\top\vec{m}}
\end{align}

here c = 7 , by substituting the vectors for $\vec{n},\vec{h}$ and $\vec{m}$

\begin{align}
k &= \frac{7 - \myvec{2 & 1 & 1}\myvec{2\\-3\\1}}{\myvec{2 & 1 &1}\myvec{-1\\1\\6}}\\
k &= \frac{7 - 2}{5} = 1 
\end{align}

\end{frame}

\begin{frame}

Substitute $k=1$ in the line equation to obtain the point of intersection $\vec{P}$
\begin{align}
\vec{P} &= \vec{h}  +\vec{m} \\
\vec{P} &= \myvec{2\\-3\\1} + \myvec{-1\\1\\6} = \myvec{2-1\\-3+1\\1+6} = \myvec{1\\-2\\7}.
\end{align}

\textbf{Answer:}
\begin{align}
\vec{P} = \myvec{1\\-2\\7}
\end{align}
  
\end{frame} 

\begin{frame}{Plot}

\begin{figure}[h!]
  \centering
  \includegraphics[width=0.8\columnwidth]{figs/intersection.png} 
   \caption*{Fig : Line and Plane}
  \label{Fig1}
\end{figure}

\end{frame}

\section*{Appendix: Code}

% C program
\begin{frame}[fragile]{C Code: points.c}
\begin{lstlisting}[language=C]

#include <stdio.h>

double intersection() {

  double k, final = -1;

  double a[3] = {2, 1, 1};

  double dot;

  for (k = -100; k <= 100; k++) {

    double x[3] = {2 - k, -3 + k, 1 + 6 * k};

    dot = a[0] * x[0] + a[1] * x[1] + a[2] * x[2];

    if (dot == 7) {
      final = k;
      break;
    }
  }

  return final;
}

\end{lstlisting}
\end{frame}

% Python calling C
\begin{frame}[fragile]{Python: call\_c.py}
\begin{lstlisting}[language=Python]

import sys
import ctypes
import numpy as np
import matplotlib.pyplot as plt
import os

#for generating figure in figs folder
figs_folder= os.path.join("..","figs")


#loading shared object , load the file into lib which is an objedt

lib = ctypes.CDLL("./points.so")              #ctypes constructor to load a shared c library
lib.intersection.restype = ctypes.c_double    #to tell the return type is a c double
lib.intersection.argtypes=[]                  # to tell that function takes no arguments

#call the c function , by using attributes for lib
k_val = lib.intersection()

if (k_val == -1) :
    print("no solution found")
    sys.exit(0)

print(f"solution found k = {k_val}")

#writing points in the form of array for line

\end{lstlisting}
\end{frame}

\begin{frame}[fragile]{Python: call\_c.py}
\begin{lstlisting}[language=Python]

#parametric form of the given line
t = np.linspace(-5,5,200)

p = 2 + (-1)*t
q = (-3) + t
r = 1 + 6*t

#for plane

#coefficients for plane equation
a,b,c,d=2,1,1,7


l = np.linspace(-10,10,100)
m = np.linspace(-10,10,100)
l,m = np.meshgrid(l,m)

n = (d - a*l -b*m)/c

#plot

fig = plt.figure(figsize=(8,6))

ax = fig.add_subplot(111,projection="3d")

#line 
ax.plot(p,q,r,label='line',color='red',linewidth=2)

ax.scatter(1,-2,7,color='black',s=5,label='I(1,-2,7)')
ax.scatter(3,-4,-5,color='green',s=5,label='A(3,-4,-5)')
ax.scatter(2,-3,1,color='yellow',s=5,label='B(2,-3,1)')

\end{lstlisting}
\end{frame}

\begin{frame}[fragile]{Python: call\_c.py}
\begin{lstlisting}[language=Python]

#plane 
ax.plot_surface(l,m,n,alpha=0.5,color='blue',edgecolor='none')  #edgecolor='none' means edges are not drawn for the mesh and hence surface appears smooth without grid lines.

ax.set_xlabel("X Axis")
ax.set_ylabel("Y Axis")
ax.set_zlabel("Z Axis")
ax.set_title("Line and Plane")
ax.grid(True)
ax.legend()

plt.tight_layout()
fig.savefig(os.path.join(figs_folder,"intersection.png"))    #savingfig using figure object
plt.show()

\end{lstlisting}
\end{frame}

%Only Python
\begin{frame}[fragile]{Python: plot.py}
\begin{lstlisting}[language=Python]

import numpy as np
import matplotlib.pyplot as plt
import os 

#for generating figure in figs folder
figs_folder= os.path.join("..","figs")

# fixed value of k
k_val = 1

# line points
#parametric form of the given line
t = np.linspace(-5,5,200)

p = 2 + (-1)*t
q = (-3) + t
r = 1 + 6*t

# plane coefficients: 2x + y + z = 7
a, b, c, d = 2, 1, 1, 7

l = np.linspace(-10, 10, 100)
m = np.linspace(-10, 10, 100)
l, m = np.meshgrid(l, m)

n = (d - a * l - b * m) / c

# plot
fig = plt.figure(figsize=(8, 6))
ax = fig.add_subplot(111, projection="3d")

\end{lstlisting}
\end{frame}

\begin{frame}[fragile]{Python: plot.py}
\begin{lstlisting}[language=Python]

# line
ax.plot(p, q, r, label="line", color="red",linewidth=2)

# points
ax.scatter(1, -2, 7, color="black", s=5, label="I(1,-2,7)")
ax.scatter(3, -4, -5, color="green", s=5, label="A(3,-4,-5)")
ax.scatter(2, -3, 1, color="yellow", s=5, label="B(2,-3,1)")

# plane
ax.plot_surface(l, m, n, alpha=0.5, color="blue", edgecolor="none")

ax.set_xlabel("X Axis")
ax.set_ylabel("Y Axis")
ax.set_zlabel("Z Axis")
ax.set_title("Line and Plane")
ax.grid(True)
ax.legend()

plt.tight_layout()
fig.savefig(os.path.join(figs_folder,"intersection.png"))
plt.show()

\end{lstlisting}
\end{frame}

\end{document}

