\documentclass{beamer}
\mode<presentation>
\usepackage{amsmath,amssymb,mathtools}
\usepackage{textcomp}
\usepackage{gensymb}
\usepackage{adjustbox}
\usepackage{subcaption}
\usepackage{enumitem}
\usepackage{multicol}
\usepackage{listings}
\usepackage{url}
\usepackage{graphicx} % <-- needed for images
\def\UrlBreaks{\do\/\do-}

\usetheme{Boadilla}
\usecolortheme{lily}
\setbeamertemplate{footline}{
  \leavevmode%
  \hbox{%
  \begin{beamercolorbox}[wd=\paperwidth,ht=2ex,dp=1ex,right]{author in head/foot}%
    \insertframenumber{} / \inserttotalframenumber\hspace*{2ex}
  \end{beamercolorbox}}%
  \vskip0pt%
}
\setbeamertemplate{navigation symbols}{}

\lstset{
  frame=single,
  breaklines=true,
  columns=fullflexible,
  basicstyle=\ttfamily\tiny   % tiny font so code fits
}

\numberwithin{equation}{section}

% ---- your macros ----
\providecommand{\nCr}[2]{\,^{#1}C_{#2}}
\providecommand{\nPr}[2]{\,^{#1}P_{#2}}
\providecommand{\mbf}{\mathbf}
\providecommand{\pr}[1]{\ensuremath{\Pr\left(#1\right)}}
\providecommand{\qfunc}[1]{\ensuremath{Q\left(#1\right)}}
\providecommand{\sbrak}[1]{\ensuremath{{}\left[#1\right]}}
\providecommand{\lsbrak}[1]{\ensuremath{{}\left[#1\right.}}
\providecommand{\rsbrak}[1]{\ensuremath{\left.#1\right]}}
\providecommand{\brak}[1]{\ensuremath{\left(#1\right)}}
\providecommand{\lbrak}[1]{\ensuremath{\left(#1\right.}}
\providecommand{\rbrak}[1]{\ensuremath{\left.#1\right)}}
\providecommand{\cbrak}[1]{\ensuremath{\left\{#1\right\}}}
\providecommand{\lcbrak}[1]{\ensuremath{\left\{#1\right.}}
\providecommand{\rcbrak}[1]{\ensuremath{\left.#1\right\}}}
\theoremstyle{remark}
\newtheorem{rem}{Remark}
\newcommand{\sgn}{\mathop{\mathrm{sgn}}}
\providecommand{\abs}[1]{\left\vert#1\right\vert}
\providecommand{\res}[1]{\Res\displaylimits_{#1}}
\providecommand{\norm}[1]{\lVert#1\rVert}
\providecommand{\mtx}[1]{\mathbf{#1}}
\providecommand{\mean}[1]{E\left[ #1 \right]}
\providecommand{\fourier}{\overset{\mathcal{F}}{ \rightleftharpoons}}
\providecommand{\system}{\overset{\mathcal{H}}{ \longleftrightarrow}}
\providecommand{\dec}[2]{\ensuremath{\overset{#1}{\underset{#2}{\gtrless}}}}
\newcommand{\myvec}[1]{\ensuremath{\begin{pmatrix}#1\end{pmatrix}}}
\newcommand{\mydet}[1]{\ensuremath{\begin{vmatrix}#1\end{vmatrix}}}

\newenvironment{amatrix}[1]{%
  \left(\begin{array}{@{}*{#1}{c}|*{#1}{c}@{}}
}{%
  \end{array}\right)
}

\newcommand{\myaugvec}[2]{\ensuremath{\begin{amatrix}{#1}#2\end{amatrix}}}
\let\vec\mathbf
% ---------------------

\title{Matgeo Presentation - Problem 12.485}
\author{ee25btech11056 - Suraj.N}

\begin{document}

\begin{frame}
  \titlepage
\end{frame}

\begin{frame}{Problem Statement}

Let 

\begin{align*}
  \vec{M} = \myvec{0 & 1\\0 & 1}
\end{align*}

Which of the following is correct 
\begin{itemize}
  \item (1) Rank of $\vec{M}$ is 1 and $\vec{M}$ is diagonalizable
  \item (2) Rank of $\vec{M}$ is 2 and $\vec{M}$ is diagonalizable
  \item (3) 1 is the only eigenvalue and $\vec{M}$ is diagonalizable
  \item (4) 1 is the only eigenvalue and $\vec{M}$ is not diagonalizable
\end{itemize}

\end{frame}

\begin{frame}{Data}

\begin{table}[h!]
  \centering
  \begin{tabular}[12pt]{ |c| c|}
    \hline
    \textbf{Name} & \textbf{Point}\\ 
    \hline
	Point A &\myvec{h \\ k}\\
    \hline 
 Point B &\myvec{x1 \\ y1}\\
    \hline
	  Point R &\myvec{x2 \\ y2}\\
    \hline
    
    \end{tabular}

  \caption*{Table : Matrix}
  \label{12.485}
\end{table}

\end{frame}

\begin{frame}{Solution}

First convert $\vec{M}$ into echelon form by applying row reduction

\begin{align}
\myvec{0 & 1 \\ 0 & 1}
\xleftrightarrow{\;R_2 \to R_2 - R_1}
\myvec{0 & 1 \\ 0 & 0}
\end{align}

From the echelon form we see that there is one nonzero row, hence

\begin{align}
\text{rank}(\vec{M}) = 1
\end{align}

Next find the eigenvalues. Because $\vec{M}$ is upper triangular, its eigenvalues are the diagonal entries:

\begin{align}
  \lambda_1 &= 0 & \lambda_2 &= 1 
\end{align}

Now find eigenvectors by solving 
\begin{align}
  (\vec{M}-\lambda \vec{I})\vec{v} = \vec{0}
\end{align}

\end{frame}

\begin{frame}{Solution}

For $\lambda = 0$ solve 
\begin{align}
\vec{M}\vec{v}=\vec{0}\\
\myvec{0 & 1 \\ 0 & 1}\vec{v} = \myvec{0\\0}\\
\myvec{0 & 1\\0 & 1}\myvec{x\\y} = \myvec{0\\0}
\end{align}

\pagebreak

This gives 

\begin{align}
y = 0
\end{align}

And x can be anything\\

Thus an eigenvector for $\lambda=0$ is 
\begin{align} 
\vec{v}_1 = \myvec{1\\0}
\end{align}

\end{frame}

\begin{frame}{Solution}

For $\lambda = 1$ solve 

\begin{align}
(\vec{M}-\vec{I})\vec{v}=\vec{0}\\
\myvec{-1 & 1 \\ 0 & 0}\vec{v} = \myvec{0\\0}\\
\myvec{-1 & 1 \\ 0 & 0}\myvec{x\\y} = \myvec{0\\0}
\end{align}
This gives 
\begin{align}
-x + y = 0 \\
y = x
\end{align}
Thus an eigenvector for $\lambda=1$ is
\begin{align}
\vec{v}_2 = \myvec{1\\1}
\end{align}

\end{frame}

\begin{frame}{Solution}

Since the eigenvalues $\lambda_1$ and $\lambda_2$ are distinct, the matrix $\vec{M}$ is diagonalizable.

Form the matrix $\vec{P}$ with eigenvectors as columns and the diagonal matrix $\vec{D}$ of eigenvalues:
\begin{align}
\vec{P} = \myvec{1 & 1 \\ 0 & 1}\\
\vec{D} = \myvec{0 & 0 \\ 0 & 1}
\end{align}

Compute $\vec{P}^{-1}$ 

\begin{align}
\myaugvec{2}{1 & 1 & 1 & 0\\0 & 1 & 0 & 1}
\xleftrightarrow{\;R_1 \to R_1 - R_2}
\myaugvec{2}{1 & 0 & 1 & -1\\0 & 1 & 0 & 1}
\end{align}

The right block gives
\begin{align}
\vec{P}^{-1} = \myvec{1 & -1 \\ 0 & 1}
\end{align}

\end{frame}

\begin{frame}{Solution}

Finally, the diagonalization:
\begin{align}
  \vec{M} &= \vec{P}\vec{D}\vec{P}^{-1} \\
  \vec{M} &= \myvec{1 & 1 \\ 0 & 1}\myvec{0 & 0 \\ 0 & 1}\myvec{1 & -1 \\ 0 & 1}
\end{align}

\textbf{Conclusion :} The matrix $\vec{M}$ has $rank(\vec{M}) = 1$ and is \textbf{diagonizable}.\\
Therefore the correct option is (1).

\end{frame}

\end{document}


