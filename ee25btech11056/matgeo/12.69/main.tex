\let\negmedspace\undefined
\let\negthickspace\undefined
\documentclass[journal,12pt,onecolumn]{IEEEtran}
\usepackage{cite}
\usepackage{amsmath,amssymb,amsfonts,amsthm}
\usepackage{algorithmic}
\usepackage{graphicx}
\graphicspath{{./figs/}}
\usepackage{textcomp}
\usepackage{xcolor}
\usepackage{txfonts}
\usepackage{listings}
\usepackage{enumitem}
\usepackage{mathtools}
\usepackage{gensymb}
\usepackage{comment}
\usepackage{caption}
\usepackage[breaklinks=true]{hyperref}
\usepackage{tkz-euclide} 
\usepackage{listings}
\usepackage{gvv}                                        
%\def\inputGnumericTable{}                                 
\usepackage[latin1]{inputenc}     
\usepackage{xparse}
\usepackage{color}                                            
\usepackage{array}
\usepackage{longtable}                                       
\usepackage{calc}                                             
\usepackage{multirow}
\usepackage{multicol}
\usepackage{hhline}                                           
\usepackage{ifthen}                                           
\usepackage{lscape}
\usepackage{tabularx}
\usepackage{array}
\usepackage{float}
\newtheorem{theorem}{Theorem}[section]
\newtheorem{problem}{Problem}
\newtheorem{proposition}{Proposition}[section]
\newtheorem{lemma}{Lemma}[section]
\newtheorem{corollary}[theorem]{Corollary}
\newtheorem{example}{Example}[section]
\newtheorem{definition}[problem]{Definition}
\newcommand{\BEQA}{\begin{eqnarray}}
\newcommand{\EEQA}{\end{eqnarray}}
\newcommand{\define}{\stackrel{\triangle}{=}}
\theoremstyle{remark}
\newtheorem{rem}{Remark}

\begin{document}

\title{12.69}
\author{ee25btech11056 - Suraj.N}
\maketitle
\renewcommand{\thefigure}{\theenumi}
\renewcommand{\thetable}{\theenumi}

\begin{document}

\textbf{Question :} Find the \textbf{condition number} for the matrix
\[
\vec{A} = \myvec{2 & 1 \\ 0 & 3}
\]

\textbf{Solution :}

\begin{table}[h!]
  \centering
  \begin{tabular}[12pt]{ |c| c|}
    \hline
    \textbf{Name} & \textbf{Point}\\ 
    \hline
	Point A &\myvec{h \\ k}\\
    \hline 
 Point B &\myvec{x1 \\ y1}\\
    \hline
	  Point R &\myvec{x2 \\ y2}\\
    \hline
    
    \end{tabular}

  \caption*{Table : Matrix}
  \label{12.69}
\end{table}

The \textbf{condition number} of a matrix measures how sensitive the solution of a linear system involving that matrix is to small changes or errors in the input data. More precisely, it is the ratio of the largest singular value of the matrix to the smallest singular value

\begin{align}
\kappa(\vec{A}) = \frac{\sigma_{\max}(\vec{A})}{\sigma_{\min}(\vec{A})} 
\end{align}

\textbf{SVD / singular-value method}

Calculate $\vec{A}^\top\vec{A}$

\begin{align}
\vec{A}^\top &= \myvec{2 & 0 \\ 1 & 3} \\
\vec{A}^\top\vec{A} &= \myvec{2 & 0 \\ 1 & 3}\myvec{2 & 1 \\ 0 & 3}
= \myvec{4 & 2 \\ 2 & 10}
\end{align}

Then , find the eigen values of $\vec{A}^\top\vec{A}$

\begin{align}
\mydet{\vec{A}^\top\vec{A}-\lambda\vec{I}} = 0 
\end{align}
\begin{align}
\mydet{4-\lambda & 2 \\ 2 & 10-\lambda} = 0 
\end{align}
\begin{align}
\mydet{4-\lambda & 2 \\ 2 & 10-\lambda} 
\xleftrightarrow{R_2 \to R_2 -\tfrac{2}{(4-\lambda)}R_1}
\mydet{4-\lambda & 2 \\ 0 & \frac{(4-\lambda)(10-\lambda)-4}{(4-\lambda)}} = 0
\end{align}

By calculating the determinant 

\begin{align}
(4-\lambda)(10-\lambda)-4 = 0 \\
\lambda^2 - 14\lambda + 36 = 0
\end{align}

\pagebreak

The eigenvalues are

\begin{align}
\lambda_i &= \frac{14 \pm \sqrt{196 - 144}}{2}
= \frac{14 \pm \sqrt{52}}{2}
= 7 \pm \sqrt{13}
\end{align}

So,

\begin{align}
\lambda_1 &= 7+\sqrt{13} & \lambda_2 &= 7-\sqrt{13}
\end{align}

The singular values are

\begin{align}
\sigma_{\max} &= \sqrt{7+\sqrt{13}} & \sigma_{\min} = \sqrt{7-\sqrt{13}}
\end{align}

Finally, the \textbf{condition number} is

\begin{align}
\kappa(\vec{A}) = \frac{\sigma_{\max}(\vec{A})}{\sigma_{\min}(\vec{A})} = \sqrt{\frac{7+\sqrt{13}}{7-\sqrt{13}}} = 1.768
\end{align}

The \textbf{condition number} of $\vec{A}$ is 

\begin{align}
\kappa(\vec{A}) = 1.768
\end{align}

\end{document}
