\let\negmedspace\undefined
\let\negthickspace\undefined
\documentclass[journal,12pt,onecolumn]{IEEEtran}
\usepackage{cite}
\usepackage{amsmath,amssymb,amsfonts,amsthm}
\usepackage{algorithmic}
\usepackage{graphicx}
\graphicspath{{./figs/}}
\usepackage{textcomp}
\usepackage{xcolor}
\usepackage{txfonts}
\usepackage{listings}
\usepackage{enumitem}
\usepackage{mathtools}
\usepackage{gensymb}
\usepackage{comment}
\usepackage{caption}
\usepackage[breaklinks=true]{hyperref}
\usepackage{tkz-euclide} 
\usepackage{listings}
\usepackage{gvv}                                        
%\def\inputGnumericTable{}                                 
\usepackage[latin1]{inputenc}     
\usepackage{xparse}
\usepackage{color}                                            
\usepackage{array}                                            
\usepackage{longtable}                                       
\usepackage{calc}                                             
\usepackage{multirow}
\usepackage{multicol}
\usepackage{hhline}                                           
\usepackage{ifthen}                                           
\usepackage{lscape}
\usepackage{tabularx}
\usepackage{array}
\usepackage{float}
%\newtheorem{theorem}{Theorem}[section]
%\newtheorem{theorem}{Theorem}[section]
%\newtheorem{problem}{Problem}
%\newtheorem{proposition}{Proposition}[section]
%\newtheorem{lemma}{Lemma}[section]
%\newtheorem{corollary}[theorem]{Corollary}
%\newtheorem{example}{Example}[section]
%\newtheorem{definition}[problem]{Definition}

\begin{document}

%\textbf{\Large 1.8.9} \\
%\textbf{\large AI25BTECH11027 - NAGA BHUVANA} \\
\title{1.8.9}
\author{AI25BTECH11033 - SPOORTHI}
% \maketitle
% \newpage
% \bigskip
%\begin{document}
{\let\newpage\relax\maketitle}
%\renewcommand{\thefigure}{\theenumi}
%\renewcommand{\thetable}{\theenumi}

	
\textbf{Question}:\\
\noindent The distance of the point \( \Vec{P}(-6, 8) \) from the origin is 

\textbf{solution}:

Let the point be represented as a column matrix or (vector).
\begin{align}
    \Vec{P}=\myvec{-6 \\8} \;  \text{and}  \; \Vec{O}=\myvec{0\\0}
\end{align}
Consider\\
\begin{align}
\Vec{P-O}=\myvec{-6\\8}
\end{align}
Transpose the vector
\begin{align}
    \Vec{(P-O)}^T=\myvec{-6 & 8} 
\end{align}

multiply the transpose with the original vector.
\begin{align}
\Vec{(P-O)}^T \Vec{(P-O)} &= (-6)^2 + 8^2 \\
    &= 36 + 64 \\
    &= 100
\end{align}

\begin{align}
d=\|\Vec{P-O}\|= \sqrt{100}=10
\end{align}
The distance of the point $\Vec{P}\myvec{-6,8}$ from the origin is 10 units
\begin{figure}[h!]
 \centering
 \includegraphics[width=0.8\linewidth]{figs/fig1.png}
 \caption{}
 \label{fig}
 \end{figure}
\end{document}


