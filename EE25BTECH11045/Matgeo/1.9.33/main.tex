\let\negmedspace\undefined
\let\negthickspace\undefined
\documentclass[journal,12pt,onecolumn]{IEEEtran}
\usepackage{cite}
\usepackage{amsmath,amssymb,amsfonts,amsthm}
\usepackage{algorithmic}
\usepackage{graphicx}
\graphicspath{{./figs/}}
\usepackage{textcomp}
\usepackage{xcolor}
\usepackage{txfonts}
\usepackage{listings}
\usepackage{enumitem}
\usepackage{mathtools}
\usepackage{gensymb}
\usepackage{comment}
\usepackage{caption}
\usepackage[breaklinks=true]{hyperref}
\usepackage{tkz-euclide} 
\usepackage{listings}
\usepackage{gvv}                                        
%\def\inputGnumericTable{}                                 
\usepackage[latin1]{inputenc}     
\usepackage{xparse}
\usepackage{color}                                            
\usepackage{array}
\usepackage{longtable}                                       
\usepackage{calc}                                             
\usepackage{multirow}
\usepackage{multicol}
\usepackage{hhline}                                           
\usepackage{ifthen}                                           
\usepackage{lscape}
\usepackage{tabularx}
\usepackage{array}
\usepackage{float}
\newtheorem{theorem}{Theorem}[section]
\newtheorem{problem}{Problem}
\newtheorem{proposition}{Proposition}[section]
\newtheorem{lemma}{Lemma}[section]
\newtheorem{corollary}[theorem]{Corollary}
\newtheorem{example}{Example}[section]
\newtheorem{definition}[problem]{Definition}
\newcommand{\BEQA}{\begin{eqnarray}}
\newcommand{\EEQA}{\end{eqnarray}}
\newcommand{\define}{\stackrel{\triangle}{=}}
\theoremstyle{remark}
\newtheorem{rem}{Remark}

\begin{document}
\title{1.9.33}
\author{EE25BTECH11045 - P.Navya Priya}
\maketitle
\renewcommand{\thefigure}{\theenumi}
\renewcommand{\thetable}{\theenumi}

\textbf{Question:} If $\textbf{Q}(0,1)$ is equidistant from $\textbf{P}(5,-3)$ and $\textbf{R}(x,6)$. Find the value of $x$.
\vspace{0.5cm}

\textbf{Solution:}
\begin{table}[H]
\centering
\renewcommand{\arraystretch}{1.5}
\begin{tabular}{|m{1cm}|m{1cm}|}
\hline
  $\vec{P}$   &  $\myvec{5\\-3}$ \\ \hline 
  $\vec{Q}$   &  $\myvec{0\\1}$ \\ \hline
  $\vec{R}$   &  $\myvec{x\\6}$ \\ \hline
\end{tabular}
\end{table}

Since $\vec{Q}$ is equidistant from $\vec{P}$ and $\vec{R}$,

\begin{align}
    \norm{\myvec{\vec{Q} - \vec{P}}} = \norm{\myvec{\vec{Q} - \vec{R}}}
\end{align}

\begin{align}
     \norm{\myvec{\vec{Q} - \vec{P}}}^2 = \norm{\myvec{\vec{Q} - \vec{R}}}^2
\end{align}

\begin{align}
 {||\vec{Q}||}^2 \, - \, 2\vec{Q}^\top\vec{P} \, + \, {||\vec{P}||}^2 \, = \, {||\vec{Q}||}^2 \, - \, 2\vec{Q}^\top\vec{R} \, + \, {||\vec{R}||}^2
\end{align}

\begin{align}
    {(\vec{P} - \vec{R})}^\top \vec{Q} \, &= \, \frac{{||\vec{P}||}^2\, - \, {||\vec{R}||}^2}{2}
\end{align}

After substituting the values,

\begin{align}
    \myvec{5-x\\-9}^\top\myvec{0\\1}\, = \, \frac{34-x^2-36}{2}
\end{align}

\begin{align}
    -18 \, = \, -2 -x^2
\end{align}

Therefore,
\begin{align}
    x \, = \, \pm 4
\end{align}

\newpage

\begin{figure}[H]
\centering
\includegraphics[width=1\columnwidth]{figs/graph.png}
 \caption*{Equidistant Points from $\vec{Q}$ with Distances}
\label{fig:graph.png}
\end{figure}
\end{document}





































































\end{document}
