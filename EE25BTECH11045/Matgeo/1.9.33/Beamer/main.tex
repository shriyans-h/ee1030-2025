\documentclass{beamer}
\usepackage[utf8]{inputenc}

\usetheme{Madrid}
\usecolortheme{default}
\usepackage{amsmath,amssymb,amsfonts,amsthm}
\usepackage{txfonts}
\usepackage{tkz-euclide}
\usepackage{listings}
\usepackage{adjustbox}
\usepackage{array}
\usepackage{tabularx}
\usepackage{gvv}
\usepackage{lmodern}
\usepackage{circuitikz}
\usepackage{tikz}
\usepackage{graphicx}
\usepackage{caption}
\captionsetup{labelformat=empty}  % removes "Figure:"


\setbeamertemplate{page number in head/foot}[totalframenumber]

\usepackage{tcolorbox}
\tcbuselibrary{minted,breakable,xparse,skins}



\definecolor{bg}{gray}{0.95}
\DeclareTCBListing{mintedbox}{O{}m!O{}}{%
	breakable=true,
	listing engine=minted,
	listing only,
	minted language=#2,
	minted style=default,
	minted options={%
		linenos,
		gobble=0,
		breaklines=true,
		breakafter=,,
		fontsize=\small,
		numbersep=8pt,
		#1},
	boxsep=0pt,
	left skip=0pt,
	right skip=0pt,
	left=25pt,
	right=0pt,
	top=3pt,
	bottom=3pt,
	arc=5pt,
	leftrule=0pt,
	rightrule=0pt,
	bottomrule=2pt,
	toprule=2pt,
	colback=bg,
	colframe=orange!70,
	enhanced,
	overlay={%
		\begin{tcbclipinterior}
			\fill[orange!20!white] (frame.south west) rectangle ([xshift=20pt]frame.north west);
	\end{tcbclipinterior}},
	#3,
}
\lstset{
	language=C,
	basicstyle=\ttfamily\small,
	keywordstyle=\color{blue},
	stringstyle=\color{orange},
	commentstyle=\color{green!60!black},
	numbers=left,
	numberstyle=\tiny\color{gray},
	breaklines=true,
	showstringspaces=false,
}
\begin{document}

\title 
{1.9.33}
\date{september 5,2025}

\author 
{Navya Priya - EE25BTECH11045}
\graphicspath{./figs}


\frame{\titlepage}
\begin{frame}{Question}
If $\textbf{Q}(0,1)$ is equidistant from $\textbf{P}(5,-3)$ and $\textbf{R}(x,6)$. Find the value of $x$.\\[5pt]
\textbf{Variables taken}:

\begin{table}[H]
\centering
\renewcommand{\arraystretch}{1}
\begin{tabular}{|m{1cm}|m{1cm}|}
\hline
  $\vec{P}$   &  $\myvec{5\\-3}$ \\ \hline 
  $\vec{Q}$   &  $\myvec{0\\1}$ \\ \hline
  $\vec{R}$   &  $\myvec{x\\6}$ \\ \hline
\end{tabular}
\end{table}
\end{frame}

\begin{frame}{Theoretical Solution}
Since $\vec{Q}$ is equidistant from $\vec{p}$ and $\vec{R}$
\begin{align}
    \norm{\myvec{\vec{Q} - \vec{P}}} = \norm{\myvec{\vec{Q} - \vec{R}}}
\end{align}

\begin{align}
     \norm{\myvec{\vec{Q} - \vec{P}}}^2 = \norm{\myvec{\vec{Q} - \vec{R}}}^2
\end{align}

\begin{align}
 {||\vec{Q}||}^2 \, - \, 2\vec{Q}^\top\vec{P} \, + \, {||\vec{P}||}^2 \, = \, {||\vec{Q}||}^2 \, - \, 2\vec{Q}^\top\vec{R} \, + \, {||\vec{R}||}^2
\end{align}

\begin{align}
    {(\vec{P} - \vec{R})}^\top \vec{Q} \, &= \, \frac{{||\vec{P}||}^2\, - \, {||\vec{R}||}^2}{2}
\end{align}
\end{frame}

\begin{frame}
After substituting the values,

\begin{align}
    \myvec{5-x\\-9}^\top\myvec{0\\1}\, = \, \frac{34-x^2-36}{2}
\end{align}

\begin{align}
    -18 \, = \, -2 -x^2
\end{align}

Therefore,
\begin{align}
    x \, = \, \pm 4
\end{align}
\end{frame}

\begin{frame}[fragile]{C code}
\begin{lstlisting}
#include <stdio.h>
#include <math.h>

int main() {
    int qx = 0, qy = 1;
    int px = 5, py = -3;
    int ry = 6;
    // Distance QP^2
    int dQP2 = (qx - px) * (qx - px) + (qy - py) * (qy - py);
    // Equation: (qx - x)^2 + (qy - ry)^2 = dQP^2
    // => (0 - x)^2 + (1 - 6)^2 = dQP2
    // => x^2 + 25 = dQP2
    int rhs = dQP2 - 25;
    int x1 = (int)sqrt(rhs);
    int x2 = -x1;
     printf("The value of x can be %d or %d\n", x1, x2);
    return 0;
}
\end{lstlisting}    
\end{frame}

\begin{frame}[fragile]{Call C.py}
\begin{lstlisting}
import subprocess

# Compile the C program (only once)
subprocess.run(["gcc", "equidistant.c", "-o", "equidistant", "-lm"])

# Run the compiled program and capture output
result = subprocess.run(["./equidistant"], capture_output=True, text=True)

print("Output from C program:")
print(result.stdout)
\end{lstlisting}
\end{frame}

\begin{frame}[fragile]{Plot.py}
\begin{lstlisting}
import matplotlib.pyplot as plt

# Points
Q = (0, 1)
P = (5, -3)
R1 = (4, 6)
R2 = (-4, 6)

# Plot points with markers
plt.scatter(*Q, color='red', s=100, marker='o', label='Q(0,1)')
plt.scatter(*P, color='blue', s=100, marker='o', label='P(5,-3)')
plt.scatter(*R1, color='green', s=100, marker='o', label='R1(4,6)')
plt.scatter(*R2, color='purple', s=100, marker='o', label='R2(-4,6)')
# Draw lines QP, QR1, QR2
plt.plot([Q[0], P[0]], [Q[1], P[1]], 'b--')
plt.plot([Q[0], R1[0]], [Q[1], R1[1]], 'g--')
plt.plot([Q[0], R2[0]], [Q[1], R2[1]], 'm--')
\end{lstlisting}
\end{frame}

\begin{frame}[fragile]{Plot.py}
\begin{lstlisting}
# Annotate points
plt.text(Q[0]+0.2, Q[1], "Q(0,1)", fontsize=10, color='red')
plt.text(P[0]+0.2, P[1], "P(5,-3)", fontsize=10, color='blue')
plt.text(R1[0]+0.2, R1[1], "R1(4,6)", fontsize=10, color='green')
plt.text(R2[0]-1, R2[1], "R2(-4,6)", fontsize=10, color='purple')

# Labels and grid
plt.xlabel('X-axis')
plt.ylabel('Y-axis')
plt.title('Equidistant Points from Q')
plt.legend()
plt.grid(True)
plt.axhline(0, color='black', linewidth=0.5)
plt.axvline(0, color='black', linewidth=0.5)

plt.show()
\end{lstlisting}
\end{frame}

\begin{frame}{Plot}
   \begin{figure}[H]
\centering
\includegraphics[width=1\columnwidth]{figs/graph.png}
 \caption*{Equidistant Points from $\vec{Q}$ with Distances}
\label{fig:graph.png}
\end{figure}
\end{frame}

\end{document}