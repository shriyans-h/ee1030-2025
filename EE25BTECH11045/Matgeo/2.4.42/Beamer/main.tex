\documentclass{beamer}
\usepackage[utf8]{inputenc}

\usetheme{Madrid}
\usecolortheme{default}
\usepackage{amsmath,amssymb,amsfonts,amsthm}
\usepackage{txfonts}
\usepackage{tkz-euclide}
\usepackage{listings}
\usepackage{adjustbox}
\usepackage{array}
\usepackage{tabularx}
\usepackage{gvv}
\usepackage{lmodern}
\usepackage{circuitikz}
\usepackage{tikz}
\usepackage{graphicx}
\usepackage{caption}
\captionsetup{labelformat=empty}  % removes "Figure:"


\setbeamertemplate{page number in head/foot}[totalframenumber]

\usepackage{tcolorbox}
\tcbuselibrary{minted,breakable,xparse,skins}



\definecolor{bg}{gray}{0.95}
\DeclareTCBListing{mintedbox}{O{}m!O{}}{%
	breakable=true,
	listing engine=minted,
	listing only,
	minted language=#2,
	minted style=default,
	minted options={%
		linenos,
		gobble=0,
		breaklines=true,
		breakafter=,,
		fontsize=\small,
		numbersep=8pt,
		#1},
	boxsep=0pt,
	left skip=0pt,
	right skip=0pt,
	left=25pt,
	right=0pt,
	top=3pt,
	bottom=3pt,
	arc=5pt,
	leftrule=0pt,
	rightrule=0pt,
	bottomrule=2pt,
	toprule=2pt,
	colback=bg,
	colframe=orange!70,
	enhanced,
	overlay={%
		\begin{tcbclipinterior}
			\fill[orange!20!white] (frame.south west) rectangle ([xshift=20pt]frame.north west);
	\end{tcbclipinterior}},
	#3,
}
\lstset{
	language=C,
	basicstyle=\ttfamily\small,
	keywordstyle=\color{blue},
	stringstyle=\color{orange},
	commentstyle=\color{green!60!black},
	numbers=left,
	numberstyle=\tiny\color{gray},
	breaklines=true,
	showstringspaces=false,
}
\begin{document}

\title 
{2.4.42}
\date{september 9,2025}

\author 
{Navya Priya - EE25BTECH11045}
\graphicspath{./figs}

\frame{\titlepage}
\begin{frame}{Question}
 Show that the points $\textbf{A}(1,2,3)$ , $\textbf{B}(-1,-2,-1)$ , $\textbf{C}(2,3,2)$ and $\textbf{D}(4,7,6)$ are the vertices of a parallelogram $ABCD$, but it is not a rectangle.
\end{frame}

\begin{frame}{Variables Taken}
    \begin{table}[H]
\centering
\renewcommand{\arraystretch}{0.5}
\begin{tabular}{|m{2cm}|m{2cm}|}
\hline
  $\vec{A}$   &  $\myvec{1\\2\\3}$ \\ \hline 
  $\vec{B}$   &  $\myvec{-1\\-2\\-1}$ \\ \hline
  $\vec{C}$   &  $\myvec{2\\3\\2}$ \\ \hline
  $\vec{D}$   &  $\myvec{4\\7\\6}$ \\ \hline
\end{tabular}
\end{table}
\end{frame}

\begin{frame}{Equations Used}
For a quadrilateral $ABCD$ to be a parallelogram, it has to satisfy the following coditions:

\begin{align*}
  (\text{a}) & \quad\vec{B}\,-\,\vec{A}\,=\,\vec{C}\,-\,\vec{D}\\
  (\text{b}) & \quad\myvec{\vec{B-A}}^\top\myvec{\vec{C-B}}\,\neq\,0\\
  (\text{c}) & \quad\myvec{\vec{C-A}}^\top\myvec{\vec{D-B}}\,\neq\,0
  \end{align*}
\end{frame}

\begin{frame}{Theoretical Solution}
Let us solve the given equation theoretically and then verify the solution computationally\\
From the given points in the question,\\
\hspace{1cm} From (a)\\
\begin{align}
   \myvec{-1\\-2\\-1}-\myvec{1\\2\\3}=\myvec{2\\3\\2}-\myvec{4\\7\\6}\equiv \, \myvec{2\\4\\4}
\end{align}
From (b)
\begin{align}
    \vec{B}-\vec{A}=\myvec{-2\\-4\\-4} \,\text{and}\, \vec{C} \vec{B}=\myvec{3\\5\\3}\\
    \implies \myvec{\vec{B-A}}^\top\myvec{\vec{C-B}}\,=\,-38(\neq0)
\end{align}
\end{frame}

\begin{frame}{Theoretical solution}
From (c)
\begin{align}
    \vec{C}-\vec{A}=\myvec{1\\1\\-1}\,\text{and}\, \vec{D}-\vec{B}=\myvec{5\\9\\7}\\
    \implies \myvec{\vec{C-A}}^\top\myvec{\vec{D-B}}\,=\,7(\neq0)
\end{align}
From (1), it is clear that the opposite sides have the same slope, which means they are parallel.\\
From (3) and (5) we can say that the sides are not perpendicular. Hence $ABCD$ is not a Rectangle.\\[4pt]
As the quadrilateral $ABCD$ satisfies all the above conditions, it is a Parallelogram.\\
\vspace{0.5cm}
\begin{align*}
    \therefore ABCD\; \text{is a Parallelogram.}
\end{align*}
\end{frame}

\begin{frame}[fragile]{C code}
\begin{lstlisting}
    #include <stdio.h>
int main() {
    int A[3] = {1, 2, 3};
    int B[3] = {-1, -2, -1};
    int C[3] = {2, 3, 2};
    int D[3] = {4, 7, 6};
    int AB[3], DC[3], AD[3], BC[3];
    // Calculate vectors
    for (int i = 0; i < 3; i++) {
        AB[i] = B[i] - A[i];
        DC[i] = C[i] - D[i];
        AD[i] = D[i] - A[i];
        BC[i] = C[i] - B[i];
    }
    
 // Check parallelogram
    int isPara = 1;
     \end{lstlisting}
\end{frame}

\begin{frame}[fragile]{C code}
\begin{lstlisting}
    for (int i = 0; i < 3; i++) {
        if (AB[i] != DC[i] || AD[i] != BC[i]) {
            isPara = 0;
            break;
        }
    }
    if (isPara) {
        printf("ABCD is a parallelogram.\n");
        // Check rectangle (dot product AB·AD)
        int dot = AB[0]*AD[0] + AB[1]*AD[1] + AB[2]*AD[2];
        if (dot == 0)
            printf("It is also a rectangle.\n");
        else
            printf("It is not a rectangle.\n");
    } else {
        printf("ABCD is not a parallelogram.\n");
    }
   return 0;
}
\end{lstlisting}
\end{frame}

\begin{frame}[fragile]{Call C.py}
\begin{lstlisting}
import ctypes
import platform

# Load shared library
if platform.system() == "Windows":
    quad_lib = ctypes.CDLL("./parallelogram.dll")
else:
    quad_lib = ctypes.CDLL("./parallelogram.so")

# Declare function argument & return types
quad_lib.check_quad.argtypes = [ctypes.c_double, ctypes.c_double, ctypes.c_double,
                                ctypes.c_double, ctypes.c_double, ctypes.c_double,
                                ctypes.c_double, ctypes.c_double, ctypes.c_double,
                                ctypes.c_double, ctypes.c_double, ctypes.c_double]
quad_lib.check_quad.restype = ctypes.c_int
 \end{lstlisting}
\end{frame}

\begin{frame}[fragile]{Call C.py}
\begin{lstlisting}

# Points: A(1,2,3), B(-1,-2,-1), C(2,3,2), D(4,7,6)
x1,y1,z1 = 1, 2, 3
x2,y2,z2 = -1, -2, -1
x3,y3,z3 = 2, 3, 2
x4,y4,z4 = 4, 7, 6

# Call C function
result = quad_lib.check_quad(x1,y1,z1, x2,y2,z2, x3,y3,z3, x4,y4,z4)

# Interpret result
if result == 0:
    print("ABCD is not a parallelogram.")
elif result == 1:
    print("ABCD is a parallelogram but not a rectangle.")
elif result == 2:
    print("ABCD is a rectangle.")

 \end{lstlisting}
\end{frame}

\begin{frame}[fragile]{Plot.py}
\begin{lstlisting}
import matplotlib.pyplot as plt
from mpl_toolkits.mplot3d.art3d import Poly3DCollection

# Points
A = (1, 2, 3)
B = (-1, -2, -1)
C = (2, 3, 2)
D = (4, 7, 6)

# Plot setup
fig = plt.figure()
ax = fig.add_subplot(111, projection='3d')

# Plot points
ax.scatter(*A, color="r", label="A(1,2,3)")
ax.scatter(*B, color="g", label="B(-1,-2,-1)")
ax.scatter(*C, color="b", label="C(2,3,2)")
ax.scatter(*D, color="orange", label="D(4,7,6)")

\end{lstlisting}
\end{frame}

\begin{frame}[fragile]{Plot.py}
\begin{lstlisting}

# Draw parallelogram edges
edges = [(A, B), (B, C), (C, D), (D, A)]
for p1, p2 in edges:
    ax.plot([p1[0], p2[0]], [p1[1], p2[1]], [p1[2], p2[2]], 'k-')

# Fill parallelogram surface
verts = [[A, B, C, D]]
ax.add_collection3d(Poly3DCollection(verts, alpha=0.2, facecolor='cyan'))

# Labels
ax.set_xlabel("X-axis")
ax.set_ylabel("Y-axis")
ax.set_zlabel("Z-axis")
ax.legend()

plt.show()
\end{lstlisting}
\end{frame}
\begin{frame}{Plot}
From the graph, theoretical solution matches with the computational solution.

\begin{figure}[H]
\centering
\includegraphics[width=0.5\columnwidth]{figs/graph.png}
 \caption*{A Parallelogram with vertices $\vec{A}$,$\vec{B},\vec{C}$ and $\vec{D}$}
\label{fig:graph.png}
\end{figure}
\end{frame}
\end{document}