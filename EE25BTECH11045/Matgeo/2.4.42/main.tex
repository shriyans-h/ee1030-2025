\let\negmedspace\undefined
\let\negthickspace\undefined
\documentclass[journal,12pt,onecolumn]{IEEEtran}
\usepackage{cite}
\usepackage{amsmath,amssymb,amsfonts,amsthm}
\usepackage{algorithmic}
\usepackage{graphicx}
\graphicspath{{./figs/}}
\usepackage{textcomp}
\usepackage{xcolor}
\usepackage{txfonts}
\usepackage{listings}
\usepackage{enumitem}
\usepackage{mathtools}
\usepackage{gensymb}
\usepackage{comment}
\usepackage{caption}
\usepackage[breaklinks=true]{hyperref}
\usepackage{tkz-euclide} 
\usepackage{listings}
\usepackage{gvv}                                        
%\def\inputGnumericTable{}                                 
\usepackage[latin1]{inputenc}     
\usepackage{xparse}
\usepackage{color}                                            
\usepackage{array}
\usepackage{longtable}                                       
\usepackage{calc}                                             
\usepackage{multirow}
\usepackage{multicol}
\usepackage{hhline}                                           
\usepackage{ifthen}                                           
\usepackage{lscape}
\usepackage{tabularx}
\usepackage{array}
\usepackage{float}
\newtheorem{theorem}{Theorem}[section]
\newtheorem{problem}{Problem}
\newtheorem{proposition}{Proposition}[section]
\newtheorem{lemma}{Lemma}[section]
\newtheorem{corollary}[theorem]{Corollary}
\newtheorem{example}{Example}[section]
\newtheorem{definition}[problem]{Definition}
\newcommand{\BEQA}{\begin{eqnarray}}
\newcommand{\EEQA}{\end{eqnarray}}
\newcommand{\define}{\stackrel{\triangle}{=}}
\theoremstyle{remark}
\newtheorem{rem}{Remark}

\begin{document}
\title{2.4.42}
\author{EE25BTECH11045 - P.Navya Priya}
\maketitle
\renewcommand{\thefigure}{\theenumi}
\renewcommand{\thetable}{\theenumi}

\textbf{Question:}  Show that the points $\textbf{A}(1,2,3)$ , $\textbf{B}(-1,-2,-1)$ , $\textbf{C}(2,3,2)$ and $\textbf{D}(4,7,6)$ are the vertices of a parallelogram $ABCD$, but it is not a rectangle.

\vspace{0.5cm}
\textbf{Soution:}

Let us solve the given equation theoretically and then verify the solution computationally.\\
\begin{table}[H]
\centering
\renewcommand{\arraystretch}{1}
\begin{tabular}{|m{2cm}|m{2cm}|}
\hline
  $\vec{A}$   &  $\myvec{1\\2\\3}$ \\ \hline 
  $\vec{B}$   &  $\myvec{-1\\-2\\-1}$ \\ \hline
  $\vec{C}$   &  $\myvec{2\\3\\2}$ \\ \hline
  $\vec{D}$   &  $\myvec{4\\7\\6}$ \\ \hline
\end{tabular}
\end{table}

For a quadrilateral $ABCD$ to be a parallelogram, it has to satisfy the following conditions:

\begin{align*}
  (\text{a}) & \quad\vec{B}\,-\,\vec{A}\,=\,\vec{C}\,-\,\vec{D}\\
  (\text{b}) & \quad\myvec{\vec{B-A}}^\top\myvec{\vec{C-B}}\,\neq\,0\\
  (\text{c}) & \quad\myvec{\vec{C-A}}^\top\myvec{\vec{D-B}}\,\neq\,0
\end{align*}
From the given points in the question,\\
\hspace{1cm} From (a)\\
\begin{align}
   \myvec{-1\\-2\\-1}-\myvec{1\\2\\3}=\myvec{2\\3\\2}-\myvec{4\\7\\6}\equiv \, \myvec{2\\4\\4}
\end{align}
From (b)
\begin{align}
    \vec{B}-\vec{A}=\myvec{-2\\-4\\-4} \,\text{and}\, \vec{C} \vec{B}=\myvec{3\\5\\3}\\
    \implies \myvec{\vec{B-A}}^\top\myvec{\vec{C-B}}\,=\,-38(\neq0)
\end{align}
From (c)
\begin{align}
    \vec{C}-\vec{A}=\myvec{1\\1\\-1}\,\text{and}\, \vec{D}-\vec{B}=\myvec{5\\9\\7}\\
    \implies \myvec{\vec{C-A}}^\top\myvec{\vec{D-B}}\,=\,7(\neq0)
\end{align}
From (1) it is clear that the opposite sides have the same slope, which means that they are parallel.\\
From (3) and (5) we can say that the sides are not perpendicular. Hence $ABCD$ is not a Rectangle.\\[4pt]
As the quadrilateral $ABCD$ satisfies all the above conditions, it is a Parallelogram.\\
\vspace{0.5cm}
\begin{align*}
    \therefore ABCD\; \text{is a Parallelogram.}
\end{align*}

\vspace{1cm}
From the graph, theoretical solution matches with the computational solution.

\begin{figure}[H]
\centering
\includegraphics[width=0.8\columnwidth]{figs/graph.png}
 \caption*{A Parallelogram with vertices $\vec{A}$,$\vec{B},\vec{C}$ and $\vec{D}$}
\label{fig:graph.png}
\end{figure}
\end{document}