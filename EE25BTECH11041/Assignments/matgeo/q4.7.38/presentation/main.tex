\documentclass{beamer}
\usepackage[utf8]{inputenc}

\usetheme{Madrid}
\usecolortheme{default}
\usepackage{amsmath,amssymb,amsfonts,amsthm}
\usepackage{txfonts}
\usepackage{tkz-euclide}
\usepackage{listings}
\usepackage{adjustbox}
\usepackage{array}
\usepackage{tabularx}
\usepackage{gvv}
\usepackage{lmodern}
\usepackage{circuitikz}
\usepackage{tikz}
\usepackage{graphicx}

\setbeamertemplate{page number in head/foot}[totalframenumber]

\usepackage{tcolorbox}
\tcbuselibrary{minted,breakable,xparse,skins}



\definecolor{bg}{gray}{0.95}
\DeclareTCBListing{mintedbox}{O{}m!O{}}{%
	breakable=true,
	listing engine=minted,
	listing only,
	minted language=#2,
	minted style=default,
	minted options={%
		linenos,
		gobble=0,
		breaklines=true,
		breakafter=,,
		fontsize=\small,
		numbersep=8pt,
		#1},
	boxsep=0pt,
	left skip=0pt,
	right skip=0pt,
	left=25pt,
	right=0pt,
	top=3pt,
	bottom=3pt,
	arc=5pt,
	leftrule=0pt,
	rightrule=0pt,
	bottomrule=2pt,
	toprule=2pt,
	colback=bg,
	colframe=orange!70,
	enhanced,
	overlay={%
		\begin{tcbclipinterior}
			\fill[orange!20!white] (frame.south west) rectangle ([xshift=20pt]frame.north west);
	\end{tcbclipinterior}},
	#3,
}
\lstset{
	language=C,
	basicstyle=\ttfamily\small,
	keywordstyle=\color{blue},
	stringstyle=\color{orange},
	commentstyle=\color{green!60!black},
	numbers=left,
	numberstyle=\tiny\color{gray},
	breaklines=true,
	showstringspaces=false,
}
\begin{document}

\title 
{4.7.38}
\date{14 September,2025}

\author 
{Naman Kumar-EE25BTECH11041}
\graphicspath{./figs}


\frame{\titlepage}
\begin{frame}{Question}
P(0, 2) is the point of intersection of Y axis and perpendicular bisector of line segment joining the points A(-1, 1) and B(3, 3).
\end{frame}
\begin{frame}{Solution}
Given points,
\begin{align}
\Vec{A}=\begin{pmatrix} -1\\1 \end{pmatrix},\Vec{B}=\begin{pmatrix} 3\\3 \end{pmatrix},\Vec{P}=\begin{pmatrix} 0\\2 \end{pmatrix}
\end{align}
Mid point of $\Vec{A}$ and $\Vec{B}$, Let it be $\Vec{R}$
\begin{align}
    \Vec{R}=\frac{\vec{A}+\vec{B}}{2}
\end{align}
\end{frame}
\begin{frame}{Solution}
Slope, $\vec{m}$
\begin{align}
    \vec{m}=\vec{B}-\vec{A}\\
\end{align}
Let $\vec{n}$ be the direction vector perpendicular to $\vec{m}$, If truly $\vec{P}$ is y-intercept of bisector 
\begin{align}
\vec{n}=\vec{P}-\vec{R}
\end{align}
\end{frame}
\begin{frame}{Solution}
Both $\vec{n}$ and $\vec{m}$ are perpendicular
\begin{align}
    \vec{n}^T\vec{m}=0\\
    (\vec{P}-\vec{R})^T(\vec{B}-\vec{A})=0\\
    (\vec{P}^T-(\frac{\vec{A}+\vec{B}}{2})^T)(\vec{B}-\vec{A})=0\\
    \vec{P}^T(\vec{B}-\vec{A})-\frac{\brak{\vec{A}+\vec{B}}^T\brak{\vec{B}-\vec{A}}}{2}=0\\
    \begin{pmatrix} 0&2 \end{pmatrix}\begin{pmatrix} 4\\2 \end{pmatrix}-\frac{\begin{pmatrix} 2&4 \end{pmatrix}\begin{pmatrix} 4\\2 \end{pmatrix}}{2}=0\\
    4-\frac{16}{2}\neq0
\end{align}
Hence, $\vec{P}$ is not the y-intercept of perpendicular bisector of line $\vec{A}-\vec{B}$
\end{frame}
\begin{frame}{Figure}
    \begin{figure}
        \centering
        \includegraphics[width=0.8\columnwidth]{figs/figure.png}
        \caption{Caption}
        \label{fig:placeholder}
    \end{figure}
\end{frame}
\begin{frame}[fragile]
\frametitle{C code}
\begin{lstlisting}
#include <stdio.h>

void midpoint(double x1, double y1, double x2, double y2, double *mx, double *my) {
    *mx = (x1 + x2) / 2.0;
    *my = (y1 + y2) / 2.0;
}
\end{lstlisting}
\end{frame}
\begin{frame}[fragile]
\frametitle{Python code}
\begin{lstlisting}
import ctypes
import matplotlib.pyplot as plt

# Load shared object
lib = ctypes.CDLL('./libmidpoint.so')

# Define argument and return types
lib.midpoint.argtypes = [ctypes.c_double, ctypes.c_double,
                         ctypes.c_double, ctypes.c_double,
                         ctypes.POINTER(ctypes.c_double),
                         ctypes.POINTER(ctypes.c_double)]

# Input points A and B
x1, y1 = -1, 1
x2, y2 = 3, 3
\end{lstlisting}
\end{frame}
\begin{frame}[fragile]
\frametitle{Python code}
\begin{lstlisting}
# Output variables
mx = ctypes.c_double()
my = ctypes.c_double()

# Call C function
lib.midpoint(x1, y1, x2, y2, ctypes.byref(mx), ctypes.byref(my))

print(f"Midpoint of A and B: ({mx.value}, {my.value})")

# Points
A = (x1, y1)
B = (x2, y2)
M = (mx.value, my.value)
P = (0, 2)  # intersection with Y-axis
\end{lstlisting}
\end{frame}
\begin{frame}[fragile]
\frametitle{Python code}
\begin{lstlisting}
# Plot
plt.figure(figsize=(6,6))
plt.plot([A[0], B[0]], [A[1], B[1]], 'b-', label="Line AB")
plt.plot([M[0],P[0]], [M[1],P[1]], 'b-', label="Line MP")
plt.scatter(*A, color='red', label="A(-1,1)")
plt.scatter(*B, color='green', label="B(3,3)")
plt.scatter(*M, color='purple', label=f"M{M}")
plt.scatter(*P, color='orange', label="P(0,2)")
\end{lstlisting}
\end{frame}
\begin{frame}[fragile]
\frametitle{Python code}
\begin{lstlisting}
plt.text(A[0]+0.08,A[1]+0.1,'A', color='red')
plt.text(B[0]-0.08,B[1]-0.1,'B', color='green')
plt.text(M[0]+0.08,M[1]+0.1,'M', color='purple')
plt.text(0+0.08,2+0.1,'P', color='orange')


# Draw perpendicular bisector line
plt.axvline(x=0, color='gray', linestyle='--', label="Y-axis")
plt.legend()
plt.grid(True)
\end{lstlisting}
\end{frame}
\begin{frame}[fragile]
\frametitle{Python code}
\begin{lstlisting}
plt.xlabel("X-axis")
plt.ylabel("Y-axis")
plt.title("Midpoint and Perpendicular Bisector Intersection")
plt.savefig("figure.png", dpi=150)
plt.show()
\end{lstlisting}
\end{frame}
\begin{frame}[fragile]
\frametitle{Direct Python code}
\begin{lstlisting}
import numpy as np
import matplotlib.pyplot as plt

plt.figure(figsize=(8, 6), dpi=100)
a=0
b=0
x=np.array([-1,3,a,0])
y=np.array([1,3,b,2])
x[2]=(x[0]+x[1])/2
y[2]=(y[0]+y[1])/2
\end{lstlisting}
\end{frame}
\begin{frame}[fragile]
\frametitle{Direct Python code}
\begin{lstlisting}
plt.scatter(x,y, color='blue', alpha=0.5, )
plt.text(x[0]+0.15, y[0]+0.15, "A(-1,1)", color='blue')
plt.text(x[1]-0.15, y[1]-0.15, "B(3,3)", color='blue')
plt.text(x[2]+0.15, y[2]+0.15, "R(1,2)", color='blue')
plt.text(x[3]+0.15, y[3]+0.15, "P(0,2)", color='blue')
plt.title("Graph")

plt.grid()
plt.plot(x,y, 'o-', color='orange', mfc='blue', ms=0,alpha=0.5, label='Line AB and PR')
plt.legend()
plt.savefig('figure.png', dpi=300, bbox_inches='tight')
plt.show()
\end{lstlisting}
\end{frame}

\end{document}