\let\negmedspace\undefined
\let\negthickspace\undefined
\documentclass[journal]{IEEEtran}
\usepackage[a5paper, margin=10mm, onecolumn]{geometry}
%\usepackage{lmodern} % Ensure lmodern is loaded for pdflatex
\usepackage{tfrupee} % Include tfrupee package

\setlength{\headheight}{1cm} % Set the height of the header box
\setlength{\headsep}{0mm}     % Set the distance between the header box and the top of the text

\usepackage{gvv-book}
\usepackage{gvv}
\usepackage{cite}
\usepackage{amsmath,amssymb,amsfonts,amsthm}
\usepackage{algorithmic}
\usepackage{graphicx}
\usepackage{textcomp}
\usepackage{xcolor}
\usepackage{txfonts}
\usepackage{listings}
\usepackage{enumitem}
\usepackage{mathtools}
\usepackage{gensymb}
\usepackage{comment}
\usepackage[breaklinks=true]{hyperref}
\usepackage{tkz-euclide} 
\usepackage{listings}
% \usepackage{gvv}                                        
\def\inputGnumericTable{}                                 
\usepackage[latin1]{inputenc}                                
\usepackage{color}                                            
\usepackage{array}                                            
\usepackage{longtable}                                       
\usepackage{calc}                                             
\usepackage{multirow}                                         
\usepackage{hhline}                                           
\usepackage{ifthen}                                           
\usepackage{lscape}
\usepackage{circuitikz}



\author{EE25BTECH11041-Naman Kumar }
\graphicspath{./figs/}

\begin{document}
\begin{center}
    \huge{4.7.38}\\
    \large{EE25BTECH11041 - Naman Kumar}
\end{center}
Question:\\
P(0, 2) is the point of intersection of Y axis and perpendicular bisector of line segment joining the points A(-1, 1) and B(3, 3).\\
\solution \\
Given points,
\begin{align}
\Vec{A}=\begin{pmatrix} -1\\1 \end{pmatrix},\Vec{B}=\begin{pmatrix} 3\\3 \end{pmatrix},\Vec{P}=\begin{pmatrix} 0\\2 \end{pmatrix}
\end{align}
Mid point of $\Vec{A}$ and $\Vec{B}$, Let it be $\Vec{R}$
\begin{align}
    \Vec{R}=\frac{\vec{A}+\vec{B}}{2}
\end{align}
Slope, $\vec{m}$
\begin{align}
    \vec{m}=\vec{B}-\vec{A}\\
\end{align}
Let $\vec{n}$ be the direction vector perpendicular to $\vec{m}$, If truly $\vec{P}$ is y-intercept of bisector 
\begin{align}
\vec{n}=\vec{P}-\vec{R}
\end{align}
Both $\vec{n}$ and $\vec{m}$ are perpendicular
\begin{align}
    \vec{n}^T\vec{m}=0\\
    (\vec{P}-\vec{R})^T(\vec{B}-\vec{A})=0\\
    (\vec{P}^T-(\frac{\vec{A}+\vec{B}}{2})^T)(\vec{B}-\vec{A})=0\\
    \vec{P}^T(\vec{B}-\vec{A})-\frac{\brak{\vec{A}+\vec{B}}^T\brak{\vec{B}-\vec{A}}}{2}=0\\
    \begin{pmatrix} 0&2 \end{pmatrix}\begin{pmatrix} 4\\2 \end{pmatrix}-\frac{\begin{pmatrix} 2&4 \end{pmatrix}\begin{pmatrix} 4\\2 \end{pmatrix}}{2}=0\\
    4-\frac{16}{2}\neq0
\end{align}
Hence, $\vec{P}$ is not the y-intercept of perpendicular bisector of line $\vec{A}-\vec{B}$
\newpage
\begin{figure}[H]
    \centering
    \includegraphics[width=\columnwidth]{figs/figure.png}
    \caption{}
    \label{fig:placeholder}
\end{figure}
\end{document}
