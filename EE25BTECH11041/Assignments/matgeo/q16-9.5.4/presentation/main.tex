\documentclass{beamer}
\usepackage[utf8]{inputenc}

\usetheme{Madrid}
\usecolortheme{default}
\usepackage{amsmath,amssymb,amsfonts,amsthm}
\usepackage{txfonts}
\usepackage{tkz-euclide}
\usepackage{listings}
\usepackage{adjustbox}
\usepackage{array}
\usepackage{tabularx}
\usepackage{gvv}
\usepackage{lmodern}
\usepackage{circuitikz}
\usepackage{tikz}
\usepackage{graphicx}

\setbeamertemplate{page number in head/foot}[totalframenumber]

\usepackage{tcolorbox}
\tcbuselibrary{minted,breakable,xparse,skins}



\definecolor{bg}{gray}{0.95}
\DeclareTCBListing{mintedbox}{O{}m!O{}}{%
	breakable=true,
	listing engine=minted,
	listing only,
	minted language=#2,
	minted style=default,
	minted options={%
		linenos,
		gobble=0,
		breaklines=true,
		breakafter=,,
		fontsize=\small,
		numbersep=8pt,
		#1},
	boxsep=0pt,
	left skip=0pt,
	right skip=0pt,
	left=25pt,
	right=0pt,
	top=3pt,
	bottom=3pt,
	arc=5pt,
	leftrule=0pt,
	rightrule=0pt,
	bottomrule=2pt,
	toprule=2pt,
	colback=bg,
	colframe=orange!70,
	enhanced,
	overlay={%
		\begin{tcbclipinterior}
			\fill[orange!20!white] (frame.south west) rectangle ([xshift=20pt]frame.north west);
	\end{tcbclipinterior}},
	#3,
}
\lstset{
	language=C,
	basicstyle=\ttfamily\small,
	keywordstyle=\color{blue},
	stringstyle=\color{orange},
	commentstyle=\color{green!60!black},
	numbers=left,
	numberstyle=\tiny\color{gray},
	breaklines=true,
	showstringspaces=false,
}
\begin{document}

\title 
{9.5.4}
\date{1 Oct,2025}

\author 
{Naman Kumar-EE25BTECH11041}
\graphicspath{./figs}


\frame{\titlepage}
\begin{frame}{Question)}
If one zero of the polynomial $6x^2 + 37x-(k-2)$ is the reciprocal of the other, then what is the value of k?
\end{frame}
\begin{frame}{Solution}
General equation of conic
\begin{align}
    g(\vec{x})=\vec{x^T}\vec{V}\vec{x}+2\vec{u^T}\vec{x}+f
\end{align}
Equation of quadratic,
\begin{align}
    \vec{x^T}\begin{pmatrix}6&0\\0&0\end{pmatrix}\vec{x}+2\begin{pmatrix}\frac{37}{2}\\0\end{pmatrix}^T\vec{x}-(k-2)=0
\end{align}
\end{frame}
\begin{frame}{Solution}
Equation of line,
\begin{align}
    \vec{x}=\vec{h}+k\vec{m} \\
    \vec{h}=\begin{pmatrix}0\\0\end{pmatrix},\vec{m}=\begin{pmatrix}1\\ 0 \end{pmatrix}
\end{align}
Using following equation to find point of intersection of conic and line
\begin{align}
    k_i=\frac{1}{\vec{m}^T\vec{V}\vec{m}}\brak{-\vec{m}^T\brak{\vec{V}\vec{h}+\vec{u}} \pm \sqrt{\sbrak{\vec{m}^T\brak{\vec{V}\vec{h}+\vec{u}}}^2-g(\vec{h})(\vec{m}^T\vec{V}\vec{m})} } \label{main}
\end{align}
\end{frame}
\begin{frame}{Solution}
Solving for $g(\vec{h})$
\begin{align}
    g(\vec{h})=\vec{h^T}\begin{pmatrix}6&0\\0&0\end{pmatrix}\vec{h}+2\begin{pmatrix}\frac{37}{2}\\0\end{pmatrix}^T\vec{h}-(k-2) \\
    g(\vec{h})=-(k-2) \label{g}
\end{align}
Solving for $\vec{m}^T\vec{V}\vec{m}$
\begin{align}
    \vec{m}^T\vec{V}\vec{m}=\begin{pmatrix}1 \\0\end{pmatrix}^T\begin{pmatrix}6&0\\0&0\end{pmatrix}\begin{pmatrix}1 \\ 0\end{pmatrix}\\
    =6\label{mv}
\end{align}
\end{frame}
\begin{frame}{Solution}
Solving for $\vec{m}^T\brak{\vec{V}\vec{h}+\vec{u}}$
\begin{align}
    \begin{pmatrix}1 \\ 0\end{pmatrix}^T\brak{\begin{pmatrix}6&0\\0&0\end{pmatrix}\begin{pmatrix}0\\0\end{pmatrix}+\begin{pmatrix}\frac{37}{2}\\0\end{pmatrix}}\\
    =\frac{37}{2}
\end{align}
Solving $\eqref{main}$
\begin{align}
    k_i=\frac{1}{6}\brak{-\frac{37}{2} \pm \sqrt{\frac{1369}{4} +(k-2)\times 6}}
\end{align}
\end{frame}
\begin{frame}{Solution}
Given condition
\begin{align}
    k_1=\frac{1}{k_2}
\end{align}
Therefore
\begin{align}
    \frac{1}{6}\brak{-\frac{37}{2} - \sqrt{\frac{1369}{4} +(k-2)\times 6}}=\frac{1}{\frac{1}{6}\brak{-\frac{37}{2} + \sqrt{\frac{1369}{4} +(k-2)\times 6}}}
\end{align}
\begin{align}
    \frac{37}{2}^2-\brak{\frac{1369}{4}+6(k-2)}=36\\
    -6(k-2)=36\\
    k=-4
\end{align}
\end{frame}

\begin{frame}{Figure}
    \begin{figure}[H]
        \centering
        \includegraphics[width=0.9\columnwidth]{figs/figure.png}
        \caption{}
        \label{fig:placeholder}
    \end{figure}
\end{frame}
\begin{frame}[fragile]
\frametitle{Direct Python}
\begin{lstlisting}
import numpy as np
import matplotlib.pyplot as plt

x=np.linspace(-8,2,300)
y=6*x*x+37*x+6

plt.xlabel("X-axis")
plt.ylabel("Y-axis")
xp = np.array([-6, -1/6])
yp=np.array([0,0])
\end{lstlisting}
\end{frame}
\begin{frame}[fragile]
\frametitle{Direct Python}
\begin{lstlisting}
plt.axhline(y=0, color='black')
plt.axvline(x=0, color='black')

plt.grid()
plt.plot(x,y, label='Quadratic')

plt.scatter(xp,yp, label='roots', color='orange')
plt.legend()
plt.text(xp[0]+0.05, yp[0]+0.05, "(-6,0)")
plt.text(xp[1]+0.05, yp[1]+0.05, "(-1/6,0)")

plt.savefig("figure.png", dpi=300)
plt.show()
\end{lstlisting}
\end{frame}
\begin{frame}[fragile]
\frametitle{C code}
\begin{lstlisting}
#include <stdio.h>

double find_k() {
    double a = 6, b = 37, c; 
    double k;

    c = a;

    k = 2 - a;
    \end{lstlisting}
\end{frame}
\begin{frame}[fragile]
\frametitle{C code}
\begin{lstlisting}
    return k;
}

int main() {
    double k = find_k();
    printf("The value of k = %.2lf\n", k);
    return 0;
}
\end{lstlisting}
\end{frame}
\begin{frame}[fragile]
\frametitle{Python code with shared object}
\begin{lstlisting}
import ctypes
import numpy as np
import matplotlib.pyplot as plt

# Load shared library
# Compile C code using: gcc -shared -fPIC -o main.so main.c
so = ctypes.CDLL('./main.so')
so.find_k.restype = ctypes.c_double

# Get value of k from C function
k = so.find_k()
print(f"The value of k = {k}")
\end{lstlisting}
\end{frame}
\begin{frame}[fragile]
\frametitle{Python code with shared object}
\begin{lstlisting}
# Define polynomial: 6x^2 + 37x - (k - 2)
a, b, c = 6, 37, -(k - 2)

# Generate x values
x = np.linspace(-10, 2, 400)
y = a * x**2 + b * x + c

# Plot
plt.figure(figsize=(8,6))
plt.plot(x, y, label=r'$6x^2 + 37x - (k - 2)$', color='blue')
\end{lstlisting}
\end{frame}
\begin{frame}[fragile]
\frametitle{Python code with shared object}
\begin{lstlisting}
# X and Y axis lines
plt.axhline(0, color='black', linewidth=1)
plt.axvline(0, color='black', linewidth=1)

# Roots of polynomial
roots = np.roots([a, b, c])
plt.scatter(roots, [0, 0], color='red', zorder=5, label='Roots')

# Labels and Title
plt.title(f"Graph of 6x² + 37x - (k - 2),  where k = {k:.2f}")
plt.xlabel('x-axis')
plt.ylabel('y-axis')
plt.legend()
plt.grid(True)
plt.show()
\end{lstlisting}
\end{frame}

\end{document}