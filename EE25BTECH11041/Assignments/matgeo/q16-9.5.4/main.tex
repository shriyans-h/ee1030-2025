\let\negmedspace\undefined
\let\negthickspace\undefined
\documentclass[a5paper,10pt]{article}
\usepackage[margin=10mm]{geometry}
%\usepackage{lmodern} % Ensure lmodern is loaded for pdflatex
\usepackage{tfrupee} % Include tfrupee package

\setlength{\headheight}{1cm} % Set the height of the header box
\setlength{\headsep}{0mm}     % Set the distance between the header box and the top of the text

\usepackage{gvv-book}
\usepackage{gvv}
\usepackage{cite}
\usepackage{amsmath,amssymb,amsfonts,amsthm}
\usepackage{algorithmic}
\usepackage{graphicx}
\usepackage{textcomp}
\usepackage{xcolor}
\usepackage{txfonts}
\usepackage{listings}
\usepackage{enumitem}
\usepackage{mathtools}
\usepackage{gensymb}
\usepackage{comment}
\usepackage[breaklinks=true]{hyperref}
\usepackage{tkz-euclide} 
\usepackage{listings}
% \usepackage{gvv}                                        
\def\inputGnumericTable{}                                 
\usepackage[latin1]{inputenc}                                
\usepackage{color}                                            
\usepackage{array}                                            
\usepackage{longtable}                                       
\usepackage{calc}                                             
\usepackage{multirow}                                         
\usepackage{hhline}                                           
\usepackage{ifthen}                                           
\usepackage{lscape}
\usepackage{circuitikz}



\author{EE25BTECH11041-Naman Kumar }
\graphicspath{./figs/}

\begin{document}
\begin{center}
    \huge{9.5.4}\\
    \large{EE25BTECH11041 - Naman Kumar}
\end{center}
Question:\\
If one zero of the polynomial $6x^2 + 37x-(k-2)$ is the reciprocal of the other, then what is the value of k?\\
\solution \\
General equation of conic
\begin{align}
    g(\vec{x})=\vec{x^T}\vec{V}\vec{x}+2\vec{u^T}\vec{x}+f
\end{align}
Equation of quadratic,
\begin{align}
    \vec{x^T}\begin{pmatrix}6&0\\0&0\end{pmatrix}\vec{x}+2\begin{pmatrix}\frac{37}{2}\\0\end{pmatrix}^T\vec{x}-(k-2)=0
\end{align}
Equation of line,
\begin{align}
    \vec{x}=\vec{h}+k\vec{m} \\
    \vec{h}=\begin{pmatrix}0\\0\end{pmatrix},\vec{m}=\begin{pmatrix}1\\ 0 \end{pmatrix}
\end{align}
Using following equation to find point of intersection of conic and line
\begin{align}
    k_i=\frac{1}{\vec{m}^T\vec{V}\vec{m}}\brak{-\vec{m}^T\brak{\vec{V}\vec{h}+\vec{u}} \pm \sqrt{\sbrak{\vec{m}^T\brak{\vec{V}\vec{h}+\vec{u}}}^2-g(\vec{h})(\vec{m}^T\vec{V}\vec{m})} } \label{main}
\end{align}
Solving for $g(\vec{h})$
\begin{align}
    g(\vec{h})=\vec{h^T}\begin{pmatrix}6&0\\0&0\end{pmatrix}\vec{h}+2\begin{pmatrix}\frac{37}{2}\\0\end{pmatrix}^T\vec{h}-(k-2) \\
    g(\vec{h})=-(k-2) \label{g}
\end{align}
Solving for $\vec{m}^T\vec{V}\vec{m}$
\begin{align}
    \vec{m}^T\vec{V}\vec{m}=\begin{pmatrix}1 \\0\end{pmatrix}^T\begin{pmatrix}6&0\\0&0\end{pmatrix}\begin{pmatrix}1 \\ 0\end{pmatrix}\\
    =6\label{mv}
\end{align}
Solving for $\vec{m}^T\brak{\vec{V}\vec{h}+\vec{u}}$
\begin{align}
    \begin{pmatrix}1 \\ 0\end{pmatrix}^T\brak{\begin{pmatrix}6&0\\0&0\end{pmatrix}\begin{pmatrix}0\\0\end{pmatrix}+\begin{pmatrix}\frac{37}{2}\\0\end{pmatrix}}\\
    =\frac{37}{2}
\end{align}
Solving $\eqref{main}$
\begin{align}
    k_i=\frac{1}{6}\brak{-\frac{37}{2} \pm \sqrt{\frac{1369}{4} +(k-2)\times 6}}
\end{align}
Given condition
\begin{align}
    k_1=\frac{1}{k_2}
\end{align}
Therefore
\begin{align}
    \frac{1}{6}\brak{-\frac{37}{2} - \sqrt{\frac{1369}{4} +(k-2)\times 6}}=\frac{1}{\frac{1}{6}\brak{-\frac{37}{2} + \sqrt{\frac{1369}{4} +(k-2)\times 6}}}\\
    \frac{37}{2}^2-\brak{\frac{1369}{4}+6(k-2)}=36\\
    -6(k-2)=36\\
    k=-4
\end{align}
\begin{figure}[H]
    \centering
    \includegraphics[width=\columnwidth]{figs/figure.png}
    \caption{}
    \label{fig:placeholder}
\end{figure}
\end{document}
