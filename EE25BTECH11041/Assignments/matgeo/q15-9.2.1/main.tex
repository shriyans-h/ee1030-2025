\let\negmedspace\undefined
\let\negthickspace\undefined
\documentclass[a5paper,10pt]{article}
\usepackage[margin=10mm]{geometry}
%\usepackage{lmodern} % Ensure lmodern is loaded for pdflatex
\usepackage{tfrupee} % Include tfrupee package

\setlength{\headheight}{1cm} % Set the height of the header box
\setlength{\headsep}{0mm}     % Set the distance between the header box and the top of the text

\usepackage{gvv-book}
\usepackage{gvv}
\usepackage{cite}
\usepackage{amsmath,amssymb,amsfonts,amsthm}
\usepackage{algorithmic}
\usepackage{graphicx}
\usepackage{textcomp}
\usepackage{xcolor}
\usepackage{txfonts}
\usepackage{listings}
\usepackage{enumitem}
\usepackage{mathtools}
\usepackage{gensymb}
\usepackage{comment}
\usepackage[breaklinks=true]{hyperref}
\usepackage{tkz-euclide} 
\usepackage{listings}
% \usepackage{gvv}                                        
\def\inputGnumericTable{}                                 
\usepackage[latin1]{inputenc}                                
\usepackage{color}                                            
\usepackage{array}                                            
\usepackage{longtable}                                       
\usepackage{calc}                                             
\usepackage{multirow}                                         
\usepackage{hhline}                                           
\usepackage{ifthen}                                           
\usepackage{lscape}
\usepackage{circuitikz}



\author{EE25BTECH11041-Naman Kumar }
\graphicspath{./figs/}

\begin{document}
\begin{center}
    \huge{9.2.1}\\
    \large{EE25BTECH11041 - Naman Kumar}
\end{center}
Question:\\
Find the area bounded by the curve $y=\sqrt{x},x=2y+3$, in the first quadrant and
x-axis.
\solution \\
General equation of conic
\begin{align}
    g(\vec{x})=\vec{x^T}\vec{V}\vec{x}+2\vec{u^T}\vec{x}+f
\end{align}
Equation of parabola,
\begin{align}
    \vec{x^T}\begin{pmatrix}0&0\\0&1\end{pmatrix}\vec{x}+2\begin{pmatrix}-\frac{1}{2}\\0\end{pmatrix}^T\vec{x}=0
\end{align}
Equation of line,
\begin{align}
    \vec{x}=\vec{h}+k\vec{m} \\
    \vec{h}=\begin{pmatrix}0\\-\frac{3}{2}\end{pmatrix},\vec{m}=\begin{pmatrix}1\\ \frac{1}{2} \end{pmatrix}
\end{align}
Using following equation to find point of intersection of conic and line
\begin{align}
    k_i=\frac{1}{\vec{m}^T\vec{V}\vec{m}}\brak{-\vec{m}^T\brak{\vec{V}\vec{h}+\vec{u}} \pm \sqrt{\sbrak{\vec{m}^T\brak{\vec{V}\vec{h}+\vec{u}}}^2-g(\vec{h})(\vec{m}^T\vec{V}\vec{m})} } \label{main}
\end{align}
Solving for $g(\vec{h})$
\begin{align}
    g(\vec{h})=\vec{h^T}\begin{pmatrix}0&0\\0&1\end{pmatrix}\vec{h}+2\begin{pmatrix}-\frac{1}{2}\\0\end{pmatrix}^T\vec{h}\\
    g(\vec{h})=\frac{9}{4} \label{g}
\end{align}
Solving for $\vec{m}^T\vec{V}\vec{m}$
\begin{align}
    \vec{m}^T\vec{V}\vec{m}=\begin{pmatrix}1 \\ \frac{1}{2}\end{pmatrix}^T\begin{pmatrix}0&0\\0&1\end{pmatrix}\begin{pmatrix}1 \\ \frac{1}{2}\end{pmatrix}\\
    =\frac{1}{4}\label{mv}
\end{align}
Solving for $\vec{m}^T\brak{\vec{V}\vec{h}+\vec{u}}$
\begin{align}
    \begin{pmatrix}1 \\ \frac{1}{2}\end{pmatrix}^T\brak{\begin{pmatrix}0&0\\0&1\end{pmatrix}\begin{pmatrix}0\\-\frac{3}{2}\end{pmatrix}+\begin{pmatrix}-\frac{1}{2}\\0\end{pmatrix}}\\
    =-\frac{5}{4}
\end{align}
Solving $\eqref{main}$
\begin{align}
    k_i=\frac{1}{\frac{1}{4}}\brak{\frac{5}{4} \pm \sqrt{\frac{25}{16} - \frac{9}{4}\times \frac{1}{4}}}\\
    k_i=4\brak{\frac{5}{4} \pm 1}\\
    k_1=9,k_2=1
\end{align}
So with these values points are
\begin{align}
    \vec{x_1}=\begin{pmatrix}0\\-\frac{3}{2}\end{pmatrix}+9\times \begin{pmatrix}1\\\frac{1}{2}\end{pmatrix}\\
    \vec{x_1}= \begin{pmatrix}9\\ 3\end{pmatrix} \\
    \vec{x_2}=\begin{pmatrix}0\\-\frac{3}{2}\end{pmatrix}+1\times \begin{pmatrix}1\\\frac{1}{2}\end{pmatrix}\\
    \vec{x_2}= \begin{pmatrix}1\\ -1\end{pmatrix}
\end{align}
Area under curve in first quadrant between parabola and line
\begin{align}
    \int_0^{3}\sqrt{x}+\int_3^{9}\sqrt{x}-\brak{\frac{x-3}{2}}\\
    \sbrak{\frac{x^{\frac{3}{2}}}{\frac{3}{2}}}^9_0-\sbrak{\frac{x^2}{4}-\frac{3x}{2}}^9_3\\
    area=9
\end{align}
\begin{figure}[H]
    \centering
    \includegraphics[width=\columnwidth]{figs/figure.png}
    \caption{}
    \label{fig:placeholder}
\end{figure}

\end{document}
