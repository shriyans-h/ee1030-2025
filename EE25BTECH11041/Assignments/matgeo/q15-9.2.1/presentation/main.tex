\documentclass{beamer}
\usepackage[utf8]{inputenc}

\usetheme{Madrid}
\usecolortheme{default}
\usepackage{amsmath,amssymb,amsfonts,amsthm}
\usepackage{txfonts}
\usepackage{tkz-euclide}
\usepackage{listings}
\usepackage{adjustbox}
\usepackage{array}
\usepackage{tabularx}
\usepackage{gvv}
\usepackage{lmodern}
\usepackage{circuitikz}
\usepackage{tikz}
\usepackage{graphicx}

\setbeamertemplate{page number in head/foot}[totalframenumber]

\usepackage{tcolorbox}
\tcbuselibrary{minted,breakable,xparse,skins}



\definecolor{bg}{gray}{0.95}
\DeclareTCBListing{mintedbox}{O{}m!O{}}{%
	breakable=true,
	listing engine=minted,
	listing only,
	minted language=#2,
	minted style=default,
	minted options={%
		linenos,
		gobble=0,
		breaklines=true,
		breakafter=,,
		fontsize=\small,
		numbersep=8pt,
		#1},
	boxsep=0pt,
	left skip=0pt,
	right skip=0pt,
	left=25pt,
	right=0pt,
	top=3pt,
	bottom=3pt,
	arc=5pt,
	leftrule=0pt,
	rightrule=0pt,
	bottomrule=2pt,
	toprule=2pt,
	colback=bg,
	colframe=orange!70,
	enhanced,
	overlay={%
		\begin{tcbclipinterior}
			\fill[orange!20!white] (frame.south west) rectangle ([xshift=20pt]frame.north west);
	\end{tcbclipinterior}},
	#3,
}
\lstset{
	language=C,
	basicstyle=\ttfamily\small,
	keywordstyle=\color{blue},
	stringstyle=\color{orange},
	commentstyle=\color{green!60!black},
	numbers=left,
	numberstyle=\tiny\color{gray},
	breaklines=true,
	showstringspaces=false,
}
\begin{document}

\title 
{9.2.1}
\date{1 Oct,2025}

\author 
{Naman Kumar-EE25BTECH11041}
\graphicspath{./figs}


\frame{\titlepage}
\begin{frame}{Question)}
Find the area bounded by the curve $y=\sqrt{x},x=2y+3$, in the first quadrant and x-axis.
\end{frame}
\begin{frame}{Solution}
General equation of conic
\begin{align}
    g(\vec{x})=\vec{x^T}\vec{V}\vec{x}+2\vec{u^T}\vec{x}+f
\end{align}
Equation of parabola,
\begin{align}
    \vec{x^T}\begin{pmatrix}0&0\\0&1\end{pmatrix}\vec{x}+2\begin{pmatrix}-\frac{1}{2}\\0\end{pmatrix}^T\vec{x}=0
\end{align}
\end{frame}
\begin{frame}{Solution}
Equation of line,
\begin{align}
    \vec{x}=\vec{h}+k\vec{m} \\
    \vec{h}=\begin{pmatrix}0\\-\frac{3}{2}\end{pmatrix},\vec{m}=\begin{pmatrix}1\\ \frac{1}{2} \end{pmatrix}
\end{align}
Using following equation to find point of intersection of conic and line
\begin{align}
    k_i=\frac{1}{\vec{m}^T\vec{V}\vec{m}}\brak{-\vec{m}^T\brak{\vec{V}\vec{h}+\vec{u}} \pm \sqrt{\sbrak{\vec{m}^T\brak{\vec{V}\vec{h}+\vec{u}}}^2-g(\vec{h})(\vec{m}^T\vec{V}\vec{m})} } \label{main}
\end{align}
\end{frame}
\begin{frame}{Solution}
Solving for $g(\vec{h})$
\begin{align}
    g(\vec{h})=\vec{h^T}\begin{pmatrix}0&0\\0&1\end{pmatrix}\vec{h}+2\begin{pmatrix}-\frac{1}{2}\\0\end{pmatrix}^T\vec{h}\\
    g(\vec{h})=\frac{9}{4} \label{g}
\end{align}
Solving for $\vec{m}^T\vec{V}\vec{m}$
\begin{align}
    \vec{m}^T\vec{V}\vec{m}=\begin{pmatrix}1 \\ \frac{1}{2}\end{pmatrix}^T\begin{pmatrix}0&0\\0&1\end{pmatrix}\begin{pmatrix}1 \\ \frac{1}{2}\end{pmatrix}\\
    =\frac{1}{4}\label{mv}
\end{align}
\end{frame}
\begin{frame}{Solution}
Solving for $\vec{m}^T\brak{\vec{V}\vec{h}+\vec{u}}$
\begin{align}
    \begin{pmatrix}1 \\ \frac{1}{2}\end{pmatrix}^T\brak{\begin{pmatrix}0&0\\0&1\end{pmatrix}\begin{pmatrix}0\\-\frac{3}{2}\end{pmatrix}+\begin{pmatrix}-\frac{1}{2}\\0\end{pmatrix}}\\
    =-\frac{5}{4}
\end{align}
Solving $\eqref{main}$
\begin{align}
    k_i=\frac{1}{\frac{1}{4}}\brak{\frac{5}{4} \pm \sqrt{\frac{25}{16} - \frac{9}{4}\times \frac{1}{4}}}\\
    k_i=4\brak{\frac{5}{4} \pm 1}\\
    k_1=9,k_2=1
\end{align}
\end{frame}
\begin{frame}{Solution}
So with these values points are
\begin{align}
    \vec{x_1}=\begin{pmatrix}0\\-\frac{3}{2}\end{pmatrix}+9\times \begin{pmatrix}1\\\frac{1}{2}\end{pmatrix}\\
    \vec{x_1}= \begin{pmatrix}9\\ 3\end{pmatrix} \\
    \vec{x_2}=\begin{pmatrix}0\\-\frac{3}{2}\end{pmatrix}+1\times \begin{pmatrix}1\\\frac{1}{2}\end{pmatrix}\\
    \vec{x_2}= \begin{pmatrix}1\\ -1\end{pmatrix}
\end{align}
\end{frame}
\begin{frame}{Solution}
Area under curve in first quadrant between parabola and line
\begin{align}
    \int_0^{3}\sqrt{x}+\int_3^{9}\sqrt{x}-\brak{\frac{x-3}{2}}\\
    \sbrak{\frac{x^{\frac{3}{2}}}{\frac{3}{2}}}^9_0-\sbrak{\frac{x^2}{4}-\frac{3x}{2}}^9_3\\
    area=9
\end{align}
\end{frame}

\begin{frame}{Figure}
    \begin{figure}[H]
        \centering
        \includegraphics[width=0.9\columnwidth]{figs/figure.png}
        \caption{}
        \label{fig:placeholder}
    \end{figure}
\end{frame}
\begin{frame}[fragile]
\frametitle{Direct Python}
\begin{lstlisting}
import numpy as np
import matplotlib.pyplot as plt
import math

y = np.linspace(-2,4,300)
x = y*y
xl = np.linspace(0,10,300)
yl = (xl-3)/2

x1 = np.linspace(0,3, 200)
y1 = np.sqrt(x1)
x2 = np.linspace(3,9, 200)
y2 = np.sqrt(x2)
yl2 = (x2-3)/2

\end{lstlisting}
\end{frame}
\begin{frame}[fragile]
\frametitle{Direct Python}
\begin{lstlisting}
plt.fill_between(x1, y1, 0, color='skyblue', alpha=0.4) 
plt.fill_between(x2,y2, yl2, color='skyblue', alpha=0.4)
plt.annotate('Intersection (9, 3)', xy=(9, 3), xytext=(7, 4),
             arrowprops=dict(facecolor='black', shrink=0.05))
plt.xlabel('x-axis', fontsize=12)
plt.ylabel('y-axis', fontsize=12)
plt.plot(x,y, label="parabola")
plt.plot(xl,yl, label="line")
plt.grid()
plt.legend()

plt.savefig("figure.png", dpi=300)

plt.show()
\end{lstlisting}
\end{frame}
\begin{frame}[fragile]
\frametitle{C code}
\begin{lstlisting}
#include <stdio.h>

double area_bounded() {
    double y1 = 0, y2 = 3;
    double area;


\end{lstlisting}
\end{frame}
\begin{frame}[fragile]
\frametitle{C code}
\begin{lstlisting}
    area = (y2 * y2 + 3 * y2 - (y2 * y2 * y2) / 3.0) -
           (y1 * y1 + 3 * y1 - (y1 * y1 * y1) / 3.0);
    return area;
}
int main() {
    printf("Area bounded = %.2f\n", area_bounded());
    return 0;
}
\end{lstlisting}
\end{frame}
\begin{frame}[fragile]
\frametitle{Python code with shared object}
\begin{lstlisting}
import ctypes
import matplotlib.pyplot as plt
import numpy as np

lib = ctypes.CDLL('./area.so')
lib.area_bounded.restype = ctypes.c_double

area = lib.area_bounded()
print("Area bounded =", area)

\end{lstlisting}
\end{frame}
\begin{frame}[fragile]
\frametitle{Python code with shared object}
\begin{lstlisting}
y = np.linspace(0, 3, 100)
x1 = y**2          # x = y^2  (y = sqrt(x))
x2 = 2*y + 3       # x = 2y + 3

plt.plot(x1, y, label='y = sqrt(x) → x = y²')
plt.plot(x2, y, label='x = 2y + 3')

plt.fill_betweenx(y, x1, x2, color='lightblue', alpha=0.5)
plt.xlabel("x")
plt.ylabel("y")
plt.title(f"Area bounded by curves = {area:.2f}")
plt.legend()
plt.grid(True)
plt.show()
\end{lstlisting}
\end{frame}

\end{document}