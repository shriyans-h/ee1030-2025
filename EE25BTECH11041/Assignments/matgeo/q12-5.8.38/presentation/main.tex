\documentclass{beamer}
\usepackage[utf8]{inputenc}

\usetheme{Madrid}
\usecolortheme{default}
\usepackage{amsmath,amssymb,amsfonts,amsthm}
\usepackage{txfonts}
\usepackage{tkz-euclide}
\usepackage{listings}
\usepackage{adjustbox}
\usepackage{array}
\usepackage{tabularx}
\usepackage{gvv}
\usepackage{lmodern}
\usepackage{circuitikz}
\usepackage{tikz}
\usepackage{graphicx}

\setbeamertemplate{page number in head/foot}[totalframenumber]

\usepackage{tcolorbox}
\tcbuselibrary{minted,breakable,xparse,skins}



\definecolor{bg}{gray}{0.95}
\DeclareTCBListing{mintedbox}{O{}m!O{}}{%
	breakable=true,
	listing engine=minted,
	listing only,
	minted language=#2,
	minted style=default,
	minted options={%
		linenos,
		gobble=0,
		breaklines=true,
		breakafter=,,
		fontsize=\small,
		numbersep=8pt,
		#1},
	boxsep=0pt,
	left skip=0pt,
	right skip=0pt,
	left=25pt,
	right=0pt,
	top=3pt,
	bottom=3pt,
	arc=5pt,
	leftrule=0pt,
	rightrule=0pt,
	bottomrule=2pt,
	toprule=2pt,
	colback=bg,
	colframe=orange!70,
	enhanced,
	overlay={%
		\begin{tcbclipinterior}
			\fill[orange!20!white] (frame.south west) rectangle ([xshift=20pt]frame.north west);
	\end{tcbclipinterior}},
	#3,
}
\lstset{
	language=C,
	basicstyle=\ttfamily\small,
	keywordstyle=\color{blue},
	stringstyle=\color{orange},
	commentstyle=\color{green!60!black},
	numbers=left,
	numberstyle=\tiny\color{gray},
	breaklines=true,
	showstringspaces=false,
}
\begin{document}

\title 
{5.8.38}
\date{26 September,2025}

\author 
{Naman Kumar-EE25BTECH11041}
\graphicspath{./figs}


\frame{\titlepage}
\begin{frame}{Question)}
Alwar tells his daughter,``Seven years ago, I was seven times as old as you were then. Also, three years from now, I shall be three times as old as you will be''\ Represent this situation algebraically and graphically.
\end{frame}
\begin{frame}{Solution}
\begin{tabular}[12pt]{ |c| c|}
    \hline
    \textbf{Name} & \textbf{Point}\\ 
    \hline
	Point A &\myvec{h \\ k}\\
    \hline 
 Point B &\myvec{x1 \\ y1}\\
    \hline
	  Point R &\myvec{x2 \\ y2}\\
    \hline
    
    \end{tabular}

from given statements, we can write
\begin{align}
    y-7=7\times(x-7)\And y+3=3\times(x+3)\\
    7x-y=42\And3x-y=-6
\end{align}
This can be written as
\begin{align}
\begin{pmatrix}7&-1\\3&-1\end{pmatrix}\begin{pmatrix}x\\y\end{pmatrix}=\begin{pmatrix}42\\-6\end{pmatrix}\\
\vec{A}\vec{x}=\vec{c}
\end{align}
\end{frame}
\begin{frame}{Solution}
Row reduced form of $\sbrak{\vec{A}\vert \vec{I}}$
\begin{align}
    \begin{amatrix}{2}7&-1&42\\3&-1&-6\end{amatrix}\xrightarrow[]{R_2-\frac{3}{7}R_1}\begin{amatrix}{2}7&-1&42\\0&-\frac{4}{7}&-24\end{amatrix}\\
    \xrightarrow[]{\frac{-7}{4}R_2}\begin{amatrix}{2}7&-1&42\\0&1&42\end{amatrix}
\end{align}
Therefore
\begin{align}
    \begin{pmatrix}7&-1\\0&1\end{pmatrix}\begin{pmatrix}x\\y\end{pmatrix}=\begin{pmatrix}42\\42\end{pmatrix}\\
    \begin{pmatrix}x\\y\end{pmatrix}=\begin{pmatrix}12\\42\end{pmatrix}
\end{align}
\end{frame}
\begin{frame}{Figure}
\begin{figure}[H]
    \centering
    \includegraphics[width=0.6\columnwidth]{figs/figure.png}
    \caption{}
    \label{fig:placeholder}
\end{figure}
\end{frame}
\begin{frame}[fragile]
\frametitle{Direct Python}
\begin{lstlisting}
import numpy as np
import matplotlib.pyplot as plt

c=np.array([42,-6])
a= np.array([[7,-1],[3,-1]])

inv=np.linalg.inv(a)

ans=np.dot(inv,c)

print("x=", ans[0],"y=",ans[1])

x=np.linspace(9,15,200)
l1=7*x-42
l2=3*x+6
\end{lstlisting}
\end{frame}
\begin{frame}[fragile]
\frametitle{Direct python code}
\begin{lstlisting}
plt.plot(x,l1, label="line 1, 7x-y=42")
plt.plot(x,l2,label="line 2, 3x-y=-6")
plt.scatter(12,42, c='r', zorder=5)
plt.text(12.3,41, "(x,y)")
plt.xlabel("daughter")
plt.ylabel("father")
plt.legend()
plt.grid()
plt.savefig("figure.png", dpi=200)
plt.show()

\end{lstlisting}
\end{frame}
\begin{frame}[fragile]
\frametitle{C code}
\begin{lstlisting}
#include <stdio.h>

typedef struct {
    double A; // Alwar's age
    double D; // Daughter's age
} Solution;

Solution solve_ages() {
    Solution S;

    // Coefficients:
    // eqn1: 1*A -7*D = -42
    // eqn2: 1*A -3*D = 6
    double a1 = 1, b1 = -7, c1 = -42;
    double a2 = 1, b2 = -3, c2 = 6;
\end{lstlisting}
\end{frame}
\begin{frame}[fragile]
\frametitle{C code}
\begin{lstlisting}
    // Solve using Cramer's rule
    double det  = a1*b2 - a2*b1;
    double detA = c1*b2 - c2*b1;
    double detD = a1*c2 - a2*c1;

    S.A = detA / det;
    S.D = detD / det;

    return S;
}\end{lstlisting}
\end{frame}
\begin{frame}[fragile]
\frametitle{C code}
\begin{lstlisting}

#ifdef TEST_C
int main(){
    Solution S = solve_ages();
    printf("Alwar's age: %.2f\n", S.A);
    printf("Daughter's age: %.2f\n", S.D);
    return 0;
}
#endif
\end{lstlisting}
\end{frame}
\begin{frame}[fragile]
\frametitle{Python code with shared object}
\begin{lstlisting}
# main.py
import ctypes
from ctypes import Structure, c_double
import numpy as np
import matplotlib.pyplot as plt

class Solution(Structure):
    _fields_ = [("A", c_double), ("D", c_double)]

# Load the shared object
lib = ctypes.CDLL("./libsolver.so")
lib.solve_ages.restype = Solution
\end{lstlisting}
\end{frame}
\begin{frame}[fragile]
\frametitle{Python code with shared object}
\begin{lstlisting}
# Call C function
sol = lib.solve_ages()
print(f"Alwar's present age: {sol.A}")
print(f"Daughter's present age: {sol.D}")

# Graphical representation
D_vals = np.linspace(0, 30, 400)

# From eqn (1): A = 7D - 42
A1 = 7*D_vals - 42
# From eqn (2): A = 3D + 6
A2 = 3*D_vals + 6
\end{lstlisting}
\end{frame}
\begin{frame}[fragile]
\frametitle{Python code with shared object}
\begin{lstlisting}
plt.figure(figsize=(6,6))
plt.plot(D_vals, A1, label="A - 7D = -42")
plt.plot(D_vals, A2, label="A - 3D = 6")

# Plot solution from C
plt.scatter(sol.D, sol.A, c="red", zorder=5)
plt.text(sol.D+0.5, sol.A, f"({int(sol.D)}, {int(sol.A)})", fontsize=10)

plt.xlabel("Daughter's Present Age (D)")
plt.ylabel("Alwar's Present Age (A)")
plt.title("Graphical Solution of Age Problem")
plt.grid(True)
plt.legend()
plt.show()
\end{lstlisting}
\end{frame}


\end{document}