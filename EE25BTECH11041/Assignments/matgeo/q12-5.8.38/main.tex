\let\negmedspace\undefined
\let\negthickspace\undefined
\documentclass[journal]{IEEEtran}
\usepackage[a5paper, margin=10mm, onecolumn]{geometry}
%\usepackage{lmodern} % Ensure lmodern is loaded for pdflatex
\usepackage{tfrupee} % Include tfrupee package

\setlength{\headheight}{1cm} % Set the height of the header box
\setlength{\headsep}{0mm}     % Set the distance between the header box and the top of the text

\usepackage{gvv-book}
\usepackage{gvv}
\usepackage{cite}
\usepackage{amsmath,amssymb,amsfonts,amsthm}
\usepackage{algorithmic}
\usepackage{graphicx}
\usepackage{textcomp}
\usepackage{xcolor}
\usepackage{txfonts}
\usepackage{listings}
\usepackage{enumitem}
\usepackage{mathtools}
\usepackage{gensymb}
\usepackage{comment}
\usepackage[breaklinks=true]{hyperref}
\usepackage{tkz-euclide} 
\usepackage{listings}
% \usepackage{gvv}                                        
\def\inputGnumericTable{}                                 
\usepackage[latin1]{inputenc}                                
\usepackage{color}                                            
\usepackage{array}                                            
\usepackage{longtable}                                       
\usepackage{calc}                                             
\usepackage{multirow}                                         
\usepackage{hhline}                                           
\usepackage{ifthen}                                           
\usepackage{lscape}
\usepackage{circuitikz}



\author{EE25BTECH11041-Naman Kumar }
\graphicspath{./figs/}

\begin{document}
\begin{center}
    \huge{5.8.38}\\
    \large{EE25BTECH11041 - Naman Kumar}
\end{center}
Question:\\
Alwar tells his daughter,``Seven years ago, I was seven times as old as you were then. Also, three years from now, I shall be three times as old as you will be''\ Represent this situation algebraically and graphically.\\
\solution \\
We have,
\begin{tabular}[12pt]{ |c| c|}
    \hline
    \textbf{Name} & \textbf{Point}\\ 
    \hline
	Point A &\myvec{h \\ k}\\
    \hline 
 Point B &\myvec{x1 \\ y1}\\
    \hline
	  Point R &\myvec{x2 \\ y2}\\
    \hline
    
    \end{tabular}

from given statements, we can write
\begin{align}
    y-7=7\times(x-7)\And y+3=3\times(x+3)\\
    7x-y=42\And3x-y=-6
\end{align}
This can be written as
\begin{align}
\begin{pmatrix}7&-1\\3&-1\end{pmatrix}\begin{pmatrix}x\\y\end{pmatrix}=\begin{pmatrix}42\\-6\end{pmatrix}\\
\vec{A}\vec{x}=\vec{c}
\end{align}
Row reduced form of $\sbrak{\vec{A}\vert \vec{I}}$
\begin{align}
    \begin{amatrix}{2}7&-1&42\\3&-1&-6\end{amatrix}\xrightarrow[]{R_2-\frac{3}{7}R_1}\begin{amatrix}{2}7&-1&42\\0&-\frac{4}{7}&-24\end{amatrix}\\
    \xrightarrow[]{\frac{-7}{4}R_2}\begin{amatrix}{2}7&-1&42\\0&1&42\end{amatrix}
\end{align}
Therefore
\begin{align}
    \begin{pmatrix}7&-1\\0&1\end{pmatrix}\begin{pmatrix}x\\y\end{pmatrix}=\begin{pmatrix}42\\42\end{pmatrix}\\
    \begin{pmatrix}x\\y\end{pmatrix}=\begin{pmatrix}12\\42\end{pmatrix}
\end{align}
\begin{figure}[H]
    \centering
    \includegraphics[width=\columnwidth]{figs/figure.png}
    \caption{}
    \label{fig:placeholder}
\end{figure}
\end{document}
