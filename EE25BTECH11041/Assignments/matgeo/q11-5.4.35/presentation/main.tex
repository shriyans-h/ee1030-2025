\documentclass{beamer}
\usepackage[utf8]{inputenc}

\usetheme{Madrid}
\usecolortheme{default}
\usepackage{amsmath,amssymb,amsfonts,amsthm}
\usepackage{txfonts}
\usepackage{tkz-euclide}
\usepackage{listings}
\usepackage{adjustbox}
\usepackage{array}
\usepackage{tabularx}
\usepackage{gvv}
\usepackage{lmodern}
\usepackage{circuitikz}
\usepackage{tikz}
\usepackage{graphicx}

\setbeamertemplate{page number in head/foot}[totalframenumber]

\usepackage{tcolorbox}
\tcbuselibrary{minted,breakable,xparse,skins}



\definecolor{bg}{gray}{0.95}
\DeclareTCBListing{mintedbox}{O{}m!O{}}{%
	breakable=true,
	listing engine=minted,
	listing only,
	minted language=#2,
	minted style=default,
	minted options={%
		linenos,
		gobble=0,
		breaklines=true,
		breakafter=,,
		fontsize=\small,
		numbersep=8pt,
		#1},
	boxsep=0pt,
	left skip=0pt,
	right skip=0pt,
	left=25pt,
	right=0pt,
	top=3pt,
	bottom=3pt,
	arc=5pt,
	leftrule=0pt,
	rightrule=0pt,
	bottomrule=2pt,
	toprule=2pt,
	colback=bg,
	colframe=orange!70,
	enhanced,
	overlay={%
		\begin{tcbclipinterior}
			\fill[orange!20!white] (frame.south west) rectangle ([xshift=20pt]frame.north west);
	\end{tcbclipinterior}},
	#3,
}
\lstset{
	language=C,
	basicstyle=\ttfamily\small,
	keywordstyle=\color{blue},
	stringstyle=\color{orange},
	commentstyle=\color{green!60!black},
	numbers=left,
	numberstyle=\tiny\color{gray},
	breaklines=true,
	showstringspaces=false,
}
\begin{document}

\title 
{5.4.35}
\date{26 September,2025}

\author 
{Naman Kumar-EE25BTECH11041}
\graphicspath{./figs}


\frame{\titlepage}
\begin{frame}{Question)}
Find inverse with elementary transformations of matrix
\begin{align}
\begin{pmatrix}0&1&2\\-1&0&-3\\-2&3&0\end{pmatrix}
\end{align}
\end{frame}
\begin{frame}{Solution}
For elementary transformation, matrix can be written in form
\begin{align}
    \sbrak{\begin{array}{ccc|ccc}
        0  & 1 & 2 & 1 & 0 & 0\\
        -1 & 0 &-3 & 0 & 1 & 0\\
        -2 & 3 & 0 & 0 & 0 & 1
    \end{array}} \label{1}
\end{align}
Here, it is in form
\begin{align}
    \sbrak{\vec{A}\vert \vec{I}}
\end{align}
With elementary transformation, we get
\begin{align}
    \sbrak{\vec{I}\vert \vec{A}^{-1}}
\end{align}
\end{frame}
\begin{frame}{Solution}
So now in $\eqref{1}$
\begin{align}
    \sbrak{\begin{array}{ccc|ccc}
        0  & 1 & 2 & 1 & 0 & 0\\
        -1 & 0 &-3 & 0 & 1 & 0\\
        -2 & 3 & 0 & 0 & 0 & 1
    \end{array}}
\end{align}
But before that check determinant of $\vec{A}$
\begin{align}
    \begin{vmatrix}
         0  & 1 & 2\\
        -1 & 0 & -3\\
        -2 & 3 & 0
    \end{vmatrix}
    \\
    0(0+9)-1(0-6)+2(-3-0)\\
    0+6-6=0
\end{align}
Since determinant is zero , No inverse exists
\end{frame}
\begin{frame}[fragile]
\frametitle{Direct Python}
\begin{lstlisting}
import numpy as np

a= np.array([[0,1,2],[-1,0,-3],[-2,3,0]])
det= np.linalg.det(a)

if det==0:
    print("No inverse exist")
else:
    inv = np.linalg.inv(a)
    print(inv)
\end{lstlisting}
\end{frame}

\end{document}