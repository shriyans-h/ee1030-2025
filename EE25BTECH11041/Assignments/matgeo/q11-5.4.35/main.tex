\let\negmedspace\undefined
\let\negthickspace\undefined
\documentclass[journal]{IEEEtran}
\usepackage[a5paper, margin=10mm, onecolumn]{geometry}
%\usepackage{lmodern} % Ensure lmodern is loaded for pdflatex
\usepackage{tfrupee} % Include tfrupee package

\setlength{\headheight}{1cm} % Set the height of the header box
\setlength{\headsep}{0mm}     % Set the distance between the header box and the top of the text

\usepackage{gvv-book}
\usepackage{gvv}
\usepackage{cite}
\usepackage{amsmath,amssymb,amsfonts,amsthm}
\usepackage{algorithmic}
\usepackage{graphicx}
\usepackage{textcomp}
\usepackage{xcolor}
\usepackage{txfonts}
\usepackage{listings}
\usepackage{enumitem}
\usepackage{mathtools}
\usepackage{gensymb}
\usepackage{comment}
\usepackage[breaklinks=true]{hyperref}
\usepackage{tkz-euclide} 
\usepackage{listings}
% \usepackage{gvv}                                        
\def\inputGnumericTable{}                                 
\usepackage[latin1]{inputenc}                                
\usepackage{color}                                            
\usepackage{array}                                            
\usepackage{longtable}                                       
\usepackage{calc}                                             
\usepackage{multirow}                                         
\usepackage{hhline}                                           
\usepackage{ifthen}                                           
\usepackage{lscape}
\usepackage{circuitikz}



\author{EE25BTECH11041-Naman Kumar }
\graphicspath{./figs/}

\begin{document}
\begin{center}
    \huge{5.4.35}\\
    \large{EE25BTECH11041 - Naman Kumar}
\end{center}
Question:\\
Find inverse with elementary transformations of matrix
\begin{align}
\begin{pmatrix}0&1&2\\-1&0&-3\\-2&3&0\end{pmatrix}
\end{align} \\
\solution \\
For elementary transformation, matrix can be written in form
\begin{align}
    \sbrak{\begin{array}{ccc|ccc}
        0  & 1 & 2 & 1 & 0 & 0\\
        -1 & 0 &-3 & 0 & 1 & 0\\
        -2 & 3 & 0 & 0 & 0 & 1
    \end{array}} \label{1}
\end{align}
Here, it is in form
\begin{align}
    \sbrak{\vec{A}\vert \vec{I}}
\end{align}
With elementary transformation, we get
\begin{align}
    \sbrak{\vec{I}\vert \vec{A}^{-1}}
\end{align}
So now in $\eqref{1}$
\begin{align}
    \sbrak{\begin{array}{ccc|ccc}
        0  & 1 & 2 & 1 & 0 & 0\\
        -1 & 0 &-3 & 0 & 1 & 0\\
        -2 & 3 & 0 & 0 & 0 & 1
    \end{array}}\\ \xrightarrow[]{R_1\rightarrow R_1-R_2}
    \sbrak{\begin{array}{ccc|ccc}
        1  & 1 & 5 & 1 & -1 & 0\\
        -1 & 0 &-3 & 0 & 1 & 0\\
        -2 & 3 & 0 & 0 & 0 & 1
    \end{array}}\\ \xrightarrow[]{R_2\rightarrow R_1+R_2}
    \sbrak{\begin{array}{ccc|ccc}
        1  & 1 & 5 & 1 & -1 & 0\\
        0 & 1 &2 & 1 & 0 & 0\\
        -2 & 3 & 0 & 0 & 0 & 1
    \end{array}}\\ \xrightarrow[]{R_1\rightarrow R_1-R_2}
    \sbrak{\begin{array}{ccc|ccc}
        1  & 0 & 3 & 0 & -1 & 0\\
        0 & 1 &2 & 1 & 0 & 0\\
        -2 & 3 & 0 & 0 & 0 & 1
    \end{array}}\\ \xrightarrow[]{R_3\rightarrow R_3+2R_1}
    \sbrak{\begin{array}{ccc|ccc}
        1  & 0 & 3 & 0 & -1 & 0\\
        0 & 1 &2 & 1 & 0 & 0\\
        0 & 3 & 6 & 0 & -2 & 1
    \end{array}}\\ \xrightarrow[]{R_3\rightarrow R_3-3R_2}
    \sbrak{\begin{array}{ccc|ccc}
        1  & 0 & 3 & 0 & -1 & 0\\
        0 & 1 &2 & 1 & 0 & 0\\
        0 & 0 & 0 & -3 & -2 & 1
    \end{array}}
\end{align}
At this point, we have obtained a row of all zeros on the left side of the augmented matrix. It's now impossible to continue the process to form the identity matrix.\\
Because the RREF of the original matrix is not the identity matrix, the matrix is singular and its inverse does not exist.
\end{document}
