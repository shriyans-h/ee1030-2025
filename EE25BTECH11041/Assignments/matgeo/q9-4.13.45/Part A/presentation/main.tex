\documentclass{beamer}
\usepackage[utf8]{inputenc}

\usetheme{Madrid}
\usecolortheme{default}
\usepackage{amsmath,amssymb,amsfonts,amsthm}
\usepackage{txfonts}
\usepackage{tkz-euclide}
\usepackage{listings}
\usepackage{adjustbox}
\usepackage{array}
\usepackage{tabularx}
\usepackage{gvv}
\usepackage{lmodern}
\usepackage{circuitikz}
\usepackage{tikz}
\usepackage{graphicx}

\setbeamertemplate{page number in head/foot}[totalframenumber]

\usepackage{tcolorbox}
\tcbuselibrary{minted,breakable,xparse,skins}



\definecolor{bg}{gray}{0.95}
\DeclareTCBListing{mintedbox}{O{}m!O{}}{%
	breakable=true,
	listing engine=minted,
	listing only,
	minted language=#2,
	minted style=default,
	minted options={%
		linenos,
		gobble=0,
		breaklines=true,
		breakafter=,,
		fontsize=\small,
		numbersep=8pt,
		#1},
	boxsep=0pt,
	left skip=0pt,
	right skip=0pt,
	left=25pt,
	right=0pt,
	top=3pt,
	bottom=3pt,
	arc=5pt,
	leftrule=0pt,
	rightrule=0pt,
	bottomrule=2pt,
	toprule=2pt,
	colback=bg,
	colframe=orange!70,
	enhanced,
	overlay={%
		\begin{tcbclipinterior}
			\fill[orange!20!white] (frame.south west) rectangle ([xshift=20pt]frame.north west);
	\end{tcbclipinterior}},
	#3,
}
\lstset{
	language=C,
	basicstyle=\ttfamily\small,
	keywordstyle=\color{blue},
	stringstyle=\color{orange},
	commentstyle=\color{green!60!black},
	numbers=left,
	numberstyle=\tiny\color{gray},
	breaklines=true,
	showstringspaces=false,
}
\begin{document}

\title 
{4.13.45 }
\date{25 September,2025}

\author 
{Naman Kumar-EE25BTECH11041}
\graphicspath{./figs}


\frame{\titlepage}
\begin{frame}{Question a)}
Two vertices of a triangle are $(5, -1)$ and $(2, -3)$. If the orthocentre of the triangle
is the origin, find the coordinates of the third point.
\end{frame}
\begin{frame}{Solution}
Given,
\begin{align}
\vec{O}=\begin{pmatrix}0\\0\end{pmatrix},\vec{A}=\begin{pmatrix}5\\-1\end{pmatrix},\vec{B}=\begin{pmatrix}2\\-3\end{pmatrix}
\end{align}
Where,
\begin{tabular}[12pt]{ |c| c|}
    \hline
    \textbf{Name} & \textbf{Point}\\ 
    \hline
	Point A &\myvec{h \\ k}\\
    \hline 
 Point B &\myvec{x1 \\ y1}\\
    \hline
	  Point R &\myvec{x2 \\ y2}\\
    \hline
    
    \end{tabular}

\end{frame}
\begin{frame}{Solution}
From General Triangle Properties.
\begin{align}
    m_1^{T}A_1=0,m_2^{T}A_2=0\\
    (\vec{A}-\vec{O})^{T}(\vec{C}-\vec{B})=0;\\
    \vec{A}^{T}(\vec{C}-\vec{B})=0, \label{1}\text{Since $\vec{O}$ is origin}
\end{align}
and 
\begin{align}
    (\vec{B}-\vec{O})^{T}(\vec{C}-\vec{A})=0;\\
    (\vec{B})^{T}(\vec{C}-\vec{A})=0,\label{2}\text{Since $\vec{O}$ is origin}
\end{align}
\end{frame}
\begin{frame}{Solution}
Modifying $\eqref{1}$ and $\eqref{2}$
\begin{align}
\vec{A}^T\vec{C}-\vec{A}^T\vec{B}=0,\vec{B}^T\vec{C}-\vec{B}^T\vec{A}=0\\
\vec{A}^T\vec{C}=\vec{A}^T\vec{B} \label{3}\\ \vec{B}^T\vec{C}=\vec{B}^T\vec{A} \label{4}
\end{align}
This can be written as
\begin{align}
    \begin{pmatrix}\vec{A}^T \\ \vec{B}^T \end{pmatrix}\vec{C}=\begin{pmatrix}\vec{A}^T\vec{B}\\ \vec{B}^T\vec{A} \end{pmatrix}\\
    \begin{pmatrix}5 & -1\\ 2 & -3\end{pmatrix}\vec{C}=\begin{pmatrix}\vec{A}^T\vec{B}\\ \vec{B}^T\vec{A} \end{pmatrix} \label{5}
\end{align}
\end{frame}
\begin{frame}{Solution}
Let
\begin{align}
    \begin{amatrix}{2}5 & -1& \vec{A}^T\vec{B}\\ 2 & -3 &\vec{B}^T\vec{A} \end{amatrix} = \begin{amatrix}{2}5 & -1& 13\\ 2 & -3 &13 \end{amatrix}
\end{align}
By Gaussian Elimination
\begin{align}
    \begin{amatrix}{2}5 & -1& 13\\ 2 & -3 &13 \end{amatrix}\xrightarrow{R_2-\frac{2}{5}R_1}\begin{amatrix}{2}5 & -1& 13\\ 0 & \frac{-13}{5} &\frac{39}{5} \end{amatrix}\\
    \xrightarrow{-\frac{5}{13}R_2}\begin{amatrix}{2}5 & -1& 13\\ 0 & 1 & -3 \end{amatrix}
\end{align}
\end{frame}
\begin{frame}{Solution}
In equation $\eqref{5}$ 
\begin{align}
    \begin{pmatrix}5 & -1\\ 0 & 1\end{pmatrix}\begin{pmatrix}x\\ y\end{pmatrix}=\begin{pmatrix}13\\ -3\end{pmatrix}
\end{align}
Therefore, $\vec{C}$ is
\begin{align}
    \begin{pmatrix}x\\y\end{pmatrix} = \begin{pmatrix}2\\-3\end{pmatrix}
\end{align}
\end{frame}
\begin{frame}{Solution}
But,Now
\begin{align}
    \vec{B}=\vec{C}
\end{align}
Which is not possible for a triangle,\\
Slope of line $\vec{A}$ to $\vec{B}$ or $\vec{L}$ be $\vec{m}$
\begin{align}
\vec{m}=\vec{A}-\vec{B}\\
\vec{m}=\begin{pmatrix}3\\2\end{pmatrix}
\end{align}
\end{frame}
\begin{frame}{Solution}
We can see, that
\begin{align}
    \vec{m}^T\vec{A_2}=0,\vec{m_2}^T\vec{A_2}=0
\end{align}
Therefore, for this to be possible when
\begin{align}
    \vec{L} \parallel \vec{m_2}, \\ \vec{A}-\vec{C} \parallel \vec{A}-\vec{B}
\end{align}

Since both lines have a point in common $\vec{A}$, therefore they must be collinear.\\
So, $\vec{A},\vec{B}$ and $\vec{C}$ is just a straight line
\end{frame}
\begin{frame}{Figure}
\begin{figure}[H]
    \centering
    \includegraphics[width=0.6\columnwidth]{figs/Figure.png}
    \caption{}
    \label{fig:placeholder}
\end{figure}
\end{frame}



\begin{frame}[fragile]
\frametitle{direct python}
\begin{lstlisting}
import numpy as np
import matplotlib.pyplot as plt

plt.figure(figsize=(6,6), dpi = 200)


abX= np.array([5,2])
abY= np.array([-1,-3])
acX= np.array([5,2])
acY=np.array([-1,-3])
bcX=np.array([2,2])
bcY=np.array([-3,-3])
obx=np.array([0,2])
oby=np.array([0,-3])
\end{lstlisting}
\end{frame}
\begin{frame}[fragile]
\frametitle{direct python}
\begin{lstlisting}

plt.plot(abX,abY, ':r',marker='o', label="Line AB")
plt.plot(bcX,bcY, ':b',marker='o', label="Line BC")
plt.plot(acX,acY, ':',marker='o', color='orange', label="Line AC")
plt.plot(obx,oby, '-', color='pink', label="Line OB")


plt.annotate("A",xy=(abX[0], abY[0]))
plt.annotate("B",xy=(abX[1]+0.05, abY[1]-0.02))
plt.annotate("C",xy=(abX[1]+0.05, abY[1]+0.1))
\end{lstlisting}
\end{frame}
\begin{frame}[fragile]
\frametitle{direct python}
\begin{lstlisting}
plt.title("Graph")
plt.legend()
plt.grid()
plt.savefig("Figure.png", dpi=200)
plt.show()

\end{lstlisting}
\end{frame}
\begin{frame}[fragile]
\frametitle{C code}
\begin{lstlisting}
#include <stdio.h>

typedef struct {
    double x;
    double y;
} Point;

// Function to find third vertex given two vertices and orthocenter
Point third_vertex(double x1, double y1, double x2, double y2, double hx, double hy){
    Point C;
    double m_alt_A, m_alt_B;
    double m_perp_BC, m_perp_AC;

    // Slope of altitude from A passing through H
    if(x1 != hx)
        m_alt_A = (hy - y1)/(hx - x1);
    else
        m_alt_A = 1e9; // vertical slope

    // Slope of altitude from B passing through H
    if(x2 != hx)
        m_alt_B = (hy - y2)/(hx - x2);
    else
\end{lstlisting}
\end{frame}
\begin{frame}[fragile]
\frametitle{C code}
\begin{lstlisting}
        m_alt_B = 1e9;
    // Equation for perpendicular slope relation
    if(m_alt_A != 1e9)
        m_perp_BC = -1.0 / m_alt_A;
    else
        m_perp_BC = 0; // horizontal BC

    if(m_alt_B != 1e9)
        m_perp_AC = -1.0 / m_alt_B;
    else
        m_perp_AC = 0; // horizontal AC
\end{lstlisting}
\end{frame}
\begin{frame}[fragile]
\frametitle{C code}
\begin{lstlisting}
    // Solve system: slope formula
    // y3 - y2 = m_perp_BC * (x3 - x2)
    // y3 - y1 = m_perp_AC * (x3 - x1)
    double x3 = (m_perp_BC * x2 - m_perp_AC * x1 + y1 - y2) / (m_perp_BC - m_perp_AC);
    double y3 = m_perp_BC * (x3 - x2) + y2;

    C.x = x3;
    C.y = y3;
    return C;
}
\end{lstlisting}
\end{frame}
\begin{frame}[fragile]
\frametitle{Python code with shared object}
\begin{lstlisting}
import ctypes
from ctypes import Structure, c_double
import matplotlib.pyplot as plt

# Define Point struct
class Point(Structure):
    _fields_ = [("x", c_double), ("y", c_double)]

# Load shared object
lib = ctypes.CDLL("./libtriangle.so")
lib.third_vertex.restype = Point
lib.third_vertex.argtypes = [c_double, c_double, c_double, c_double, c_double, c_double]
\end{lstlisting}
\end{frame}
\begin{frame}[fragile]
\frametitle{Python code with shared object}
\begin{lstlisting}
# Given vertices and orthocenter
A_x, A_y = 5, -1
B_x, B_y = 2, -3
H_x, H_y = 0, 0

# Call C function
C = lib.third_vertex(A_x, A_y, B_x, B_y, H_x, H_y)
print(f"Third vertex: ({C.x:.2f}, {C.y:.2f})")

# Plot triangle
x_coords = [A_x, B_x, C.x, A_x]
y_coords = [A_y, B_y, C.y, A_y]
\end{lstlisting}
\end{frame}
\begin{frame}[fragile]
\frametitle{Python code with shared object}
\begin{lstlisting}
plt.figure(figsize=(6,6))
plt.plot(x_coords, y_coords, 'b-o', label='Triangle')
plt.plot(H_x, H_y, 'r*', markersize=12, label='Orthocenter')
plt.text(A_x, A_y, 'A', fontsize=12, ha='right')
plt.text(B_x, B_y, 'B', fontsize=12, ha='right')
plt.text(C.x, C.y, 'C', fontsize=12, ha='right')
plt.text(H_x, H_y, 'H', fontsize=12, ha='right')
plt.grid(True)
plt.legend()
plt.xlabel('X-axis')
plt.ylabel('Y-axis')
plt.title('Triangle with Orthocenter')
plt.show()
\end{lstlisting}
\end{frame}

\end{document}