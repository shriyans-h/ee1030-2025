\let\negmedspace\undefined
\let\negthickspace\undefined
\documentclass[journal]{IEEEtran}
\usepackage[a5paper, margin=10mm, onecolumn]{geometry}
%\usepackage{lmodern} % Ensure lmodern is loaded for pdflatex
\usepackage{tfrupee} % Include tfrupee package

\setlength{\headheight}{1cm} % Set the height of the header box
\setlength{\headsep}{0mm}     % Set the distance between the header box and the top of the text

\usepackage{gvv-book}
\usepackage{gvv}
\usepackage{cite}
\usepackage{amsmath,amssymb,amsfonts,amsthm}
\usepackage{algorithmic}
\usepackage{graphicx}
\usepackage{textcomp}
\usepackage{xcolor}
\usepackage{txfonts}
\usepackage{listings}
\usepackage{enumitem}
\usepackage{mathtools}
\usepackage{gensymb}
\usepackage{comment}
\usepackage[breaklinks=true]{hyperref}
\usepackage{tkz-euclide} 
\usepackage{listings}
% \usepackage{gvv}                                        
\def\inputGnumericTable{}                                 
\usepackage[latin1]{inputenc}                                
\usepackage{color}                                            
\usepackage{array}                                            
\usepackage{longtable}                                       
\usepackage{calc}                                             
\usepackage{multirow}                                         
\usepackage{hhline}                                           
\usepackage{ifthen}                                           
\usepackage{lscape}
\usepackage{circuitikz}



\author{EE25BTECH11041-Naman Kumar }
\graphicspath{./figs/}

\begin{document}
\begin{center}
    \huge{4.13.45}\\
    \large{EE25BTECH11041 - Naman Kumar}
\end{center}
Question a):\\
Two vertices of a triangle are $(5, -1)$ and $(2, -3)$. If the orthocentre of the triangle
is the origin, find the coordinates of the third point.\\
\solution \\
Given,
\begin{align}
\vec{O}=\begin{pmatrix}0\\0\end{pmatrix},\vec{A}=\begin{pmatrix}5\\-1\end{pmatrix},\vec{B}=\begin{pmatrix}2\\-3\end{pmatrix}
\end{align}
Where,
\begin{tabular}[12pt]{ |c| c|}
    \hline
    \textbf{Name} & \textbf{Point}\\ 
    \hline
	Point A &\myvec{h \\ k}\\
    \hline 
 Point B &\myvec{x1 \\ y1}\\
    \hline
	  Point R &\myvec{x2 \\ y2}\\
    \hline
    
    \end{tabular}

From General Triangle Properties.
\begin{align}
    m_1^{T}A_1=0,m_2^{T}A_2=0\\
    (\vec{A}-\vec{O})^{T}(\vec{C}-\vec{B})=0;\\
    \vec{A}^{T}(\vec{C}-\vec{B})=0, \label{1}\text{Since $\vec{O}$ is origin}
\end{align}
and 
\begin{align}
    (\vec{B}-\vec{O})^{T}(\vec{C}-\vec{A})=0;\\
    (\vec{B})^{T}(\vec{C}-\vec{A})=0,\label{2}\text{Since $\vec{O}$ is origin}
\end{align}
Modifying $\eqref{1}$ and $\eqref{2}$
\begin{align}
\vec{A}^T\vec{C}-\vec{A}^T\vec{B}=0,\vec{B}^T\vec{C}-\vec{B}^T\vec{A}=0\\
\vec{A}^T\vec{C}=\vec{A}^T\vec{B} \label{3}\\ \vec{B}^T\vec{C}=\vec{B}^T\vec{A} \label{4}
\end{align}
This can be written as
\begin{align}
    \begin{pmatrix}\vec{A}^T \\ \vec{B}^T \end{pmatrix}\vec{C}=\begin{pmatrix}\vec{A}^T\vec{B}\\ \vec{B}^T\vec{A} \end{pmatrix}\\
    \begin{pmatrix}5 & -1\\ 2 & -3\end{pmatrix}\vec{C}=\begin{pmatrix}\vec{A}^T\vec{B}\\ \vec{B}^T\vec{A} \end{pmatrix} \label{5}
\end{align}
Let
\begin{align}
    \begin{amatrix}{2}5 & -1& \vec{A}^T\vec{B}\\ 2 & -3 &\vec{B}^T\vec{A} \end{amatrix} = \begin{amatrix}{2}5 & -1& 13\\ 2 & -3 &13 \end{amatrix}
\end{align}
By Gaussian Elimination
\begin{align}
    \begin{amatrix}{2}5 & -1& 13\\ 2 & -3 &13 \end{amatrix}\xrightarrow{R_2-\frac{2}{5}R_1}\begin{amatrix}{2}5 & -1& 13\\ 0 & \frac{-13}{5} &\frac{39}{5} \end{amatrix}\\
    \xrightarrow{-\frac{5}{13}R_2}\begin{amatrix}{2}5 & -1& 13\\ 0 & 1 & -3 \end{amatrix}
\end{align}
In equation $\eqref{5}$ 
\begin{align}
    \begin{pmatrix}5 & -1\\ 0 & 1\end{pmatrix}\begin{pmatrix}x\\ y\end{pmatrix}=\begin{pmatrix}13\\ -3\end{pmatrix}
\end{align}
Therefore, $\vec{C}$ is
\begin{align}
    \begin{pmatrix}x\\y\end{pmatrix} = \begin{pmatrix}2\\-3\end{pmatrix}
\end{align}
But,Now
\begin{align}
    \vec{B}=\vec{C}
\end{align}
Which is not possible for a triangle,\\
Slope of line $\vec{A}$ to $\vec{B}$ or $\vec{L}$ be $\vec{m}$
\begin{align}
\vec{m}=\vec{A}-\vec{B}\\
\vec{m}=\begin{pmatrix}3\\2\end{pmatrix}
\end{align}
We can see, that
\begin{align}
    \vec{m}^T\vec{A_2}=0,\vec{m_2}^T\vec{A_2}=0
\end{align}
Therefore, for this to be possible when
\begin{align}
    \vec{L} \parallel \vec{m_2}, \\ \vec{A}-\vec{C} \parallel \vec{A}-\vec{B}
\end{align}
Since both lines have a point in common $\vec{A}$, therefore they must be collinear.\\
So, $\vec{A},\vec{B}$ and $\vec{C}$ is just a straight line
\begin{figure}[H]
    \centering
    \includegraphics[width=\columnwidth]{figs/Figure.png}
    \caption{}
    \label{fig:placeholder}
\end{figure}

\end{document}
