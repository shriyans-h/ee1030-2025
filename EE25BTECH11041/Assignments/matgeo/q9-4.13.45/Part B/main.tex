\let\negmedspace\undefined
\let\negthickspace\undefined
\documentclass[journal]{IEEEtran}
\usepackage[a5paper, margin=10mm, onecolumn]{geometry}
%\usepackage{lmodern} % Ensure lmodern is loaded for pdflatex
\usepackage{tfrupee} % Include tfrupee package

\setlength{\headheight}{1cm} % Set the height of the header box
\setlength{\headsep}{0mm}     % Set the distance between the header box and the top of the text

\usepackage{gvv-book}
\usepackage{gvv}
\usepackage{cite}
\usepackage{amsmath,amssymb,amsfonts,amsthm}
\usepackage{algorithmic}
\usepackage{graphicx}
\usepackage{textcomp}
\usepackage{xcolor}
\usepackage{txfonts}
\usepackage{listings}
\usepackage{enumitem}
\usepackage{mathtools}
\usepackage{gensymb}
\usepackage{comment}
\usepackage[breaklinks=true]{hyperref}
\usepackage{tkz-euclide} 
\usepackage{listings}
% \usepackage{gvv}                                        
\def\inputGnumericTable{}                                 
\usepackage[latin1]{inputenc}                                
\usepackage{color}                                            
\usepackage{array}                                            
\usepackage{longtable}                                       
\usepackage{calc}                                             
\usepackage{multirow}                                         
\usepackage{hhline}                                           
\usepackage{ifthen}                                           
\usepackage{lscape}
\usepackage{circuitikz}



\author{EE25BTECH11041-Naman Kumar }
\graphicspath{./figs/}

\begin{document}
\begin{center}
    \huge{4.13.45}\\
    \large{EE25BTECH11041 - Naman Kumar}
\end{center}
Question b):\\
Find the equation of the line which bisects the obtuse angle between the lines
$x -2y + 4 = 0$ and $4x - 3y + 2 = 0$.\\
\solution \\
Given,
\begin{tabular}[12pt]{ |c| c|}
    \hline
    \textbf{Name} & \textbf{Point}\\ 
    \hline
	Point A &\myvec{h \\ k}\\
    \hline 
 Point B &\myvec{x1 \\ y1}\\
    \hline
	  Point R &\myvec{x2 \\ y2}\\
    \hline
    
    \end{tabular}

\begin{align}
\vec{n_1}^T\vec{x}=c_1,\vec{n_2}^T\vec{x}
\end{align}
Where,
\begin{align}
\vec{n_1}=\begin{pmatrix}1\\-2\end{pmatrix},\vec{n_2}=\begin{pmatrix}4\\-3\end{pmatrix},c_1=-4\text{ and }c_2=-2
\end{align}
Equation for bisectors is
\begin{align}
    \lvert\frac{\vec{n_1}^T\vec{B}-c_1}{\parallel\vec{n_1}\parallel}\rvert=\lvert\frac{\vec{n_2}^T\vec{B}-c_2}{\parallel\vec{n_2}\parallel}\rvert\\
    \frac{\vec{n_1}^T\vec{B}-c_1}{\parallel\vec{n_1}\parallel}=\pm \frac{\vec{n_2}^T\vec{B}-c_2}{\parallel\vec{n_2}\parallel}\\
    \frac{\vec{n_1}^T\vec{B_1}-c_1}{\parallel\vec{n_1}\parallel}=\frac{\vec{n_2}^T\vec{B_1}-c_2}{\parallel\vec{n_2}\parallel},\text{ and }\frac{\vec{n_1}^T\vec{B_2}-c_1}{\parallel\vec{n_1}\parallel}=-\frac{\vec{n_2}^T\vec{B_2}-c_2}{\parallel\vec{n_2}\parallel}
\end{align}
Can be written as
\begin{align}
    \brak{\frac{\vec{n_1}^T}{\parallel\vec{n_1}\parallel} - \frac{\vec{n_2}^T}{\parallel\vec{n_2}\parallel}}\vec{B_1}=\brak{\frac{c_1}{\parallel\vec{n_1}\parallel} - \frac{c_2}{\parallel\vec{n_2}\parallel}}\label{b1} \\
    \vec{n_{B_1}}^TB_1=c_{B_1} \label{b1}
\end{align}
and
\begin{align}
    \brak{\frac{\vec{n_1}^T}{\parallel\vec{n_1}\parallel} + \frac{\vec{n_2}^T}{\parallel\vec{n_2}\parallel}}\vec{B_2}=\brak{\frac{c_2}{\parallel\vec{n_2}\parallel} + \frac{c_1}{\parallel\vec{n_1}\parallel}} {b2}\\
    \vec{n_{B_2}}^TB_2=c_{B_2} \label{b2}
\end{align}
Now for obtuse angle bisector
\begin{align}
    \cos\theta_1=\frac{\vec{n_{B_1}}^T\vec{n_1}}{\parallel\vec{n_1}\parallel\parallel\vec{n_{B_1}}\parallel}\\
    \cos\theta_2=\frac{\vec{n_{B_2}}^T\vec{n_1}}{\parallel\vec{n_1}\parallel\parallel\vec{n_{B_2}}\parallel}
\end{align}
Solving with $\eqref{b1}$ and $\eqref{b2}$
\begin{align}
    \cos\theta_1=\frac{\brak{\frac{\vec{n_1}^T}{\parallel\vec{n_1}\parallel} - \frac{\vec{n_2}^T}{\parallel\vec{n_2}\parallel}}\vec{n_1}}{\parallel\vec{n_1}\parallel\parallel\vec{n_{B_1}}\parallel}\\
    \cos\theta_1=\frac{\brak{\frac{\vec{n_1}^T}{\parallel\vec{n_1}\parallel}\vec{n_1} - \frac{\vec{n_2}^T}{\parallel\vec{n_2}\parallel}\vec{n_1}}}{\parallel\vec{n_1}\parallel\parallel\vec{n_{B_1}}\parallel}\\
    \cos\theta_1=\frac{\brak{\sqrt{5} - \frac{10}{5}}}{\parallel\vec{n_1}\parallel\parallel\vec{n_{B_1}}\parallel}\\
    \cos\theta_1=\frac{\sqrt{5}-2}{\sqrt{5}\sqrt{\brak{\frac{1}{\sqrt{5}}-\frac{4}{5}}^2+\brak{\frac{3}{5}-\frac{2}{\sqrt{5}}}^2}}\approx 0.22 \label{a1}
\end{align}
Similarly
\begin{align}
    \cos\theta_2=\frac{\sqrt{5}+2}{\sqrt{5}\sqrt{\brak{\frac{1}{\sqrt{5}}+\frac{4}{5}}^2+\brak{\frac{3}{5}+\frac{2}{\sqrt{5}}}^2}}\approx 0.97\label{a2}
\end{align}
by comparing $\eqref{a1}$ and $\eqref{a2}$
\begin{align}
    \theta_1>\theta_2
\end{align}
So $B_1$ is obtuse angle bisector
\begin{align}
\vec{n_{B_1}}^TB_1=c_{B_1}\\
\vec{n_{\vec{B_1}}}= \begin{pmatrix}\frac{1}{\sqrt{5}}-\frac{4}{5}\\ \frac{3}{5}-\frac{2}{\sqrt{5}} \end{pmatrix}
\end{align}
and
\begin{align}
    c_{B_1}=\frac{2}{5}+\frac{-4}{\sqrt{5}}
\end{align}
\begin{figure}
    \centering
    \includegraphics[width=\columnwidth]{figs/figure.png}
    \caption{}
    \label{fig:placeholder}
\end{figure}

\end{document}
