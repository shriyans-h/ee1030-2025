\let\negmedspace\undefined
\let\negthickspace\undefined
\documentclass[journal]{IEEEtran}
\usepackage[a5paper, margin=10mm, onecolumn]{geometry}
%\usepackage{lmodern} % Ensure lmodern is loaded for pdflatex
\usepackage{tfrupee} % Include tfrupee package

\setlength{\headheight}{1cm} % Set the height of the header box
\setlength{\headsep}{0mm}     % Set the distance between the header box and the top of the text

\usepackage{gvv-book}
\usepackage{gvv}
\usepackage{cite}
\usepackage{amsmath,amssymb,amsfonts,amsthm}
\usepackage{algorithmic}
\usepackage{graphicx}
\usepackage{textcomp}
\usepackage{xcolor}
\usepackage{txfonts}
\usepackage{listings}
\usepackage{enumitem}
\usepackage{mathtools}
\usepackage{gensymb}
\usepackage{comment}
\usepackage[breaklinks=true]{hyperref}
\usepackage{tkz-euclide} 
\usepackage{listings}
% \usepackage{gvv}                                        
\def\inputGnumericTable{}                                 
\usepackage[latin1]{inputenc}                                
\usepackage{color}                                            
\usepackage{array}                                            
\usepackage{longtable}                                       
\usepackage{calc}                                             
\usepackage{multirow}                                         
\usepackage{hhline}                                           
\usepackage{ifthen}                                           
\usepackage{lscape}
\usepackage{circuitikz}



\author{EE25BTECH11041-Naman Kumar }
\graphicspath{./figs/}

\begin{document}
\begin{center}
    \huge{4.11.18}\\
    \large{EE25BTECH11041 - Naman Kumar}
\end{center}
Question:\\
Find the equation of the plane which contains the line of intersection of the planes $\vec{r}\cdot(\imath-2\jmath+3\hat{k})-4=0$ and  $\vec{r}\cdot(-2\imath+\jmath+\hat{k})+5=0$ and whose intercept on X axis is equal to that of on Y axis.\\
\solution \\
Given Planes,
\begin{align}
    n_1^Tx=c_1,n_2^Tx=c_2
\end{align}
Where
\begin{align}
n_1=\begin{pmatrix}1\\-2\\3\end{pmatrix},n_2=\begin{pmatrix}-2\\1\\1\end{pmatrix},c_1=4,c_2=-5
\end{align}
Let Required equation of plane
\begin{align}
    n_3^Tx=c_3
\end{align}
Since we can write,
\begin{align}
P_3=P_1-\lambda P_2 \text{ (Where $P_1,P_2,P_3$ are equation of planes)}
\end{align}
Because All three planes intersect at same line,Therefore
\begin{align}
    (n_1-\lambda n_2)^Tx=c_1-\lambda c_2\\
\end{align}
Given,
\begin{align}
    X-intercept=Y-intercept\\
\end{align}
for X-intercept
\begin{align}
    (n_1-\lambda n_2)^T\begin{pmatrix}x\\0\\0\end{pmatrix}=c_1-\lambda c_2\\
    (n_1-\lambda n_2)^Txe_1=c_1-\lambda c_2
\end{align}
Therefore,
\begin{align}
    X-intercept= \frac{c_1-\lambda c_2}{(n_1-\lambda n_2)^Te_1}\label{1}
\end{align}
Similarly
\begin{align}
    Y-intercept=\frac{c_1-\lambda c_2}{(n_1-\lambda n_2)^Te_2} \label{2}
\end{align}
\newpage
Comparing equations $\eqref{1}$ and $\eqref{2}$
\begin{align}
    \frac{c_1-\lambda c_2}{(n_1-\lambda n_2)^Te_1}=\frac{c_1-\lambda c_2}{(n_1-\lambda n_2)^Te_2}\\
    (n_1-\lambda n_2)^Te_1=(n_1-\lambda n_2)^Te_2\\
    \begin{pmatrix}1+2\lambda\\-2-1\lambda\\3-1\lambda\end{pmatrix}^Te_1=\begin{pmatrix}1+2\lambda\\-2-1\lambda\\3-1\lambda\end{pmatrix}^Te_2\\
    1+2\lambda=-2-1\lambda\\
    \lambda=-1
\end{align}
Therefore equation of required plane is
\begin{align}
    \begin{pmatrix}1+2(-1)\\-2-1(-1)\\3-1(-1)\end{pmatrix}^Tx=4+5(-1)\\
    \begin{pmatrix}-1\\-1\\4\end{pmatrix}^Tx=-1
\end{align}
\begin{figure}[H]
    \centering
    \includegraphics[width=\columnwidth]{figs/Figure.png}
    \caption{}
    \label{fig:placeholder}
\end{figure}
\end{document}
