\let\negmedspace\undefined
\let\negthickspace\undefined
\documentclass[a5paper,10pt]{article}
\usepackage[margin=10mm]{geometry}
%\usepackage{lmodern} % Ensure lmodern is loaded for pdflatex
\usepackage{tfrupee} % Include tfrupee package

\setlength{\headheight}{1cm} % Set the height of the header box
\setlength{\headsep}{0mm}     % Set the distance between the header box and the top of the text

\usepackage{gvv-book}
\usepackage{gvv}
\usepackage{cite}
\usepackage{amsmath,amssymb,amsfonts,amsthm}
\usepackage{algorithmic}
\usepackage{graphicx}
\usepackage{textcomp}
\usepackage{xcolor}
\usepackage{txfonts}
\usepackage{listings}
\usepackage{enumitem}
\usepackage{mathtools}
\usepackage{gensymb}
\usepackage{comment}
\usepackage[breaklinks=true]{hyperref}
\usepackage{tkz-euclide} 
\usepackage{listings}
% \usepackage{gvv}                                        
\def\inputGnumericTable{}                                 
\usepackage[latin1]{inputenc}                                
\usepackage{color}                                            
\usepackage{array}                                            
\usepackage{longtable}                                       
\usepackage{calc}                                             
\usepackage{multirow}                                         
\usepackage{hhline}                                           
\usepackage{ifthen}                                           
\usepackage{lscape}
\usepackage{circuitikz}



\author{EE25BTECH11041-Naman Kumar }
\graphicspath{./figs/}

\begin{document}
\begin{center}
    \huge{8.2.3}\\
    \large{EE25BTECH11041 - Naman Kumar}
\end{center}
Question:\\
\begin{align}
y^2 = -8x \label{1} 
\end{align}
\solution \\
Since it is a parabola we have
\begin{tabular}[12pt]{ |c| c|}
    \hline
    \textbf{Name} & \textbf{Point}\\ 
    \hline
	Point A &\myvec{h \\ k}\\
    \hline 
 Point B &\myvec{x1 \\ y1}\\
    \hline
	  Point R &\myvec{x2 \\ y2}\\
    \hline
    
    \end{tabular}

General equation of conic
\begin{align}
    \vec{x^T}\vec{V}\vec{x}+2\vec{u^T}\vec{x}+f=0
\end{align}
For the equation $\eqref{1}$, we can write
\begin{align}
    \vec{x^T}\begin{pmatrix}0&0\\0&1\end{pmatrix}\vec{x}+2\begin{pmatrix}4\\0\end{pmatrix}^T\vec{x}=0
\end{align}
\begin{table}[h!]
    \centering
    \begin{tabular}{|c|c|c|}
        \hline
        Point & For $k=3$ & For $k=-\tfrac{9}{2}$ \\
        \hline
        $A$ & $(1,\,-1)$ & $(1,\,-1)$ \\
        $B$ & $(-4,\,6)$ & $(-4,\,-9)$ \\
        $C$ & $(-3,\,-5)$ & $\left(\tfrac{9}{2},\,-5\right)$ \\
        \hline
    \end{tabular}
    \caption{Vertices of $\triangle ABC$ after substituting $k$ values}
    \label{tab:triangle_values}
\end{table}

using general equations we know for any conic
\begin{align}
    \vec{V}=\lVert\vec{n}\rVert^2\vec{I}-e^2\vec{n}\vec{n}^T \label{2} \\
    \vec{u}=ce^2\vec{n}-\lVert\vec{n}\rVert^2\vec{F} \label{3} \\
    f=\lVert\vec{n}\rVert^2\lVert\vec{F}\rVert^2-c^2e^2 \label{4}
\end{align}
Let
\begin{table}[H]
    \centering
    \begin{tabular}{|l|l|l|}
    \hline
      $\vec{n}$   & $\begin{pmatrix}a\\b\end{pmatrix}$ & normal vector to directrix\\
      \hline
      $\vec{F}$   & $\begin{pmatrix}g\\h\end{pmatrix}$ & focus\\
      \hline
      c   & & be constant of directrix \\
      \hline
    \end{tabular}
    \label{tab:tables/table3.tex}
\end{table}
Firstly in $\eqref{2}$
\begin{align}
    \begin{pmatrix}0&0\\0&1\end{pmatrix}=\vec{n}^T\vec{n}\vec{I}-(1)^2\vec{n}\vec{n}^T\\
    \begin{pmatrix}0&0\\0&1\end{pmatrix}=\begin{pmatrix}a^2+b^2&0\\0&a^2+b^2\end{pmatrix}-\begin{pmatrix}a^2&ab\\ab&b^2\end{pmatrix}\\
    \vec{n}=\begin{pmatrix}1\\0\end{pmatrix}=\vec{e_1}
\end{align}
In $\eqref{3}$
\begin{align}
    4\vec{e_1}=c(1)^2\vec{e_1}-(1)^2\begin{pmatrix}g\\h\end{pmatrix}\\
    \begin{pmatrix}g\\h\end{pmatrix}=(c-4)\vec{e_1}=(c-4)\begin{pmatrix}1\\0\end{pmatrix}\\
    \vec{F}=\begin{pmatrix}c-4\\0\end{pmatrix}
\end{align}
In $\eqref{4}$
\begin{align}
    0=(1)^2\begin{pmatrix}c-4\\0\end{pmatrix}^T\begin{pmatrix}c-4\\0\end{pmatrix}-c^2(1)\\
    c^2+16-8c-c^2=0\\
    c=2\\
    \vec{F}=-2\vec{e_1}
\end{align}
Directrix is
\begin{align}
    \vec{n^T}\vec{x}=c\\
    \vec{e_1}^T\vec{x}=2
\end{align}
\begin{figure}[H]
    \centering
    \includegraphics[width=\columnwidth]{figs/figure.png}
    \caption{}
    \label{fig:placeholder}
\end{figure}
\end{document}
