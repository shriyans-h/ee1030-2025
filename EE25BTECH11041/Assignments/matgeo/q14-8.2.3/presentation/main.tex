\documentclass{beamer}
\usepackage[utf8]{inputenc}

\usetheme{Madrid}
\usecolortheme{default}
\usepackage{amsmath,amssymb,amsfonts,amsthm}
\usepackage{txfonts}
\usepackage{tkz-euclide}
\usepackage{listings}
\usepackage{adjustbox}
\usepackage{array}
\usepackage{tabularx}
\usepackage{gvv}
\usepackage{lmodern}
\usepackage{circuitikz}
\usepackage{tikz}
\usepackage{graphicx}

\setbeamertemplate{page number in head/foot}[totalframenumber]

\usepackage{tcolorbox}
\tcbuselibrary{minted,breakable,xparse,skins}



\definecolor{bg}{gray}{0.95}
\DeclareTCBListing{mintedbox}{O{}m!O{}}{%
	breakable=true,
	listing engine=minted,
	listing only,
	minted language=#2,
	minted style=default,
	minted options={%
		linenos,
		gobble=0,
		breaklines=true,
		breakafter=,,
		fontsize=\small,
		numbersep=8pt,
		#1},
	boxsep=0pt,
	left skip=0pt,
	right skip=0pt,
	left=25pt,
	right=0pt,
	top=3pt,
	bottom=3pt,
	arc=5pt,
	leftrule=0pt,
	rightrule=0pt,
	bottomrule=2pt,
	toprule=2pt,
	colback=bg,
	colframe=orange!70,
	enhanced,
	overlay={%
		\begin{tcbclipinterior}
			\fill[orange!20!white] (frame.south west) rectangle ([xshift=20pt]frame.north west);
	\end{tcbclipinterior}},
	#3,
}
\lstset{
	language=C,
	basicstyle=\ttfamily\small,
	keywordstyle=\color{blue},
	stringstyle=\color{orange},
	commentstyle=\color{green!60!black},
	numbers=left,
	numberstyle=\tiny\color{gray},
	breaklines=true,
	showstringspaces=false,
}
\begin{document}

\title 
{8.2.3}
\date{1 Oct,2025}

\author 
{Naman Kumar-EE25BTECH11041}
\graphicspath{./figs}


\frame{\titlepage}
\begin{frame}{Question)}
\begin{align}
y^2 = -8x \label{1} 
\end{align}
\end{frame}
\begin{frame}{Solution}
Since it is a parabola we have
\begin{tabular}[12pt]{ |c| c|}
    \hline
    \textbf{Name} & \textbf{Point}\\ 
    \hline
	Point A &\myvec{h \\ k}\\
    \hline 
 Point B &\myvec{x1 \\ y1}\\
    \hline
	  Point R &\myvec{x2 \\ y2}\\
    \hline
    
    \end{tabular}

General equation of conic
\begin{align}
    \vec{x^T}\vec{V}\vec{x}+2\vec{u^T}\vec{x}+f=0
\end{align}
\end{frame}
\begin{frame}{Solution}
For the equation $\eqref{1}$, we can write
\begin{align}
    \vec{x^T}\begin{pmatrix}0&0\\0&1\end{pmatrix}\vec{x}+2\begin{pmatrix}4\\0\end{pmatrix}^T\vec{x}=0
\end{align}
\begin{table}[h!]
    \centering
    \begin{tabular}{|c|c|c|}
        \hline
        Point & For $k=3$ & For $k=-\tfrac{9}{2}$ \\
        \hline
        $A$ & $(1,\,-1)$ & $(1,\,-1)$ \\
        $B$ & $(-4,\,6)$ & $(-4,\,-9)$ \\
        $C$ & $(-3,\,-5)$ & $\left(\tfrac{9}{2},\,-5\right)$ \\
        \hline
    \end{tabular}
    \caption{Vertices of $\triangle ABC$ after substituting $k$ values}
    \label{tab:triangle_values}
\end{table}

\end{frame}
\begin{frame}{Solution}
using general equations we know for any conic
\begin{align}
    \vec{V}=\lVert\vec{n}\rVert^2\vec{I}-e^2\vec{n}\vec{n}^T \label{2} \\
    \vec{u}=ce^2\vec{n}-\lVert\vec{n}\rVert^2\vec{F} \label{3} \\
    f=\lVert\vec{n}\rVert^2\lVert\vec{F}\rVert^2-c^2e^2 \label{4}
\end{align}
Let
\begin{table}[H]
    \centering
    \begin{tabular}{|l|l|l|}
    \hline
      $\vec{n}$   & $\begin{pmatrix}a\\b\end{pmatrix}$ & normal vector to directrix\\
      \hline
      $\vec{F}$   & $\begin{pmatrix}g\\h\end{pmatrix}$ & focus\\
      \hline
      c   & & be constant of directrix \\
      \hline
    \end{tabular}
    \label{tab:tables/table3.tex}
\end{table}
\end{frame}
\begin{frame}{Solution}
Firstly in $\eqref{2}$
\begin{align}
    \begin{pmatrix}0&0\\0&1\end{pmatrix}=\vec{n}^T\vec{n}\vec{I}-(1)^2\vec{n}\vec{n}^T\\
    \begin{pmatrix}0&0\\0&1\end{pmatrix}=\begin{pmatrix}a^2+b^2&0\\0&a^2+b^2\end{pmatrix}-\begin{pmatrix}a^2&ab\\ab&b^2\end{pmatrix}\\
    \vec{n}=\begin{pmatrix}1\\0\end{pmatrix}=\vec{e_1}
\end{align}
\end{frame}
\begin{frame}{Solution}
In $\eqref{3}$
\begin{align}
    4\vec{e_1}=c(1)^2\vec{e_1}-(1)^2\begin{pmatrix}g\\h\end{pmatrix}\\
    \begin{pmatrix}g\\h\end{pmatrix}=(c-4)\vec{e_1}=(c-4)\begin{pmatrix}1\\0\end{pmatrix}\\
    \vec{F}=\begin{pmatrix}c-4\\0\end{pmatrix}
\end{align}
\end{frame}
\begin{frame}{Solution}
In $\eqref{4}$
\begin{align}
    0=(1)^2\begin{pmatrix}c-4\\0\end{pmatrix}^T\begin{pmatrix}c-4\\0\end{pmatrix}-c^2(1)\\
    c^2+16-8c-c^2=0\\
    c=2\\
    \vec{F}=-2\vec{e_1}
\end{align}
\end{frame}
\begin{frame}{Solution}
Directrix is
\begin{align}
    \vec{n^T}\vec{x}=c\\
    \vec{e_1}^T\vec{x}=2
\end{align}
\end{frame}
\begin{frame}{Figure}
    \begin{figure}[H]
        \centering
        \includegraphics[width=0.9\columnwidth]{figs/figure.png}
        \caption{}
        \label{fig:placeholder}
    \end{figure}
\end{frame}
\begin{frame}[fragile]
\frametitle{Direct Python}
\begin{lstlisting}
import numpy as np
import matplotlib.pyplot as plt

y = np.linspace(-7,7,200)
x = -y*y/8

plt.axvline(x=2, color='blue', label="Directrix")
plt.axvline(x=-2, color='orange', label="Latus Rectum")

xp = np.array([0,2,-2])
yp = np.array([0,0,0])
plt.scatter(xp,yp)
\end{lstlisting}
\end{frame}
\begin{frame}[fragile]
\frametitle{Direct Python}
\begin{lstlisting}
plt.annotate('O', xy=(0, 0)) 
plt.annotate('F', xy=(-2, 0), xytext=(-1.7,0.5)) 

plt.plot(x,y, label="Parabola", color='red')
plt.grid()
plt.legend()
plt.savefig("figure.png", dpi=300)
plt.show()
\end{lstlisting}
\end{frame}
\begin{frame}[fragile]
\frametitle{C code}
\begin{lstlisting}
// main.c
#include <stdio.h>
#include <math.h>

int generate_points(double x[], double y[], int n) {
    double xmin = -10.0;  // start of x range
    double xmax = 0.0;    // parabola is defined for x <= 0
    double step = (xmax - xmin) / (n/2); // half because we store ±y

    int idx = 0;
\end{lstlisting}
\end{frame}
\begin{frame}[fragile]
\frametitle{C code}
\begin{lstlisting}
    for (int i = 0; i <= n/2 && idx < n-1; i++) {
        double xval = xmin + i * step;
        double yval = sqrt(-8.0 * xval);

        x[idx] = xval;
        y[idx] = yval;
        idx++;

        x[idx] = xval;
        y[idx] = -yval;
        idx++;
    }

    return idx;
}
\end{lstlisting}
\end{frame}
\begin{frame}[fragile]
\frametitle{Python code with shared object}
\begin{lstlisting}
# main.py
import ctypes
import numpy as np
import matplotlib.pyplot as plt

# Load the shared object
lib = ctypes.CDLL("./libparabola.so")

# Define function prototype
lib.generate_points.argtypes = [
    np.ctypeslib.ndpointer(dtype=np.float64, ndim=1, flags="C_CONTIGUOUS"),
    np.ctypeslib.ndpointer(dtype=np.float64, ndim=1, flags="C_CONTIGUOUS"),
    ctypes.c_int
]
lib.generate_points.restype = ctypes.c_int

\end{lstlisting}
\end{frame}
\begin{frame}[fragile]
\frametitle{Python code with shared object}
\begin{lstlisting}
# Allocate arrays
n = 4000
x = np.zeros(n, dtype=np.float64)
y = np.zeros(n, dtype=np.float64)

# Call the C function
count = lib.generate_points(x, y, n)

# Slice to actual filled points
x = x[:count]
y = y[:count]

\end{lstlisting}
\end{frame}
\begin{frame}[fragile]
\frametitle{Python code with shared object}
\begin{lstlisting}
# Plot parabola
plt.figure(figsize=(6,6))
plt.scatter(x, y, s=5, c='b', label=r"$y^2=-8x$")
plt.axhline(0, color='k', linewidth=0.8)
plt.axvline(0, color='k', linewidth=0.8)
plt.xlabel("x-axis")
plt.ylabel("y-axis")
plt.title("Parabola: $y^2 = -8x$")
plt.legend()
plt.grid(True)
plt.axis("equal")
plt.show()

\end{lstlisting}
\end{frame}

\end{document}