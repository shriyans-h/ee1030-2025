\let\negmedspace\undefined
\let\negthickspace\undefined
\documentclass[a5paper,10pt]{article}
\usepackage[margin=10mm]{geometry}
%\usepackage{lmodern} % Ensure lmodern is loaded for pdflatex
\usepackage{tfrupee} % Include tfrupee package

\setlength{\headheight}{1cm} % Set the height of the header box
\setlength{\headsep}{0mm}     % Set the distance between the header box and the top of the text

\usepackage{gvv-book}
\usepackage{gvv}
\usepackage{cite}
\usepackage{amsmath,amssymb,amsfonts,amsthm}
\usepackage{algorithmic}
\usepackage{graphicx}
\usepackage{textcomp}
\usepackage{xcolor}
\usepackage{txfonts}
\usepackage{listings}
\usepackage{enumitem}
\usepackage{mathtools}
\usepackage{gensymb}
\usepackage{comment}
\usepackage[breaklinks=true]{hyperref}
\usepackage{tkz-euclide} 
\usepackage{listings}
% \usepackage{gvv}                                        
\def\inputGnumericTable{}                                 
\usepackage[latin1]{inputenc}                                
\usepackage{color}                                            
\usepackage{array}                                            
\usepackage{longtable}                                       
\usepackage{calc}                                             
\usepackage{multirow}                                         
\usepackage{hhline}                                           
\usepackage{ifthen}                                           
\usepackage{lscape}
\usepackage{circuitikz}



\author{EE25BTECH11041-Naman Kumar }
\graphicspath{./figs/}

\begin{document}
\begin{center}
    \huge{10.5.8}\\
    \large{EE25BTECH11041 - Naman Kumar}
\end{center}
Question:\\
Draw two concentric circles of radii 3 cm and 5 cm. Taking a point on outer circle construct the pair of tangents to the other. Measure the length of a tangent and verify it by actual calculation.\\
\solution \\
General equation of conic
\begin{align}
    g(\vec{x})=\vec{x^T}\vec{V}\vec{x}+2\vec{u^T}\vec{x}+f 
\end{align}
Equation of circle,
\begin{align}
    \vec{x^T}\begin{pmatrix}1&0\\0&1\end{pmatrix}\vec{x}+2\begin{pmatrix}0\\0\end{pmatrix}^T\vec{x}-r^2=0,\label{1} r=\text{radius od circle}\\
    r_1=3cm , r_2=5cm
\end{align}
A point lies on the tangent to the conic if it satisfies the following equation
\begin{align}
    \vec{m}^T\sbrak{\brak{\vec{V}\vec{h}+\vec{u}}\brak{\vec{V}\vec{h}+\vec{u}}^T-\vec{V}g(\vec{h})}\vec{m}=0 \label{2}
\end{align}
Assuming a point on outer circle as $\vec{A}(5,0)$\\
putting $\vec{A}$ in $\eqref{1}$ for inner circle
\begin{align}
    \vec{A^T}\begin{pmatrix}1&0\\0&1\end{pmatrix}\vec{A}+2\begin{pmatrix}0\\0\end{pmatrix}^T\vec{A}-(r_1)^2\\
    25-9=16\\
    g(\vec{A})_1=16
\end{align}
Calculating $\brak{\vec{V}\vec{A}+\vec{u}}$
\begin{align}
    \begin{pmatrix}1&0\\0&1\end{pmatrix}\begin{pmatrix}5\\0\end{pmatrix}+\begin{pmatrix}0\\0\end{pmatrix}\\
    \begin{pmatrix}5\\0\end{pmatrix}    
\end{align}
putting in $\eqref{2}$
\begin{align}
    \vec{m}^T\sbrak{\brak{\vec{V}\vec{A}+\vec{u}}\brak{\vec{V}\vec{A}+\vec{u}}^T-\vec{V}g(\vec{A})_1}\vec{m}=0\\
    \vec{m}^T\sbrak{\begin{pmatrix}5\\0\end{pmatrix}\begin{pmatrix}5\\0\end{pmatrix}^T-\begin{pmatrix}1&0\\0&1\end{pmatrix}\times16}\vec{m}=0\\
    \vec{m}^T\sbrak{\begin{pmatrix}9&0\\0&-16\end{pmatrix}}\vec{m}=0\\
    \begin{pmatrix}1\\m\end{pmatrix}^T\begin{pmatrix}9&0\\0&-16\end{pmatrix}\begin{pmatrix}1\\m\end{pmatrix}=0\\
    9-16m^2=0\\
    m=\pm \frac{3}{4}\\
    \vec{m}=\begin{pmatrix}1\\\pm \frac{3}{4}\end{pmatrix}
\end{align}
Using following formula to find point of contact of tangent
\begin{align}
    \vec{q}_{j}=\brak{\pm r \frac{\vec{n}_j}{\lVert \vec{n}_j \rVert} -\vec{u}}, j=1,2 \label{3} \\
    \vec{q_1}=\brak{\pm 3 \frac{\begin{pmatrix}\frac{3}{4}\\1\end{pmatrix}}{\sqrt{\brak{\frac{3}{4}}^2+1}}} \\
    \vec{q}_1=\pm \begin{pmatrix}\frac{9}{5}\\ \frac{12}{5}\end{pmatrix}\\
    Similarly, \vec{q}_2=\pm\begin{pmatrix}\frac{9}{5}\\ \frac{-12}{5}\end{pmatrix}
\end{align}
To take the ones passing through $\vec{A}$ taking $\vec{q}_1$ and $\vec{q}_2$ as
\begin{align}
\vec{q}_1=\begin{pmatrix}\frac{9}{5}\\ \frac{12}{5}\end{pmatrix}\\
\vec{q}_2=\begin{pmatrix}\frac{9}{5}\\ \frac{-12}{5}\end{pmatrix}    
\end{align}
Length of both tangent will be equal and will be
\begin{align}
    \lVert \vec{q_1}-\vec{A} \rVert\\
    \lVert \begin{pmatrix}\frac{9}{5}\\ \frac{12}{5}\end{pmatrix} -\begin{pmatrix}5\\0\end{pmatrix}\rVert\\
    \lVert \begin{pmatrix}\frac{-16}{5}\\ \frac{12}{5}\end{pmatrix}\rVert\\
    =4
\end{align}
\begin{figure}[H]
    \centering
    \includegraphics[width=\columnwidth]{figs/figure.png}
    \caption{}
    \label{fig:placeholder}
\end{figure}
\end{document}
