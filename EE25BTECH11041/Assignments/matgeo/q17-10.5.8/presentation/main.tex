\documentclass{beamer}
\usepackage[utf8]{inputenc}

\usetheme{Madrid}
\usecolortheme{default}
\usepackage{amsmath,amssymb,amsfonts,amsthm}
\usepackage{txfonts}
\usepackage{tkz-euclide}
\usepackage{listings}
\usepackage{adjustbox}
\usepackage{array}
\usepackage{tabularx}
\usepackage{gvv}
\usepackage{lmodern}
\usepackage{circuitikz}
\usepackage{tikz}
\usepackage{graphicx}

\setbeamertemplate{page number in head/foot}[totalframenumber]

\usepackage{tcolorbox}
\tcbuselibrary{minted,breakable,xparse,skins}



\definecolor{bg}{gray}{0.95}
\DeclareTCBListing{mintedbox}{O{}m!O{}}{%
	breakable=true,
	listing engine=minted,
	listing only,
	minted language=#2,
	minted style=default,
	minted options={%
		linenos,
		gobble=0,
		breaklines=true,
		breakafter=,,
		fontsize=\small,
		numbersep=8pt,
		#1},
	boxsep=0pt,
	left skip=0pt,
	right skip=0pt,
	left=25pt,
	right=0pt,
	top=3pt,
	bottom=3pt,
	arc=5pt,
	leftrule=0pt,
	rightrule=0pt,
	bottomrule=2pt,
	toprule=2pt,
	colback=bg,
	colframe=orange!70,
	enhanced,
	overlay={%
		\begin{tcbclipinterior}
			\fill[orange!20!white] (frame.south west) rectangle ([xshift=20pt]frame.north west);
	\end{tcbclipinterior}},
	#3,
}
\lstset{
	language=C,
	basicstyle=\ttfamily\small,
	keywordstyle=\color{blue},
	stringstyle=\color{orange},
	commentstyle=\color{green!60!black},
	numbers=left,
	numberstyle=\tiny\color{gray},
	breaklines=true,
	showstringspaces=false,
}
\begin{document}

\title 
{10.5.8}
\date{5 Oct,2025}

\author 
{Naman Kumar-EE25BTECH11041}
\graphicspath{./figs}


\frame{\titlepage}
\begin{frame}{Question)}
Draw two concentric circles of radii 3 cm and 5 cm. Taking a point on outer circle construct the pair of tangents to the other. Measure the length of a tangent and verify it by actual calculation.
\end{frame}
\begin{frame}{Solution}
General equation of conic
\begin{align}
    g(\vec{x})=\vec{x^T}\vec{V}\vec{x}+2\vec{u^T}\vec{x}+f 
\end{align}
Equation of circle,
\begin{align}
    \vec{x^T}\begin{pmatrix}1&0\\0&1\end{pmatrix}\vec{x}+2\begin{pmatrix}0\\0\end{pmatrix}^T\vec{x}-r^2=0,\label{1} r=\text{radius od circle}\\
    r_1=3cm , r_2=5cm
\end{align}
\end{frame}
\begin{frame}{Solution}
A point lies on the tangent to the conic if it satisfies the following equation
\begin{align}
    \vec{m}^T\sbrak{\brak{\vec{V}\vec{h}+\vec{u}}\brak{\vec{V}\vec{h}+\vec{u}}^T-\vec{V}g(\vec{h})}\vec{m}=0 \label{2}
\end{align}
Assuming a point on outer circle as $\vec{A}(5,0)$\\
putting $\vec{A}$ in $\eqref{1}$ for inner circle
\begin{align}
    \vec{A^T}\begin{pmatrix}1&0\\0&1\end{pmatrix}\vec{A}+2\begin{pmatrix}0\\0\end{pmatrix}^T\vec{A}-(r_1)^2\\
    25-9=16\\
    g(\vec{A})_1=16
\end{align}
\end{frame}
\begin{frame}{Solution}
Calculating $\brak{\vec{V}\vec{A}+\vec{u}}$
\begin{align}
    \begin{pmatrix}1&0\\0&1\end{pmatrix}\begin{pmatrix}5\\0\end{pmatrix}+\begin{pmatrix}0\\0\end{pmatrix}\\
    \begin{pmatrix}5\\0\end{pmatrix}    
\end{align}
\end{frame}
\begin{frame}{Solution}
putting in $\eqref{2}$
\begin{align}
    \vec{m}^T\sbrak{\brak{\vec{V}\vec{A}+\vec{u}}\brak{\vec{V}\vec{A}+\vec{u}}^T-\vec{V}g(\vec{A})_1}\vec{m}=0\\
    \vec{m}^T\sbrak{\begin{pmatrix}5\\0\end{pmatrix}\begin{pmatrix}5\\0\end{pmatrix}^T-\begin{pmatrix}1&0\\0&1\end{pmatrix}\times16}\vec{m}=0\\
    \vec{m}^T\sbrak{\begin{pmatrix}9&0\\0&-16\end{pmatrix}}\vec{m}=0\end{align}
    \end{frame}
\begin{frame}{Solution}
    \begin{align}
    \begin{pmatrix}1\\m\end{pmatrix}^T\begin{pmatrix}9&0\\0&-16\end{pmatrix}\begin{pmatrix}1\\m\end{pmatrix}=0\\
    9-16m^2=0\\
    m=\pm \frac{3}{4}\\
    \vec{m}=\begin{pmatrix}1\\\pm \frac{3}{4}\end{pmatrix}
\end{align}
\end{frame}
\begin{frame}{Solution}
Using following formula to find point of contact of tangent
\begin{align}
    \vec{q}_{j}=\brak{\pm r \frac{\vec{n}_j}{\lVert \vec{n}_j \rVert} -\vec{u}}, j=1,2 \label{3} \\
    \vec{q_1}=\brak{\pm 3 \frac{\begin{pmatrix}\frac{3}{4}\\1\end{pmatrix}}{\sqrt{\brak{\frac{3}{4}}^2+1}}} \\
    \vec{q}_1=\pm \begin{pmatrix}\frac{9}{5}\\ \frac{12}{5}\end{pmatrix}\\
    Similarly, \vec{q}_2=\pm\begin{pmatrix}\frac{9}{5}\\ \frac{-12}{5}\end{pmatrix}
\end{align}
\end{frame}
\begin{frame}{Solution}
To take the ones passing through $\vec{A}$ taking $\vec{q}_1$ and $\vec{q}_2$ as
\begin{align}
\vec{q}_1=\begin{pmatrix}\frac{9}{5}\\ \frac{12}{5}\end{pmatrix}\\
\vec{q}_2=\begin{pmatrix}\frac{9}{5}\\ \frac{-12}{5}\end{pmatrix}    
\end{align}
Length of both tangent will be equal and will be
\begin{align}
    \lVert \vec{q_1}-\vec{A} \rVert\\
    \lVert \begin{pmatrix}\frac{9}{5}\\ \frac{12}{5}\end{pmatrix} -\begin{pmatrix}5\\0\end{pmatrix}\rVert\\
    \lVert \begin{pmatrix}\frac{-16}{5}\\ \frac{12}{5}\end{pmatrix}\rVert\\
    =4
\end{align}
\end{frame}

\begin{frame}{Figure}
    \begin{figure}[H]
        \centering
        \includegraphics[width=0.7\columnwidth]{figs/figure.png}
        \caption{}
        \label{fig:placeholder}
    \end{figure}
\end{frame}
\begin{frame}[fragile]
\frametitle{Direct Python}
\begin{lstlisting}
import numpy as np
import matplotlib.pyplot as plt


center = (0, 0)
r_inner = 3.0
r_outer = 5.0


point_A = np.array([5.0, 0.0])


tangent_point_1 = np.array([9/5, 12/5])  # (1.8, 2.4)
tangent_point_2 = np.array([9/5, -12/5]) # (1.8, -2.4)

theta = np.linspace(0, 2 * np.pi, 200)
\end{lstlisting}
\end{frame}
\begin{frame}[fragile]
\frametitle{Direct Python}
\begin{lstlisting}

x_inner = center[0] + r_inner * np.cos(theta)
y_inner = center[1] + r_inner * np.sin(theta)


x_outer = center[0] + r_outer * np.cos(theta)
y_outer = center[1] + r_outer * np.sin(theta)


plt.figure(figsize=(8, 8))
ax = plt.gca()


ax.plot(x_inner, y_inner, label=f'Inner Circle (r={r_inner})', color='blue')
ax.plot(x_outer, y_outer, label=f'Outer Circle (r={r_outer})', color='orange')
\end{lstlisting}
\end{frame}
\begin{frame}[fragile]
\frametitle{Direct Python}
\begin{lstlisting}

ax.plot([point_A[0], tangent_point_1[0]], [point_A[1], tangent_point_1[1]], 'g-', label='Tangent 1')
ax.plot([point_A[0], tangent_point_2[0]], [point_A[1], tangent_point_2[1]], 'g-', label='Tangent 2')

ax.plot([center[0], tangent_point_1[0]], [center[1], tangent_point_1[1]], 'r--', label='Radius to Tangent Point')
ax.plot([center[0], tangent_point_2[0]], [center[1], tangent_point_2[1]], 'r--')


ax.plot(center[0], center[1], 'ko', label='Center (0,0)')
ax.plot(point_A[0], point_A[1], 'go', markersize=8, label=f'Point A {tuple(point_A)}')
ax.plot(tangent_point_1[0], tangent_point_1[1], 'ro', label='Tangent Points')
ax.plot(tangent_point_2[0], tangent_point_2[1], 'ro')
\end{lstlisting}
\end{frame}
\begin{frame}[fragile]
\frametitle{Direct Python}
\begin{lstlisting}
# --- 5. Final Plot Adjustments ---
# Ensure the plot has an equal aspect ratio to make circles look circular
ax.set_aspect('equal', adjustable='box')

# Add titles and labels for clarity
plt.title('Construction of Tangents to Concentric Circles')
plt.xlabel('X-axis')
plt.ylabel('Y-axis')
plt.legend()
plt.grid(True)

# Set axis limits for a nice view
plt.xlim(-6, 6)
plt.ylim(-6, 6)

plt.savefig("figure.png", dpi=300)
# Show the plot
plt.show()
\end{lstlisting}
\end{frame}
\begin{frame}[fragile]
\frametitle{C code}
\begin{lstlisting}
// main.c
#include <stdio.h>
#include <math.h>
double tangent_length(double R, double r) {
    if (R <= r) {
        printf("Invalid input: Outer radius must be greater than inner radius.\n");
        return -1;
    }
    return sqrt((R * R) - (r * r));
}
\end{lstlisting}
\end{frame}
\begin{frame}[fragile]
\frametitle{C code}
\begin{lstlisting}
    return k;
}

int main() {
    double k = find_k();
    printf("The value of k = %.2lf\n", k);
    return 0;
}
\end{lstlisting}
\end{frame}
\begin{frame}[fragile]
\frametitle{Python code with shared object}
\begin{lstlisting}
# main.py
import ctypes
import numpy as np
import matplotlib.pyplot as plt
from math import acos, sqrt

# Load the shared object file (compiled C code)
lib = ctypes.CDLL("./main.so")

# Define argument and return types for the C function
lib.tangent_length.argtypes = [ctypes.c_double, ctypes.c_double]
lib.tangent_length.restype = ctypes.c_double

# Given radii
r = 3.0
R = 5.0
\end{lstlisting}
\end{frame}
\begin{frame}[fragile]
\frametitle{Python code with shared object}
\begin{lstlisting}
# Call C function to get tangent length
tangent_len_c = lib.tangent_length(R, r)

if tangent_len_c < 0:
    raise ValueError("Invalid radii: R must be greater than r")

print(f"[C Function] Tangent length (√(R²−r²)) = {tangent_len_c:.3f} cm")

# Now use geometry to verify and plot
theta = np.deg2rad(60)
O = np.array([0.0, 0.0])
P = np.array([R * np.cos(theta), R * np.sin(theta)])
\end{lstlisting}
\end{frame}
\begin{frame}[fragile]
\frametitle{Python code with shared object}
\begin{lstlisting}
# Compute tangent points
beta = acos(r / R)
T1 = np.array([r * np.cos(theta + beta), r * np.sin(theta + beta)])
T2 = np.array([r * np.cos(theta - beta), r * np.sin(theta - beta)])

# Verify length geometrically
L1 = np.linalg.norm(P - T1)
L2 = np.linalg.norm(P - T2)

print(f"[Python Geometry] PT1 = {L1:.3f} cm, PT2 = {L2:.3f} cm")

# Plot
fig, ax = plt.subplots(figsize=(6,6))
ax.set_aspect('equal', 'box')
ax.set_title("Tangents from a Point on Outer Circle to Inner Circle")
\end{lstlisting}
\end{frame}
\begin{frame}[fragile]
\frametitle{Python code with shared object}
\begin{lstlisting}
# Circles
outer = plt.Circle((0,0), R, fill=False, color='blue', linestyle='--', label='Outer Circle (R=5 cm)')
inner = plt.Circle((0,0), r, fill=False, color='green', linestyle='--', label='Inner Circle (r=3 cm)')
ax.add_patch(outer)
ax.add_patch(inner)

# Lines
ax.plot([0, P[0]], [0, P[1]], 'k-', label='OP')
ax.plot([P[0], T1[0]], [P[1], T1[1]], 'r-', label='Tangent 1')
ax.plot([P[0], T2[0]], [P[1], T2[1]], 'r-')
ax.plot([0, T1[0]], [0, T1[1]], 'gray', linestyle=':')
ax.plot([0, T2[0]], [0, T2[1]], 'gray', linestyle=':')
\end{lstlisting}
\end{frame}
\begin{frame}[fragile]
\frametitle{Python code with shared object}
\begin{lstlisting}
# Points
ax.plot(0, 0, 'ko')
ax.plot(P[0], P[1], 'ro')
ax.plot(T1[0], T1[1], 'go')
ax.plot(T2[0], T2[1], 'go')

ax.text(0.2, -0.3, 'O', fontsize=10)
ax.text(P[0]*1.05, P[1]*1.05, 'P', fontsize=10)
ax.text(T1[0]*1.05, T1[1]*1.05, 'T1', fontsize=10)
ax.text(T2[0]*1.05, T2[1]*1.05, 'T2', fontsize=10)

ax.legend()
ax.grid(True)
plt.show()
\end{lstlisting}
\end{frame}

\end{document}