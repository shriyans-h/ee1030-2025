\documentclass{beamer}
\usepackage[utf8]{inputenc}

\usetheme{Madrid}
\usecolortheme{default}
\usepackage{amsmath,amssymb,amsfonts,amsthm}
\usepackage{txfonts}
\usepackage{tkz-euclide}
\usepackage{listings}
\usepackage{adjustbox}
\usepackage{array}
\usepackage{tabularx}
\usepackage{gvv}
\usepackage{lmodern}
\usepackage{circuitikz}
\usepackage{tikz}
\usepackage{graphicx}

\setbeamertemplate{page number in head/foot}[totalframenumber]

\usepackage{tcolorbox}
\tcbuselibrary{minted,breakable,xparse,skins}



\definecolor{bg}{gray}{0.95}
\DeclareTCBListing{mintedbox}{O{}m!O{}}{%
	breakable=true,
	listing engine=minted,
	listing only,
	minted language=#2,
	minted style=default,
	minted options={%
		linenos,
		gobble=0,
		breaklines=true,
		breakafter=,,
		fontsize=\small,
		numbersep=8pt,
		#1},
	boxsep=0pt,
	left skip=0pt,
	right skip=0pt,
	left=25pt,
	right=0pt,
	top=3pt,
	bottom=3pt,
	arc=5pt,
	leftrule=0pt,
	rightrule=0pt,
	bottomrule=2pt,
	toprule=2pt,
	colback=bg,
	colframe=orange!70,
	enhanced,
	overlay={%
		\begin{tcbclipinterior}
			\fill[orange!20!white] (frame.south west) rectangle ([xshift=20pt]frame.north west);
	\end{tcbclipinterior}},
	#3,
}
\lstset{
	language=C,
	basicstyle=\ttfamily\small,
	keywordstyle=\color{blue},
	stringstyle=\color{orange},
	commentstyle=\color{green!60!black},
	numbers=left,
	numberstyle=\tiny\color{gray},
	breaklines=true,
	showstringspaces=false,
}
\begin{document}

\title 
{5.2.43}
\date{26 September,2025}

\author 
{Naman Kumar-EE25BTECH11041}
\graphicspath{./figs}


\frame{\titlepage}
\begin{frame}{Question)}
Solve the linear equation:
\begin{align}
    6x + 3y = 6xy \label{eq1}\\2x + 4y = 5xy \label{eq2}
\end{align}
\end{frame}
\begin{frame}{Solution}
General equation of conic
\begin{align}
    \vec{x}^T\vec{V}\vec{x}+2\vec{u}^T\vec{x}+f
\end{align}
Given set of equations in the form of general conic can be written as
\begin{align}
\vec{x}^T\begin{pmatrix}0&3\\3&0\end{pmatrix}\vec{x}+2\begin{pmatrix}-3\\-1.5\end{pmatrix}^T\vec{x}=0\\
\vec{x}^T\vec{V_1}\vec{x}+2\vec{u_1}^T\vec{x}=0
\end{align}
Similarly
\begin{align}
    \vec{x}^T\begin{pmatrix}0&2.5\\2.5&0\end{pmatrix}\vec{x}+2\begin{pmatrix}-1\\-2\end{pmatrix}^T\vec{x}=0\\
\vec{x}^T\vec{V_2}\vec{x}+2\vec{u_2}^T\vec{x}=0
\end{align}
\end{frame}
\begin{frame}{Solution}
Intersection of two conic
\begin{align}
    \vec{x}^T(\vec{V_1}+\mu \vec{V_2})\vec{x}+2(\vec{u_1}+\mu \vec{u_2})^T\vec{x}=0 \label{comb}
\end{align}
General equation of conic represent pair of lines if
\begin{align}
    \begin{vmatrix}\vec{V}&\vec{u}\\ \vec{u}^T&f\end{vmatrix}=0
\end{align}
From $\eqref{comb}$
\begin{align}
    \begin{vmatrix}\vec{V_1}+\mu \vec{V_2}&\vec{u_1}+\mu \vec{u_2}\\ (\vec{u_1}+\mu \vec{u_2})^T&0\end{vmatrix}=0 \label{1}
\end{align}
\end{frame}
\begin{frame}{Solution}
Here
\begin{align}
    \vec{A}=\vec{V_1}+\mu \vec{V_2}=\begin{pmatrix}0&3+2.5\mu\\3+2.5\mu&0\end{pmatrix} \label{a} \\
    \vec{A}=\begin{pmatrix}a_{11}&a_{12}\\a_{21}&a_{22}\end{pmatrix}\\ \\
    \vec{B}=\vec{u_1}+\mu \vec{u_2}=\begin{pmatrix}-3+\mu(-1)\\-1.5+\mu(-2)\end{pmatrix} \label{b}\\ \vec{B}=\begin{pmatrix}b_{1}\\b_{2}\end{pmatrix}
\end{align}
\end{frame}
\begin{frame}{Solution}
Putting values in $\eqref{1}$
\begin{align}
    \begin{vmatrix}a_{11}&a_{12}&b_1\\a_{21}&a_{22}&b_2\\b_1&b_2&0\end{vmatrix}\\
    -b_2(b_2a_{11}-b_1a_{21})+b_1(b_2a_{12}-b_1a_{22})
\end{align}
Putting values from $\eqref{a}$ $\eqref{b}$
\begin{align}
    (-3-2.5\mu)(3+\mu)(1.5+2\mu)\\
    \mu=\frac{-6}{5},-3,\frac{-3}{4}
\end{align}
\end{frame}
\begin{frame}{Solution}
Case 1:$\mu=-3$ in $\eqref{comb}$
\begin{align}
    \vec{x}^T\begin{pmatrix}0&4.5\\4.5&0\end{pmatrix}\vec{x}+2\begin{pmatrix}0\\-4.5\end{pmatrix}^T\vec{x}=0\\
    \begin{pmatrix}x\\y\end{pmatrix}^T\begin{pmatrix}0&4.5\\4.5&0\end{pmatrix}\begin{pmatrix}x\\y\end{pmatrix}+2\begin{pmatrix}0\\-4.5\end{pmatrix}^T\begin{pmatrix}x\\y\end{pmatrix}=0\\
    2\times4.5xy+2(0-4.5y)\\=9xy-9y=9y(x-1)=0\\
    y=0,x=1
\end{align}
\end{frame}
\begin{frame}{Solution}
Case 2: $\mu=\frac{-6}{5}$ in $\eqref{comb}$
\begin{align}
    \vec{x}^T\begin{pmatrix}0&0\\0&0\end{pmatrix}\vec{x}+2\begin{pmatrix}1.8\\-0.9\end{pmatrix}^T\vec{x}=0\\
    \begin{pmatrix}x\\y\end{pmatrix}^T\begin{pmatrix}0&0\\0&0\end{pmatrix}\begin{pmatrix}x\\y\end{pmatrix}+2\begin{pmatrix}1.8\\-0.9\end{pmatrix}^T\begin{pmatrix}x\\y\end{pmatrix}=0\\
    2x-y=0
\end{align}
\end{frame}
\begin{frame}{Solution}
Case 3: $\mu=\frac{-3}{4}$ in $\eqref{comb}$
\begin{align}
    \vec{x}^T\begin{pmatrix}0&-1.125\\-1.125&0\end{pmatrix}\vec{x}+2\begin{pmatrix}2.25\\0\end{pmatrix}^T\vec{x}=0\\
    \begin{pmatrix}x\\y\end{pmatrix}^T\begin{pmatrix}0&-1.125\\-1.125&0\end{pmatrix}\begin{pmatrix}x\\y\end{pmatrix}+2\begin{pmatrix}2.25\\0\end{pmatrix}^T\begin{pmatrix}x\\y\end{pmatrix}=0\\
    -2.25xy+4.5=2.25x(2-y)=0\\
    x=0,y=2
\end{align}
\end{frame}
\begin{frame}{Solution}
Now checking point of intersection with conic\\
from $\mu=-3$ factors y=0 and x=1
\begin{itemize}
    \item y=0 in $\eqref{eq1}$ 6x=6x.0 $\implies$ x=0 and in $\eqref{eq2}$ 2x=0, so point (0,0)
    \item x=1 in $\eqref{eq1}$ 6+3y=6y $\implies$ y=2, so (1,2)
\end{itemize}
similarly for $\mu=\frac{-3}{4}$ factors are x=0 and y=2
\begin{itemize}
    \item x=0 gives (0,0)
    \item y=2 gives (1,2)
\end{itemize}
And for $\mu=\frac{-6}{5}$ line y=2x
\begin{itemize}
    \item put y=2x in $\eqref{eq1}$ $12x=12x^2\implies x=0 ,1$ so (0,0),(1,2)
\end{itemize}
All three cases have same points ,\\so Points are (0,0) and (1,2)
\end{frame}
\begin{frame}{Figure}
    \begin{figure}[H]
        \centering
        \includegraphics[width=0.6\columnwidth]{figs/figure.png}
        \caption{}
        \label{fig:placeholder}
    \end{figure}
\end{frame}
\begin{frame}[fragile]
\frametitle{Direct Python}
\begin{lstlisting}
import numpy as np
import matplotlib.pyplot as plt

plt.figure(figsize=(5,5), dpi=200)
plt.xlim(-1,4)
plt.ylim(-1,4)
A=np.array([[3,6],[4,2]])
c=np.array([6,5])

an=np.linalg.inv(A)
ans=np.dot(an,c)

Ans=np.linalg.solve(A,c)
\end{lstlisting}
\end{frame}
\begin{frame}[fragile]
\frametitle{Direct Python}
\begin{lstlisting}
print("x=",Ans[0],"y=",Ans[1])
a=1/ans[0]
b=1/ans[1]

x = np.array([a,b]).reshape(-1,1)

x1= np.linspace(-1,4,100)
l1= (6*x1)/(6*x1-3)
l2= (2*x1)/(5*x1-4)

plt.plot(x1,l1, color='blue', label="Line 1")
\end{lstlisting}
\end{frame}
\begin{frame}[fragile]
\frametitle{Direct Python}
\begin{lstlisting}
plt.plot(x1,l2, color='orange', label="Line 2")
plt.scatter(1,2, c='r', zorder=5)
plt.text(1,2,"(1,2)")
plt.text(0,0,"Origin",)

plt.xlabel("x")
plt.ylabel("y")
plt.grid()
plt.legend()
plt.savefig("figure.png", dpi=200)
plt.show()
\end{lstlisting}
\end{frame}
\begin{frame}[fragile]
\frametitle{C code}
\begin{lstlisting}
#include <stdio.h>

typedef struct {
    double x;
    double y;
} Point;

typedef struct {
    Point sols[2];
    int count;
} SolutionSet;

// Solve 6x+3y=6xy, 2x+4y=5xy
SolutionSet solve_equations() {
    SolutionSet S;
    S.count = 0;
\end{lstlisting}
\end{frame}
\begin{frame}[fragile]
\frametitle{C code}
\begin{lstlisting}
    // Solution 1: (0,0)
    S.sols[S.count].x = 0;
    S.sols[S.count].y = 0;
    S.count++;

    // Solution 2: (1,2)
    S.sols[S.count].x = 1;
    S.sols[S.count].y = 2;
    S.count++;

    return S;
}
\end{lstlisting}
\end{frame}
\begin{frame}[fragile]
\frametitle{C code}
\begin{lstlisting}
#ifdef TEST_C
int main(){
    SolutionSet S = solve_equations();
    for(int i=0; i<S.count; i++){
        printf("Solution %d: (%.2f, %.2f)\n", i+1, S.sols[i].x, S.sols[i].y);
    }
    return 0;
}
#endif
\end{lstlisting}
\end{frame}
\begin{frame}[fragile]
\frametitle{Python code with shared object}
\begin{lstlisting}
# main.py
import ctypes
from ctypes import Structure, c_double, c_int
import matplotlib.pyplot as plt

class Point(Structure):
    _fields_ = [("x", c_double), ("y", c_double)]

class SolutionSet(Structure):
    _fields_ = [("sols", Point * 2), ("count", c_int)]

# Load C lib
lib = ctypes.CDLL("./libsolver.so")
lib.solve_equations.restype = SolutionSet
\end{lstlisting}
\end{frame}
\begin{frame}[fragile]
\frametitle{Python code with shared object}
\begin{lstlisting}
# Call function
solutions = lib.solve_equations()
print(f"Found {solutions.count} solutions:")
for i in range(solutions.count):
    x, y = solutions.sols[i].x, solutions.sols[i].y
    print(f"Solution {i+1}: ({x}, {y})")

# Plot equations and solutions
import numpy as np
x_vals = np.linspace(-1, 3, 400)
y1 = (6*x_vals)/(6*x_vals - 3)   # from eqn (1)
y2 = (2*x_vals)/(5*x_vals - 4)   # from eqn (2)
\end{lstlisting}
\end{frame}
\begin{frame}[fragile]
\frametitle{Python code with shared object}
\begin{lstlisting}
plt.figure(figsize=(6,6))
plt.plot(x_vals, y1, label="6x+3y=6xy")
plt.plot(x_vals, y2, label="2x+4y=5xy")
for i in range(solutions.count):
    x, y = solutions.sols[i].x, solutions.sols[i].y
    plt.scatter(x, y, c='r', zorder=5)
    plt.text(x, y, f"({x:.0f},{y:.0f})", fontsize=10)

plt.ylim(-1,4)
plt.grid(True)
plt.legend()
plt.xlabel("x")
plt.ylabel("y")
plt.title("Solutions of nonlinear system")
plt.show()


\end{lstlisting}
\end{frame}

\end{document}