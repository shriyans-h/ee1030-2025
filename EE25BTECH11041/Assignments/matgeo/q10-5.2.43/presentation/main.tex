\documentclass{beamer}
\usepackage[utf8]{inputenc}

\usetheme{Madrid}
\usecolortheme{default}
\usepackage{amsmath,amssymb,amsfonts,amsthm}
\usepackage{txfonts}
\usepackage{tkz-euclide}
\usepackage{listings}
\usepackage{adjustbox}
\usepackage{array}
\usepackage{tabularx}
\usepackage{gvv}
\usepackage{lmodern}
\usepackage{circuitikz}
\usepackage{tikz}
\usepackage{graphicx}

\setbeamertemplate{page number in head/foot}[totalframenumber]

\usepackage{tcolorbox}
\tcbuselibrary{minted,breakable,xparse,skins}



\definecolor{bg}{gray}{0.95}
\DeclareTCBListing{mintedbox}{O{}m!O{}}{%
	breakable=true,
	listing engine=minted,
	listing only,
	minted language=#2,
	minted style=default,
	minted options={%
		linenos,
		gobble=0,
		breaklines=true,
		breakafter=,,
		fontsize=\small,
		numbersep=8pt,
		#1},
	boxsep=0pt,
	left skip=0pt,
	right skip=0pt,
	left=25pt,
	right=0pt,
	top=3pt,
	bottom=3pt,
	arc=5pt,
	leftrule=0pt,
	rightrule=0pt,
	bottomrule=2pt,
	toprule=2pt,
	colback=bg,
	colframe=orange!70,
	enhanced,
	overlay={%
		\begin{tcbclipinterior}
			\fill[orange!20!white] (frame.south west) rectangle ([xshift=20pt]frame.north west);
	\end{tcbclipinterior}},
	#3,
}
\lstset{
	language=C,
	basicstyle=\ttfamily\small,
	keywordstyle=\color{blue},
	stringstyle=\color{orange},
	commentstyle=\color{green!60!black},
	numbers=left,
	numberstyle=\tiny\color{gray},
	breaklines=true,
	showstringspaces=false,
}
\begin{document}

\title 
{5.2.43}
\date{26 September,2025}

\author 
{Naman Kumar-EE25BTECH11041}
\graphicspath{./figs}


\frame{\titlepage}
\begin{frame}{Question)}
Solve the linear equation:
\begin{align}
    6x + 3y = 6xy\\2x + 4y = 5xy
\end{align}
\end{frame}
\begin{frame}{Solution}
Dividing both equations with $xy$
\begin{align}
    \frac{6}{y}+\frac{3}{x}=6\\
    \frac{2}{y}+\frac{4}{x}=5
\end{align}
Let
\begin{align}
    \frac{1}{x}=a,\frac{1}{y}=b
\end{align}
\end{frame}
\begin{frame}{Solution}
So, new equations
\begin{align}
    3a+6b=6\\
    4a+2b=5\\
    \begin{pmatrix}3&6\\4&2\end{pmatrix}\begin{pmatrix}a\\b\end{pmatrix}\begin{pmatrix}6\\5\end{pmatrix}\label{1} \\
    \vec{A}\vec{x}=\vec{c}
\end{align}
Gaussian elimination on $\vec{A}$
\begin{align}
    \begin{amatrix}{2}3&6&6\\4&2&5\end{amatrix}\xrightarrow[]{R_2-\frac{4R_1}{3}}\begin{amatrix}{2}3&6&6\\0&-6&-3\end{amatrix}\\
    \xrightarrow[]{\frac{R_2}{-6}}\begin{amatrix}{2}3&6&6\\0&1&\frac{1}{2}\end{amatrix}
\end{align}
\end{frame}
\begin{frame}{Solution}
Therefore,by putting values in $\eqref{1}$
\begin{align}
    \begin{pmatrix}a\\b\end{pmatrix}=\begin{pmatrix}1 \\ \frac{1}{2}\end{pmatrix}
\end{align}
For x,y
\begin{align}
    \begin{pmatrix}\frac{1}{x}\\\frac{1}{y}\end{pmatrix}=\begin{pmatrix}1 \\ \frac{1}{2}\end{pmatrix}\\
    \begin{pmatrix}x\\y\end{pmatrix}=\begin{pmatrix}1 \\ 2\end{pmatrix}
\end{align}
\end{frame}
\begin{frame}{Figure}
    \begin{figure}[H]
        \centering
        \includegraphics[width=0.6\columnwidth]{figs/figure.png}
        \caption{}
        \label{fig:placeholder}
    \end{figure}
\end{frame}
\begin{frame}[fragile]
\frametitle{Direct Python}
\begin{lstlisting}
import numpy as np
import matplotlib.pyplot as plt

plt.figure(figsize=(5,5), dpi=200)
plt.xlim(-1,4)
plt.ylim(-1,4)
A=np.array([[3,6],[4,2]])
c=np.array([6,5])

an=np.linalg.inv(A)
ans=np.dot(an,c)

Ans=np.linalg.solve(A,c)
\end{lstlisting}
\end{frame}
\begin{frame}[fragile]
\frametitle{Direct Python}
\begin{lstlisting}
print("x=",Ans[0],"y=",Ans[1])
a=1/ans[0]
b=1/ans[1]

x = np.array([a,b]).reshape(-1,1)

x1= np.linspace(-1,4,100)
l1= (6*x1)/(6*x1-3)
l2= (2*x1)/(5*x1-4)

plt.plot(x1,l1, color='blue', label="Line 1")
\end{lstlisting}
\end{frame}
\begin{frame}[fragile]
\frametitle{Direct Python}
\begin{lstlisting}
plt.plot(x1,l2, color='orange', label="Line 2")
plt.scatter(1,2, c='r', zorder=5)
plt.text(1,2,"(1,2)")
plt.text(0,0,"Origin",)

plt.xlabel("x")
plt.ylabel("y")
plt.grid()
plt.legend()
plt.savefig("figure.png", dpi=200)
plt.show()
\end{lstlisting}
\end{frame}
\begin{frame}[fragile]
\frametitle{C code}
\begin{lstlisting}
#include <stdio.h>

typedef struct {
    double x;
    double y;
} Point;

typedef struct {
    Point sols[2];
    int count;
} SolutionSet;

// Solve 6x+3y=6xy, 2x+4y=5xy
SolutionSet solve_equations() {
    SolutionSet S;
    S.count = 0;
\end{lstlisting}
\end{frame}
\begin{frame}[fragile]
\frametitle{C code}
\begin{lstlisting}
    // Solution 1: (0,0)
    S.sols[S.count].x = 0;
    S.sols[S.count].y = 0;
    S.count++;

    // Solution 2: (1,2)
    S.sols[S.count].x = 1;
    S.sols[S.count].y = 2;
    S.count++;

    return S;
}
\end{lstlisting}
\end{frame}
\begin{frame}[fragile]
\frametitle{C code}
\begin{lstlisting}
#ifdef TEST_C
int main(){
    SolutionSet S = solve_equations();
    for(int i=0; i<S.count; i++){
        printf("Solution %d: (%.2f, %.2f)\n", i+1, S.sols[i].x, S.sols[i].y);
    }
    return 0;
}
#endif
\end{lstlisting}
\end{frame}
\begin{frame}[fragile]
\frametitle{Python code with shared object}
\begin{lstlisting}
# main.py
import ctypes
from ctypes import Structure, c_double, c_int
import matplotlib.pyplot as plt

class Point(Structure):
    _fields_ = [("x", c_double), ("y", c_double)]

class SolutionSet(Structure):
    _fields_ = [("sols", Point * 2), ("count", c_int)]

# Load C lib
lib = ctypes.CDLL("./libsolver.so")
lib.solve_equations.restype = SolutionSet
\end{lstlisting}
\end{frame}
\begin{frame}[fragile]
\frametitle{Python code with shared object}
\begin{lstlisting}
# Call function
solutions = lib.solve_equations()
print(f"Found {solutions.count} solutions:")
for i in range(solutions.count):
    x, y = solutions.sols[i].x, solutions.sols[i].y
    print(f"Solution {i+1}: ({x}, {y})")

# Plot equations and solutions
import numpy as np
x_vals = np.linspace(-1, 3, 400)
y1 = (6*x_vals)/(6*x_vals - 3)   # from eqn (1)
y2 = (2*x_vals)/(5*x_vals - 4)   # from eqn (2)
\end{lstlisting}
\end{frame}
\begin{frame}[fragile]
\frametitle{Python code with shared object}
\begin{lstlisting}
plt.figure(figsize=(6,6))
plt.plot(x_vals, y1, label="6x+3y=6xy")
plt.plot(x_vals, y2, label="2x+4y=5xy")
for i in range(solutions.count):
    x, y = solutions.sols[i].x, solutions.sols[i].y
    plt.scatter(x, y, c='r', zorder=5)
    plt.text(x, y, f"({x:.0f},{y:.0f})", fontsize=10)

plt.ylim(-1,4)
plt.grid(True)
plt.legend()
plt.xlabel("x")
plt.ylabel("y")
plt.title("Solutions of nonlinear system")
plt.show()


\end{lstlisting}
\end{frame}

\end{document}