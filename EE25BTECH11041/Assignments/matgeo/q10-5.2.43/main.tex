\let\negmedspace\undefined
\let\negthickspace\undefined
\documentclass[a5paper,10pt]{article}
\usepackage[margin=10mm]{geometry}
%\usepackage{lmodern} % Ensure lmodern is loaded for pdflatex
\usepackage{tfrupee} % Include tfrupee package

\setlength{\headheight}{1cm} % Set the height of the header box
\setlength{\headsep}{0mm}     % Set the distance between the header box and the top of the text

\usepackage{gvv-book}
\usepackage{gvv}
\usepackage{cite}
\usepackage{amsmath,amssymb,amsfonts,amsthm}
\usepackage{algorithmic}
\usepackage{graphicx}
\usepackage{textcomp}
\usepackage{xcolor}
\usepackage{txfonts}
\usepackage{listings}
\usepackage{enumitem}
\usepackage{mathtools}
\usepackage{gensymb}
\usepackage{comment}
\usepackage[breaklinks=true]{hyperref}
\usepackage{tkz-euclide} 
\usepackage{listings}
% \usepackage{gvv}                                        
\def\inputGnumericTable{}                                 
\usepackage[latin1]{inputenc}                                
\usepackage{color}                                            
\usepackage{array}                                            
\usepackage{longtable}                                       
\usepackage{calc}                                             
\usepackage{multirow}                                         
\usepackage{hhline}                                           
\usepackage{ifthen}                                           
\usepackage{lscape}
\usepackage{circuitikz}



\author{EE25BTECH11041-Naman Kumar }
\graphicspath{./figs/}

\begin{document}
\begin{center}
    \huge{5.2.43}\\
    \large{EE25BTECH11041 - Naman Kumar}
\end{center}
Question:\\
Solve the linear equation:
\begin{align}
    6x + 3y = 6xy \label{eq1}\\2x + 4y = 5xy \label{eq2}
\end{align} \\
\solution \\
General equation of conic
\begin{align}
    \vec{x}^T\vec{V}\vec{x}+2\vec{u}^T\vec{x}+f
\end{align}
Given set of equations in the form of general conic can be written as
\begin{align}
\vec{x}^T\begin{pmatrix}0&3\\3&0\end{pmatrix}\vec{x}+2\begin{pmatrix}-3\\-1.5\end{pmatrix}^T\vec{x}=0\\
\vec{x}^T\vec{V_1}\vec{x}+2\vec{u_1}^T\vec{x}=0
\end{align}
Similarly
\begin{align}
    \vec{x}^T\begin{pmatrix}0&2.5\\2.5&0\end{pmatrix}\vec{x}+2\begin{pmatrix}-1\\-2\end{pmatrix}^T\vec{x}=0\\
\vec{x}^T\vec{V_2}\vec{x}+2\vec{u_2}^T\vec{x}=0
\end{align}
Intersection of two conic
\begin{align}
    \vec{x}^T(\vec{V_1}+\mu \vec{V_2})\vec{x}+2(\vec{u_1}+\mu \vec{u_2})^T\vec{x}=0 \label{comb}
\end{align}
General equation of conic represent pair of lines if
\begin{align}
    \begin{vmatrix}\vec{V}&\vec{u}\\ \vec{u}^T&f\end{vmatrix}=0
\end{align}
From $\eqref{comb}$
\begin{align}
    \begin{vmatrix}\vec{V_1}+\mu \vec{V_2}&\vec{u_1}+\mu \vec{u_2}\\ (\vec{u_1}+\mu \vec{u_2})^T&0\end{vmatrix}=0 \label{1}
\end{align}
Here
\begin{align}
    \vec{A}=\vec{V_1}+\mu \vec{V_2}=\begin{pmatrix}0&3+2.5\mu\\3+2.5\mu&0\end{pmatrix} \label{a} \\
    \vec{A}=\begin{pmatrix}a_{11}&a_{12}\\a_{21}&a_{22}\end{pmatrix}\\ \\
    \vec{B}=\vec{u_1}+\mu \vec{u_2}=\begin{pmatrix}-3+\mu(-1)\\-1.5+\mu(-2)\end{pmatrix} \label{b}\\ \vec{B}=\begin{pmatrix}b_{1}\\b_{2}\end{pmatrix}
\end{align}
Putting values in $\eqref{1}$
\begin{align}
    \begin{vmatrix}a_{11}&a_{12}&b_1\\a_{21}&a_{22}&b_2\\b_1&b_2&0\end{vmatrix}\\
    -b_2(b_2a_{11}-b_1a_{21})+b_1(b_2a_{12}-b_1a_{22})
\end{align}
Putting values from $\eqref{a}$ $\eqref{b}$
\begin{align}
    (-3-2.5\mu)(3+\mu)(1.5+2\mu)\\
    \mu=\frac{-6}{5},-3,\frac{-3}{4}
\end{align}
Case 1:$\mu=-3$ in $\eqref{comb}$
\begin{align}
    \vec{x}^T\begin{pmatrix}0&4.5\\4.5&0\end{pmatrix}\vec{x}+2\begin{pmatrix}0\\-4.5\end{pmatrix}^T\vec{x}=0\\
    \begin{pmatrix}x\\y\end{pmatrix}^T\begin{pmatrix}0&4.5\\4.5&0\end{pmatrix}\begin{pmatrix}x\\y\end{pmatrix}+2\begin{pmatrix}0\\-4.5\end{pmatrix}^T\begin{pmatrix}x\\y\end{pmatrix}=0\\
    2\times4.5xy+2(0-4.5y)\\=9xy-9y=9y(x-1)=0\\
    y=0,x=1
\end{align}
Case 2: $\mu=\frac{-6}{5}$ in $\eqref{comb}$
\begin{align}
    \vec{x}^T\begin{pmatrix}0&0\\0&0\end{pmatrix}\vec{x}+2\begin{pmatrix}1.8\\-0.9\end{pmatrix}^T\vec{x}=0\\
    \begin{pmatrix}x\\y\end{pmatrix}^T\begin{pmatrix}0&0\\0&0\end{pmatrix}\begin{pmatrix}x\\y\end{pmatrix}+2\begin{pmatrix}1.8\\-0.9\end{pmatrix}^T\begin{pmatrix}x\\y\end{pmatrix}=0\\
    2x-y=0
\end{align}
Case 3: $\mu=\frac{-3}{4}$ in $\eqref{comb}$
\begin{align}
    \vec{x}^T\begin{pmatrix}0&-1.125\\-1.125&0\end{pmatrix}\vec{x}+2\begin{pmatrix}2.25\\0\end{pmatrix}^T\vec{x}=0\\
    \begin{pmatrix}x\\y\end{pmatrix}^T\begin{pmatrix}0&-1.125\\-1.125&0\end{pmatrix}\begin{pmatrix}x\\y\end{pmatrix}+2\begin{pmatrix}2.25\\0\end{pmatrix}^T\begin{pmatrix}x\\y\end{pmatrix}=0\\
    -2.25xy+4.5=2.25x(2-y)=0\\
    x=0,y=2
\end{align}
Now checking point of intersection with conic\\
from $\mu=-3$ factors y=0 and x=1
\begin{itemize}
    \item y=0 in $\eqref{eq1}$ 6x=6x.0 $\implies$ x=0 and in $\eqref{eq2}$ 2x=0, so point (0,0)
    \item x=1 in $\eqref{eq1}$ 6+3y=6y $\implies$ y=2, so (1,2)
\end{itemize}
similarly for $\mu=\frac{-3}{4}$ factors are x=0 and y=2
\begin{itemize}
    \item x=0 gives (0,0)
    \item y=2 gives (1,2)
\end{itemize}
And for $\mu=\frac{-6}{5}$ line y=2x
\begin{itemize}
    \item put y=2x in $\eqref{eq1}$ $12x=12x^2\implies x=0 ,1$ so (0,0),(1,2)
\end{itemize}
All three cases have same points ,\\so Points are (0,0) and (1,2)
\newpage
\begin{figure}[H]
    \centering
    \includegraphics[width=\columnwidth]{figs/figure.png}
    \caption{}
    \label{fig:placeholder}
\end{figure}
\end{document}
