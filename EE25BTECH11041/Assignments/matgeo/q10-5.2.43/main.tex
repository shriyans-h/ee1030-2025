\let\negmedspace\undefined
\let\negthickspace\undefined
\documentclass[journal]{IEEEtran}
\usepackage[a5paper, margin=10mm, onecolumn]{geometry}
%\usepackage{lmodern} % Ensure lmodern is loaded for pdflatex
\usepackage{tfrupee} % Include tfrupee package

\setlength{\headheight}{1cm} % Set the height of the header box
\setlength{\headsep}{0mm}     % Set the distance between the header box and the top of the text

\usepackage{gvv-book}
\usepackage{gvv}
\usepackage{cite}
\usepackage{amsmath,amssymb,amsfonts,amsthm}
\usepackage{algorithmic}
\usepackage{graphicx}
\usepackage{textcomp}
\usepackage{xcolor}
\usepackage{txfonts}
\usepackage{listings}
\usepackage{enumitem}
\usepackage{mathtools}
\usepackage{gensymb}
\usepackage{comment}
\usepackage[breaklinks=true]{hyperref}
\usepackage{tkz-euclide} 
\usepackage{listings}
% \usepackage{gvv}                                        
\def\inputGnumericTable{}                                 
\usepackage[latin1]{inputenc}                                
\usepackage{color}                                            
\usepackage{array}                                            
\usepackage{longtable}                                       
\usepackage{calc}                                             
\usepackage{multirow}                                         
\usepackage{hhline}                                           
\usepackage{ifthen}                                           
\usepackage{lscape}
\usepackage{circuitikz}



\author{EE25BTECH11041-Naman Kumar }
\graphicspath{./figs/}

\begin{document}
\begin{center}
    \huge{5.2.43}\\
    \large{EE25BTECH11041 - Naman Kumar}
\end{center}
Question:\\
Solve the linear equation:
\begin{align}
    6x + 3y = 6xy\\2x + 4y = 5xy
\end{align} \\
\solution \\
Dividing both equations with $xy$
\begin{align}
    \frac{6}{y}+\frac{3}{x}=6\\
    \frac{2}{y}+\frac{4}{x}=5
\end{align}
Let
\begin{align}
    \frac{1}{x}=a,\frac{1}{y}=b
\end{align}
So, new equations
\begin{align}
    3a+6b=6\\
    4a+2b=5\\
    \begin{pmatrix}3&6\\4&2\end{pmatrix}\begin{pmatrix}a\\b\end{pmatrix}\begin{pmatrix}6\\5\end{pmatrix}\label{1} \\
    \vec{A}\vec{x}=\vec{c}
\end{align}
Gaussian elimination on $\vec{A}$
\begin{align}
    \begin{amatrix}{2}3&6&6\\4&2&5\end{amatrix}\xrightarrow[]{R_2-\frac{4R_1}{3}}\begin{amatrix}{2}3&6&6\\0&-6&-3\end{amatrix}\\
    \xrightarrow[]{\frac{R_2}{-6}}\begin{amatrix}{2}3&6&6\\0&1&\frac{1}{2}\end{amatrix}
\end{align}
Therefore,by putting values in $\eqref{1}$
\begin{align}
    \begin{pmatrix}a\\b\end{pmatrix}=\begin{pmatrix}1 \\ \frac{1}{2}\end{pmatrix}
\end{align}
For x,y
\begin{align}
    \begin{pmatrix}\frac{1}{x}\\\frac{1}{y}\end{pmatrix}=\begin{pmatrix}1 \\ \frac{1}{2}\end{pmatrix}\\
    \begin{pmatrix}x\\y\end{pmatrix}=\begin{pmatrix}1 \\ 2\end{pmatrix}
\end{align}
\begin{figure}[H]
    \centering
    \includegraphics[width=\columnwidth]{figs/figure.png}
    \caption{}
    \label{fig:placeholder}
\end{figure}


\end{document}
