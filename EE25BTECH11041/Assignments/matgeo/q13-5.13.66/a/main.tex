\let\negmedspace\undefined
\let\negthickspace\undefined
\documentclass[journal]{IEEEtran}
\usepackage[a5paper, margin=10mm, onecolumn]{geometry}
%\usepackage{lmodern} % Ensure lmodern is loaded for pdflatex
\usepackage{tfrupee} % Include tfrupee package

\setlength{\headheight}{1cm} % Set the height of the header box
\setlength{\headsep}{0mm}     % Set the distance between the header box and the top of the text

\usepackage{gvv-book}
\usepackage{gvv}
\usepackage{cite}
\usepackage{amsmath,amssymb,amsfonts,amsthm}
\usepackage{algorithmic}
\usepackage{graphicx}
\usepackage{textcomp}
\usepackage{xcolor}
\usepackage{txfonts}
\usepackage{listings}
\usepackage{enumitem}
\usepackage{mathtools}
\usepackage{gensymb}
\usepackage{comment}
\usepackage[breaklinks=true]{hyperref}
\usepackage{tkz-euclide} 
\usepackage{listings}
% \usepackage{gvv}                                        
\def\inputGnumericTable{}                                 
\usepackage[latin1]{inputenc}                                
\usepackage{color}                                            
\usepackage{array}                                            
\usepackage{longtable}                                       
\usepackage{calc}                                             
\usepackage{multirow}                                         
\usepackage{hhline}                                           
\usepackage{ifthen}                                           
\usepackage{lscape}
\usepackage{circuitikz}



\author{EE25BTECH11041-Naman Kumar }
\graphicspath{./figs/}

\begin{document}
\begin{center}
    \huge{5.4.35}\\
    \large{EE25BTECH11041 - Naman Kumar}
\end{center}
Question:\\
Let p be an odd prime number and $\vec{T_p}$ be the following set of $2\times2$ matrices
\begin{align}
\vec{T_p}=\cbrak{\Vec{A}=\begin{pmatrix}a&b\\c&a\end{pmatrix}:a,b,c \in \cbrak{0,1,2,\dots,p-1} }
\end{align}
a) The number of $\vec{A}$ in $\vec{T_p}$ such that $\vec{A}$ is either symmetric or skew-symmetric or
both, and det(A) divisible by p is\\
\solution \\
Case 1: Symmentric
\begin{align}
    \vec{A}=\vec{A}^T\\
    \begin{pmatrix}a&b\\c&a\end{pmatrix}=\begin{pmatrix}a&c\\b&a\end{pmatrix}\\
    b=c\\
    \vec{A}=\begin{pmatrix}a&b\\b&a\end{pmatrix}
\end{align}
Determinant divisible by p
\begin{align}
    \vert\vec{A}\vert=\begin{vmatrix}a&b\\b&a\end{vmatrix}\\
    =a^2-b^2\\
    =(a-b)(a+b)\\
    \vert \vec{A}\vert\mod p =0\\
    (a-b)(a+b) \mod p=0
\end{align}
Since p is a prime number, \\
\begin{enumerate}[label=\roman*)]
    \item $a-b\mod p =0$ only at $a-b=0$:
        \begin{align}
            a-b=0\\
            a=b
        \end{align}
        So their are p pairs for (a,b) at a-b=0\\
    \item   \begin{align}
                a+b\mod p =0\\
            \end{align}
            \begin{center}    
                at a=1, b=p-1, a+b=p\\
                at a=2, b=p-2, a+b=p\\
                \[ \vdots \] \\
                so,similarly for all a, their is b in pair(a,b)\\
                therefore their are p pairs
            \end{center}
\end{enumerate}
Total number of such $\vec{A}$ in case 1 = ${i}+{ii}-{i\cap ii}$\\
Total = p + p -1 (case of elements = 0)\\
=2p-1\\\\
Case 2:Skew-Symmetric
\begin{align}
    \vec{A}+\vec{A}^T=0\\
    \begin{pmatrix}a&b\\c&a\end{pmatrix}+\begin{pmatrix}a&c\\b&a\end{pmatrix}=0\\
    \begin{pmatrix}2a&b+c\\c+b&2a\end{pmatrix}=0\\
    a=0,b=-c\\
    \vec{A}=\begin{pmatrix}0&b\\-b&0\end{pmatrix}
\end{align}
Now, for $\mod$
\begin{align}
    \vert \vec{A} \vert \mod p =0\\
    (0+b^2) \mod p=0\\
    b^2 \mod p=0
\end{align}
Only when b=0, so
\begin{align}
    \vec{A}=\begin{pmatrix}0&0\\0&0\end{pmatrix}
\end{align}
Already included in case 1
therefore, final answer = 2p-1
\begin{figure}[H]
    \centering
    \includegraphics[width=\columnwidth]{figs/figure.png}
    \caption{}
    \label{fig:placeholder}
\end{figure}
\end{document}
