\let\negmedspace\undefined
\let\negthickspace\undefined
\documentclass[journal]{IEEEtran}
\usepackage[a5paper, margin=10mm, onecolumn]{geometry}
%\usepackage{lmodern} % Ensure lmodern is loaded for pdflatex
\usepackage{tfrupee} % Include tfrupee package

\setlength{\headheight}{1cm} % Set the height of the header box
\setlength{\headsep}{0mm}     % Set the distance between the header box and the top of the text

\usepackage{gvv-book}
\usepackage{gvv}
\usepackage{cite}
\usepackage{amsmath,amssymb,amsfonts,amsthm}
\usepackage{algorithmic}
\usepackage{graphicx}
\usepackage{textcomp}
\usepackage{xcolor}
\usepackage{txfonts}
\usepackage{listings}
\usepackage{enumitem}
\usepackage{mathtools}
\usepackage{gensymb}
\usepackage{comment}
\usepackage[breaklinks=true]{hyperref}
\usepackage{tkz-euclide} 
\usepackage{listings}
% \usepackage{gvv}                                        
\def\inputGnumericTable{}                                 
\usepackage[latin1]{inputenc}                                
\usepackage{color}                                            
\usepackage{array}                                            
\usepackage{longtable}                                       
\usepackage{calc}                                             
\usepackage{multirow}                                         
\usepackage{hhline}                                           
\usepackage{ifthen}                                           
\usepackage{lscape}
\usepackage{circuitikz}



\author{EE25BTECH11041-Naman Kumar }
\graphicspath{./figs/}

\begin{document}
\begin{center}
    \huge{5.4.35}\\
    \large{EE25BTECH11041 - Naman Kumar}
\end{center}
Question:\\
Let p be an odd prime number and $\vec{T_p}$ be the following set of $2\times2$ matrices
\begin{align}
\vec{T_p}=\cbrak{\Vec{A}=\begin{pmatrix}a&b\\c&a\end{pmatrix}:a,b,c \in \cbrak{0,1,2,\dots,p-1} }
\end{align}
b) The number of $\vec{A}$ in $\vec{T}_p$ such that the trace of $\vec{A}$ is not divisible by p but det($\vec{A}$) is divisible by p is\\
\solution \\
Step 1: Trace of $\vec{A}$
\begin{align}
    \vec{A}=\begin{pmatrix}a&b\\c&a \end{pmatrix}\\
    tr(\vec{A})=a+a=2a\\
    2a \mod p \not\equiv  0
\end{align}
p is a odd prime , the number 2 is not a multiple of p, so 'a' must also be non-zero,\\
Therefore the condition simplifies to:
\begin{align}
    a\mod p \not\equiv 0
\end{align}
so for a their are p-1 chooses.\\
Step 2: $det(\vec{A})\mod p\equiv0$
\begin{align}
    det(\vec{A})=\begin{vmatrix}a&b\\c&a \end{vmatrix}\\
    =a^2-bc\\
    a^2-bc\mod p \equiv 0 \implies bc\equiv a^2 \brak{\mod p}
\end{align}
'a' as p-1 choices leaving a=0,let $a^2=k$
\begin{align}
    bc=k(k\neq0)
\end{align}
neither of  'b' and 'c' be zero\\
for 'b' we have p-1 choices leaving zero
\begin{align}
    bc\equiv k\\
    c \equiv k.b^{-1}\text{ ($b^{-1}$multiplicative inverse of b modulo p)}
\end{align}
so for every 'b' we have 'c'\\
Therefore their are p-1 pairs of (b,c)\\
Finally, total number matrix $\vec{A}$
\begin{align}
    =(p-1)(p-1)
    =(p-1)^2
\end{align}
\begin{figure}[H]
    \centering
    \includegraphics[width=\columnwidth]{figs/figure_b.png}
    \caption{}
    \label{fig:placeholder}
\end{figure}
\end{document}
