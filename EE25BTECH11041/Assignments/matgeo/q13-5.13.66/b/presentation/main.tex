\documentclass{beamer}
\usepackage[utf8]{inputenc}

\usetheme{Madrid}
\usecolortheme{default}
\usepackage{amsmath,amssymb,amsfonts,amsthm}
\usepackage{txfonts}
\usepackage{tkz-euclide}
\usepackage{listings}
\usepackage{adjustbox}
\usepackage{array}
\usepackage{tabularx}
\usepackage{gvv}
\usepackage{lmodern}
\usepackage{circuitikz}
\usepackage{tikz}
\usepackage{graphicx}

\setbeamertemplate{page number in head/foot}[totalframenumber]

\usepackage{tcolorbox}
\tcbuselibrary{minted,breakable,xparse,skins}



\definecolor{bg}{gray}{0.95}
\DeclareTCBListing{mintedbox}{O{}m!O{}}{%
	breakable=true,
	listing engine=minted,
	listing only,
	minted language=#2,
	minted style=default,
	minted options={%
		linenos,
		gobble=0,
		breaklines=true,
		breakafter=,,
		fontsize=\small,
		numbersep=8pt,
		#1},
	boxsep=0pt,
	left skip=0pt,
	right skip=0pt,
	left=25pt,
	right=0pt,
	top=3pt,
	bottom=3pt,
	arc=5pt,
	leftrule=0pt,
	rightrule=0pt,
	bottomrule=2pt,
	toprule=2pt,
	colback=bg,
	colframe=orange!70,
	enhanced,
	overlay={%
		\begin{tcbclipinterior}
			\fill[orange!20!white] (frame.south west) rectangle ([xshift=20pt]frame.north west);
	\end{tcbclipinterior}},
	#3,
}
\lstset{
	language=C,
	basicstyle=\ttfamily\small,
	keywordstyle=\color{blue},
	stringstyle=\color{orange},
	commentstyle=\color{green!60!black},
	numbers=left,
	numberstyle=\tiny\color{gray},
	breaklines=true,
	showstringspaces=false,
}
\begin{document}

\title 
{5.13.66}
\date{26 September,2025}

\author 
{Naman Kumar-EE25BTECH11041}
\graphicspath{./figs}


\frame{\titlepage}
\begin{frame}{Question b)}
Let p be an odd prime number and $\vec{T_p}$ be the following set of $2\times2$ matrices
\begin{align}
\vec{T_p}=\cbrak{\Vec{A}=\begin{pmatrix}a&b\\c&a\end{pmatrix}:a,b,c \in \cbrak{0,1,2,\dots,p-1} }
\end{align}
b) The number of $\vec{A}$ in $\vec{T}_p$ such that the trace of $\vec{A}$ is not divisible by p but det($\vec{A}$) is divisible by p is
\end{frame}
\begin{frame}{Solution}
Step 1: Trace of $\vec{A}$
\begin{align}
    \vec{A}=\begin{pmatrix}a&b\\c&a \end{pmatrix}\\
    tr(\vec{A})=a+a=2a\\
    2a \mod p \not\equiv  0
\end{align}
p is a odd prime , the number 2 is not a multiple of p, so 'a' must also be non-zero,\\
Therefore the condition simplifies to:
\begin{align}
    a\mod p \not\equiv 0
\end{align}
\end{frame}
\begin{frame}{Solution}
so for a their are p-1 chooses.\\
Step 2: $det(\vec{A})\mod p\equiv0$
\begin{align}
    det(\vec{A})=\begin{vmatrix}a&b\\c&a \end{vmatrix}\\
    =a^2-bc\\
    a^2-bc\mod p \equiv 0 \implies bc\equiv a^2 \brak{\mod p}
\end{align}
'a' as p-1 choices leaving a=0,let $a^2=k$
\begin{align}
    bc=k(k\neq0)
\end{align}
\end{frame}
\begin{frame}{Solution}
neither of  'b' and 'c' be zero\\
for 'b' we have p-1 choices leaving zero
\begin{align}
    bc\equiv k\\
    c \equiv k.b^{-1}\text{ ($b^{-1}$multiplicative inverse of b modulo p)}
\end{align}
so for every 'b' we have 'c'\\
Therefore their are p-1 pairs of (b,c)\\
Finally, total number matrix $\vec{A}$
\begin{align}
    =(p-1)(p-1)
    =(p-1)^2
\end{align}
\end{frame}
\begin{frame}{Figure}
    \begin{figure}[H]
    \centering
    \includegraphics[width=\columnwidth]{figs/figure_b.png}
    \caption{}
    \label{fig:placeholder}
\end{figure}
\end{frame}
\end{document}