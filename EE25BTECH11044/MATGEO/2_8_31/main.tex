\let\negmedspace\undefined
\let\negthickspace\undefined
\documentclass[journal]{IEEEtran}
\usepackage[a5paper, margin=10mm, onecolumn]{geometry}
%\usepackage{lmodern} % Ensure lmodern is loaded for pdflatex
\usepackage{tfrupee} % Include tfrupee package

\setlength{\headheight}{1cm} % Set the height of the header box
\setlength{\headsep}{0mm}     % Set the distance between the header box and the top of the text

\usepackage{gvv-book}
\usepackage{gvv}
\usepackage{cite}
\usepackage{amsmath,amssymb,amsfonts,amsthm}
\usepackage{algorithmic}
\usepackage{graphicx}
\usepackage{textcomp}
\usepackage{xcolor}
\usepackage{txfonts}
\usepackage{listings}
\usepackage{enumitem}
\usepackage{mathtools}
\usepackage{gensymb}
\usepackage{comment}
\usepackage[breaklinks=true]{hyperref}
\usepackage{tkz-euclide} 
\usepackage{listings}
% \usepackage{gvv}                                        
\def\inputGnumericTable{}                                 
\usepackage[latin1]{inputenc}                                
\usepackage{color}                                            
\usepackage{array}                                            
\usepackage{longtable}                                       
\usepackage{calc}                                             
\usepackage{multirow}                                         
\usepackage{hhline}                                           
\usepackage{ifthen}                                           
\usepackage{lscape}
\usepackage{circuitikz}
\tikzstyle{block} = [rectangle, draw, fill=blue!20, 
    text width=4em, text centered, rounded corners, minimum height=3em]
\tikzstyle{sum} = [draw, fill=blue!10, circle, minimum size=1cm, node distance=1.5cm]
\tikzstyle{input} = [coordinate]
\tikzstyle{output} = [coordinate]


\begin{document}

\bibliographystyle{IEEEtran}
\vspace{3cm}

\title{2.8.31}
\author{EE25BTECH11044 - Sai Hasini Pappula}
 \maketitle
% \newpage
% \bigskip
{\let\newpage\relax\maketitle}

\renewcommand{\thefigure}{\theenumi}
\renewcommand{\thetable}{\theenumi}
\setlength{\intextsep}{10pt} % Space between text and floats


\numberwithin{equation}{enumi}
\numberwithin{figure}{enumi}
\renewcommand{\thetable}{\theenumi}

\textbf{Question}
Given $A(1,-2)$, $B(2,3)$, $C(a,2)$ and $D(-4,-3)$ which form a parallelogram. Using only matrices and norms (no coordinate geometry formulas), find $a$ and the height when $AB$ is taken as base. 

\bigskip

\textbf{Solution}

Represent the points as column vectors:
\begin{equation}
A=\myvec{1\\-2},\quad B=\myvec{2\\3},\quad C=\myvec{a\\2},\quad D=\myvec{-4\\-3}.
\end{equation}

Parallelogram condition (diagonals bisect): 
\begin{equation}
A+C=B+D.
\end{equation}
Hence
\begin{equation}
\myvec{1\\-2}+\myvec{a\\2}=\myvec{2\\3}+\myvec{-4\\-3}.
\end{equation}
Compute the right-hand side and equate:
\begin{equation}
\myvec{1+a\\0}=\myvec{-2\\0}\quad\Longrightarrow\quad a=-3.
\end{equation}
Thus
\begin{equation}
C=\myvec{-3\\2}.
\end{equation}

\bigskip

Now let the base vector and the AC vector be
\begin{equation}
\mathbf{u}=AB=B-A=\myvec{1\\5},\qquad
\mathbf{v}=AC=C-A=\myvec{-4\\4}.
\end{equation}

Form the orthogonal projector onto $\mathbf{u}$ (matrix form):
\begin{equation}
P_{\mathbf{u}}=\frac{\mathbf{u}\mathbf{u}^\top}{\mathbf{u}^\top\mathbf{u}}.
\end{equation}

The component of $\mathbf{v}$ orthogonal to $\mathbf{u}$ is
\begin{equation}
\mathbf{w}=(I-P_{\mathbf{u}})\,\mathbf{v}.
\end{equation}

The height $h$ (distance from $C$ to line through $AB$) is the norm of $\mathbf{w}$:
\begin{equation}
\label{eq:height-norm}
h=\|\mathbf{w}\|=\big\|(I-P_{\mathbf{u}})\mathbf{v}\big\|.
\end{equation}

Compute the scalar products needed:
\begin{align}
\mathbf{u}^\top\mathbf{u}&=1^2+5^2=26, \\
\mathbf{u}^\top\mathbf{v}&=(-4)\cdot1 + 4\cdot5 = -4+20=16.
\end{align}

Thus the projector acting on $\mathbf{v}$ is
\begin{equation}
P_{\mathbf{u}}\mathbf{v}=\frac{\mathbf{u}(\mathbf{u}^\top\mathbf{v})}{\mathbf{u}^\top\mathbf{u}}
=\frac{16}{26}\,\mathbf{u}=\frac{8}{13}\,\myvec{1\\5}
=\myvec{\tfrac{8}{13}\\[4pt]\tfrac{40}{13}}.
\end{equation}

So
\begin{equation}
\mathbf{w}=\mathbf{v}-P_{\mathbf{u}}\mathbf{v}
=\myvec{-4\\4}-\myvec{\tfrac{8}{13}\\[4pt]\tfrac{40}{13}}
=\myvec{-\tfrac{60}{13}\\[4pt]\tfrac{12}{13}}.
\end{equation}

Therefore the height is
\begin{align}
h=\|\mathbf{w}\|
&= \sqrt{\Big(-\tfrac{60}{13}\Big)^{\!2}+\Big(\tfrac{12}{13}\Big)^{\!2}}
= \frac{\sqrt{60^2+12^2}}{13}
= \frac{\sqrt{3744}}{13} \nonumber\\[4pt]
&= \frac{12\sqrt{26}}{13} \;=\; \frac{24}{\sqrt{26}}.
\end{align}

\bigskip

\noindent\textbf{Final results:}
\begin{equation}
\boxed{a=-3},\qquad \boxed{h=\dfrac{24}{\sqrt{26}}=\dfrac{12\sqrt{26}}{13}}.
\end{equation}

\bigskip

\begin{center}
    \includegraphics[width=0.8\columnwidth]{figs/fig4.png}
\end{center}

\end{document}

