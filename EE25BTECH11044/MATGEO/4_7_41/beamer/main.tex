\documentclass{beamer}
\usepackage[utf8]{inputenc}

\usetheme{Madrid}
\usecolortheme{default}
\usepackage{amsmath,amssymb,amsfonts,amsthm}
\usepackage{txfonts}
\usepackage{tkz-euclide}
\usepackage{listings}
\usepackage{adjustbox}
\usepackage{array}
\usepackage{tabularx}
\usepackage{gvv}
\usepackage{lmodern}
\usepackage{circuitikz}
\usepackage{tikz}
\usepackage{graphicx}

\setbeamertemplate{page number in head/foot}[totalframenumber]

\usepackage{tcolorbox}
\tcbuselibrary{minted,breakable,xparse,skins}



\definecolor{bg}{gray}{0.95}
\DeclareTCBListing{mintedbox}{O{}m!O{}}{%
  breakable=true,
  listing engine=minted,
  listing only,
  minted language=#2,
  minted style=default,
  minted options={%
    linenos,
    gobble=0,
    breaklines=true,
    breakafter=,,
    fontsize=\small,
    numbersep=8pt,
    #1},
  boxsep=0pt,
  left skip=0pt,
  right skip=0pt,
  left=25pt,
  right=0pt,
  top=3pt,
  bottom=3pt,
  arc=5pt,
  leftrule=0pt,
  rightrule=0pt,
  bottomrule=2pt,
  toprule=2pt,
  colback=bg,
  colframe=orange!70,
  enhanced,
  overlay={%
    \begin{tcbclipinterior}
    \fill[orange!20!white] (frame.south west) rectangle ([xshift=20pt]frame.north west);
    \end{tcbclipinterior}},
  #3,
}
\lstset{
    language=C,
    basicstyle=\ttfamily\small,
    keywordstyle=\color{blue},
    stringstyle=\color{orange},
    commentstyle=\color{green!60!black},
    numbers=left,
    numberstyle=\tiny\color{gray},
    breaklines=true,
    showstringspaces=false,
}
%------------------------------------------------------------
%This block of code defines the information to appear in the
%Title page
\title %optional
{4.7.41}
\date{September 20, 2025}
%\subtitle{A short story}

\author % (optional)
{Sai Hasini  - EE25BTECH11044}




\begin{document}

% --- Frame: Problem Statement ---
\begin{frame}{Problem Statement}
\begin{block}{Question}
Find the distance of the point
\[
\vec{P} = \begin{bmatrix} 2 \\ 4 \\ -1 \end{bmatrix}
\]
from the line
\[
\frac{x+5}{1} = \frac{y+3}{4} = \frac{z-6}{-9}.
\]
\end{block}
\end{frame}


% --- Frame: Vector form of line ---
\begin{frame}{Line in Vector Form}
\begin{equation}\label{eq:vectorform}
\vec{r}=\vec{A}+\lambda\myvec{1\\4\\-9},\qquad
\vec{A}=\myvec{-5\\-3\\6},\ \lambda\in\mathbb{R}.
\end{equation}
We seek the foot \(\vec{Q}=(x,y,z)^T\) on the line such that
\((\vec{P}-\vec{Q})\cdot\myvec{1\\4\\-9}=0\).
\end{frame}

% --- Frame: Augmented matrix ---
\begin{frame}{Matrix Form}
From the perpendicularity condition we obtain the linear system
\begin{equation}\label{eq:matrix}
\left[\begin{array}{cccc}
1 & 0 & 0 & -1\\[4pt]
0 & 1 & 0 & -4\\[4pt]
0 & 0 & 1 & 9 \\[4pt]
-1&-4 & 9 & 0
\end{array}\right]
\begin{bmatrix}x\\y\\z\\\lambda\end{bmatrix}
=
\begin{bmatrix}-5\\-3\\6\\-27\end{bmatrix}.
\end{equation}
Write the augmented matrix for elimination:
\begin{equation}\label{eq:aug}
\left[\begin{array}{cccc|c}
1 & 0 & 0 & -1 & -5\\[4pt]
0 & 1 & 0 & -4 & -3\\[4pt]
0 & 0 & 1 & 9  & 6 \\[4pt]
-1&-4 & 9 & 0  & -27
\end{array}\right]
\end{equation}
\end{frame}

% --- Frame: Row operation 1-4 (numbered) ---
\begin{frame}{Row-Reduction — Steps}
\begin{equation}\label{eq:step1}
R_4 \leftarrow R_4 + R_1
\qquad\Longrightarrow\qquad
\left[\begin{array}{cccc|c}
1 & 0 & 0 & -1 & -5\\[4pt]
0 & 1 & 0 & -4 & -3\\[4pt]
0 & 0 & 1 & 9  & 6 \\[4pt]
0 & -4& 9 & -1 & -32
\end{array}\right]
\end{equation}

\begin{equation}\label{eq:step2}
R_4 \leftarrow R_4 + 4R_2
\qquad\Longrightarrow\qquad
\left[\begin{array}{cccc|c}
1 & 0 & 0 & -1 & -5\\[4pt]
0 & 1 & 0 & -4 & -3\\[4pt]
0 & 0 & 1 & 9  & 6 \\[4pt]
0 & 0 & 9 & -17 & -44
\end{array}\right]
\end{equation}

\begin{equation}\label{eq:step3}
R_4 \leftarrow R_4 - 9R_3
\qquad\Longrightarrow\qquad
\left[\begin{array}{cccc|c}
1 & 0 & 0 & -1 & -5\\[4pt]
0 & 1 & 0 & -4 & -3\\[4pt]
0 & 0 & 1 & 9  & 6 \\[4pt]
0 & 0 & 0 & -98 & -98
\end{array}\right]
\end{equation}
\end{frame}

% --- Frame: Final elimination to RREF ---
\begin{frame}{Final Steps to RREF}
\begin{equation}\label{eq:step4}
R_4 \leftarrow \frac{1}{-98}R_4
\qquad\Longrightarrow\qquad
\left[\begin{array}{cccc|c}
1 & 0 & 0 & -1 & -5\\[4pt]
0 & 1 & 0 & -4 & -3\\[4pt]
0 & 0 & 1 & 9  & 6 \\[4pt]
0 & 0 & 0 & 1 & 1
\end{array}\right]
\end{equation}

Now eliminate the $\lambda$-entries above:
\[
R_1\leftarrow R_1+R_4,\quad
R_2\leftarrow R_2+4R_4,\quad
R_3\leftarrow R_3-9R_4
\]
which yields the RREF:
\begin{equation}\label{eq:rref}
\left[\begin{array}{cccc|c}
1 & 0 & 0 & 0 & -4\\[4pt]
0 & 1 & 0 & 0 & 1\\[4pt]
0 & 0 & 1 & 0 & -3\\[4pt]
0 & 0 & 0 & 1 & 1
\end{array}\right].
\end{equation}
\end{frame}

% --- Frame: Solution and distance ---
\begin{frame}{Solution and Distance}
Read off the solution from \eqref{eq:rref}:
\begin{equation}\label{eq:sol}
x=-4,\quad y=1,\quad z=-3,\quad \lambda=1,
\qquad \vec{Q}=\myvec{-4\\1\\-3}.
\end{equation}

Compute the distance:
\begin{equation}\label{eq:distance}
\begin{aligned}
d &= \|\vec{P}-\vec{Q}\|
= \sqrt{(2-(-4))^2+(4-1)^2+(-1-(-3))^2}\\[4pt]
  &= \sqrt{6^2+3^2+2^2}=\sqrt{49}=7.
\end{aligned}
\end{equation}

\[
\boxed{\,d=7\,}
\]
\end{frame}

% --- Frame 2: C Code (part 1) ---
\begin{frame}[fragile]{C Code (Part 1)}
\begin{lstlisting}[language=C]
#include <stdio.h>
#include <math.h>

#define ROWS 4
#define COLS 5   // 4 variables + 1 RHS

// Function to perform Gaussian elimination to RREF
void gaussJordan(double mat[ROWS][COLS]) {
    int i, j, k;
    for (i = 0; i < ROWS; i++) {
        // Make the pivot element = 1
        double pivot = mat[i][i];
        if (pivot != 0) {
            for (j = 0; j < COLS; j++) {
                mat[i][j] /= pivot;
            }
        }
\end{lstlisting}
\end{frame}

% --- Frame 3: C Code (part 2) ---
\begin{frame}[fragile]{C Code (Part 2)}
\begin{lstlisting}[language=C]
        // Eliminate all other entries in column i
        for (k = 0; k < ROWS; k++) {
            if (k != i) {
                double factor = mat[k][i];
                for (j = 0; j < COLS; j++) {
                    mat[k][j] -= factor * mat[i][j];
                }
            }
        }
    }
}
int main() {
    // Augmented matrix for system in (x,y,z,lambda)
    double mat[ROWS][COLS] = {
        { 1,  0,  0, -1, -5},
        { 0,  1,  0, -4, -3},
        { 0,  0,  1,  9,  6},
        {-1, -4,  9,  0, -27}
    };
\end{lstlisting}
\end{frame}

% --- Frame 4: C Code (main) ---
\begin{frame}[fragile]{C Code (Part 3)}
\begin{lstlisting}[language=C]


    gaussJordan(mat);

    double x = mat[0][4], y = mat[1][4];
    double z = mat[2][4], lambda = mat[3][4];

    printf("Solution: x = %.2f, y = %.2f, z = %.2f, λ = %.2f\n",
           x, y, z, lambda);

    // Given point P(2,4,-1)
    double px = 2, py = 4, pz = -1;
    double dist = sqrt((px-x)*(px-x) + (py-y)*(py-y) + (pz-z)*(pz-z));

    printf("Distance = %.2f\n", dist);
    return 0;
}
\end{lstlisting}
\end{frame}


% --- Frame 4: Python Code (part 1) ---
\begin{frame}[fragile]{Python Code (Part 1)}
\begin{lstlisting}[language=Python]
import ctypes
import numpy as np
import matplotlib.pyplot as plt

# Load shared library
lib = ctypes.CDLL("gj.so")

# Define function signature
lib.distance.argtypes = [
    ctypes.c_double, ctypes.c_double, ctypes.c_double
]
lib.distance.restype = ctypes.c_double

# Point P
px, py, pz = 2.0, 4.0, -1.0
\end{lstlisting}
\end{frame}

% --- Frame 5: Python Code (part 2) ---
\begin{frame}[fragile]{Python Code (Part 2)}
\begin{lstlisting}[language=Python]
# Call C function
dist = lib.distance(px, py, pz)
print("Distance =", dist)

# Foot of perpendicular (from Gauss-Jordan result)
Q = np.array([-1/7, 19/7, 45/7])
P = np.array([px, py, pz])

# Line definition
A_line = np.array([-5, -3, 6])
d = np.array([1, 4, -9])

# Generate line points
t_vals = np.linspace(-2, 3, 100)
line_points = A_line.reshape(3,1) + d.reshape(3,1)*t_vals
\end{lstlisting}
\end{frame}

% --- Frame 6: Python Code (Plot) ---
\begin{frame}[fragile]{Python Code (Plot)}
\begin{lstlisting}[language=Python]
# Plot
fig = plt.figure(figsize=(8,6))
ax = fig.add_subplot(111, projection='3d')

# Line, points, perpendicular
ax.plot(line_points[0], line_points[1], line_points[2],
        'b-', label="Line")
ax.scatter(P[0], P[1], P[2], color='r', s=60, label="Point P")
ax.scatter(Q[0], Q[1], Q[2], color='g', s=60, label="Foot Q")
ax.plot([P[0], Q[0]], [P[1], Q[1]], [P[2], Q[2]],
        'k--', label="Perpendicular")

ax.set_xlabel("X-axis")
ax.set_ylabel("Y-axis")
ax.set_zlabel("Z-axis")
ax.legend()
plt.show()
\end{lstlisting}
\end{frame}


% --- Frame 6: Plot ---
\begin{frame}{Plot}
\centering
\includegraphics[width=0.8\columnwidth]{figs/plot7.png}
\end{frame}

\end{document}
