\let\negmedspace\undefined
\let\negthickspace\undefined
\documentclass[journal]{IEEEtran}
\usepackage[a5paper, margin=10mm, onecolumn]{geometry}
\usepackage{tfrupee} 

\setlength{\headheight}{1cm}
\setlength{\headsep}{0mm}     

\usepackage{gvv-book}
\usepackage{gvv}
\usepackage{cite}
\usepackage{amsmath,amssymb,amsfonts,amsthm}
\usepackage{algorithmic}
\usepackage{graphicx}
\usepackage{textcomp}
\usepackage{xcolor}
\usepackage{txfonts}
\usepackage{listings}
\usepackage{enumitem}
\usepackage{mathtools}
\usepackage{gensymb}
\usepackage{comment}
\usepackage[breaklinks=true]{hyperref}
\usepackage{tkz-euclide} 
\usepackage{listings}
\def\inputGnumericTable{}                                 
\usepackage[latin1]{inputenc}                                
\usepackage{color}                                            
\usepackage{array}                                            
\usepackage{longtable}                                       
\usepackage{calc}                                             
\usepackage{multirow}                                         
\usepackage{hhline}                                           
\usepackage{ifthen}                                           
\usepackage{lscape}
\usepackage{circuitikz}


\tikzstyle{block} = [rectangle, draw, fill=blue!20, 
    text width=4em, text centered, rounded corners, minimum height=3em]
\tikzstyle{sum} = [draw, fill=blue!10, circle, minimum size=1cm, node distance=1.5cm]
\tikzstyle{input} = [coordinate]
\tikzstyle{output} = [coordinate]

\begin{document}

\bibliographystyle{IEEEtran}
\vspace{3cm}

\title{4.7.41}
\author{EE25BTECH11044 - Sai Hasini Pappula}
 \maketitle
{\let\newpage\relax\maketitle}

\renewcommand{\thefigure}{\theenumi}
\renewcommand{\thetable}{\theenumi}
\setlength{\intextsep}{10pt} 

\numberwithin{equation}{enumi}
\numberwithin{figure}{enumi}
\renewcommand{\thetable}{\theenumi}
\section*{Question}
Find the distance of the point \(\vec{P}=\myvec{2\\4\\-1}\) from the line
\begin{equation}
\frac{x+5}{1}=\frac{y+3}{4}=\frac{z-6}{-9}.
\end{equation}

\section*{Solution}
Write the line in parametric form:
\begin{equation}
\vec{Q}=\myvec{-5\\-3\\6}+\lambda\myvec{1\\4\\-9},\qquad \lambda\in\mathbb{R},
\end{equation}
so that a general point on the line is
\begin{equation}
\vec{Q}=\myvec{-5+\lambda\\-3+4\lambda\\6-9\lambda}.
\end{equation}
The direction vector of the line is
\begin{equation}
\myvec{1\\4\\-9}.
\end{equation}
The perpendicularity condition for the foot of the perpendicular from \(\vec{P}\) to the line is
\begin{equation}
\myvec{1\\4\\-9}\cdot\big(\vec{P}-\vec{Q}\big)=0.
\end{equation}

Expanding and collecting the linear equations in the unknowns \(x,y,z,\lambda\) (where \(x,y,z\) are the coordinates of \(\vec{Q}\)) gives the system
\begin{align}
x-\lambda &= -5, \label{eq:1}\\
y-4\lambda &= -3, \label{eq:2}\\
z+9\lambda &= 6, \label{eq:3}\\
-\,x-4y+9z &= -27. \label{eq:4}
\end{align}

\section*{Augmented matrix and full row-reduction steps}
Write the augmented matrix for the linear system in the unknown order \((x,y,z,\lambda)\):
\begin{equation}\label{eq:aug0}
\left[\begin{array}{cccc|c}
1 & 0 & 0 & -1 & -5\\[4pt]
0 & 1 & 0 & -4 & -3\\[4pt]
0 & 0 & 1 & 9  & 6 \\[4pt]
-1& -4& 9 & 0  & -27
\end{array}\right].
\end{equation}

We perform elementary row operations step by step.

\medskip
1. Eliminate the \(1^{\text{st}}\) column entry of row 4 by adding row 1 to row 4:
\begin{equation}\label{eq:step1}
R_4 \leftarrow R_4+R_1
\qquad\Longrightarrow\qquad
\left[\begin{array}{cccc|c}
1 & 0 & 0 & -1 & -5\\[4pt]
0 & 1 & 0 & -4 & -3\\[4pt]
0 & 0 & 1 & 9  & 6 \\[4pt]
0 & -4& 9 & -1 & -32
\end{array}\right]
\end{equation}

2. Eliminate the \(2^{\text{nd}}\) column entry of row 4 using row 2:
\begin{equation}\label{eq:step2}
R_4 \leftarrow R_4 + 4R_2
\qquad\Longrightarrow\qquad
\left[\begin{array}{cccc|c}
1 & 0 & 0 & -1 & -5\\[4pt]
0 & 1 & 0 & -4 & -3\\[4pt]
0 & 0 & 1 & 9  & 6 \\[4pt]
0 & 0 & 9 & -17 & -44
\end{array}\right]
\end{equation}


3. Eliminate the \(3^{\text{rd}}\) column entry of row 4 using row 3:
\begin{equation}\label{eq:step3}
R_4 \leftarrow R_4 - 9R_3
\qquad\Longrightarrow\qquad
\left[\begin{array}{cccc|c}
1 & 0 & 0 & -1 & -5\\[4pt]
0 & 1 & 0 & -4 & -3\\[4pt]
0 & 0 & 1 & 9  & 6 \\[4pt]
0 & 0 & 0 & -98 & -98
\end{array}\right]
\end{equation}

4. Scale row 4 to make a leading 1 (divide by \(-98\)):
\begin{equation}\label{eq:step4}
R_4 \leftarrow \frac{1}{-98}R_4
\qquad\Longrightarrow\qquad
\left[\begin{array}{cccc|c}
1 & 0 & 0 & -1 & -5\\[4pt]
0 & 1 & 0 & -4 & -3\\[4pt]
0 & 0 & 1 & 9  & 6 \\[4pt]
0 & 0 & 0 & 1 & 1
\end{array}\right]
\end{equation}


5. Use the pivot in row 4 to eliminate the \(\lambda\)-entries above it:

\begin{equation}\label{eq:step5}
\begin{aligned}
& R_1 \leftarrow R_1 + R_4,\\
& R_2 \leftarrow R_2 + 4R_4,\\
& R_3 \leftarrow R_3 - 9R_4,
\end{aligned}
\qquad\Longrightarrow\qquad
\left[\begin{array}{cccc|c}
1 & 0 & 0 & 0 & -4\\[4pt]
0 & 1 & 0 & 0 & 1\\[4pt]
0 & 0 & 1 & 0 & -3\\[4pt]
0 & 0 & 0 & 1 & 1
\end{array}\right].
\end{equation}
This matrix is now in RREF.

\section*{Solution from RREF}
 from (0.15):
\begin{align}
x &= -4, \\
y &= 1, \\
z &= -3, \\
\lambda &= 1.
\end{align}
Thus the foot of the perpendicular (point on the line closest to \(\vec{P}\)) is
\begin{equation}
\vec{Q}=\myvec{-4\\1\\-3}.
\end{equation}

\section*{Distance}
Compute \(\vec{P}-\vec{Q}\) and its norm:
\begin{equation}
\vec{P}-\vec{Q}=\myvec{2\\4\\-1}-\myvec{-4\\1\\-3}=\myvec{6\\3\\2},
\end{equation}
\begin{equation}
\|\vec{P}-\vec{Q}\|=\sqrt{6^2+3^2+2^2}=\sqrt{36+9+4}=\sqrt{49}=7.
\end{equation}

\bigskip
\noindent\textbf{Final Answer:} The distance from \(\vec{P}=\myvec{2\\4\\-1}\) to the given line is \(\boxed{7}\).


\begin{center}
    \includegraphics[width=0.8\columnwidth]{figs/plot7.png}
\end{center}

\end{document}