\let\negmedspace\undefined
\let\negthickspace\undefined
\documentclass[journal]{IEEEtran}
\usepackage[a5paper, margin=10mm, onecolumn]{geometry}
%\usepackage{lmodern} % Ensure lmodern is loaded for pdflatex
\usepackage{tfrupee} % Include tfrupee package

\setlength{\headheight}{1cm} % Set the height of the header box
\setlength{\headsep}{0mm}     % Set the distance between the header box and the top of the text

\usepackage{gvv-book}
\usepackage{gvv}
\usepackage{cite}
\usepackage{amsmath,amssymb,amsfonts,amsthm}
\usepackage{algorithmic}
\usepackage{graphicx}
\usepackage{textcomp}
\usepackage{xcolor}
\usepackage{txfonts}
\usepackage{listings}
\usepackage{enumitem}
\usepackage{mathtools}
\usepackage{gensymb}
\usepackage{comment}
\usepackage[breaklinks=true]{hyperref}
\usepackage{tkz-euclide} 
\usepackage{listings}
% \usepackage{gvv}                                        
\def\inputGnumericTable{}                                 
\usepackage[latin1]{inputenc}                                
\usepackage{color}                                            
\usepackage{array}                                            
\usepackage{longtable}                                       
\usepackage{calc}                                             
\usepackage{multirow}                                         
\usepackage{hhline}                                           
\usepackage{ifthen}                                           
\usepackage{lscape}
\usepackage{circuitikz}
\tikzstyle{block} = [rectangle, draw, fill=blue!20, 
    text width=4em, text centered, rounded corners, minimum height=3em]
\tikzstyle{sum} = [draw, fill=blue!10, circle, minimum size=1cm, node distance=1.5cm]
\tikzstyle{input} = [coordinate]
\tikzstyle{output} = [coordinate]


\begin{document}

\bibliographystyle{IEEEtran}
\vspace{3cm}

\title{4.3.34}
\author{EE25BTECH11044 - Sai Hasini Pappula}
 \maketitle
% \newpage
% \bigskip
{\let\newpage\relax\maketitle}

\renewcommand{\thefigure}{\theenumi}
\renewcommand{\thetable}{\theenumi}
\setlength{\intextsep}{10pt} % Space between text and floats


\numberwithin{equation}{enumi}
\numberwithin{figure}{enumi}
\renewcommand{\thetable}{\theenumi}
\textbf{Question:}  
If the line 
\[
\frac{x}{a} + \frac{y}{b} = 1
\]
passes through the points $(2,-3)$ and $(4,-5)$, then find $(a,b)$ 

\textbf{Solution:}  

For $(2,-3)$:
\[
\frac{2}{a} + \frac{-3}{b} = 1 \quad \Rightarrow \quad \frac{2}{a} - \frac{3}{b} = 1
\]

For $(4,-5)$:
\[
\frac{4}{a} + \frac{-5}{b} = 1 \quad \Rightarrow \quad \frac{4}{a} - \frac{5}{b} = 1
\]

Let 
\[
u = \frac{1}{a}, \quad v = \frac{1}{b}.
\]

Then the system becomes:
\[
2u - 3v = 1, \qquad 4u - 5v = 1
\]

or in matrix form:
\[
\begin{bmatrix}
2 & -3 \\
4 & -5
\end{bmatrix}
\begin{bmatrix}
u \\ v
\end{bmatrix}
=
\begin{bmatrix}
1 \\ 1
\end{bmatrix}.
\]


\[
A = \begin{bmatrix} 2 & -3 \\ 4 & -5 \end{bmatrix}.
\]

We compute
\[
A A^T =
\begin{bmatrix}
13 & 23 \\
23 & 41
\end{bmatrix}.
\]

Hence, the norm of $A$ is
\[
\|A\|_F = \sqrt{\mathrm{trace}(A A^T)} = \sqrt{13 + 41} = \sqrt{54} \neq 0,
\]
so the matrix is invertible.


\[
A^{-1} = \frac{1}{2}
\begin{bmatrix}
-5 & 3 \\
-4 & 2
\end{bmatrix}.
\]

Thus,
\[
\begin{bmatrix}
u \\ v
\end{bmatrix}
=
A^{-1}
\begin{bmatrix}
1 \\ 1
\end{bmatrix}
=
\frac{1}{2}
\begin{bmatrix}
-5 & 3 \\
-4 & 2
\end{bmatrix}
\begin{bmatrix}
1 \\ 1
\end{bmatrix}
=
\frac{1}{2}
\begin{bmatrix}
-2 \\ -2
\end{bmatrix}
=
\begin{bmatrix}
-1 \\ -1
\end{bmatrix}.
\]

Hence,
\[
u = -1 \;\Rightarrow\; a = -1, 
\qquad 
v = -1 \;\Rightarrow\; b = -1.
\]

---

\textbf{Verification:}  
Equation of the line:
\[
\frac{x}{-1} + \frac{y}{-1} = 1 \quad \Rightarrow \quad -x - y = 1.
\]

For $(2,-3)$:
\[
-(2) - (-3) = -2 + 3 = 1 \quad \checkmark
\]

For $(4,-5)$:
\[
-(4) - (-5) = -4 + 5 = 1 \quad \checkmark
\]

---

\textbf{Final Answer:}  
\[
(a,b) = (-1,-1).
\]

\begin{center}
    \includegraphics[width=0.8\columnwidth]{figs/plot6.png}
\end{center}

\end{document}