\let\negmedspace\undefined
\let\negthickspace\undefined
\documentclass[journal]{IEEEtran}
\usepackage[a5paper, margin=10mm, onecolumn]{geometry}
%\usepackage{lmodern} % Ensure lmodern is loaded for pdflatex
\usepackage{tfrupee} % Include tfrupee package

\setlength{\headheight}{1cm} % Set the height of the header box
\setlength{\headsep}{0mm}     % Set the distance between the header box and the top of the text

\usepackage{gvv-book}
\usepackage{gvv}
\usepackage{cite}
\usepackage{amsmath,amssymb,amsfonts,amsthm}
\usepackage{algorithmic}
\usepackage{graphicx}
\usepackage{textcomp}
\usepackage{xcolor}
\usepackage{txfonts}
\usepackage{listings}
\usepackage{enumitem}
\usepackage{mathtools}
\usepackage{gensymb}
\usepackage{comment}
\usepackage[breaklinks=true]{hyperref}
\usepackage{tkz-euclide} 
\usepackage{listings}
% \usepackage{gvv}                                        
\def\inputGnumericTable{}                                 
\usepackage[latin1]{inputenc}                                
\usepackage{color}                                            
\usepackage{array}                                            
\usepackage{longtable}                                       
\usepackage{calc}                                             
\usepackage{multirow}                                         
\usepackage{hhline}                                           
\usepackage{ifthen}                                           
\usepackage{lscape}
\usepackage{circuitikz}
\tikzstyle{block} = [rectangle, draw, fill=blue!20, 
    text width=4em, text centered, rounded corners, minimum height=3em]
\tikzstyle{sum} = [draw, fill=blue!10, circle, minimum size=1cm, node distance=1.5cm]
\tikzstyle{input} = [coordinate]
\tikzstyle{output} = [coordinate]


\begin{document}

\bibliographystyle{IEEEtran}
\vspace{3cm}

\title{4.3.34}
\author{EE25BTECH11044 - Sai Hasini Pappula}
 \maketitle
% \newpage
% \bigskip
{\let\newpage\relax\maketitle}

\renewcommand{\thefigure}{\theenumi}
\renewcommand{\thetable}{\theenumi}
\setlength{\intextsep}{10pt} % Space between text and floats


\numberwithin{equation}{enumi}
\numberwithin{figure}{enumi}
\renewcommand{\thetable}{\theenumi}
\textbf{Question:}  
If the line 
\[
\frac{x}{a} + \frac{y}{b} = 1
\]
passes through the points $(2,-3)$ and $(4,-5)$, then find $(a,b)$ 

\textbf{Solution (using $\vec{n}^T \vec{x} = c$):}

The equation of a line can be expressed as
\[
\vec{n}^T \vec{x} = c,
\]
where $\vec{n}$ is the normal vector to the line.

---

\textbf{Step 1: Direction vector of the line}

The line passes through
\[
\vec{x}_1 = \begin{bmatrix}2\\-3\end{bmatrix}, \quad
\vec{x}_2 = \begin{bmatrix}4\\-5\end{bmatrix}.
\]

Hence, its direction vector is
\[
\vec{m} = \vec{x}_2 - \vec{x}_1
= \begin{bmatrix}4-2 \\ -5 - (-3)\end{bmatrix}
= \begin{bmatrix}2 \\ -2\end{bmatrix}.
\]

---

\textbf{Step 2: Find the normal vector $\vec{n}$}

The normal vector $\vec{n} = \begin{bmatrix}n_1 \\ n_2\end{bmatrix}$ must satisfy
\[
\vec{n}^T \vec{m} = 0.
\]

That is,
\[
\begin{bmatrix}n_1 & n_2\end{bmatrix}
\begin{bmatrix}2 \\ -2\end{bmatrix} = 0,
\]
\[
2n_1 - 2n_2 = 0 \quad \Rightarrow \quad n_1 = n_2.
\]

So, a valid choice is
\[
\vec{n} = \begin{bmatrix}1 \\ 1\end{bmatrix}.
\]

---

\textbf{Step 3: Find $c$}

Using the equation
\[
\vec{n}^T \vec{x} = c,
\]
substitute $\vec{x}_1 = \begin{bmatrix}2\\-3\end{bmatrix}$:
\[
\begin{bmatrix}1 & 1\end{bmatrix}\begin{bmatrix}2 \\ -3\end{bmatrix}
= 2 - 3 = -1.
\]

Thus,
\[
c = -1.
\]

---

\textbf{Final Answer:}
\[
\vec{n}^T \vec{x} = -1 \quad \Rightarrow \quad x+y=-1.
\]

\begin{center}
    \includegraphics[width=0.8\columnwidth]{figs/plot6.png}
\end{center}

\end{document}