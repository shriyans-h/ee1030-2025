\let\negmedspace\undefined
\let\negthickspace\undefined
\documentclass[journal]{IEEEtran}
\usepackage[a5paper, margin=10mm, onecolumn]{geometry}
\usepackage{tfrupee} 

\setlength{\headheight}{1cm}
\setlength{\headsep}{0mm}     

\usepackage{gvv-book}
\usepackage{gvv}
\usepackage{cite}
\usepackage{amsmath,amssymb,amsfonts,amsthm}
\usepackage{algorithmic}
\usepackage{graphicx}
\usepackage{textcomp}
\usepackage{xcolor}
\usepackage{txfonts}
\usepackage{listings}
\usepackage{enumitem}
\usepackage{mathtools}
\usepackage{gensymb}
\usepackage{comment}
\usepackage[breaklinks=true]{hyperref}
\usepackage{tkz-euclide} 
\usepackage{listings}
\def\inputGnumericTable{}                                 
\usepackage[latin1]{inputenc}                                
\usepackage{color}                                            
\usepackage{array}                                            
\usepackage{longtable}                                       
\usepackage{calc}                                             
\usepackage{multirow}                                         
\usepackage{hhline}                                           
\usepackage{ifthen}                                           
\usepackage{lscape}
\usepackage{circuitikz}


\tikzstyle{block} = [rectangle, draw, fill=blue!20, 
    text width=4em, text centered, rounded corners, minimum height=3em]
\tikzstyle{sum} = [draw, fill=blue!10, circle, minimum size=1cm, node distance=1.5cm]
\tikzstyle{input} = [coordinate]
\tikzstyle{output} = [coordinate]

\begin{document}

\bibliographystyle{IEEEtran}
\vspace{3cm}

\title{4.3.34}
\author{EE25BTECH11044 - Sai Hasini Pappula}
 \maketitle
{\let\newpage\relax\maketitle}

\renewcommand{\thefigure}{\theenumi}
\renewcommand{\thetable}{\theenumi}
\setlength{\intextsep}{10pt} 

\numberwithin{equation}{enumi}
\numberwithin{figure}{enumi}
\renewcommand{\thetable}{\theenumi}

\textbf{Question:}  
If the line 
\[
\frac{x}{a} + \frac{y}{b} = 1
\]
passes through the points $(2,-3)$ and $(4,-5)$, then find $(a,b)$. 

---

\textbf{Solution}

The equation of a line can be expressed as
\begin{equation}
\vec{n}^T \vec{x} = c,
\end{equation}
where $\vec{n}$ is the normal vector to the line.

---

\textbf{Step 1: Direction vector of the line}

The line passes through
\begin{equation}
\myvec{x}_1 = \myvec{2\\-3}, \quad
\myvec{x}_2 = \myvec{4\\-5}.
\end{equation}

Hence, its direction vector is
\begin{equation}
\myvec{m} = \myvec{x}_2 - \myvec{x}_1
= \myvec{2 \\ -2}.
\end{equation}

---

\textbf{Step 2: Find the normal vector $\myvec{n}$}

The normal vector $\myvec{n} = \myvec{n_1 \\ n_2}$ must satisfy
\begin{equation}
\vec{n}^T \vec{m} = 0.
\end{equation}

That is,
\begin{equation}
\myvec{n_1 & n_2}\myvec{2 \\ -2} = 0,
\end{equation}
\begin{equation}
2n_1 - 2n_2 = 0 \quad \Rightarrow \quad n_1 = n_2.
\end{equation}

So, a valid choice is
\begin{equation}
\myvec{n} = \myvec{1 \\ 1}.
\end{equation}

---

\textbf{Step 3: Find $c$}

Using the equation
\begin{equation}
\vec{n}^T \vec{x} = c,
\end{equation}
substitute $\myvec{x}_1 = \myvec{2\\-3}$:
\begin{equation}
\myvec{1 & 1}\myvec{2 \\ -3} = -1.
\end{equation}

Thus,
\begin{equation}
c = -1.
\end{equation}

---

\textbf{Final Answer (matrix form):}
\begin{equation}
\myvec{1 & 1}\myvec{x \\ y} = -1
\quad \Rightarrow \quad x+y=-1.
\end{equation}

\begin{center}
    \includegraphics[width=0.8\columnwidth]{figs/plot6.png}
\end{center}

\end{document}
