\documentclass{beamer}
\usepackage[utf8]{inputenc}

\usetheme{Madrid}
\usecolortheme{default}
\usepackage{amsmath,amssymb,amsfonts,amsthm}
\usepackage{txfonts}
\usepackage{tkz-euclide}
\usepackage{listings}
\usepackage{adjustbox}
\usepackage{array}
\usepackage{tabularx}
\usepackage{gvv}
\usepackage{lmodern}
\usepackage{circuitikz}
\usepackage{tikz}
\usepackage{graphicx}

\setbeamertemplate{page number in head/foot}[totalframenumber]

\usepackage{tcolorbox}
\tcbuselibrary{minted,breakable,xparse,skins}



\definecolor{bg}{gray}{0.95}
\DeclareTCBListing{mintedbox}{O{}m!O{}}{%
  breakable=true,
  listing engine=minted,
  listing only,
  minted language=#2,
  minted style=default,
  minted options={%
    linenos,
    gobble=0,
    breaklines=true,
    breakafter=,,
    fontsize=\small,
    numbersep=8pt,
    #1},
  boxsep=0pt,
  left skip=0pt,
  right skip=0pt,
  left=25pt,
  right=0pt,
  top=3pt,
  bottom=3pt,
  arc=5pt,
  leftrule=0pt,
  rightrule=0pt,
  bottomrule=2pt,
  toprule=2pt,
  colback=bg,
  colframe=orange!70,
  enhanced,
  overlay={%
    \begin{tcbclipinterior}
    \fill[orange!20!white] (frame.south west) rectangle ([xshift=20pt]frame.north west);
    \end{tcbclipinterior}},
  #3,
}
\lstset{
    language=C,
    basicstyle=\ttfamily\small,
    keywordstyle=\color{blue},
    stringstyle=\color{orange},
    commentstyle=\color{green!60!black},
    numbers=left,
    numberstyle=\tiny\color{gray},
    breaklines=true,
    showstringspaces=false,
}
%------------------------------------------------------------
%This block of code defines the information to appear in the
%Title page
\title %optional
{4.3.34}
\date{September 15, 2025}
%\subtitle{A short story}

\author % (optional)
{Sai Hasini Pappula - EE25BTECH11044}



\begin{document}
\begin{frame}{Question}
\begin{block}{Problem}
If the line
\[
\frac{x}{a} + \frac{y}{b} = 1
\]
passes through the points $(2,-3)$ and $(4,-5)$, then find $(a,b)$.
\end{block}
\end{frame}

% Solution (Frame 1)
\begin{frame}{Solution (Step 1)}
For $(2,-3)$:
\[
\frac{2}{a} - \frac{3}{b} = 1
\]
For $(4,-5)$:
\[
\frac{4}{a} - \frac{5}{b} = 1
\]

Let
\[
u = \frac{1}{a}, \quad v = \frac{1}{b}.
\]
\end{frame}

% Solution (Frame 2)
\begin{frame}{Solution (Step 2)}
System becomes:
\[
2u - 3v = 1, \qquad 4u - 5v = 1
\]

Matrix form:
\[
\begin{bmatrix}2 & -3 \\ 4 & -5 \end{bmatrix}
\begin{bmatrix}u \\ v\end{bmatrix}
=
\begin{bmatrix}1 \\ 1\end{bmatrix}.
\]

Compute:
\[
A^{-1} = \frac{1}{2}
\begin{bmatrix}-5 & 3 \\ -4 & 2\end{bmatrix}.
\]
\end{frame}

% Solution (Frame 3)
\begin{frame}{Solution (Step 3)}
\[
\begin{bmatrix}u \\ v\end{bmatrix}
=
A^{-1}\begin{bmatrix}1 \\ 1\end{bmatrix}
=
\frac{1}{2}\begin{bmatrix}-2 \\ -2\end{bmatrix}
=
\begin{bmatrix}-1 \\ -1\end{bmatrix}.
\]

Thus,
\[
a = -1, \quad b = -1.
\]

Equation of line:
\[
-x - y = 1.
\]
\end{frame}

% C Code Part 1
\begin{frame}[fragile]{C Code (Part 1)}
\begin{lstlisting}[language=C]
#include <stdio.h>

int main() {
    double A[2][2] = {{2, -3}, {4, -5}};
    double B[2] = {1, 1};
    double det, u, v, a, b;

    // Determinant of A
    det = A[0][0]*A[1][1] - A[0][1]*A[1][0];
\end{lstlisting}
\end{frame}

% C Code Part 2
\begin{frame}[fragile]{C Code (Part 2)}
\begin{lstlisting}[language=C]
    if(det == 0) {
        printf("No unique solution.\n");
        return 0;
    }

    // Cramer's Rule
    u = (B[0]*A[1][1] - B[1]*A[0][1]) / det;
    v = (A[0][0]*B[1] - A[1][0]*B[0]) / det;

    a = 1.0 / u;
    b = 1.0 / v;

    printf("Solution: a = %.2f, b = %.2f\n", a, b);
    return 0;
}
\end{lstlisting}
\end{frame}

% Python Code Part 1
\begin{frame}[fragile]{Python Code (Part 1)}
\begin{lstlisting}[language=Python]
import ctypes
import numpy as np
import matplotlib.pyplot as plt

# Load the shared library
lib = ctypes.CDLL("./c.so")

# Define the function signature for points
lib.points.argtypes = [
    ctypes.c_float,  # x_0
    ctypes.c_float,  # y_0
    ctypes.c_float,  # x_end
    ctypes.c_float,  # h
    np.ctypeslib.ndpointer(dtype=np.float32, ndim=1),  
    np.ctypeslib.ndpointer(dtype=np.float32, ndim=1),  
    ctypes.c_int     # steps
]
\end{lstlisting}
\end{frame}

% Python Code Part 2
\begin{frame}[fragile]{Python Code (Part 2)}
\begin{lstlisting}[language=Python]
# Parameters for simulation
x_0, y_0 = 0.0, 2.0
x_end, step_size = 1.0, 0.001
steps = int((x_end - x_0) / step_size) + 1

x_points = np.zeros(steps, dtype=np.float32)
y_points = np.zeros(steps, dtype=np.float32)

# Call the points function
lib.points(x_0, y_0, x_end, step_size, 
           x_points, y_points, steps)

# Theoretical solution (C = -2)
def theoretical_solution(x):
    return (-x + 4 - 2*np.exp(x))

x_theory = np.linspace(x_0, x_end, 1000)
y_theory = theoretical_solution(x_theory)

# Plot results
plt.plot(x_points, y_points, 'ro-', 
         markersize=2, linewidth=4, label="sim")
plt.plot(x_theory, y_theory, 'b-', 
         linewidth=2, label="theory")
plt.xlabel("x"); plt.ylabel("y")
plt.legend(); plt.grid(True, linestyle="--")
plt.show()
\end{lstlisting}
\end{frame}


% Plot Frame
\begin{frame}{Plot}
\begin{center}
\includegraphics[width=0.8\columnwidth]{figs/plot6.png}
\end{center}
\end{frame}

\end{document}
