\documentclass{beamer}
\usepackage[utf8]{inputenc}

\usetheme{Madrid}
\usecolortheme{default}
\usepackage{amsmath,amssymb,amsfonts,amsthm}
\usepackage{txfonts}
\usepackage{tkz-euclide}
\usepackage{listings}
\usepackage{adjustbox}
\usepackage{array}
\usepackage{tabularx}
\usepackage{gvv}
\usepackage{lmodern}
\usepackage{circuitikz}
\usepackage{tikz}
\usepackage{graphicx}

\setbeamertemplate{page number in head/foot}[totalframenumber]

\usepackage{tcolorbox}
\tcbuselibrary{minted,breakable,xparse,skins}



\definecolor{bg}{gray}{0.95}
\DeclareTCBListing{mintedbox}{O{}m!O{}}{%
  breakable=true,
  listing engine=minted,
  listing only,
  minted language=#2,
  minted style=default,
  minted options={%
    linenos,
    gobble=0,
    breaklines=true,
    breakafter=,,
    fontsize=\small,
    numbersep=8pt,
    #1},
  boxsep=0pt,
  left skip=0pt,
  right skip=0pt,
  left=25pt,
  right=0pt,
  top=3pt,
  bottom=3pt,
  arc=5pt,
  leftrule=0pt,
  rightrule=0pt,
  bottomrule=2pt,
  toprule=2pt,
  colback=bg,
  colframe=orange!70,
  enhanced,
  overlay={%
    \begin{tcbclipinterior}
    \fill[orange!20!white] (frame.south west) rectangle ([xshift=20pt]frame.north west);
    \end{tcbclipinterior}},
  #3,
}
\lstset{
    language=C,
    basicstyle=\ttfamily\small,
    keywordstyle=\color{blue},
    stringstyle=\color{orange},
    commentstyle=\color{green!60!black},
    numbers=left,
    numberstyle=\tiny\color{gray},
    breaklines=true,
    showstringspaces=false,
}
%------------------------------------------------------------
%This block of code defines the information to appear in the
%Title page
\title %optional
{4.3.34}
\date{September 15, 2025}
%\subtitle{A short story}

\author % (optional)
{Sai Hasini Pappula - EE25BTECH11044}



\begin{document}
\begin{frame}{Question}
\begin{block}{Problem}
If the line
\[
\frac{x}{a} + \frac{y}{b} = 1
\]
passes through the points $(2,-3)$ and $(4,-5)$, find $(a,b)$.
\end{block}
\end{frame}

\begin{frame}{Solution: Step 1}
The given points are
\begin{equation}
\myvec{x}_1 = \begin{bmatrix}2\\-3\end{bmatrix}, \quad
\myvec{x}_2 = \begin{bmatrix}4\\-5\end{bmatrix}.
\end{equation}

\bigskip
The direction vector of the line is
\begin{equation}
\myvec{m} = \myvec{x}_2 - \myvec{x}_1
= \begin{bmatrix}2 \\ -2\end{bmatrix}.
\end{equation}
\end{frame}

\begin{frame}{Solution: Step 2}
The normal vector $\myvec{n} = \begin{bmatrix}n_1 \\ n_2\end{bmatrix}$ must satisfy
\begin{equation}
\vec{n}^T \vec{m} = 0.
\end{equation}

\begin{equation}
\begin{bmatrix}n_1 & n_2\end{bmatrix}
\begin{bmatrix}2 \\ -2\end{bmatrix} = 0
\quad \Rightarrow \quad n_1 = n_2.
\end{equation}

So we can take
\begin{equation}
\myvec{n} = \begin{bmatrix}1 \\ 1\end{bmatrix}.
\end{equation}
\end{frame}

\begin{frame}{Solution: Step 3}
The line equation is
\begin{equation}
\vec{n}^T \vec{x} = c
\quad \Rightarrow \quad
\begin{bmatrix}1 & 1\end{bmatrix}
\begin{bmatrix}x \\ y\end{bmatrix} = c.
\end{equation}

Substitute point $\myvec{x}_1 = \begin{bmatrix}2\\-3\end{bmatrix}$:
\begin{equation}
c = 2 - 3 = -1.
\end{equation}

\bigskip
\textbf{Final Equation:}
\begin{equation}
x + y = -1.
\end{equation}
The line equation can be expressed as
\begin{equation}
\begin{bmatrix}1 & 1\end{bmatrix}
\begin{bmatrix}x \\ y\end{bmatrix} = -1
\end{equation}
\end{frame}



% ---------------- C code 1 ----------------
\begin{frame}[fragile]{C Code (Part 1)}
\lstset{language=C}
\begin{lstlisting}
#include <stdio.h>

int main() {
    double x1 = 2, y1 = -3;
    double x2 = 4, y2 = -5;

    // Direction vector
    double m1 = x2 - x1;
    double m2 = y2 - y1;

    // Normal vector (perpendicular)
    double n1 = m2;
    double n2 = -m1;
\end{lstlisting}
\end{frame}

% ---------------- C code 2 ----------------
\begin{frame}[fragile]{C Code (Part 2)}
\lstset{language=C}
\begin{lstlisting}
    // Constant c
    double c = n1*x1 + n2*y1;

    // Line equation
    printf("Equation of line: %.2lf*x + %.2lf*y = %.2lf\n",
           n1, n2, c);

    return 0;
}
\end{lstlisting}
\end{frame}

% ---------------- Python code 1 ----------------
\begin{frame}[fragile]{Python Code (Part 1)}
\lstset{language=Python}
\begin{lstlisting}
import ctypes
import numpy as np
import matplotlib.pyplot as plt

# Load the shared library
lib = ctypes.CDLL("./c.so")

# Define the function signature for points
lib.points.argtypes = [
    ctypes.c_float,  # x_0
    ctypes.c_float,  # y_0
    ctypes.c_float,  # x_end
    ctypes.c_float,  # h
    np.ctypeslib.ndpointer(dtype=np.float32, ndim=1),  
    np.ctypeslib.ndpointer(dtype=np.float32, ndim=1),  
    ctypes.c_int     # steps
]
\end{lstlisting}
\end{frame}

% ---------------- Python code 2 ----------------
\begin{frame}[fragile]{Python Code (Part 2)}
\lstset{language=Python}
\begin{lstlisting}
# Parameters for simulation
x_0, y_0 = 0.0, 2.0
x_end, step_size = 1.0, 0.001
steps = int((x_end - x_0) / step_size) + 1

x_points = np.zeros(steps, dtype=np.float32)
y_points = np.zeros(steps, dtype=np.float32)

# Call the points function
lib.points(x_0, y_0, x_end, step_size, 
           x_points, y_points, steps)

# Theoretical solution (C = -2)
def theoretical_solution(x):
    return (-x + 4 - 2*np.exp(x))
\end{lstlisting}
\end{frame}

% ---------------- Python code 3 ----------------
\begin{frame}[fragile]{Python Code (Part 3)}
\lstset{language=Python}
\begin{lstlisting}
# Generate theory curve
x_theory = np.linspace(x_0, x_end, 1000)
y_theory = theoretical_solution(x_theory)

# Plot results
plt.plot(x_points, y_points, 'ro-', 
         markersize=2, linewidth=4, label="sim")
plt.plot(x_theory, y_theory, 'b-', 
         linewidth=2, label="theory")

plt.xlabel("x")
plt.ylabel("y")
plt.legend()
plt.grid(True, linestyle="--")
plt.show()
\end{lstlisting}
\end{frame}


% ---------------- Plot ----------------
\begin{frame}{Plot of the Line}
\begin{center}
\includegraphics[width=0.7\columnwidth]{figs/plot6.png}
\end{center}
\end{frame}

\end{document}
