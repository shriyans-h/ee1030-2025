\documentclass{beamer}
\usepackage{amsmath}
\usepackage{gvv}

\title{Question 5.2.3}
\author{AI25BTECH11040 - Vivaan Parashar}
\date{\today}

\begin{document}

\frame{\titlepage}

\begin{frame}
    \frametitle{Question: }
    Solve the following system of linear equations:
    \begin{align*}
        9x + 3y + 12 = 0\\
        18x + 6y + 24 = 0
    \end{align*}
\end{frame}

\begin{frame}
    \frametitle{Solution: }
    The given equations can be rewritten as:
    \begin{align}
        9x + 3y  & = -12 \\
        18x + 6y & = -24
    \end{align}
    We can represent this system of equations in matrix form as:
    \begin{align}
        \vec{M}\vec{X} = \vec{D} \\
        \implies \myvec{9 & 3    \\ 18 & 6} \myvec{x \\ y} = \myvec{-12 \\ -24}
    \end{align}
    To obtain values of $x$ and $y$, we can multiply both sides by the inverse of the coefficient matrix on the left side, but in this case, the coefficient matrix has a rank of 1 ($R_2 = 2R_1$). This implies that either the system has no solutions or infinite solutions.\\
    If the system has infinite solution, then the rank of the augmented matrix $\myvec{\vec{M} & \vec{D}}$ should also have a rank of 1.
\end{frame}
\begin{frame}
    \begin{align}
        \rank(\myvec{\vec{M}            & \vec{D}}) = \rank(\myvec{9 & 3   & -12 \\ 18 & 6 & -24})\\
        \myvec{9                        & 3                          & -12       \\ 18 & 6 & -24} \xleftrightarrow{R_2 = R_2 - 2R_1} \myvec{9 & 3 & -12 \\ 0 & 0 & 0}\\
        \therefore \rank(\myvec{\vec{M} & \vec{D}}) = 1
    \end{align}
    Therefore, this system of equations has infinite solutions, which are all values of $(x, y) \in \mathbb{R}^2$ that satisfy the equation $3x + y = -4$.\\
\end{frame}

\begin{frame}
    \frametitle{Plot: }
    \begin{figure}[h!]
        \centering
        \includegraphics[width=0.9\columnwidth]{../figs/plot.png}
        \caption{Graph of line representing all possible solutions}
        \label{fig:5.2.3}
    \end{figure}
\end{frame}

\end{document}
