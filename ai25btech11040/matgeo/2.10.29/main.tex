\documentclass[a4paper, 12pt]{article}

\usepackage{geometry}
\usepackage{amsmath}
\usepackage{gvv}

\title{Question 2.10.29}
\author{AI25BTECH11040 - Vivaan Parashar}
\date{\today}

\begin{document}

\maketitle

\section{Question: }
The volume of the parallelopiped whose sides are given by $\textit{OA} = 2\vec{i}-2\vec{j}$, $\textit{OB} = \vec{i}+\vec{j}-\vec{k}$, $\textit{OC} = 3\vec{i} - \vec{k}$, is

\section{Solution: }
To find the volume of the parallelopiped, we can use the scalar triple product formula:
\begin{align}
    V = \norm{\vec{OA}^{\mathrm{T}}(\vec{OB} \times \vec{OC})} \\
    \vec{OA} = \myvec{2 \\ -2 \\ 0}, \quad
    \vec{OB} = \myvec{1 \\ 1 \\ -1}, \quad
    \vec{OC} = \myvec{3 \\ 0 \\ -1}\\
\end{align}
To find the cross product between two vectors $\vec{a} = \myvec{a_1 \\ a_2 \\ a_3}$ and $\vec{b} = \myvec{b_1 \\ b_2 \\ b_3}$, we can use the matrix multiplication method, by first defining a new matrix $[\vec{a}]_{\times}$ as follows:
\begin{align}
    [\vec{a}]_{\times} = \myvec{0 & -a_3 & a_2 \\ a_3 & 0 & -a_1 \\ -a_2 & a_1 & 0} \\
    \text{Now, }\vec{a} \times \vec{b} = [\vec{a}]_{\times}\vec{b}
\end{align}
Using this method, we can find the cross product $\vec{OB} \times \vec{OC}$ as follows:
\begin{align}
    [\vec{OB}]_{\times} = \myvec{0 & 1 & 1 \\ -1 & 0 & -1 \\ -1 & 1 & 0} \\
    \vec{OB} \times \vec{OC} = [\vec{OB}]_{\times}\vec{OC} = \myvec{0 & 1 & 1 \\ -1 & 0 & -1 \\ -1 & 1 & 0}\myvec{3 \\ 0 \\ -1} = \myvec{-1 \\ -2 \\ 3}
\end{align}
Now, we can find the scalar triple product $\vec{OA}^{\mathrm{T}}(\vec{OB} \times \vec{OC})$ as follows:
\begin{align}
    \vec{OA}^{\mathrm{T}}(\vec{OB} \times \vec{OC}) = \myvec{2 & -2 & 0}\myvec{-1 \\ -2 \\ 3} = \myvec{-2 + 4 + 0} = \myvec{2}
\end{align}
Finally, we can find the volume of the parallelopiped as follows:
\begin{align}
    V = \norm{\vec{OA}^{\mathrm{T}}(\vec{OB} \times \vec{OC})} = \norm{\myvec{2}} = 2
\end{align}

\end{document}
