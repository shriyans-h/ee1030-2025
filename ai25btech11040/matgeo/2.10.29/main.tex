\documentclass[a4paper, 12pt]{article}

\usepackage{geometry}
\usepackage{amsmath}
\usepackage{gvv}

\title{Question 2.10.29}
\author{AI25BTECH11040 - Vivaan Parashar}
\date{\today}

\begin{document}

\maketitle

\section{Question: }
The volume of the parallelopiped whose sides are given by $\textit{OA} = 2\vec{i}-2\vec{j}$, $\textit{OB} = \vec{i}+\vec{j}-\vec{k}$, $\textit{OC} = 3\vec{i} - \vec{k}$, is

\section{Solution: }
To find the volume of the parallelopiped, we can use the Gram matrix formula:

\begin{align}
    V = \sqrt{\det(\vec{G})} \\
    \vec{OA} = \myvec{2      \\ -2 \\ 0}, \quad
    \vec{OB} = \myvec{1      \\ 1 \\ -1}, \quad
    \vec{OC} = \myvec{3      \\ 0 \\ -1}\\
\end{align}
We define the Gram matrix $\vec{G}$ as:
\begin{align}
    \vec{G} = \myvec{\vec{OA}^{\mathrm{T}}\vec{OA} & \vec{OA}^{\mathrm{T}}\vec{OB} & \vec{OA}^{\mathrm{T}}\vec{OC}  \\
    \vec{OB}^{\mathrm{T}}\vec{OA}                  & \vec{OB}^{\mathrm{T}}\vec{OB} & \vec{OB}^{\mathrm{T}}\vec{OC}  \\
    \vec{OC}^{\mathrm{T}}\vec{OA}                  & \vec{OC}^{\mathrm{T}}\vec{OB} & \vec{OC}^{\mathrm{T}}\vec{OC}}
\end{align}
Calculating the dot products:
\begin{align}
    \vec{OA}^{\mathrm{T}}\vec{OA} = 8, \quad
    \vec{OA}^{\mathrm{T}}\vec{OB} = 0, \quad
    \vec{OA}^{\mathrm{T}}\vec{OC} = 6 \\
    \vec{OB}^{\mathrm{T}}\vec{OB} = 3, \quad
    \vec{OB}^{\mathrm{T}}\vec{OC} = 4 \\
    \vec{OC}^{\mathrm{T}}\vec{OC} = 10
\end{align}
The Gram matrix $\vec{G}$ becomes:
\begin{align}
    \vec{G} = \myvec{
    8 & 0 & 6  \\
    0 & 3 & 4  \\
    6 & 4 & 10
    }
\end{align}

Therefore, $V = \sqrt{|\det(\vec{G})|} = \sqrt{4} = 2$.

\section{Plot: }
\begin{figure}[h!]
    \centering
    \includegraphics[width=\columnwidth]{figs/plot.png}
    \caption{Parallelopiped formed by vectors $\vec{OA}$, $\vec{OB}$ and $\vec{OC}$}
    \label{fig:2.10.29}
\end{figure}


\end{document}