\documentclass[a4paper, 12pt]{article}

\usepackage{geometry}
\usepackage{amsmath}
\usepackage{gvv}

\title{Question 2.3.3}
\author{AI25BTECH11040 - Vivaan Parashar}
\date{\today}

\begin{document}

\maketitle

\section{Question: }
If $\vec{a}$, $\vec{b}$, $\vec{c}$ are three non-zero unequal vectors such that $\vec{a}^{\mathrm{T}}\vec{b} = \vec{a}^{\mathrm{T}}\vec{c}$, then find the angle between $\vec{a}$ and $\vec{b}-\vec{c}$.

\section{Solution: }
The angle $\theta$ between two vectors $\vec{a}$ and $\vec{b}$ is given by the formula:
\begin{align}
    \theta = \arccos \left(\ \frac{\norm{\vec{a}^{\mathrm{T}}\vec{b}}}{\norm{\vec{a}} \norm{\vec{b}}} \right)
\end{align}
In this case, we would need to find
\begin{align}
    \theta = \arccos(\frac{ \norm{\vec{a}^{\mathrm{T}}(\vec{b}-\vec{c})}}{\norm{\vec{a}} \norm{\vec{b}-\vec{c}}})\\
    \theta = \arccos(\frac{\norm{\vec{a}^{\mathrm{T}}\vec{b}-\vec{a}^{\mathrm{T}}\vec{c}}}{|\vec{a}||\vec{b}-\vec{c}|})\\
    \theta = \arccos(0) = 90^\circ\\
    \because \vec{a}^{\mathrm{T}}\vec{b}=\vec{a}^{\mathrm{T}}\vec{c} \text{ and } |\vec{a}| \ne 0, |\vec{b}-\vec{c}| \ne 0
\end{align}
Therefore, the angle between the vectors $\vec{a}$ and $\vec{b}-\vec{c}$ is $90^\circ$.

\end{document}
