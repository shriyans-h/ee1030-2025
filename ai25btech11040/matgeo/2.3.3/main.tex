\documentclass[a4paper, 12pt]{article}

\usepackage{geometry}
\usepackage{amsmath}
\usepackage{gvv}

\title{Question 2.3.3}
\author{AI25BTECH11040 - Vivaan Parashar}
\date{\today}

\begin{document}

\maketitle

\section{Question: }
If $\vec{a}$, $\vec{b}$, $\vec{c}$ are three non-zero unequal vectors such that $\vec{a}^{\mathrm{T}}\vec{b} = \vec{a}^{\mathrm{T}}\vec{c}$, then find the angle between $\vec{a}$ and $\vec{b}-\vec{c}$.

\section{Solution: }
Given that $\vec{a}^{\mathrm{T}}\vec{b} = \vec{a}^{\mathrm{T}}\vec{c}$, we can rewrite this as:
\begin{align}
    \vec{a}^{\mathrm{T}}\vec{b} - \vec{a}^{\mathrm{T}}\vec{c} = 0 \\
    \vec{a}^{\mathrm{T}}(\vec{b} - \vec{c}) = 0
\end{align}
This implies that the dot product of $\vec{a}$ and $\vec{b} - \vec{c}$ is zero, ie these are orthogonal vectors.
Therefore, the angle between $\vec{a}$ and $\vec{b} - \vec{c}$ is $90^\circ$.

\end{document}
