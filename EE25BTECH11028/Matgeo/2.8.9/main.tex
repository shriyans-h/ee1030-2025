\let\negmedspace\undefined
\let\negthickspace\undefined
\documentclass[journal]{IEEEtran}
\usepackage[a5paper, margin=10mm, onecolumn]{geometry}
%\usepackage{lmodern} % Ensure lmodern is loaded for pdflatex
\usepackage{tfrupee} % Include tfrupee package

\setlength{\headheight}{1cm} % Set the height of the header box
\setlength{\headsep}{0mm}     % Set the distance between the header box and the top of the text

\usepackage{gvv-book}
\usepackage{gvv}
\usepackage{cite}
\usepackage{amsmath,amssymb,amsfonts,amsthm}
\usepackage{algorithmic}
\usepackage{graphicx}
\usepackage{textcomp}
\usepackage{xcolor}
\usepackage{txfonts}
\usepackage{listings}
\usepackage{enumitem}
\usepackage{mathtools}
\usepackage{gensymb}
\usepackage{comment}
\usepackage[breaklinks=true]{hyperref}
\usepackage{tkz-euclide} 
\usepackage{listings}
% \usepackage{gvv}                                        
\def\inputGnumericTable{}                                 
\usepackage[latin1]{inputenc}                                
\usepackage{color}                                            
\usepackage{array}                                            
\usepackage{longtable}                                       
\usepackage{calc}                                             
\usepackage{multirow}                                         
\usepackage{hhline}                                           
\usepackage{ifthen}                                           
\usepackage{lscape}
\begin{document}

\bibliographystyle{IEEEtran}
\vspace{3cm}

\title{2.8.9}
\author{EE25btech11028 - J.Navya sri}
% \maketitle
% \newpage
% \bigskip
{\let\newpage\relax\maketitle}


\textbf{Question:} \\
Let $\vec{a}, \vec{b}, \vec{c}$ be three vectors such that 
$|\vec{a}|=3,\; |\vec{b}|=4,\; |\vec{c}|=5$, and each one of them is perpendicular to the sum of the other two. 
Find $|\vec{a}+\vec{b}+\vec{c}|$.

\bigskip

\textbf{Solution:} \\
Let the Gram matrix $G$ of the three vectors $(a), (b), (c)$ be
\begin{equation}
G = 
\myvec{
(a,a) & (a,b) & (a,c) \\
(b,a) & (b,b) & (b,c) \\
(c,a) & (c,b) & (c,c)
}
=
\myvec{
9 & x & z \\
x & 16 & y \\
z & y & 25
}
\end{equation}
where
\begin{equation}
x = (a,b), \quad y = (b,c), \quad z = (c,a).
\end{equation}

The conditions ``each vector is perpendicular to the sum of the other two'' give
\begin{align}
(a,(b)+(c)) &= 0, \\
(b,(c)+(a)) &= 0, \\
(c,(a)+(b)) &= 0.
\end{align}

In terms of $x,y,z$, equations (3)--(5) become
\begin{align}
x+z &= 0, \\
x+y &= 0, \\
y+z &= 0.
\end{align}

From (6) we get $z=-x$, and from (7) we get $y=-x$. Substituting into (8) gives
\begin{equation}
(-x)+(-x) = 0 \quad \Rightarrow \quad x=0.
\end{equation}

Hence
\begin{equation}
x=y=z=0.
\end{equation}

So $(a),(b),(c)$ are pairwise orthogonal.  

Therefore
\begin{align}
|(a)+(b)+(c)|^2 &= (a+b+c)\cdot(a+b+c) \\
&= (a,a) + (b,b) + (c,c) \\
&= |a|^2 + |b|^2 + |c|^2 \\
&= 9 + 16 + 25 \\
&= 50.
\end{align}

Thus
\begin{equation}
|(a)+(b)+(c)| = \sqrt{50} = 5\sqrt{2}.
\end{equation}

\textbf{Graphical Representation:}

\begin{figure}[h!]
\centering
\begin{tikzpicture}[tdplot_main_coords, scale=0.9]

  % axes
  \draw[->] (0,0,0) -- (6.5,0,0) node[anchor=north east]{$x$};
  \draw[->] (0,0,0) -- (0,6.5,0) node[anchor=north west]{$y$};
  \draw[->] (0,0,0) -- (0,0,6.5) node[anchor=south]{$z$};

  % vector lengths: |a|=3, |b|=4, |c|=5
  \coordinate (O) at (0,0,0);
  \coordinate (A) at (3,0,0);
  \coordinate (B) at (0,4,0);
  \coordinate (C) at (0,0,5);

  % resultant a+b+c
  \coordinate (R) at (3,4,5);

  % draw vectors
  \draw[->, line width=1pt] (O) -- (A) node[midway, below right] {$\vec a$};
  \draw[->, line width=1pt] (O) -- (B) node[midway, above left] {$\vec b$};
  \draw[->, line width=1pt] (O) -- (C) node[midway, right] {$\vec c$};

  % resultant
  \draw[->, very thick, red] (O) -- (R) node[pos=0.85, above right] {$\vec a+\vec b+\vec c$};

  % dashed parallelepiped edges
  \draw[dashed] (A) -- (3,4,0) -- (B);
  \draw[dashed] (A) -- (3,0,5) -- (C);
  \draw[dashed] (B) -- (0,4,5) -- (C);
  \draw[dashed] (3,4,0) -- (3,4,5);

\end{tikzpicture}
        % Add "Fig. 4" text down the figure
    \vspace{0.5cm}% space between figure and text
\centerline{\textbf{Fig. 4}}
\end{figure}
\end{document}