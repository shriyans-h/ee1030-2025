\let\negmedspace\undefined
\let\negthickspace\undefined
\documentclass[journal]{IEEEtran}
\usepackage[a5paper, margin=10mm, onecolumn]{geometry}
%\usepackage{lmodern} % Ensure lmodern is loaded for pdflatex
\usepackage{tfrupee} % Include tfrupee package

\setlength{\headheight}{1cm} % Set the height of the header box
\setlength{\headsep}{0mm}     % Set the distance between the header box and the top of the text

\usepackage{gvv-book}
\usepackage{gvv}
\usepackage{cite}
\usepackage{amsmath,amssymb,amsfonts,amsthm}
\usepackage{algorithmic}
\usepackage{graphicx}
\usepackage{textcomp}
\usepackage{xcolor}
\usepackage{txfonts}
\usepackage{listings}
\usepackage{enumitem}
\usepackage{mathtools}
\usepackage{gensymb}
\usepackage{comment}
\usepackage[breaklinks=true]{hyperref}
\usepackage{tkz-euclide} 
\usepackage{listings}
% \usepackage{gvv}                                        
\def\inputGnumericTable{}                                 
\usepackage[latin1]{inputenc}                                
\usepackage{color}                                            
\usepackage{array}                                            
\usepackage{longtable}                                       
\usepackage{calc}                                             
\usepackage{multirow}                                         
\usepackage{hhline}                                           
\usepackage{ifthen}                                           
\usepackage{lscape}
\begin{document}

\bibliographystyle{IEEEtran}
\vspace{3cm}

\title{2.8.9}
\author{EE25btech11028 - J.Navya sri}
% \maketitle
% \newpage
% \bigskip
{\let\newpage\relax\maketitle}


\textbf{Question:} \\
Let $\vec{a}, \vec{b}, \vec{c}$ be three vectors such that 
$|\vec{a}|=3,\; |\vec{b}|=4,\; |\vec{c}|=5$, and each one of them is perpendicular to the sum of the other two. 
Find $|\vec{a}+\vec{b}+\vec{c}|$.

\bigskip

\textbf{Solution:} \\

Let 
\begin{equation}
|\vec{a}|=3,\quad |\vec{b}|=4,\quad |\vec{c}|=5
\end{equation}

Since each vector is perpendicular to the sum of the other two, we have:
\begin{equation}
\vec{a}\cdot(\vec{b}+\vec{c})=0,\quad 
\vec{b}\cdot(\vec{c}+\vec{a})=0,\quad 
\vec{c}\cdot(\vec{a}+\vec{b})=0
\end{equation}

Introduce notation:
\begin{equation}
s=\vec{a}\cdot\vec{b},\quad t=\vec{b}\cdot\vec{c},\quad u=\vec{c}\cdot\vec{a}
\end{equation}

From (2), the equations become:
\begin{equation}
s+u=0,\quad t+s=0,\quad u+t=0
\end{equation}

From the first equation,
\begin{equation}
u=-s
\end{equation}

From the second equation,
\begin{equation}
t=-s
\end{equation}

Substitute (5) and (6) into the third equation:
\begin{equation}
(-s)+(-s)=-2s=0 \;\;\Rightarrow\;\; s=0
\end{equation}

Hence,
\begin{equation}
s=t=u=0
\end{equation}

This shows that $\vec{a},\vec{b},\vec{c}$ are mutually perpendicular.

Now,
\begin{equation}
\|\vec{a}+\vec{b}+\vec{c}\|^2
= \|\vec{a}\|^2+\|\vec{b}\|^2+\|\vec{c}\|^2 + 2(s+t+u)
\end{equation}

Substitute values from (1) and (8):
\begin{equation}
= 3^2+4^2+5^2+2(0+0+0)
\end{equation}

\begin{equation}
= 9+16+25=50
\end{equation}

Therefore,
\begin{equation}
\|\vec{a}+\vec{b}+\vec{c}\|=\sqrt{50}=5\sqrt{2}
\end{equation}

\bigskip

\textbf{Final Answer:} 
\[
\boxed{5\sqrt{2}}
\]


\textbf{Graphical Representation:}

\begin{figure}[h!]
\centering
\begin{tikzpicture}[tdplot_main_coords, scale=0.9]

  % axes
  \draw[->] (0,0,0) -- (6.5,0,0) node[anchor=north east]{$x$};
  \draw[->] (0,0,0) -- (0,6.5,0) node[anchor=north west]{$y$};
  \draw[->] (0,0,0) -- (0,0,6.5) node[anchor=south]{$z$};

  % vector lengths: |a|=3, |b|=4, |c|=5
  \coordinate (O) at (0,0,0);
  \coordinate (A) at (3,0,0);
  \coordinate (B) at (0,4,0);
  \coordinate (C) at (0,0,5);

  % resultant a+b+c
  \coordinate (R) at (3,4,5);

  % draw vectors
  \draw[->, line width=1pt] (O) -- (A) node[midway, below right] {$\vec a$};
  \draw[->, line width=1pt] (O) -- (B) node[midway, above left] {$\vec b$};
  \draw[->, line width=1pt] (O) -- (C) node[midway, right] {$\vec c$};

  % resultant
  \draw[->, very thick, red] (O) -- (R) node[pos=0.85, above right] {$\vec a+\vec b+\vec c$};

  % dashed parallelepiped edges
  \draw[dashed] (A) -- (3,4,0) -- (B);
  \draw[dashed] (A) -- (3,0,5) -- (C);
  \draw[dashed] (B) -- (0,4,5) -- (C);
  \draw[dashed] (3,4,0) -- (3,4,5);

\end{tikzpicture}
        % Add "Fig. 4" text down the figure
    \vspace{0.5cm}
\centerline{\textbf{Fig. 4}}
\end{figure}
\end{document}
