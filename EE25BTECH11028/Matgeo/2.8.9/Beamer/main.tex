\documentclass{beamer}
\usepackage[utf8]{inputenc}

\usetheme{Madrid}
\usecolortheme{default}
\usepackage{amsmath,amssymb,amsfonts,amsthm}
\usepackage{txfonts}
\usepackage{tkz-euclide}
\usepackage{listings}
\usepackage{adjustbox}
\usepackage{array}
\usepackage{tabularx}
\usepackage{gvv}
\usepackage{lmodern}
\usepackage{circuitikz}
\usepackage{tikz}
\usepackage{graphicx}

\setbeamertemplate{page number in head/foot}[totalframenumber]

\usepackage{tcolorbox}
\tcbuselibrary{minted,breakable,xparse,skins}



\definecolor{bg}{gray}{0.95}
\DeclareTCBListing{mintedbox}{O{}m!O{}}{%
  breakable=true,
  listing engine=minted,
  listing only,
  minted language=#2,
  minted style=default,
  minted options={%
    linenos,
    gobble=0,
    breaklines=true,
    breakafter=,,
    fontsize=\small,
    numbersep=8pt,
    #1},
  boxsep=0pt,
  left skip=0pt,
  right skip=0pt,
  left=25pt,
  right=0pt,
  top=3pt,
  bottom=3pt,
  arc=5pt,
  leftrule=0pt,
  rightrule=0pt,
  bottomrule=2pt,

  colback=bg,
  colframe=orange!70,
  enhanced,
  overlay={%
    \begin{tcbclipinterior}
    \fill[orange!20!white] (frame.south west) rectangle ([xshift=20pt]frame.north west);
    \end{tcbclipinterior}},
  #3,
}
\lstset{
    language=C,
    basicstyle=\ttfamily\small,
    keywordstyle=\color{blue},
    stringstyle=\color{orange},
    commentstyle=\color{green!60!black},
    numbers=left,
    numberstyle=\tiny\color{gray},
    breaklines=true,
    showstringspaces=false,
}
%------------------------------------------------------------
%This block of code defines the information to appear in the
%Title page
\title %optional
{2.8.9}
\date{september 2025}
%\subtitle{A short story}

\author % (optional)
{J.NAVYASRI- EE25BTECH11028}

\begin{document}

\frame{\titlepage}
\begin{frame}{Question}
Let $\vec{a}, \vec{b}, \vec{c}$ be three vectors such that 
$|\vec{a}|=3,\; |\vec{b}|=4,\; |\vec{c}|=5$, and each one of them is perpendicular to the sum of the other two. 
Find $|\vec{a}+\vec{b}+\vec{c}|$.
\end{frame}

% Step 1: Theoretical solution
\begin{frame}{Theoretical solution}
Let 
\begin{equation}
|\vec{a}|=3,\quad |\vec{b}|=4,\quad |\vec{c}|=5
\end{equation}

Since each vector is perpendicular to the sum of the other two, we have:
\begin{equation}
\vec{a}\cdot(\vec{b}+\vec{c})=0,\quad 
\vec{b}\cdot(\vec{c}+\vec{a})=0,\quad 
\vec{c}\cdot(\vec{a}+\vec{b})=0
\end{equation}

Introduce notation:
\begin{equation}
s=\vec{a}\cdot\vec{b},\quad t=\vec{b}\cdot\vec{c},\quad u=\vec{c}\cdot\vec{a}
\end{equation}

From (2), the equations become:
\begin{equation}
s+u=0,\quad t+s=0,\quad u+t=0
\end{equation}
\end{frame}

% Step 2: Theoretical solution 
\begin{frame}{Theoretical solution}
From the first equation,
\begin{equation}
u=-s
\end{equation}

From the second equation,
\begin{equation}
t=-s
\end{equation}

Substitute (5) and (6) into the third equation:
\begin{equation}
(-s)+(-s)=-2s=0 \;\;\Rightarrow\;\; s=0
\end{equation}

Hence,
\begin{equation}
s=t=u=0
\end{equation}

This shows that $\vec{a},\vec{b},\vec{c}$ are mutually perpendicular.

Now,
\begin{equation}
\|\vec{a}+\vec{b}+\vec{c}\|^2
= \|\vec{a}\|^2+\|\vec{b}\|^2+\|\vec{c}\|^2 + 2(s+t+u)
\end{equation}
\end{frame}

% Step 3: Theoretical solution 
\begin{frame}{Theoretical solution}
Substitute values from (1) and (8):
\begin{equation}
= 3^2+4^2+5^2+2(0+0+0)
\end{equation}

\begin{equation}
= 9+16+25=50
\end{equation}

Therefore,
\begin{equation}
\|\vec{a}+\vec{b}+\vec{c}\|=\sqrt{50}=5\sqrt{2}
\end{equation}

\bigskip

\textbf{Final Answer:} 
\[
\boxed{5\sqrt{2}}
\]
\end{frame}

\begin{frame}[fragile]
    \frametitle{Python Code}
    \begin{lstlisting}
import matplotlib.pyplot as plt
import numpy as np
from mpl_toolkits.mplot3d import Axes3D

# Define vectors
a = np.array([3, 0, 0])   # vector a
b = np.array([0, 4, 0])   # vector b
c = np.array([0, 0, 5])   # vector c

# Origin
O = np.array([0, 0, 0])

# Resultant
R = a + b + c

\end{lstlisting}
\end{frame}


\begin{frame}[fragile]
    \frametitle{Python Code}
    \begin{lstlisting}
fig = plt.figure(figsize=(8, 6))
ax = fig.add_subplot(111, projection='3d')

# Plot coordinate axes
ax.quiver(0, 0, 0, 6, 0, 0, arrow_length_ratio=0.1, color='k')
ax.quiver(0, 0, 0, 0, 6, 0, arrow_length_ratio=0.1, color='k')
ax.quiver(0, 0, 0, 0, 0, 6, arrow_length_ratio=0.1, color='k')

\end{lstlisting}
\end{frame}


\begin{frame}[fragile]
    \frametitle{Python Code}
    \begin{lstlisting}
# Plot vectors a, b, c
ax.quiver(*O, *a, color='black', linewidth=2)
ax.text(*a, r'$\vec{a}$', color='black')

ax.quiver(*O, *b, color='black', linewidth=2)
ax.text(*b, r'$\vec{b}$', color='black')

ax.quiver(*O, *c, color='black', linewidth=2)
ax.text(*c, r'$\vec{c}$', color='black')

# Plot resultant
ax.quiver(*O, *R, color='red', linewidth=2)
ax.text(*R, r'$\vec{a}+\vec{b}+\vec{c}$', color='red')

\end{lstlisting}
\end{frame}

\begin{frame}[fragile]
    \frametitle{Python Code}
    \begin{lstlisting}
# Draw dashed parallelepiped edges
ax.plot([a[0], R[0]], [a[1], R[1]], [a[2], R[2]], 'k--')
ax.plot([b[0], R[0]], [b[1], R[1]], [b[2], R[2]], 'k--')
ax.plot([c[0], R[0]], [c[1], R[1]], [c[2], R[2]], 'k--')

# Labels
ax.set_xlabel('x')
ax.set_ylabel('y')
ax.set_zlabel('z')
ax.set_title("Fig. 4")

# Set limits
ax.set_xlim([0, 6])
ax.set_ylim([0, 6])
ax.set_zlim([0, 6])

plt.show()
\end{lstlisting}
\end{frame}


\begin{frame}[fragile]
\frametitle{C Code}
\begin{lstlisting}
#include <stdio.h>
#include <math.h>

int main() {
    // Given magnitudes
    double a = 3.0;
    double b = 4.0;
    double c = 5.0;

    // Since a, b, c are mutually perpendicular:
    double sumSq = a*a + b*b + c*c;

    // Magnitude of (a+b+c)
    double result = sqrt(sumSq);

    printf("||a + b + c|| = sqrt(%.0f) = %.4f\n", sumSq, result);

    return 0;
}
\end{lstlisting}

\end{frame}

    \begin{frame}[fragile]
\frametitle{Python and C Code}

\begin{lstlisting}
import numpy as np

# Part 1: Input vectors
a = np.array([3.0, 0, 0])   # Along X-axis
b = np.array([0, 4.0, 0])   # Along Y-axis
c = np.array([0, 0, 5.0])   # Along Z-axis

print("---- Part 1: Input Vectors ----")
print("a =", a)
print("b =", b)
print("c =", c)

\end{lstlisting}

\end{frame}


    \begin{frame}[fragile]
\frametitle{Python and C Code}

\begin{lstlisting}
import numpy as np

# Part 2: Resultant vector calculation
a = np.array([3.0, 0, 0])
b = np.array([0, 4.0, 0])
c = np.array([0, 0, 5.0])

res = a + b + c
sumSq = np.dot(a, a) + np.dot(b, b) + np.dot(c, c)
result = np.linalg.norm(res)

print("\n---- Part 2: Resultant ----")
print(f"||a + b + c|| = sqrt({sumSq:.0f}) = {result:.4f}")

\end{lstlisting}

\end{frame}


    \begin{frame}[fragile]
\frametitle{Python and C Code}

\begin{lstlisting}
import numpy as np
import matplotlib.pyplot as plt

# Part 3: Plot vectors
a = np.array([3.0, 0, 0])
b = np.array([0, 4.0, 0])
c = np.array([0, 0, 5.0])
res = a + b + c

fig = plt.figure()
ax = fig.add_subplot(111, projection='3d')

ax.quiver(0, 0, 0, a[0], a[1], a[2], color='r', label='a (3)')
ax.quiver(0, 0, 0, b[0], b[1], b[2], color='g', label='b (4)')
ax.quiver(0, 0, 0, c[0], c[1], c[2], color='b', label='c (5)')
ax.quiver(0, 0, 0, res[0], res[1], res[2], color='k', linewidth=2, label='a+b+c')

\end{lstlisting}

\end{frame}



    \begin{frame}[fragile]
\frametitle{Python and C Code}

\begin{lstlisting}
import matplotlib.pyplot as plt

# Part 4: Plot setup and show
fig = plt.figure()
ax = fig.add_subplot(111, projection='3d')

ax.set_xlim([0, 6])
ax.set_ylim([0, 6])
ax.set_zlim([0, 6])
ax.set_xlabel('X')
ax.set_ylabel('Y')
ax.set_zlabel('Z')
ax.set_title('Vector Addition: a + b + c')
ax.legend()

plt.show()

\end{lstlisting}

\end{frame}



\textbf{Graphical Representation:}

\begin{figure}[h!]
\centering
\begin{tikzpicture}[tdplot_main_coords, scale=0.9]

  % axes
  \draw[->] (0,0,0) -- (6.5,0,0) node[anchor=north east]{$x$};
  \draw[->] (0,0,0) -- (0,6.5,0) node[anchor=north west]{$y$};
  \draw[->] (0,0,0) -- (0,0,6.5) node[anchor=south]{$z$};

  % vector lengths: |a|=3, |b|=4, |c|=5matgeo
  \coordinate (O) at (0,0,0);
  \coordinate (A) at (3,0,0);
  \coordinate (B) at (0,4,0);
  \coordinate (C) at (0,0,5);

  % resultant a+b+c
  \coordinate (R) at (3,4,5);

  % draw vectors
  \draw[->, line width=1pt] (O) -- (A) node[midway, below right] {$\vec a$};
  \draw[->, line width=1pt] (O) -- (B) node[midway, above left] {$\vec b$};
  \draw[->, line width=1pt] (O) -- (C) node[midway, right] {$\vec c$};

  % resultant
  \draw[->, very thick, red] (O) -- (R) node[pos=0.85, above right] {$\vec a+\vec b+\vec c$};

  % dashed parallelepiped edges
  \draw[dashed] (A) -- (3,4,0) -- (B);
  \draw[dashed] (A) -- (3,0,5) -- (C);
  \draw[dashed] (B) -- (0,4,5) -- (C);
  \draw[dashed] (3,4,0) -- (3,4,5);

\end{tikzpicture}
        % Add "Fig. 4" text down the figure
    \vspace{0.5cm}
\centerline{\textbf{Fig. 4}}
\end{figure}
\end{document}

