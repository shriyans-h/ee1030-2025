 \let\negmedspace\undefined
\let\negthickspace\undefined
\documentclass[journal]{IEEEtran}
\usepackage[a5paper, margin=10mm, onecolumn]{geometry}
%\usepackage{lmodern} % Ensure lmodern is loaded for pdflatex
\usepackage{tfrupee} % Include tfrupee package

\setlength{\headheight}{1cm} % Set the height of the header box
\setlength{\headsep}{0mm}     % Set the distance between the header box and the top of the text

\usepackage{gvv-book}
\usepackage{gvv}
\usepackage{cite}
\usepackage{amsmath,amssymb,amsfonts,amsthm}
\usepackage{algorithmic}
\usepackage{graphicx}
\usepackage{textcomp}
\usepackage{xcolor}
\usepackage{txfonts}
\usepackage{listings}
\usepackage{enumitem}
\usepackage{mathtools}
\usepackage{gensymb}
\usepackage{comment}
\usepackage[breaklinks=true]{hyperref}
\usepackage{tkz-euclide} 
\usepackage{listings}
% \usepackage{gvv}                                        
\def\inputGnumericTable{}                                 
\usepackage[latin1]{inputenc}                                
\usepackage{color}                                            
\usepackage{array}                                            
\usepackage{longtable}                                       
\usepackage{calc}                                             
\usepackage{multirow}                                         
\usepackage{hhline}                                           
\usepackage{ifthen}                                           
\usepackage{lscape}
\begin{document}

\bibliographystyle{IEEEtran}
\vspace{3cm}


\title{1.5.16}
\author{EE25btech11028 - J.Navya sri}
% \maketitle
% \newpage
% \bigskip
{\let\newpage\relax\maketitle}

\renewcommand{\thefigure}{\theenumi}
\renewcommand{\thetable}{\theenumi}
\setlength{\intextsep}{10pt} % Space between text and floats

\textbf{Question:} Find the point \(A\) if \(AB\) is a diameter of the circle with center \(C=(3,-1)\) and point \(B=(2,6)\).
\bigskip

\textbf{Solution:}
\bigskip

\begin{tabular}[12pt]{ |c| c| c|} 
    \hline
    {Point} & {Vector} \\ 
    \hline
    B & $ \myvec{2 \\ 6} $  \\
    \hline
    C & $ \myvec{3 \\ -1} $   \\
    \hline  
    \end{tabular}

\bigskip

\textbf{Section Formula:}

If a point \(P\) divides the line joining \(A\) and \(B\) internally in the ratio \(m:n\), then
\[
\vec P = \frac{k\vec B + \vec A}{k+1}
= \begin{pmatrix}\vec A & \vec B\end{pmatrix}
\myvec{\tfrac{1}{k+1}\\[6pt]\tfrac{k}{k+1}}
\]

Here, \(C\) is the midpoint of \(AB\), i.e.\ ratio \(1:1\).
\[
\vec{C} = \frac{\vec{A}+\vec{B}}{2}
= \begin{pmatrix}\vec A & \vec B\end{pmatrix}
\myvec{\tfrac{1}{2}\\[6pt]\tfrac{1}{2}}.
\]

Substitute values:
\[
\myvec{3\\-1} =
\begin{pmatrix}\vec A & \myvec{2\\6}\end{pmatrix}
\myvec{\tfrac{1}{2}\\[6pt]\tfrac{1}{2}}.
\]

\[
2\myvec{3\\-1}=\vec A+\myvec{2\\6}
\quad\Rightarrow\quad
\vec A=2\myvec{3\\-1}-\myvec{2\\6}=\myvec{4\\-8}.
\]

\bigskip

\textbf{Rank Verification:}

Check collinearity of \(A,B,C\):
\[
\begin{pmatrix}\vec C-\vec A & \vec B-\vec A\end{pmatrix}
=\begin{pmatrix}3-4 & 2-4\\ -1-(-8) & 6-(-8)\end{pmatrix}
=\begin{pmatrix}-1 & -2\\ 7 & 14\end{pmatrix} = 0
\]


Thus, rank \(=1\) and points are collinear.

\[
\boxed{\vec A=\myvec{4\\-8}}
\]

\begin{figure}[H]
    \centering
    \includegraphics[width=0.5\linewidth]{figs/fig.png}
    \caption{}
    \label{fig:placeholder}
\end{figure}
\end{document}


