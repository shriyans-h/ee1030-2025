\let\negmedspace\undefined
\let\negthickspace\undefined
\documentclass[journal]{IEEEtran}
\usepackage[a5paper, margin=10mm, onecolumn]{geometry}
%\usepackage{lmodern} % Ensure lmodern is loaded for pdflatex
\usepackage{tfrupee} % Include tfrupee package

\setlength{\headheight}{1cm} % Set the height of the header box
\setlength{\headsep}{0mm}     % Set the distance between the header box and the top of the text

\usepackage{gvv-book}
\usepackage{gvv}
\usepackage{cite}
\usepackage{amsmath,amssymb,amsfonts,amsthm}
\usepackage{algorithmic}
\usepackage{graphicx}
\usepackage{textcomp}
\usepackage{xcolor}
\usepackage{txfonts}
\usepackage{listings}
\usepackage{enumitem}
\usepackage{mathtools}
\usepackage{gensymb}
\usepackage{comment}
\usepackage[breaklinks=true]{hyperref}
\usepackage{tkz-euclide} 
\usepackage{listings}
% \usepackage{gvv}                                        
\def\inputGnumericTable{}                                 
\usepackage[latin1]{inputenc}                                
\usepackage{color}                                            
\usepackage{array}                                            
\usepackage{longtable}                                       
\usepackage{calc}                                             
\usepackage{multirow}                                         
\usepackage{hhline}                                           
\usepackage{ifthen}                                           
\usepackage{lscape}
\begin{document}

\bibliographystyle{IEEEtran}
\vspace{3cm}

\title{1.5.16}
\author{EE25btech11028 - J.Navya sri}
% \maketitle
% \newpage
% \bigskip
{\let\newpage\relax\maketitle}

\renewcommand{\thefigure}{\theenumi}
\renewcommand{\thetable}{\theenumi}
\setlength{\intextsep}{10pt} % Space between text and floats


\numberwithin{equation}{enumi}
\numberwithin{figure}{enumi}
\renewcommand{\thetable}{\theenumi}


\textbf{Question}:\\
 Find the point $A$ where $AB$ is a diameter of a circle with center $(3,-1)$ and the point $B$ is $(2,6)$

\bigskip

\begin{tabular}[12pt]{ |c| c| c|} 
    \hline
    {Point} & {Vector} \\ 
    \hline
    B & $ \myvec{2 \\ 6} $  \\
    \hline
    C & $ \myvec{3 \\ -1} $   \\
    \hline  
\end{tabular}

\bigskip

\textbf{Using rank of matrix:}
\bigskip
Three points \( \vec{A}, \vec{C}, \vec{B} \) are collinear if
\[
\rank{\begin{pmatrix} \vec{C} - \vec{A} & \vec{B} - \vec{A} \end{pmatrix}} = 1
\]

\[
\vec{C} - \vec{A} = \myvec{3 - x\\ -1 - y}, \quad
\vec{B} - \vec{A} = \myvec{2 - x\\ 6 - y}
\]

Form matrix:
\[
\begin{pmatrix}
3 - x & 2 - x \\
-1 - y & 6 - y
\end{pmatrix}
\]

Apply row operation:
\[
R_2 \rightarrow (3 - x)R_2 - (-1 - y)R_1
\]

\[
(3 - x)(6 - y) + (1 + y)(2 - x) = 0
\Rightarrow 20 - y - 7x = 0 \Rightarrow \boxed{7x + y = 20}
\]

\bigskip

\textbf{Using midpoint formula:}
\[
\vec{C} = \frac{\vec{A} + \vec{B}}{2} \Rightarrow \vec{A} = 2\vec{C} - \vec{B}
\]

\[
\vec{A} = 2\myvec{3\\-1} - \myvec{2\\6} = \myvec{6\\-2} - \myvec{2\\6} = \boxed{\vec{A} = \myvec{4\\-8}}
\]

\begin{figure}[H]
    \centering
    \includegraphics[width=0.5\linewidth]{figs/fig.png}
    \caption{}
    \label{fig:placeholder}
\end{figure}
\end{document}

Midpoint of $A(4, -8)$ and $B(2, 6)$ is  
\[
\left( \frac{4+2}{2}, \; \frac{-8+6}{2} \right) = (3, -1)

\]

\begin{figure} [H]
    \centering
    \includegraphics[width=0.8\columnwidth]{figs/fig.png}
    
    \label{fig:placeholder}
\end{figure}
\end{document}
