\let\negmedspace\undefined
\let\negthickspace\undefined
\documentclass[journal]{IEEEtran}
\usepackage[a5paper, margin=10mm, onecolumn]{geometry}
%\usepackage{lmodern} % Ensure lmodern is loaded for pdflatex
\usepackage{tfrupee} % Include tfrupee package

\setlength{\headheight}{1cm} % Set the height of the header box
\setlength{\headsep}{0mm}     % Set the distance between the header box and the top of the text

\usepackage{gvv-book}
\usepackage{gvv}
\usepackage{cite}
\usepackage{amsmath,amssymb,amsfonts,amsthm}
\usepackage{algorithmic}
\usepackage{graphicx}
\usepackage{textcomp}
\usepackage{xcolor}
\usepackage{txfonts}
\usepackage{listings}
\usepackage{enumitem}
\usepackage{mathtools}
\usepackage{gensymb}
\usepackage{comment}
\usepackage[breaklinks=true]{hyperref}
\usepackage{tkz-euclide} 
\usepackage{listings}
% \usepackage{gvv}                                        
\def\inputGnumericTable{}                                 
\usepackage[latin1]{inputenc}                                
\usepackage{color}                                            
\usepackage{array}                                            
\usepackage{longtable}                                       
\usepackage{calc}                                             
\usepackage{multirow}                                         
\usepackage{hhline}                                           
\usepackage{ifthen}                                           
\usepackage{lscape}
\begin{document}

\bibliographystyle{IEEEtran}
\vspace{3cm}

\title{1.5.16}
\author{EE25btech11028 - J.Navya sri}
% \maketitle
% \newpage
% \bigskip
{\let\newpage\relax\maketitle}

\renewcommand{\thefigure}{\theenumi}
\renewcommand{\thetable}{\theenumi}
\setlength{\intextsep}{10pt} % Space between text and floats


\numberwithin{equation}{enumi}
\numberwithin{figure}{enumi}
\renewcommand{\thetable}{\theenumi}
\textbf{Question:}  
Find the coordinates of a point $A$ where $AB$ is a diameter of the circle with center  
$(3, -1)$ and the point $B$ is $(2, 6)$.
\bigskip

\textbf{Solution:}
let $C$ be the center of circle
\bigskip

 
\begin{tabular}[12pt]{ |c| c| c|} 
    \hline
    {Point} & {Vector} \\ 
    \hline
    B & $ \myvec{2 \\ 6} $  \\
    \hline
    C & $ \myvec{3 \\ -1} $   \\
    \hline  
    \end{tabular}
\bigskip
\begin{document}

Given points:
\[
\begin{tabular}{c}
A(x,y)\\
B(2,6)\\
C(3,-1) 
\end{tabular}
\]
As $C$ is the center of the circle, it divides $AB$ in $1:1$ ratio.  
If $P$ divides $QR$ in $k:1$ ratio, then

\[
P = \frac{kR + 1(Q)}{k+1}
\]

Now,
\[
C =  \begin{pmatrix}
\frac{A+B} {2} 
\end{pmatrix}
\]

\[
2 \begin{pmatrix} 3 \\ -1 \end{pmatrix}
=
\begin{pmatrix} x \\ y \end{pmatrix}
+
\begin{pmatrix} 2 \\ 6 \end{pmatrix}
\]

\[
\begin{pmatrix} 6 \\ -2 \end{pmatrix}
=
\begin{pmatrix} x+2 \\ y+6 \end{pmatrix}
\]

\[
x+2 = 6 \;\;\Rightarrow\;\; x=4
\]

\[
y+6 = -2 \;\;\Rightarrow\;\; y=-8
\]
Hence,
\[
(x,y)=(4,-8)
\]
\begin{figure}[H]
    \centering
    \includegraphics[width=0.5\linewidth]{figs/fig.png}
    \caption{Caption}
    \label{fig:placeholder}
\end{figure}
\end{document}

Midpoint of $A(4, -8)$ and $B(2, 6)$ is  
\[
\left( \frac{4+2}{2}, \; \frac{-8+6}{2} \right) = (3, -1)

\]

\begin{figure} [H]
    \centering
    \includegraphics[width=0.8\columnwidth]{figs/fig.png}
    
    \label{fig:placeholder}
\end{figure}
\end{document}
