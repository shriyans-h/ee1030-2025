 \let\negmedspace\undefined
\let\negthickspace\undefined
\documentclass[journal]{IEEEtran}
\usepackage[a5paper, margin=10mm, onecolumn]{geometry}
%\usepackage{lmodern} % Ensure lmodern is loaded for pdflatex
\usepackage{tfrupee} % Include tfrupee package

\setlength{\headheight}{1cm} % Set the height of the header box
\setlength{\headsep}{0mm}     % Set the distance between the header box and the top of the text

\usepackage{gvv-book}
\usepackage{gvv}
\usepackage{cite}
\usepackage{amsmath,amssymb,amsfonts,amsthm}
\usepackage{algorithmic}
\usepackage{graphicx}
\usepackage{textcomp}
\usepackage{xcolor}
\usepackage{txfonts}
\usepackage{listings}
\usepackage{enumitem}
\usepackage{mathtools}
\usepackage{gensymb}
\usepackage{comment}
\usepackage[breaklinks=true]{hyperref}
\usepackage{tkz-euclide} 
\usepackage{listings}
% \usepackage{gvv}                                        
\def\inputGnumericTable{}                                 
\usepackage[latin1]{inputenc}                                
\usepackage{color}                                            
\usepackage{array}                                            
\usepackage{longtable}                                       
\usepackage{calc}                                             
\usepackage{multirow}                                         
\usepackage{hhline}                                           
\usepackage{ifthen}                                           
\usepackage{lscape}
\begin{document}

\bibliographystyle{IEEEtran}
\vspace{3cm}


\title{1.5.16}
\author{EE25btech11028 - J.Navya sri}
% \maketitle
% \newpage
% \bigskip
{\let\newpage\relax\maketitle}

\renewcommand{\thefigure}{\theenumi}
\renewcommand{\thetable}{\theenumi}
\setlength{\intextsep}{10pt} % Space between text and floats


% Define custom vector macro

\textbf{Question:} Find the point \(A\) if \(AB\) is a diameter of the circle with center \(C=(3,-1)\) and point \(B=(2,6)\).

\bigskip

\textbf{Solution:}

\bigskip

\begin{tabular}{ |c|c| } 
    \hline
    Point & Vector \\ 
    \hline
    B & $ \myvec{2 \\ 6} $  \\
    \hline
    C & $ \myvec{3 \\ -1} $   \\
    \hline  
\end{tabular}

\bigskip

\textbf{Section Formula:} \\
If a point \(P\) divides the line joining \(A\) and \(B\) internally in the ratio \(m:n\), then
\begin{equation}
\vec{P} = \frac{k \vec{B} + \vec{A}}{k+1}
= \begin{pmatrix} \vec{A} & \vec{B} \end{pmatrix}
\myvec{\frac{1}{k+1} \\[6pt] \frac{k}{k+1}}.
\tag{1}
\end{equation}

Here, \(C\) is the midpoint of \(AB\), i.e. ratio \(1:1\).
\begin{equation}
\vec{C} = \frac{\vec{A} + \vec{B}}{2}
= \begin{pmatrix} \vec{A} & \vec{B} \end{pmatrix}
\myvec{\frac{1}{2} \\[6pt] \frac{1}{2}}.
\tag{2}
\end{equation}

\bigskip

Express \(\vec{A}\) in terms of \(\vec{B}\) and \(\vec{C}\):
\begin{equation}
\vec{C} = \frac{\vec{A} + \vec{B}}{2}
\quad \Rightarrow \quad
2 \vec{C} = \vec{A} + \vec{B}
\quad \Rightarrow \quad
\vec{A} = 2 \vec{C} - \vec{B}.
\tag{3}
\end{equation}

Using matrix notation:
\begin{equation}
\vec{A} = 2 \vec{C} - \vec{B} 
= \begin{pmatrix} \vec{B} & \vec{C} \end{pmatrix}
\myvec{-1 \\[6pt] 2}.
\tag{4}
\end{equation}

\bigskip

\textbf{Substitute values:}

Given:
\[
\vec{B} = \myvec{2 \\ 6}, \quad \vec{C} = \myvec{3 \\ -1}
\]
we have,
\begin{equation}
\vec{A} = 2 \myvec{3 \\ -1} - \myvec{2 \\ 6} 
= \myvec{6 \\ -2} - \myvec{2 \\ 6} 
= \myvec{4 \\ -8}.
\tag{5}
\end{equation}

\bigskip

\begin{equation}
\boxed{\vec{A} = \myvec{4 \\ -8}}
\tag{6}
\end{equation}




\begin{figure}[H]
    \centering
    \includegraphics[width=0.5\linewidth]{figs/fig.png}
    \caption{}
    \label{fig:placeholder}
\end{figure}
\end{document}


