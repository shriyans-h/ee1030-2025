\documentclass{beamer}
\usepackage[utf8]{inputenc}

\usetheme{Madrid}
\usecolortheme{default}
\usepackage{amsmath,amssymb,amsfonts,amsthm}
\usepackage{txfonts}
\usepackage{tkz-euclide}
\usepackage{listings}
\usepackage{adjustbox}
\usepackage{array}
\usepackage{tabularx}
\usepackage{gvv}
\usepackage{lmodern}
\usepackage{circuitikz}
\usepackage{tikz}
\usepackage{graphicx}

\setbeamertemplate{page number in head/foot}[totalframenumber]

\usepackage{tcolorbox}
\tcbuselibrary{minted,breakable,xparse,skins}



\definecolor{bg}{gray}{0.95}
\DeclareTCBListing{mintedbox}{O{}m!O{}}{%
  breakable=true,
  listing engine=minted,
  listing only,
  minted language=#2,
  minted style=default,
  minted options={%
    linenos,
    gobble=0,
    breaklines=true,
    breakafter=,,
    fontsize=\small,
    numbersep=8pt,
    #1},
  boxsep=0pt,
  left skip=0pt,
  right skip=0pt,
  left=25pt,
  right=0pt,
  top=3pt,
  bottom=3pt,
  arc=5pt,
  leftrule=0pt,
  rightrule=0pt,
  bottomrule=2pt,

  colback=bg,
  colframe=orange!70,
  enhanced,
  overlay={%
    \begin{tcbclipinterior}
    \fill[orange!20!white] (frame.south west) rectangle ([xshift=20pt]frame.north west);
    \end{tcbclipinterior}},
  #3,
}
\lstset{
    language=C,
    basicstyle=\ttfamily\small,
    keywordstyle=\color{blue},
    stringstyle=\color{orange},
    commentstyle=\color{green!60!black},
    numbers=left,
    numberstyle=\tiny\color{gray},
    breaklines=true,
    showstringspaces=false,
}
%------------------------------------------------------------
%This block of code defines the information to appear in the
%Title page
\title %optional
{1.5.16}
\date{August  2025}
\author{EE25btech11028 - J.Navya sri}
% \maketitle
% \newpage
% \bigskip
{\let\newpage\relax\maketitle}

\renewcommand{\thefigure}{\theenumi}
\renewcommand{\thetable}{\theenumi}
\setlength{\intextsep}{10pt} % Space between text and floats


\numberwithin{equation}{enumi}
\numberwithin{figure}{enumi}
\renewcommand{\thetable}{\theenumi}


\begin{document}

\textbf{Question:} \\
Find the point $A$ where $AB$ is a diameter of a circle with center $(3,-1)$ and the point $B$ is $(2,6)$

\bigskip

\begin{tabular}[12pt]{ |c| c| c|} 
    \hline
    {Point} & {Vector} \\ 
    \hline
    B & $ \myvec{2 \\ 6} $  \\
    \hline
    C & $ \myvec{3 \\ -1} $   \\
    \hline  
\end{tabular}

\bigskip

\textbf{Using rank of matrix:}
\bigskip
Three points \( \vec{A}, \vec{C}, \vec{B} \) are collinear if
\[
\rank{\begin{pmatrix} \vec{C} - \vec{A} & \vec{B} - \vec{A} \end{pmatrix}} = 1
\]

\[
\vec{C} - \vec{A} = \myvec{3 - x\\ -1 - y}, \quad
\vec{B} - \vec{A} = \myvec{2 - x\\ 6 - y}
\]

Form matrix:
\[
\begin{pmatrix}
3 - x & 2 - x \\
-1 - y & 6 - y
\end{pmatrix}
\]

Apply row operation:
\[
R_2 \rightarrow (3 - x)R_2 - (-1 - y)R_1
\]

\[
(3 - x)(6 - y) + (1 + y)(2 - x) = 0
\Rightarrow 20 - y - 7x = 0 \Rightarrow \boxed{7x + y = 20}
\]

\bigskip

\textbf{Using midpoint formula:}
\[
\vec{C} = \frac{\vec{A} + \vec{B}}{2} \Rightarrow \vec{A} = 2\vec{C} - \vec{B}
\]

\[
\vec{A} = 2\myvec{3\\-1} - \myvec{2\\6} = \myvec{6\\-2} - \myvec{2\\6} = \boxed{\vec{A} = \myvec{4\\-8}}
\]

\begin{figure}[H]
    \centering
    \includegraphics[width=0.5\linewidth]{figs/fig.png}
    \caption{}
    \label{fig:placeholder}
\end{figure}
\end{document}

