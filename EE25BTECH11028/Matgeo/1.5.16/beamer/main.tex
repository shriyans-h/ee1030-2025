\documentclass{beamer}
\usepackage[utf8]{inputenc}

\usetheme{Madrid}
\usecolortheme{default}
\usepackage{amsmath,amssymb,amsfonts,amsthm}
\usepackage{txfonts}
\usepackage{tkz-euclide}
\usepackage{listings}
\usepackage{adjustbox}
\usepackage{array}
\usepackage{tabularx}
\usepackage{gvv}
\usepackage{lmodern}
\usepackage{circuitikz}
\usepackage{tikz}
\usepackage{graphicx}

\setbeamertemplate{page number in head/foot}[totalframenumber]

\usepackage{tcolorbox}
\tcbuselibrary{minted,breakable,xparse,skins}



\definecolor{bg}{gray}{0.95}
\DeclareTCBListing{mintedbox}{O{}m!O{}}{%
  breakable=true,
  listing engine=minted,
  listing only,
  minted language=#2,
  minted style=default,
  minted options={%
    linenos,
    gobble=0,
    breaklines=true,
    breakafter=,,
    fontsize=\small,
    numbersep=8pt,
    #1},
  boxsep=0pt,
  left skip=0pt,
  right skip=0pt,
  left=25pt,
  right=0pt,
  top=3pt,
  bottom=3pt,
  arc=5pt,
  leftrule=0pt,
  rightrule=0pt,
  bottomrule=2pt,

  colback=bg,
  colframe=orange!70,
  enhanced,
  overlay={%
    \begin{tcbclipinterior}
    \fill[orange!20!white] (frame.south west) rectangle ([xshift=20pt]frame.north west);
    \end{tcbclipinterior}},
  #3,
}
\lstset{
    language=C,
    basicstyle=\ttfamily\small,
    keywordstyle=\color{blue},
    stringstyle=\color{orange},
    commentstyle=\color{green!60!black},
    numbers=left,
    numberstyle=\tiny\color{gray},
    breaklines=true,
    showstringspaces=false,
}
%------------------------------------------------------------
%This block of code defines the information to appear in the
%Title page
\title %optional
{1.5.16}
\date{August  2025}
%\subtitle{A short story}

\author % (optional)
{J.NAVYASRI- EE25BTECH11028}



\begin{document}


\frame{\titlepage}
\begin{frame}{Question}

 \textbf{Question:} Find the point \(A\) if \(AB\) is a diameter of the circle with center \(C=(3,-1)\) and point \(B=(2,6)\).
 
\end{frame}
 
\begin{frame}{given data}
 \[
\begin{tabular}[12pt]{ |c| c| c|} 
    \hline
    {Point} & {Vector} \\ 
    \hline
    B & $ \myvec{2 \\ 6} $  \\
    \hline
    C & $ \myvec{3 \\ -1} $   \\
    \hline  
    \end{tabular}
\]

   
\end{frame}

\begin{frame}{Section Formula}
If a point \(P\) divides the line joining \(A\) and \(B\) internally in the ratio \(m:n\), then
\[
\vec P = \frac{k\vec B + \vec A}{k+1}
= \begin{pmatrix}\vec A & \vec B\end{pmatrix}
\myvec{\tfrac{1}{k+1}\\[6pt]\tfrac{k}{k+1}}
\]

\begin{align*}
  Here, \(C\) is the midpoint of \(AB\), i.e.\ ratio \(1:1\).
\[
\vec{C} = \frac{\vec{A}+\vec{B}}{2}
= \begin{pmatrix}\vec A & \vec B\end{pmatrix}
\myvec{\tfrac{1}{2}\\[6pt]\tfrac{1}{2}}.
\]
 \end{align*}
\end{frame}
 


 \begin{frame}{substitute values:}
\[
\myvec{3\\-1} =
\begin{pmatrix}\vec A & \myvec{2\\6}\end{pmatrix}
\myvec{\tfrac{1}{2}\\[6pt]\tfrac{1}{2}}.
\]

\[
2\myvec{3\\-1}=\vec A+\myvec{2\\6}
\quad\Rightarrow\quad
\vec A=2\myvec{3\\-1}-\myvec{2\\6}=\myvec{4\\-8}.
\]

\end{frame}

\begin{frame}{Rank Verification}
   Check collinearity of \(A,B,C\):
\[
\begin{pmatrix}\vec C-\vec A & \vec B-\vec A\end{pmatrix}
=\begin{pmatrix}3-4 & 2-4\\ -1-(-8) & 6-(-8)\end{pmatrix}
=\begin{pmatrix}-1 & -2\\ 7 & 14\end{pmatrix} = 0
\]


Thus, rank \(=1\) and points are collinear.

\[
\boxed{\vec A=\myvec{4\\-8}}
\]
 
\end{frame}

\begin{frame}{Figure}
  
\begin{figure}[H]
    \centering
    \includegraphics[width=0.5\linewidth]{figs/fig.png}
    \caption{}
    \label{fig:placeholder}
\end{figure}
  
\end{frame}

\begin{frame}[fragile]
    \frametitle{Python Code}
    \begin{lstlisting}
    import matplotlib.pyplot as plt

# Center C = (3, -1)
# B = (2, 6)
# Let A = (x, y). Midpoint formula: center = (A + B) / 2 =>
# 3 = (x + 2) / 2,  -1 = (y + 6) / 2
# Solve for (x, y):
# x = 2*3 - 2 = 4
# y = 2*(-1) - 6 = -8

A = np.array([4, -8])
B = np.array([2, 6])
C = np.array([3, -1])
\end{lstlisting}
\end{frame}

\begin{frame}[fragile]
    \frametitle{Python Code}

    \begin{lstlisting}
# For the circle, radius = distance(center, B)
import numpy as np
def dist(P, Q):
    return np.sqrt((P[0] - Q[0])**2 + (P[1] - Q[1])**2)
radius = dist(C, B)

fig, ax = plt.subplots(figsize=(7,7))

# Plot the circle
circle = plt.Circle(C, radius, color='blue', fill=False, linestyle='dotted', label='Circle')
ax.add_patch(circle)

# Plot A, B, C
ax.scatter(*A, color='red', label='A (unknown, solved)')
ax.scatter(*B, color='green', label='B (2, 6)')
ax.scatter(*C, color='orange', label='Center (3, -1)')



    \end{lstlisting}
\end{frame}

\begin{frame}[fragile]
    \frametitle{Python Code}

    \begin{lstlisting}
 # Plot line AC
ax.plot([A[0], C[0]], [A[1], C[1]], [A[2], C[2]], color='purple', label='Line AC')

# Annotate points
ax.text(*A, ' A', color='red', fontsize=10)
ax.text(*B, ' B', color='green', fontsize=10)
ax.text(*C, ' C', color='blue', fontsize=10)


    \end{lstlisting}
\end{frame}

\begin{frame}[fragile]
    \frametitle{Python Code}

    \begin{lstlisting}
 # Draw diameter AB
ax.plot([A[0], B[0]], [A[1], B[1]], color='purple', linewidth=2, linestyle='--', label='Diameter AB')

# Annotate
ax.annotate('A'+str(A), (A[0], A[1]), xytext=(10, -10), textcoords='offset points')
ax.annotate('B'+str(B), (B[0], B[1]), xytext=(-40, 10), textcoords='offset points')
ax.annotate('C'+str(C), (C[0], C[1]), xytext=(5, -10), textcoords='offset points')

ax.set_xlim(C[0] - radius - 2, C[0] + radius + 2)
ax.set_ylim(C[1] - radius - 2, C[1] + radius + 2)
ax.set_aspect('equal')
ax.grid(True)
plt.legend()
plt.title('Circle with Diameter AB')
plt.xlabel('x')
plt.ylabel('y')
plt.show()


\end{lstlisting}
\end{frame}

 



 


\begin{frame}[fragile]
\frametitle{C Code}
\begin{lstlisting}
  
#include <stdio.h>

int main() {
    // Given values
    int xB = 2, yB = 6;
    int xC = 3, yC = -1;  // Center of the circle

    // Calculate coordinates of A using midpoint formula
    int xA = 2 * xC - xB;
    int yA = 2 * yC - yB;

    // Print result
    printf("Coordinates of point A are: (%d, %d)\n", xA, yA);

    // Verify midpoint
    float midX = (xA + xB) / 2.0;
    float midY = (yA + yB) / 2.0;
    printf("Midpoint of A and B is: (%.1f, %.1f)\n", midX, midY);

    return 0;
}






\end{lstlisting}

\end{frame}


\begin{frame}[fragile]
\frametitle{Python and C Code}

\begin{lstlisting}
 import subprocess

# Compile the C program
subprocess.run(["gcc", "midpoint.c", "-o", "midpoint"])

# Run the compiled C program
result = subprocess.run(["./midpoint"], capture_output=True, text=True)

# Print the output from the C program 
print(result.stdout)
\end{lstlisting}

\end{frame}

\begin{figure}[H]
    \centering
    \includegraphics[width=0.5\linewidth]{beamer/fig.png}
    \caption{}
    \label{fig:placeholder}
\end{figure} 



\end{document}