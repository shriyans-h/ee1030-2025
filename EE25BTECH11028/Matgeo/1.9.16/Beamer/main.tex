\documentclass{beamer}
\usepackage[utf8]{inputenc}

\usetheme{Madrid}
\usecolortheme{default}
\usepackage{amsmath,amssymb,amsfonts,amsthm}
\usepackage{txfonts}
\usepackage{tkz-euclide}
\usepackage{listings}
\usepackage{adjustbox}
\usepackage{array}
\usepackage{tabularx}
\usepackage{gvv}
\usepackage{lmodern}
\usepackage{circuitikz}
\usepackage{tikz}
\usepackage{graphicx}

\setbeamertemplate{page number in head/foot}[totalframenumber]

\usepackage{tcolorbox}
\tcbuselibrary{minted,breakable,xparse,skins}



\definecolor{bg}{gray}{0.95}
\DeclareTCBListing{mintedbox}{O{}m!O{}}{%
  breakable=true,
  listing engine=minted,
  listing only,
  minted language=#2,
  minted style=default,
  minted options={%
    linenos,
    gobble=0,
    breaklines=true,
    breakafter=,,
    fontsize=\small,
    numbersep=8pt,
    #1},
  boxsep=0pt,
  left skip=0pt,
  right skip=0pt,
  left=25pt,
  right=0pt,
  top=3pt,
  bottom=3pt,
  arc=5pt,
  leftrule=0pt,
  rightrule=0pt,
  bottomrule=2pt,

  colback=bg,
  colframe=orange!70,
  enhanced,
  overlay={%
    \begin{tcbclipinterior}
    \fill[orange!20!white] (frame.south west) rectangle ([xshift=20pt]frame.north west);
    \end{tcbclipinterior}},
  #3,
}
\lstset{
    language=C,
    basicstyle=\ttfamily\small,
    keywordstyle=\color{blue},
    stringstyle=\color{orange},
    commentstyle=\color{green!60!black},
    numbers=left,
    numberstyle=\tiny\color{gray},
    breaklines=true,
    showstringspaces=false,
}
%------------------------------------------------------------
%This block of code defines the information to appear in the
%Title page
\title %optional
{1.5.16}
\date{August  2025}
%\subtitle{A short story}

\author % (optional)
{J.NAVYASRI- EE25BTECH11028}

\begin{document}

\frame{\titlepage}
\begin{frame}{Question}
\textbf{Question:} \\
Write the coordinates of a point \(\mathbf{P}\) on the \(x\)-axis which is equidistant from the points \(\mathbf{A(-2, 0)}\) and \(\mathbf{B(6, 0)}\).
\end{frame}

% Step 1: Theoretical solution
\begin{frame}{Theoretical solution}
Let the point \(P\) lie on the \(x\)-axis with coordinates
\begin{equation}
    \mathbf{P} = (x, 0)
\end{equation}

The position vectors of points \(A\), \(B\), and \(P\) are
\begin{align}
    \vec{A} &= \langle -2, 0 \rangle \\
    \vec{B} &= \langle 6, 0 \rangle \\
    \vec{P} &= \langle x, 0 \rangle
\end{align}

Since \(P\) is equidistant from \(A\) and \(B\), their distances are equal:
\begin{equation}
    |\vec{P} - \vec{A}| = |\vec{P} - \vec{B}|
\end{equation}

\end{frame}

% Step 2: Theoretical solution 
\begin{frame}{Theoretical solution}
Using vector subtraction:
\begin{align}
    \vec{P} - \vec{A} &= \langle x - (-2), 0 - 0 \rangle = \langle x + 2, 0 \rangle \\
    \vec{P} - \vec{B} &= \langle x - 6, 0 - 0 \rangle = \langle x - 6, 0 \rangle
\end{align}

Now equate the magnitudes:
\begin{equation}
    \sqrt{(x+2)^2 + 0^2} = \sqrt{(x-6)^2 + 0^2}
\end{equation}

Simplifying, we get:
\begin{equation}
    |x + 2| = |x - 6|
\end{equation}
\end{frame}

% Step 3: Solve Cases
\begin{frame}{Solving Cases}
Consider the two cases:

\textbf{Case 1:}
\begin{equation}
    x + 2 = x - 6 \implies 2 = -6 \quad \text{(not possible)}
\end{equation}

\textbf{Case 2:}
\begin{equation}
    x + 2 = -(x - 6) \implies x + 2 = -x + 6 \implies 2x = 4 \implies x = 2
\end{equation}
\end{frame}

% Final Answer
\begin{frame}{Answer}
Therefore, the coordinates of point \(P\) are
\begin{equation}
    \boxed{(2, 0)}
\end{equation}

\end{frame}

\begin{frame}[fragile]
    \frametitle{Python Code}
    \begin{lstlisting}
import matplotlib.pyplot as plt

def setup_plot():
    # Points A, P, and B
    A = (-2, 0)
    P = (2, 0)
    B = (6, 0)

    fig, ax = plt.subplots(figsize=(8, 3))

    # Set axis limits and aspect ratio
    ax.set_xlim(-4, 8)
    ax.set_ylim(-1, 2)
    ax.set_aspect('equal')

    # Remove ticks and spines except left and bottom
    ax.set_xticks([])
    ax.set_yticks([])
    for spine in ['top', 'right']:
        ax.spines[spine].set_visible(False)

    return fig, ax, A, P, B

\end{lstlisting}
\end{frame}

\begin{frame}[fragile]
    \frametitle{Python Code}
    \begin{lstlisting}
def draw_elements(ax, A, P, B):
    # Plot points
    ax.plot(A[0], A[1], 'ro')  # red
    ax.plot(P[0], P[1], 'go')  # green
    ax.plot(B[0], B[1], 'bo')  # blue

    # Labels below points
    ax.text(A[0], A[1] - 0.25, r'$A(-2,0)$', color='red', ha='center', fontsize=12)
    ax.text(P[0], P[1] - 0.25, r'$P(2,0)$', color='green', ha='center', fontsize=12)
    ax.text(B[0], B[1] - 0.25, r'$B(6,0)$', color='blue', ha='center', fontsize=12)

    # Dashed line between A and B
    ax.plot([A[0], B[0]], [A[1], B[1]], 'k--', linewidth=1)

    # Annotation above dashed line
    ax.text(2, 0.3, r'$P$ is equidistant from $A$ and $B$', fontsize=12, style='italic', ha='center')

    # Draw x and y axis arrows
    ax.arrow(-4, 0, 12, 0, head_width=0.15, head_length=0.3, fc='black', ec='black', length_includes_head=True)
    ax.arrow(0, -1, 0, 2, head_width=0.15, head_length=0.3, fc='black', ec='black', length_includes_head=True)

    # Axis labels
    ax.text(7.9, -0.15, 'x', fontsize=12)
    ax.text(-0.2, 1.8, 'y', fontsize=12)

\end{lstlisting}
\end{frame}

\begin{frame}[fragile]
    \frametitle{Python Code}
      \begin{lstlisting}
def add_caption_and_show(fig):
    fig.text(0.5, 0.02, 'Fig. 0', ha='center', fontsize=14, weight='bold')
    plt.show()

if __name__ == "__main__":
    fig, ax, A, P, B = setup_plot()
    draw_elements(ax, A, P, B)
    add_caption_and_show(fig)
\end{lstlisting}
\end{frame}

\begin{frame}[fragile]
\frametitle{C Code}
\begin{lstlisting}
// Part 1: Problem Setup and Case 1 Checking
#include <stdio.h>
#include <math.h>
#include <stdlib.h>

int main() {
    // Coordinates of points A and B
    int Ax = -2, Ay = 0;
    int Bx = 6, By = 0;

    // Point P lies on the x-axis, so P = (x, 0)
    int x;

    // Case 1: x + 2 = x - 6 --> 2 = -6 (not possible)
    int lhs1 = 0 + 2;  // placeholder for demonstration
    int rhs1 = 0 - 6;

    if (lhs1 == rhs1) {
        printf("Case 1 valid, unexpected result\n");
    } else {
        printf("Case 1: %d = %d --> Not possible\n", lhs1, rhs1);
    }


\end{lstlisting}

\end{frame}

\begin{frame}[fragile]
\frametitle{C Code}
\begin{lstlisting}
    // Case 2: x + 2 = -(x - 6) => x + 2 = -x + 6
    // => 2x = 4 => x = 2
    x = 2;

    // Output result
    printf("The point P that is equidistant from A and B on the x-axis is: (%d, 0)\n", x);

    // Optional: verify the distances
    double distance_PA = sqrt(pow(x - Ax, 2) + pow(0 - Ay, 2));
    double distance_PB = sqrt(pow(x - Bx, 2) + pow(0 - By, 2));

    printf("Distance PA = %.2f\n", distance_PA);
    printf("Distance PB = %.2f\n", distance_PB);

    return 0;
}

\end{lstlisting}

\end{frame}

\begin{frame}[fragile]
\frametitle{Python and C Code}

\begin{lstlisting}
# Compile the C program
subprocess.run(["gcc", "equidiistance.c", "-o", "equidistance"])

# Run the compiled C program
result = subprocess.run(["./equidistance"], capture_output=True, text=True)

# Print the output from the C program 
print(result.stdout)
\end{lstlisting}

\end{frame}

\begin{frame}{Graphical Representation:}

\begin{center}
    \begin{tikzpicture}[scale=1.0]
        % axes
        \draw[->] (-4,0) -- (8,0) node[right] {\(x\)};
        \draw[->] (0,-1) -- (0,2) node[above] {\(y\)};
        
        % points
        \filldraw[red] (-2,0) circle (2pt) node[below] {\(A(-2,0)\)};
        \filldraw[blue] (6,0) circle (2pt) node[below] {\(B(6,0)\)};
        \filldraw[green!70!black] (2,0) circle (2pt) node[below] {\(P(2,0)\)};
        
        % dotted line
        \draw[dashed] (-2,0) -- (6,0);
        
        % annotation
        \node at (4,1.0) {\(P\) is equidistant from \(A\) and \(B\)};
    \end{tikzpicture}
    
    % Add "Fig. 0" text below the figure
    \vspace{0.5cm} % space between figure and text
    \textbf{Fig. 0}
\end{center}
\end{frame}
\end{document}

