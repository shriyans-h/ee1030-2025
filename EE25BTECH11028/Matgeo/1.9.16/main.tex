\let\negmedspace\undefined
\let\negthickspace\undefined
\documentclass[journal]{IEEEtran}
\usepackage[a5paper, margin=10mm, onecolumn]{geometry}
%\usepackage{lmodern} % Ensure lmodern is loaded for pdflatex
\usepackage{tfrupee} % Include tfrupee package

\setlength{\headheight}{1cm} % Set the height of the header box
\setlength{\headsep}{0mm}     % Set the distance between the header box and the top of the text

\usepackage{gvv-book}
\usepackage{gvv}
\usepackage{cite}
\usepackage{amsmath,amssymb,amsfonts,amsthm}
\usepackage{algorithmic}
\usepackage{graphicx}
\usepackage{textcomp}
\usepackage{xcolor}
\usepackage{txfonts}
\usepackage{listings}
\usepackage{enumitem}
\usepackage{mathtools}
\usepackage{gensymb}
\usepackage{comment}
\usepackage[breaklinks=true]{hyperref}
\usepackage{tkz-euclide} 
\usepackage{listings}
% \usepackage{gvv}                                        
\def\inputGnumericTable{}                                 
\usepackage[latin1]{inputenc}                                
\usepackage{color}                                            
\usepackage{array}                                            
\usepackage{longtable}                                       
\usepackage{calc}                                             
\usepackage{multirow}                                         
\usepackage{hhline}                                           
\usepackage{ifthen}                                           
\usepackage{lscape}
\begin{document}

\bibliographystyle{IEEEtran}
\vspace{3cm}

\title{1.9.17}
\author{EE25btech11028 - J.Navya sri}
% \maketitle
% \newpage
% \bigskip
{\let\newpage\relax\maketitle}



\textbf{Question:} \\
Write the coordinates of a point \(\mathbf{P}\) on the \(x\)-axis which is equidistant from points \(\mathbf{A}(-2,0)\) and \(\mathbf{B}(6,0)\).

\vspace{0.5cm}
\textbf{Solution:} Let
\begin{equation}
\mathbf{A} = \begin{pmatrix} a \\ 0 \end{pmatrix}, \quad
\mathbf{B} = \begin{pmatrix} b \\ 0 \end{pmatrix}, \quad
\mathbf{P} = \begin{pmatrix} p \\ 0 \end{pmatrix}
\end{equation}

Since \(\mathbf{P}\) is equidistant from \(\mathbf{A}\) and \(\mathbf{B}\), their distances satisfy:
\begin{equation}
\|\mathbf{P} - \mathbf{A}\| = \|\mathbf{P} - \mathbf{B}\|
\end{equation}

Square both sides:
\begin{equation}
\|\mathbf{P} - \mathbf{A}\|^2 = \|\mathbf{P} - \mathbf{B}\|^2
\end{equation}

Using the norm squared definition:
\begin{equation}
(\mathbf{P} - \mathbf{A})^\top (\mathbf{P} - \mathbf{A}) = (\mathbf{P} - \mathbf{B})^\top (\mathbf{P} - \mathbf{B})
\end{equation}

Expand both sides:
\begin{equation}
\mathbf{P}^\top \mathbf{P} - 2 \mathbf{A}^\top \mathbf{P} + \mathbf{A}^\top \mathbf{A} = \mathbf{P}^\top \mathbf{P} - 2 \mathbf{B}^\top \mathbf{P} + \mathbf{B}^\top \mathbf{B}
\end{equation}

Cancel \(\mathbf{P}^\top \mathbf{P}\) from both sides:
\begin{equation}
- 2 \mathbf{A}^\top \mathbf{P} + \mathbf{A}^\top \mathbf{A} = - 2 \mathbf{B}^\top \mathbf{P} + \mathbf{B}^\top \mathbf{B}
\end{equation}

Rearranged:
\begin{equation}
2 (\mathbf{B} - \mathbf{A})^\top \mathbf{P} = \mathbf{B}^\top \mathbf{B} - \mathbf{A}^\top \mathbf{A}
\end{equation}

Substitute the vectors:
\begin{equation}
2 (b - a) p = b^2 - a^2
\end{equation}

Rewrite right side as difference of squares:
\begin{equation}
2 (b - a) p = (b - a)(b + a)
\end{equation}

Since \(b \neq a\), divide both sides by \((b - a)\):
\begin{equation}
2 p = b + a
\end{equation}

Solve for \(p\):
\begin{equation}
p = \frac{a + b}{2}
\end{equation}

Now substitute \(a = -2\), \(b = 6\):
\begin{equation}
p = \frac{-2 + 6}{2} = \frac{4}{2} = 2
\end{equation}

Hence, the coordinates of \(\mathbf{P}\) are:
\begin{equation}
\boxed{
\mathbf{P} = \begin{pmatrix} 2 \\ 0 \end{pmatrix}
}
\end{equation}


\vspace{0.5cm}

\documentclass{}
\usepackage{}


\textbf{Graphical Representation:}

\begin{center}
    \begin{tikzpicture}[scale=1.0]
        % axes
        \draw[->] (-4,0) -- (8,0) node[right] {\(x\)};
        \draw[->] (0,-1) -- (0,2) node[above] {\(y\)};
        
        % points
        \filldraw[red] (-2,0) circle (2pt) node[below] {\(A(-2,0)\)};
        \filldraw[blue] (6,0) circle (2pt) node[below] {\(B(6,0)\)};
        \filldraw[green!70!black] (2,0) circle (2pt) node[below] {\(P(2,0)\)};
        
        % dotted line
        \draw[dashed] (-2,0) -- (6,0);
        
        % annotation
        \node at (4,1.0) {\(P\) is equidistant from \(A\) and \(B\)};
    \end{tikzpicture}
    
    % Add "Fig. 0" text below the figure
    \vspace{0.5cm} % space between figure and text
    \textbf{Fig. 0}
\end{center}

\end{document}


