\let\negmedspace\undefined
\let\negthickspace\undefined
\documentclass[journal]{IEEEtran}
\usepackage[a5paper, margin=10mm, onecolumn]{geometry}
%\usepackage{lmodern} % Ensure lmodern is loaded for pdflatex
\usepackage{tfrupee} % Include tfrupee package

\setlength{\headheight}{1cm} % Set the height of the header box
\setlength{\headsep}{0mm}     % Set the distance between the header box and the top of the text

\usepackage{gvv-book}
\usepackage{gvv}
\usepackage{cite}
\usepackage{amsmath,amssymb,amsfonts,amsthm}
\usepackage{algorithmic}
\usepackage{graphicx}
\usepackage{textcomp}
\usepackage{xcolor}
\usepackage{txfonts}
\usepackage{listings}
\usepackage{enumitem}
\usepackage{mathtools}
\usepackage{gensymb}
\usepackage{comment}
\usepackage[breaklinks=true]{hyperref}
\usepackage{tkz-euclide} 
\usepackage{listings}
% \usepackage{gvv}                                        
\def\inputGnumericTable{}                                 
\usepackage[latin1]{inputenc}                                
\usepackage{color}                                            
\usepackage{array}                                            
\usepackage{longtable}                                       
\usepackage{calc}                                             
\usepackage{multirow}                                         
\usepackage{hhline}                                           
\usepackage{ifthen}                                           
\usepackage{lscape}
\begin{document}

\bibliographystyle{IEEEtran}
\vspace{3cm}

\title{1.9.17}
\author{EE25btech11028 - J.Navya sri}
% \maketitle
% \newpage
% \bigskip
{\let\newpage\relax\maketitle}



\textbf{Question:} \\
Write the coordinates of a point \(\mathbf{P}\) on the \(x\)-axis which is equidistant from the points 
\(A(-2, 0)\) and \(B(6, 0)\).

\vspace{0.5cm}

\textbf{Solution:}

Let the point \(P\) lie on the \(x\)-axis with coordinates
\begin{equation}
    \mathbf{P} = (x, 0)
\end{equation}

The position vectors of points \(A\), \(B\), and \(P\) are
\begin{align}
    \vec{A} &= \langle -2, 0 \rangle \\
    \vec{B} &= \langle 6, 0 \rangle \\
    \vec{P} &= \langle x, 0 \rangle
\end{align}

Since \(P\) is equidistant from \(A\) and \(B\), their distances are equal:
\begin{equation}
    |\vec{P} - \vec{A}| = |\vec{P} - \vec{B}|
\end{equation}

Using vector subtraction:
\begin{align}
    \vec{P} - \vec{A} &= \langle x - (-2), 0 - 0 \rangle = \langle x + 2, 0 \rangle \\
    \vec{P} - \vec{B} &= \langle x - 6, 0 - 0 \rangle = \langle x - 6, 0 \rangle
\end{align}

Now equate the magnitudes:
\begin{equation}
    \sqrt{(x+2)^2 + 0^2} = \sqrt{(x-6)^2 + 0^2}
\end{equation}

Simplifying, we get:
\begin{equation}
    |x + 2| = |x - 6|
\end{equation}

Consider the two cases:

\textbf{Case 1:}
\begin{equation}
    x + 2 = x - 6 \implies 2 = -6 \quad \text{(not possible)}
\end{equation}

\textbf{Case 2:}
\begin{equation}
    x + 2 = -(x - 6) \implies x + 2 = -x + 6 \implies 2x = 4 \implies x = 2
\end{equation}

Therefore, the coordinates of point \(P\) are
\begin{equation}
    \boxed{(2, 0)}
\end{equation}

\vspace{0.5cm}

\documentclass{}
\usepackage{}


\textbf{Graphical Representation:}

\begin{center}
    \begin{tikzpicture}[scale=1.0]
        % axes
        \draw[->] (-4,0) -- (8,0) node[right] {\(x\)};
        \draw[->] (0,-1) -- (0,2) node[above] {\(y\)};
        
        % points
        \filldraw[red] (-2,0) circle (2pt) node[below] {\(A(-2,0)\)};
        \filldraw[blue] (6,0) circle (2pt) node[below] {\(B(6,0)\)};
        \filldraw[green!70!black] (2,0) circle (2pt) node[below] {\(P(2,0)\)};
        
        % dotted line
        \draw[dashed] (-2,0) -- (6,0);
        
        % annotation
        \node at (4,1.0) {\(P\) is equidistant from \(A\) and \(B\)};
    \end{tikzpicture}
    
    % Add "Fig. 0" text below the figure
    \vspace{0.5cm} % space between figure and text
    \textbf{Fig. 0}
\end{center}

\end{document}


