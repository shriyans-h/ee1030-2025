\let\negmedspace\undefined
\let\negthickspace\undefined
\documentclass[journal]{IEEEtran}
\usepackage[a5paper, margin=10mm, onecolumn]{geometry}
%\usepackage{lmodern} % Ensure lmodern is loaded for pdflatex
\usepackage{tfrupee} % Include tfrupee package

\setlength{\headheight}{1cm} % Set the height of the header box
\setlength{\headsep}{0mm}     % Set the distance between the header box and the top of the text

\usepackage{gvv-book}
\usepackage{gvv}
\usepackage{cite}
\usepackage{amsmath,amssymb,amsfonts,amsthm}
\usepackage{algorithmic}
\usepackage{graphicx}
\usepackage{textcomp}
\usepackage{xcolor}
\usepackage{txfonts}
\usepackage{listings}
\usepackage{enumitem}
\usepackage{mathtools}
\usepackage{gensymb}
\usepackage{comment}
\usepackage[breaklinks=true]{hyperref}
\usepackage{tkz-euclide} 
\usepackage{listings}
% \usepackage{gvv}                                        
\def\inputGnumericTable{}                                 
\usepackage[latin1]{inputenc}                                
\usepackage{color}                                            
\usepackage{array}                                            
\usepackage{longtable}                                       
\usepackage{calc}                                             
\usepackage{multirow}                                         
\usepackage{hhline}                                           
\usepackage{ifthen}                                           
\usepackage{lscape}
\begin{document}

\bibliographystyle{IEEEtran}
\vspace{3cm}

\title{1.9.17}
\author{EE25btech11028 - J.Navya sri}
% \maketitle
% \newpage
% \bigskip
{\let\newpage\relax\maketitle}

\renewcommand{\thefigure}{\theenumi}
\renewcommand{\thetable}{\theenumi}
\setlength{\intextsep}{10pt} % Space between text and floats


\bigskip
\textbf{Question:} \\
Write the coordinates of a point $\mathbf{P}$ on the $x$-axis which is equidistant from the points $\mathbf{A(-2, 0)}$ and $\mathbf{B(6, 0)}$.

\bigskip

\textbf{Solution:} \\

Let the point $P$ be on the $x$-axis with coordinates:
\begin{equation}
P(x, 0)
\end{equation}

Since $P$ is equidistant from $A$ and $B$, their distances from $P$ are equal:
\begin{equation}
PA = PB
\end{equation}

Using the distance formula:
\begin{equation}
\sqrt{(x + 2)^2 + (0 - 0)^2} = \sqrt{(x - 6)^2 + (0 - 0)^2}
\end{equation}

This simplifies to:
\begin{equation}
|x + 2| = |x - 6|
\end{equation}

Consider two cases:

\textbf{Case 1:}
\begin{equation}
x + 2 = x - 6 \quad \Rightarrow \quad 2 = -6 \quad \text{(Not possible)}
\end{equation}

\textbf{Case 2:}
\begin{equation}
x + 2 = -(x - 6) 
\end{equation}
\[ x + 2 = -x + 6 \] 
\[ 2x = 4 \quad \Rightarrow \quad x = 2 \]
Therefore, the coordinates of point $P$ are:
\begin{equation}
\boxed{(2, 0)}
\end{equation}

\textbf{Graphical Representation:}

\begin{center}
\begin{tikzpicture}[scale=0.8]
    % Draw axes
    \draw[->] (-4,0) -- (8,0) node[right] {$x$};
    \draw[->] (0,-1) -- (0,2) node[above] {$y$};

    % Plot points
    \filldraw[red] (-2,0) circle (3pt) node[below] {$A(-2,0)$};
    \filldraw[blue] (6,0) circle (3pt) node[below] {$B(6,0)$};
    \filldraw[green!70!black] (2,0) circle (3pt) node[below] {$P(2,0)$};

    % Draw dashed lines showing equal distance
    \draw[dashed, gray] (-2,0) -- (2,0);
    \draw[dashed, gray] (6,0) -- (2,0);

    % Add label
    
  \node at (2,1) {P is equidistant from A and B};
    
\end{tikzpicture}
\end{center}

\end{document}