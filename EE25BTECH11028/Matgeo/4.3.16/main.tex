\let\negmedspace\undefined
\let\negthickspace\undefined
\documentclass[journal]{IEEEtran}
\usepackage[a5paper, margin=10mm, onecolumn]{geometry}
%\usepackage{lmodern} % Ensure lmodern is loaded for pdflatex
\usepackage{tfrupee} % Include tfrupee package

\setlength{\headheight}{1cm} % Set the height of the header box
\setlength{\headsep}{0mm}     % Set the distance between the header box and the top of the text

\usepackage{gvv-book}
\usepackage{gvv}
\usepackage{cite}
\usepackage{amsmath,amssymb,amsfonts,amsthm}
\usepackage{algorithmic}
\usepackage{graphicx}
\usepackage{textcomp}
\usepackage{xcolor}
\usepackage{txfonts}
\usepackage{listings}
\usepackage{enumitem}
\usepackage{mathtools}
\usepackage{gensymb}
\usepackage{comment}
\usepackage[breaklinks=true]{hyperref}
\usepackage{tkz-euclide} 
\usepackage{listings}
% \usepackage{gvv}                                        
\def\inputGnumericTable{}                                 
\usepackage[latin1]{inputenc}                                
\usepackage{color}                                            
\usepackage{array}                                            
\usepackage{longtable}                                       
\usepackage{calc}                                             
\usepackage{multirow}                                         
\usepackage{hhline}                                           
\usepackage{ifthen}                                           
\usepackage{lscape}
\begin{document}

\bibliographystyle{IEEEtran}
\vspace{3cm}

\title{4.3.16}
\author{EE25btech11028 - J.Navya sri}
% \maketitle
% \newpage
% \bigskip
{\let\newpage\relax\maketitle}


\textbf{Question:} \\
Find the equation of the plane through the points  
\[
(2,1,0), \quad (3,-2,-2), \quad (3,1,7).
\]
\textbf{Solution:}
Given three points:
\[
P_1(2,1,0), \quad P_2(3,-2,-2), \quad P_3(3,1,7)
\]

The direction vectors are:
\begin{equation}
\myvec{v}_1 =\myvec{P_2 - P_1} = (3-2,\,-2-1,\,-2-0) = (1,-3,-2)
\end{equation}

\begin{equation}
\myvec{v}_2 =\myvec{P_3 - P_1 }= (3-2,\;1-1,\;7-0) = (1,0,7)
\end{equation}

The normal vector to the plane is given by the cross product:
\begin{equation}
\myvec{n} = \myvec{v}_1 \times \myvec{v}_2
\end{equation}

\begin{equation}
\myvec{n} = 
\begin{vmatrix}
\mathbf{i} & \mathbf{j} & \mathbf{k} \\
1 & -3 & -2 \\
1 & 0 & 7
\end{vmatrix}
\end{equation}

\begin{equation}
\myvec{n} = (-21,\,-9,\;3) = (-7,\,-3,\;1)
\end{equation}

Hence, the equation of the plane is:
\begin{equation}
-7(x-2) - 3(y-1) + (z-0) = 0
\end{equation}

Simplifying:
\begin{equation}
-7x + 14 - 3y + 3 + z = 0
\end{equation}

\begin{equation}
7x + 3y - z - 17 = 0
\end{equation}
\textbf{Final answer:}
\[
\boxed{7x + 3y - z - 17 = 0}
\]


\begin{figure}[H]
\begin{center}
\includegraphics[width=0.6\columnwidth]{figs/fig6.png}
\end{center}
\caption{}
\label{fig:Fig}
\end{figure}


\end{document}