\documentclass{beamer}
\usepackage[utf8]{inputenc}

\usetheme{Madrid}
\usecolortheme{default}
\usepackage{amsmath,amssymb,amsfonts,amsthm}
\usepackage{txfonts}
\usepackage{tkz-euclide}
\usepackage{listings}
\usepackage{adjustbox}
\usepackage{array}
\usepackage{tabularx}
\usepackage{gvv}
\usepackage{lmodern}
\usepackage{circuitikz}
\usepackage{tikz}
\usepackage{graphicx}

\setbeamertemplate{page number in head/foot}[totalframenumber]

\usepackage{tcolorbox}
\tcbuselibrary{minted,breakable,xparse,skins}



\definecolor{bg}{gray}{0.95}
\DeclareTCBListing{mintedbox}{O{}m!O{}}{%
  breakable=true,
  listing engine=minted,
  listing only,
  minted language=#2,
  minted style=default,
  minted options={%
    linenos,
    gobble=0,
    breaklines=true,
    breakafter=,,
    fontsize=\small,
    numbersep=8pt,
    #1},
  boxsep=0pt,
  left skip=0pt,
  right skip=0pt,
  left=25pt,
  right=0pt,
  top=3pt,
  bottom=3pt,
  arc=5pt,
  leftrule=0pt,
  rightrule=0pt,
  bottomrule=2pt,

  colback=bg,
  colframe=orange!70,
  enhanced,
  overlay={%
    \begin{tcbclipinterior}
    \fill[orange!20!white] (frame.south west) rectangle ([xshift=20pt]frame.north west);
    \end{tcbclipinterior}},
  #3,
}
\lstset{
    language=C,
    basicstyle=\ttfamily\small,
    keywordstyle=\color{blue},
    stringstyle=\color{orange},
    commentstyle=\color{green!60!black},
    numbers=left,
    numberstyle=\tiny\color{gray},
    breaklines=true,
    showstringspaces=false,
}
%------------------------------------------------------------
%This block of code defines the information to appear in the
%Title page
\title %optional
{1.5.16}
\date{August  2025}
%\subtitle{A short story}

\author % (optional)
{J.Navya sri - EE25BTECH11028}



\begin{document}


\frame{\titlepage}
\begin{frame}{Question}
Find the coordinates of a point $A$ where $AB$ is a diameter of the circle with center  
$(3, -1)$ and the point $B$ is $(2, 6)$.


\end{frame}
    
\begin{frame}{given data}
 
let C be the center of circle
\[
\begin{array}{|c|c|c|c|}
\hline
\textbf{Point} & \textbf{x} & \textbf{y}\\
\hline
B & 2 & 6 \\
C & 3 & -1 \\
\hline
\end{array}
\]

   
\end{frame}

\begin{frame}{Formula}
  Midpoint formula : If C is the Midpoint of AB . where A and B are the end points of diameter

\end{frame}
 
\[
\begin{array}{|c|c|c|c|}
\hline
\textbf{Point} & \textbf{x} & \textbf{y}\\
\hline
B & 2 & 6 \\
C & 3 & -1 \\
\hline
\end{array}
\]

\begin{enumerate}
 Circle center is the \textbf{midpoint} of diameter $AB$.  
    So, midpoint formula:  
    \[
    \left( \frac{x_A + x_B}{2}, \; \frac{y_A + y_B}{2} \right) = (3, -1)
    \]

 Solve for $x_A$:  
    \[
    \frac{x_A + 2}{2} = 3 \;\;\Rightarrow\;\; x_A + 2 = 6 \;\;\Rightarrow\;\; x_A = 6 - 2 = 4
    \]

    \ Solve for $y_A$:  
    \[
    \frac{y_A + 6}{2} = -1 \;\;\Rightarrow\;\; y_A + 6 = -2 \;\;\Rightarrow\;\; y_A = -2 - 6 = -8
    \]
Hence,  
    \[
    A = (4, -8)
    \]
\end{enumerate}

Midpoint of $A(4, -8)$ and $B(2, 6)$ is  
\[
\left( \frac{4+2}{2}, \; \frac{-8+6}{2} \right) = (3, -1)

\]

 







\begin{frame}[fragile]
    \frametitle{Python Code}
    \begin{lstlisting}
 # Plotting points A(1, -2, -8), B(5, 0, -2), and C(11, 3, 7)
 
import numpy as np
import matplotlib.pyplot as plt
from mpl_toolkits.mplot3d import Axes3D

# Define the points as numpy arrays
A = np.array([1, -2, -8])
B = np.array([5, 0, -2])
C = np.array([11, 3, 7])
\end{lstlisting}
\end{frame}

\begin{frame}[fragile]
    \frametitle{Python Code}

    \begin{lstlisting}
 # Create a 3D plot
fig = plt.figure(figsize=(8, 6))
ax = fig.add_subplot(111, projection='3d')

# Plot the points
ax.scatter(*A, color='red', s=100, label='A(1, -2, -8)')
ax.scatter(*B, color='green', s=100, label='B(5, 0, -2)')
ax.scatter(*C, color='blue', s=100, label='C(11, 3, 7)')



    \end{lstlisting}
\end{frame}

\begin{frame}[fragile]
    \frametitle{Python Code}

    \begin{lstlisting}
 # Plot line AC
ax.plot([A[0], C[0]], [A[1], C[1]], [A[2], C[2]], color='purple', label='Line AC')

# Annotate points
ax.text(*A, ' A', color='red', fontsize=10)
ax.text(*B, ' B', color='green', fontsize=10)
ax.text(*C, ' C', color='blue', fontsize=10)


    \end{lstlisting}
\end{frame}

\begin{frame}[fragile]
    \frametitle{Python Code}

    \begin{lstlisting}
 # Set axes labels
ax.set_xlabel('X-axis')
ax.set_ylabel('Y-axis')
ax.set_zlabel('Z-axis')
ax.set_title('3D Plot of Points A, B, C and Line AC')
ax.legend()
ax.grid(True)

# Show the plot
plt.show()



\end{lstlisting}
\end{frame}

 



 


\begin{frame}[fragile]
\frametitle{C Code}
\begin{lstlisting}
#include <stdio.h>

int main() {
    // Given values
    int xB = 2, yB = 6;
    int xC = 3, yC = -1;  // Center of the circle

    // Calculate coordinates of A using midpoint formula
    int xA = 2 * xC - xB;
    int yA = 2 * yC - yB;

    // Print result
    printf("Coordinates of point A are: (%d, %d)\n", xA, yA);

    // Verify midpoint
    float midX = (xA + xB) / 2.0;
    float midY = (yA + yB) / 2.0;
    printf("Midpoint of A and B is: (%.1f, %.1f)\n", midX, midY);

    return 0;
}




\end{lstlisting}

\end{frame}


\begin{frame}[fragile]
\frametitle{Python and C Code}

\begin{lstlisting}
 import subprocess

# Compile the C program
subprocess.run(["gcc", "midpoint.c", "-o", "midpoint"])

# Run the compiled C program
result = subprocess.run(["./midpoint"], capture_output=True, text=True)

# Print the output from the C program (solution steps for k=2/3)
print(result.stdout) 

\end{lstlisting}

\end{frame}

 


\begin{figure}
    \centering
    \includegraphics[width=0.8\columnwidth]{Fig.png}
    \caption{Plot}
    \label{fig:placeholder}
\end{figure}


\end{document}