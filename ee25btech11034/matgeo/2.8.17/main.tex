\let\negmedspace\undefined
\let\negthickspace\undefined
\documentclass[journal]{IEEEtran}
\usepackage[a5paper, margin =  10mm, onecolumn]{geometry}
%\usepackage{lmodern} % Ensure lmodern is loaded for pdflatex
\usepackage{tfrupee} % Include tfrupee package

\setlength{\headheight}{1cm} % Set the height of the header box
\setlength{\headsep}{0mm}     % Set the distance between the header box and the top of the text

\usepackage{gvv-book}
\usepackage{gvv}
\usepackage{cite}
\usepackage{amsmath,amssymb,amsfonts,amsthm}
\usepackage{algorithmic}
\usepackage{graphicx}
\usepackage{textcomp}
\usepackage{xcolor}
\usepackage{txfonts}
\usepackage{listings}
\usepackage{enumitem}
\usepackage{mathtools}
\usepackage{gensymb}
\usepackage{comment}
\usepackage[breaklinks =  true]{hyperref}
\usepackage{tkz-euclide} 
\usepackage{listings}
% \usepackage{gvv}                                        
\def\inputGnumericTable{}                                 
\usepackage[latin1]{inputenc}                                
\usepackage{color}                                            
\usepackage{array}                                            
\usepackage{longtable}                                       
\usepackage{calc}                                             
\usepackage{multirow}                                         
\usepackage{hhline}                                           
\usepackage{ifthen}                                           
\usepackage{lscape}
\begin{document}

\bibliographystyle{IEEEtran}
\vspace{3cm}

\title{2.8.17}
\author{EE25BTECH11034 - Kishora Karthik}
% \maketitle
% \newpage
% \bigskip
{\let\newpage\relax\maketitle}

\renewcommand{\thefigure}{\theenumi}
\renewcommand{\thetable}{\theenumi}
%\setlength{\intextsep}{10pt} % Space between text and floats
\textbf{Question:}\\
Show that the straight lines whose direction cosines $(l, m, n)$ are given by the equations $2l + 2m - n = 0$ and $mn + nl + lm = 0$ are at right angles.\\
\textbf{Solution:}\\
Let the direction cosines be represented by the column vector,
\begin{align}
    \vec{v}=\myvec{l \\ m \\ n}
\end{align}
The linear equation $2l + 2m - n = 0$ can be written as,
\begin{align}
    \vec{c}^\top\vec{v}=0
\end{align}
Where,
\begin{align}
    \vec{c}=\myvec{2\\2\\-1}
\end{align}
The quadratic equation $mn + nl + lm = 0$ is equivalent to $2mn + 2nl + 2lm = 0$. This can be written as a quadratic form $\vec{v}^T A \vec{v} = 0$, where $A$ is the symmetric matrix:
\begin{align}
     A = \myvec{ 0 & 1 & 1 \\ 1 & 0 & 1 \\ 1 & 1 & 0} 
\end{align}
For a system defined by $\vec{c}^\top \vec{v} = 0$ and $\vec{v}^\top A \vec{v} = 0$, the two solution vectors are orthogonal if and only if the following algebraic condition is met:
\begin{align}
     \vec{c}^\top \brak{\text{Tr}(A)I - A}\vec{c} = 0
\end{align}
\begin{align}
    \text{Tr}(A) = 0 + 0 + 0 = 0
\end{align}
Substituting this into the condition, it simplifies to:
\begin{align}
    \vec{c}^T (0 \cdot I - A) \vec{c} = -\vec{c}^\top A \vec{c} = 0
\end{align}
We therefore only need to verify that $\vec{c}^\top A \vec{c} = 0$. 
\begin{align}
   \vec{c}^\top A \vec{c} &= \myvec{ 2 & 2 & -1} \myvec{0 & 1 & 1 \\ 1 & 0 & 1 \\ 1 & 1 & 0} \myvec{2 \\ 2 \\ -1} 
\end{align}
\begin{align}
    \vec{c}^\top A \vec{c}&= \myvec{(2 \cdot 0 + 2 \cdot 1 - 1 \cdot 1) & (2 \cdot 1 + 2 \cdot 0 - 1 \cdot 1) & (2 \cdot 1 + 2 \cdot 1 - 1 \cdot 0)}   \myvec{2 \\ 2 \\ -1} 
\end{align}
\begin{align}
    \vec{c}^\top A \vec{c}&= \myvec{1 & 1 & 4} \myvec{2 \\ 2 \\ -1} 
\end{align}
\begin{align}
    \vec{c}^\top A \vec{c}&= (1)(2) + (1)(2) + (4)(-1) 
\end{align}
\begin{align}
    \vec{c}^\top A \vec{c}&= 2 + 2 - 4 
\end{align}
\begin{align} 
    \vec{c}^\top A \vec{c}&= 0
\end{align}
Since the condition is satisfied, the given lines are at right angles.\\\\

    \centering   \includegraphics[width=\columnwidth, height=1\textheight, keepaspectratio]{figs/fig1.png} 

\end{document}
