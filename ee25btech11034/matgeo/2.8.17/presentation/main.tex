\documentclass{beamer}
\usepackage[utf8]{inputenc}

\usetheme{Madrid}
\usecolortheme{default}
\usepackage{amsmath,amssymb,amsfonts,amsthm}
\usepackage{txfonts}
\usepackage{tkz-euclide}
\usepackage{listings}
\usepackage{adjustbox}
\usepackage{array}
\usepackage{tabularx}
\usepackage{gvv}
\usepackage{lmodern}
\usepackage{circuitikz}
\usepackage{tikz}
\usepackage{graphicx}

\setbeamertemplate{page number in head/foot}[totalframenumber]

\usepackage{tcolorbox}
\tcbuselibrary{minted,breakable,xparse,skins}



\definecolor{bg}{gray}{0.95}
\DeclareTCBListing{mintedbox}{O{}m!O{}}{%
  breakable=true,
  listing engine=minted,
  listing only,
  minted language=#2,
  minted style=default,
  minted options={%
    linenos,
    gobble=0,
    breaklines=true,
    breakafter=,,
    fontsize=\small,
    numbersep=8pt,
    #1},
  boxsep=0pt,
  left skip=0pt,
  right skip=0pt,
  left=25pt,
  right=0pt,
  top=3pt,
  bottom=3pt,
  arc=5pt,
  leftrule=0pt,
  rightrule=0pt,
  bottomrule=2pt,
  toprule=2pt,
  colback=bg,
  colframe=orange!70,
  enhanced,
  overlay={%
    \begin{tcbclipinterior}
    \fill[orange!20!white] (frame.south west) rectangle ([xshift=20pt]frame.north west);
    \end{tcbclipinterior}},
  #3,
}
\lstset{
    language=C,
    basicstyle=\ttfamily\small,
    keywordstyle=\color{blue},
    stringstyle=\color{orange},
    commentstyle=\color{green!60!black},
    numbers=left,
    numberstyle=\tiny\color{gray},
    breaklines=true,
    showstringspaces=false,
}
\begin{document}

\title 
{2.8.17}
\date{September 12,2025}


\author 
{Kishora Karthik-EE25BTECH11034}
\frame{\titlepage}
\begin{frame}{Question}
Show that the straight lines whose direction cosines $(l, m, n)$ are given by the equations $2l + 2m - n = 0$ and $mn + nl + lm = 0$ are at right angles.\\
\end{frame}



\begin{frame}{ Solution}
Let the direction cosines be represented by the column vector,
\begin{align}
    \vec{v}=\myvec{l \\ m \\ n}
\end{align}
The linear equation $2l + 2m - n = 0$ can be written as,
\begin{align}
    \vec{c}^\top\vec{v}=0
\end{align}
Where,
\begin{align}
    \vec{c}=\myvec{2\\2\\-1}
\end{align}
The quadratic equation $mn + nl + lm = 0$ is equivalent to $2mn + 2nl + 2lm = 0$. This can be written as a quadratic form $\vec{v}^T A \vec{v} = 0$, where $A$ is the symmetric matrix:
\end{frame}

\begin{frame}{Solution}
\begin{align}
     A = \myvec{ 0 & 1 & 1 \\ 1 & 0 & 1 \\ 1 & 1 & 0} 
\end{align}
For a system defined by $\vec{c}^\top \vec{v} = 0$ and $\vec{v}^\top A \vec{v} = 0$, the two solution vectors are orthogonal if and only if the following algebraic condition is met:
\begin{align}
     \vec{c}^\top \brak{\text{Tr}(A)I - A}\vec{c} = 0
\end{align}
\begin{align}
    \text{Tr}(A) = 0 + 0 + 0 = 0
\end{align}
Substituting this into the condition, it simplifies to:
\begin{align}
    \vec{c}^\top (0 \cdot I - A) \vec{c} = -\vec{c}^\top A \vec{c} = 0
\end{align}
We therefore only need to verify that $\vec{c}^\top A \vec{c} = 0$.
\end{frame}
\begin{frame}{Solution}
\begin{align}
   \vec{c}^\top A \vec{c} &= \myvec{ 2 & 2 & -1} \myvec{0 & 1 & 1 \\ 1 & 0 & 1 \\ 1 & 1 & 0} \myvec{2 \\ 2 \\ -1} 
\end{align}
\begin{align}
    \vec{c}^\top A \vec{c}&= \myvec{(2 \cdot 0 + 2 \cdot 1 - 1 \cdot 1) & (2 \cdot 1 + 2 \cdot 0 - 1 \cdot 1) & (2 \cdot 1 + 2 \cdot 1 - 1 \cdot 0)} \myvec{2 \\ 2 \\ -1} 
\end{align}
\begin{align}
    \vec{c}^\top A \vec{c}&= \myvec{1 & 1 & 4} \myvec{2 \\ 2 \\ -1} 
\end{align}
\begin{align}
    \vec{c}^\top A \vec{c}&= (1)(2) + (1)(2) + (4)(-1) 
\end{align}
\end{frame}
\begin{frame}{Solution}
\begin{align}
    \vec{c}^\top A \vec{c}&= 2 + 2 - 4 
\end{align}
\begin{align} 
    \vec{c}^\top A \vec{c}&= 0
\end{align}
Since the condition is satisfied, the given lines are at right angles.
\end{frame}
\begin{frame}{Plot}
    \centering
    \includegraphics[width=\columnwidth, height=1\textheight, keepaspectratio]{figs/fig1.png} 
\end{frame}

\end{document}

