\documentclass[journal]{IEEEtran}
\usepackage[a5paper, margin=10mm]{geometry}
%\usepackage{lmodern} % Ensure lmodern is loaded for pdflatex
\usepackage{tfrupee} % Include tfrupee package


\setlength{\headheight}{1cm} % Set the height of the header box
\setlength{\headsep}{0mm}     % Set the distance between the header box and the top of the text


%\usepackage[a5paper, top=10mm, bottom=10mm, left=10mm, right=10mm]{geometry}

%
\setlength{\intextsep}{10pt} % Space between text and floats

\makeindex


\usepackage{cite}
\usepackage{amsmath,amssymb,amsfonts,amsthm}
\usepackage{algorithmic}
\usepackage{graphicx}
\usepackage{textcomp}
\usepackage{xcolor}
\usepackage{txfonts}
\usepackage{listings}
\usepackage{enumitem}
\usepackage{mathtools}
\usepackage{gensymb}
\usepackage{comment}
\usepackage[breaklinks=true]{hyperref}
\usepackage{tkz-euclide} 
\usepackage{listings}
\usepackage{multicol}
\usepackage{xparse}
\usepackage{gvv}
%\def\inputGnumericTable{}                                 
\usepackage[latin1]{inputenc}                                
\usepackage{color}                                            
\usepackage{array}                                            
\usepackage{longtable}                                       
\usepackage{calc}                                             
\usepackage{multirow}                                         
\usepackage{hhline}                                           
\usepackage{ifthen}                                               
\usepackage{lscape}
\usepackage{tabularx}
\usepackage{array}
\usepackage{float}
\usepackage{ar}
\usepackage[version=4]{mhchem}


\newtheorem{theorem}{Theorem}[section]
\newtheorem{problem}{Problem}
\newtheorem{proposition}{Proposition}[section]
\newtheorem{lemma}{Lemma}[section]
\newtheorem{corollary}[theorem]{Corollary}
\newtheorem{example}{Example}[section]
\newtheorem{definition}[problem]{Definition}
\newcommand{\BEQA}{\begin{eqnarray}}
\newcommand{\EEQA}{\end{eqnarray}}

\theoremstyle{remark}


\begin{document}
\bibliographystyle{IEEEtran}
\onecolumn

\title{2.10.56}
\author{INDHIRESH S- EE25BTECH11027}
\maketitle


\renewcommand{\thefigure}{\theenumi}
\renewcommand{\thetable}{\theenumi}

\textbf{Question}.  Let two non-collinear unit vectors $\hat{a}$ and $\hat{b}$ form an acute angle. A point $\vec{P}$ moves so that at any time $t$ the position vector $\overrightarrow{OP}$ (where $\vec{O}$ is the origin) is given by $\hat{\vec{a}}\cos{t} + \hat{\vec{b}}\sin{t}$. When $\vec{P}$ is farthest from origin $\vec{O}$, let $M$ be the length of $\overrightarrow{OP}$ and $\hat{\vec{u}}$ be the unit vector along $\overrightarrow{OP}$. Then,
\begin{multicols}{2} 
    \begin{enumerate}
	    \item $\hat{\vec{u}} = \frac{\hat{\vec{a}}+\hat{\vec{b}}}{|\hat{\vec{a}}+\hat{\vec{b}}|} \text{ and } M = (1+\hat{\vec{a}} \cdot \hat{\vec{b}})^{\frac{1}{2}}$
    	\item $\hat{\vec{u}} = \frac{\hat{\vec{a}}-\hat{\vec{b}}}{|\hat{\vec{a}}-\hat{\vec{b}}|} \text{ and } M = (1+\hat{\vec{a}} \cdot \hat{\vec{b}})^{\frac{1}{2}}$
    	\item $\hat{\vec{u}} = \frac{\hat{\vec{a}}+\hat{\vec{b}}}{|\hat{\vec{a}}+\hat{\vec{b}}|} \text{ and } M = (1+2\hat{\vec{a}} \cdot \hat{\vec{b}})^{\frac{1}{2}}$
    	\item $\hat{\vec{u}} = \frac{\hat{\vec{a}}-\hat{\vec{b}}}{|\hat{\vec{a}}-\hat{\vec{b}}|} \text{ and } M = (1+2\hat{\vec{a}} \cdot \hat{\vec{b}})^{\frac{1}{2}}$
    \end{enumerate}
    \end{multicols}
\textbf{Solution}:\\
Let us solve the given equation theoretically and then verify the solution computationally. \\
Given equation:
\begin{align}
     \Vec{P}=\vec{a}\cos{t} + \vec{b}\sin{t}
\end{align}
Which can be written as :
\begin{align}
    \Vec{P}=\myvec{\vec{a}&\vec{b}}\myvec{\cos{t}\\\sin{t}}
\end{align}
\begin{align}
    \Vec{P}=\myvec{\Vec{a}&\Vec{b}}\Vec{x}
\end{align}
Let
\begin{align}
    \Vec{x}= \myvec{\cos{t} \\ \sin{t}} \;\;and\;\;\vec{G}=\myvec{1&(\Vec{a})^T(\Vec{b})\\(\Vec{a})^T(\Vec{b})&1}
\end{align}
From given if \textbf{P} is farthest from origin , then length of $\vec{P}$ is given as M.From this we can say  that
\begin{align}
   M=\text{max}\norm{\Vec{P}}
\end{align}
Now,
\begin{align}
    \norm{\Vec{P}}=\sqrt{(\Vec{P})^T(\Vec{P})}
\end{align}
\begin{align}
    \norm{\Vec{P}}=\sqrt{\brak{\myvec{\Vec{a}&\Vec{b}}\Vec{x}}^T\brak{\myvec{\Vec{a}&\Vec{b}}\Vec{x}}}
\end{align}
\begin{align}
    \norm{\Vec{P}}=\sqrt{\Vec{x}^T\myvec{\Vec{a}&\Vec{b}}^T\myvec{\Vec{a}&\Vec{b}}\Vec{x}}
\end{align}
Let $\Vec{G}$ be a gram matrix:
\begin{align}
    \Vec{G}=\myvec{\Vec{a}&\Vec{b}}^T\myvec{\Vec{a}&\Vec{b}}=\myvec{1&(\Vec{a})^T(\Vec{b})\\(\Vec{a})^T(\Vec{b})&1}
\end{align}
\begin{align}
       \norm{\Vec{P}}^2=\Vec{x}^T\myvec{1&(\Vec{a})^T(\Vec{b})\\(\Vec{a})^T(\Vec{b})&1}\Vec{x}
\end{align}

\begin{align}
\norm{\Vec{P}}^2=\Vec{x}^T\vec{G}\Vec{x}
  \end{align}
Now we should find the maximum value of \textbf{$x^TGx$} such that $\norm{x}=1$\\\\
By \textbf{Rayleigh-Ritz theorem}, the maximum value of the quadratic form if \textbf{x} is a unit vector will be the largest eigenvalue $(\lambda_{max})$ of the matrix G.\\
So,
\begin{align}
    max\norm{\Vec{P}}=\sqrt{\lambda_{max}}
\end{align}
Now we will calculate the Eigen value for the matrix G:
\begin{align}
    \mydet{\Vec{G}-\lambda \Vec{I}}=0
\end{align}
\begin{align}
    \mydet{\myvec{1-\lambda&(\Vec{a})^T(\Vec{b})\\(\Vec{a})^T(\Vec{b})&1-\lambda}}=0
\end{align}
\begin{align}
    (1-\lambda)^2-((\Vec{a})^T(\Vec{b}))^2=0
\end{align}
\begin{align}
    1-\lambda=(\Vec{a})^T(\Vec{b})\;\;or\;\;1-\lambda=-(\Vec{a})^T(\Vec{b})
\end{align}
\begin{align}
    \lambda=1+(\Vec{a})^T(\Vec{b})\;\;or \;\;\lambda=1-(\Vec{a})^T(\Vec{b})
\end{align}
It is already given that $(\Vec{a})^T(\Vec{b})>0 $($\Vec{a}$ and $\Vec{b}$ form an acute angle) . so,
\begin{align}
 \lambda_{max}=1+  (\Vec{a})^T(\Vec{b}) 
\end{align}
From Eq.12
\begin{align}
    max\norm{\Vec{P}}=\sqrt{1+  (\Vec{a})^T(\Vec{b}) }
\end{align}
The above equation can be written as
\begin{align}
   max \norm{\Vec{P}}=\sqrt{1+\Vec{a}.\Vec{b}}
\end{align}
From Eq.5:
\begin{align}
    M=\sqrt{1+\Vec{a}.\Vec{b}}
\end{align}
Now let us find the value of t for which $\norm{\Vec{P}}$ is max\\
With eigenvalue equation,We'll use matrix G and largest eigenvalue $\lambda_{max}$ such that,
\begin{align}
    \myvec{\Vec{G}-\lambda \Vec{I}}\Vec{x}=0
\end{align}
\begin{align}
    \brak{\myvec{1&(\Vec{a})^T(\Vec{b})\\(\Vec{a})^T(\Vec{b})&1}-\myvec{\lambda&0\\0&\lambda}}\Vec{x}=\myvec{0\\0}
\end{align}
\begin{align}
    \myvec{1-\lambda&(a)^T(b)\\(a)^T(b)&1-\lambda}\Vec{x}=\myvec{0\\0}
\end{align}
By substituting  $\lambda = 1+  (\Vec{a})^T(\Vec{b}) $ . We get:

\begin{align}
    \myvec{-(\Vec{a})^T(\Vec{b})&(\Vec{a})^T(\Vec{b})\\(\Vec{a})^T(\Vec{b})&-(\Vec{a})^T(\Vec{b})}\Vec{x}=\myvec{0\\0}
\end{align}
\begin{align}
    \myvec{-(\Vec{a})^T(\Vec{b})&(\Vec{a})^T(\Vec{b})\\(\Vec{a})^T(\Vec{b})&-(\Vec{a})^T(\Vec{b})}\myvec{\cos{t}\\sin{t}}=\myvec{0\\0}
\end{align}
\begin{align}
    \myvec{-(\Vec{a})^T(\Vec{b})\cos{t}+(\Vec{a})^T(\Vec{b})\sin{t}\\(\Vec{a})^T(\Vec{b})\cos{t}-(\Vec{a})^T(\Vec{b})\sin{t}}=\myvec{0\\0}
\end{align}
\begin{align}
    -(\Vec{a})^T(\Vec{b})\cos{t}+(\Vec{a})^T(\Vec{b})\sin{t}=0
\end{align}
\begin{align}
    (\Vec{a})^T(\Vec{b})\cos{t}=(\Vec{a})^T(\Vec{b})\sin{t}
\end{align}
\begin{align}
    \cos{t}=\sin{t}
\end{align}
We already know that:
\begin{align}
    \sin^2{t}+\cos^2{t}=1
\end{align}
So,
\begin{align}
    \sin{t}=\frac{1}{\sqrt{2}}\;\;and\;\; \cos{t}=\frac{1}{\sqrt{2}}
\end{align}
From above result
\begin{align}
    t=\frac{\pi}{4}
\end{align}

Now unit vector $\Vec{u}$ along $\Vec{P}$ is given by:
\begin{align}
    \Vec{{\Vec{u}}}=\frac{\Vec{P}}{\norm{\Vec{P}}}
\end{align}
\begin{align}
\Vec{\Vec{u}}=\frac{\vec{a}\cos{t} + \vec{b}\sin{t}}{\norm{\vec{a}\cos{t} + \vec{b}\sin{t}}}
\end{align}
Now subtituiting the value of $t$ in above equation:
\begin{align}
    \Vec{u}=\frac{\Vec{a}\frac{1}{\sqrt{2}} + \Vec{b}\frac{1}{\sqrt{2}}}{\norm{\Vec{a}\frac{1}{\sqrt{2}} + \Vec{b}\frac{1}{\sqrt{2}}}}
\end{align}
\begin{align}
    \Vec{u}=\frac{\Vec{a} + \Vec{b}}{\norm{\Vec{a} + \Vec{b}}}
\end{align}
From Eq.21 and Eq.37 (a) is correct\\
From the figure it is clearly verified that the theoretical solution matches with the computational solution.\\
\begin{figure}[h]
    \centering
    \includegraphics[height=0.5\textheight, keepaspectratio]{figs/figure1.png}
    \label{figure_1}
\end{figure}

\end{document}
