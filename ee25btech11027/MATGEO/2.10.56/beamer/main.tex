\documentclass{beamer}
\usepackage[utf8]{inputenc}

\usetheme{Madrid}
\usecolortheme{default}
\usepackage{amsmath,amssymb,amsfonts,amsthm}
\usepackage{txfonts}
\usepackage{tkz-euclide}
\usepackage{listings}
\usepackage{adjustbox}
\usepackage{array}
\usepackage{tabularx}
\usepackage{gvv}
\usepackage{lmodern}
\usepackage{circuitikz}
\usepackage{tikz}
\usepackage{graphicx}

\setbeamertemplate{page number in head/foot}[totalframenumber]

\usepackage{tcolorbox}
\tcbuselibrary{minted,breakable,xparse,skins}



\definecolor{bg}{gray}{0.95}
\DeclareTCBListing{mintedbox}{O{}m!O{}}{%
  breakable=true,
  listing engine=minted,
  listing only,
  minted language=#2,
  minted style=default,
  minted options={%
    linenos,
    gobble=0,
    breaklines=true,
    breakafter=,,
    fontsize=\small,
    numbersep=8pt,
    #1},
  boxsep=0pt,
  left skip=0pt,
  right skip=0pt,
  left=25pt,
  right=0pt,
  top=3pt,
  bottom=3pt,
  arc=5pt,
  leftrule=0pt,
  rightrule=0pt,
  bottomrule=2pt,
  toprule=2pt,
  colback=bg,
  colframe=orange!70,
  enhanced,
  overlay={%
    \begin{tcbclipinterior}
    \fill[orange!20!white] (frame.south west) rectangle ([xshift=20pt]frame.north west);
    \end{tcbclipinterior}},
  #3,
}
\lstset{
    language=C,
    basicstyle=\ttfamily\small,
    keywordstyle=\color{blue},
    stringstyle=\color{orange},
    commentstyle=\color{green!60!black},
    numbers=left,
    numberstyle=\tiny\color{gray},
    breaklines=true,
    showstringspaces=false,
}
%------------------------------------------------------------
%This block of code defines the information to appear in the
%Title page
\title %optional
{2.10.56}
\date{8 September, 2025}
%\subtitle{A short story}

\author % (optional)
{INDHIRESH S - EE25BTECH11027}



\begin{document}


\frame{\titlepage}
\begin{frame}{Question}
Let two non-collinear unit vectors $\hat{a}$ and $\hat{b}$ form an acute angle. A point $\vec{P}$ moves so that at any time $t$ the position vector $\Vec{P}$ (where $\vec{O}$ is the origin) is given by $\Vec{a}\cos{t} + \vec{b}\sin{t}$. When $\vec{P}$ is farthest from origin $\vec{O}$, let $M$ be the length of $\Vec{P}$ and $\hat{\vec{u}}$ be the unit vector along $\Vec{P}$. Then,

    \begin{enumerate}
	    \item $\hat{\vec{u}} = \frac{\Vec{a}+\vec{b}}{|\Vec{a}+\vec{b}|} \text{ and } M = (1+\Vec{a} \cdot \vec{b})^{\frac{1}{2}}$
    	\item $\hat{\vec{u}} = \frac{\Vec{a}-\vec{b}}{|\Vec{a}-\vec{b}|} \text{ and } M = (1+\Vec{a} \cdot \vec{b})^{\frac{1}{2}}$
    	\item $\hat{\vec{u}} = \frac{\Vec{a}+\vec{b}}{|\Vec{a}+\vec{b}|} \text{ and } M = (1+2\Vec{a} \cdot \vec{b})^{\frac{1}{2}}$
    	\item $\hat{\vec{u}} = \frac{\Vec{a}-\vec{b}}{|\Vec{a}-\vec{b}|} \text{ and } M = (1+2\Vec{a} \cdot \vec{b})^{\frac{1}{2}}$
    \end{enumerate}
    
\end{frame}
\begin{frame}{allowframebreaks}
\frametitle{Equation}

    \centering
    
    \label{tab:parameters}
Given equation:
\begin{align}
     \Vec{P}=\vec{a}\cos{t} + \vec{b}\sin{t}
\end{align}
Which can be written as :
\begin{align}
    \Vec{P}=\myvec{\vec{a}&\vec{b}}\myvec{\cos{t}\\\sin{t}}
\end{align}
\begin{align}
    \Vec{P}=\myvec{\Vec{a}&\Vec{b}}\Vec{x}
\end{align}
Let
\begin{align}
    \Vec{x}= \myvec{\cos{t} \\ \sin{t}} \;\;and\;\;\vec{G}=\myvec{1&(\Vec{a})^T(\Vec{b})\\(\Vec{a})^T(\Vec{b})&1}
\end{align}

   
\end{frame}


\begin{frame}
\frametitle{Theoretical Solution}
From given if \textbf{P} is farthest from origin , then length of $\vec{P}$ is given as M.From this we can say  that
\begin{align}
   M=\text{max}\norm{\Vec{P}}
\end{align}
Now,
\begin{align}
    \norm{\Vec{P}}=\sqrt{(\Vec{P})^T(\Vec{P})}
\end{align}
\begin{align}
    \norm{\Vec{P}}=\sqrt{\brak{\myvec{\Vec{a}&\Vec{b}}\Vec{x}}^T\brak{\myvec{\Vec{a}&\Vec{b}}\Vec{x}}}
\end{align}
\begin{align}
    \norm{\Vec{P}}=\sqrt{\Vec{x}^T\myvec{\Vec{a}&\Vec{b}}^T\myvec{\Vec{a}&\Vec{b}}\Vec{x}}
\end{align}






\end{frame}
\begin{frame}
\frametitle{Theoretical solution}
Let $\Vec{G}$ be a gram matrix:
\begin{align}
    \Vec{G}=\myvec{\Vec{a}&\Vec{b}}^T\myvec{\Vec{a}&\Vec{b}}=\myvec{1&(\Vec{a})^T(\Vec{b})\\(\Vec{a})^T(\Vec{b})&1}
\end{align}
\begin{align}
       \norm{\Vec{P}}^2=\Vec{x}^T\myvec{1&(\Vec{a})^T(\Vec{b})\\(\Vec{a})^T(\Vec{b})&1}\Vec{x}
\end{align}

\begin{align}
\norm{\Vec{P}}^2=\Vec{x}^T\vec{G}\Vec{x}
  \end{align}
Now we should find the maximum value of \textbf{$x^TGx$} such that $\norm{x}=1$\\\\
By \textbf{Rayleigh-Ritz theorem}, the maximum value of the quadratic form if \textbf{x} is a unit vector will be the largest eigenvalue $(\lambda_{max})$ of the matrix G.\\
So,
\begin{align}
    max\norm{\Vec{P}}=\sqrt{\lambda_{max}}
\end{align}


\end{frame}
\begin{frame}
\frametitle{Theoretical Solution}
   Now we will calculate the Eigen value for the matrix G:
\begin{align}
    \mydet{\Vec{G}-\lambda \Vec{I}}=0
\end{align}
\begin{align}
    \mydet{\myvec{1-\lambda&(\Vec{a})^T(\Vec{b})\\(\Vec{a})^T(\Vec{b})&1-\lambda}}=0
\end{align}
\begin{align}
    (1-\lambda)^2-((\Vec{a})^T(\Vec{b}))^2=0
\end{align}
\begin{align}
    1-\lambda=(\Vec{a})^T(\Vec{b})\;\;or\;\;1-\lambda=-(\Vec{a})^T(\Vec{b})
\end{align}
\begin{align}
    \lambda=1+(\Vec{a})^T(\Vec{b})\;\;or \;\;\lambda=1-(\Vec{a})^T(\Vec{b})
\end{align}
It is already given that $(\Vec{a})^T(\Vec{b})>0 $($\Vec{a}$ and $\Vec{b}$ form an acute angle) . so,
\begin{align}
 \lambda_{max}=1+  (\Vec{a})^T(\Vec{b}) 
\end{align}


\end{frame}
\begin{frame}
\frametitle{Theoretical Solution}
  From Eq.12
\begin{align}
    max\norm{\Vec{P}}=\sqrt{1+  (\Vec{a})^T(\Vec{b}) }
\end{align}
The above equation can be written as
\begin{align}
   max \norm{\Vec{P}}=\sqrt{1+\Vec{a}.\Vec{b}}
\end{align}
From Eq.5:
\begin{align}
    M=\sqrt{1+\Vec{a}.\Vec{b}}
\end{align}
Now let us find the value of t for which $\norm{\Vec{P}}$ is max\\
With eigenvalue equation,We'll use matrix G and largest eigenvalue $\lambda_{max}$ such that,
\begin{align}
    \myvec{\Vec{G}-\lambda \Vec{I}}\Vec{x}=0
\end{align}
\begin{align}
    \brak{\myvec{1&(\Vec{a})^T(\Vec{b})\\(\Vec{a})^T(\Vec{b})&1}-\myvec{\lambda&0\\0&\lambda}}\Vec{x}=\myvec{0\\0}
\end{align}

\end{frame}
\begin{frame}
\frametitle{Theoretical Solution}
\begin{align}
    \myvec{1-\lambda&(a)^T(b)\\(a)^T(b)&1-\lambda}\Vec{x}=\myvec{0\\0}
\end{align}
By substituting  $\lambda = 1+  (\Vec{a})^T(\Vec{b}) $ . We get:

\begin{align}
    \myvec{-(\Vec{a})^T(\Vec{b})&(\Vec{a})^T(\Vec{b})\\(\Vec{a})^T(\Vec{b})&-(\Vec{a})^T(\Vec{b})}\Vec{x}=\myvec{0\\0}
\end{align}
\begin{align}
    \myvec{-(\Vec{a})^T(\Vec{b})&(\Vec{a})^T(\Vec{b})\\(\Vec{a})^T(\Vec{b})&-(\Vec{a})^T(\Vec{b})}\myvec{\cos{t}\\\sin{t}}=\myvec{0\\0}
\end{align}
\begin{align}
    \myvec{-(\Vec{a})^T(\Vec{b})\cos{t}+(\Vec{a})^T(\Vec{b})\sin{t}\\(\Vec{a})^T(\Vec{b})\cos{t}-(\Vec{a})^T(\Vec{b})\sin{t}}=\myvec{0\\0}
\end{align}
\begin{align}
    -(\Vec{a})^T(\Vec{b})\cos{t}+(\Vec{a})^T(\Vec{b})\sin{t}=0
\end{align}




\end{frame}
\begin{frame}
\frametitle{Theoretical Solution}
   \begin{align}
    (\Vec{a})^T(\Vec{b})\cos{t}=(\Vec{a})^T(\Vec{b})\sin{t}
\end{align}
\begin{align}
    \cos{t}=\sin{t}
\end{align}
We already know that:
\begin{align}
    \sin^2{t}+\cos^2{t}=1
\end{align}
So,
\begin{align}
    \sin{t}=\frac{1}{\sqrt{2}}\;\;and\;\; \cos{t}=\frac{1}{\sqrt{2}}
\end{align}
From above result
\begin{align}
    t=\frac{\pi}{4}
\end{align}


\end{frame}

\begin{frame}
\frametitle{Theoretical Solution}
    Now unit vector $\Vec{u}$ along $\Vec{P}$ is given by:
\begin{align}
    \Vec{{\Vec{u}}}=\frac{\Vec{P}}{\norm{\Vec{P}}}
\end{align}
\begin{align}
\Vec{\Vec{u}}=\frac{\vec{a}\cos{t} + \vec{b}\sin{t}}{\norm{\vec{a}\cos{t} + \vec{b}\sin{t}}}
\end{align}
Now subtituiting the value of $t$ in above equation:
\begin{align}
    \Vec{u}=\frac{\Vec{a}\frac{1}{\sqrt{2}} + \Vec{b}\frac{1}{\sqrt{2}}}{\norm{\Vec{a}\frac{1}{\sqrt{2}} + \Vec{b}\frac{1}{\sqrt{2}}}}
\end{align}
\begin{align}
    \Vec{u}=\frac{\Vec{a} + \Vec{b}}{\norm{\Vec{a} + \Vec{b}}}
\end{align}
From Eq.21 and Eq.37 (a) is correct

    
\end{frame}




\begin{frame}[fragile]
    \frametitle{C Code  }

    \begin{lstlisting}
#include <stdio.h>
#include <math.h>

// Dot product of two 2D vectors
double dot(double a[], double b[]) {
    return a[0]*b[0] + a[1]*b[1];
}

// Magnitude of a 2D vector
double magnitude(double a[]) {
    return sqrt(dot(a, a));
}

// Compute max length M and unit vector u using matrix method
void compute(double a[], double b[], double *M, double u[]) {
    double c = dot(a, b);        // a · b
    *M = sqrt(1 + c);            // largest eigenvalue's sqrt

   
}



    \end{lstlisting}
\end{frame}

\begin{frame}[fragile]
    \frametitle{C Code  }

    \begin{lstlisting}


    // Direction = a + b
    double temp[2] = {a[0] + b[0], a[1] + b[1]};
    double norm = magnitude(temp);
    u[0] = temp[0] / norm;
    u[1] = temp[1] / norm;
}



    \end{lstlisting}
\end{frame}


\begin{frame}[fragile]
    \frametitle{Python Code}
    \begin{lstlisting}
import ctypes
import numpy as np
import matplotlib.pyplot as plt

# Load shared library
lib = ctypes.CDLL("./vec.so")   # use "vec.dll" on Windows

# Define argument & return types
lib.compute.argtypes = [
    np.ctypeslib.ndpointer(dtype=np.double, ndim=1, flags="C_CONTIGUOUS"),
    np.ctypeslib.ndpointer(dtype=np.double, ndim=1, flags="C_CONTIGUOUS"),
    ctypes.POINTER(ctypes.c_double),
    np.ctypeslib.ndpointer(dtype=np.double, ndim=1, flags="C_CONTIGUOUS")
]











    \end{lstlisting}
\end{frame}

\begin{frame}[fragile]
    \frametitle{Python Code}
    \begin{lstlisting}
# Example vectors
a = np.array([1.0, 0.0], dtype=np.double)
b = np.array([0.6, 0.8], dtype=np.double)

M = ctypes.c_double()
u = np.zeros(2, dtype=np.double)

# Call C function
lib.compute(a, b, ctypes.byref(M), u)

print("From C library:")
print("M =", M.value)
print("u =", u)

# Plot in same style as attachment
O = np.array([0.0, 0.0])
P = u * M.value












    \end{lstlisting}
\end{frame}

\begin{frame}[fragile]
    \frametitle{Python Code}

    \begin{lstlisting}
plt.plot([O[0], P[0]], [O[1], P[1]], 'b-', label="Vector OP")
plt.scatter(*O, color="red", s=100, label="O(0,0)")
plt.scatter(*P, color="green", s=100, label=f"P({P[0]:.2f},{P[1]:.2f})")
plt.scatter(u, color="purple", marker="", s=200, label=f"u({u[0]:.2f},{u[1]:.2f})")
plt.axhline(0, color='black')
plt.axvline(0, color='black')
plt.legend()
plt.title("Figure")
plt.savefig("/media/indhiresh-s/New Volume/Matrix/ee1030-2025/ee25btech11027/MATGEO/2.10.56/figs/figure1.png")
plt.show()






    \end{lstlisting}
\end{frame}



\begin{frame}{Plot}
    \begin{center}
        \includegraphics[width=\columnwidth, height=0.8\textheight, keepaspectratio]{figs/figure1.png}
    \end{center}
\end{frame}




\end{document}