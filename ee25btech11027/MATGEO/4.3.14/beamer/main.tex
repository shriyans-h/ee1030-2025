\documentclass{beamer}
\usepackage[utf8]{inputenc}

\usetheme{Madrid}
\usecolortheme{default}
\usepackage{amsmath,amssymb,amsfonts,amsthm}
\usepackage{txfonts}
\usepackage{tkz-euclide}
\usepackage{listings}
\usepackage{adjustbox}
\usepackage{array}
\usepackage{tabularx}
\usepackage{gvv}
\usepackage{lmodern}
\usepackage{circuitikz}
\usepackage{tikz}
\usepackage{graphicx}

\setbeamertemplate{page number in head/foot}[totalframenumber]

\usepackage{tcolorbox}
\tcbuselibrary{minted,breakable,xparse,skins}



\definecolor{bg}{gray}{0.95}
\DeclareTCBListing{mintedbox}{O{}m!O{}}{%
  breakable=true,
  listing engine=minted,
  listing only,
  minted language=#2,
  minted style=default,
  minted options={%
    linenos,
    gobble=0,
    breaklines=true,
    breakafter=,,
    fontsize=\small,
    numbersep=8pt,
    #1},
  boxsep=0pt,
  left skip=0pt,
  right skip=0pt,
  left=25pt,
  right=0pt,
  top=3pt,
  bottom=3pt,
  arc=5pt,
  leftrule=0pt,
  rightrule=0pt,
  bottomrule=2pt,
  toprule=2pt,
  colback=bg,
  colframe=orange!70,
  enhanced,
  overlay={%
    \begin{tcbclipinterior}
    \fill[orange!20!white] (frame.south west) rectangle ([xshift=20pt]frame.north west);
    \end{tcbclipinterior}},
  #3,
}
\lstset{
    language=C,
    basicstyle=\ttfamily\small,
    keywordstyle=\color{blue},
    stringstyle=\color{orange},
    commentstyle=\color{green!60!black},
    numbers=left,
    numberstyle=\tiny\color{gray},
    breaklines=true,
    showstringspaces=false,
}
%------------------------------------------------------------
%This block of code defines the information to appear in the
%Title page
\title %optional
{4.3.14}
\date{9 September, 2025}
%\subtitle{A short story}

\author % (optional)
{INDHIRESH S - EE25BTECH11027}



\begin{document}


\frame{\titlepage}
\begin{frame}{Question}
 A line intersects the Y axis and X axis at the points $\Vec{P}$ and $\vec{Q}$, respectively. If $(2, 5)$ is the mid-point of PQ, then the coordinates of $\Vec{P}$ and $\Vec{Q}$ are
    
\end{frame}
\begin{frame}{allowframebreaks}
\frametitle{Equation}

    \centering
    
    \label{tab:parameters}
Let,
\begin{align}
     \Vec{P}=\myvec{0\\a}\;\;and\;\;\Vec{Q}=\myvec{b\\0}
\end{align}
   
\end{frame}


\begin{frame}
\frametitle{Theoretical Solution}
Let
\begin{align}
    \Vec{C}=\myvec{2\\5}
\end{align}
Given that $\Vec{C}$ is the midpoint of $\Vec{P}$ and $\Vec{Q}$. So,
\begin{align}
    \Vec{C}=\frac{\Vec{P}+\Vec{Q}}{2}
\end{align}
Now,
\begin{align}
\myvec{2\\5}=\frac{\myvec{0\\a}+\myvec{b\\0}}{2}
\end{align}



\end{frame}
\begin{frame}
\frametitle{Theoretical solution}
\begin{align}
    \myvec{2\\5}=\myvec{\frac{b}{2}\\\frac{a}{2}}
\end{align}

\begin{align}
    a=10\;\;and\;\;b=4
\end{align}
Subtituting the value of a and b in Eq.1, we get:
\begin{align}
    \Vec{P}=\myvec{0\\10}\;\;and\;\;\Vec{Q}=\myvec{4\\0}
\end{align}


\end{frame}





\begin{frame}[fragile]
    \frametitle{C Code  }

    \begin{lstlisting}
#include <stdio.h>

// Function to fill coordinates of P and Q
void get_points(int *Px, int *Py, int *Qx, int *Qy) {
    int mx = 2, my = 5;   // midpoint given

    *Px = 0;
    *Py = 2 * my;   // y1 = 10
    *Qx = 2 * mx;   // x1 = 4
    *Qy = 0;
}



    \end{lstlisting}
\end{frame}


\begin{frame}[fragile]
    \frametitle{Python Code}
    \begin{lstlisting}
import ctypes
import matplotlib.pyplot as plt

# Load compiled C library
lib = ctypes.CDLL("./midpoint.so")

# Define return type and argument types
lib.get_points.argtypes = [ctypes.POINTER(ctypes.c_int), ctypes.POINTER(ctypes.c_int),
                           ctypes.POINTER(ctypes.c_int), ctypes.POINTER(ctypes.c_int)]

# Prepare variables
Px, Py, Qx, Qy = ctypes.c_int(), ctypes.c_int(), ctypes.c_int(), ctypes.c_int()











    \end{lstlisting}
\end{frame}

\begin{frame}[fragile]
    \frametitle{Python Code}
    \begin{lstlisting}
# Call C function
lib.get_points(ctypes.byref(Px), ctypes.byref(Py), ctypes.byref(Qx), ctypes.byref(Qy))

# Extract results
P = (Px.value, Py.value)
Q = (Qx.value, Qy.value)
M = (2, 5)  # midpoint given

# --- Plotting ---
plt.plot([P[0], Q[0]], [P[1], Q[1]], 'b-', label="Line PQ")

# Plot P, Q, M
plt.scatter(*P, color="red", s=100, label=f"P{P}")
plt.scatter(*Q, color="green", s=100, label=f"Q{Q}")
plt.scatter(*M, color="purple", s=150, marker="*", label=f"M{M}")













    \end{lstlisting}
\end{frame}

\begin{frame}[fragile]
    \frametitle{Python Code}

    \begin{lstlisting}
# Annotate
plt.text(P[0]+0.2, P[1], f"P{P}", fontsize=10)
plt.text(Q[0]+0.2, Q[1], f"Q{Q}", fontsize=10)
plt.text(M[0]+0.2, M[1], f"M{M}", fontsize=10, color="purple")

# Axes
plt.axhline(0, color='black')
plt.axvline(0, color='black')

plt.title("Figure")
plt.xlabel("X-axis")
plt.ylabel("Y-axis")
plt.legend()
plt.grid(True)
plt.savefig("/media/indhiresh-s/New Volume/Matrix/ee1030-2025/ee25btech11027/MATGEO/4.3.14/figs/figure1.png")
plt.show()







    \end{lstlisting}
\end{frame}



\begin{frame}{Plot}
    \begin{center}
        \includegraphics[width=\columnwidth, height=0.8\textheight, keepaspectratio]{figs/figure1.png}
    \end{center}
\end{frame}




\end{document}
