\documentclass[journal]{IEEEtran}
\usepackage[a5paper, margin=10mm]{geometry}
%\usepackage{lmodern} % Ensure lmodern is loaded for pdflatex
\usepackage{tfrupee} % Include tfrupee package


\setlength{\headheight}{1cm} % Set the height of the header box
\setlength{\headsep}{0mm}     % Set the distance between the header box and the top of the text


%\usepackage[a5paper, top=10mm, bottom=10mm, left=10mm, right=10mm]{geometry}

%
\setlength{\intextsep}{10pt} % Space between text and floats

\makeindex


\usepackage{cite}
\usepackage{amsmath,amssymb,amsfonts,amsthm}
\usepackage{algorithmic}
\usepackage{graphicx}
\usepackage{textcomp}
\usepackage{xcolor}
\usepackage{txfonts}
\usepackage{listings}
\usepackage{enumitem}
\usepackage{mathtools}
\usepackage{gensymb}
\usepackage{comment}
\usepackage[breaklinks=true]{hyperref}
\usepackage{tkz-euclide} 
\usepackage{listings}
\usepackage{multicol}
\usepackage{xparse}
\usepackage{gvv}
%\def\inputGnumericTable{}                                 
\usepackage[latin1]{inputenc}                                
\usepackage{color}                                            
\usepackage{array}                                            
\usepackage{longtable}                                       
\usepackage{calc}                                             
\usepackage{multirow}                                         
\usepackage{hhline}                                           
\usepackage{ifthen}                                               
\usepackage{lscape}
\usepackage{tabularx}
\usepackage{array}
\usepackage{float}
\usepackage{ar}
\usepackage[version=4]{mhchem}


\newtheorem{theorem}{Theorem}[section]
\newtheorem{problem}{Problem}
\newtheorem{proposition}{Proposition}[section]
\newtheorem{lemma}{Lemma}[section]
\newtheorem{corollary}[theorem]{Corollary}
\newtheorem{example}{Example}[section]
\newtheorem{definition}[problem]{Definition}
\newcommand{\BEQA}{\begin{eqnarray}}
\newcommand{\EEQA}{\end{eqnarray}}

\theoremstyle{remark}


\begin{document}
\bibliographystyle{IEEEtran}
\onecolumn

\title{4.7.13}
\author{INDHIRESH S- EE25BTECH11027}
\maketitle


\renewcommand{\thefigure}{\theenumi}
\renewcommand{\thetable}{\theenumi}

\textbf{Question}.   Find the distance between the lines $l_1$ and $l_2$ given by
\begin{align*}
    \overrightarrow{r}=\hat{i}+2\hat{j}-4\hat{k}+\lambda(2\hat{i}+3\hat{j}+6\hat{k})\\\overrightarrow{r}=3\hat{i}+3\hat{j}-5\hat{k}+\mu(2\hat{i}+3\hat{j}+6\hat{k})
\end{align*}
\textbf{Solution}:\\
Let us solve the given equation theoretically and then verify the solution computationally. \\
Given equation:
\begin{align}
   \overrightarrow{\Vec{r}}=\hat{i}+2\hat{j}-4\hat{k}+\lambda(2\hat{i}+3\hat{j}+6\hat{k}) \\
\overrightarrow{\Vec{r}}=3\hat{i}+3\hat{j}-5\hat{k}+\mu(2\hat{i}+3\hat{j}+6\hat{k})\\
\end{align}
The given lines are in the form
\begin{align}
   \Vec{r} = \Vec{a_1} + \lambda\Vec{b}\\
   \Vec{r} = \Vec{a_2} + \mu\Vec{b}
\end{align}
Where,
\begin{align}
    \Vec{a_1}=\myvec{1\\2\\-4}\;\;\Vec{a_2}=\myvec{3\\3\\-5}\;\;\Vec{b}=\myvec{2\\3\\6}
\end{align}
The given two lines are parallel.The distance between two parallel lines is given by:
\begin{align}
     d = \frac{\norm{(\Vec{a}_2 - \Vec{a}_1) \times \Vec{b}}}{\norm{\Vec{b}}}
\end{align}

\begin{align}
   \Vec{a_2}-\Vec{a_1}=\myvec{2\\1\\-1}\;\;and\;\;\Vec{b}=\myvec{2\\3\\6}
\end{align}
Let,
\begin{align}
    \Vec{a_2}-\Vec{a_1}=\Vec{a}
\end{align}
Now finding:
\begin{align}
    (\Vec{a}_2 - \Vec{a}_1) \times \Vec{b}=\Vec{a}\times\Vec{b}
\end{align}
\begin{align}
   \mydet{\Vec{A_{23} \;\; \Vec{B_{23}}}}=\mydet{1&3\\-1&6}=9
\end{align}
\begin{align}
     \mydet{\Vec{A_{31} \;\; \Vec{B_{31}}}}=\mydet{2&2\\-1&6}=14
\end{align}
\begin{align}
     \mydet{\Vec{A_{12} \;\; \Vec{B_{12}}}}=\mydet{2&2\\1&3}=4
\end{align}
\begin{align}
     \norm{\Vec{a}\times\Vec{b}}=\left\|\myvec{ \mydet{\Vec{A_{23} \;\; \Vec{B_{23}}}} \\\\ \mydet{\Vec{A_{31} \;\; \Vec{B_{31}}}} \\\\ \mydet{\Vec{A_{12} \;\; \Vec{B_{12}}}}}  \right\|
\end{align}
\begin{align}
      \norm{\Vec{a}\times\Vec{b}}=\left\|\myvec{9\\ 14\\ 4}  \right\|
\end{align}
\begin{align}
       \norm{\Vec{a}\times\Vec{b}}=\sqrt{293}
\end{align}

\begin{align}
     \norm{(\Vec{a}_2 - \Vec{a}_1) \times \Vec{b}}=\sqrt{293}
\end{align}

\begin{align}
\norm{\Vec{b}}=\sqrt{\Vec{b^T}\Vec{b}}
  \end{align}

\begin{align}
   \norm{\Vec{b}}=\sqrt{4+9+36}=\sqrt{49}
\end{align}

\begin{align}
    \norm{\Vec{b}}=7
\end{align}
Substituting the values in Eq.7:

\begin{align}
  d=\frac{\sqrt{293}}{7}
\end{align}
Therefore the distance between the lines $l_1$ and $l_2$ is $\frac{\sqrt{293}}{7}$

From the figure it is clearly verified that the theoretical solution matches with the computational solution.\\
\begin{figure}[h]
    \centering
    \includegraphics[height=0.5\textheight, keepaspectratio]{figs/figure1.png}
    \label{figure_1}
\end{figure}

\end{document}