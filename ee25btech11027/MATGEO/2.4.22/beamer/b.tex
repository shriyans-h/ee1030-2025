\documentclass{beamer}
\usepackage[utf8]{inputenc}

\usetheme{Madrid}
\usecolortheme{default}
\usepackage{amsmath,amssymb,amsfonts,amsthm}
\usepackage{txfonts}
\usepackage{tkz-euclide}
\usepackage{listings}
\usepackage{adjustbox}
\usepackage{array}
\usepackage{tabularx}
\usepackage{gvv}
\usepackage{lmodern}
\usepackage{circuitikz}
\usepackage{tikz}
\usepackage{graphicx}

\setbeamertemplate{page number in head/foot}[totalframenumber]

\usepackage{tcolorbox}
\tcbuselibrary{minted,breakable,xparse,skins}



\definecolor{bg}{gray}{0.95}
\DeclareTCBListing{mintedbox}{O{}m!O{}}{%
  breakable=true,
  listing engine=minted,
  listing only,
  minted language=#2,
  minted style=default,
  minted options={%
    linenos,
    gobble=0,
    breaklines=true,
    breakafter=,,
    fontsize=\small,
    numbersep=8pt,
    #1},
  boxsep=0pt,
  left skip=0pt,
  right skip=0pt,
  left=25pt,
  right=0pt,
  top=3pt,
  bottom=3pt,
  arc=5pt,
  leftrule=0pt,
  rightrule=0pt,
  bottomrule=2pt,
  toprule=2pt,
  colback=bg,
  colframe=orange!70,
  enhanced,
  overlay={%
    \begin{tcbclipinterior}
    \fill[orange!20!white] (frame.south west) rectangle ([xshift=20pt]frame.north west);
    \end{tcbclipinterior}},
  #3,
}
\lstset{
    language=C,
    basicstyle=\ttfamily\small,
    keywordstyle=\color{blue},
    stringstyle=\color{orange},
    commentstyle=\color{green!60!black},
    numbers=left,
    numberstyle=\tiny\color{gray},
    breaklines=true,
    showstringspaces=false,
}
%------------------------------------------------------------
%This block of code defines the information to appear in the
%Title page
\title %optional
{2.4.22}
\date{August, 2025}
%\subtitle{A short story}

\author % (optional)
{INDHIRESH S - EE25BTECH11027}



\begin{document}


\frame{\titlepage}
\begin{frame}{Question}
 Find the equation of a plane which bisects perpendicularly the line joining the points $A(2, 3, 4)$ and $B(4, 5, 8)$ at right angles.\\
\end{frame}
\begin{frame}{allowframebreaks}
\frametitle{Equation}

    \centering
    
    \label{tab:parameters}
  Let,
\begin{align}
    \Vec{A}=\myvec{2\\3\\4}\;\;and\;\;\Vec{B}=\myvec{4\\5\\8}
\end{align}
   
\end{frame}


\begin{frame}{Theoretical Solution}
Given that the plane is a perpendicular bisector to the line joining points A and B. Since it is a perpendicular bisector to the line joining points A and B , the midpoint of the line joining points A and B lies on the plane.\\
Let the midpoint of points A and B be C. Then
\begin{align}
norm({\Vec{C}-\Vec{A}})=norm({\Vec{C}-\Vec{B}})
\end{align}
\begin{align}
    \sqrt{(\Vec{C}-\Vec{A})^T(\Vec{C}-\Vec{A})}=\sqrt{(\Vec{C}-\Vec{B})^T(\Vec{C}-\Vec{B})}
\end{align}
\begin{align}
    (\Vec{C}-\Vec{A})^T(\Vec{C}-\Vec{A})=(\Vec{C}-\Vec{B})^T(\Vec{C}-\Vec{B})
\end{align}


\end{frame}
\begin{frame}
\frametitle{Theoretical Solution}
Let,
\begin{align}
    C=\myvec{x \\ y \\ z}
\end{align}
\begin{align}
    \brak{\myvec{x \\ y \\ z}-\myvec{2\\3\\4}}^T\brak{\myvec{x \\ y \\ z}-\myvec{2\\3\\4}}=\brak{\myvec{x \\ y \\ z}-\myvec{4\\5\\8}^T}\brak{\myvec{x \\ y \\ z}-\myvec{4\\5\\8}}
\end{align}
\begin{align}
    \myvec{x-2\\y-3\\z-4}^T\myvec{x-2\\y-3\\z-4}=\myvec{x-4\\y-5\\z-8}^T\myvec{x-4\\y-5\\z-8}
\end{align}


\end{frame}
\begin{frame}
\frametitle{Theoretical Solution}
\begin{align}
(x-2)^2+(y-3)^2+(z-4)^2=(x-4)^2+(y-5)^2+(z-8)^2
\end{align}
\begin{align}
    x^2+4-4x+y^2+9-6y+z^2+16-8z=x^2+16-8x+y^2+25-10y+z^2+64-16z
\end{align}
\begin{align}
   4x+4y+8z=76
\end{align}

\begin{align}
     x+y+2z=19
\end{align}


\end{frame}
\begin{frame}
\frametitle{Theoretical Solution}
Now the equation of plane is :
\begin{align}
     x+y+2z=19
\end{align}
In matrix form:
\begin{align}
     \myvec{1\\1\\2}^T\Vec{R}=19
\end{align}
Where \textbf{R} is the equation of the plane


\end{frame}




\begin{frame}[fragile]
    \frametitle{C Code - Midpoint formula }

    \begin{lstlisting}
#include<stdio.h>
void midpoint(float* out, float* A, float* B) {
    out[0] = (A[0] + B[0]) / 2.0f; // X-coordinate
    out[1] = (A[1] + B[1]) / 2.0f; // Y-coordinate
    out[2] = (A[2] + B[2]) / 2.0f; // Z-coordinate
}



    \end{lstlisting}
\end{frame}


\begin{frame}[fragile]
    \frametitle{Python Code}
    \begin{lstlisting}
import numpy as np
import ctypes
import matplotlib.pyplot as plt
from mpl_toolkits.mplot3d import Axes3D

# Load C library
libmid = ctypes.CDLL('./plane.so')

# Define arrays (float32)
A = np.array([2.0, 3.0, 4.0], dtype=np.float32)
B = np.array([4.0, 5.0, 8.0], dtype=np.float32)
M = np.zeros(3, dtype=np.float32)

# Set argtypes/restype for C function
libmid.midpoint.argtypes = [ctypes.POINTER(ctypes.c_float),
                            ctypes.POINTER(ctypes.c_float),
                            ctypes.POINTER(ctypes.c_float)]
libmid.midpoint.restype = None







    \end{lstlisting}
\end{frame}

\begin{frame}[fragile]
    \frametitle{Python Code}
    \begin{lstlisting}
# Call C function to compute midpoint
libmid.midpoint(M.ctypes.data_as(ctypes.POINTER(ctypes.c_float)),
                A.ctypes.data_as(ctypes.POINTER(ctypes.c_float)),
                B.ctypes.data_as(ctypes.POINTER(ctypes.c_float)))

print("Midpoint:", M)

# Prepare plane x + y + z = 10
xx, yy = np.meshgrid(np.linspace(0, 6, 20), np.linspace(0, 8, 20))
zz = 10 - xx - yy

# Plot
fig = plt.figure()
ax = fig.add_subplot(111, projection='3d')

# Plane
ax.plot_surface(xx, yy, zz, alpha=0.3, color='cyan')










    \end{lstlisting}
\end{frame}

\begin{frame}[fragile]
    \frametitle{Python Code}

    \begin{lstlisting}
# Points and line
ax.scatter(*A, color='red', s=60, label='A(2,3,4)')
ax.scatter(*B, color='green', s=60, label='B(4,5,8)')
ax.scatter(*M, color='purple', s=100, marker='*', label='M(3,4,6)')
ax.plot([A[0], B[0]],    # x coordinates
        [A[1], B[1]],    # y coordinates
        [A[2], B[2]],    # z coordinates
        color='blue', linewidth=2, label='Line AB')
ax.text(*A, 'A(2,3,4)', fontsize=9, color='red')
ax.text(*B, 'B(4,5,8)', fontsize=9, color='green')
ax.text(*M, 'M(3,4,6)', fontsize=9, color='purple')

ax.set_xlabel('X-axis')
ax.set_ylabel('Y-axis')
ax.set_zlabel('Z-axis')
ax.legend()



    \end{lstlisting}
\end{frame}
\begin{frame}[fragile]
    \frametitle{Python Code}

    \begin{lstlisting}
plt.title('Midpoint using C + Python')
plt.savefig("/media/indhiresh-s/New Volume/Matrix/ee1030-2025/ee25btech11027/MATGEO/2.4.22/figs/figure1.png")
plt.show()



    \end{lstlisting}
\end{frame}



\begin{frame}{Plot}
    \begin{center}
        \includegraphics[width=\columnwidth, height=0.8\textheight, keepaspectratio]{figs/figure1.png}
    \end{center}
\end{frame}




\end{document}
