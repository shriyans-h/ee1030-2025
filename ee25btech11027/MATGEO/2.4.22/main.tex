
\documentclass[journal]{IEEEtran}
\usepackage[a5paper, margin=10mm]{geometry}
%\usepackage{lmodern} % Ensure lmodern is loaded for pdflatex
\usepackage{tfrupee} % Include tfrupee package


\setlength{\headheight}{1cm} % Set the height of the header box
\setlength{\headsep}{0mm}     % Set the distance between the header box and the top of the text


%\usepackage[a5paper, top=10mm, bottom=10mm, left=10mm, right=10mm]{geometry}

%
\setlength{\intextsep}{10pt} % Space between text and floats

\makeindex


\usepackage{cite}
\usepackage{amsmath,amssymb,amsfonts,amsthm}
\usepackage{algorithmic}
\usepackage{graphicx}
\usepackage{textcomp}
\usepackage{xcolor}
\usepackage{txfonts}
\usepackage{listings}
\usepackage{enumitem}
\usepackage{mathtools}
\usepackage{gensymb}
\usepackage{comment}
\usepackage[breaklinks=true]{hyperref}
\usepackage{tkz-euclide} 
\usepackage{listings}
\usepackage{multicol}
\usepackage{xparse}
\usepackage{gvv}
%\def\inputGnumericTable{}                                 
\usepackage[latin1]{inputenc}                                
\usepackage{color}                                            
\usepackage{array}                                            
\usepackage{longtable}                                       
\usepackage{calc}                                             
\usepackage{multirow}                                         
\usepackage{hhline}                                           
\usepackage{ifthen}                                               
\usepackage{lscape}
\usepackage{tabularx}
\usepackage{array}
\usepackage{float}
\usepackage{ar}
\usepackage[version=4]{mhchem}


\newtheorem{theorem}{Theorem}[section]
\newtheorem{problem}{Problem}
\newtheorem{proposition}{Proposition}[section]
\newtheorem{lemma}{Lemma}[section]
\newtheorem{corollary}[theorem]{Corollary}
\newtheorem{example}{Example}[section]
\newtheorem{definition}[problem]{Definition}
\newcommand{\BEQA}{\begin{eqnarray}}
\newcommand{\EEQA}{\end{eqnarray}}

\theoremstyle{remark}


\begin{document}
\bibliographystyle{IEEEtran}
\onecolumn

\title{2.4.22}
\author{INDHIRESH S- EE25BTECH11027}
\maketitle


\renewcommand{\thefigure}{\theenumi}
\renewcommand{\thetable}{\theenumi}

\textbf{Question}  Find the equation of a plane which bisects perpendicularly the line joining the points $A(2, 3, 4)$ and $B(4, 5, 8)$ at right angles.\\
\textbf{Solution}:\\
Let us solve the given equation theoretically and then verify the solution computationally. \\
Let,
\begin{align}
    \Vec{A}=\myvec{2\\3\\4}\;\;and\;\;\Vec{B}=\myvec{4\\5\\8}
\end{align}
Given that the plane is a perpendicular bisector to the line joining points A and B. Since it is a perpendicular bisector to the line joining points A and B , the midpoint of the line joining points A and B lies on the plane.\\
Let the midpoint of points A and B be C. Then
\begin{align}
norm({\Vec{C}-\Vec{A}})=norm({\Vec{C}-\Vec{B}})
\end{align}
\begin{align}
    \sqrt{(\Vec{C}-\Vec{A})^T(\Vec{C}-\Vec{A})}=\sqrt{(\Vec{C}-\Vec{B})^T(\Vec{C}-\Vec{B})}
\end{align}
\begin{align}
    (\Vec{C}-\Vec{A})^T(\Vec{C}-\Vec{A})=(\Vec{C}-\Vec{B})^T(\Vec{C}-\Vec{B})
\end{align}
Let,
\begin{align}
    C=\myvec{x \\ y \\ z}
\end{align}
\begin{align}
    \brak{\myvec{x \\ y \\ z}-\myvec{2\\3\\4}}^T\brak{\myvec{x \\ y \\ z}-\myvec{2\\3\\4}}=\brak{\myvec{x \\ y \\ z}-\myvec{4\\5\\8}^T}\brak{\myvec{x \\ y \\ z}-\myvec{4\\5\\8}}
\end{align}
\begin{align}
    \myvec{x-2\\y-3\\z-4}^T\myvec{x-2\\y-3\\z-4}=\myvec{x-4\\y-5\\z-8}^T\myvec{x-4\\y-5\\z-8}
\end{align}
\begin{align}
(x-2)^2+(y-3)^2+(z-4)^2=(x-4)^2+(y-5)^2+(z-8)^2
\end{align}
\begin{align}
    x^2+4-4x+y^2+9-6y+z^2+16-8z=x^2+16-8x+y^2+25-10y+z^2+64-16z
\end{align}
\begin{align}
   4x+4y+8z=76
\end{align}

\begin{align}
     x+y+2z=19
\end{align}

Now the equation of plane is :
\begin{align}
     x+y+2z=19
\end{align}
In matrix form:
\begin{align}
     \myvec{1\\1\\2}^T\Vec{R}=19
\end{align}
Where \textbf{R} is the equation of the plane\\\\



From the figure it is clearly verified that the theoretical solution matches with the computational solution.\\
\begin{figure}[h]
    \centering
    \includegraphics[height=0.5\textheight, keepaspectratio]{figs/figure1.png}
    \label{figure_1}
\end{figure}

\end{document}
\end{document}