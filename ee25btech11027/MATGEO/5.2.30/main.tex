\documentclass[journal]{IEEEtran}
\usepackage[a5paper, margin=10mm]{geometry}
%\usepackage{lmodern} % Ensure lmodern is loaded for pdflatex
\usepackage{tfrupee} % Include tfrupee package


\setlength{\headheight}{1cm} % Set the height of the header box
\setlength{\headsep}{0mm}     % Set the distance between the header box and the top of the text


%\usepackage[a5paper, top=10mm, bottom=10mm, left=10mm, right=10mm]{geometry}

%
\setlength{\intextsep}{10pt} % Space between text and floats

\makeindex


\usepackage{cite}
\usepackage{amsmath,amssymb,amsfonts,amsthm}
\usepackage{algorithmic}
\usepackage{graphicx}
\usepackage{textcomp}
\usepackage{xcolor}
\usepackage{txfonts}
\usepackage{listings}
\usepackage{enumitem}
\usepackage{mathtools}
\usepackage{gensymb}
\usepackage{comment}
\usepackage[breaklinks=true]{hyperref}
\usepackage{tkz-euclide} 
\usepackage{listings}
\usepackage{multicol}
\usepackage{xparse}
\usepackage{gvv}
%\def\inputGnumericTable{}                                 
\usepackage[latin1]{inputenc}                                
\usepackage{color}                                            
\usepackage{array}                                            
\usepackage{longtable}                                       
\usepackage{calc}                                             
\usepackage{multirow}                                         
\usepackage{hhline}                                           
\usepackage{ifthen}                                               
\usepackage{lscape}
\usepackage{tabularx}
\usepackage{array}
\usepackage{float}
\usepackage{ar}
\usepackage[version=4]{mhchem}


\newtheorem{theorem}{Theorem}[section]
\newtheorem{problem}{Problem}
\newtheorem{proposition}{Proposition}[section]
\newtheorem{lemma}{Lemma}[section]
\newtheorem{corollary}[theorem]{Corollary}
\newtheorem{example}{Example}[section]
\newtheorem{definition}[problem]{Definition}
\newcommand{\BEQA}{\begin{eqnarray}}
\newcommand{\EEQA}{\end{eqnarray}}

\theoremstyle{remark}


\begin{document}
\bibliographystyle{IEEEtran}
\onecolumn

\title{5.2.30}
\author{INDHIRESH S- EE25BTECH11027}
\maketitle


\renewcommand{\thefigure}{\theenumi}
\renewcommand{\thetable}{\theenumi}

\textbf{Question}. Solve the following system of linear equations.
\begin{align*}
    7x - 15y = 2\\
    x + 2y = 3
\end{align*}
\textbf{Solution}:\\
Let us solve the given equation theoretically and then verify the solution computationally. \\
The given equation can be written as:
\begin{align}
   \myvec{7&-15\\1&2}\Vec{x}=\myvec{2\\3}
   \end{align}


\begin{align}
   \augvec{2}{1}{7&-15&2\\1&2&3}\xleftrightarrow{R_2\longleftarrow R_2-\frac{1}{7}R_1}\augvec{2}{1}{7&-15&2\\0&\frac{29}{7}&\frac{19}{7}}
\end{align}

\begin{align}
   \augvec{2}{1}{7&-15&2\\0&\frac{29}{7}&\frac{19}{7}}\xleftrightarrow[R_1\longleftarrow\frac{1}{7}R_1]{R_2\longleftarrow\frac{7}{29}R_2} \augvec{2}{1}{1&\frac{-15}{7}&\frac{2}{7}\\0&1&\frac{19}{29}}
\end{align}

\begin{align}
   \augvec{2}{1}{1&\frac{-15}{7}&\frac{2}{7}\\0&1&\frac{19}{29}}\xleftrightarrow{R_1\longleftarrow R_1+\frac{15}{7}}\augvec{2}{1}{1&0&\frac{49}{29}\\0&1&\frac{19}{29}}
\end{align}

From this we can say that:
\begin{align}
    \Vec{x}=\myvec{\frac{49}{29}\\\\ \frac{19}{29}}
\end{align}
From the figure it is clearly verified that the theoretical solution matches with the computational solution.\\
\begin{figure}[h]
    \centering
    \includegraphics[height=0.5\textheight, keepaspectratio]{figs/figure1.png}
    \label{figure_1}
\end{figure}

\end{document}