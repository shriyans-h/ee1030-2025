

\let\negmedspace\undefined
\let\negthickspace\undefined
\documentclass[journal]{IEEEtran}
\setlength{\headheight}{1cm} % Set the height of the header box
\setlength{\headsep}{0mm}     % Set the distance between the header box and the top of the text
\usepackage[a4paper,margin=10mm, onecolumn]{geometry}
\usepackage{gvv-book}
\usepackage{gvv}
\usepackage{cite}
\usepackage{amsmath,amssymb,amsfonts,amsthm}
\usepackage{algorithmic}
\usepackage{graphicx}
\usepackage{textcomp}
\usepackage{xcolor}
\usepackage{txfonts}
\usepackage{listings}
\usepackage{enumitem}
\usepackage{mathtools}
\usepackage{gensymb}
\usepackage{comment}
\usepackage[breaklinks=true]{hyperref}
\usepackage{tkz-euclide}
\usepackage{listings}
\def\inputGnumericTable{}
\usepackage[latin1]{inputenc}
\usepackage{color}
\usepackage{array}
\usepackage{longtable}
\usepackage{calc}
\usepackage{multirow}
\usepackage{hhline}
\usepackage{ifthen}
\usepackage{lscape}
\usepackage{circuitikz}
\tikzstyle{block} = [rectangle, draw, fill=blue!20,
    text width=4em, text centered, rounded corners, minimum height=3em]
\tikzstyle{sum} = [draw, fill=blue!10, circle, minimum size=1cm, node distance=1.5cm]
\tikzstyle{input} = [coordinate]
\tikzstyle{output} = [coordinate]
\begin{document}
\bibliographystyle{IEEEtran}
\vspace{3cm}
\title{2.10.2}
\author{AI25BTECH11013-Gautham}
\maketitle
% \newpage
% \bigskip
{\let\newpage\relax\maketitle}
\renewcommand{\thefigure}{\theenumi}
\renewcommand{\thetable}{\theenumi}
\setlength{\intextsep}{10pt} % Space between text and floats
\numberwithin{equation}{enumi}
\numberwithin{figure}{enumi}
\renewcommand{\thetable}{\theenumi}
\textbf{Question}:\\
Let $\Vec{A}$, $\Vec{B}$, and $\Vec{C}$ be vectors of lengths 3, 4, and 5 respectively such that 
$\Vec{A} \perp \Vec{B} + \Vec{C}$,
$\Vec{B} \perp \Vec{C} + \Vec{A}$, and
$\Vec{C} \perp \Vec{A} + \Vec{B}$. Find the length of the vector $\Vec{A} + \Vec{B} + \Vec{C}$.\\
\solution \\

Let the Gram matrix $G$ for the vectors $\Vec{A}, \Vec{B}, \Vec{C}$ be:
\begin{align}
G = \myvec{
\Vec{A}^T \Vec{A} & \Vec{A}^T \Vec{B} & \Vec{A}^T \Vec{C} \\
\Vec{B}^T \Vec{A} & \Vec{B}^T \Vec{B} & \Vec{B}^T \Vec{C} \\
\Vec{C}^T \Vec{A} & \Vec{C}^T \Vec{B} & \Vec{C}^T \Vec{C}
}
= \myvec{
9 & a & b \\
a & 16 & c \\
b & c & 25
}
\end{align}
where $a = \Vec{A}^T \Vec{B}$, $b = \Vec{A}^T \Vec{C}$, and $c = \Vec{B}^T \Vec{C}$.

Given the orthogonality conditions:
\begin{align}
\Vec{A} \perp \Vec{B} + \Vec{C} &\implies \Vec{A}^T (\Vec{B} + \Vec{C}) = 0 \implies a + b = 0, \\
\Vec{B} \perp \Vec{C} + \Vec{A} &\implies \Vec{B}^T (\Vec{C} + \Vec{A}) = 0 \implies c + a = 0, \\
\Vec{C} \perp \Vec{A} + \Vec{B} &\implies \Vec{C}^T (\Vec{A} + \Vec{B}) = 0 \implies b + c = 0.
\end{align}
This system can be written as:
\begin{align}
a + b = 0 \\
c + a = 0 \\
b + c = 0.
\end{align}
In matrix form:
\begin{align}
\myvec{
1 & 1 & 0 \\
1 & 0 & 1 \\
0 & 1 & 1
}
\myvec{
a \\ b \\ c
}
= \myvec{
0 \\ 0 \\ 0
}
\end{align}

Convert the coefficient matrix to upper triangular form by row operations:
\begin{align}
\myvec{
1 & 1 & 0 \\
1 & 0 & 1 \\
0 & 1 & 1
}
\overset{R_2 \to R_2 - R_1}{\longrightarrow}
\myvec{
1 & 1 & 0 \\
0 & -1 & 1 \\
0 & 1 & 1
}
\overset{R_3 \to R_3 + R_2}{\longrightarrow}
\myvec{
1 & 1 & 0 \\
0 & -1 & 1 \\
0 & 0 & 2
}
\end{align}

From the last row:
\begin{align}
2c = 0 \implies c = 0
\end{align}
From the second row:
\begin{align}
- b + c = 0 \implies b = 0
\end{align}
From the first row:
\begin{align}
a + b = 0 \implies a = 0
\end{align}

Thus, the Gram matrix is:
\begin{align}
G = \myvec{
9 & 0 & 0 \\
0 & 16 & 0 \\
0 & 0 & 25
}
\end{align}
Let $\Vec{u}=\myvec{1 \\ 1 \\ 1}$\\
Now, the squared length of $\Vec{A} + \Vec{B} + \Vec{C}$ is:
\begin{align}
||\Vec{A} + \Vec{B} + \Vec{C}||^2 = \Vec{u}^T \Vec{G} \Vec{u}\\
||\Vec{A} + \Vec{B} + \Vec{C}||^2 = 50
\end{align}
Therefore,
\begin{align}
||\Vec{A} + \Vec{B} + \Vec{C}|| = \sqrt{50} = 5 \sqrt{2}
\end{align}
\end{document}
