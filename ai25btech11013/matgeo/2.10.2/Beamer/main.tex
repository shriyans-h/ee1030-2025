\documentclass{beamer}
\usepackage[utf8]{inputenc}

\usetheme{Madrid}
\usecolortheme{default}
\usepackage{amsmath,amssymb,amsfonts,amsthm}
\usepackage{txfonts}
\usepackage{tkz-euclide}
\usepackage{listings}
\usepackage{adjustbox}
\usepackage{array}
\usepackage{tabularx}
\usepackage{gvv}
\usepackage{lmodern}
\usepackage{circuitikz}
\usepackage{tikz}
\usepackage{graphicx}

\setbeamertemplate{page number in head/foot}[totalframenumber]

\usepackage{tcolorbox}
\tcbuselibrary{minted,breakable,xparse,skins}



\definecolor{bg}{gray}{0.95}
\DeclareTCBListing{mintedbox}{O{}m!O{}}{%
  breakable=true,
  listing engine=minted,
  listing only,
  minted language=#2,
  minted style=default,
  minted options={%
    linenos,
    gobble=0,
    breaklines=true,
    breakafter=,,
    fontsize=\small,
    numbersep=8pt,
    #1},
  boxsep=0pt,
  left skip=0pt,
  right skip=0pt,
  left=25pt,
  right=0pt,
  top=3pt,
  bottom=3pt,
  arc=5pt,
  leftrule=0pt,
  rightrule=0pt,
  bottomrule=2pt,
  toprule=2pt,
  colback=bg,
  colframe=orange!70,
  enhanced,
  overlay={%
    \begin{tcbclipinterior}
    \fill[orange!20!white] (frame.south west) rectangle ([xshift=20pt]frame.north west);
    \end{tcbclipinterior}},
  #3,
}
\lstset{
    language=C,
    basicstyle=\ttfamily\small,
    keywordstyle=\color{blue},
    stringstyle=\color{orange},
    commentstyle=\color{green!60!black},
    numbers=left,
    numberstyle=\tiny\color{gray},
    breaklines=true,
    showstringspaces=false,
}
%------------------------------------------------------------
%This block of code defines the information to appear in the
%Title page
\title %optional
{2.10.2}
%\subtitle{A short story}

\author % (optional)
{Gautham-AI25BTECH11013}



\begin{document}


\frame{\titlepage}
\begin{frame}{Question}
Let $\Vec{A}$, $\Vec{B}$, and $\Vec{C}$ be vectors of lengths 3, 4, and 5 respectively such that
$\Vec{A} \perp \Vec{B} + \Vec{C}$, 
$\Vec{B} \perp \Vec{C} + \Vec{A}$, and
$\Vec{C} \perp \Vec{A} + \Vec{B}$. Find the length of the vector $\Vec{A} + \Vec{B} + \Vec{C}$.\\
\end{frame}
\begin{frame}{Theoretical Solution}
Given the orthogonality conditions:
\begin{align}
\Vec{A} \perp \Vec{B} + \Vec{C} \implies \Vec{A}^T (\Vec{B} + \Vec{C}) = 0 \\
\Vec{B} \perp \Vec{C} + \Vec{A} \implies \Vec{B}^T (\Vec{C} + \Vec{A}) = 0 \\
\Vec{C} \perp \Vec{A} + \Vec{B} \implies \Vec{C}^T (\Vec{A} + \Vec{B}) = 0
\end{align}

Expanding, we get:
\begin{align}
\Vec{A}^T \Vec{B} + \Vec{A}^T \Vec{C} &= 0 \\
\Vec{B}^T \Vec{C} + \Vec{B}^T \Vec{A} &= 0  \\
\Vec{C}^T \Vec{A} + \Vec{C}^T \Vec{B} &= 0 
\end{align}
\end{frame}
\begin{frame}{Theoretical Solution}
   Adding :
\begin{align}
(\Vec{A}^T \Vec{B} + \Vec{A}^T \Vec{C}) + (\Vec{B}^T \Vec{C} + \Vec{B}^T \Vec{A}) + (\Vec{C}^T \Vec{A} + \Vec{C}^T \Vec{B}) = 0
\end{align}

Grouping like terms and noting dot products are symmetric:
\begin{align}
2 \big( \Vec{A}^T \Vec{B} + \Vec{B}^T \Vec{C} + \Vec{C}^T \Vec{A} \big) = 0 \implies \Vec{A}^T \Vec{B} + \Vec{B}^T \Vec{C} + \Vec{C}^T \Vec{A} = 0
\end{align}
\end{frame}
\begin{frame}{Theoretical Solution}
 
Now compute the squared length of $\Vec{A} + \Vec{B} + \Vec{C}$:
\begin{align}
||\Vec{A} + \Vec{B} + \Vec{C}||^2 &= (\Vec{A} + \Vec{B} + \Vec{C})^T (\Vec{A} + \Vec{B} + \Vec{C}) \\
&= \Vec{A}^T \Vec{A} + \Vec{B}^T \Vec{B} + \Vec{C}^T \Vec{C} + 2(\Vec{A}^T \Vec{B} + \Vec{B}^T \Vec{C} + \Vec{C}^T \Vec{A}) \\
&=||\Vec{A}||^2+||\Vec{B}||^2+||\Vec{C}||^2 + 2(\Vec{A}^T \Vec{B} + \Vec{B}^T \Vec{C} + \Vec{C}^T \Vec{A}) \\
&= 3^2 + 4^2 + 5^2 + 2 \times 0 \\
&= 9 + 16 + 25 \\
&= 50
\end{align}

Therefore,
\begin{align}
||\Vec{A} + \Vec{B} + \Vec{C}||= \sqrt{50} = 5 \sqrt{2}
\end{align}
\end{frame}
\begin{frame}[fragile]
\frametitle{Python Code}
   \begin{lstlisting}
import numpy as np
import numpy.linalg as la
import math
a=3
b=4
c=5
#x=a.b,y=b.c,z=c.a
#x+y=0,y+z=0,x+z=0
B=np.array([0,0,0])
A=np.array([[1,1,0],[0,1,1],[1,0,1]])
X=la.solve(A,B)
d=a*a+b*b+c*c+2*np.sum(X)
print(math.sqrt(d))
   \end{lstlisting}
\end{frame}


\end{document}