\documentclass{beamer}
\usepackage[utf8]{inputenc}

\usetheme{Madrid}
\usecolortheme{default}
\usepackage{amsmath,amssymb,amsfonts,amsthm}
\usepackage{txfonts}
\usepackage{tkz-euclide}
\usepackage{listings}
\usepackage{adjustbox}
\usepackage{array}
\usepackage{tabularx}
\usepackage{gvv}
\usepackage{lmodern}
\usepackage{circuitikz}
\usepackage{tikz}
\usepackage{graphicx}

\setbeamertemplate{page number in head/foot}[totalframenumber]

\usepackage{tcolorbox}
\tcbuselibrary{minted,breakable,xparse,skins}

\definecolor{bg}{gray}{0.95}
\DeclareTCBListing{mintedbox}{O{}m!O{}}{%
	breakable=true,
	listing engine=minted,
	listing only,
	minted language=#2,
	minted style=default,
	minted options={%
		linenos,
		gobble=0,
		breaklines=true,
		breakafter=,,
		fontsize=\small,
		numbersep=8pt,
		#1},
	boxsep=0pt,
	left skip=0pt,
	right skip=0pt,
	left=25pt,
	right=0pt,
	top=3pt,
	bottom=3pt,
	arc=5pt,
	leftrule=0pt,
	rightrule=0pt,
	bottomrule=2pt,
	toprule=2pt,
	colback=bg,
	colframe=orange!70,
	enhanced,
	overlay={%
		\begin{tcbclipinterior}
			\fill[orange!20!white] (frame.south west) rectangle ([xshift=20pt]frame.north west);
	\end{tcbclipinterior}},
	#3,
}
\lstset{
	language=C,
	basicstyle=\ttfamily\small,
	keywordstyle=\color{blue},
	stringstyle=\color{orange},
	commentstyle=\color{green!60!black},
	numbers=left,
	numberstyle=\tiny\color{gray},
	breaklines=true,
	showstringspaces=false,
}

\title{5.4.42}
\date{October 2, 2025}
\author{Revanth Siva Kumar - EE25BTECH11048}

\begin{document}

\frame{\titlepage}

\begin{frame}{Question}
Using elementary transformations, find the inverse of the following matrix:
\begin{align*}
\myvec{1 & -1 & 2\\0 & 2 & -3\\3 & -2 & 4}
\end{align*}
\end{frame}

\begin{frame}{Theoretical Solution}
We solve using Gauss-Jordan elimination.
\begin{align}
    \augvec{3}{3}{1 & -1 & 2 & 1 & 0 & 0\\ 0 & 2 & -3 & 0 & 1 & 0\\ 3 & -2 & 4 & 0 & 0 & 1}\\
    &\xleftrightarrow{\,R_3 \gets R_3 - 3R_1}
    \augvec{3}{3}{1 & -1 & 2 & 1 & 0 & 0\\ 0 & 2 & -3 & 0 & 1 & 0\\ 0 & 1 & -2 & -3 & 0 & 1} \\[1em]
    &\xleftrightarrow{\,R_2 \gets \frac{1}{2}R_2}
    \augvec{3}{3}{1 & -1 & 2 & 1 & 0 & 0\\ 0 & 1 & -3/2 & 0 & 1/2 & 0\\ 0 & 1 & -2 & -3 & 0 & 1} 
    \end{align}
    \end{frame}
\begin{frame}{Theoretical Solution}
\begin{align}
    &\xleftrightarrow{\,R_3 \gets R_3 - R_2}
    \augvec{3}{3}{1 & -1 & 2 & 1 & 0 & 0\\ 0 & 1 & -3/2 & 0 & 1/2 & 0\\ 0 & 0 & -1/2 & -3 & -1/2 & 1} \\[1em]
    &\xleftrightarrow{\,R_3 \gets -2R_3}
    \augvec{3}{3}{1 & -1 & 2 & 1 & 0 & 0\\ 0 & 1 & -3/2 & 0 & 1/2 & 0\\ 0 & 0 & 1 & 6 & 1 & -2} \\[1em]
    &\xleftrightarrow[\,R_1 \gets R_1 - 2R_3]{\,R_2 \gets R_2 + 3/2 R_3}
    \augvec{3}{3}{1 & -1 & 0 & -11 & -2 & 4\\ 0 & 1 & 0 & 9 & 2 & -3\\ 0 & 0 & 1 & 6 & 1 & -2} \\[1em]
    &\xleftrightarrow{\,R_1 \gets R_1 + R_2}
    \augvec{3}{3}{1 & 0 & 0 & -2 & 0 & 1\\ 0 & 1 & 0 & 9 & 2 & -3\\ 0 & 0 & 1 & 6 & 1 & -2}
\end{align}
\end{frame}
\begin{frame}{Conclusion}
\begin{align*}
    \therefore \text{Inverse of the given Matrix:}\myvec{-2 & 0 & 1\\9 & 2 & -3\\6 & 1 & -2}
\end{align*}

\end{frame}

\begin{frame}[fragile]
\frametitle{C Code }
\begin{lstlisting}
#include <stdio.h>
#define N 3
void inverse(double A[N][N], double inv[N][N]) {
	double aug[N][2*N];
	for (int i = 0; i < N; i++) {
		for (int j = 0; j < N; j++) {
			aug[i][j] = A[i][j];
			aug[i][j+N] = (i==j)?1:0;
		}
	}
\end{lstlisting}
\end{frame}

\begin{frame}[fragile]
\frametitle{C Code}
\begin{lstlisting}
// Gauss-Jordan elimination
for (int i=0;i<N;i++){
	double pivot = aug[i][i];
	for(int j=0;j<2*N;j++) aug[i][j]/=pivot;
	for(int k=0;k<N;k++){
		if(k!=i){
			double factor = aug[k][i];
			for(int j=0;j<2*N;j++)
				aug[k][j]-=factor*aug[i][j];
		}
	}
}
// Extract inverse
for(int i=0;i<N;i++)
	for(int j=0;j<N;j++)
		inv[i][j]=aug[i][j+N];
}
\end{lstlisting}
\end{frame}

\begin{frame}[fragile]
\frametitle{Python Code using shared output}
\begin{lstlisting}
import sympy as sp

A = sp.Matrix([[1, -1, 2], [0, 2, -3], [3, -2, 4]])
A_inv = A.inv()
sp.pprint(A_inv)
\end{lstlisting}
\end{frame}

\begin{frame}[fragile]
\frametitle{Python:plot.py}
\begin{lstlisting}
import matplotlib.pyplot as plt
import numpy as np

# Inverse matrix from 5.4.42
A_inv = np.array([
    [-2, 0, 1],
    [9, 2, -3],
    [6, 1, -2]
])

fig, ax = plt.subplots()
cax = ax.matshow(A_inv, cmap='coolwarm')
fig.colorbar(cax)
for (i, j), val in np.ndenumerate(A_inv):
    ax.text(j, i, f'{val}', ha='center', va='center', color='black')
ax.set_title('Inverse of Matrix 5.4.42')
plt.show()

\end{lstlisting}
\end{frame}

\end{document}
