\let\negmedspace\undefined
\let\negthickspace\undefined
\documentclass[journal]{IEEEtran}
\usepackage[a5paper, margin=10mm, onecolumn]{geometry}
\usepackage{tfrupee}

\setlength{\headheight}{1cm} 
\setlength{\headsep}{0mm}     

\usepackage{gvv-book}
\usepackage{gvv}
\usepackage{cite}
\usepackage{amsmath,amssymb,amsfonts,amsthm}
\usepackage{algorithmic}
\usepackage{graphicx}
\usepackage{textcomp}
\usepackage{xcolor}
\usepackage{txfonts}
\usepackage{listings}
\usepackage{enumitem}
\usepackage{mathtools}
\usepackage{gensymb}
\usepackage[breaklinks=true]{hyperref}
\usepackage{tkz-euclide} 
\usepackage{longtable}
\usepackage{multirow}
\usepackage{circuitikz}

\begin{document}
	
	\bibliographystyle{IEEEtran}
	\vspace{3cm}
	
	\title{5.4.42}
	\author{EE25BTECH11048 - Revanth Siva Kumar}
	\maketitle
	{\let\newpage\relax\maketitle}
	
	\renewcommand{\thefigure}{\theenumi}
	\renewcommand{\thetable}{\theenumi}
	\setlength{\intextsep}{10pt}
	
	\numberwithin{equation}{enumi}
	\numberwithin{figure}{enumi}
	\renewcommand{\thetable}{\theenumi}
	
	\textbf{Question}:\\
	Using elementary transformations, find the inverse of the following matrix. 
	\begin{align*}
		\myvec{1 & -1 & 2\\0 & 2 & -3\\3 & -2 & 4}
	\end{align*}
	
	\textbf{Solution}: \\
	We solve using Gauss-Jordan elimination.
	\begin{align}
		\augvec{3}{3}{1 & -1 & 2 & 1 & 0 & 0\\ 0 & 2 & -3 & 0 & 1 & 0\\ 3 & -2 & 4 & 0 & 0 & 1}
		&\xleftrightarrow{\,R_3 \gets R_3 - 3R_1}
		\augvec{3}{3}{1 & -1 & 2 & 1 & 0 & 0\\ 0 & 2 & -3 & 0 & 1 & 0\\ 0 & 1 & -2 & -3 & 0 & 1} \\[1em]
		&\xleftrightarrow{\,R_2 \gets \frac{1}{2}R_2}
		\augvec{3}{3}{1 & -1 & 2 & 1 & 0 & 0\\ 0 & 1 & -3/2 & 0 & 1/2 & 0\\ 0 & 1 & -2 & -3 & 0 & 1} \\[1em]
		&\xleftrightarrow{\,R_3 \gets R_3 - R_2}
		\augvec{3}{3}{1 & -1 & 2 & 1 & 0 & 0\\ 0 & 1 & -3/2 & 0 & 1/2 & 0\\ 0 & 0 & -1/2 & -3 & -1/2 & 1} \\[1em]
		&\xleftrightarrow{\,R_3 \gets -2R_3}
		\augvec{3}{3}{1 & -1 & 2 & 1 & 0 & 0\\ 0 & 1 & -3/2 & 0 & 1/2 & 0\\ 0 & 0 & 1 & 6 & 1 & -2} \\[1em]
		&\xleftrightarrow[\,R_1 \gets R_1 - 2R_3]{\,R_2 \gets R_2 + 3/2 R_3}
		\augvec{3}{3}{1 & -1 & 0 & -11 & -2 & 4\\ 0 & 1 & 0 & 9 & 2 & -3\\ 0 & 0 & 1 & 6 & 1 & -2} \\[1em]
		&\xleftrightarrow{\,R_1 \gets R_1 + R_2}
		\augvec{3}{3}{1 & 0 & 0 & -2 & 0 & 1\\ 0 & 1 & 0 & 9 & 2 & -3\\ 0 & 0 & 1 & 6 & 1 & -2}
	\end{align}
	
	\begin{align}
		\therefore \text{Inverse of the given Matrix:}\myvec{-2 & 0 & 1\\9 & 2 & -3\\6 & 1 & -2}
	\end{align}
	
\end{document}

