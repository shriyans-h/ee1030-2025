\documentclass{beamer}
\mode<presentation>
\usepackage{amsmath,amssymb,mathtools}
\usepackage{textcomp}
\usepackage{gensymb}
\usepackage{adjustbox}
\usepackage{subcaption}
\usepackage{enumitem}
\usepackage{multicol}
\usepackage{listings}
\usepackage{url}
\usepackage{graphicx}
\def\UrlBreaks{\do/\do-}

\usetheme{Boadilla}
\usecolortheme{lily}
\setbeamertemplate{footline}{
\leavevmode
\hbox{
\begin{beamercolorbox}[wd=\paperwidth,ht=2ex,dp=1ex,right]{author in head/foot}
\insertframenumber{} / \inserttotalframenumber\hspace*{2ex}
\end{beamercolorbox}}
\vskip0pt
}
\setbeamertemplate{navigation symbols}{}

\lstset{
frame=single,
breaklines=true,
columns=fullflexible,
basicstyle=\ttfamily\tiny
}

\numberwithin{equation}{section}

\newcommand{\myvec}[1]{\ensuremath{\begin{pmatrix}#1\end{pmatrix}}}
\let\vec\mathbf

\title{Matgeo Presentation - Problem 4.7.45}
\author{Revanth Siva Kumar.D -- EE25BTECH11048}

\begin{document}

\begin{frame}
\titlepage
\end{frame}

\begin{frame}{QUESTION}
For any three vectors $\vec{a},\vec{b},\vec{c}$, prove or disprove
\begin{align*}
    (\vec{a}-\vec{b})\cdot \big((\vec{b}-\vec{c})\times(\vec{c}-\vec{a})\big) = 2\vec{a}\cdot(\vec{b}\times \vec{c}).
\end{align*}\end{frame}

\begin{frame}{Solution:}
We write the scalar triple product in determinant form:
\begin{align}
(\vec{a}-\vec{b})\cdot \big((\vec{b}-\vec{c})\times(\vec{c}-\vec{a})\big)
= \det\myvec{ (\vec{a}-\vec{b})^T \\ (\vec{b}-\vec{c})^T \\ (\vec{c}-\vec{a})^T }.
\end{align}

Now observe that
\begin{align}
(\vec{a}-\vec{b}) + (\vec{b}-\vec{c}) + (\vec{c}-\vec{a}) = \vec{0}.
\end{align}
Thus the three rows of the determinant are linearly dependent. From matrix theory, the determinant of a matrix with linearly dependent rows is zero. Hence
\begin{align}
(\vec{a}-\vec{b})\cdot \big((\vec{b}-\vec{c})\times(\vec{c}-\vec{a})\big) = 0.
\end{align}
\end{frame}
\begin{frame}{Solution:}
On the other hand, the right-hand side is
\begin{align}
2\vec{a}\cdot(\vec{b}\times\vec{c})
= 2\det\myvec{\vec{a}^T \\ \vec{b}^T \\ \vec{c}^T},
\end{align}
which is not identically zero for arbitrary $\vec{a},\vec{b},\vec{c}$.
\end{frame}
\begin{frame}{Conclusion}The given statement is \textbf{false}. 

The left-hand side  $(\vec{a}-\vec{b})\cdot \big((\vec{b}-\vec{c})\times(\vec{c}-\vec{a})\big)$is  always zero, 
while the right-hand side $2\vec{a}\cdot(\vec{b}\times \vec{c})$ can be nonzero.
\end{frame}
\begin{frame}[fragile]{C Code}
\begin{lstlisting}
#include <stdio.h>

// Cross product u × v
void cross(double u[3], double v[3], double result[3]) {
    result[0] = u[1]*v[2] - u[2]*v[1];
    result[1] = u[2]*v[0] - u[0]*v[2];
    result[2] = u[0]*v[1] - u[1]*v[0];
}

// Dot product u · v
double dot(double u[3], double v[3]) {
    return u[0]*v[0] + u[1]*v[1] + u[2]*v[2];
}
\end{lstlisting}
\end{frame}
\begin{frame}[fragile]{C Code}
\begin{lstlisting}
// Compute lhs (always 0) and rhs
void compute(double a[3], double b[3], double c[3], double* lhs, double* rhs) {
    double cross_b_c[3];

    // b × c
    cross(b, c, cross_b_c);

    // LHS always 0 due to linear dependence
    *lhs = 0.0;

    // RHS = 2a · (b × c)
    *rhs = 2 * dot(a, cross_b_c);
}
\end{lstlisting}
\end{frame}
\begin{frame}[fragile]{callc.py Python code}
\begin{lstlisting}
import ctypes

# Load shared library
lib = ctypes.CDLL("./points.so")

# Argument and return types
lib.compute.argtypes = [
    ctypes.POINTER(ctypes.c_double),  # a
    ctypes.POINTER(ctypes.c_double),  # b
    ctypes.POINTER(ctypes.c_double),  # c
    ctypes.POINTER(ctypes.c_double),  # lhs
    ctypes.POINTER(ctypes.c_double)   # rhs
]

def compute(a, b, c):
    a_arr = (ctypes.c_double*3)(*a)
    b_arr = (ctypes.c_double*3)(*b)
    c_arr = (ctypes.c_double*3)(*c)
    lhs = ctypes.c_double()
    rhs = ctypes.c_double()
    lib.compute(a_arr, b_arr, c_arr, ctypes.byref(lhs), ctypes.byref(rhs))
    return lhs.value, rhs.value
\end{lstlisting}
\end{frame}
\begin{frame}[fragile]{callc.py Python code}
\begin{lstlisting}
if __name__ == "__main__":
    # Input vectors
    a = list(map(float, input("Enter vector a (3 values): ").split()))
    b = list(map(float, input("Enter vector b (3 values): ").split()))
    c = list(map(float, input("Enter vector c (3 values): ").split()))

    lhs, rhs = compute(a, b, c)

    print(f"LHS = {lhs:.4f}")
    print(f"RHS = {rhs:.4f}")

    if abs(lhs - rhs) < 1e-6:
        print(" The equality holds (both sides are zero).")
    else:
        print(" The equality does NOT hold.")

\end{lstlisting}
\end{frame}
\end{document}
