\let\negmedspace\undefined
\let\negthickspace\undefined
\documentclass[journal]{IEEEtran}
\usepackage[a5paper, margin=10mm, onecolumn]{geometry}
\usepackage{tfrupee}

\setlength{\headheight}{1cm}
\setlength{\headsep}{0mm}

\usepackage{gvv-book}
\usepackage{gvv}
\usepackage{cite}
\usepackage{amsmath,amssymb,amsfonts,amsthm}
\usepackage{algorithmic}
\usepackage{graphicx}
\usepackage{textcomp}
\usepackage{xcolor}
\usepackage{txfonts}
\usepackage{listings}
\usepackage{enumitem}
\usepackage{mathtools}
\usepackage{gensymb}
\usepackage{comment}
\usepackage[breaklinks=true]{hyperref}
\usepackage{tkz-euclide}
\usepackage{array}
\usepackage{longtable}
\usepackage{multirow}

\begin{document}

\bibliographystyle{IEEEtran}
\vspace{3cm}

\title{2.10.76}
\author{EE25BTECH11048 - Revanth Siva Kumar}
{\let\newpage\relax\maketitle}

\textbf{Question}:\
For any three vectors $\vec{a},\vec{b},\vec{c}$, prove or disprove
\begin{align*}
    (\vec{a}-\vec{b})\cdot \big((\vec{b}-\vec{c})\times(\vec{c}-\vec{a})\big) = 2\vec{a}\cdot(\vec{b}\times \vec{c}).
\end{align*}



\textbf{Solution}:\
We write the scalar triple product in determinant form:
\begin{align}
(\vec{a}-\vec{b})\cdot \big((\vec{b}-\vec{c})\times(\vec{c}-\vec{a})\big)
= \det\myvec{ (\vec{a}-\vec{b})^T \\ (\vec{b}-\vec{c})^T \\ (\vec{c}-\vec{a})^T }.
\end{align}

Now observe that
\begin{align}
(\vec{a}-\vec{b}) + (\vec{b}-\vec{c}) + (\vec{c}-\vec{a}) = \vec{0}.
\end{align}
Thus the three rows of the determinant are linearly dependent. From matrix theory, the determinant of a matrix with linearly dependent rows is zero. Hence
\begin{align}
(\vec{a}-\vec{b})\cdot \big((\vec{b}-\vec{c})\times(\vec{c}-\vec{a})\big) = 0.
\end{align}

On the other hand, the right-hand side is
\begin{align}
2\vec{a}\cdot(\vec{b}\times\vec{c})
= 2\det\myvec{\vec{a}^T \\ \vec{b}^T \\ \vec{c}^T},
\end{align}
which is not identically zero for arbitrary $\vec{a},\vec{b},\vec{c}$.

\textbf{Conclusion}: The given statement is \textbf{false}. The left-hand side is always zero, while the right-hand side can be nonzero.

\end{document}












