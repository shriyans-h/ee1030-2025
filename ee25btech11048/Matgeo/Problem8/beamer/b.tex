\documentclass{beamer}
\mode<presentation>
\usepackage{amsmath,amssymb,mathtools}
\usepackage{textcomp}
\usepackage{gensymb}
\usepackage{adjustbox}
\usepackage{subcaption}
\usepackage{enumitem}
\usepackage{multicol}
\usepackage{listings}
\usepackage{url}
\usepackage{graphicx}
\def\UrlBreaks{\do/\do-}

\usetheme{Boadilla}
\usecolortheme{lily}
\setbeamertemplate{footline}{
\leavevmode
\hbox{
\begin{beamercolorbox}[wd=\paperwidth,ht=2ex,dp=1ex,right]{author in head/foot}
\insertframenumber{} / \inserttotalframenumber\hspace*{2ex}
\end{beamercolorbox}}
\vskip0pt
}
\setbeamertemplate{navigation symbols}{}

\lstset{
frame=single,
breaklines=true,
columns=fullflexible,
basicstyle=\ttfamily\tiny
}

\numberwithin{equation}{section}

\newcommand{\myvec}[1]{\ensuremath{\begin{pmatrix}#1\end{pmatrix}}}
\let\vec\mathbf

\title{Matgeo Presentation - Problem 4.7.45}
\author{Revanth Siva Kumar.D -- EE25BTECH11048}

\begin{document}

\begin{frame}
\titlepage
\end{frame}

\begin{frame}{QUESTION}
For any three vectors $\vec{a},\vec{b},\vec{c}$, prove or disprove
\begin{align*}
    (\vec{a}-\vec{b})\cdot \big((\vec{b}-\vec{c})\times(\vec{c}-\vec{a})\big) = 2\vec{a}\cdot(\vec{b}\times \vec{c}).
\end{align*}\end{frame}

\begin{frame}{Solution:}
We write the scalar triple product in determinant form:
\begin{align}
(\vec{a}-\vec{b})\cdot \big((\vec{b}-\vec{c})\times(\vec{c}-\vec{a})\big)
= \det\myvec{ (\vec{a}-\vec{b})^T \\ (\vec{b}-\vec{c})^T \\ (\vec{c}-\vec{a})^T }.
\end{align}

Now observe that
\begin{align}
(\vec{a}-\vec{b}) + (\vec{b}-\vec{c}) + (\vec{c}-\vec{a}) = \vec{0}.
\end{align}
Thus the three rows of the determinant are linearly dependent. From matrix theory, the determinant of a matrix with linearly dependent rows is zero. Hence
\begin{align}
(\vec{a}-\vec{b})\cdot \big((\vec{b}-\vec{c})\times(\vec{c}-\vec{a})\big) = 0.
\end{align}
\end{frame}
\begin{frame}{Solution:}
On the other hand, the right-hand side is
\begin{align}
2\vec{a}\cdot(\vec{b}\times\vec{c})
= 2\det\myvec{\vec{a}^T \\ \vec{b}^T \\ \vec{c}^T},
\end{align}
which is not identically zero for arbitrary $\vec{a},\vec{b},\vec{c}$.
\end{frame}
\begin{frame}{Conclusion}The given statement is \textbf{false}. 

The left-hand side  $(\vec{a}-\vec{b})\cdot \big((\vec{b}-\vec{c})\times(\vec{c}-\vec{a})\big)$is  always zero, 
while the right-hand side $2\vec{a}\cdot(\vec{b}\times \vec{c})$ can be nonzero.
\end{frame}
\end{document}
