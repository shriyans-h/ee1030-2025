\let\negmedspace\undefined
\let\negthickspace\undefined
\documentclass[journal]{IEEEtran}
\usepackage[a5paper, margin=10mm, onecolumn]{geometry}
\usepackage{tfrupee}

\setlength{\headheight}{1cm} 
\setlength{\headsep}{0mm}     

\usepackage{gvv-book}
\usepackage{gvv}
\usepackage{cite}
\usepackage{amsmath,amssymb,amsfonts,amsthm}
\usepackage{algorithmic}
\usepackage{graphicx}
\usepackage{textcomp}
\usepackage{xcolor}
\usepackage{txfonts}
\usepackage{listings}
\usepackage{enumitem}
\usepackage{mathtools}
\usepackage{gensymb}
\usepackage{comment}
\usepackage[breaklinks=true]{hyperref}
\usepackage{tkz-euclide} 
\usepackage{array}                                            
\usepackage{longtable}                                       
\usepackage{multirow}                                         

\begin{document}

\bibliographystyle{IEEEtran}
\vspace{3cm}

\title{4.11.25}
\author{EE25BTECH11048 - Revanth Siva Kumar.D}
{\let\newpage\relax\maketitle}

\textbf{Question} \\
Find the distance of the point $\myvec{1,-2,9}$ from the point of intersection of the line
\[
\vec{r} = 4\hat{i} + 2\hat{j} + 7\hat{k} + \lambda \myvec{3\hat{i} + 4\hat{j} + 2\hat{k}}
\]
and the plane
\[
\vec{r}\cdot\myvec{\hat{i} - \hat{j} + \hat{k}} = 10.
\]

\textbf{Solution:} \\

The line is
\begin{align}
    \vec{r} &= \vec{r}_0 + \lambda \vec{d}, \\
    \vec{r}_0 &= \myvec{4 \\ 2 \\ 7}, \quad
    \vec{d} = \myvec{3 \\ 4 \\ 2}.
\end{align}

The plane has normal
\begin{align}
    \vec{n} = \myvec{1 \\ -1 \\ 1}, \quad \vec{r}^T \vec{n} = 10.
\end{align}

Substitute $\vec{r} = \vec{r}_0 + \lambda \vec{d}$ into the plane equation:
\begin{align}
    \vec{n}^T\myvec{\vec{r}_0 + \lambda \vec{d}} &= 10 \\
    \implies \vec{n}^T\vec{d}\,\lambda &= 10 - \vec{n}^T\vec{r}_0.
\end{align}

Now,
\begin{align}
    \vec{n}^T\vec{d} &= \myvec{1 & -1 & 1}\myvec{3 \\ 4 \\ 2} = 1, \\
    \vec{n}^T\vec{r}_0 &= \myvec{1 & -1 & 1}\myvec{4 \\ 2 \\ 7} = 9.
\end{align}

Thus,
\begin{align}
    \lambda &= \frac{10-9}{1} = 1.
\end{align}

Hence, the intersection point is
\begin{align}
    \vec{P} &= \vec{r}_0 + \lambda \vec{d} \\
      &= \myvec{4 \\ 2 \\ 7} + \myvec{3 \\ 4 \\ 2} \\
      &= \myvec{7 \\ 6 \\ 9}.
\end{align}
Given point is
\begin{align}
    \vec{A} = \myvec{1 \\ -2 \\ 9}.
\end{align}
The displacement vector is
\begin{align}
    \vec{v} &= \vec{P} - \vec{A} = \myvec{6 \\ 8 \\ 0}.
\end{align}
Therefore, the distance is
\begin{align}
    d &= \norm{\vec{v}} = \sqrt{\vec{v}^T\vec{v}} \\
      &= \sqrt{6^2 + 8^2 + 0^2} \\
      &= \sqrt{100} = 10.
\end{align}
\textbf{Final Answer: } The required distance is 
\[
\boxed{10}
\]
\pagebreak
\begin{figure}[H]
    \centering
    \includegraphics[width=0.7\columnwidth]{figs/Figure_1.png}
    \caption{PLOT}
    \label{fig:fig1}
\end{figure}
\end{document}














