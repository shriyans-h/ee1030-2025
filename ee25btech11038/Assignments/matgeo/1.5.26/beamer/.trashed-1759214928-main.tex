\documentclass{beamer}
\usepackage[utf8]{inputenc}

\usetheme{Madrid}
\usecolortheme{default}
\usepackage{amsmath,amssymb,amsfonts,amsthm}
\usepackage{txfonts}
\usepackage{tkz-euclide}
\usepackage{listings}
\usepackage{adjustbox}
\usepackage{array}
\usepackage{tabularx}
\usepackage{gvv}
\usepackage{lmodern}
\usepackage{circuitikz}
\usepackage{tikz}
\usepackage{graphicx}

\setbeamertemplate{page number in head/foot}[totalframenumber]

\usepackage{tcolorbox}
\tcbuselibrary{minted,breakable,xparse,skins}



\definecolor{bg}{gray}{0.95}
\DeclareTCBListing{mintedbox}{O{}m!O{}}{%
  breakable=true,
  listing engine=minted,
  listing only,
  minted language=#2,
  minted style=default,
  minted options={%
    linenos,
    gobble=0,
    breaklines=true,
    breakafter=,,
    fontsize=\small,
    numbersep=8pt,
    #1},
  boxsep=0pt,
  left skip=0pt,
  right skip=0pt,
  left=25pt,
  right=0pt,
  top=3pt,
  bottom=3pt,
  arc=5pt,
  leftrule=0pt,
  rightrule=0pt,
  bottomrule=2pt,
  toprule=2pt,
  colback=bg,
  colframe=orange!70,
  enhanced,
  overlay={%
    \begin{tcbclipinterior}
    \fill[orange!20!white] (frame.south west) rectangle ([xshift=20pt]frame.north west);
    \end{tcbclipinterior}},
  #3,
}
\lstset{
    language=C,
    basicstyle=\ttfamily\small,
    keywordstyle=\color{blue},
    stringstyle=\color{orange},
    commentstyle=\color{green!60!black},
    numbers=left,
    numberstyle=\tiny\color{gray},
    breaklines=true,
    showstringspaces=false,
}
\begin{document}

\title 
{1.5.26}
\date{August 23,2025}


\author 
{GNANTHIK LUCKY -EE25BTECH11038}






\frame{\titlepage}
\begin{frame}{Question}
Let  $\vec{P}$ and $\vec{Q}$ be the points of trisection of the line segment that join the points  $\vec{A}$ (2,-2) and  $\vec{B}$ (-7,4) such that  $\vec{P}$ is closer to  $\vec{A}$. Find the coordinates of  $\vec{P}$ and  $\vec{Q}$.
\end{frame}

\begin{frame}{formula}
  \noindent
\textbf{D} divides $BC$ in the ratio $k : 1$, 
\[
    \mathbf{D} = \frac{k\mathbf{C} + \mathbf{B}}{k + 1}
\]
\end{frame}


\begin{frame}{Theoretical Solution}
\[
\text{Let } 
A = \begin{pmatrix} 2 \\ -2 \end{pmatrix}, \qquad
B = \begin{pmatrix} -7 \\ 4 \end{pmatrix}
\]

\textbf{Point \( P \) (Nearer to \( A \), Ratio 1 : 2):}
\[
P = \frac{1}{3} B + \frac{2}{3} A 
= \frac{1}{3} \begin{pmatrix} -7 \\ 4 \end{pmatrix} 
  + \frac{2}{3} \begin{pmatrix} 2 \\ -2 \end{pmatrix}
\]
\[
P = \begin{pmatrix}
\frac{1 \times (-7) + 2 \times 2}{3} \\
\frac{1 \times 4 + 2 \times (-2)}{3}
\end{pmatrix}
= \begin{pmatrix} -1 \\ 0 \end{pmatrix}
\]
\end{frame}

\begin{frame}
\textbf{Point \( Q \) (Further from \( A \), Ratio 2 : 1):}
\[
Q = \frac{2}{3} B + \frac{1}{3} A 
= \frac{2}{3} \begin{pmatrix} -7 \\ 4 \end{pmatrix}
  + \frac{1}{3} \begin{pmatrix} 2 \\ -2 \end{pmatrix}
\]
\[
Q = \begin{pmatrix}
\frac{2 \times (-7) + 1 \times 2}{3} \\
\frac{2 \times 4 + 1 \times (-2)}{3}
\end{pmatrix}
= \begin{pmatrix} -4 \\ 2 \end{pmatrix}
\]

\[
\boxed{P = (-1,\,0)\qquad Q = (-4,\,2)}
\]
\end{frame}

\begin{frame}
    \vspace{5em}
\textbf{Graph of the line segment AB with trisection points P and Q}
\begin{figure}[H]
    \centering
    \includegraphics[width=0.75\columnwidth]{figs/1.jpg}
    \caption{Figure for 1.5.26}
    \label{fig1}
\end{figure}
\end{frame}

\end{document}