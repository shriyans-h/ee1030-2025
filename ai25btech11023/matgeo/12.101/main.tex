 \let\negmedspace\undefined
\let\negthickspace\undefined
\documentclass[journal]{IEEEtran}
\usepackage[a5paper, margin=10mm, onecolumn]{geometry}
%\usepackage{lmodern} % Ensure lmodern is loaded for pdflatex
\usepackage{tfrupee} % Include tfrupee package

\setlength{\headheight}{1cm} % Set the height of the header box
\setlength{\headsep}{0mm}     % Set the distance between the header box and the top of the text
\usepackage{gvv-book}
\usepackage{gvv}
\usepackage{cite}
\usepackage{amsmath,amssymb,amsfonts,amsthm}
\usepackage{algorithmic}
\usepackage{graphicx}
\usepackage{textcomp}
\usepackage{xcolor}
\usepackage{txfonts}
\usepackage{listings}
\usepackage{enumitem}
\usepackage{mathtools}
\usepackage{gensymb}
\usepackage{comment}
\usepackage[breaklinks=true]{hyperref}
\usepackage{tkz-euclide} 
\usepackage{listings}
% \usepackage{gvv}                                        
\def\inputGnumericTable{}                                 
\usepackage[latin1]{inputenc}                                
\usepackage{color}                                            
\usepackage{array}                                            
\usepackage{longtable}                                       
\usepackage{calc}                                             
\usepackage{multirow}                                         
\usepackage{hhline}                                           
\usepackage{ifthen}                                           
\usepackage{lscape}



\usepackage{amsmath,amssymb}
\usepackage{booktabs}
\usepackage{tikz}
\usetikzlibrary{arrows.meta,angles,quotes}





\begin{document}

\bibliographystyle{IEEEtran}
\vspace{3cm}

\title{12.101}
\author{AI25BTECH11023 - Pratik R}
% \maketitle
% \newpage
% \bigskip
{\let\newpage\relax\maketitle}

\renewcommand{\thefigure}{\theenumi}
\renewcommand{\thetable}{\theenumi}
\setlength{\intextsep}{10pt} % Space between text and floats


\numberwithin{equation}{enumi}
\numberwithin{figure}{enumi}
\renewcommand{\thetable}{\theenumi}


\section*{\textbf{Question}}

$\vec{A}$ is $2\times2$ with $\operatorname{tr}(\vec{A}) = 5$, $\det(\vec{A}) = 6$. Let the characteristic polynomial of $(\vec{A} + \vec{I_2})^{-1}$ be $x^2 - bx + c$. Find $b/c =$ (integer).

\subsection*{\textbf{Solution}}
Given:
\begin{align}
\operatorname{tr}(\vec{A}) &= 5 \\
\det(\vec{A}) &= 6
\end{align}

Let the eigenvalues of $\vec{A}$ be $\lambda_1$ and $\lambda_2$:
\begin{align}
\lambda_1 + \lambda_2 &= 5 \\
\lambda_1 \lambda_2 &= 6
\end{align}

Eigenvalues of $\vec{A} + \vec{I_2}$ are:
\begin{align}
\lambda_1 + 1,\quad \lambda_2 + 1
\end{align}

Eigenvalues of $(\vec{A} + \vec{I_2})^{-1}$ are:
\begin{align}
\frac{1}{\lambda_1 + 1},\quad \frac{1}{\lambda_2 + 1}
\end{align}

Thus, its characteristic polynomial is:
\begin{align}
x^2 - \left(\frac{1}{\lambda_1 + 1} + \frac{1}{\lambda_2 + 1}\right)x + \frac{1}{(\lambda_1 + 1)(\lambda_2 + 1)}
\end{align}

Calculate:
\begin{align}
(\lambda_1+1)(\lambda_2+1) &= \lambda_1\lambda_2 + (\lambda_1+\lambda_2) + 1 = 6 + 5 + 1 = 12 \\
\frac{1}{\lambda_1+1} + \frac{1}{\lambda_2+1} &= \frac{(\lambda_1+1) + (\lambda_2+1)}{(\lambda_1+1)(\lambda_2+1)} = \frac{5+2}{12} = \frac{7}{12}
\end{align}

Therefore:
\begin{align}
b &= \frac{7}{12},\quad c = \frac{1}{12} \\
\frac{b}{c} &= \frac{\frac{7}{12}}{\frac{1}{12}} = 7
\end{align}

Hence, the answer is $7$.
\end{document}
