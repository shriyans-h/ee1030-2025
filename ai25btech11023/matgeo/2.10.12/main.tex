\let\negmedspace\undefined
\let\negthickspace\undefined
\documentclass[journal]{IEEEtran}
\usepackage[a5paper, margin=10mm, onecolumn]{geometry}
%\usepackage{lmodern} % Ensure lmodern is loaded for pdflatex
\usepackage{tfrupee} % Include tfrupee package

\setlength{\headheight}{1cm} % Set the height of the header box
\setlength{\headsep}{0mm}     % Set the distance between the header box and the top of the text

\usepackage{gvv-book}
\usepackage{gvv}
\usepackage{cite}
\usepackage{amsmath,amssymb,amsfonts,amsthm}
\usepackage{algorithmic}
\usepackage{graphicx}
\usepackage{textcomp}
\usepackage{xcolor}
\usepackage{txfonts}
\usepackage{listings}
\usepackage{enumitem}
\usepackage{mathtools}
\usepackage{gensymb}
\usepackage{comment}
\usepackage[breaklinks=true]{hyperref}
\usepackage{tkz-euclide} 
\usepackage{listings}
% \usepackage{gvv}                                        
\def\inputGnumericTable{}                                 
\usepackage[latin1]{inputenc}                                
\usepackage{color}                                            
\usepackage{array}                                            
\usepackage{longtable}                                       
\usepackage{calc}                                             
\usepackage{multirow}                                         
\usepackage{hhline}                                           
\usepackage{ifthen}                                           
\usepackage{lscape}
\usepackage{circuitikz}


\begin{document}

\bibliographystyle{IEEEtran}
\vspace{3cm}

\title{2.10.12}
\author{AI25BTECH11023-Pratik R}
 \maketitle
% \newpage
% \bigskip
{\let\newpage\relax\maketitle}

\renewcommand{\thefigure}{\theenumi}
\renewcommand{\thetable}{\theenumi}
\setlength{\intextsep}{10pt} % Space between text and floats


\numberwithin{equation}{enumi}
\numberwithin{figure}{enumi}
\renewcommand{\thetable}{\theenumi}

\textbf{Question}:\\
A unit vector perpendicular to the plane determined by the points
$P\brak{1,-1,2}$,$Q\brak{2,0,-1}$ and $R\brak{0,2,1}$ is \\ 
\solution \\
According to the question, \\
Given the position vectors,
\begin{align}
    \vec{P}=\myvec{1\\-1\\2};
    \vec{Q}=\myvec{2\\0\\-1};
    \vec{R}=\myvec{0\\2\\1}
\end{align}
\begin{align}
    \vec{A}= \vec{Q}-\vec{P} = \myvec{1\\1\\-3}\\
    \vec B = \vec{R}-\vec{P} = \myvec{-1\\3\\-1}
\end{align}

we need to find the unit vector which is perpendicular to the vectors $\vec{A}$ and $\vec{B}$.The vector perpendicular to $\vec{A}$ and $\vec{B}$ is given by their cross-product.\\
Let the perpendicular vector be $\vec{X}^T=\myvec{X_1&&X_2&&X_3}$
\begin{align}
    \because \vec{A}^T\vec{X}=0\\
    \vec{B}^T\vec{X}=0\;,
\end{align}
\begin{align}
    \therefore \myvec{\vec{A}^T\\\vec{B}^T}\vec{X}=0
\end{align}

\begin{align}
    \myvec{1&&1&&-3\\-1&&3&&-1}\myvec{X_1\\X_2\\X_3}=0
\end{align}
This can be represented as,
\begin{align}
    \myvec{1&&1&&-3\\-1&&3&&-1}
    \xleftrightarrow{\,R_2 \gets R_2+R_1}
    \myvec{1&&1&&-3\\0&&4&&-4}
\end{align}

yielding,
\begin{align}
    x_1+x_2-3x_3=0\\
    4x_2-4x_3=0 \\
    \implies x_2=x_3 \\
    x_1 = 2x_3
\end{align}
\begin{align}
    \vec{x}=x_3\myvec{2\\1\\1}
\end{align}

The unit vector perpendicular to the plane is given by \\
\begin{align}
     \vec{x}=\frac{1}{\sqrt{6}}\myvec{2\\1\\1}
\end{align}
\\

\begin{figure}[H]
    \centering
    \includegraphics[width=0.8\columnwidth]{figs/fig.png}
    \label{fig-1}
\end{figure}

 


\end{document}
