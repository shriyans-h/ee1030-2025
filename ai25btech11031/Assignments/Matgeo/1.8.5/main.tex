\let\negmedspace\undefined
\let\negthickspace\undefined
\documentclass[journal]{IEEEtran}
\usepackage[a5paper, margin=10mm, onecolumn]{geometry}
%\usepackage{lmodern} % Ensure lmodern is loaded for pdflatex
\usepackage{tfrupee} % Include tfrupee package

\setlength{\headheight}{1cm} % Set the height of the header box
\setlength{\headsep}{0mm}     % Set the distance between the header box and the top of the text

\usepackage{gvv-book}
\usepackage{gvv}
\usepackage{cite}
\usepackage{amsmath,amssymb,amsfonts,amsthm}
\usepackage{algorithmic}
\usepackage{graphicx}
\usepackage{textcomp}
\usepackage{xcolor}
\usepackage{txfonts}
\usepackage{listings}
\usepackage{enumitem}
\usepackage{mathtools}
\usepackage{gensymb}
\usepackage{comment}
\usepackage[breaklinks=true]{hyperref}
\usepackage{tkz-euclide} 
\usepackage{listings}
% \usepackage{gvv}                                        
\def\inputGnumericTable{}                                 
\usepackage[latin1]{inputenc}                                
\usepackage{color}                                            
\usepackage{array}                                            
\usepackage{longtable}                                       
\usepackage{calc}                                             
\usepackage{multirow}                                         
\usepackage{hhline}                                           
\usepackage{ifthen}                                           
\usepackage{lscape}
\usepackage{circuitikz}
\tikzstyle{block} = [rectangle, draw, fill=blue!20, 
    text width=4em, text centered, rounded corners, minimum height=3em]
\tikzstyle{sum} = [draw, fill=blue!10, circle, minimum size=1cm, node distance=1.5cm]
\tikzstyle{input} = [coordinate]
\tikzstyle{output} = [coordinate]


\begin{document}

\bibliographystyle{IEEEtran}
\vspace{3cm}

\title{1.8.5}
\author{AI25BTECH110031 \\ Shivam Sawarkar}
 \maketitle
% \newpage
% \bigskip
{\let\newpage\relax\maketitle}

\renewcommand{\thefigure}{\theenumi}
\renewcommand{\thetable}{\theenumi}
\setlength{\intextsep}{10pt} % Space between text and floats


\numberwithin{equation}{enumi}
\numberwithin{figure}{enumi}
\renewcommand{\thetable}{\theenumi}

\textbf{Question(1.8.5)}
If $\vec{A}$ and $\vec{B}$ be the points $(3,4,5)$ and $(-1,3,-7)$ respectively, find the equation of the set of a point $\vec{P}$ such that $PA^2+{PB}^2=k^2$

\textbf{Solution}:
Given ponits 
\begin{align}
    \vec{A}=\myvec{3 \\ 4 \\ 5} \quad \vec{B}=\myvec{-1 \\ 3 \\ -7}
\end{align}
According to the question,  
\begin{align}
    PA^2+PB^2=k^2
\end{align}
where, $PA=\norm{P-A}$ and $PB=\norm{P-B}$

The squared distances can be written as dot products:
\begin{align}
    PA^2=(\vec{P}-\vec{A})^T(\vec{P}-\vec{A}) \\
    PB^2=(\vec{P}-\vec{B})^T(\vec{P}-\vec{B})
\end{align}
Thus:
\begin{align}
    PA^2+PB^2=(\vec{P}-\vec{A})^T(\vec{P}-\vec{A})+(\vec{P}-\vec{B})^T(\vec{P}-\vec{B}) \\
    PA^2+PB^2=2\vec{P}^T\vec{P}-2\vec{A}^T\vec{P}+\vec{A}^T\vec{A}+\vec{P}^T\vec{P}-2\vec{B}^T\vec{P}+\vec{B}^T\vec{B}\\ 
    2\vec{P}^T\vec{P}-2(\vec{A}+\vec{B})^T\vec{P}+\vec{A}^T\vec{A}+\vec{B}^T\vec{B}-k^2=0
\end{align}

Complete the square,\\ 
Let, \\ 
\begin{align}
    \vec{M}:=\frac{\vec{A}+\vec{B}}{2} \\ 
    R^2:=\norm{M}^2-\frac{\vec{A}^T\vec{A}+\vec{B}^T\vec{B}-k^2}{2}\\ 
\end{align}

Then the equation becomes\\ 
\begin{align}
    \norm{P-M}^2=R^2 \\ 
    \norm{P-\frac{\vec{A}+\vec{B}}{2}}^2=\norm{\frac{\vec{A}+\vec{B}}{2}}^2-\frac{\norm{A}^2+\norm{B}^2-k^2}{2}
\end{align}

Substitute the known values
\begin{align}
    \norm{A}=3^2+4^2+5^2=50 \\
    \norm{B}=(-1)^2+3^2+(-7)^2=59 \\
    \frac{\vec{A}+\vec{B}}{2}=\myvec{1 \\ 3.5 \\ -1}
\end{align}

The equation of the locus is:
\begin{align}
    \norm{P-\myvec{1 \\ 3.5 \\ -1}}^2=\frac{2k^2-161}{4}
\end{align}
\\ \\ \\ 
The plot show the locus for $k=20$

\begin{figure}[H]
    \centering
    \includegraphics[width=1\columnwidth]{figs/plot2_1.png}
    \caption{}
    \label{fig:plot}
\end{figure}

\end{document}