\documentclass{beamer}
\usepackage[utf8]{inputenc}

\usetheme{Madrid}
\usecolortheme{default}
\usepackage{amsmath,amssymb,amsfonts,amsthm}
\usepackage{txfonts}
\usepackage{tkz-euclide}
\usepackage{listings}
\usepackage{adjustbox}
\usepackage{array}
\usepackage{tabularx}
\usepackage{gvv}
\usepackage{lmodern}
\usepackage{circuitikz}
\usepackage{tikz}
\usepackage{graphicx}

\setbeamertemplate{page number in head/foot}[totalframenumber]

\usepackage{tcolorbox}
\tcbuselibrary{minted,breakable,xparse,skins}



\definecolor{bg}{gray}{0.95}
\DeclareTCBListing{mintedbox}{O{}m!O{}}{%
  breakable=true,
  listing engine=minted,
  listing only,
  minted language=#2,
  minted style=default,
  minted options={%
    linenos,
    gobble=0,
    breaklines=true,
    breakafter=,,
    fontsize=\small,
    numbersep=8pt,
    #1},
  boxsep=0pt,
  left skip=0pt,
  right skip=0pt,
  left=25pt,
  right=0pt,
  top=3pt,
  bottom=3pt,
  arc=5pt,
  leftrule=0pt,
  rightrule=0pt,
  bottomrule=2pt,

  colback=bg,
  colframe=orange!70,
  enhanced,
  overlay={%
    \begin{tcbclipinterior}
    \fill[orange!20!white] (frame.south west) rectangle ([xshift=20pt]frame.north west);
    \end{tcbclipinterior}},
  #3,
}
\lstset{
    language=C,
    basicstyle=\ttfamily\small,
    keywordstyle=\color{blue},
    stringstyle=\color{orange},
    commentstyle=\color{green!60!black},
    numbers=left,
    numberstyle=\tiny\color{gray},
    breaklines=true,
    showstringspaces=false,
}
%------------------------------------------------------------
%This block of code defines the information to appear in the
%Title page
\title %optional
{1.8.5}
\date{August  2025}
%\subtitle{A short story}

\author % (optional)
{Shivam Sawarkar \\ AI25BTECH11031}



\begin{document}


\frame{\titlepage}
\begin{frame}{Question (1.8.5)}
If $\vec{A}$ and $\vec{B}$ be the points $(3,4,5)$ and $(-1,3,-7)$ respectively, find the equation of the set of a point $\vec{P}$ such that $PA^2+PB^2=k^2$
\end{frame}

\begin{frame}{Given}
    \begin{align}
    \vec{A}=\myvec{3 \\ 4 \\ 5} \quad \vec{B}=\myvec{-1 \\ 3 \\ -7}
\end{align}
According to the question,  
\begin{align}
    PA^2+PB^2=k^2
\end{align}
where, $PA^2=\norm{P-A}$ and $PB^2=\norm{P-B}$
\end{frame}

\begin{frame}{Solution}
    The squared distances can be written as dot products:
\begin{align}
    PA^2=(\vec{P}-\vec{A})^T(\vec{P}-\vec{A}) \\
    PB^2=(\vec{P}-\vec{B})^T(\vec{P}-\vec{B})
\end{align}
Thus:
\begin{align}
    PA^2+PB^2=(\vec{P}-\vec{A})^T(\vec{P}-\vec{A})+(\vec{P}-\vec{B})^T(\vec{P}-\vec{B}) \\
    PA^2+PB^2=\vec{P}^T\vec{P}-2\vec{A}^T\vec{P}+\vec{A}^T\vec{A}+\vec{P}^T\vec{P}-2\vec{B}^T\vec{P}+\vec{B}^T\vec{B}
\end{align}
\end{frame}

\begin{frame}{Compliting Square}
    Let, \\ 
\begin{align}
    \vec{M}:=\frac{\vec{A}+\vec{B}}{2} \\ 
    R^2:=\norm{M}^2-\frac{\vec{A}^T\vec{A}+\vec{B}^T\vec{B}-k^2}{2}
\end{align}

Then the equation becomes\\ 
\begin{align}
    \norm{P-M}^2=R^2 \\ 
    \norm{P-\frac{\vec{A}+\vec{B}}{2}}^2=\norm{\frac{\vec{A}+\vec{B}}{2}}^2-\frac{\norm{A}^2+\norm{B}^2-k^2}{2}
\end{align}
\end{frame}

\begin{frame}{Substitute the known values}
\begin{align}
    \norm{A}=3^2+4^2+5^2=50 \\
    \norm{B}=(-1)^2+3^2+(-7)^2=59 \\
    \frac{\vec{A}+\vec{B}}{2}=\myvec{1 \\ 3.5 \\ -1}
\end{align}
\end{frame}

\begin{frame}{Result}
    The equation of the locus is:
\begin{align}
    \norm{P-\myvec{1 \\ 3.5 \\ -1}}^2=\frac{2k^2-161}{4}
\end{align}
\end{frame}

\begin{frame}{Plot}
    The plot show the locus for $k=20$

\begin{figure}[H]
    \centering
    \includegraphics[width=0.8\columnwidth]{figs/plot2_1.png}
    \caption{}
    \label{fig:plot}
\end{figure}
\end{frame}

\begin{frame}[fragile]{C Code}
    \begin{verbatim}
#include <stdio.h>
#include <math.h>

int main() {
    double Ax, Ay, Az, Bx, By, Bz, K;
    double Mx, My, Mz, diffx, diffy, diffz;
    double diff_sq, r2, radius;

    // Input
    printf("Enter coordinates of A (x y z): ");
    scanf("%lf %lf %lf", &Ax, &Ay, &Az);

    printf("Enter coordinates of B (x y z): ");
    scanf("%lf %lf %lf", &Bx, &By, &Bz);

    \end{verbatim}
\end{frame}
\begin{frame}[fragile]{C Code}
    \begin{verbatim}
        printf("Enter constant K: ");
    scanf("%lf", &K);

    // Midpoint (center of sphere)
    Mx = (Ax + Bx) / 2.0;
    My = (Ay + By) / 2.0;
    Mz = (Az + Bz) / 2.0;

    // ||A-B||^2
    diffx = Ax - Bx;
    diffy = Ay - By;
    diffz = Az - Bz;
    diff_sq = diffx*diffx + diffy*diffy + diffz*diffz;

    // r^2 formula
    r2 = (2.0 * K * K - diff_sq) / 4.0;
    \end{verbatim}
\end{frame}

\begin{frame}[fragile]{C Code}
    \begin{verbatim}
    printf("Center M = (%.2f, %.2f, %.2f)\n", Mx, My, Mz);
    printf("r^2 = %.4f\n", r2);

    if (r2 < 0) {
        printf("No real sphere exists (r^2 < 0).\n");
    } else {
        radius = sqrt(r2);
        printf("Radius = %.4f\n", radius);

        printf("\nEquation of locus (vector form):\n");
        printf("|| P - (%.2f, %.2f, %.2f) ||^2 = %.4f\n",
        Mx, My, Mz, r2);
    }

    return 0;
}
    \end{verbatim}
\end{frame}
\begin{frame}[fragile]{Python Code}
    \begin{verbatim}
    import numpy as np

def sphere_from_sum_of_squares(A, B, K):
    A = np.asarray(A, dtype=float)
    B = np.asarray(B, dtype=float)
    M = (A + B) / 2.0
    diff = A - B
    diff_sq = np.dot(diff, diff)  # ||A-B||^2
    r2 = (2.0 * K**2 - diff_sq) / 4.0
    if r2 < 0:
        return {'center': M, 'r2': r2, 'radius': None, 'real': False}
    else:
        return {'center': M, 'r2': r2, 'radius': np.sqrt(r2), 'real': True}
    \end{verbatim}
\end{frame}

\begin{frame}[fragile]{Python Code}
    \begin{verbatim}
    def main():
    # Input values
    A = tuple(map(float, input("Enter coordinates of A (x y z): ").split()))
    B = tuple(map(float, input("Enter coordinates of B (x y z): ").split()))
    K = float(input("Enter constant K: "))

    res = sphere_from_sum_of_squares(A, B, K)

    print("\n--- Results ---")
    print("Point A =", A)
    print("Point B =", B)
    print("K =", K)
    print("Center M =", res['center'])
    print("r^2 =", res['r2'])
    \end{verbatim}
\end{frame}

\begin{frame}[fragile]{Python Code}
    \begin{verbatim}
    if res['real']:
        print("Radius =", res['radius'])
        # Vector form of locus
        print("\nEquation of locus (vector form):")
        print(f"|| P - {res['center']} ||^2 = {res['r2']}")
    else:
        print("No real sphere exists (r^2 < 0).")
    \end{verbatim}
\end{frame}



\end{document}