\documentclass{beamer}
\usepackage[utf8]{inputenc}

\usetheme{Madrid}
\usecolortheme{default}
\usepackage{amsmath,amssymb,amsfonts,amsthm}
\usepackage{txfonts}
\usepackage{tkz-euclide}
\usepackage{listings}
\usepackage{adjustbox}
\usepackage{array}
\usepackage{tabularx}
\usepackage{gvv}
\usepackage{lmodern}
\usepackage{circuitikz}
\usepackage{tikz}
\usepackage{graphicx}

\setbeamertemplate{page number in head/foot}[totalframenumber]

\usepackage{tcolorbox}
\tcbuselibrary{minted,breakable,xparse,skins}



\definecolor{bg}{gray}{0.95}
\DeclareTCBListing{mintedbox}{O{}m!O{}}{%
  breakable=true,
  listing engine=minted,
  listing only,
  minted language=#2,
  minted style=default,
  minted options={%
    linenos,
    gobble=0,
    breaklines=true,
    breakafter=,,
    fontsize=\small,
    numbersep=8pt,
    #1},
  boxsep=0pt,
  left skip=0pt,
  right skip=0pt,
  left=25pt,
  right=0pt,
  top=3pt,
  bottom=3pt,
  arc=5pt,
  leftrule=0pt,
  rightrule=0pt,
  bottomrule=2pt,

  colback=bg,
  colframe=orange!70,
  enhanced,
  overlay={%
    \begin{tcbclipinterior}
    \fill[orange!20!white] (frame.south west) rectangle ([xshift=20pt]frame.north west);
    \end{tcbclipinterior}},
  #3,
}
\lstset{
    language=C,
    basicstyle=\ttfamily\small,
    keywordstyle=\color{blue},
    stringstyle=\color{orange},
    commentstyle=\color{green!60!black},
    numbers=left,
    numberstyle=\tiny\color{gray},
    breaklines=true,
    showstringspaces=false,
}
%------------------------------------------------------------
%This block of code defines the information to appear in the
%Title page
\title %optional
{1.8.5}
\date{August  2025}
%\subtitle{A short story}

\author % (optional)
{Shivam Sawarkar \\ AI25BTECH11031}



\begin{document}


\frame{\titlepage}
\begin{frame}{Question (1.8.5)}
If $\vec{A}$ and $\vec{B}$ be the points $(3,4,5)$ and $(-1,3,-7)$ respectively, find the equation of the set of a point $\vec{P}$ such that $\vec{PA}^2+\vec{PB}^2=k^2$
\end{frame}

\begin{frame}{Given}
    \begin{align}
    \vec{A}=\myvec{3 \\ 4 \\ 5} \quad \vec{B}=\myvec{-1 \\ 3 \\ -7}
\end{align}
According to the question,  
\begin{align}
    \vec{PA}^2+\vec{PB}^2=k^2
\end{align}
where, $\vec{PA}=\norm{P-A}$ and $\vec{PB}=\norm{P-B}$
\end{frame}

\begin{frame}{Solution}
    The squared distances can be written as dot products:
\begin{align}
    \vec{PA}^2=(\vec{P}-\vec{A}).(\vec{P}-\vec{A}) \\
    \vec{PB}^2=(\vec{P}-\vec{B}).(\vec{P}-\vec{B})
\end{align}
Thus:
\begin{align}
    \vec{PA}^2+\vec{PB}^2=(\vec{P}-\vec{A}).(\vec{P}-\vec{A})+(\vec{P}-\vec{B}).(\vec{P}-\vec{B}) \\
    \vec{PA}^2+\vec{PB}^2=\vec{P}.\vec{P}-2\vec{A}.\vec{P}+\vec{A}.\vec{A}+\vec{P}.\vec{P}-2\vec{B}.\vec{P}+\vec{B}.\vec{B}\\
\end{align}
Substitute the known values
\begin{align}
    \vec{A}.\vec{A}=3^2+4^2+5^2=50 \\
    \vec{B}.\vec{B}=(-1)^2+3^2+(-7)^2=59 \\
    \vec{A}+\vec{B}=\myvec{3-1 \\ 4-3 \\ 5-7}=\myvec{2 \\ 7 \\ -2}
\end{align}
\end{frame}

\begin{frame}{Result}
    The equation of the locus is:
\begin{align}
    2\vec{P}.\vec{P}-2\myvec{2 \\ 7 \\ -2}.\vec{P}+109=K^2
\end{align}

or equivalently,
\begin{align}
    2\vec{P}^T\vec{P}-2\myvec{2 & 7 & -2}.\vec{P}+109=K^2
\end{align}
\end{frame}

\begin{frame}{Plot}
    The plot show the locus for $k=20$

\begin{figure}[H]
    \centering
    \includegraphics[width=0.8\columnwidth]{figs/plot2_1.png}
    \caption{}
    \label{fig:plot}
\end{figure}
\end{frame}

\begin{frame}[fragile]{C Code}
    \begin{verbatim}
#include <stdio.h>

// Define a 3D vector struct
typedef struct {
    double x, y, z;
} Vector3;

// Dot product of two vectors
double dot(Vector3 v1, Vector3 v2) {
    return v1.x * v2.x + v1.y * v2.y + v1.z * v2.z;
}
int main() {
    Vector3 A = {3, 4, 5};
    Vector3 B = {-1, 3, -7};
    \end{verbatim}
\end{frame}
\begin{frame}[fragile]{C Code}
    \begin{verbatim}
    // Vector sum A+B
    Vector3 AplusB = {A.x + B.x, A.y + B.y, A.z + B.z};
    // Dot products |A|^2 and |B|^2
    double A_dot = dot(A, A);
    double B_dot = dot(B, B);
    // Equation in vector form:
    // 2 * (P·P) - 2 * (A+B)·P + |A|^2 + |B|^2 = K^2
    printf("Vector form equation of locus P satisfies:\n");
    printf("2 * (P · P) - 2 * (A+B) · P + |A|^2 + |B|^2 = K^2
    \n");
    printf("where\n");
    printf("A + B = (%.1f, %.1f, %.1f),\n", AplusB.x, 
    AplusB.y,AplusB.z);
    printf("|A|^2 = %.1f, |B|^2 = %.1f\n", A_dot, B_dot);
    return 0;
}
    \end{verbatim}
\end{frame}


\begin{frame}[fragile]{Python Code}
    \begin{verbatim}
        import numpy as np

# Define points A and B
A = np.array([3, 4, 5])
B = np.array([-1, 3, -7])

# Compute vector sum A+B
AplusB = A + B

# Compute dot products |A|^2 and |B|^2
A_dot = np.dot(A, A)
B_dot = np.dot(B, B)
    \end{verbatim}
\end{frame}

\begin{frame}[fragile]{Python Code}
    \begin{verbatim}
print("Vector form equation of locus P satisfies:")
print("2 * (P · P) - 2 * (A+B) · P + |A|^2 + |B|^2 = K^2")
print("where")
print(f"A + B = ({AplusB[0]:.1f}, {AplusB[1]:.1f}, {AplusB[2]:.1f}),")
print(f"|A|^2 = {A_dot:.1f}, |B|^2 = {B_dot:.1f}")

    \end{verbatim}
\end{frame}





\end{document}