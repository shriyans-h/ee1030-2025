\let\negmedspace\undefined
\let\negthickspace\undefined
\documentclass[journal]{IEEEtran}
\usepackage[a5paper, margin=10mm, onecolumn]{geometry}
%\usepackage{lmodern} % Ensure lmodern is loaded for pdflatex
\usepackage{tfrupee} % Include tfrupee package

\setlength{\headheight}{1cm} % Set the height of the header box
\setlength{\headsep}{0mm}     % Set the distance between the header box and the top of the text

\usepackage{gvv-book}
\usepackage{gvv}
\usepackage{cite}
\usepackage{amsmath,amssymb,amsfonts,amsthm}
\usepackage{algorithmic}
\usepackage{graphicx}
\usepackage{textcomp}
\usepackage{xcolor}
\usepackage{txfonts}
\usepackage{listings}
\usepackage{enumitem}
\usepackage{mathtools}
\usepackage{gensymb}
\usepackage{comment}
\usepackage[breaklinks=true]{hyperref}
\usepackage{tkz-euclide} 
\usepackage{listings}
% \usepackage{gvv}                                        
\def\inputGnumericTable{}                                 
\usepackage[latin1]{inputenc}                                
\usepackage{color}                                            
\usepackage{array}                                            
\usepackage{longtable}                                       
\usepackage{calc}                                             
\usepackage{multirow} 
\usepackage{hhline}                                           
\usepackage{ifthen}                                           
\usepackage{lscape}
\usepackage{circuitikz}
\tikzstyle{block} = [rectangle, draw, fill=blue!20, 
    text width=4em, text centered, rounded corners, minimum height=3em]
\tikzstyle{sum} = [draw, fill=blue!10, circle, minimum size=1cm, node distance=1.5cm]
\tikzstyle{input} = [coordinate]
\tikzstyle{output} = [coordinate]

\begin{document}
\bibliographystyle{IEEEtran}
\vspace{3cm}

\title{MatGeo Assignment 1.2.14}
\author{AI25BTECH11008\\Chiruvella Harshith Sharan}

 \maketitle
{\let\newpage\relax\maketitle}

\renewcommand{\thefigure}{\theenumi}
\renewcommand{\thetable}{\theenumi}
\setlength{\intextsep}{10pt} % Space between text and floats


\numberwithin{equation}{enumi}
\numberwithin{figure}{enumi}
\renewcommand{\thetable}{\theenumi}
\noindent
\textbf{Question:}\\
The fourth vertex $D$ of a parallelogram $ABCD$ whose three vertices are 
$A(-2,3)$, $B(6,7)$ and $C(8,3)$ is \\
\noindent
\textbf{Solution:}\\
Let us solve the given equation theoretically and then verify the solution computationally. \\

We are given three vertices of a parallelogram:

\[
A(-2,3), \; B(6,7), \; C(8,3).
\]

\textbf{Property: In a parallelogram, diagonals bisect each other.}

Thus, midpoint of $AC = $ midpoint of $BD$.  

Let $D(x,y)$ be the fourth vertex.  

\[
\frac{1}{2}
\myvec{
-2+8 \\
3+3
}
=
\frac{1}{2}
\myvec{
6+x \\
7+y
}
\]

\[
\myvec{
\frac{6}{2} \\
\frac{6}{2}
}
=
\myvec{
\frac{6+x}{2} \\
\frac{7+y}{2}
}
\]

\[
\myvec{
3 \\
3
}
=
\myvec{
\frac{6+x}{2} \\
\frac{7+y}{2}
}
\]

\[
\frac{6+x}{2} = 3, \quad \frac{7+y}{2} = 3
\]

\[
x = 0, \quad y = -1
\]

\[
\therefore \; D(0,-1)
\]

\begin{figure}[H]
    \centering
    \includegraphics[width=0.75\linewidth]{figs/fig1.png}
    \caption{The visual of the parallelogram with vertices $A$, $B$, $C$, $D$ and diagonals shown}
    \label{fig:figs/fig1.png}
\end{figure}


\noindent
From the figure it is clearly verified that the theoretical solution matches with the computational solution.

\end{document}
