\documentclass{beamer}
\usepackage[utf8]{inputenc}

\usetheme{Madrid}
\usecolortheme{default}
\usepackage{amsmath,amssymb,amsfonts,amsthm}
\usepackage{txfonts}
\usepackage{tkz-euclide}
\usepackage{listings}
\usepackage{adjustbox}
\usepackage{array}
\usepackage{tabularx}
\usepackage{gvv}
\usepackage{lmodern}
\usepackage{circuitikz}
\usepackage{tikz}
\usepackage{graphicx}

\setbeamertemplate{page number in head/foot}[totalframenumber]

\usepackage{tcolorbox}
\tcbuselibrary{minted,breakable,xparse,skins}

\definecolor{bg}{gray}{0.95}
\DeclareTCBListing{mintedbox}{O{}m!O{}}{%
  breakable=true,
  listing engine=minted,
  listing only,
  minted language=#2,
  minted style=default,
  minted options={%
    linenos,
    gobble=0,
    breaklines=true,
    breakafter=,,
    fontsize=\small,
    numbersep=8pt,
    #1},
  boxsep=0pt,
  left skip=0pt,
  right skip=0pt,
  left=25pt,
  right=0pt,
  top=3pt,
  bottom=3pt,
  arc=5pt,
  leftrule=0pt,
  rightrule=0pt,
  bottomrule=2pt,
  toprule=2pt,
  colback=bg,
  colframe=orange!70,
  enhanced,
  overlay={%
    \begin{tcbclipinterior}
    \fill[orange!20!white] (frame.south west) rectangle ([xshift=20pt]frame.north west);
    \end{tcbclipinterior}},
  #3,
}
\lstset{
    language=C,
    basicstyle=\ttfamily\small,
    keywordstyle=\color{blue},
    stringstyle=\color{orange},
    commentstyle=\color{green!60!black},
    numbers=left,
    numberstyle=\tiny\color{gray},
    breaklines=true,
    showstringspaces=false,
}
\title 
{MatGeo Assignment 1.2.14}

\author
{AI25BTECH11008}
\begin{document}

\frame{\titlepage}

\begin{frame}{Question}
The fourth vertex $D$ of a parallelogram $ABCD$ whose three vertices are 
$A(-2,3)$, $B(6,7)$ and $C(8,3)$ is
\end{frame}

\begin{frame}{Theoretical Solution}
\noindent
Let us solve the given equation theoretically and then verify the solution computationally. \\
According to the question, \\

We are given three vertices of a parallelogram:
$$A(-2,3), \; B(6,7), \; C(8,3).$$
\end{frame}

\begin{frame}{Property}
\textbf{In a parallelogram, the diagonals bisect each other. So, the midpoints of the diagonals are equal.}
\end{frame}

\begin{frame}{Theoretical Solution}
Let $D(x,y)$ be the fourth vertex. \\

\[
\text{Midpoint of } AC = \text{Midpoint of } BD
\]

\[
\frac{1}{2}
\myvec{
-2+8 \\
3+3
}
=
\frac{1}{2}
\myvec{
6+x \\
7+y
}
\]

\[
\myvec{
3 \\
3
}
=
\myvec{
\frac{6+x}{2} \\
\frac{7+y}{2}
}
\]

\[
\frac{6+x}{2} = 3, \quad \frac{7+y}{2} = 3
\]

\[
x = 0, \quad y = -1
\]

\[
\therefore \; D(0,-1)
\]
\end{frame}

\begin{frame}[fragile]
\frametitle{C-code}
\begin{lstlisting}
    
    #include <stdio.h>

int main() {
    // Given vertices
    int x1 , y1 ;  // A
    int x2 , y2 ;   // B
    int x3 , y3 ;   // C
    int x, y;             // D (to be calculated)

    // Using midpoint property: midpoint of AC = midpoint of BD
    x = x1 + x3 - x2;  // Derived formula
    y = y1 + y3 - y2;

    return 0;
}

\end{lstlisting}
\end{frame}

\begin{frame}{Plot}
    \centering
    \includegraphics[width=\columnwidth, height=0.8\textheight, keepaspectratio]{figs/fig1.png} 
    \label{The visual of the parallelogram with vertices A, B, C, D and diagonals shown}
\end{frame}

\begin{frame}[fragile]
\frametitle{Python code for plot}
\begin{lstlisting}
    import matplotlib.pyplot as plt

# Given points
A = (-2, 3)
B = (6, 7)
C = (8, 3)
D = (0, -1)  # calculated fourth vertex

# Plotting the parallelogram
x_coords = [A[0], B[0], C[0], D[0], A[0]]
y_coords = [A[1], B[1], C[1], D[1], A[1]]

plt.figure(figsize=(6,6))
plt.plot(x_coords, y_coords, 'b-o')

# Plot diagonals
plt.plot([A[0], C[0]], [A[1], C[1]], 'r--', label='Diagonal AC')
plt.plot([B[0], D[0]], [B[1], D[1]], 'g--', label='Diagonal BD')


\end{lstlisting}
\end{frame}


\begin{frame}[fragile]
\frametitle{Python code for plot}
\begin{lstlisting}


# Label points
plt.text(A[0]-0.4, A[1]-0.3, 'A(-2,3)', fontsize=10)
plt.text(B[0]+0.1, B[1], 'B(6,7)', fontsize=10)
plt.text(C[0]+0.1, C[1]-0.3, 'C(8,3)', fontsize=10)
plt.text(D[0]-0.6, D[1]-0.3, 'D(0,-1)', fontsize=10)

# Axes and grid
plt.axhline(0, color='black', linewidth=0.5)
plt.axvline(0, color='black', linewidth=0.5)
plt.grid(True, linestyle='--', alpha=0.5)

# Title and legend
plt.legend()
plt.title("Parallelogram ABCD with diagonals AC and BD")

\end{lstlisting}
\end{frame}

\begin{frame}{Conclusion}
From the figure it is clearly verified that the theoretical solution matches with the computational solution.
\end{frame}

\end{document}




