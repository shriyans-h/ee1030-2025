\let\negmedspace\undefined
\let\negthickspace\undefined
\documentclass[journal,12pt,onecolumn]{IEEEtran}
\usepackage{cite}
\usepackage{amsmath,amssymb,amsfonts,amsthm}
\usepackage{algorithmic}
\usepackage{graphicx}
\usepackage{textcomp}
\usepackage{xcolor}
\usepackage{txfonts}
\usepackage{listings}
\usepackage{enumitem}
\usepackage{mathtools}
\usepackage{gensymb}
\usepackage{comment}
\usepackage[breaklinks=true]{hyperref}
\usepackage{tkz-euclide} 
\usepackage{gvv}                                        
%\def\inputGnumericTable{}                                 
\usepackage[latin1]{inputenc}     
\usepackage{xparse}
\usepackage{color}                                            
\usepackage{array}                                            
\usepackage{longtable}                                       
\usepackage{calc}                                             
\usepackage{multirow}
\usepackage{multicol}
\usepackage{hhline}                                           
\usepackage{ifthen}                                           
\usepackage{lscape}
\usepackage{tabularx}
\usepackage{array}
\usepackage{float}
\newtheorem{theorem}{Theorem}[section]
\newtheorem{problem}{Problem}
\newtheorem{proposition}{Proposition}[section]
\newtheorem{lemma}{Lemma}[section]
\newtheorem{corollary}[theorem]{Corollary}
\newtheorem{example}{Example}[section]
\newtheorem{definition}[problem]{Definition}
\newcommand{\BEQA}{\begin{eqnarray}}
\newcommand{\EEQA}{\end{eqnarray}}
\newcommand{\define}{\stackrel{\triangle}{=}}
\theoremstyle{remark}
\newtheorem{rem}{Remark}
% Marks the beginning of the document
\begin{document}
\title{1.7.8}
\author{AI25btech11022 - Narshitha}
\maketitle
\renewcommand{\thefigure}{\theenumi}
\renewcommand{\thetable}{\theenumi}
\textbf{Question}:\\
  Using vectors , prove that the points $\brak{2,-1,3} ,\brak{3,-5,1}$ and $\brak{-1,11,9}$ are collinear.\\
\textbf{Solution}:\\
Let 
\begin{align}
\vec{A}=\myvec{2\\-1\\3}  , \vec{B}=\myvec{3\\-5\\1} , \vec{C}=\myvec{-1\\11\\9}
\end{align}
For the points to be collinear, the following condition should be satisfied.
\begin{align}
    rank\myvec{\vec{B-A} & \vec{C-A}} =1 \\
    \myvec{\vec{B-A} & \vec{C-A}}^T = \myvec{1 & -4 & -2\\ -3 & 12 & 6} 
\end{align} 
By doing $R_2 =3R_1 +R_2$ we get 
\begin{align}
    \myvec{\vec{B-A} & \vec{C-A}}^T = \myvec{1 & -4 & -2\\ 0 & 0 & 0} 
\end{align}
As $rank=1$ \\
$\therefore$ The points are collinear 
\begin{figure}
    \centering
    \includegraphics[width=0.9\linewidth]{figs/fig1.png}
    \caption{ }
    \label{fig:placeholder}
\end{figure}
\end{document}
