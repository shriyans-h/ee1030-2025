\documentclass[12pt]{article}
\usepackage{hyperref}
\usepackage{listings}
\usepackage[margin=1in]{geometry}
\usepackage{enumitem}
\usepackage{multicol}
\usepackage{array}
\usepackage{titlesec}
\usepackage{helvet}
\renewcommand{\familydefault}{\sfdefault}
\usepackage{amsmath}     % For math equations
\usepackage{amssymb}     % For advanced math symbols
\usepackage{amsfonts} % For math fonts
\usepackage{gvv}
\usepackage{esint}
\usepackage[utf8]{inputenc}
\usepackage{graphicx}
\usepackage{pgfplots}
\pgfplotsset{compat=1.18}
\titleformat{\section}{\bfseries\large}{\thesection.}{1em}{}
\setlength{\parindent}{0pt}
\setlength{\parskip}{6pt}
\usepackage{multirow}
\usepackage{float}
\usepackage{caption}


\begin{document}

\section*{Problem 8.3.12}
Find the equation of the set of all points the sum of whose distances 
from the points $(3,0)$ and $(9,0)$ is $12$.

\section*{Input Variables}
\begin{table}[H]
\centering
\begin{tabular}{|c|c|}
\hline
Variable & Value \\
\hline
$\vec{F_1}$ & $\myvec{3\\0}$ \\
\hline
$\vec{F_2}$ & $\myvec{9\\0}$ \\
\hline
$2a$ & $12$ \\
\hline
\end{tabular}
\caption{} \label{}
\end{table}

\section*{Solution}

\subsection*{Step 1: Center and directions}
\begin{align}
\vec{c} &= \frac{\vec{F_1}+\vec{F_2}}{2} 
= \myvec{6\\0} \\[6pt]
\vec{p_1} &= \frac{\vec{F_2}-\vec{F_1}}{\|\vec{F_2}-\vec{F_1}\|}
= \myvec{1\\0}, \quad 
\vec{p_2} = \myvec{0\\1}, \quad 
P = I
\end{align}

\subsection*{Step 2: Semi-minor axis}
\begin{align}
c_f &= \frac{\|\vec{F_2}-\vec{F_1}\|}{2} = 3 \\[6pt]
a &= 6 \\[6pt]
b^2 &= a^2 - c_f^2 = 36 - 9 = 27
\end{align}

\subsection*{Step 3: Standard ellipse form}
\begin{align}
(\vec{x}-\vec{c})^\top D (\vec{x}-\vec{c}) &= 1 \\[6pt]
D &= \myvec{1/a^2 & 0 \\ 0 & 1/b^2} 
= \myvec{1/36 & 0 \\ 0 & 1/27} \\[6pt]
V &= P D P^\top = D
\end{align}

\subsection*{Step 4: Convert to general quadratic form}
\begin{align}
(\vec{x}-\vec{c})^\top V (\vec{x}-\vec{c}) &= 1 \\[6pt]
\vec{x}^\top V \vec{x} - 2 \vec{c}^\top V \vec{x} + \vec{c}^\top V \vec{c} - 1 &= 0
\end{align}

Comparing with $\vec{x}^\top V \vec{x} + 2 \vec{u}^\top \vec{x} + f = 0$:
\begin{align}
\vec{u} &= -V\vec{c} \\[6pt]
f &= \vec{c}^\top V \vec{c} - 1
\end{align}

Compute:
\begin{align}
\vec{u} &= -\myvec{1/36 & 0 \\ 0 & 1/27}\myvec{6\\0} 
= \myvec{-1/6\\0} \\[6pt]
f &= \myvec{6 & 0}\myvec{1/36 & 0 \\ 0 & 1/27}\myvec{6\\0} - 1 = 0
\end{align}

\subsection*{Step 5: Clear denominators}
\begin{align}
V &= \myvec{3 & 0 \\ 0 & 4} \\[6pt]
\vec{u} &= \myvec{-18 \\ 0} \\[6pt]
f &= 0
\end{align}

\subsection*{Final Matrix Equation}
\begin{align}
\boxed{ \; \vec{x}^\top V \vec{x} + 2 \vec{u}^\top \vec{x} + f = 0, \quad
V = \myvec{3 & 0 \\ 0 & 4}, \;
\vec{u} = \myvec{-18 \\ 0}, \;
f = 0 \; }
\end{align}

\begin{figure}[H]
    \centering
    \includegraphics[width=0.9\columnwidth]{figs/ellipse.png}
    \caption{}
    \label{fig:placeholder}
\end{figure}


\end{document}
