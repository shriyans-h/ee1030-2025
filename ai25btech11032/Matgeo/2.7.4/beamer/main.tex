\documentclass{beamer}
\mode<presentation>
\usepackage{amsmath}
\usepackage{amssymb}
%\usepackage{advdate}
\usepackage{adjustbox}
\usepackage{subcaption}
\usepackage{enumitem}
\usepackage{multicol}
\usepackage{mathtools}
\usepackage{listings}
\usepackage{float}
\usepackage{graphicx}
\usepackage{url}
\def\UrlBreaks{\do\/\do-}
\usetheme{Boadilla}
\usecolortheme{lily}
\setbeamertemplate{footline}
{
  \leavevmode%
  \hbox{%
  \begin{beamercolorbox}[wd=\paperwidth,ht=2.25ex,dp=1ex,right]{author in head/foot}%
    \insertframenumber{} / \inserttotalframenumber\hspace*{2ex} 
  \end{beamercolorbox}}%
  \vskip0pt%
}
\setbeamertemplate{navigation symbols}{}

\providecommand{\nCr}[2]{\,^{#1}C_{#2}} % nCr
\providecommand{\nPr}[2]{\,^{#1}P_{#2}} % nPr
\providecommand{\mbf}{\mathbf}
\providecommand{\pr}[1]{\ensuremath{\Pr\left(#1\right)}}
\providecommand{\qfunc}[1]{\ensuremath{Q\left(#1\right)}}
\providecommand{\sbrak}[1]{\ensuremath{{}\left[#1\right]}}
\providecommand{\lsbrak}[1]{\ensuremath{{}\left[#1\right.}}
\providecommand{\rsbrak}[1]{\ensuremath{{}\left.#1\right]}}
\providecommand{\brak}[1]{\ensuremath{\left(#1\right)}}
\providecommand{\lbrak}[1]{\ensuremath{\left(#1\right.}}
\providecommand{\rbrak}[1]{\ensuremath{\left.#1\right)}}
\providecommand{\cbrak}[1]{\ensuremath{\left\{#1\right\}}}
\providecommand{\lcbrak}[1]{\ensuremath{\left\{#1\right.}}
\providecommand{\rcbrak}[1]{\ensuremath{\left.#1\right\}}}
\theoremstyle{remark}
\newtheorem{rem}{Remark}
\newcommand{\sgn}{\mathop{\mathrm{sgn}}}
\providecommand{\abs}[1]{\left\vert#1\right\vert}
\providecommand{\res}[1]{\Res\displaylimits_{#1}} 
\providecommand{\norm}[1]{\lVert#1\rVert}
\providecommand{\mtx}[1]{\mathbf{#1}}
\providecommand{\mean}[1]{E\left[ #1 \right]}
\providecommand{\fourier}{\overset{\mathcal{F}}{ \rightleftharpoons}}
%\providecommand{\hilbert}{\overset{\mathcal{H}}{ \rightleftharpoons}}
\providecommand{\system}{\overset{\mathcal{H}}{ \longleftrightarrow}}
	%\newcommand{\solution}[2]{\textbf{Solution:}{#1}}
%\newcommand{\solution}{\noindent \textbf{Solution: }}
\providecommand{\dec}[2]{\ensuremath{\overset{#1}{\underset{#2}{\gtrless}}}}
\newcommand{\myvec}[1]{\ensuremath{\begin{pmatrix}#1\end{pmatrix}}}
\let\vec\mathbf

\lstset{
language=C,
frame=single, 
breaklines=true,
columns=fullflexible
}

\numberwithin{equation}{section}

\title{Presentation - Matgeo}
\author{Aryansingh Sonaye \\
AI25BTECH11032 \\
EE1030 - Matrix Theory}

\date{\today} 
\begin{document}

\begin{frame}
\titlepage
\end{frame}

\section{Problem}
\begin{frame}
\frametitle{Problem Statement}
If
\begin{align}
\vec a = \myvec{2 \\1 \\3},\qquad
\vec b = \myvec{-1 \\2 \\1},\qquad
\vec c = \myvec{3 \\1 \\2},
\end{align}

find $\vec a\cdot(\vec b\times\vec c)$.
\end{frame}

\section{Solution}
\subsection{Description of Variables used}
\begin{frame}
\frametitle{Description of Variables used}
\begin{table}[H]
\centering
\begin{tabular}[12pt]{ |c| c|}
    \hline
    \textbf{Input variable} & \textbf{Value}\\ 
    \hline
    $\vec{a}$ & \myvec{2 \\1 \\3} \\
    \hline 
    $\vec{b}$ & \myvec{-1 \\2 \\1}\\
    \hline
    $\vec{c}$ & \myvec{3 \\1 \\2}\\
    \hline
    \end{tabular}
    \caption{}
    \label{}
 \end{table}


\end{frame}

\subsection{Theoretical Solution }
\begin{frame}
\frametitle{Theoretical Solution}
Write the vectors in component form:
\begin{align}
\vec a=\myvec{2\\[4pt]1\\[4pt]3},\qquad
\vec b=\myvec{-1\\[4pt]2\\[4pt]1},\qquad
\vec c=\myvec{3\\[4pt]1\\[4pt]2}.
\end{align}

Using the minor notation from the problem statement, where
\begin{align}
\vec{B_{ij}}=\myvec{b_i\\[4pt]b_j},\qquad \vec{C_{ij}}=\myvec{c_i\\[4pt]c_j},
\end{align}
we write the cross product :
\begin{align}
\vec b\times\vec c \;=\;
\myvec{\,\lvert \vec{B_{23}}\;\vec{C_{23}}\rvert \\[6pt] \lvert \vec{B_{31}}\;\vec{C_{31}}\rvert \\[6pt] \lvert \vec{B_{12}}\;\vec{C_{12}}\rvert\, }.
\end{align}
\end{frame}

\begin{frame}
\frametitle{Theoretical Solution}
Substituting the components gives
\begin{align}
\vec b\times\vec c = \myvec{3\\[4pt]5\\[4pt]-7}.
\end{align}

Now use the transpose (row-vector) method for the dot product:
\begin{align}
\vec a^{\!T}(\vec b\times\vec c)
= \myvec{2\\[4pt]1\\[4pt]3}^{\!T}\,\myvec{3\\[4pt]5\\[4pt]-7}
= \myvec{2 & 1 & 3}\,\myvec{3\\[4pt]5\\[4pt]-7}
= 2\cdot 3 + 1\cdot 5 + 3\cdot(-7) = -10.
\end{align}

Thus
\begin{align}
\boxed{\ \vec a\cdot(\vec b\times\vec c) = -10\ }.
\end{align}

\end{frame}

\subsection{Plot}
\begin{frame}
    \frametitle{Plot}
\begin{figure}[H]
   \centering
   \includegraphics[width=0.9\columnwidth]{figs/triple_product.png}
   \end{figure}
\end{frame}

\begin{frame}[fragile]
    \frametitle{Code - C}
    \begin{lstlisting}
#include <stdio.h>

// Cross product of two 3D vectors
void cross_product(double a[3], double b[3], double result[3]) {
    result[0] = a[1]*b[2] - a[2]*b[1];
    result[1] = a[2]*b[0] - a[0]*b[2];
    result[2] = a[0]*b[1] - a[1]*b[0];
}

// Dot product of two 3D vectors
double dot_product(double a[3], double b[3]) {
    return a[0]*b[0] + a[1]*b[1] + a[2]*b[2];
}

\end{lstlisting}
\end{frame}

\begin{frame}[fragile]
    \frametitle{Code - Python(with shared C code)}
    The code to obtain the required plot is
    \begin{lstlisting}
import ctypes
import numpy as np
import matplotlib.pyplot as plt

# Load compiled C library
lib = ctypes.CDLL('./libvecops.so')

# Argument/return types
lib.cross_product.argtypes = [ctypes.POINTER(ctypes.c_double),
                              ctypes.POINTER(ctypes.c_double),
                              ctypes.POINTER(ctypes.c_double)]
lib.dot_product.argtypes = [ctypes.POINTER(ctypes.c_double),
                            ctypes.POINTER(ctypes.c_double)]
lib.dot_product.restype = ctypes.c_double

\end{lstlisting}
\end{frame}
\begin{frame}[fragile]
\frametitle{Code - Python(with shared C code)}
\begin{lstlisting}
# Helper type
DoubleArray3 = ctypes.c_double * 3

# Define vectors
a = np.array([2.0, 1.0, 3.0])
b = np.array([-1.0, 2.0, 1.0])
c = np.array([3.0, 1.0, 2.0])

# Cross product (via C)
cross_res = DoubleArray3()
lib.cross_product(DoubleArray3(*b), DoubleArray3(*c), cross_res)
bx_c = np.array([cross_res[i] for i in range(3)])

# Dot product (via C)
scalar_triple = lib.dot_product(DoubleArray3(*a), cross_res)

\end{lstlisting}
\end{frame}

\begin{frame}[fragile]
\frametitle{Code - Python(with shared C code)}
\begin{lstlisting}
print("b x c =", bx_c)
print("a . (b x c) =", scalar_triple)

# ---------- Image Generation ----------
fig = plt.figure()
ax = fig.add_subplot(111, projection='3d')

def draw_vec(v, color, label):
    ax.quiver(0, 0, 0, v[0], v[1], v[2],
              color=color, arrow_length_ratio=0.1, label=label)

draw_vec(a, 'r', 'a')
draw_vec(b, 'g', 'b')
draw_vec(c, 'b', 'c')
draw_vec(bx_c, 'm', 'b x c')

\end{lstlisting}
\end{frame}

\begin{frame}[fragile]
\frametitle{Code - Python(with shared C code)}
\begin{lstlisting}
lim = 4
ax.set_xlim([-lim, lim])
ax.set_ylim([-lim, lim])
ax.set_zlim([-lim, lim])

ax.set_xlabel('X')
ax.set_ylabel('Y')
ax.set_zlabel('Z')
ax.legend()

plt.title(f"Scalar triple product = {scalar_triple}")

# Save the image instead of just showing
plt.savefig("/sdcard/ee1030-2025/ai25btech11032/Matgeo/2.7.4/figs/triple_product.png", dpi=300)
plt.show()

\end{lstlisting}
\end{frame}

\begin{frame}[fragile]
\frametitle{Code - Python only}
\begin{lstlisting}

import numpy as np
import matplotlib.pyplot as plt

# Define vectors
a = np.array([2.0, 1.0, 3.0])
b = np.array([-1.0, 2.0, 1.0])
c = np.array([3.0, 1.0, 2.0])

# Cross product (NumPy)
bx_c = np.cross(b, c)

# Scalar triple product
scalar_triple = np.dot(a, bx_c)

\end{lstlisting}
\end{frame}
\begin{frame}[fragile]
\frametitle{Code - Python only}
\begin{lstlisting}
# Print results
print("b x c =", bx_c)
print("a . (b x c) =", scalar_triple)

# ---------- Image Generation ----------
fig = plt.figure()
ax = fig.add_subplot(111, projection='3d')

def draw_vec(v, color, label):
    ax.quiver(0, 0, 0, v[0], v[1], v[2],
              color=color, arrow_length_ratio=0.1, label=label)

# Draw vectors
draw_vec(a, 'r', 'a')
draw_vec(b, 'g', 'b')
draw_vec(c, 'b', 'c')
draw_vec(bx_c, 'm', 'b x c')

\end{lstlisting}
\end{frame}

\begin{frame}[fragile]
\frametitle{Code - Python only}
\begin{lstlisting}
# Axis limits
lim = 4
ax.set_xlim([-lim, lim])
ax.set_ylim([-lim, lim])
ax.set_zlim([-lim, lim])

# Labels and title
ax.set_xlabel('X')
ax.set_ylabel('Y')
ax.set_zlabel('Z')
ax.legend()

plt.title(f"Scalar triple product = {scalar_triple}")

# Save image
plt.savefig("triple_product_python.png", dpi=300)
plt.show()

\end{lstlisting}
\end{frame}

\end{document}

