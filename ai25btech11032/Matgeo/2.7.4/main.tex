\documentclass[12pt]{article}
\usepackage{hyperref}
\usepackage{listings}
\usepackage[margin=1in]{geometry}
\usepackage{enumitem}
\usepackage{multicol}
\usepackage{array}
\usepackage{titlesec}
\usepackage{helvet}
\renewcommand{\familydefault}{\sfdefault}
\usepackage{amsmath}     % For math equations
\usepackage{amssymb}     % For advanced math symbols
\usepackage{amsfonts} % For math fonts
\usepackage{gvv}
\usepackage{esint}
\usepackage[utf8]{inputenc}
\usepackage{graphicx}
\usepackage{pgfplots}
\pgfplotsset{compat=1.18}
\titleformat{\section}{\bfseries\large}{\thesection.}{1em}{}
\setlength{\parindent}{0pt}
\setlength{\parskip}{6pt}
\usepackage{multirow}
\usepackage{float}
\usepackage{caption}


\begin{document}

\section*{Problem 2.7.4}

\textbf{Problem.}
If
\begin{align}
\vec a = \myvec{2 \\1 \\3},\qquad
\vec b = \myvec{-1 \\2 \\1},\qquad
\vec c = \myvec{3 \\1 \\2},
\end{align}

find $\vec a\cdot(\vec b\times\vec c)$.


\textbf{Solution.}

\begin{table}[H]
\centering
\begin{tabular}[12pt]{ |c| c|}
    \hline
    \textbf{Input variable} & \textbf{Value}\\ 
    \hline
    $\vec{a}$ & \myvec{2 \\1 \\3} \\
    \hline 
    $\vec{b}$ & \myvec{-1 \\2 \\1}\\
    \hline
    $\vec{c}$ & \myvec{3 \\1 \\2}\\
    \hline
    \end{tabular}
    \caption{}
    \label{}
 \end{table}
Write the vectors in component form:
\begin{align}
\vec a=\myvec{2\\[4pt]1\\[4pt]3},\qquad
\vec b=\myvec{-1\\[4pt]2\\[4pt]1},\qquad
\vec c=\myvec{3\\[4pt]1\\[4pt]2}.
\end{align}

Using the minor notation from the problem statement, where
\begin{align}
\vec{B_{ij}}=\myvec{b_i\\[4pt]b_j},\qquad \vec{C_{ij}}=\myvec{c_i\\[4pt]c_j},
\end{align}
we write the cross product :
\begin{align}
\vec b\times\vec c \;=\;
\myvec{\,\lvert \vec{B_{23}}\;\vec{C_{23}}\rvert \\[6pt] \lvert \vec{B_{31}}\;\vec{C_{31}}\rvert \\[6pt] \lvert \vec{B_{12}}\;\vec{C_{12}}\rvert\, }.
\end{align}

Substituting the components gives
\begin{align}
\vec b\times\vec c = \myvec{3\\[4pt]5\\[4pt]-7}.
\end{align}

Now use the transpose (row-vector) method for the dot product:
\begin{align}
\vec a^{\!T}(\vec b\times\vec c)
= \myvec{2\\[4pt]1\\[4pt]3}^{\!T}\,\myvec{3\\[4pt]5\\[4pt]-7}
= \myvec{2 & 1 & 3}\,\myvec{3\\[4pt]5\\[4pt]-7}
= 2\cdot 3 + 1\cdot 5 + 3\cdot(-7) = -10.
\end{align}

Thus
\begin{align}
\boxed{\ \vec a\cdot(\vec b\times\vec c) = -10\ }.
\end{align}

\begin{figure}[H]
    \centering
    \includegraphics[width=0.9\columnwidth]{figs/triple_product.png}
    \caption{}
    \label{fig:placeholder}
\end{figure}


\end{document}
