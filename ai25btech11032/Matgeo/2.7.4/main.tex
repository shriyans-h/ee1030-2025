\documentclass[12pt]{article}
\usepackage{hyperref}
\usepackage{listings}
\usepackage[margin=1in]{geometry}
\usepackage{enumitem}
\usepackage{multicol}
\usepackage{array}
\usepackage{titlesec}
\usepackage{helvet}
\renewcommand{\familydefault}{\sfdefault}
\usepackage{amsmath}     % For math equations
\usepackage{amssymb}     % For advanced math symbols
\usepackage{amsfonts} % For math fonts
\usepackage{gvv}
\usepackage{esint}
\usepackage[utf8]{inputenc}
\usepackage{graphicx}
\usepackage{pgfplots}
\pgfplotsset{compat=1.18}
\titleformat{\section}{\bfseries\large}{\thesection.}{1em}{}
\setlength{\parindent}{0pt}
\setlength{\parskip}{6pt}
\usepackage{multirow}
\usepackage{float}
\usepackage{caption}


\begin{document}

\section*{Problem 2.7.4}

\textbf{Problem.}
If
\begin{align}
\vec a = \myvec{2 \\1 \\3},\qquad
\vec b = \myvec{-1 \\2 \\1},\qquad
\vec c = \myvec{3 \\1 \\2},
\end{align}

find $\vec a\cdot(\vec b\times\vec c)$.


\textbf{Solution.}

\begin{table}[H]
\centering
\begin{tabular}[12pt]{ |c| c|}
    \hline
    \textbf{Input variable} & \textbf{Value}\\ 
    \hline
    $\vec{a}$ & \myvec{2 \\1 \\3} \\
    \hline 
    $\vec{b}$ & \myvec{-1 \\2 \\1}\\
    \hline
    $\vec{c}$ & \myvec{3 \\1 \\2}\\
    \hline
    \end{tabular}
    \caption{}
    \label{}
 \end{table}

We are asked to compute:
\begin{align}
\vec{a} \cdot (\vec{b} \times \vec{c}).
\end{align}

\subsection*{Step 1 — Vectors as column matrices}
\begin{align}
\vec{a} &= \myvec{2\\1\\3}, &
\vec{b} &= \myvec{-1\\2\\1}, &
\vec{c} &= \myvec{3\\1\\2}.
\end{align}

\subsection*{Step 2 — Form the Gram matrix}
The Gram matrix is
\begin{align}
G &=
\myvec{
\vec{a}^T \vec{a} & \vec{a}^T \vec{b} & \vec{a}^T \vec{c} \\
\vec{b}^T \vec{a} & \vec{b}^T \vec{b} & \vec{b}^T \vec{c} \\
\vec{c}^T \vec{a} & \vec{c}^T \vec{b} & \vec{c}^T \vec{c}
}.
\end{align}

Compute each entry:
\begin{align}
\vec{a}^T \vec{a} &= 14, &
\vec{b}^T \vec{b} &= 6, &
\vec{c}^T \vec{c} &= 14, \\
\vec{a}^T \vec{b} &= 3, &
\vec{b}^T \vec{c} &= 1, &
\vec{c}^T \vec{a} &= 13.
\end{align}

Thus,
\begin{align}
G = \myvec{
14 & 3 & 13 \\
3 & 6 & 1 \\
13 & 1 & 14
}.
\end{align}

\subsection*{Step 3 — Gram determinant identity}
We know
\begin{align}
\det(G) &= \big(\det([\vec{a}\ \vec{b}\ \vec{c}])\big)^2 \\
        &= \big(\vec{a} \cdot (\vec{b} \times \vec{c})\big)^2.
\end{align}

Direct computation gives
\begin{align}
\det(G) = 100.
\end{align}

Hence
\begin{align}
|\vec{a} \cdot (\vec{b} \times \vec{c})| &= \sqrt{100} = 10.
\end{align}

\subsection*{Step 4 — Find the sign}
Form the matrix
\begin{align}
A = [\vec{a}\ \vec{b}\ \vec{c}] =
\myvec{
2 & -1 & 3 \\
1 & 2  & 1 \\
3 & 1  & 2
}.
\end{align}

Then
\begin{align}
\det(A) = \vec{a} \cdot (\vec{b} \times \vec{c}).
\end{align}

Compute:
\begin{align}
\det(A) = -10.
\end{align}

\subsection*{Final Answer}
\begin{align}
\boxed{\ \vec{a} \cdot (\vec{b} \times \vec{c}) = -10\ }
\end{align}

\begin{figure}[H]
    \centering
    \includegraphics[width=0.9\columnwidth]{figs/gram_triple_product.png}
    \caption{}
    \label{fig:placeholder}
\end{figure}



\end{document}
