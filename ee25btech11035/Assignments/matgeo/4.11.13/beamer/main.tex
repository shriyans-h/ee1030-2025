\documentclass[notheorems]{beamer}
\let\vec\mathbf
\mode<presentation>
\usepackage{amsmath}
\usepackage{amssymb}
\usepackage{xparse}
%\usepackage{advdate}
\usepackage{adjustbox}
%\usepackage{subcaption}
\usepackage{enumitem}
\usepackage{multicol}
\usepackage{mathtools}
\usepackage{listings}
\usepackage{amsthm}
\usepackage{gvv}
\usepackage{url}
\usetheme{Boadilla}
\usecolortheme{lily}
\setbeamertemplate{footline}
{
  \leavevmode%
  \hbox{%
  \begin{beamercolorbox}[wd=\paperwidth,ht=2.25ex,dp=1ex,right]{author in head/foot}%
    \insertframenumber{} / \inserttotalframenumber\hspace*{2ex} 
  \end{beamercolorbox}}%
  \vskip0pt%
}
\setbeamertemplate{navigation symbols}{}
\providecommand{\nCr}[2]{\,^{#1}C_{#2}} % nCr
\providecommand{\nPr}[2]{\,^{#1}P_{#2}} % nPr
\providecommand{\mbf}{\mathbf}
\providecommand{\pr}[1]{\ensuremath{\Pr\left(#1\right)}}
\providecommand{\qfunc}[1]{\ensuremath{Q\left(#1\right)}}
\providecommand{\sbrak}[1]{\ensuremath{{}\left[#1\right]}}
\providecommand{\lsbrak}[1]{\ensuremath{{}\left[#1\right.}}
\providecommand{\rsbrak}[1]{\ensuremath{{}\left.#1\right]}}
\providecommand{\brak}[1]{\ensuremath{\left(#1\right)}}
\providecommand{\lbrak}[1]{\ensuremath{\left(#1\right.}}
\providecommand{\rbrak}[1]{\ensuremath{\left.#1\right)}}
\providecommand{\cbrak}[1]{\ensuremath{\left\{#1\right\}}}
\providecommand{\lcbrak}[1]{\ensuremath{\left\{#1\right.}}
\providecommand{\rcbrak}[1]{\ensuremath{\left.#1\right\}}}
\theoremstyle{remark}

\providecommand{\res}[1]{\Res\displaylimits_{#1}} 
\providecommand{\norm}[1]{\lVert#1\rVert}
\providecommand{\mtx}[1]{\mathbf{#1}}

\providecommand{\fourier}{\overset{\mathcal{F}}{ \rightleftharpoons}}
%\providecommand{\hilbert}{\overset{\mathcal{H}}{ \rightleftharpoons}}
\providecommand{\system}{\overset{\mathcal{H}}{ \longleftrightarrow}}
	%\newcommand{\solution}[2]{\textbf{Solution:}{#1}}
%\newcommand{\solution}{\noindent \textbf{Solution: }}
\providecommand{\dec}[2]{\ensuremath{\overset{#1}{\underset{#2}{\gtrless}}}}

\title{Matrices in Geometry - 4.11.13}
\author{EE25BTECH11035  Kushal B N}
\date{Sep, 2025}

\begin{document}

\maketitle

\section{Problem Statement}
\begin{frame}
\frametitle{Problem Statement}
Find the equation of the plane passing through the points (2,5,-3), (-2,-3,5), and (5,3,-3). Also find the point of intersection of this plane with the line passing through points (3,1,5) and (-1,-3,-1).
\end{frame}

\section{Solution}
\begin{frame}{Solution}
Given,\\
The points $\vec{A}\myvec{2\\5\\-3}$, $\vec{B}\myvec{-2\\-3\\5}$ and $\vec{C}\myvec{5\\3\\-3}$ which pass through a plane.\\
The points $\vec{D}\myvec{3\\1\\5}$ and $\vec{E}\myvec{-1\\-3\\-1}$ which pass through a line.
\end{frame}

\begin{frame}{Solution}
If $\vec{n}$ is the normal vector to the plane, then by the plane equation we can write\\
\begin{equation}
    \vec{n}^{\top}\vec{A} = c
\end{equation}
\begin{equation}
    \vec{n}^{\top}\vec{B} = c
\end{equation}
\begin{equation}
    \vec{n}^{\top}\vec{C} = c
\end{equation}

which can be writeen as
\begin{equation}
    \myvec{A&B&C}^{\top}\vec{n} = \myvec{1\\1\\1}
\end{equation}
\end{frame}

\begin{frame}{Solution}

Forming the augmented matrix for this
\begin{equation}
    \augvec{3}{3}{2&5&-3&1\\-2&-3&5&1\\5&3&-3&1}
\end{equation}

 \begin{equation}
    \augvec{3}{3}{2&5&-3&1\\-2&-3&5&1\\5&3&-3&1} \xleftrightarrow{R_1 \leftarrow \frac{R_1}{2}} \augvec{3}{3}{1&\frac{5}{2}&\frac{-3}{2}&\frac{1}{2}\\-2&-3&5&1\\5&3&-3&1}
 \end{equation}

 \begin{equation}
  \xleftrightarrow{R_2 \leftarrow \frac{R_2}{2}} \augvec{3}{3}{1&\frac{5}{2}&\frac{-3}{2}&\frac{1}{2}\\0&1&1&1\\0&\frac{-19}{2}&\frac{9}{2}&\frac{-3}{2}} \underset{R_1 \leftarrow R_1 - \frac{5}{2}R_2}{\xleftrightarrow{R_3 \leftarrow R_3 + \frac{19}{2}R_2}} \augvec{3}{3}{1&0&-4&-2\\0&1&1&1\\0&0&14&8}
 \end{equation}

\end{frame}

\begin{frame}{Solution}
\begin{equation}
    \xleftrightarrow{R_3 \leftarrow \frac{1}{14} R_3} \augvec{3}{3}{1&0&-4&-2\\0&1&1&1\\0&0&1&\frac{4}{7}} \underset{R_2 \leftarrow R_2 - R_3}{\xleftrightarrow{R_1 \leftarrow R_1 + 4R_3}} \augvec{3}{3}{1&0&0&\frac{2}{7}\\0&1&0&\frac{3}{7}\\0&0&1&\frac{4}{7}}
\end{equation}

Thus, multiplying by 7, the plane equation can be expressed as
\begin{equation}
    \myvec{2&3&4}\vec{x} = 7
\end{equation}

Now, the line passing through the two given points in parametric form
\begin{equation}
    \vec{x} = \vec{P} + \lambda \vec{m}
\end{equation}
where $\vec{P} = \myvec{3\\1\\5}$ and $\vec{m} = \myvec{3\\1\\5} - \myvec{-1\\-3\\-1} = \myvec{4\\4\\6} \equiv \myvec{2\\2\\3}$,
\end{frame}

\begin{frame}{Solution}
    Now substituting the parametric form in the plane equation $\vec{n}^{\top}\vec{x} = c$ where here $\vec{n} = \myvec{2\\3\\4}$ and $c = 7$,

\begin{equation}
    \vec{n}^{\top}\brak{\vec{P} + \lambda \vec{m}} = c
\end{equation}

\begin{equation}
    \vec{n}^{\top}\vec{P} + \lambda \vec{n}^{\top}\vec{m} = c
\end{equation}

\begin{equation}
    \implies \lambda = \frac{c - \vec{n}^{\top}\vec{P}}{\vec{n}^{\top}\vec{m}}
\end{equation}

\end{frame}

\begin{frame}{Solution}
So that, 
\begin{equation}
  \vec{x} = \vec{P} + \brak{\frac{c - \vec{n}^{\top}\vec{P}}{\vec{n}^{\top}\vec{m}}}\vec{m}
\end{equation}

Substituting the values in equation (14) we get the intersection point as,

\begin{equation}
    \implies \fbox{$\vec{x} = \myvec{1\\-1\\2}$}
\end{equation}
\end{frame}

\section{Conclusion}
\begin{frame}{Conclusion}
$\therefore$ The equation of the plane is $\myvec{2&3&4}\myvec{x\\y\\z}=7$ and the point of intersection of this plane with the line through the given two points is $\vec{P}\myvec{1\\-1\\2}$.
\begin{figure}[H]
    \centering
    \includegraphics[width=0.55\columnwidth]{figs/1.png}
    \caption{}
\end{figure}
\end{frame}
\end{document}

