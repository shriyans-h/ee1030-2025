\documentclass{beamer}
\let\vec\mathbf
\mode<presentation>
\usepackage{amsmath}
\usepackage{amssymb}
%\usepackage{advdate}
\usepackage{adjustbox}
%\usepackage{subcaption}
%\usepackage{enumitem}
\usepackage{multicol}
\usepackage{mathtools}
\usepackage{listings}
\usepackage{url}
\usetheme{Boadilla}
\usecolortheme{lily}
\setbeamertemplate{footline}
{
  \leavevmode%
  \hbox{%
  \begin{beamercolorbox}[wd=\paperwidth,ht=2.25ex,dp=1ex,right]{author in head/foot}%
    \insertframenumber{} / \inserttotalframenumber\hspace*{2ex} 
  \end{beamercolorbox}}%
  \vskip0pt%
}
\setbeamertemplate{navigation symbols}{}
\providecommand{\nCr}[2]{\,^{#1}C_{#2}} % nCr
\providecommand{\nPr}[2]{\,^{#1}P_{#2}} % nPr
\providecommand{\mbf}{\mathbf}
\providecommand{\pr}[1]{\ensuremath{\Pr\left(#1\right)}}
\providecommand{\qfunc}[1]{\ensuremath{Q\left(#1\right)}}
\providecommand{\sbrak}[1]{\ensuremath{{}\left[#1\right]}}
\providecommand{\lsbrak}[1]{\ensuremath{{}\left[#1\right.}}
\providecommand{\rsbrak}[1]{\ensuremath{{}\left.#1\right]}}
\providecommand{\brak}[1]{\ensuremath{\left(#1\right)}}
\providecommand{\lbrak}[1]{\ensuremath{\left(#1\right.}}
\providecommand{\rbrak}[1]{\ensuremath{\left.#1\right)}}
\providecommand{\cbrak}[1]{\ensuremath{\left\{#1\right\}}}
\providecommand{\lcbrak}[1]{\ensuremath{\left\{#1\right.}}
\providecommand{\rcbrak}[1]{\ensuremath{\left.#1\right\}}}
\theoremstyle{remark}
\newtheorem{rem}{Remark}
\newcommand{\sgn}{\mathop{\mathrm{sgn}}}

\providecommand{\res}[1]{\Res\displaylimits_{#1}} 
\providecommand{\norm}[1]{\lVert#1\rVert}
\providecommand{\mtx}[1]{\mathbf{#1}}

\providecommand{\fourier}{\overset{\mathcal{F}}{ \rightleftharpoons}}
%\providecommand{\hilbert}{\overset{\mathcal{H}}{ \rightleftharpoons}}
\providecommand{\system}{\overset{\mathcal{H}}{ \longleftrightarrow}}
	%\newcommand{\solution}[2]{\textbf{Solution:}{#1}}
%\newcommand{\solution}{\noindent \textbf{Solution: }}
\providecommand{\dec}[2]{\ensuremath{\overset{#1}{\underset{#2}{\gtrless}}}}
\newcommand{\myvec}[1]{\ensuremath{\begin{pmatrix}#1\end{pmatrix}}}

\title{Matrices in Geometry - 5.13.61}
\author{EE25BTECH11035  Kushal B N}
\date{}

\begin{document}

\maketitle

\section{Problem Statement}
\begin{frame}
\frametitle{Problem Statement}
Let $\vec{P} = \myvec{1&0&0\\4&1&0\\16&4&1}$ and $\vec{I}$ be the identity matrix of order 3. If $\vec{Q} = q_{ij}$ is a matrix such that $\vec{P}^{50} - \vec{Q} = \vec{I}$, then $\frac{q_{31}+q_{32}}{q_{21}}$ equals
\hfill{\brak{JEE Adv. 2016}}
\begin{enumerate}
\begin{multicols}{4}
    \item 52
    \item 103
    \item 201
    \item 205
\end{multicols}
\end{enumerate}

\end{frame}

\section{Solution}
\begin{frame}{Solution}
Given,\\
The matrix $\vec{P} = \myvec{1&0&0\\4&1&0\\16&4&1}$ and $\vec{Q} = \vec{P}^{50} - \vec{I}$

Let us express the matrix $\vec{P}$ as\\
\begin{equation}
    \vec{P} = \vec{I} + \vec{N}
\end{equation}
where 
\begin{equation}
    \vec{N} = \myvec{0&0&0\\4&0&0\\16&4&0}
\end{equation}
Now we see that 
\begin{equation}
    \vec{N}^2 = \myvec{0&0&0\\0&0&0\\16&0&0}
\end{equation}

\begin{equation}
    \vec{N}^3 = \vec{0}
\end{equation}
\end{frame}

\begin{frame}{Solution}
So that now by binomial expansion we have,
\begin{equation}
    \vec{P}^{50} = \brak{\vec{I}+\vec{N}}^{50} 
\end{equation}

from (4),
\begin{equation}
    \implies \vec{P}^{50} = \vec{I} + 50\vec{N} + 1225\vec{N}^2
\end{equation}

\begin{equation}
    \implies \vec{Q} = 50\vec{N} + 1225\vec{N}^2
\end{equation}

\begin{equation}
    \vec{Q} = \myvec{0&0&0\\200&0&0\\20400&200&0}
\end{equation}

\begin{equation}
    \implies \fbox{$\frac{q_{31}+q_{32}}{q_{21}} = 103$}
\end{equation}
\end{frame}

\section{Conclusion}
\begin{frame}{Conclusion}
$\therefore$ The value of the given expression $\frac{q_{31}+q_{32}}{q_{21}} = 103$.\\
Hence, the correct answer is (2).
\end{frame}
\end{document}