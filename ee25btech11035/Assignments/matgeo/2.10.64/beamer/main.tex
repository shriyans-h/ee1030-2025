\documentclass{beamer}
\let\vec\mathbf
\mode<presentation>
\usepackage{amsmath}
\usepackage{amssymb}
%\usepackage{advdate}
\usepackage{adjustbox}
%\usepackage{subcaption}
\usepackage{enumitem}
\usepackage{multicol}
\usepackage{mathtools}
\usepackage{listings}
\usepackage{url}
\usetheme{Boadilla}
\usecolortheme{lily}
\setbeamertemplate{footline}
{
  \leavevmode%
  \hbox{%
  \begin{beamercolorbox}[wd=\paperwidth,ht=2.25ex,dp=1ex,right]{author in head/foot}%
    \insertframenumber{} / \inserttotalframenumber\hspace*{2ex} 
  \end{beamercolorbox}}%
  \vskip0pt%
}
\setbeamertemplate{navigation symbols}{}
\providecommand{\nCr}[2]{\,^{#1}C_{#2}} % nCr
\providecommand{\nPr}[2]{\,^{#1}P_{#2}} % nPr
\providecommand{\mbf}{\mathbf}
\providecommand{\pr}[1]{\ensuremath{\Pr\left(#1\right)}}
\providecommand{\qfunc}[1]{\ensuremath{Q\left(#1\right)}}
\providecommand{\sbrak}[1]{\ensuremath{{}\left[#1\right]}}
\providecommand{\lsbrak}[1]{\ensuremath{{}\left[#1\right.}}
\providecommand{\rsbrak}[1]{\ensuremath{{}\left.#1\right]}}
\providecommand{\brak}[1]{\ensuremath{\left(#1\right)}}
\providecommand{\lbrak}[1]{\ensuremath{\left(#1\right.}}
\providecommand{\rbrak}[1]{\ensuremath{\left.#1\right)}}
\providecommand{\cbrak}[1]{\ensuremath{\left\{#1\right\}}}
\providecommand{\lcbrak}[1]{\ensuremath{\left\{#1\right.}}
\providecommand{\rcbrak}[1]{\ensuremath{\left.#1\right\}}}
\theoremstyle{remark}
\newtheorem{rem}{Remark}
\newcommand{\sgn}{\mathop{\mathrm{sgn}}}

\providecommand{\res}[1]{\Res\displaylimits_{#1}} 
\providecommand{\norm}[1]{\lVert#1\rVert}
\providecommand{\mtx}[1]{\mathbf{#1}}

\providecommand{\fourier}{\overset{\mathcal{F}}{ \rightleftharpoons}}
%\providecommand{\hilbert}{\overset{\mathcal{H}}{ \rightleftharpoons}}
\providecommand{\system}{\overset{\mathcal{H}}{ \longleftrightarrow}}
	%\newcommand{\solution}[2]{\textbf{Solution:}{#1}}
%\newcommand{\solution}{\noindent \textbf{Solution: }}
\providecommand{\dec}[2]{\ensuremath{\overset{#1}{\underset{#2}{\gtrless}}}}
\newcommand{\myvec}[1]{\ensuremath{\begin{pmatrix}#1\end{pmatrix}}}

\title{Matrices in Geometry - 2.4.32}
\author{EE25BTECH11035  Kushal B N}
\date{Sep, 2025}

\begin{document}

\maketitle

\section{Problem Statement}
\begin{frame}
\frametitle{Problem Statement}
The position vectors of the points $\vec{A}$, $\vec{B}$, $\vec{C}$ and $\vec{D}$ are $\brak{3\hat{i}-2\hat{j}-\hat{k}}$, $\brak{2\hat{i}+3\hat{j}-4\hat{k}}$, $\brak{-\hat{i}+\hat{j}+2\hat{k}}$ and $\brak{4\hat{i}+5\hat{j}+\lambda\hat{k}}$ respectively. If the points $\vec{A}$, $\vec{B}$, $\vec{C}$ and $\vec{D}$ lie on a plane, find the value of $\lambda$.

\end{frame}

\section{Solution}
\begin{frame}{Solution}
Given,\\
$\vec{A}\myvec{3\\-2\\-1}$, $\vec{B}\myvec{2\\3\\-4}$, $\vec{C}\myvec{-1\\1\\2}$ and $\vec{D}\myvec{4\\5\\\lambda}$.

\begin{equation}
\vec{B} - \vec{A} = \myvec{-1\\5\\-3}
\end{equation}

\begin{equation}
\vec{C} - \vec{A} = \myvec{-4\\3\\3}
\end{equation}

\begin{equation}
\vec{D} - \vec{A} = \myvec{1\\7\\\lambda+1}
\end{equation}

\end{frame}

\begin{frame}{Solution}
As the points $\vec{A}$, $\vec{B}$, $\vec{C}$ and $\vec{D}$ lie on a plane, this means that the vectors $\vec{B} - \vec{A}$, $\vec{C} - \vec{A}$ and $\vec{D} - \vec{A}$ are coplanar and hence, the determinant of the matrix\\
\begin{equation}
\implies {\myvec{\vec{B} - \vec{A} & \vec{C} - \vec{A} & \vec{D} - \vec{A}}} = 0
\end{equation}

\begin{equation}
    {\myvec{-1 & -4 & 1\\5 & 3 &7\\-3 & 3 & \lambda+1}} = 0
\end{equation}

Converting this matrix into row echelon form,
\begin{equation}
 \myvec{-1 & -4 & 1\\5 & 3 &7\\-3 & 3 & \lambda+1} \overset{R_2 \rightarrow R_2 + 5R_1}{\longrightarrow} \myvec{-1 & -4 & 1\\0 & -17 & 12\\-3 & 3 & \lambda+1}
\end{equation}
\end{frame}

\begin{frame}{Solution}
\begin{equation}
    \myvec{-1 & -4 & 1\\0 & -17 & 12\\-3 & 3 & \lambda+1} \overset{R_3 \rightarrow R_3 - 3R_1}{\longrightarrow} \myvec{-1 & -4 & 1\\0 & -17 & 12\\0 & 15 & \lambda-2}
\end{equation}

\begin{equation}
    \myvec{-1 & -4 & 1\\0 & -17 & 12\\0 & 15 & \lambda-2} \overset{R_2 \rightarrow -R_2}{\longrightarrow} \myvec{-1 & -4 & 1\\0 & 17 & -12\\0 & 15 & \lambda-2}
\end{equation}

\begin{equation}
    \myvec{-1 & -4 & 1\\0 & 17 & -12\\0 & 15 & \lambda-2} \overset{R_3 \rightarrow R_3 - \frac{15}{17}R_2}{\longrightarrow} \myvec{-1 & -4 & 1\\0 & 17 & -12\\0 & 0 & \lambda+\frac{146}{17}}
\end{equation}

Now for the determinant of this matrix to be zero, the complete row $R_3$ must be zero, so that

\begin{equation}
    \implies \fbox{$\lambda = \frac{-146}{17}$}
\end{equation}
\end{frame}

\section{Final Answer}
\begin{frame}{Final Answer}
The value of $\lambda$ is $\frac{-146}{17}$.

\begin{figure}
    \centering
    \includegraphics[width=0.95\columnwidth]{figs/2.png}
\end{figure}
\end{frame}
\end{document}