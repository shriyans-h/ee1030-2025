\documentclass[journal,12pt,onecolumn]{IEEEtran}
\usepackage{cite}
 \usepackage{caption}
\usepackage{graphicx}
\usepackage{amsmath,amssymb,amsfonts,amsthm}
\usepackage{algorithmic}
\usepackage{graphicx}
\usepackage{textcomp}
\usepackage{xcolor}
\usepackage{txfonts}
\usepackage{listings}
\usepackage{enumitem}
\usepackage{mathtools}
\usepackage{gensymb}
\usepackage{comment}
\usepackage[breaklinks=true]{hyperref}
\usepackage{tkz-euclide} 
\usepackage{listings}
\usepackage{gvv}
%\def\inputGnumericTable{}                                 
\usepackage[latin1]{inputenc} 
\usetikzlibrary{arrows.meta, positioning}
\usepackage{xparse}
\usepackage{color}                                            
\usepackage{array}                                            
\usepackage{longtable}                                       
\usepackage{calc}                                             
\usepackage{multirow}
\usepackage{multicol}
\usepackage{hhline}                                           
\usepackage{ifthen}                                           
\usepackage{lscape}
\usepackage{tabularx}
\usepackage{array}
\usepackage{float}
\usepackage{amssymb}

\usepackage{float}
%\newcommand{\define}{\stackrel{\triangle}{=}}
\theoremstyle{remark}
\usepackage{circuitikz}
\captionsetup{justification=centering}
\usepackage{tikz}

\title{Matrices in Geometry 2.10.64}
\author{EE25BTECH11035 - Kushal B N}
\begin{document}
\vspace{3cm}
\maketitle
{\let\newpage\relax\maketitle}
\textbf{Question: }
The position vectors of the points $\vec{A}$, $\vec{B}$, $\vec{C}$ and $\vec{D}$ are $\brak{3\hat{i}-2\hat{j}-\hat{k}}$, $\brak{2\hat{i}+3\hat{j}-4\hat{k}}$, $\brak{-\hat{i}+\hat{j}+2\hat{k}}$ and $\brak{4\hat{i}+5\hat{j}+\lambda\hat{k}}$ respectively. If the points $\vec{A}$, $\vec{B}$, $\vec{C}$ and $\vec{D}$ lie on a plane, find the value of $\lambda$.
\bigskip

\textbf{Given: } \\
$\vec{A}\myvec{3\\-2\\-1}$, $\vec{B}\myvec{2\\3\\-4}$, $\vec{C}\myvec{-1\\1\\2}$ and $\vec{D}\myvec{4\\5\\\lambda}$.
\bigskip

\textbf{Solution: }\\
\begin{equation}
\vec{B} - \vec{A} = \myvec{-1\\5\\-3}
\end{equation}

\begin{equation}
\vec{C} - \vec{A} = \myvec{-4\\3\\3}
\end{equation}

\begin{equation}
\vec{D} - \vec{A} = \myvec{1\\7\\\lambda+1}
\end{equation}

As the points $\vec{A}$, $\vec{B}$, $\vec{C}$ and $\vec{D}$ lie on a plane, this means that the vectors $\vec{B} - \vec{A}$, $\vec{C} - \vec{A}$ and $\vec{D} - \vec{A}$ are coplanar and hence,\\
\begin{equation}
\implies \abs{\myvec{\vec{B} - \vec{A} & \vec{C} - \vec{A} & \vec{D} - \vec{A}}} = 0
\end{equation}

\begin{equation}
    \abs{\myvec{-1 & -4 & 1\\5 & 3 &7\\-3 & 3 & \lambda+1}} = 0
\end{equation}

Converting this matrix into row echelon form,
\begin{equation}
 \myvec{-1 & -4 & 1\\5 & 3 &7\\-3 & 3 & \lambda+1} \overset{R_2 \rightarrow R_2 + 5R_1}{\longrightarrow} \myvec{-1 & -4 & 1\\0 & -17 & 12\\-3 & 3 & \lambda+1}
\end{equation}

\begin{equation}
    \myvec{-1 & -4 & 1\\0 & -17 & 12\\-3 & 3 & \lambda+1} \overset{R_3 \rightarrow R_3 - 3R_1}{\longrightarrow} \myvec{-1 & -4 & 1\\0 & -17 & 12\\0 & 15 & \lambda-2}
\end{equation}

\begin{equation}
    \myvec{-1 & -4 & 1\\0 & -17 & 12\\0 & 15 & \lambda-2} \overset{R_2 \rightarrow -R_2}{\longrightarrow} \myvec{-1 & -4 & 1\\0 & 17 & -12\\0 & 15 & \lambda-2}
\end{equation}

\begin{equation}
    \myvec{-1 & -4 & 1\\0 & 17 & -12\\0 & 15 & \lambda-2} \overset{R_3 \rightarrow R_3 - \frac{15}{17}R_2}{\longrightarrow} \myvec{-1 & -4 & 1\\0 & 17 & -12\\0 & 0 & \lambda+\frac{146}{17}}
\end{equation}

Now for the determinant of this matrix to be zero, the complete row $R_3$ must be zero, so that

\begin{equation}
    \implies \fbox{$\lambda = \frac{-146}{17}$}
\end{equation}

\textbf{Final Answer:}\\
The value of $\lambda$ is $\frac{-146}{17}$.

\begin{figure}[H]
    \centering
    \includegraphics[width=1.05\columnwidth]{figs/2.png}
\end{figure}

\end{document}
