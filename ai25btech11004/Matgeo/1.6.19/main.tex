\let\negmedspace\undefined
\let\negthickspace\undefined
\documentclass[journal]{IEEEtran}
\usepackage[a5paper, margin=10mm, onecolumn]{geometry}
%\usepackage{lmodern} % Ensure lmodern is loaded for pdflatex
\usepackage{tfrupee} % Include tfrupee package

\setlength{\headheight}{1cm} % Set the height of the header box
\setlength{\headsep}{0mm}     % Set the distance between the header box and the top of the text

\usepackage{gvv-book}
\usepackage{gvv}
\usepackage{cite}
\usepackage{amsmath,amssymb,amsfonts,amsthm}
\usepackage{algorithmic}
\usepackage{graphicx}
\usepackage{textcomp}
\usepackage{xcolor}
\usepackage{txfonts}
\usepackage{listings}
\usepackage{enumitem}
\usepackage{mathtools}
\usepackage{gensymb}
\usepackage{comment}
\usepackage[breaklinks=true]{hyperref}
\usepackage{tkz-euclide} 
\usepackage{listings}
% \usepackage{gvv}                                        
\def\inputGnumericTable{}                                 
\usepackage[latin1]{inputenc}                                
\usepackage{color}                                            
\usepackage{array}                                            
\usepackage{longtable}                                       
\usepackage{calc}                                             
\usepackage{multirow}                                         
\usepackage{hhline}                                           
\usepackage{ifthen}                                           
\usepackage{lscape}
\usepackage{tikz}
\usetikzlibrary{patterns}
\begin{document}


\bibliographystyle{IEEEtran}
\vspace{3cm}


\numberwithin{equation}{enumi}
\numberwithin{figure}{enumi}
\renewcommand{\thetable}{\theenumi}


% Marks the beginning of the document

\bibliographystyle{IEEEtran}
\vspace{3cm}


\title{1.6.19}
\author{AI25BTECH11004-B.JASWANTH}
% \maketitle
% \newpage
% \bigskip
{\let\newpage\relax\maketitle}


\renewcommand{\thefigure}{\theenumi}
\renewcommand{\thetable}{\theenumi}
\setlength{\intextsep}{10pt} % Space between text and floats

\textbf{Question}\\
The vectors $\lambda\hat{i}+\lambda\hat{j}+2\hat{k}$,$1\hat{i}+\lambda\hat{j}-1\hat{k}$ and $2\hat{i}-1\hat{j}+\lambda\hat{k}$ are coplanar if $\lambda$ = \\
\textbf{Solution}: \\
\begin{table}[h!]
	\centering
	\begin{tabular}[12pt]{ |c| c|}
    \hline
    \textbf{Name} & \textbf{Point}\\ 
    \hline
	Point A &\myvec{h \\ k}\\
    \hline 
 Point B &\myvec{x1 \\ y1}\\
    \hline
	  Point R &\myvec{x2 \\ y2}\\
    \hline
    
    \end{tabular}

	\caption{variables used}
	\label{}
\end{table}\\

Form the \(3\times3\) matrix whose columns are the given vectors:
\begin{align*}
A=\myvec{\lambda & 1 & 2\\
\lambda & \lambda & -1\\
2 & -1 & \lambda}.
\end{align*}
The three vectors are coplanar if the columns are linearly dependent, i.e. if there exists a nonzero vector
\(u=\myvec{x\\y\\z} \) with \(A u=0\).  Writing \(Au=0\) gives the system
\begin{align*}
\begin{cases}
\lambda x + y + 2z = 0,\\[4pt]
\lambda x + \lambda y - z = 0,\\[4pt]
2x - y + \lambda z = 0.
\end{cases}
\end{align*}

Subtract the first equation from the second to eliminate \(x\):
\begin{align*}
(\lambda x+\lambda y - z) - (\lambda x + y + 2z)=0
\implies (\lambda-1)y -3z =0
\implies z=\dfrac{\lambda-1}{3}\,y.
\end{align*}

Substitute this \(z\) into the first equation to express \(x\) in terms of \(y\):
\begin{align*}
\lambda x + y + 2\!\left(\frac{\lambda-1}{3}y\right)=0
\implies
\lambda x + \frac{2\lambda+1}{3}\,y = 0
\implies
x = -\frac{2\lambda+1}{3\lambda}\,y,
\end{align*}
(valid when \(\lambda\neq0\); the case \(\lambda=0\) is checked separately below).

Now substitute \(x\) and \(z\) (both expressed in terms of \(y\)) into the third equation:
\begin{align*}
2x - y + \lambda z = 0.
\end{align*}
Using \(x=-\dfrac{2\lambda+1}{3\lambda}y\) and \(z=\dfrac{\lambda-1}{3}y\) we get
\begin{align*}
-\,\frac{4\lambda+2}{3\lambda}\,y - y + \frac{\lambda(\lambda-1)}{3}\,y = 0.
\end{align*}
Multiply through by \(3\lambda\) and factor \(y\):
\begin{align*}
y\bigl(\lambda^3-\lambda^2-7\lambda-2\bigr)=0.
\end{align*}
A nontrivial solution requires \(y\neq0\), hence
\begin{align*}
\lambda^3-\lambda^2-7\lambda-2=0.
\end{align*}

Factor the cubic. One checks \(\lambda=-2\) is a root, and polynomial division yields
\begin{align*}
\lambda^3-\lambda^2-7\lambda-2=(\lambda+2)(\lambda^2-3\lambda-1).
\end{align*}
The quadratic factor has roots
\begin{align*}
\lambda=\frac{3\pm\sqrt{9+4}}{2}=\frac{3\pm\sqrt{13}}{2}.
\end{align*}

Finally check the special case \(\lambda=0\): the system becomes
\begin{align*}
\begin{cases}
y+2z=0,\\
-\,z=0,\\
2x-y=0,
\end{cases}
\end{align*}
which forces \(x=y=z=0\), so \(\lambda=0\) does \emph{not} give a nontrivial solution.

\begin{align*}
\boxed{\text{Therefore the vectors are coplanar exactly for }
\lambda=-2,\;\frac{3+\sqrt{13}}{2},\;\frac{3-\sqrt{13}}{2}.}
\end{align*}

\begin{figure}[H]
    \centering
    \includegraphics[width=0.66\columnwidth]{figs/01.png}
    \label{fig-1}
\end{figure}
\end{document}
