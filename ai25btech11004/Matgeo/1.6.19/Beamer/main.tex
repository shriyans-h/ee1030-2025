\documentclass{beamer}
\mode<presentation>
\usepackage{amsmath,amssymb,mathtools}
\usepackage{textcomp}
\usepackage{gensymb}
\usepackage{adjustbox}
\usepackage{subcaption}
\usepackage{enumitem}
\usepackage{multicol}
\usepackage{listings}
\usepackage{url}
\usepackage{graphicx} % <-- needed for images
\def\UrlBreaks{\do\/\do-}

\usetheme{Boadilla}
\usecolortheme{lily}
\setbeamertemplate{footline}{
  \leavevmode%
  \hbox{%
  \begin{beamercolorbox}[wd=\paperwidth,ht=2ex,dp=1ex,right]{author in head/foot}%
    \insertframenumber{} / \inserttotalframenumber\hspace*{2ex}
  \end{beamercolorbox}}%
  \vskip0pt%
}
\setbeamertemplate{navigation symbols}{}

\lstset{
  frame=single,
  breaklines=true,
  columns=fullflexible,
  basicstyle=\ttfamily\tiny   % tiny font so code fits
}

\numberwithin{equation}{section}

% ---- your macros ----
\providecommand{\nCr}[2]{\,^{#1}C_{#2}}
\providecommand{\nPr}[2]{\,^{#1}P_{#2}}
\providecommand{\mbf}{\mathbf}
\providecommand{\pr}[1]{\ensuremath{\Pr\left(#1\right)}}
\providecommand{\qfunc}[1]{\ensuremath{Q\left(#1\right)}}
\providecommand{\sbrak}[1]{\ensuremath{{}\left[#1\right]}}
\providecommand{\lsbrak}[1]{\ensuremath{{}\left[#1\right.}}
\providecommand{\rsbrak}[1]{\ensuremath{\left.#1\right]}}
\providecommand{\brak}[1]{\ensuremath{\left(#1\right)}}
\providecommand{\lbrak}[1]{\ensuremath{\left(#1\right.}}
\providecommand{\rbrak}[1]{\ensuremath{\left.#1\right)}}
\providecommand{\cbrak}[1]{\ensuremath{\left\{#1\right\}}}
\providecommand{\lcbrak}[1]{\ensuremath{\left\{#1\right.}}
\providecommand{\rcbrak}[1]{\ensuremath{\left.#1\right\}}}
\theoremstyle{remark}
\newtheorem{rem}{Remark}
\newcommand{\sgn}{\mathop{\mathrm{sgn}}}
\providecommand{\abs}[1]{\left\vert#1\right\vert}
\providecommand{\res}[1]{\Res\displaylimits_{#1}}
\providecommand{\norm}[1]{\lVert#1\rVert}
\providecommand{\mtx}[1]{\mathbf{#1}}
\providecommand{\mean}[1]{E\left[ #1 \right]}
\providecommand{\fourier}{\overset{\mathcal{F}}{ \rightleftharpoons}}
\providecommand{\system}{\overset{\mathcal{H}}{ \longleftrightarrow}}
\providecommand{\dec}[2]{\ensuremath{\overset{#1}{\underset{#2}{\gtrless}}}}
\newcommand{\myvec}[1]{\ensuremath{\begin{pmatrix}#1\end{pmatrix}}}
\let\vec\mathbf

\title{Matgeo Presentation - Problem 1.6.19}
\author{ai25btech11004 - Jaswanth}

\begin{document}


\frame{\titlepage}
\begin{frame}{Question}
The vectors $\lambda\hat{i}+\lambda\hat{j}+2\hat{k}$,$1\hat{i}+\lambda\hat{j}-1\hat{k}$ and $2\hat{i}-1\hat{j}+\lambda\hat{k}$ are coplanar if $\lambda$ = \\
\end{frame}


\begin{frame}{Solution}
   
The vectors are coplanar $\iff$ they are linearly dependent.  
Form the matrix with these vectors as columns:
\begin{align}
A = 
\begin{bmatrix}
\lambda & 1 & 2 \\[4pt]
\lambda & \lambda & -1 \\[4pt]
2 & -1 & \lambda
\end{bmatrix}.
\end{align}

The three vectors are linearly dependent $\iff \det(A)=0$.  
We compute $\det(A)$ using row reduction.

\begin{align}
A = 
\begin{bmatrix}
\lambda & 1 & 2 \\
\lambda & \lambda & -1 \\
2 & -1 & \lambda
\end{bmatrix}
\end{align}

\begin{align}
R_2 \to R_2 - R_1 \quad \Rightarrow \quad
\begin{bmatrix}
\lambda & 1 & 2 \\
0 & \lambda-1 & -3 \\
2 & -1 & \lambda
\end{bmatrix}
\end{align}
\end{frame}

\begin{frame}{Solution}
\begin{align}
R_3 \to R_3 - \tfrac{2}{\lambda}R_1 \quad \Rightarrow \quad
\begin{bmatrix}
\lambda & 1 & 2 \\
0 & \lambda-1 & -3 \\
0 & -1-\tfrac{2}{\lambda} & \lambda - \tfrac{4}{\lambda}
\end{bmatrix}
\end{align}

\begin{align}
R_3 \to R_3 - \frac{-1-\tfrac{2}{\lambda}}{\lambda-1}R_2
\quad \Rightarrow \quad
\begin{bmatrix}
\lambda & 1 & 2 \\
0 & \lambda-1 & -3 \\
0 & 0 & \dfrac{\lambda^3-\lambda^2-7\lambda-2}{\lambda(\lambda-1)}
\end{bmatrix}
\end{align}

Now the matrix is upper triangular, so

\bigskip

\textbf{Solve the cubic}  
\begin{align}
\lambda^3 - \lambda^2 - 7\lambda - 2 = 0.
\end{align}
\end{frame}
\begin{frame}{Solution}

Factor:
\begin{align}
(\lambda+2)(\lambda^2 - 3\lambda - 1) = 0.
\end{align}

So the solutions are
\begin{align}
\lambda = -2, \qquad \lambda = \frac{3+\sqrt{13}}{2}, \qquad \lambda = \frac{3-\sqrt{13}}{2}.
\end{align}

\bigskip
\textbf{Conclusion:}  
\begin{itemize}
\item For these values of $\lambda$, $\det(A)=0 \implies \operatorname{rank}(A)<3$, so the vectors are \textbf{linearly dependent} (coplanar).
\end{itemize}

$\boxed{\text{The vectors are coplanar for } \lambda=-2, \; \tfrac{3+\sqrt{13}}{2}, \; \tfrac{3-\sqrt{13}}{2}.}$
\end{frame}

\begin{frame}{Plot}
    \begin{figure}[H]
    \centering
    \includegraphics[width=0.66\columnwidth]{figs/01.png}
    \label{fig-1}
\end{figure}
\end{frame}

% --------- CODE APPENDIX ---------
\section*{Appendix: Code}

% C program
\begin{frame}[fragile]{C Code: Vector.c}
\begin{lstlisting}[language=C]
#include <stdio.h>

int main() {
    FILE *fp;
    fp = fopen("vector.dat", "w");
    if (fp == NULL) {
        printf("Error opening file!\n");
        return 1;
    }

    // The determinant expansion:
    // | λ   λ   2 |
    // | 1   λ  -1 |
    // | 2  -1   λ |
    //
    // Det = -λ^3 + λ^2 + 5λ - 4

    fprintf(fp, "Determinant condition for coplanarity:\n");
    fprintf(fp, "(-λ^3 + λ^2 + 5λ - 4) = 0\n\n");

    fprintf(fp, "Checking integer values of λ from -10 to 10:\n");

    for (int lambda = -10; lambda <= 10; lambda++) {
        int val = -lambda*lambda*lambda + lambda*lambda + 5*lambda - 4;
        if (val == 0) {
            fprintf(fp, "λ = %d is a solution.\n", lambda);
        }
    }

    fclose(fp);
    printf("Results written to vector.dat\n");
    return 0;
}

    \end{lstlisting}
\end{frame}

\begin{frame}[fragile]{Python: plot.py}
\begin{lstlisting}[language=Python]
  import numpy as np
import matplotlib.pyplot as plt
from mpl_toolkits.mplot3d import Axes3D  # noqa: F401

# ----- Correct value -----
lam = -2  # correct λ

# Vectors
A = np.array([lam, lam, 2], dtype=float)
B = np.array([1, lam, -1], dtype=float)
C = np.array([2, -1, lam], dtype=float)

# Verify coplanarity via scalar triple product: A · (B × C) = 0
triple = float(np.dot(A, np.cross(B, C)))
print(f"Scalar triple product at λ={lam}: {triple:.6g} (0 => coplanar)")

# ----- Plot -----
fig = plt.figure()
ax = fig.add_subplot(111, projection='3d')

origin = np.zeros(3)

# Plot vectors from origin
ax.quiver(*origin, *A, length=1, normalize=False, label=f"A = {A}", color='r')
ax.quiver(*origin, *B, length=1, normalize=False, label=f"B = {B}", color='g')
ax.quiver(*origin, *C, length=1, normalize=False, label=f"C = {C}", color='b')

# Plot the plane spanned by B and C (shows A lies in this plane)
s = np.linspace(-1.2, 1.2, 20)
t = np.linspace(-1.2, 1.2, 20)
S, T = np.meshgrid(s, t)
plane = np.outer(S.ravel(), B) + np.outer(T.ravel(), C)
X = plane[:, 0].reshape(S.shape)
\end{lstlisting}
\end{frame}   

\begin{frame}[fragile]{Python: plot.py}
\begin{lstlisting}[language=Python]
Y = plane[:, 1].reshape(S.shape)
Z = plane[:, 2].reshape(S.shape)
ax.plot_surface(X, Y, Z, alpha=0.2, edgecolor='none')

# Aesthetic: equal aspect & limits
all_pts = np.vstack([origin, A, B, C, plane])
mins = all_pts.min(axis=0)
maxs = all_pts.max(axis=0)
ranges = maxs - mins
center = (maxs + mins) / 2
max_range = ranges.max() * 0.55 + 1e-9
ax.set_xlim(center[0]-max_range, center[0]+max_range)
ax.set_ylim(center[1]-max_range, center[1]+max_range)
ax.set_zlim(center[2]-max_range, center[2]+max_range)
ax.set_box_aspect([1,1,1])

ax.set_xlabel('X')
ax.set_ylabel('Y')
ax.set_zlabel('Z')
ax.set_title(f"Vectors A, B, C for λ = {lam} (coplanar)")

ax.legend(loc='upper left')

# ----- Save the figure -----
plt.savefig("vectors.png", dpi=300, bbox_inches='tight')

# Show on screen too (optional)
plt.show()

\end{lstlisting}
\end{frame}   

\end{document}
