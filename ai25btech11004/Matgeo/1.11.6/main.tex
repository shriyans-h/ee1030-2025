\let\negmedspace\undefined
\let\negthickspace\undefined
\documentclass[journal]{IEEEtran}
\usepackage[a5paper, margin=10mm, onecolumn]{geometry}
%\usepackage{lmodern} % Ensure lmodern is loaded for pdflatex
\usepackage{tfrupee} % Include tfrupee package

\setlength{\headheight}{1cm} % Set the height of the header box
\setlength{\headsep}{0mm}     % Set the distance between the header box and the top of the text

\usepackage{gvv-book}
\usepackage{gvv}
\usepackage{cite}
\usepackage{amsmath,amssymb,amsfonts,amsthm}
\usepackage{algorithmic}
\usepackage{graphicx}
\usepackage{textcomp}
\usepackage{xcolor}
\usepackage{txfonts}
\usepackage{listings}
\usepackage{enumitem}
\usepackage{mathtools}
\usepackage{gensymb}
\usepackage{comment}
\usepackage[breaklinks=true]{hyperref}
\usepackage{tkz-euclide} 
\usepackage{listings}
% \usepackage{gvv}                                        
\def\inputGnumericTable{}                                 
\usepackage[latin1]{inputenc}                                
\usepackage{color}                                            
\usepackage{array}                                            
\usepackage{longtable}                                       
\usepackage{calc}                                             
\usepackage{multirow}                                         
\usepackage{hhline}                                           
\usepackage{ifthen}                                           
\usepackage{lscape}
\usepackage{tikz}
\usetikzlibrary{patterns}
\begin{document}


\bibliographystyle{IEEEtran}
\vspace{3cm}


\numberwithin{equation}{enumi}
\numberwithin{figure}{enumi}
\renewcommand{\thetable}{\theenumi}


% Marks the beginning of the document

\bibliographystyle{IEEEtran}
\vspace{3cm}


\title{1.11.6}
\author{AI25BTECH11004-B.JASWANTH}
% \maketitle
% \newpage
% \bigskip
{\let\newpage\relax\maketitle}


\renewcommand{\thefigure}{\theenumi}
\renewcommand{\thetable}{\theenumi}
\setlength{\intextsep}{10pt} % Space between text and floats

\textbf{Question}\\
Find the direction cosines of a line which makes equal angles with the coordinate axes. 
\textbf{Solution}:\\ 

Let $\theta$ be the angle made by a line with coordinate axes.  
The direction cosines of line l are given by  
$\myvec{ \cos\theta \\ \cos\theta \\ \cos\theta}$    

Since $||l|| = 1$, we have
\begin{align}
\cos^2 \theta + \cos^2 \theta + \cos^2 \theta = 1    
\end{align}
\begin{align}
3 \cos^2 \theta = 1 \quad \Rightarrow \quad \cos^2 \theta = \tfrac{1}{3}    
\end{align}


Since $\theta$ is an acute angle,
\begin{align}
\cos \theta = \frac{1}{\sqrt{3}}
\end{align}
Hence,direction cosines of a line are 
$\myvec{\tfrac{1}{\sqrt{3}} \\ \tfrac{1}{\sqrt{3}} \\ \tfrac{1}{\sqrt{3}}}$

\begin{figure}[H]
    \centering
	\includegraphics[width=0.5\columnwidth]{figs/01.png}
    \label{fig-1}
\end{figure}

\end{document}
