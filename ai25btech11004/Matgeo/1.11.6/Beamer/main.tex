\documentclass{beamer}
\mode<presentation>
\usepackage{amsmath,amssymb,mathtools}
\usepackage{textcomp}
\usepackage{gensymb}
\usepackage{adjustbox}
\usepackage{subcaption}
\usepackage{enumitem}
\usepackage{multicol}
\usepackage{listings}
\usepackage{url}
\usepackage{graphicx} % <-- needed for images
\def\UrlBreaks{\do\/\do-}

\usetheme{Boadilla}
\usecolortheme{lily}
\setbeamertemplate{footline}{
  \leavevmode%
  \hbox{%
  \begin{beamercolorbox}[wd=\paperwidth,ht=2ex,dp=1ex,right]{author in head/foot}%
    \insertframenumber{} / \inserttotalframenumber\hspace*{2ex}
  \end{beamercolorbox}}%
  \vskip0pt%
}
\setbeamertemplate{navigation symbols}{}

\lstset{
  frame=single,
  breaklines=true,
  columns=fullflexible,
  basicstyle=\ttfamily\tiny   % tiny font so code fits
}

\numberwithin{equation}{section}

% ---- your macros ----
\providecommand{\nCr}[2]{\,^{#1}C_{#2}}
\providecommand{\nPr}[2]{\,^{#1}P_{#2}}
\providecommand{\mbf}{\mathbf}
\providecommand{\pr}[1]{\ensuremath{\Pr\left(#1\right)}}
\providecommand{\qfunc}[1]{\ensuremath{Q\left(#1\right)}}
\providecommand{\sbrak}[1]{\ensuremath{{}\left[#1\right]}}
\providecommand{\lsbrak}[1]{\ensuremath{{}\left[#1\right.}}
\providecommand{\rsbrak}[1]{\ensuremath{\left.#1\right]}}
\providecommand{\brak}[1]{\ensuremath{\left(#1\right)}}
\providecommand{\lbrak}[1]{\ensuremath{\left(#1\right.}}
\providecommand{\rbrak}[1]{\ensuremath{\left.#1\right)}}
\providecommand{\cbrak}[1]{\ensuremath{\left\{#1\right\}}}
\providecommand{\lcbrak}[1]{\ensuremath{\left\{#1\right.}}
\providecommand{\rcbrak}[1]{\ensuremath{\left.#1\right\}}}
\theoremstyle{remark}
\newtheorem{rem}{Remark}
\newcommand{\sgn}{\mathop{\mathrm{sgn}}}
\providecommand{\abs}[1]{\left\vert#1\right\vert}
\providecommand{\res}[1]{\Res\displaylimits_{#1}}
\providecommand{\norm}[1]{\lVert#1\rVert}
\providecommand{\mtx}[1]{\mathbf{#1}}
\providecommand{\mean}[1]{E\left[ #1 \right]}
\providecommand{\fourier}{\overset{\mathcal{F}}{ \rightleftharpoons}}
\providecommand{\system}{\overset{\mathcal{H}}{ \longleftrightarrow}}
\providecommand{\dec}[2]{\ensuremath{\overset{#1}{\underset{#2}{\gtrless}}}}
\newcommand{\myvec}[1]{\ensuremath{\begin{pmatrix}#1\end{pmatrix}}}
\let\vec\mathbf

\title{Matgeo Presentation - Problem 1.11.6-}
\author{ai25btech11004 - Jaswanth}


\begin{document}

\frame{\titlepage}
\begin{frame}{Question}
Find the direct cosines of a line which makes equal angles with the coordinate axes.
\end{frame}

\begin{frame}{Solution}
Let $\theta$ be the angle made by a line with coordinate axes.  
The direction cosines of line l are given by  
$\myvec{ \cos\theta \\ \cos\theta \\ \cos\theta}$    

Since $||l|| = 1$, we have
\begin{align}
\cos^2 \theta + \cos^2 \theta + \cos^2 \theta = 1    
\end{align}
\begin{align}
3 \cos^2 \theta = 1 \quad \Rightarrow \quad \cos^2 \theta = \tfrac{1}{3}    
\end{align}


Since $\theta$ is an acute angle,
\begin{align}
\cos \theta = \frac{1}{\sqrt{3}}
\end{align}
Hence,direction cosines of a line are 
$\myvec{\tfrac{1}{\sqrt{3}} \\ \tfrac{1}{\sqrt{3}} \\ \tfrac{1}{\sqrt{3}}}$
\end{frame}
\begin{frame}{Plot}
    \begin{figure}[H]
    \centering
    \includegraphics[width=0.66\columnwidth]{figs/01.png}
    \label{fig-1}
\end{figure}
\end{frame}

% --------- CODE APPENDIX ---------
\section*{Appendix: Code}

% C program
\begin{frame}[fragile]{C Code: Vector.c}
\begin{lstlisting}[language=C]
 #include <stdio.h>
#include <math.h>

int main() {
    FILE *fp;
    fp = fopen("axis.dat", "w");   // Open file to write

    if (fp == NULL) {
        printf("Error opening file!\n");
        return 1;
    }

    // Since line makes equal angles with coordinate axes
    // direction cosines are ±1/sqrt(3)
    double val = 1.0 / sqrt(3.0);

    fprintf(fp, "Direction cosines of a line making equal angles with coordinate axes:\n");
    fprintf(fp, "Possible sets are:\n\n");

    // There are 8 possible combinations of signs
    int signs[8][3] = {
        { 1,  1,  1},
        { 1,  1, -1},
        { 1, -1,  1},
        { 1, -1, -1},
        {-1,  1,  1},
        {-1,  1, -1},
        {-1, -1,  1},
        {-1, -1, -1}
    };

    for (int i = 0; i < 8; i++) {
        fprintf(fp, "(%.4f, %.4f, %.4f)\n",
         \end{lstlisting}
\end{frame}
\begin{frame}[fragile]{C Code: Vector.c}
\begin{lstlisting}[language=C]
                signs[i][0] * val,
                signs[i][1] * val,
                signs[i][2] * val);
    }

    fclose(fp);
    printf("Direction cosines written to axis.dat successfully.\n");
    return 0;
}
    \end{lstlisting}
\end{frame}
\begin{frame}[fragile]{Python: plot.py}
\begin{lstlisting}[language=Python]
  import numpy as np
import matplotlib.pyplot as plt
from mpl_toolkits.mplot3d import Axes3D

# Direction cosines
l = m = n = 1/np.sqrt(3)

# Define line points
t = np.linspace(-5, 5, 100)
x = l * t
y = m * t
z = n * t

# Plotting
fig = plt.figure(figsize=(8, 6))
ax = fig.add_subplot(111, projection='3d')

# Plot the line
ax.plot(x, y, z, label="Line with equal angles", color="blue")

# Plot origin
ax.scatter(0, 0, 0, color="red", s=50, label="Origin")

# Axes labels
ax.set_xlabel('X-axis')
ax.set_ylabel('Y-axis')
ax.set_zlabel('Z-axis')
ax.set_title("Line making equal angles with coordinate axes")
ax.legend()
# Save the figure
plt.savefig("equal_angles_line.png", dpi=300, bbox_inches='tight')
# Show the plot
plt.show()
\end{lstlisting}
\end{frame}   




\end{document}
