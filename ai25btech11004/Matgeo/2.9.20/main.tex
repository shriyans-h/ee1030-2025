\let\negmedspace\undefined
\let\negthickspace\undefined
\documentclass[journal]{IEEEtran}
\usepackage[a5paper, margin=10mm, onecolumn]{geometry}
%\usepackage{lmodern} % Ensure lmodern is loaded for pdflatex
\usepackage{tfrupee} % Include tfrupee package

\setlength{\headheight}{1cm} % Set the height of the header box
\setlength{\headsep}{0mm}     % Set the distance between the header box and the top of the text

\usepackage{gvv-book}
\usepackage{gvv}
\usepackage{cite}
\usepackage{amsmath,amssymb,amsfonts,amsthm}
\usepackage{algorithmic}
\usepackage{graphicx}
\usepackage{textcomp}
\usepackage{xcolor}
\usepackage{txfonts}
\usepackage{listings}
\usepackage{enumitem}
\usepackage{mathtools}
\usepackage{gensymb}
\usepackage{comment}
\usepackage[breaklinks=true]{hyperref}
\usepackage{tkz-euclide} 
\usepackage{listings}
% \usepackage{gvv}                                        
\def\inputGnumericTable{}                                 
\usepackage[latin1]{inputenc}                                
\usepackage{color}                                            
\usepackage{array}                                            
\usepackage{longtable}                                       
\usepackage{calc}                                             
\usepackage{multirow}                                         
\usepackage{hhline}                                           
\usepackage{ifthen}                                           
\usepackage{lscape}
\usepackage{tikz}
\usetikzlibrary{patterns}
\begin{document}


\bibliographystyle{IEEEtran}
\vspace{3cm}


\numberwithin{equation}{enumi}
\numberwithin{figure}{enumi}
\renewcommand{\thetable}{\theenumi}


% Marks the beginning of the document

\bibliographystyle{IEEEtran}
\vspace{3cm}


\title{2.9.20}
\author{AI25BTECH11004-B.JASWANTH}
% \maketitle
% \newpage
% \bigskip
{\let\newpage\relax\maketitle}


\renewcommand{\thefigure}{\theenumi}
\renewcommand{\thetable}{\theenumi}
\setlength{\intextsep}{10pt} % Space between text and floats

\textbf{Question}\\
$\vec{X}$ and  $\vec{Y}$ are two points with position vectors $3\vec{a}$+$\vec{b}$  and  $\vec{a}$-$3\vec{b}$, respectively. Write the position vector of a point Z  which divides the line segment  $\vec{XY}$ in the ratio 2:1 externally.\\
\textbf{Solution}:\\
Let the position vectors of the points $X$ and $Y$ be given by
\begin{align}
  \implies  \vec{X} = \myvec{3 & 1}
\myvec{\vec{a} \\ \vec{b}}\\
\end{align}
\begin{align}
\implies \vec{Y} = 
\myvec{1 & -3}
\myvec{\vec{a} \\ \vec{b}}.
\end{align}

\vspace{1em}

The formula for the point $\mathbf{Z}$ dividing the line segment joining $\vec{X}$ and $\vec{Y}$ \textbf{externally} in the ratio $k:1$ is:
\begin{align}
\vec{Z} = \frac{k\vec{Y} - \vec{X}}{k - 1}.
\end{align}

Substituting $k = 2$ and the above matrices:
\begin{align}
\vec{Z} = \frac{2\vec{Y} - \vec{X}}{2 - 1}
= 2\vec{Y} - \vec{X}.
\end{align}

Now compute $2\vec{Y} - \vec{X}$:
\begin{align}
2\vec{Y} - \vec{X} = 
2\myvec{1 & -3}
       \myvec{\vec{a} \\ \vec{b}}
-
\myvec{3 & 1}
      \myvec{\vec{a} \\ \vec{b}}
=
\myvec{-1 & -7}
     \myvec{\vec{a} \\ \vec{b}}.
\end{align}

Hence,
\begin{align}
\vec{Z} = 
\myvec{-1 & -7}
\myvec{\vec{a} \\ \vec{b}}
.
\end{align}

\noindent
\textbf{Therefore, the position vector of $Z$ is}
\begin{align}
\boxed{\vec{Z} = 
-\vec{a}-7\vec{b}}
\end{align}


\begin{figure}[h!]
    \centering
    \includegraphics[width=0.8\linewidth]{figs/01.png}
    \caption{Caption}
    \label{fig:placeholder}
\end{figure}


\end{document}
