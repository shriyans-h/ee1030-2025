\let\negmedspace\undefined
\let\negthickspace\undefined
\documentclass[journal]{IEEEtran}
\usepackage[a5paper, margin=10mm, onecolumn]{geometry}
%\usepackage{lmodern} % Ensure lmodern is loaded for pdflatex
\usepackage{tfrupee} % Include tfrupee package

\setlength{\headheight}{1cm} % Set the height of the header box
\setlength{\headsep}{0mm}     % Set the distance between the header box and the top of the text

\usepackage{gvv-book}
\usepackage{gvv}
\usepackage{cite}
\usepackage{amsmath,amssymb,amsfonts,amsthm}
\usepackage{algorithmic}
\usepackage{graphicx}
\usepackage{textcomp}
\usepackage{xcolor}
\usepackage{txfonts}
\usepackage{listings}
\usepackage{enumitem}
\usepackage{mathtools}
\usepackage{gensymb}
\usepackage{comment}
\usepackage[breaklinks=true]{hyperref}
\usepackage{tkz-euclide} 
\usepackage{listings}
% \usepackage{gvv}                                        
\def\inputGnumericTable{}                                 
\usepackage[latin1]{inputenc}                                
\usepackage{color}                                            
\usepackage{array}                                            
\usepackage{longtable}                                       
\usepackage{calc}                                             
\usepackage{multirow}                                         
\usepackage{hhline}                                           
\usepackage{ifthen}                                           
\usepackage{lscape}
\usepackage{circuitikz}
\tikzstyle{block} = [rectangle, draw, fill=blue!20, 
    text width=4em, text centered, rounded corners, minimum height=3em]
\tikzstyle{sum} = [draw, fill=blue!10, circle, minimum size=1cm, node distance=1.5cm]
\tikzstyle{input} = [coordinate]
\tikzstyle{output} = [coordinate]

\begin{document}


\bibliographystyle{IEEEtran}
\vspace{3cm}

\title{2.9.2}
\author{EE25BTECH11051 - Shreyas Goud Burra}
\maketitle
{\let\newpage\relax\maketitle}

\renewcommand{\thefigure}{\theenumi}
\renewcommand{\thetable}{\theenumi}
\setlength{\intextsep}{10pt}


\numberwithin{equation}{enumi}
\numberwithin{figure}{enumi}
\renewcommand{\thetable}{\theenumi}

\textbf{Question}
If (-5, 3) and (5, 3) are two vertices of an equilateral triangle, then the
coordinates of the third vertex, given that the origin lies inside the triangle (take $\sqrt{3}$ = 1.7), are\\

\solution\\

Let us find the solution theoretically first and then verify it computationally.\\
Let the two given points be represented as vectors, \textbf{A} and \textbf{B}, respectively

\begin{align}
    \textbf{A} = \myvec{-5\\3}, \textbf{B} = \myvec{5\\3}
    \label{0.1}
\end{align}

Let us assume the third point be \textbf{C}.\\
\textbf{C} must be equidistant from both \textbf{A} and \textbf{B}, and it lies on the perpendicular bisector to both \textbf{A} and \textbf{B}.

The distance between \textbf{A} and \textbf{B}, is given by
\begin{align}
    \norm{\textbf{A}-\textbf{B}}=\norm{\myvec{-10\\0}}
    \label{0.2}
\end{align}

We know that the norm of a vector is given by

\begin{align}
    \norm{\textbf{A}-\textbf{B}}^2 = (\textbf{A}-\textbf{B})^\text{T}(\textbf{A}-\textbf{B}) \implies \myvec{-10&0}.\myvec{-10\\0}= 100
    \label{0.3}
\end{align}

As the norm of a vector is always greater than or equal to zero. From \ref{0.3} we get

\begin{align}
    \norm{\textbf{A}-\textbf{B}} = 10
    \label{0.4}
\end{align}

The midpoint to the line segment \textbf{AB} is given by

\begin{align}
        \frac{\textbf{A}+\textbf{B}}{2} = \frac{\myvec{-5\\3}+\myvec{5\\3}}{2} = \myvec{0\\3}
        \label{0.5}
\end{align}

Slope of line segment \textbf{AB} is given by
\begin{align}
    \textbf{B}-\textbf{A}=k\myvec{1\\m}\text{, where m is the slope of the line segment}
    \label{0.6}
\end{align}

On further solving

\begin{align}
    \textbf{B}-\textbf{A}=\myvec{10\\0} \implies m=0
    \label{0.7}
\end{align}

Therefore the perpendicular bisector for this line segment is a vertical line passing through the midpoint (0, 3).\\

In parametric form

\begin{align}
    \textbf{}
    \textbf{C}= \myvec{0\\t+3}\text{, where t is the distance between the point \textbf{C} and the line segment \textbf{AB}}
    \label{0.8}
\end{align}

We know for an equilateral triangle, distance between a point and the opposite edge is $\frac{\sqrt{3}}{2}$ times the length of an edge of that triangle.

\begin{align}
    t=\pm \frac{\sqrt{3}}{2}\norm{\textbf{A}-\textbf{B}} \implies t=\pm 5\sqrt{3}
    \label{0.9}
\end{align}

Therefore the required points for \textbf{C} are given by

\begin{align}
    \textbf{C} = \myvec{0\\\pm 5\sqrt{3}+3}
    \label{0.9}
\end{align}

On plotting this gives us

\begin{figure}[H]
    \centering
    \includegraphics[width=0.8\columnwidth]{figs/fig1.png}
    \caption{2D Plot}
    \label{3D Plot}
\end{figure}






\end{document}
