% Preamble templated from Mihir-Divyansh/Course-Setup
%iffalse
\let\negmedspace\undefined
\let\negthickspace\undefined
\documentclass[journal,12pt,onecolumn]{IEEEtran}
\usepackage{cite}
\usepackage{amsmath,amssymb,amsfonts,amsthm}
\usepackage{algorithmic}
\usepackage{graphicx}
\usepackage{textcomp}
\usepackage{xcolor}
\usepackage{txfonts}
\usepackage{listings}
\usepackage{enumitem}
\usepackage{mathtools}
\usepackage{gensymb}
\usepackage{comment}
\usepackage[breaklinks=true]{hyperref}
\usepackage{tkz-euclide}
\usepackage{listings}
\usepackage{gvv}
%\def\inputGnumericTable{}
\usepackage[latin1]{inputenc}
\usepackage{color}
\usepackage{array}
\usepackage{longtable}
\usepackage{calc}
\usepackage{multirow}
\usepackage{hhline}
\usepackage{ifthen}
\usepackage{lscape}
\usepackage{tabularx}
\usepackage{array}
\usepackage{float}
\usepackage{caption}
\usepackage{multicol}

\newtheorem{theorem}{Theorem}[section]
\newtheorem{problem}{Problem}
\newtheorem{proposition}{Proposition}[section]
\newtheorem{lemma}{Lemma}[section]
\newtheorem{corollary}[theorem]{Corollary}
\newtheorem{example}{Example}[section]
\newtheorem{definition}[problem]{Definition}
\newcommand{\BEQA}{\begin{eqnarray}}
\newcommand{\EEQA}{\end{eqnarray}}
%\newcommand{\define}{\stackrel{\triangle}{=}}
\theoremstyle{remark}
%\newtheorem{rem}{Remark}

% Marks the beginning of the document
\begin{document}
\bibliographystyle{IEEEtran}
\vspace{3cm}

\title{Assignment 11: 5.5.9}
\author{EE25BTECH11055 - Subhodeep Chakraborty}
\maketitle
\hrulefill
\bigskip

\renewcommand{\thefigure}{\theenumi}
\renewcommand{\thetable}{\theenumi}

\textbf{Question:}\par
Using elementary row transformations find the inverse of the matrix\hfill\brak{12, 2019}
$$\myvec{3&0&-1\\2&3&0\\0&4&1}$$
\par
\textbf{Solution:}\par

Given:
\begin{align}
 \vec{A} = \myvec{3&0&-1\\2&3&0\\0&4&1}
\end{align}
Let $\vec{A^{-1}}$ be inverse of $\vec{A}$
We know \begin{align}\vec{A}\vec{A^{-1}} = \vec{I}\end{align}
Thus augmented matrix to find $\vec{A^{-1}}$ is given by: $\augvec{1}{1}{\vec{A}&\vec{I}}$
\begin{align}
\alpha
\end{align}
So we have:
\begin{align}
 \vec{A^{-1}}=\myvec{x\\y\\z}=\myvec{1\\1/2\\-3/2}
\end{align}
\begin{figure}[H]
    \centering
    \includegraphics{figs/plot.png}
    \caption*{}
    \label{fig:plot}
\end{figure}
\end{document}
