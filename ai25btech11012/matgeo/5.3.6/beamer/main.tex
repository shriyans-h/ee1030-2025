\documentclass{beamer}
\usetheme{Madrid}

\usepackage{amsmath, amssymb, amsthm}
\usepackage{graphicx}
\usepackage{gensymb}
\usepackage[utf8]{inputenc}
\usepackage{hyperref}
\usepackage{tikz}
\usepackage{amsmath}

\title{5.3.6 Matgeo}
\author{AI25BTECH11012 - Garige Unnathi}
\date{}

\begin{document}

\frame{\titlepage}

% Question frame
\begin{frame}
\frametitle{Question}
If the pair of equations 3x - y + 8 = 0 and 6x - ry + 16 = 0 represents coincident
lines, then the value of r is 
\end{frame}


% Solution steps
\begin{frame}
\frametitle{Solution}
Let :
\begin{align}
    \textbf{$r_1$} = \begin{bmatrix}3 & -1\end{bmatrix}\textbf{x} = -8 \\
    \textbf{$r_2$} = \begin{bmatrix}6 & -r\end{bmatrix}\textbf{x} = -16
\end{align}
For coincident lines:
\begin{align}
    \textbf{$r_2$} = \kappa  \textbf{$r_1$}
\end{align}
\end{frame}

\begin{frame}
\frametitle{Solution}
Solving using above equation 
\begin{align}
    \begin{bmatrix}6 & -r\end{bmatrix}\textbf{x} + 16 = \kappa(\begin{bmatrix}3 & -1\end{bmatrix}\textbf{x} + 8)\\
     = \begin{bmatrix}3\kappa & -1\kappa\end{bmatrix}\textbf{x} + 8\kappa
\end{align}

By comparing we get :
\begin{align}
    \kappa = 2\\
     \begin{bmatrix}6 & -r\end{bmatrix}\textbf{x} + 16 = \begin{bmatrix}6 & -2 \end{bmatrix}\textbf{x} + 16
\end{align}
\end{frame}


\begin{frame}
\frametitle{Solution}
since LHS should be equal to RHS :
\begin{align}
    r = 2
\end{align}
\end{frame}


\begin{frame}

\frametitle{Graphical Representation}
\begin{center}
\includegraphics[width=0.6\linewidth]{/Users/unnathi/Documents/ee1030-2025/ai25btech11012/matgeo/5.3.6/figs/figs.png}
\end{center}
\end{frame}


\end{document}
