\let\negmedspace\undefined
\let\negthickspace\undefined
\documentclass[journal]{IEEEtran}
\usepackage[a5paper, margin=10mm, onecolumn]{geometry}
%\usepackage{lmodern} % Ensure lmodern is loaded for pdflatex
\usepackage{tfrupee} % Include tfrupee package

\setlength{\headheight}{1cm} % Set the height of the header box
\setlength{\headsep}{0mm}     % Set the distance between the header box and the top of the text

\usepackage{gvv-book}
\usepackage{gvv}
\usepackage{cite}
\usepackage{amsmath,amssymb,amsfonts,amsthm}
\usepackage{algorithmic}
\usepackage{graphicx}
\usepackage{textcomp}
\usepackage{xcolor}
\usepackage{txfonts}
\usepackage{listings}
\usepackage{enumitem}
\usepackage{mathtools}
\usepackage{gensymb}
\usepackage{comment}
\usepackage[breaklinks=true]{hyperref}
\usepackage{tkz-euclide} 
\usepackage{listings}
\def\inputGnumericTable{}                                 
\usepackage[latin1]{inputenc}                                
\usepackage{color}                                            
\usepackage{array}                                            
\usepackage{longtable}                                       
\usepackage{calc}                                             
\usepackage{multirow}                                         
\usepackage{hhline}                                           
\usepackage{ifthen}                                           
\usepackage{lscape}
\begin{document}

\bibliographystyle{IEEEtran}


\title{5.5.29}
\author{AI25BTECH11012 - GARIGE UNNATHI}
% \maketitle
% \newpage
% \bigskip
{\let\newpage\relax\maketitle}


\renewcommand{\thefigure}{\theenumi}
\renewcommand{\thetable}{\theenumi}
\setlength{\intextsep}{10pt} % Space between text and floats


\numberwithin{equation}{enumi}
\numberwithin{figure}{enumi}

\vspace{-1cm}

\textbf{Question}:\\
If the inverse of the matrix \myvec{7 &-3&-3\\
                                    -1&1&0\\
                                     -1&0&1}is the matrix \myvec{1&3&3\\1&\lambda&3\\1&3&4} , then find the value of $\lambda$ .


\textbf{Solution: }\\
Let : 
\begin{align*}
    \vec{A}  = \myvec{7 &-3&-3\\
                      -1&1&0\\
                      -1&0&1}
\end{align*}
The characteristic equation for a matrix $\vec{A}$ is  
\begin{align}
 f(\lambda) =  \lvert \vec{A} - \lambda\vec{I} \rvert = 0\\
 f(\lambda) = \begin{vmatrix}7-\lambda &-3&-3\\
                      -1&1-\lambda&0\\
                      -1&0&1-\lambda\end{vmatrix} = 0
\end{align}
Solving the above equation we get :

\begin{align}
  \lambda^3 - 9\lambda^2 + 9\lambda - 1 = 0
\end{align}

By Cayley-Hamilton theorem :
\begin{align}
  f(\lambda) = f(\vec{A}) = 0 \\
  \vec{A}^3 -9\vec{A}^2 + 9\vec{A} -1 =0
\end{align}

Multiplying the equation 0.5 by $\vec{A}^-1$ we get :
\begin{align}
  \vec{A}^2 -9\vec{A} +9\vec{I} - \vec{A}^{-1} = 0 \\
  \vec{A}^{-1} = \vec{A}^2 -9\vec{A} +9\vec{I}
\end{align}

Solving the above equation we get :
\begin{align}
   \vec{A}^{-1} = \myvec{1&3&3\\1&4&3\\1&3&4}
\end{align}
Hence ,
\begin{align}
    \lambda = 4
\end{align}





\end{document}


