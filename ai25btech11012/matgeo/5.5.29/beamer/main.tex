\documentclass{beamer}
\usetheme{Madrid}

\usepackage{amsmath, amssymb, amsthm}
\usepackage{graphicx}
\usepackage{gensymb}
\usepackage[utf8]{inputenc}
\usepackage{hyperref}
\usepackage{tikz}


\title{5.5.29 Matgeo}
\author{AI25BTECH11012 - Garige Unnathi}
\date{}

\begin{document}

\frame{\titlepage}

% Question frame
\begin{frame}
\frametitle{Question}
If the inverse of the matrix  $\begin{bmatrix}7 &-3&-3\\
                                    -1&1&0\\
                                     -1&0&1\end{bmatrix}$is the matrix  $\begin{bmatrix}1&3&3\\1&\lambda&3\\1&3&4\end{bmatrix}$ , then find the value of $\lambda$ .
\end{frame}


% Solution steps
\begin{frame}
\frametitle{Solution}
   
Let : 
\begin{align*}
    \textbf{A}  = \begin{bmatrix}7 &-3&-3\\
                      -1&1&0\\
                      -1&0&1\end{bmatrix}
\end{align*}
The characteristic equation for a matrix $\vec{A}$ is  
\begin{align}
 f(\lambda) =  \lvert \textbf{A} - \lambda\textbf{I} \rvert = 0\\
 f(\lambda) = \begin{vmatrix}7-\lambda &-3&-3\\
                      -1&1-\lambda&0\\
                      -1&0&1-\lambda\end{vmatrix} = 0
\end{align}
\end{frame}



\begin{frame}
Solving the above equation we get :

\begin{align}
  \lambda^3 - 9\lambda^2 + 9\lambda - 1 = 0
\end{align}

By Cayley-Hamilton theorem :
\begin{align}
  f(\lambda) = f(\textbf{A}) = 0 \\
  \textbf{A}^3 -9\textbf{A}^2 + 9\textbf{A} -1 =0
\end{align}
\end{frame}

\begin{frame}
\frametitle{Solution}
Multiplying the equation 5 by $\textbf{A}^-1$ we get :
\begin{align}
  \textbf{A}^2 -9\textbf{A} +9\textbf{I} - \textbf{A}^{-1} = 0 \\
  \textbf{A}^{-1} = \textbf{A}^2 -9\textbf{A} +9\textbf{I}
\end{align}

Solving the above equation we get :
\begin{align}
   \textbf{A}^{-1} = \begin{bmatrix}1&3&3\\1&4&3\\1&3&4\end{bmatrix}
\end{align}
Hence ,
\begin{align}
    \lambda = 4
\end{align}
\end{frame}


\end{document}
