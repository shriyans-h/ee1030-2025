\let\negmedspace\undefined
\let\negthickspace\undefined
\documentclass[journal]{IEEEtran}
\usepackage[a5paper, margin=10mm, onecolumn]{geometry}
%\usepackage{lmodern} % Ensure lmodern is loaded for pdflatex
\usepackage{tfrupee} % Include tfrupee package

\setlength{\headheight}{1cm} % Set the height of the header box
\setlength{\headsep}{0mm}     % Set the distance between the header box and the top of the text

\usepackage{gvv-book}
\usepackage{gvv}
\usepackage{cite}
\usepackage{amsmath,amssymb,amsfonts,amsthm}
\usepackage{algorithmic}
\usepackage{graphicx}
\usepackage{textcomp}
\usepackage{xcolor}
\usepackage{txfonts}
\usepackage{listings}
\usepackage{enumitem}
\usepackage{mathtools}
\usepackage{gensymb}
\usepackage{comment}
\usepackage[breaklinks=true]{hyperref}
\usepackage{tkz-euclide} 
\usepackage{listings}
\def\inputGnumericTable{}                                 
\usepackage[latin1]{inputenc}                                
\usepackage{color}                                            
\usepackage{array}                                            
\usepackage{longtable}                                       
\usepackage{calc}                                             
\usepackage{multirow}                                         
\usepackage{hhline}                                           
\usepackage{ifthen}                                           
\usepackage{lscape}
\begin{document}

\bibliographystyle{IEEEtran}


\title{4.8.8}
\author{AI25BTECH11012 - GARIGE UNNATHI}
% \maketitle
% \newpage
% \bigskip
{\let\newpage\relax\maketitle}


\renewcommand{\thefigure}{\theenumi}
\renewcommand{\thetable}{\theenumi}
\setlength{\intextsep}{10pt} % Space between text and floats


\numberwithin{equation}{enumi}
\numberwithin{figure}{enumi}

\vspace{-1cm}

\textbf{Question}:\\
Find the equation of the plane passing through the point (-1,3,2) and perpendicular
to the planes x + 2y + 3z = 5 and 3x + 3y + z = 0. .


\textbf{Solution: }

 

The equation of a plane can be given by the formula :
         
\begin{align*}
    \mathbf{n^{T}}\vec{x} = c
\end{align*}
From the above formula we can write :
\begin{align}
  x + 2y + 3z = 5  = \myvec{1\\2\\3}^{T}\vec{x} = 5\\
  3x + 3y + z = 0  = \myvec{3\\3\\1}^{T}\vec{x} = 0
\end{align}

\begin{table}[h!]    
      \centering
      \let\negmedspace\undefined
\let\negthickspace\undefined
\documentclass[journal,12pt,onecolumn]{IEEEtran}
\usepackage{cite}
\usepackage{amsmath,amssymb,amsfonts,amsthm}
\usepackage{algorithmic}
\usepackage{graphicx}
\graphicspath{{./figs/}}
\usepackage{textcomp}
\usepackage{xcolor}
\usepackage{txfonts}
\usepackage{listings}
\usepackage{enumitem}
\usepackage{mathtools}
\usepackage{gensymb}
\usepackage{comment}
\usepackage{caption}
\usepackage[breaklinks=true]{hyperref}
\usepackage{tkz-euclide} 
\usepackage{listings}
\usepackage{gvv}                                        
%\def\inputGnumericTable{}                                 
\usepackage[latin1]{inputenc}     
\usepackage{xparse}
\usepackage{color}                                            
\usepackage{array}
\usepackage{longtable}                                       
\usepackage{calc}                                             
\usepackage{multirow}
\usepackage{multicol}
\usepackage{hhline}                                           
\usepackage{ifthen}                                           
\usepackage{lscape}
\usepackage{tabularx}
\usepackage{array}
\usepackage{float}
\usepackage{parskip}
\newtheorem{theorem}{Theorem}[section]
\newtheorem{problem}{Problem}
\newtheorem{proposition}{Proposition}[section]
\newtheorem{lemma}{Lemma}[section]
\newtheorem{corollary}[theorem]{Corollary}
\newtheorem{example}{Example}[section]
\newtheorem{definition}[problem]{Definition}
\newcommand{\BEQA}{\begin{eqnarray}}
\newcommand{\EEQA}{\end{eqnarray}}
\newcommand{\define}{\stackrel{\triangle}{=}}
\theoremstyle{remark}
\newtheorem{rem}{Remark}

\begin{document}
\title{2.10.73}
\author{EE25BTECH11045 - P.Navya Priya}
\maketitle
\renewcommand{\thefigure}{\theenumi}
\renewcommand{\thetable}{\theenumi}

\textbf{Question:}

 Let $\vec{A}$, $\vec{B}$ and $\vec{C}$ be unit vectors. Suppose that  $\vec{A}\cdot\vec{B}$ =  $\vec{A}\cdot\vec{C}$= 0, and that the angle between  $\vec{B}$ and  $\vec{C}$ is $\frac{\pi}{6}$. Then $\vec{A}= \pm2(\vec{B}\times\vec{C})$
 \vspace{0.5cm}

\textbf{Solution:}

Let us solve the given equation theoretically and then verify the solution computationally.\\

Since $\vec{A}\cdot\vec{B}$ =  $\vec{A}\cdot\vec{C}$= 0, it follows that $\vec{A}$ is perpendicular to both $\vec{B}$ and $\vec{C}$. Therefore A is parallel(or anti-parallel) to the cross product of $\vec{B}$ and $\vec{C}$.

\begin{align}
    \vec{A}\,=\,\lambda(\vec{B}\times\vec{C})
\end{align}
From the given question,
\begin{align}
    \vec{B}^\top\vec{C}=\text{cos}\brak{\frac{\pi}{6}}
\end{align}
We know that,
\begin{align}
 \vec{B}^\top\vec{C}^{2}\,+\,||\vec{B}\times\vec{C}||^{2}=||\vec{B}||^{2}||\vec{C}||^{2}
\end{align}
\begin{align}
 \implies   ||\vec{B}\times\vec{C}||^{2}\,=\,\frac{1}{4}
\end{align}
\begin{align}
   \implies   ||\vec{B}\times\vec{C}||\,=\,\frac{1}{2}
\end{align}
As $\vec{A}$ is a unit vector,\\
from(1)
\begin{align}
    ||\vec{A}||\,=\,||\lambda(\vec{B}\times\vec{C})||
\end{align}
\begin{align}
    1\,=\,|\lambda|\frac{1}{2}
\end{align}
Hence

\begin{align}
 \lambda\,=\,\pm 2
\end{align}

\begin{align}
    \therefore \vec{A}\,=\,\pm 2(\vec{B}\times\vec{C})
\end{align}
\newpage
To verify the solution computationally let us assume the vectors $\vec{B}$ and $\vec{C}$ as 

\centering
$\vec{B}=\myvec{1\\0\\0}$ and $\vec{C}=\myvec{\frac{\sqrt{3}}{2}\\[4pt]\frac{1}{2}\\0}$

\begin{figure}[H]
\centering
\includegraphics[width=0.7\columnwidth]{figs/graph.png}
\label{fig:graph.png}
\end{figure}
\end{figure}
\end{document}




































      \caption{Variables Used}
      \label{}
    \end{table}


Let us assume the equation of the plane to be

\begin{align}
      \mathbf{n^{T}}\vec{x} = 1\\
      or \\
      \mathbf{x^{T}}\vec{n} = 1
\end{align}

As point $\vec{A}$ lies on the plane we can write :
\begin{align}
    \mathbf{A^{T}}\vec{n} = 1
\end{align}

If two planes are perpendicular then there normal vectors must also be perpendicular ,using this we can write :
\begin{align}
    \mathbf{n_1^{T}}\vec{n} = 0\\
    \mathbf{n_2^{T}}\vec{n} = 0
\end{align}

Combining equations 0.6,0.7 and 0.8 ,we get :
\begin{align}
    \mathbf{(A\quad n_1\quad n_2)^{T}}\vec{n} = \myvec{-1 & 3& 2\\
                                                      1 & 2 & 3\\
                                                      3 & 3& 1}\vec{n} = \myvec{1\\0\\0}
\end{align}
Solving the above equation by row reduction we get :
\begin{align}
    \vec{n} = \myvec{-\frac{7}{25} \\ \frac{8}{25} \\ -\frac{3}{25}} = \frac{1}{25}\myvec{-7\\8\\-3}
\end{align}

From the equation 0.3 we can write the plane euation as :
\begin{align}
    \myvec{-7\\8\\-3}^{T}\vec{x} = 25
\end{align}

\begin{figure}[h!]
   \centering
   \includegraphics[width=0.7\linewidth]{/Users/unnathi/Documents/ee1030-2025/ai25btech11012/matgeo/4.8.8/figs/fig.png}
   \caption{}
   \label{stemplot}
\end{figure}


\end{document}



