\documentclass{beamer}
\usepackage[utf8]{inputenc}

\usetheme{Madrid}
\usecolortheme{default}
\usepackage{amsmath,amssymb,amsfonts,amsthm}
\usepackage{txfonts}
\usepackage{tkz-euclide}
\usepackage{listings}
\usepackage{adjustbox}
\usepackage{array}
\usepackage{tabularx}
\usepackage{gvv}
\usepackage{lmodern}
\usepackage{circuitikz}
\usepackage{tikz}
\usepackage{graphicx}
\usepackage{mathtools}
\setbeamertemplate{page number in head/foot}[totalframenumber]

\usepackage{tcolorbox}
\tcbuselibrary{minted,breakable,xparse,skins}



\definecolor{bg}{gray}{0.95}
\DeclareTCBListing{mintedbox}{O{}m!O{}}{%
  breakable=true,
  listing engine=minted,
  listing only,
  minted language=#2,
  minted style=default,
  minted options={%
    linenos,
    gobble=0,
    breaklines=true,
    breakafter=,,
    fontsize=\small,
    numbersep=8pt,
    #1},
  boxsep=0pt,
  left skip=0pt,
  right skip=0pt,
  left=25pt,
  right=0pt,
  top=3pt,
  bottom=3pt,
  arc=5pt,
  leftrule=0pt,
  rightrule=0pt,
  bottomrule=2pt,
  toprule=2pt,
  colback=bg,
  colframe=orange!70,
  enhanced,
  overlay={%
    \begin{tcbclipinterior}
    \fill[orange!20!white] (frame.south west) rectangle ([xshift=20pt]frame.north west);
    \end{tcbclipinterior}},
  #3,
}
\lstset{
    language=C,
    basicstyle=\ttfamily\small,
    keywordstyle=\color{blue},
    stringstyle=\color{orange},
    commentstyle=\color{green!60!black},
    numbers=left,
    numberstyle=\tiny\color{gray},
    breaklines=true,
    showstringspaces=false,
}

\title 
{4.7.59}
\date{September 30, 2025}


\author 
{Vivek K Kumar - EE25BTECH11062}

\begin{document}


\frame{\titlepage}
\begin{frame}{Question}
Find the equation of a line perpendicular to the line $x + 2y + 3 = 0$ and passing
through the point $\brak{1, -2}$.
\end{frame}

\begin{frame}{Variables used}
\begin{align}
\end{align}
\begin{table}[H]    
  \centering
  \begin{tabular}[12pt]{ |c| c|}
    \hline
    \textbf{Name} & \textbf{Point}\\ 
    \hline
	Point A &\myvec{h \\ k}\\
    \hline 
 Point B &\myvec{x1 \\ y1}\\
    \hline
	  Point R &\myvec{x2 \\ y2}\\
    \hline
    
    \end{tabular}

  \caption{Variables used}
  \label{tab:4.7.59}
\end{table}

\end{frame}

\begin{frame}{Solution}

The given line can be expressed as 
\begin{align}
    \vec{n_1}^\top \vec{x} = c \\
    \myvec{1 & 2}\vec{x} = -3\\
\end{align}

As the given lines are perpendicular
\begin{align}
    \vec{n_1}^\top \vec{n_2} = 0 \\
    k = \frac{-1}{2}\\
    \vec{n_2} = \myvec{1 \\ -1/2}
\end{align}
\end{frame}

\begin{frame}{Solution}
The equation of the resulting line can be expressed as
\begin{align}
    \vec{n_2}^\top\brak{\vec{x} - \vec{A}} = 0\\
    \myvec{1 & \frac{-1}{2}}\vec{x} = \myvec{1 & \frac{-1}{2}}\myvec{1 \\ -2}\\
    \myvec{1 & \frac{-1}{2}}\vec{x} = 2
\end{align}
\end{frame}

\begin{frame}[fragile]
    \frametitle{Python - Importing libraries and checking system}
    \begin{lstlisting}
import sys
import numpy as np
import numpy.linalg as LA
import matplotlib.pyplot as plt
import matplotlib.image as mpimg
import math

from libs.line.funcs import *
from libs.triangle.funcs import *
from libs.conics.funcs import circ_gen

import subprocess
import shlex

print('Using termux?(y/n)')
y = input()
\end{lstlisting}
\end{frame}

\begin{frame}[fragile]
    \frametitle{Python - Writing coordinates of point and corresponding line direction vectors}
    \begin{lstlisting}
x = np.array([1, -2]).reshape(-1,1)
m1 = np.array([-2, 1]).reshape(-1,1)
m2 = np.array([1/2, 1]).reshape(-1,1)
\end{lstlisting}
\end{frame}

\begin{frame}[fragile]
    \frametitle{Python - Generating points and plotting}
    \begin{lstlisting}
p_l1 = line_gen(x-5*m1, x+5*m1)
p_l2 = line_gen(x-5*m2, x+5*m2)
fig = plt.figure()
ax = fig.add_subplot(111)
ax.plot(p_l1[0, :], p_l1[1, :], label = 'Given line')
ax.plot(p_l2[0, :], p_l2[1, :], label = 'Required Line')
\end{lstlisting}
\end{frame}

\begin{frame}[fragile]
    \frametitle{Python - Labelling points}
    \begin{lstlisting}
ax.scatter(np.array([x[0]]), np.array([x[1]]))
ax.text(x[0], x[1], s='(1,-2)')

ax.set_xlabel('$X$')
ax.set_ylabel('$Y$')
ax.legend(loc='best')
ax.grid(True) 
ax.axis('equal')
    \end{lstlisting}
\end{frame}

\begin{frame}[fragile]
    \frametitle{Python - Saving figure and opening it}
    \begin{lstlisting}
fig.savefig('../figs/fig.png')
print('Saved figure to ../figs/fig.png')

if(y == 'y'):
    subprocess.run(shlex.split('termux-open ../figs/fig.png'))
else:
    subprocess.run(["open",  "../figs/fig.png"])
    \end{lstlisting}
\end{frame}


\begin{frame}{Plot-Using only Python}
    \centering
    \includegraphics[width=\columnwidth, height=0.8\textheight, keepaspectratio]{../figs/fig.png}     
\end{frame}

\begin{frame}[fragile]
    \frametitle{C Code (0) - Importing libraries}

    \begin{lstlisting}
#include <stdio.h>
#include <stdlib.h>
#include <string.h>
#include <math.h>
#include <sys/socket.h>
#include <netinet/in.h>
#include <unistd.h>
#include "libs/matfun.h"
#include "libs/geofun.h"
    \end{lstlisting}
\end{frame}
\begin{frame}[fragile]
    \frametitle{C Code (1) - Function to Generate Points on a Line}

    \begin{lstlisting}

void point_gen(FILE *p_file, double **A, double **B, int rows, int cols, int npts){
    for(int i = 0; i <= npts; i++){
     double **output = Matadd(A, Matscale(Matsub(B, A, rows, cols), rows, cols, (double)i/npts), rows, cols);
     fprintf(p_file, "%lf, %lf\n", output[0][0], output[1][0]);
     freeMat(output, rows);
    }
}

    \end{lstlisting}
\end{frame}


\begin{frame}[fragile]
    \frametitle{C Code (2) - Function to write points of a line by using the given points to a file}

    \begin{lstlisting}
void write_points(double x1, double y1, double x2, double y2, double x3, double y3, int npts){
    int m = 2;
    int n = 1;

    double **A = createMat(m, n);
    double **B = createMat(m, n);
    double **C = createMat(m, n);

    B[0][0] = x2;
    B[1][0] = y2;
    \end{lstlisting}
\end{frame}
\begin{frame}[fragile]
    \frametitle{C Code (2) - Function to write points of a line by using the given points to a file}

    \begin{lstlisting}
    A[0][0] = x1;
    A[1][0] = y1;
    
    C[0][0] = x3;
    C[1][0] = y3;

    double **L1_1 = Matsub(A, Matscale(B, m, n, -5), m, n);
    double **L1_2 = Matsub(A, Matscale(B, m, n, 5), m, n);
    double **L2_1 = Matsub(A, Matscale(C, m, n, -5), m, n);
    double **L2_2 = Matsub(A, Matscale(C, m, n, 5), m, n);
    
    FILE *p_file;
    p_file = fopen("plot.dat", "w");
    
    if(p_file == NULL)
        printf("Error opening one of the data files\n");
    \end{lstlisting}
\end{frame}

\begin{frame}[fragile]
    \frametitle{C Code (2) - Function to write points of a line by using the given points to a file}
    \begin{lstlisting}
    point_gen(p_file, L1_1, L1_2, m, n, npts);
    point_gen(p_file, L2_1, L2_2, m, n, npts);

    freeMat(A, m);
    freeMat(B, m);
    freeMat(C, m);
    freeMat(L1_1, m);
    freeMat(L1_2, m);
    freeMat(L2_1, m);
    freeMat(L2_2, m);

    fclose(p_file);
}
    \end{lstlisting}
\end{frame}

\begin{frame}[fragile]
    \frametitle{Python Code (0) - Importing libraries and checking system}
    \begin{lstlisting}
import numpy as np
import matplotlib.pyplot as plt
import ctypes
import os
import sys
import subprocess
import math

print('Using termux? (y/n)')
termux = input()
\end{lstlisting}
\end{frame}

\begin{frame}[fragile]
    \frametitle{Python Code (1) - Using Shared Object}
    \begin{lstlisting}
lib_path = os.path.join(os.path.dirname(__file__), 'plot.so')
my_lib = ctypes.CDLL(lib_path)

my_lib.write_points.argtypes = [ctypes.c_double, ctypes.c_double, ctypes.c_double, ctypes.c_double, ctypes.c_double, ctypes.c_double, ctypes.c_int]
my_lib.write_points.restype = None
x = np.array([1, -2]).reshape(-1, 1)
m1 = np.array([-2, 1]).reshape(-1, 1)
m2 = np.array([1/2, 1]).reshape(-1, 1)
npts = 20000
\end{lstlisting}
\end{frame}

\begin{frame}[fragile]
    \frametitle{Python Code (2) - Loading points and plotting them}
    \begin{lstlisting}
my_lib.write_points(x[0][0], x[1][0], m1[0][0], m1[1][0], m2[0][0], m2[1][0], npts)

fig = plt.figure()
ax = fig.add_subplot(111)
labels = ['Given Line', 'Required Line']
pts = np.block([x])

for i,label in enumerate(labels):
    points = np.loadtxt('plot.dat', delimiter = ',', usecols=(0,1))[i*(npts+1):(i+1)*(npts+1)]
    ax.plot(points[:, 0], points[:, 1], label = label)

ax.text(pts[:, 0][0], pts[:, 0][1], s=f'(1, -2)')
\end{lstlisting}
\end{frame}

\begin{frame}[fragile]
    \frametitle{Python Code (3) - Labelling plot}
    \begin{lstlisting}
ax.set_xlabel('$X$')
ax.set_ylabel('$Y$')
ax.legend(loc='best')
ax.grid() 
ax.axis('equal')
    \end{lstlisting}
\end{frame}

\begin{frame}[fragile]
    \frametitle{Python Code (4) - Saving and displaying plot}
    \begin{lstlisting}
fig.savefig('../figs/fig2.png')
print('Saved figure to ../figs/fig2.png')

if(termux == 'y'):
    subprocess.run(shlex.split('termux-open ../figs/fig2.png'))
else:
    subprocess.run(["open",  "../figs/fig2.png"])
\end{lstlisting}
\end{frame}

\begin{frame}{Plot-Using Both C and Python}
    \centering
    \includegraphics[width=\columnwidth, height=0.8\textheight, keepaspectratio]{../figs/fig2.png}     
\end{frame}

\end{document}