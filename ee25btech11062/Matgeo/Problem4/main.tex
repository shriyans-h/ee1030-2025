\let\negmedspace\undefined
\let\negthickspace\undefined
\documentclass[journal]{IEEEtran}
\usepackage[a5paper, margin=10mm, onecolumn]{geometry}
\usepackage{lmodern} 
\usepackage{tfrupee} 
\setlength{\headheight}{1cm}
\setlength{\headsep}{0mm}   

\usepackage{gvv-book}
\usepackage{gvv}
\usepackage{cite}
\usepackage{amsmath,amssymb,amsfonts,amsthm}
\usepackage{algorithmic}
\usepackage{graphicx}
\usepackage{textcomp}
\usepackage{xcolor}
\usepackage{txfonts}
\usepackage{listings}
\usepackage{enumitem}
\usepackage{mathtools}
\usepackage{gensymb}
\usepackage{comment}
\usepackage[breaklinks=true]{hyperref}
\usepackage{tkz-euclide} 
\usepackage{listings}                             
\def\inputGnumericTable{}                                 
\usepackage[latin1]{inputenc}                                
\usepackage{color}                                            
\usepackage{array}                                            
\usepackage{longtable}                                       
\usepackage{calc}                                             
\usepackage{multirow}                                         
\usepackage{hhline}                                           
\usepackage{ifthen}                                           
\usepackage{lscape}
\usepackage{xparse}

\bibliographystyle{IEEEtran}

\title{2.9.13}
\author{EE25BTECH11062 - Vivek K Kumar}

\begin{document}
\maketitle

\renewcommand{\thefigure}{\theenumi}
\renewcommand{\thetable}{\theenumi}

\numberwithin{equation}{enumi}
\numberwithin{figure}{enumi} 

\textbf{Question}:\\
If $\vec{A}, \vec{B}$ and $\vec{C}$ are the position vectors of the points $\vec{A}\brak{2, 3, -4}$, $\vec{B}\brak{3, -4, -5}$ and $\vec{C}\brak{3, 2, -3}$ respectively, then $\norm{\vec{A}+\vec{B}+\vec{C}}$ is equal to:

\textbf{Solution: }
\begin{table}[H]    
  \centering
  \begin{tabular}[12pt]{ |c| c|}
    \hline
    \textbf{Name} & \textbf{Point}\\ 
    \hline
	Point A &\myvec{h \\ k}\\
    \hline 
 Point B &\myvec{x1 \\ y1}\\
    \hline
	  Point R &\myvec{x2 \\ y2}\\
    \hline
    
    \end{tabular}

  \caption{Variables Used}
  \label{tab:2.5.30}
\end{table}
Finding $\vec{A} + \vec{B} + \vec{C}$,

\begin{align}
\vec{A} + \vec{B} + \vec{C} &= \myvec{2\\3\\-4} + \myvec{3\\-4\\-5} + \myvec{3\\2\\-3}\\
&= \myvec{8\\1\\-12}
\end{align}

Now finding $\norm{\vec{A} + \vec{B} + \vec{C}}$,

\begin{align}
\norm{\vec{A} + \vec{B} + \vec{C}}^2 &= \brak{\vec{A+B+C}}^\top\brak{\vec{A+B+C}} \\
&= \myvec{8 & 1 & -12}\myvec{8 \\ 1 \\ -12}\\
&= 8^2 + 1^2 + \brak{-12}^2\\
&= 209
\end{align}
Hence, 
\begin{align}
\norm{\vec{A+B+C}} = \sqrt{209}
\end{align}
\begin{figure}[H]
   \centering
  \includegraphics[width=0.64\columnwidth]{figs/fig.png}
   \caption{Vectors $\vec{A}, \vec{B}, \vec{C} \text{ and } \vec{A+B+C}$}
   \label{stemplot}
\end{figure}
\end{document}  

