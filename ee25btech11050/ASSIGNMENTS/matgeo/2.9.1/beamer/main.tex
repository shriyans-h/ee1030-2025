\documentclass{beamer}
\usepackage[utf8]{inputenc}

\usetheme{Madrid}
\usecolortheme{default}
\usepackage{amsmath,amssymb,amsfonts,amsthm}
\usepackage{txfonts}
\usepackage{tkz-euclide}
\usepackage{listings}
\usepackage{adjustbox}
\usepackage{array}
\usepackage{tabularx}
\usepackage{gvv}
\usepackage{lmodern}
\usepackage{circuitikz}
\usepackage{tikz}
\usepackage{graphicx}

\setbeamertemplate{page number in head/foot}[totalframenumber]

\usepackage{tcolorbox}
\tcbuselibrary{minted,breakable,xparse,skins}



\definecolor{bg}{gray}{0.95}
\DeclareTCBListing{mintedbox}{O{}m!O{}}{%
	breakable=true,
	listing engine=minted,
	listing only,
	minted language=#2,
	minted style=default,
	minted options={%
		linenos,
		gobble=0,
		breaklines=true,
		breakafter=,,
		fontsize=\small,
		numbersep=8pt,
		#1},
	boxsep=0pt,
	left skip=0pt,
	right skip=0pt,
	left=25pt,
	right=0pt,
	top=3pt,
	bottom=3pt,
	arc=5pt,
	leftrule=0pt,
	rightrule=0pt,
	bottomrule=2pt,
	toprule=2pt,
	colback=bg,
	colframe=orange!70,
	enhanced,
	overlay={%
		\begin{tcbclipinterior}
			\fill[orange!20!white] (frame.south west) rectangle ([xshift=20pt]frame.north west);
	\end{tcbclipinterior}},
	#3,
}
\lstset{
	language=C,
	basicstyle=\ttfamily\small,
	keywordstyle=\color{blue},
	stringstyle=\color{orange},
	commentstyle=\color{green!60!black},
	numbers=left,
	numberstyle=\tiny\color{gray},
	breaklines=true,
	showstringspaces=false,
}
%------------------------------------------------------------
%This block of code defines the information to appear in the
%Title page
\title %optional
{2.9.1}
%\subtitle{A short story}

\author % (optional)
{Hema Havil - EE25BTECH11050}



\begin{document}
	
	\frame{\titlepage}
	\begin{frame}{Question}
		Jagdish has a field which is in the shape of a right-angled triangle AQC. He wants to leave a space in the form of a square PQRS inside the field for growing wheat and the remaining space for growing vegetables. In the field, there is a pole marked as O. Based on the above information, answer the following equations
        \begin{enumerate}[label=\alph*)]
            \item Taking O as the origin, P = (-200, 0) and Q = (200, 0). PQRS being a square, what are the coordinates of R and S?
            \item 
            \begin{enumerate}[label=\roman*)]
                \item What is the area of square PQRS ?
                \item What is the length of diagonal PR in PQRS ?
            \end{enumerate}
            \item If S divides CA in the ratio K : 1, what is the value of K, where A = (200, 800)?
        \end{enumerate}
	\end{frame}

	
\begin{frame}{Theoretical Solution}
         Given that,\\ AQC is a right angled triangle at point Q and PQRS is a square inside the $\Delta$AQC,\\ 
          (a)
             We were given two points 
            \begin{align}
                P=(-200,0),Q=(200,0)
            \end{align}
            Let,\\ X be the vector along the side PQ,\\ Y be the vector along the side QR,\\Z be the vector along the side PS then, \\
            \begin{align}
                \vec{X}=\vec{Q}-\vec{P}=\myvec{200\\0}-\myvec{-200\\0}
            \end{align}
            \begin{align}
                \vec{X}=\myvec{400\\0}
            \end{align}
            
            
\end{frame}
\begin{frame}{Theoretical Solution}
        Rotation vector for 2x2 matrix is 
            \begin{align}
                \vec{R_\theta}=\myvec{cos \theta \; -sin \theta\\sin \theta\;\;\;cos \theta}
            \end{align}
            Rotate the vector $\vec{X}$ by $90^{\circ}$ anticlockwise to get Y
            \begin{align}
                \vec{Y}=\vec{R_{90}}\vec{X}
            \end{align}
            \begin{align}
                \vec{Y}=\myvec{0\;-1\\1\;\;\;\;0}\myvec{400\\0}
            \end{align}
            \begin{align}
                \vec{Y}=\myvec{0\\400}
            \end{align}
            So the vector along the side QR is $\vec{Y}=\myvec{0\\400}$ then,
            \begin{align}
                \vec{Y}=\vec{R}-\vec{Q}
            \end{align}
            
	\end{frame}
    \begin{frame}{Theoretical Solution}
        \begin{align}
                \vec{R}=\vec{Y}+\vec{Q}
            \end{align}
            \begin{align}
                \vec{R}=\myvec{0\\400}+\myvec{200\\0}
            \end{align}
            \begin{align}
                \vec{R}=\myvec{200\\400}
            \end{align}
            Since the sides QR and PS are parallel, vectors $\vec{Y}=\vec{Z}$ then
            \begin{align}
                \vec{Z}=\vec{S}-\vec{P}
            \end{align}
            \begin{align}
                \vec{S}=\vec{Z}+\vec{P}
            \end{align}
            \begin{align}
                \vec{S}=\myvec{0\\400}+\myvec{-200\\0}
            \end{align}
    \end{frame}
    \begin{frame}{Theoretical Solution}
        \begin{align}
                \vec{S}=\myvec{-200\\400}
            \end{align}
            Therefore the coordinates of the points R and S are (200,400) and (-200,400)
            (b)
            \begin{enumerate}[label=(\roman*)]
                \item We know the points P(-200,0) and Q(200,0)\\
                   Let length of the side of the square PQRS be x then,
                   \begin{align}
                       x=\norm{\vec{Q}-\vec{P}}
                   \end{align}
                   \begin{align}
                       x=\norm{\myvec{400\\0}}=400
                   \end{align}\\
                
                Area of the square = $x^2$ = $(400)^2$ = 160000 sq units\\
                \item Length of diagnol of the square = $x\sqrt{2}$ = $400\sqrt{2}$ units\\
            \end{enumerate}
    \end{frame}
    \begin{frame}{Theoretical Solution}
        (c) Given the point A=(200,800)\\
            Since it was given that point S divides CA in the ratio K:1, this shows that points A,C and S are collinear. Since AQC is a right angled triangle, from this we can say that point C lies on X axis\\
            Let point C be (t,0), Consider the matrix M\\
            \begin{align}
                M = \myvec{200\;800\;\;1\\-200\;400\;1\\t\;\;\;\;\;0\;\;\;\;\;\;1}
            \end{align}
            {\large$R_1\rightarrow{}\frac{1}{200}R_1$}
            \begin{align}
                 M = \myvec{1\;\;\;\;\;\;4\;\;\;\;\;\frac{1}{200}\\-200\;400\;1\\t\;\;\;\;\;0\;\;\;\;\;\;1}
            \end{align}
            {\large$R_2\rightarrow{}R_2 + 200R_1$}\hspace{2cm}
            {\large$R_3\rightarrow{R_3 - tR_1}$}
    \end{frame}
    \begin{frame}{Theoretical Solution}
        \begin{align}
                 M = \myvec{1\;\;\;\;\;\;4\;\;\;\;\;\frac{1}{200}\\0\;\;\;\;\;1200\;\;\;\;\;2\\0\;\;\;-4t\;\;\;1-\frac{t}{200}}
            \end{align}
            {\large$R_2\rightarrow{\frac{1}{200}R_2}$}
            \begin{align}
                M = \myvec{1\;\;\;\;\;\;4\;\;\;\;\;\frac{1}{200}\\0\;\;\;\;\;1\;\;\;\;\;\frac{1}{600}\\0\;\;\;-4t\;\;\;1-\frac{t}{200}}
            \end{align}
            {\large$R_3\rightarrow{R_3 + 4tR_2}$}
            \begin{align}
                M = \myvec{1\;\;\;\;\;\;4\;\;\;\;\;\frac{1}{200}\\0\;\;\;\;\;1\;\;\;\;\;\frac{1}{600}\\0\;\;\;0\;\;\;\;1-\frac{t}{200}+\frac{4t}{600}}
            \end{align}
            Since the three points A,S and C are collinear,\\
            Rank of M = 2
    \end{frame}
    \begin{frame}{Theoretical Solution}
        \begin{align}
                1-\frac{t}{200}+\frac{4t}{600} = 0
            \end{align}
            \begin{align}
                1 + \frac{t}{600} = 0
            \end{align}
            \begin{align}
                \frac{t}{600}=-1
            \end{align}
            \begin{align}
                t= -600
            \end{align}
            Therefore point C=(-600,0), Now S divides CA in the ratio K:1,
            \begin{align}
                S = \frac{KA+C}{K+1}
            \end{align}
    \end{frame}
    \begin{frame}{Theoretical Solution}
            \begin{align}
                K=\frac{(S-A)^T(C-S)}{\norm{S-A}^2}
            \end{align}
        \begin{align}
                K=\frac{1}{(400)^2+(400)^2}\myvec{-400\;-400}\myvec{-400\\-400}
            \end{align}
            By solving ((c).12) we get K=1
    \end{frame}
    
	
	\begin{frame}[fragile]
	\frametitle{C Code- Ploting the given vectors}
	
	\begin{lstlisting}

#include <stdio.h>

typedef struct {
    int x, y;
} Point;

void get_triangle(Point* pts) {
    pts[0].x = 200; pts[0].y = 800;     // A
    pts[1].x = 200; pts[1].y = 0;       // Q
    pts[2].x = -600; pts[2].y = 0;      // C
}
	\end{lstlisting}
\end{frame}
\begin{frame}[fragile]
\frametitle{C Code- Ploting the given vectors}
    \begin{lstlisting}
    void get_square(Point* pts) {
    pts[0].x = -200; pts[0].y = 0;      // P
    pts[1].x = 200;  pts[1].y = 0;    // Q
    pts[2].x = 200; pts[2].y = 400;    // R
    pts[3].x = -200;  pts[3].y = 400;      // S (shared corner, so Q again)
}

    \end{lstlisting}
\end{frame}

\begin{frame}[fragile]
	\frametitle{Python Code using shared output}
	\begin{lstlisting}
		from ctypes import * 
import matplotlib.pyplot as plt

# Load the C shared library
lib = CDLL("./2.9.1.so")  # Ensure path matches your compiled output

class Point(Structure):
    _fields_ = [("x", c_int), ("y", c_int)]

triangle = (Point * 3)()
square = (Point * 4)()

lib.get_triangle(triangle)
lib.get_square(square)
	\end{lstlisting}
\end{frame}
\begin{frame}[fragile]
	\frametitle{Python Code using shared output}
	\begin{lstlisting}	
    tri_x = [triangle[i].x for i in range(3)] + [triangle[0].x]
    tri_y = [triangle[i].y for i in range(3)] + [triangle[0].y]
     sqr_x = [square[i].x for i in range(4)] + [square[0].x]
sqr_y = [square[i].y for i in range(4)] + [square[0].y]

plt.figure(figsize=(8, 8))
plt.plot(tri_x, tri_y, 'r-', label='Triangle')
plt.plot(sqr_x, sqr_y, 'b-', label='Square')

# Annotate triangle points
labels_tri = ['A', 'Q', 'C']
for i in range(3):
    plt.text(triangle[i].x, triangle[i].y, labels_tri[i])
	\end{lstlisting}
\end{frame}
\begin{frame}[fragile]
	\frametitle{Python Code using shared output}
	\begin{lstlisting}
# Annotate square points
labels_sqr = ['P', 'Q', 'R', 'S']
for i in range(4):
    plt.text(square[i].x, square[i].y, labels_sqr[i])

plt.scatter([0], [0], marker='x', color='black')
plt.text(0, 0, 'O', fontsize=12)
plt.xlim(-700, 300)
plt.ylim(-100, 900)
plt.xlabel('X')
plt.ylabel('Y')
plt.title('Solution plot')
plt.grid(True)
plt.legend()
plt.show()
	\end{lstlisting}
\end{frame}
\begin{frame}{Plot by python using shared output}
	\begin{center}
	\begin{figure}[H]
		\centering
		\includegraphics[width = 0.7\columnwidth]{figs/fig1.png}
		\caption{Plot of the vectors when y=3}
		\label{fig1}
	\end{figure}
	\end{center}
\end{frame}
\begin{frame}[fragile]
     \frametitle{Python code for the plot}
\begin{lstlisting}
    import matplotlib.pyplot as plt

# Triangle points
A = (200, 800)
Q = (200, 0)
C = (-600, 0)

# Square points
P = (-200, 0)
R = (200, 400)
S = (-200, 400)

# Lists for triangle
triangle_x = [A[0], Q[0], C[0], A[0]]
triangle_y = [A[1], Q[1], C[1], A[1]]


\end{lstlisting}
\end{frame}
\begin{frame}[fragile]
   \frametitle{Python code for the plot}
    \begin{lstlisting}
# Lists for square
square_x = [P[0], Q[0], R[0], S[0], P[0]]
square_y = [P[1], Q[1], R[1], S[1], P[1]]

plt.figure(figsize=(8, 8))

# Plot triangle
plt.plot(triangle_x, triangle_y, 'r-', label='Triangle')

# Plot square
plt.plot(square_x, square_y, 'b-', label='Square')

# Label triangle points
plt.text(A[0], A[1], 'A')
plt.text(Q[0], Q[1], 'Q')
plt.text(C[0], C[1], 'C')



 \end{lstlisting}
\end{frame}
 \begin{frame}[fragile]
       \frametitle{Python code for plot}
       \begin{lstlisting}
      # Label square points
plt.text(P[0], P[1], 'P')
plt.text(R[0], R[1], 'R')
plt.text(S[0], S[1], 'S')

# Mark and label the origin
plt.scatter([0], [0], marker='x', color='black')
plt.text(0, 0, 'O')

# To match axes and layout
plt.xlabel('X')
plt.ylabel('Y')
plt.title('Solution plot')
plt.xlim(-700, 300)
plt.ylim(-100, 900)
plt.grid(True)
plt.legend()
plt.show()
    \end{lstlisting}
 \end{frame}
     \begin{frame}{Plot of triangle and square}
       \begin{figure}
           \centering
           \includegraphics[width=0.7\linewidth]{figs/fig1.png}
           \caption{Plot of the points}
           \label{fig:placeholder}
       \end{figure}
         
     \end{frame}
\end{document}
