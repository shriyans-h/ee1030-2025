\documentclass{beamer}
\usepackage[utf8]{inputenc}
\usetheme{Madrid}
\usecolortheme{default}
\usepackage{amsmath,amssymb,amsfonts,amsthm}
\usepackage{txfonts}
\usepackage{tkz-euclide}
\usepackage{listings}
\usepackage{adjustbox}
\usepackage{array}
\usepackage{tabularx}
\usepackage{gvv}
\usepackage{lmodern}
\usepackage{circuitikz}
\usepackage{tikz}
\usepackage{graphicx}

\setbeamertemplate{page number in head/foot}[totalframenumber]

\usepackage{tcolorbox}
\tcbuselibrary{minted,breakable,xparse,skins}



\definecolor{bg}{gray}{0.95}
\DeclareTCBListing{mintedbox}{O{}m!O{}}{%
  breakable=true,
  listing engine=minted,
  listing only,
  minted language=#2,
  minted style=default,
  minted options={%
    linenos,
    gobble=0,
    breaklines=true,
    breakafter=,,
    fontsize=\small,
    numbersep=8pt,
    #1},
  boxsep=0pt,
  left skip=0pt,
  right skip=0pt,
  left=25pt,
  right=0pt,
  top=3pt,
  bottom=3pt,
  arc=5pt,
  leftrule=0pt,
  rightrule=0pt,
  bottomrule=2pt,
  toprule=2pt,
  colback=bg,
  colframe=orange!70,
  enhanced,
  overlay={%
    \begin{tcbclipinterior}
    \fill[orange!20!white] (frame.south west) rectangle ([xshift=20pt]frame.north west);
    \end{tcbclipinterior}},
  #3,
}
\lstset{
    language=C,
    basicstyle=\ttfamily\small,
    keywordstyle=\color{blue},
    stringstyle=\color{orange},
    commentstyle=\color{green!60!black},
    numbers=left,
    numberstyle=\tiny\color{gray},
    breaklines=true,
    showstringspaces=false,
}
%------------------------------------------------------------
%This block of code defines the information to appear in the
%Title page
\title %optional
{3.3.1}

%\subtitle{A short story}

\author % (optional)
{BALU-ai25btech11017}



\begin{document}


\frame{\titlepage}
\begin{frame}{Question}
Draw a triangle $\triangle ABC$ with 
\begin{align}
BC = 6 \,\text{cm}, \quad AB = 5 \,\text{cm}, \quad \text{and } \angle ABC = 60^\circ.
\end{align}\\ 
\end{frame}
\begin{frame}{Theoretical Solution}
Given three points\\
Let us solve the given equation theoretically and then verify the solution computationally \\
According to the question, \\
Take\\
\begin{align}
\vec{B}=\begin{myvec}{0\\0}\end{myvec}\
\vec{C}=\begin{myvec}{6\\0}\end{myvec}\
\vec{A}=\begin{myvec}{5\cos{60}\\5\sin{60}}\end{myvec}
\end{align}
   \begin{align}
 \vec{A}=\begin{myvec}{2.5\\2.5\sqrt{3}}\end{myvec}\
\end{align}
\end{frame}
\begin{frame}[fragile]
    \frametitle{C Code}

    \begin{lstlisting}
#include <stdio.h>
#include <math.h>

int main() {
    // Given values
    double BC = 6.0;
    double AB = 5.0;
    double angle_ABC = 60.0; // in degrees

    // Convert degrees to radians
    double angle_rad = angle_ABC * M_PI / 180.0;

    // Apply cosine rule: AC^2 = AB^2 + BC^2 - 2*AB*BC*cos(angle)
    double AC = sqrt(AB*AB + BC*BC - 2*AB*BC*cos(angle_rad));

    // Print results
    printf("Given: BC = %.2f cm, AB = %.2f cm, Angle ABC = %.2f degrees\n", BC, AB, angle_ABC);
   
     \end{lstlisting}
\end{frame}
\begin{frame}[fragile]
    \frametitle{C Code - Resultant velocity}
    \begin{lstlisting}
  printf("The length of AC = %.2f cm\n", AC);

    return 0;
}
    \end{lstlisting}
\end{frame}
\begin{frame}[fragile]
    \frametitle{Python Code}
    \begin{lstlisting}
import matplotlib.pyplot as plt
import numpy as np

# Given values
BC = 6
AB = 5
angle_ABC = np.radians(60)  # convert degrees to radians

# Use cosine rule to find AC
AC = np.sqrt(AB**2 + BC**2 - 2*AB*BC*np.cos(angle_ABC))

# Coordinates of points
B = np.array([0, 0])
C = np.array([BC, 0])
A = np.array([AB * np.cos(angle_ABC), AB * np.sin(angle_ABC)])



    \end{lstlisting}
\end{frame}

\begin{frame}[fragile]
    \frametitle{Python Code}
    \begin{lstlisting}
# Plot triangle
x_coords = [A[0], B[0], C[0], A[0]]
y_coords = [A[1], B[1], C[1], A[1]]

plt.figure(figsize=(6,6))
plt.plot(x_coords, y_coords, 'b-o')

# Label points
plt.text(A[0], A[1]+0.2, 'A', fontsize=12, color='red')
plt.text(B[0]-0.3, B[1]-0.3, 'B', fontsize=12, color='red')
plt.text(C[0]+0.1, C[1]-0.3, 'C', fontsize=12, color='red')
# Add side labels
plt.text((A[0]+B[0])/2 -0.5, (A[1]+B[1])/2, f"AB={AB}", fontsize=10, color="green")
    \end{lstlisting}
\end{frame}

\begin{frame}[fragile]
    \frametitle{Python Code}
    \begin{lstlisting}

plt.text((B[0]+C[0])/2, (B[1]+C[1])/2 -0.3, f"BC={BC}", fontsize=10, color="green")
plt.text((A[0]+C[0])/2 +0.2, (A[1]+C[1])/2, f"AC={AC:.2f}", fontsize=10, color="green")

# Formatting
plt.axis("equal")
plt.grid(True, linestyle="--", alpha=0.5)
plt.title("Triangle ABC with BC=6 cm, AB=5 cm, ∠ABC=60°")
# Save as image
plt.savefig("triangle_solution.png", dpi=300)
plt.show()

print("Triangle saved as 'triangle_solution.png'")
    \end{lstlisting}
\end{frame}

\begin{frame}{Plot}
    \centering
    \includegraphics[width=\columnwidth, height=0.8\textheight, keepaspectratio]{figs/triangle_solution.png}     
\end{frame}




\end{document}
