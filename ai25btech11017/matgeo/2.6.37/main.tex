 \let\negmedspace\undefined
\let\negthickspace\undefined
\documentclass[journal]{IEEEtran}
\usepackage[a5paper, margin=10mm, onecolumn]{geometry}
\usepackage{lmodern} % Ensure lmodern is loaded for pdflatex
\usepackage{tfrupee} % Include tfrupee package

\setlength{\headheight}{1cm} % Set the height of the header box
\setlength{\headsep}{0mm}     % Set the distance between the header box and the top of the text

\usepackage{gvv-book}
\usepackage{gvv}
\usepackage{cite}
\usepackage{amsmath,amssymb,amsfonts,amsthm}
\usepackage{algorithmic}
\usepackage{graphicx}
\usepackage{textcomp}
\usepackage{xcolor}
\usepackage{txfonts}
\usepackage{listings}
\usepackage{enumitem}
\usepackage{mathtools}
\usepackage{gensymb}
\usepackage{comment}
\usepackage[breaklinks=true]{hyperref}
\usepackage{tkz-euclide} 
\usepackage{listings}
\usepackage{gvv}                                        
\def\inputGnumericTable{}                       
\usepackage[latin1]{inputenc}                                
\usepackage{color}                                            
\usepackage{array}                                            
\usepackage{longtable}                                       
\usepackage{calc}                                             
\usepackage{multirow}                                         
\usepackage{hhline}                                           
\usepackage{ifthen}                                           
\usepackage{lscape}  
\usetikzlibrary{patterns}
\begin{document}
\bibliographystyle{IEEEtran}


\textbf{Question 2.6.37:} \\
The vector from origin to the points $A$ and $B$ are \begin{align}
\mathbf{a} = 2\hat{i} - 3\hat{j} + 2\hat{k} \quad \text{and} 
 \quad \mathbf{b} = 2\hat{i} + 3\hat{j} + \hat{k},
 \end{align}respectively, then the area of $\triangle OAB$ is \underline{\hspace{2cm}}.

\textbf{Solution:}
Given

\begin{align}
\vec a=\myvec{2\\-3\\2},\qquad
\vec b=\myvec{2\\3\\1}.
\end{align}
Using the triangle-area formula,
\begin{align}
\text{ar}(\triangle OAB)=\frac12\|(A-O)\times(B-O)\|
=\frac12\|\vec a\times\vec b\|. \; \; \; \; \; \; \; \;
 \end{align}
\begin{align}
\vec a\times\vec b=
\myvec{
\hat{\imath}&\hat{\jmath}&\hat{k}\\
2&-3&2\\
2&\phantom{-}3&1
}
=-9\,\hat{\imath}+2\,\hat{\jmath}+12\,\hat{k},
 \end{align}
hence
\begin{align}
\|\vec a\times\vec b\|=\sqrt{(-9)^2+2^2+12^2}=\sqrt{229}.
\end{align}
Therefore,
\begin{align}
\boxed{\text{area}(\triangle OAB)=\dfrac{\sqrt{229}}{2}}.
\end{align}

\newpage
\begin{figure}
    \centering
\includegraphics[width=0.5\linewidth]{figs/matg4.jpeg}
    \caption{}
    \label{fig:placeholder}
\end{figure}
\end{document}