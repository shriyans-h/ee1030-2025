\documentclass{beamer}
\usepackage[utf8]{inputenc}
\usetheme{Madrid}
\usecolortheme{default}
\usepackage{amsmath,amssymb,amsfonts,amsthm}
\usepackage{txfonts}
\usepackage{tkz-euclide}
\usepackage{listings}
\usepackage{adjustbox}
\usepackage{array}
\usepackage{tabularx}
\usepackage{gvv}
\usepackage{lmodern}
\usepackage{circuitikz}
\usepackage{tikz}
\usepackage{graphicx}

\setbeamertemplate{page number in head/foot}[totalframenumber]

\usepackage{tcolorbox}
\tcbuselibrary{minted,breakable,xparse,skins}



\definecolor{bg}{gray}{0.95}
\DeclareTCBListing{mintedbox}{O{}m!O{}}{%
  breakable=true,
  listing engine=minted,
  listing only,
  minted language=#2,
  minted style=default,
  minted options={%
    linenos,
    gobble=0,
    breaklines=true,
    breakafter=,,
    fontsize=\small,
    numbersep=8pt,
    #1},
  boxsep=0pt,
  left skip=0pt,
  right skip=0pt,
  left=25pt,
  right=0pt,
  top=3pt,
  bottom=3pt,
  arc=5pt,
  leftrule=0pt,
  rightrule=0pt,
  bottomrule=2pt,
  toprule=2pt,
  colback=bg,
  colframe=orange!70,
  enhanced,
  overlay={%
    \begin{tcbclipinterior}
    \fill[orange!20!white] (frame.south west) rectangle ([xshift=20pt]frame.north west);
    \end{tcbclipinterior}},
  #3,
}
\lstset{
    language=C,
    basicstyle=\ttfamily\small,
    keywordstyle=\color{blue},
    stringstyle=\color{orange},
    commentstyle=\color{green!60!black},
    numbers=left,
    numberstyle=\tiny\color{gray},
    breaklines=true,
    showstringspaces=false,
}
%------------------------------------------------------------
%This block of code defines the information to appear in the
%Title page
\title %optional
{2.6.37}

%\subtitle{A short story}

\author % (optional)
{RATHLAVATH JEEVAN -AI25BTECH11026}



\begin{document}


\frame{\titlepage}
\begin{frame}{Question}
The vector from origin to the points $A$ and $B$ are
\begin{align}
\mathbf{a} = 2\hat{i} - 3\hat{j} + 2\hat{k} 
\quad \text{and} \quad \mathbf{b} = 2\hat{i} + 3\hat{j} + \hat{k},
 \end{align}
 respectively, then the area of $\triangle OAB$ is \underline{\hspace{2cm}}.
 
\end{frame}
\begin{frame}{Theoretical Solution} 
\textbf{Solution:}\\
 \textbf{Given}  
\begin{align}
\vec a=\myvec{2\\-3\\2},\qquad
\vec b=\myvec{2\\3\\1}.
\end{align}
Using the triangle-area formula,
\begin{align}
\text{ar}(\triangle OAB)=\frac12\|(A-O)\times(B-O)\|
=\frac12\|\vec a\times\vec b\|. \; \; \; \; \; \; \; \;
 \end{align}
\begin{align}
\vec a\times\vec b=
\myvec{
\hat{\imath}&\hat{\jmath}&\hat{k}\\
2&-3&2\\
2&\phantom{-}3&1
}
=-9\,\hat{\imath}+2\,\hat{\jmath}+12\,\hat{k},
 \end{align}
 \end{frame}
 \begin{frame}{Theoretical Solution} 
\textbf{Solution:}\\
 \textbf{Given} 
hence
\begin{align}
\|\vec a\times\vec b\|=\sqrt{(-9)^2+2^2+12^2}=\sqrt{229}.
\end{align}
Therefore,
\begin{align}
\boxed{\text{area}(\triangle OAB)=\dfrac{\sqrt{229}}{2}}.
\end{align}

\end{frame}

\begin{frame}[fragile]
    \frametitle{C Code}
    \begin{lstlisting}
#include <stdio.h>
#include <math.h>

int main() {
    // Vectors a and b
    double ax = 2, ay = -3, az = 2;
    double bx = 2, by = 3,  bz = 1;

    // Cross product a * b
    double cx = ay*bz - az*by;
    double cy = az*bx - ax*bz;
    double cz = ax*by - ay*bx;

    // Magnitude of cross product
    double magnitude = sqrt(cx*cx + cy*cy + cz*cz);

     \end{lstlisting}
\end{frame}
\begin{frame}[fragile]
    \frametitle{C Code }
    \begin{lstlisting}
// Area of triangle OAB
    double area = 0.5 * magnitude;

    printf("The area of triangle OAB is: %.2f\n", area);

    return 0;
}
    \end{lstlisting}
\end{frame}
\begin{frame}[fragile]
    \frametitle{Python Code}
    \begin{lstlisting}
import numpy as np
import matplotlib.pyplot as plt
from mpl_toolkits.mplot3d.art3d import Poly3DCollection

# Vectors
a = np.array([2, -3, 2])
b = np.array([2, 3, 1])

# Cross product and area
cross = np.cross(a, b)
area = 0.5 * np.linalg.norm(cross)
print("Area of triangle OAB:", area)

# Points
origin = np.array([0, 0, 0])
A = a
B = b


    \end{lstlisting}
\end{frame}

\begin{frame}[fragile]
    \frametitle{Python Code}
    \begin{lstlisting}
    # Create 3D plot
fig = plt.figure()
ax = fig.add_subplot(111, projection='3d')
# Plot vectors a and b
ax.quiver(0, 0, 0, a[0], a[1], a[2], color='r', label='a = (2,-3,2)')
ax.quiver(0, 0, 0, b[0], b[1], b[2], color='b', label='b = (2,3,1)')

# Draw triangle OAB
verts = [[origin, A, B]]
ax.add_collection3d(Poly3DCollection(verts, alpha=0.3, facecolor='cyan'))

# Labels and legend
ax.set_xlabel('X')
ax.set_ylabel('Y')
ax.set_zlabel('Z')

    \end{lstlisting}
\end{frame}

\begin{frame}[fragile]
    \frametitle{Python Code}
    \begin{lstlisting}
    ax.legend()
# Set equal aspect ratio
ax.set_box_aspect([1,1,1])

# Save figure as image
plt.savefig("triangle_OAB.png", dpi=300)
plt.show()

    \end{lstlisting}
\end{frame}

\begin{frame}{Plot}
    \centering
    \includegraphics[width=\columnwidth, height=0.8\textheight, keepaspectratio]{beamer/figs/matg4.jpeg}     
\end{frame}




\end{document}