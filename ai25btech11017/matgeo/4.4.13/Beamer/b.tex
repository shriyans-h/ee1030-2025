\documentclass{beamer}
\usepackage[utf8]{inputenc}
\usetheme{Madrid}
\usecolortheme{default}
\usepackage{amsmath,amssymb,amsfonts,amsthm}
\usepackage{txfonts}
\usepackage{tkz-euclide}
\usepackage{listings}
\usepackage{adjustbox}
\usepackage{array}
\usepackage{tabularx}
\usepackage{gvv}
\usepackage{lmodern}
\usepackage{circuitikz}
\usepackage{tikz}
\usepackage{graphicx}

\setbeamertemplate{page number in head/foot}[totalframenumber]

\usepackage{tcolorbox}
\tcbuselibrary{minted,breakable,xparse,skins}



\definecolor{bg}{gray}{0.95}
\DeclareTCBListing{mintedbox}{O{}m!O{}}{%
  breakable=true,
  listing engine=minted,
  listing only,
  minted language=#2,
  minted style=default,
  minted options={%
    linenos,
    gobble=0,
    breaklines=true,
    breakafter=,,
    fontsize=\small,
    numbersep=8pt,
    #1},
  boxsep=0pt,
  left skip=0pt,
  right skip=0pt,
  left=25pt,
  right=0pt,
  top=3pt,
  bottom=3pt,
  arc=5pt,
  leftrule=0pt,
  rightrule=0pt,
  bottomrule=2pt,
  toprule=2pt,
  colback=bg,
  colframe=orange!70,
  enhanced,
  overlay={%
    \begin{tcbclipinterior}
    \fill[orange!20!white] (frame.south west) rectangle ([xshift=20pt]frame.north west);
    \end{tcbclipinterior}},
  #3,
}
\lstset{
    language=C,
    basicstyle=\ttfamily\small,
    keywordstyle=\color{blue},
    stringstyle=\color{orange},
    commentstyle=\color{green!60!black},
    numbers=left,
    numberstyle=\tiny\color{gray},
    breaklines=true,
    showstringspaces=false,
}
%------------------------------------------------------------
%This block of code defines the information to appear in the
%Title page
\title %optional
{4.4.13}

%\subtitle{A short story}

\author % (optional)
{BALU-ai25btech11017}



\begin{document}


\frame{\titlepage}
\begin{frame}{Question}
A line passes through the point with position vector 
\begin{align}
2\hat{i} - \hat{j} + 4\hat{k}
\end{align}
and is in the direction of the vector 
\begin{align}
\hat{i} + \hat{j} - 2\hat{k}.
\end{align}
Find the equation of the line.\\ 
\end{frame}
\begin{frame}{Theoretical Solution}
Let us solve the given equation theoretically and then verify the solution computationally \\
According to the question, \\
Given\\
\begin{align}
\vec{P}=\begin{myvec}{2\\-1\\4}\end{myvec}\
\vec{D}=\begin{myvec}{1\\1\\-2}\end{myvec}\
\end{align}
The equation of line is
   \begin{align}
 \vec{r}=\begin{myvec}{2\\-1\\4}\end{myvec}+\lambda\begin{myvec}{1\\1\\-2}
 \end{myvec}
\end{align}
\end{frame}
\begin{frame}[fragile]
    \frametitle{C Code}

    \begin{lstlisting}
#include <stdio.h>

int main() {
    // Point on line
    int x0 = 2, y0 = -1, z0 = 4;
    // Direction vector
    int a = 1, b = 1, c = -2;

    printf("Equation of the line passing through (2, -1, 4)\n");
    printf("and parallel to vector (1, 1, -2):\n\n");

    // Vector form
    printf("Vector form:\n");
    printf("r = (2, -1, 4) + t(1, 1, -2)\n\n");

    // Parametric form
    printf("Parametric form:\n");
    printf("x = %d + t\n", x0);
     \end{lstlisting}
\end{frame}
\begin{frame}[fragile]
    \frametitle{C Code }
    \begin{lstlisting}
     printf("y = %d + t\n", y0);
    printf("z = %d - 2t\n\n");

    // Symmetric form
    printf("Symmetric form:\n");
    printf("(x - %d)/%d = (y - %d)/%d = (z - %d)/%d\n",
           x0, a, y0, b, z0, c);

    return 0;
}
    \end{lstlisting}
\end{frame}
\begin{frame}[fragile]
    \frametitle{Python Code}
    \begin{lstlisting}
import numpy as np
import matplotlib.pyplot as plt
from mpl_toolkits.mplot3d import Axes3D

# Point and direction vector
P = np.array([2, -1, 4])
d = np.array([1, 1, -2])

# Parameter range for t
t = np.linspace(-5, 5, 100)

# Line points
x = P[0] + d[0]*t
y = P[1] + d[1]*t
z = P[2] + d[2]*t

# Plotting
fig = plt.figure()
ax = fig.add_subplot(111, projection='3d')



    \end{lstlisting}
\end{frame}

\begin{frame}[fragile]
    \frametitle{Python Code}
    \begin{lstlisting}
ax.plot(x, y, z, label="Line")
ax.scatter(P[0], P[1], P[2], color='red', s=50, label="Point (2,-1,4)")

# Labels
ax.set_xlabel('X axis')
ax.set_ylabel('Y axis')
ax.set_zlabel('Z axis')
ax.set_title('3D Line through (2,-1,4) in direction (1,1,-2)')
ax.legend()

# Save as picture
plt.savefig("line_3d.png", dpi=300)
plt.show()
    \end{lstlisting}
\end{frame}
\begin{frame}{Plot}
    \centering
    \includegraphics[width=\columnwidth, height=0.8\textheight, keepaspectratio]{figs/linefig.png}     
\end{frame}
\end{document}