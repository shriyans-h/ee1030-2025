\let\negmedspace\undefined
\let\negthickspace\undefined
\documentclass[journal]{IEEEtran}
\usepackage[a5paper, margin=10mm, onecolumn]{geometry}
%\usepackage{lmodern} % Ensure lmodern is loaded for pdflatex
\usepackage{tfrupee} % Include tfrupee package

\setlength{\headheight}{1cm} % Set the height of the header box
\setlength{\headsep}{0mm}     % Set the distance between the header box and the top of the text

\usepackage{gvv-book}
\usepackage{gvv}
\usepackage{cite}
\usepackage{amsmath,amssymb,amsfonts,amsthm}
\usepackage{algorithmic}
\usepackage{graphicx}
\usepackage{textcomp}
\usepackage{xcolor}
\usepackage{txfonts}
\usepackage{listings}
\usepackage{enumitem}
\usepackage{mathtools}
\usepackage{gensymb}
\usepackage{comment}
\usepackage[breaklinks=true]{hyperref}
\usepackage{tkz-euclide} 
\usepackage{listings}
% \usepackage{gvv}                                        
\def\inputGnumericTable{}                                 
\usepackage[latin1]{inputenc}                                
\usepackage{color}                                            
\usepackage{array}                                            
\usepackage{longtable}                                       
\usepackage{calc}                                             
\usepackage{multirow}                                         
\usepackage{hhline}                                           
\usepackage{ifthen}                                           
\usepackage{lscape}
\usepackage{circuitikz}
\tikzstyle{block} = [rectangle, draw, fill=blue!20, 
    text width=4em, text centered, rounded corners, minimum height=3em]
\tikzstyle{sum} = [draw, fill=blue!10, circle, minimum size=1cm, node distance=1.5cm]
\tikzstyle{input} = [coordinate]
\tikzstyle{output} = [coordinate]


\begin{document}

\bibliographystyle{IEEEtran}
\vspace{3cm}

\title{2.10.5}
\author{AI25BTECH11016-Varun}
 \maketitle
% \newpage
% \bigskip
{\let\newpage\relax\maketitle}
\renewcommand{\thefigure}{\theenumi}
\renewcommand{\thetable}{\theenumi}
\setlength{\intextsep}{10pt} % Space between text and floats

\numberwithin{figure}{enumi}
\renewcommand{\thetable}{\theenumi}
\textbf{Question}:\\

$A, B, C$ and $D$, are four points in a plane respectively such that 
$(A - D) \cdot (B - C) = (B - D) \cdot (C - A) = 0.$  
The point $D$, then, is the \underline{\hspace{1cm}} of $\triangle ABC$. 

  
\solution \\
Consider the equation,

\begin{align}
(\vec{A}- \vec{D} )^{T} (\vec{B}- \vec{C})
&=0
\end{align}
This implies line joining A and D is perpendicular to line joining B and C


Consider the equation,
\begin{align}
(\vec{B}- \vec{D})^{T}(\vec{C}- \vec{A})
&=0
\end{align}
This implies line joining B and D is perpendicular to line joining A and C

In $\triangle ABC$ ,\\
  side BC is perpendicular to AD\\
side AC is perpendicular to BD\\
\textbf{We know that,}\\
The altitudes(The perpendiculars drawn from a vertex to opposite sides) are concurrent at Orthocentre.\\
\textbf{Therefore,}\\
D must be Orthocentre of $\triangle ABC$\\
\textbf{Verification by example:}


\textbf{Let us take the points}\\
$\vec{A}=\myvec{1\\1},
\vec{B}=\myvec{5\\1}, 
\vec{C}=\myvec{3\\4}, 
\vec{D}=\myvec{3\\\tfrac{7}{3}}.$


\textbf{Checking the First condition:}
\begin{align}
(\vec{A}- \vec{D} )^{T} (\vec{B}- \vec{C})
&=0\\
L.H.S&=\Bigg(\myvec{1\\1}- \myvec{3\\\tfrac{7}{3}}\Bigg )^{T} \Bigg(\myvec{5\\1}- \myvec{3\\4}\Bigg)\\
&=\myvec{-2\\\tfrac{-4}{3}} ^{T} \myvec{2\\-3}\\
&=0\\
&=R.H.S\\
L.H.S&=R.H.S
\end{align}
\textbf{Checking the Second condition:}
\begin{align}
(\vec{B}- \vec{D} )^{T} (\vec{C}- \vec{A})
&=0\\
L.H.S&=\Bigg(\myvec{5\\1}- \myvec{3\\\tfrac{7}{3}}\Bigg )^{T} \Bigg(\myvec{3\\4}- \myvec{1\\1}\Bigg)\\\nonumber
&=\myvec{2\\\tfrac{-4}{3}} ^{T} \myvec{2\\3}\\\nonumber
&=0\\\nonumber
&=R.H.S\\
L.H.S&=R.H.S
\end{align}
\textbf{Let's take two points F and E which are foot of perpendiculars of altitudes drawn from vertices A and B respectively.}

\textbf{1.}The normal vector of    $\vec{F}-\vec{A}$ is\\ 
\begin{align}
\vec{n}&=\myvec{2\\-3}
\end{align}
The equation of the altitude from A (i.e AF) is
\begin{align}
\vec{n}^{T}(\vec{x-A})&=0\\
\myvec{2\\-3}^{T}(\vec{x}-\myvec{1\\1})&=0\\
\myvec{2&-3}(\vec{x}-\myvec{1\\1})&=0\\
\myvec{2&-3}(\vec{x})&=-1\\
\end{align}

\textbf{2.}The normal vector of    $\vec{E}-\vec{B}$ is\\ 
\begin{align}
\vec{n}&=\myvec{2\\3}
\end{align}
The equation of the altitude from B (i.e BE) is
\begin{align}
\vec{n}^{T}(\vec{x-B})&=0\\
\myvec{2\\3}^{T}(\vec{x}-\myvec{5\\1})&=0\\
\myvec{2&3}(\vec{x}-\myvec{5\\1})&=0\\
\myvec{2&3}(\vec{x})&=13\\
\end{align}
The intersection point of altitudes \textbf{orthocenter:H} can be obtained by solving the above two equations

\begin{align}
\myvec{2 & -3 \\ 2 & 3}\vec{x} = \myvec{-1 \\ 13}
\end{align}

\begin{align}
\myvec{2 & -3 & -1 \\ 2 & 3 & 13}
&\xrightarrow{\;R_2 \gets R_2 - R_1\;}
\myvec{2 & -3 & -1 \\ 0 & 6 & 14} \\
&\xrightarrow{\;R_2 \gets \tfrac{1}{6}R_2\;}
\myvec{2 & -3 & -1 \\ 0 & 1 & \tfrac{7}{3}} \\
&\xrightarrow{\;R_1 \gets R_1 + 3R_2\;}
\myvec{2 & 0 & 6 \\ 0 & 1 & \tfrac{7}{3}} \\
&\xrightarrow{\;R_1 \gets \tfrac{1}{2}R_1\;}
\myvec{1 & 0 & 3 \\ 0 & 1 & \tfrac{7}{3}}
\end{align}
which gives,
\begin{align}
H&=\myvec{3\\\tfrac{7}{3}}
\end{align}

Therefore,\\
The D we have taken matches with the orthocenter H of the given triangle


\begin{figure}[h]
    \centering
    \includegraphics[scale=0.5]{figs/2.10.5.jpg}
    \caption{}
    \label{fig:1}
\end{figure}
\end{document}
