\documentclass{beamer}
\usepackage[utf8]{inputenc}
\usetheme{Madrid}
\usecolortheme{default}
\usepackage{amsmath,amssymb,amsfonts,amsthm}
\usepackage{txfonts}
\usepackage{tkz-euclide}
\usepackage{listings}
\usepackage{adjustbox}
\usepackage{array}
\usepackage{tabularx}
\usepackage{gvv}
\usepackage{lmodern}
\usepackage{circuitikz}
\usepackage{tikz}
\usepackage{graphicx}

\setbeamertemplate{page number in head/foot}[totalframenumber]

\usepackage{tcolorbox}
\tcbuselibrary{minted,breakable,xparse,skins}



\definecolor{bg}{gray}{0.95}
\DeclareTCBListing{mintedbox}{O{}m!O{}}{%
  breakable=true,
  listing engine=minted,
  listing only,
  minted language=#2,
  minted style=default,
  minted options={%
    linenos,
    gobble=0,
    breaklines=true,
    breakafter=,,
    fontsize=\small,
    numbersep=8pt,
    #1},
  boxsep=0pt,
  left skip=0pt,
  right skip=0pt,
  left=25pt,
  right=0pt,
  top=3pt,
  bottom=3pt,
  arc=5pt,
  leftrule=0pt,
  rightrule=0pt,
  bottomrule=2pt,
  toprule=2pt,
  colback=bg,
  colframe=orange!70,
  enhanced,
  overlay={%
    \begin{tcbclipinterior}
    \fill[orange!20!white] (frame.south west) rectangle ([xshift=20pt]frame.north west);
    \end{tcbclipinterior}},
  #3,
}
\lstset{
    language=C,
    basicstyle=\ttfamily\small,
    keywordstyle=\color{blue},
    stringstyle=\color{orange},
    commentstyle=\color{green!60!black},
    numbers=left,
    numberstyle=\tiny\color{gray},
    breaklines=true,
    showstringspaces=false,
}
%------------------------------------------------------------
%This block of code defines the information to appear in the
%Title page
\title %optional
{2.10.5}

%\subtitle{A short story}

\author % (optional)
{Varun-ai25btech11016}



\begin{document}


\frame{\titlepage}
\begin{frame}{Question}
$A, B, C$ and $D$, are four points in a plane respectively such that 
$(A - D) \cdot (B - C) = (B - D) \cdot (C - A) = 0.$  
The point $D$, then, is the \underline{\hspace{1cm}} of $\triangle ABC$. 
\end{frame}



\begin{frame}{Theoretical Solution }

Consider the equation,

\begin{align}
\myvec{A - D} ^{T} \myvec{B - C}
&=0
\end{align}
This implies line joining A and D is perpendicular to line joining B and C


Consider the equation,
\begin{align}
\myvec{B - D} ^{T} \myvec{C - A}
&=0
\end{align}
This implies line joining B and D is perpendicular to line joining A and C

In $\triangle ABC$ ,\\
  side BC is perpendicular to AD\\
side AC is perpendicular to BD\\
\end{frame}
\begin{frame}{Conclusion}
\textbf{Therefore,}\\
D must be Orthocenter of $\triangle ABC$\\
\textbf{Since}\\
The line joining vertex and orthocenter is perpendicular to opposite side
\end{frame}
\begin{frame}{Verification by example:}
\textbf{Let us take the points}\\
$\vec{A}=\myvec{1\\1},
\vec{B}=\myvec{5\\1}, 
\vec{C}=\myvec{3\\4}, 
\vec{D}=\myvec{3\\\tfrac{7}{3}}.$


\textbf{Checking the First condition:}
\begin{align}
(\vec{A}- \vec{D} )^{T} (\vec{B}- \vec{C})
&=0\\
L.H.S&=\Bigg(\myvec{1\\1}- \myvec{3\\\tfrac{7}{3}}\Bigg )^{T} \Bigg(\myvec{5\\1}- \myvec{3\\4}\Bigg)\\
&=\myvec{-2\\\tfrac{-4}{3}} ^{T} \myvec{2\\-3}\\
&=0\\
&=R.H.S\\
L.H.S&=R.H.S
\end{align}
\end{frame}
\begin{frame}{Verification by example:}
\textbf{Checking the Second condition:}
\begin{align}
(\vec{B}- \vec{D} )^{T} (\vec{C}- \vec{A})
&=0\\
L.H.S&=\Bigg(\myvec{5\\1}- \myvec{3\\\tfrac{7}{3}}\Bigg )^{T} \Bigg(\myvec{3\\4}- \myvec{1\\1}\Bigg)\\\nonumber
&=\myvec{2\\\tfrac{-4}{3}} ^{T} \myvec{2\\3}\\\nonumber
&=0\\\nonumber
&=R.H.S\\
L.H.S&=R.H.S
\end{align}
\textbf{Let's take two points F and E which are foot of perpendiculars of altitudes drawn from vertices A and B respectively.}
\end{frame}
\begin{frame}{Verification by example:}
\textbf{1.}The normal vector of    $\vec{F}-\vec{A}$ is\\ 
\begin{align}
\vec{n}&=\myvec{2\\-3}
\end{align}
The equation of the altitude from A (i.e AF) is
\begin{align}
\vec{n}^{T}(\vec{x-A})&=0\\
\myvec{2\\-3}^{T}(\vec{x}-\myvec{1\\1})&=0\\
\myvec{2&-3}(\vec{x}-\myvec{1\\1})&=0\\
\myvec{2&-3}(\vec{x})&=-1\\
\end{align}
\end{frame}
\begin{frame}{Verification by example:}
\textbf{2.}The normal vector of    $\vec{E}-\vec{B}$ is\\ 
\begin{align}
\vec{n}&=\myvec{2\\3}
\end{align}
The equation of the altitude from B (i.e BE) is
\begin{align}
\vec{n}^{T}(\vec{x-B})&=0\\
\myvec{2\\3}^{T}(\vec{x}-\myvec{5\\1})&=0\\
\myvec{2&3}(\vec{x}-\myvec{5\\1})&=0\\
\myvec{2&3}(\vec{x})&=13\\
\end{align}
\end{frame}
\begin{frame}{Verification by example:}
The intersection point of altitudes \textbf{orthocenter:H} can be obtained by solving the above two equations

\begin{align}
\myvec{2 & -3 \\ 2 & 3}\vec{x} = \myvec{-1 \\ 13}
\end{align}

\begin{align}
\myvec{2 & -3 & -1 \\ 2 & 3 & 13}
&\xrightarrow{\;R_2 \gets R_2 - R_1\;}
\myvec{2 & -3 & -1 \\ 0 & 6 & 14} \\
&\xrightarrow{\;R_2 \gets \tfrac{1}{6}R_2\;}
\myvec{2 & -3 & -1 \\ 0 & 1 & \tfrac{7}{3}} \\
&\xrightarrow{\;R_1 \gets R_1 + 3R_2\;}
\myvec{2 & 0 & 6 \\ 0 & 1 & \tfrac{7}{3}} \\
&\xrightarrow{\;R_1 \gets \tfrac{1}{2}R_1\;}
\myvec{1 & 0 & 3 \\ 0 & 1 & \tfrac{7}{3}}
\end{align}
\end{frame}
\begin{frame}{Verification by example:}
which gives,
\begin{align}
H&=\myvec{3\\\tfrac{7}{3}}
\end{align}

Therefore,\\
The D we have taken coincides with the orthocenter H of the given triangle
\end{frame}
\begin{frame}[fragile]
    \frametitle{C code}
    \begin{lstlisting}



// orthocenter.c
#include <stdio.h>

// Function to compute orthocenter of triangle ABC
// A, B, C are arrays of length 2: [x, y]
// D is output array of length 2: [x, y]
void orthocenter(double *A, double *B, double *C, double *D) {
    // Slopes of sides
    double m_BC = (C[1] - B[1]) / (C[0] - B[0]);
    double m_AC = (C[1] - A[1]) / (C[0] - A[0]);
\end{lstlisting}
 
\end{frame}
\begin{frame}[fragile]
    \frametitle{C code}
    \begin{lstlisting}
    // Slopes of altitudes (negative reciprocal)
    double m_alt_A = -1.0 / m_BC;
    double m_alt_B = -1.0 / m_AC;

    // Equation of altitude from A: y - A_y = m_alt_A(x - A_x)
    // Equation of altitude from B: y - B_y = m_alt_B(x - B_x)

    double x_num = (m_alt_A*A[0] - m_alt_B*B[0] + B[1] - A[1]);
    double x_den = (m_alt_A - m_alt_B);
    double x = x_num / x_den;
    double y = m_alt_A*(x - A[0]) + A[1];

    D[0] = x;
    D[1] = y;
}


\end{lstlisting}
 
\end{frame}
\begin{frame}[fragile]
    \frametitle{C plus Python code}
    \begin{lstlisting}

import ctypes
import numpy as np
import matplotlib.pyplot as plt

# Load shared library (make sure libortho.so is in the same folder)
lib = ctypes.CDLL("./libortho.so")

# Define C function signature
lib.orthocenter.argtypes = [ctypes.POINTER(ctypes.c_double),
                            ctypes.POINTER(ctypes.c_double),
                            ctypes.POINTER(ctypes.c_double),
                            ctypes.POINTER(ctypes.c_double)]
\end{lstlisting}
 
\end{frame}
\begin{frame}[fragile]
    \frametitle{C plus Python code}
    \begin{lstlisting}
# Define triangle vertices
A = np.array([1.0, 1.0], dtype=np.double)
B = np.array([5.0, 1.0], dtype=np.double)
C = np.array([3.0, 4.0], dtype=np.double)
D = np.zeros(2, dtype=np.double)

# Call C function
lib.orthocenter(A.ctypes.data_as(ctypes.POINTER(ctypes.c_double)),
                B.ctypes.data_as(ctypes.POINTER(ctypes.c_double)),
                C.ctypes.data_as(ctypes.POINTER(ctypes.c_double)),
                D.ctypes.data_as(ctypes.POINTER(ctypes.c_double)))

print("Orthocenter D =", D)

# ---- Plotting ----
plt.figure(figsize=(6,6))
\end{lstlisting}
 
\end{frame}
\begin{frame}[fragile]
    \frametitle{C plus Python code}
    \begin{lstlisting}
# Triangle
plt.plot([A[0],B[0]],[A[1],B[1]],'b')
plt.plot([B[0],C[0]],[B[1],C[1]],'b')
plt.plot([C[0],A[0]],[C[1],A[1]],'b')

# Lines for perpendicularity check
plt.plot([A[0], D[0]], [A[1], D[1]], 'g--', label="AD")
plt.plot([B[0], C[0]], [B[1], C[1]], 'r--', label="BC")
plt.plot([B[0], D[0]], [B[1], D[1]], 'g--', label="BD")
plt.plot([A[0], C[0]], [A[1], C[1]], 'r--', label="AC")

# Points
plt.scatter(*A, color='red')
plt.scatter(*B, color='red')
plt.scatter(*C, color='red')
plt.scatter(*D, color='purple')
\end{lstlisting}
 
\end{frame}
\begin{frame}[fragile]
    \frametitle{C plus Python code}
    \begin{lstlisting}
# Labels
plt.text(A[0]+0.1, A[1], 'A')
plt.text(B[0]+0.1, B[1], 'B')
plt.text(C[0]+0.1, C[1], 'C')
plt.text(D[0]+0.1, D[1], 'D (Orthocenter)')

plt.legend()
plt.gca().set_aspect('equal', adjustable='box')
plt.grid(True)
plt.savefig("/sdcard/Matrix/ee1030-2025/ai25btech11016/Matgeo
/2.10.5/figs/2.10.5.png")
plt.show()


\end{lstlisting}
 
\end{frame}
\begin{frame}[fragile]
    \frametitle{Python}
    \begin{lstlisting}

 import numpy as np
import matplotlib.pyplot as plt

# Function to find line coefficients Ax + By = C given two points
def line_coeffs(p1, p2):
    A = p2[1] - p1[1]
    B = p1[0] - p2[0]
    C = A*p1[0] + B*p1[1]
    return A, B, C
\end{lstlisting}
 
\end{frame}
\begin{frame}[fragile]
    \frametitle{Python}
    \begin{lstlisting}
# Function to find intersection of two lines (given in Ax+By=C form)
def intersection(L1, L2):
    A1, B1, C1 = L1
    A2, B2, C2 = L2
    det = A1*B2 - A2*B1
    if det == 0:
        raise ValueError("Lines are parallel, no intersection.")
    x = (C1*B2 - C2*B1) / det
    y = (A1*C2 - A2*C1) / det
    return np.array([x, y])

# Define triangle vertices
A = np.array([1, 1])
B = np.array([5, 1])
C = np.array([3, 4])
\end{lstlisting}
 
\end{frame}
\begin{frame}[fragile]
    \frametitle{Python}
    \begin{lstlisting}
# Slopes of sides
L_BC = line_coeffs(B, C)
L_AC = line_coeffs(A, C)

# Altitude from A (perpendicular to BC, passes through A)
A1, B1, _ = L_BC
L_alt_A = (-B1, A1, -B1*A[0] + A1*A[1])

# Altitude from B (perpendicular to AC, passes through B)
A2, B2, _ = L_AC
L_alt_B = (-B2, A2, -B2*B[0] + A2*B[1])

# Orthocenter (D)
D = intersection(L_alt_A, L_alt_B)
\end{lstlisting}
 
\end{frame}
\begin{frame}[fragile]
    \frametitle{Python}
    \begin{lstlisting}
# Plotting
plt.figure(figsize=(6,6))
# Triangle
plt.plot([A[0],B[0]],[A[1],B[1]],'b')
plt.plot([B[0],C[0]],[B[1],C[1]],'b')
plt.plot([C[0],A[0]],[C[1],A[1]],'b')

# Lines showing perpendicularity
plt.plot([A[0], D[0]], [A[1], D[1]], 'g--', label="AD")
plt.plot([B[0], C[0]], [B[1], C[1]], 'r--', label="BC")
plt.plot([B[0], D[0]], [B[1], D[1]], 'g--', label="BD")
plt.plot([A[0], C[0]], [A[1], C[1]], 'r--', label="AC")

# Points
plt.scatter(*A, color='red')
plt.scatter(*B, color='red')
plt.scatter(*C, color='red')
plt.scatter(*D, color='purple')
\end{lstlisting}
 
\end{frame}
\begin{frame}[fragile]
    \frametitle{Python}
    \begin{lstlisting}
# Labels
plt.text(A[0]+0.1, A[1], 'A')
plt.text(B[0]+0.1, B[1], 'B')
plt.text(C[0]+0.1, C[1], 'C')
plt.text(D[0]+0.1, D[1], 'D (Orthocenter)')

plt.legend()
plt.gca().set_aspect('equal', adjustable='box')
plt.grid(True)
plt.savefig("/sdcard/Matrix/ee1030-2025/ai25btech11016/Matgeo/2.10.5/figs/2.10.5.png")
plt.show()
\end{lstlisting}
\end{frame}
\begin{frame}{Plot}


\begin{figure}[h]
    \centering
    \includegraphics[scale=0.5]{figs/2.10.5.jpg}
    \caption{}
    \label{fig:1}
\end{figure}
\end{frame}
\end{document}