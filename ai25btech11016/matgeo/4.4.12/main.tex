\let\negmedspace\undefined
\let\negthickspace\undefined
\documentclass[journal]{IEEEtran}
\usepackage[a5paper, margin=10mm, onecolumn]{geometry}
%\usepackage{lmodern} % Ensure lmodern is loaded for pdflatex
\usepackage{tfrupee} % Include tfrupee package

\setlength{\headheight}{1cm} % Set the height of the header box
\setlength{\headsep}{0mm}     % Set the distance between the header box and the top of the text

\usepackage{gvv-book}
\usepackage{gvv}
\usepackage{cite}
\usepackage{amsmath,amssymb,amsfonts,amsthm}
\usepackage{algorithmic}
\usepackage{graphicx}
\usepackage{textcomp}
\usepackage{xcolor}
\usepackage{txfonts}
\usepackage{listings}
\usepackage{enumitem}
\usepackage{mathtools}
\usepackage{gensymb}
\usepackage{comment}
\usepackage[breaklinks=true]{hyperref}
\usepackage{tkz-euclide} 
\usepackage{listings}
% \usepackage{gvv}                                        
\def\inputGnumericTable{}                                 
\usepackage[latin1]{inputenc}                                
\usepackage{color}                                            
\usepackage{array}                                            
\usepackage{longtable}                                       
\usepackage{calc}                                             
\usepackage{multirow}                                         
\usepackage{hhline}                                           
\usepackage{ifthen}                                           
\usepackage{lscape}
\usepackage{circuitikz}
\tikzstyle{block} = [rectangle, draw, fill=blue!20, 
    text width=4em, text centered, rounded corners, minimum height=3em]
\tikzstyle{sum} = [draw, fill=blue!10, circle, minimum size=1cm, node distance=1.5cm]
\tikzstyle{input} = [coordinate]
\tikzstyle{output} = [coordinate]


\begin{document}

\bibliographystyle{IEEEtran}
\vspace{3cm}

\title{4.4.12}
\author{AI25BTECH11016-Varun}
 \maketitle
% \newpage
% \bigskip
{\let\newpage\relax\maketitle}
\renewcommand{\thefigure}{\theenumi}
\renewcommand{\thetable}{\theenumi}
\setlength{\intextsep}{10pt} % Space between text and floats

\numberwithin{figure}{enumi}
\renewcommand{\thetable}{\theenumi}
\textbf{Question}:\\

Find the equation of the plane passing through the points $(2, 5, -3)$ , $(-2, -3, 5)$ and
$(5, 3, -3)$. Also find the point of intersection of this plane with the line passing
through points $(3, 1, 5)$ and $(-1, -3, -1)$.
.
  
\textbf{Solution}:\\
Let the vectors be
\begin{align}
\vec{A} &= \myvec{2 \\ 5 \\ -3}, &
\vec{B} &= \myvec{-2 \\ -3\\ 5}, &
\vec{C} &= \myvec{5 \\ 3 \\ -3}.
\end{align}

The vectors lying on the plane are
\begin{align}
\vec{B} - \vec{A} = \myvec{-2 \\ -3 \\ 5} - \myvec{2 \\ 5 \\ -3} 
= \myvec{-4 \\ -8 \\ 8}, \\
\vec{C} - \vec{A} = \myvec{5 \\ 3 \\ -3} - \myvec{2 \\ 5 \\ -3} 
= \myvec{3 \\ -2 \\ 0}.
\end{align}

The normal vector to the plane is given by the cross product
\begin{align}
\vec{n} &= (\vec{B} - \vec{A} ) \times (\vec{C} - \vec{A})\\
&= \myvec{-4 \\ -8 \\ 8} \times \myvec{3 \\ -2 \\ 0}\nonumber\\
&= \myvec{16 \\ 24 \\ 32}\nonumber
\end{align}

Equation of the plane passing through $\vec{A}$ is
\begin{align}
\vec{n}^{T} (\vec{x} - \vec{A}) &= 0, \\
\myvec{16 & 24 & 32} \left( \myvec{x \\ y \\ z} - \myvec{2 \\ 5 \\ -3} \right) &= 0.\nonumber
\end{align}
Hence, the equation of the plane is
\begin{align}
2x + 3y + 4z = 7.
\end{align}

Now, the line passes through 
\begin{align}
  \vec{P} = \myvec{3 \\ 1 \\ 5}, \quad \vec{Q} = \myvec{-1 \\ -3 \\ -1}.  
\end{align}


The direction vector is
\begin{align}
\vec{d} &= \vec{Q} - \vec{P} = \myvec{-1 \\ -3 \\ -1} - \myvec{3 \\ 1 \\ 5} 
= \myvec{-4 \\ -4 \\ -6}.
\end{align}

Thus, the parametric equation of the line is
\begin{align}
\vec{r} &= \vec{P} + \lambda \vec{d}, \\
\vec{r} &= \myvec{3 - 4\lambda \\ 1 - 4\lambda \\ 5 - 6\lambda}.
\end{align}

Substitute into the plane equation:
\begin{align} 
2x + 3y + 4z = 7\\
2(3 - 4\lambda) + 3(1 - 4\lambda) + 4(5 - 6\lambda) &= 7, \\
\lambda &= \tfrac{1}{2}.
\end{align}

Thus, the point of intersection is
\begin{align}
\vec{r} &= \myvec{3 - 4\left(\tfrac{1}{2}\right) \\ 1 - 4\left(\tfrac{1}{2}\right) \\ 5 - 6\left(\tfrac{1}{2}\right)} \\
&= \myvec{1 \\ -1 \\ 2}.
\end{align}

\textbf{Final Answer:} \\ 
The plane equation is
$2x + 3y + 4z = 7,$ and the point of intersection is
$\myvec{1 \\ -1 \\ 2}.$



\begin{figure}[H]
    \centering
    \includegraphics[scale=0.7]{figs/4.4.12.jpg}
    \caption{}
    \label{fig:1}
\end{figure}
\end{document}

