\let\negmedspace\undefined
\let\negthickspace\undefined
\documentclass[journal]{IEEEtran}
\usepackage[a5paper, margin=10mm, onecolumn]{geometry}
%\usepackage{lmodern} % Ensure lmodern is loaded for pdflatex
\usepackage{tfrupee} % Include tfrupee package

\setlength{\headheight}{1cm} % Set the height of the header box
\setlength{\headsep}{0mm}     % Set the distance between the header box and the top of the text

\usepackage{gvv-book}
\usepackage{gvv}
\usepackage{cite}
\usepackage{amsmath,amssymb,amsfonts,amsthm}
\usepackage{algorithmic}
\usepackage{graphicx}
\usepackage{textcomp}
\usepackage{xcolor}
\usepackage{txfonts}
\usepackage{listings}
\usepackage{enumitem}
\usepackage{mathtools}
\usepackage{gensymb}
\usepackage{comment}
\usepackage[breaklinks=true]{hyperref}
\usepackage{tkz-euclide} 
\usepackage{listings}
% \usepackage{gvv}                                        
\def\inputGnumericTable{}                                 
\usepackage[latin1]{inputenc}                                
\usepackage{color}                                            
\usepackage{array}                                            
\usepackage{longtable}                                       
\usepackage{calc}                                             
\usepackage{multirow}                                         
\usepackage{hhline}                                           
\usepackage{ifthen}                                           
\usepackage{lscape}
\usepackage{circuitikz}
\tikzstyle{block} = [rectangle, draw, fill=blue!20, 
    text width=4em, text centered, rounded corners, minimum height=3em]
\tikzstyle{sum} = [draw, fill=blue!10, circle, minimum size=1cm, node distance=1.5cm]
\tikzstyle{input} = [coordinate]
\tikzstyle{output} = [coordinate]


\begin{document}

\bibliographystyle{IEEEtran}
\vspace{3cm}

\title{4.4.12}
\author{AI25BTECH11016-Varun}
 \maketitle
% \newpage
% \bigskip
{\let\newpage\relax\maketitle}
\renewcommand{\thefigure}{\theenumi}
\renewcommand{\thetable}{\theenumi}
\setlength{\intextsep}{10pt} % Space between text and floats

\numberwithin{figure}{enumi}
\renewcommand{\thetable}{\theenumi}
\textbf{Question}:\\

Find the equation of the plane passing through the points $(2, 5, -3)$ , $(-2, -3, 5)$ and
$(5, 3, -3)$. Also find the point of intersection of this plane with the line passing
through points $(3, 1, 5)$ and $(-1, -3, -1)$.

\textbf{Solution}:\\
The points are 
\begin{align}
\vec{A}=\myvec{2\\5\\-3}  ,\vec{B}=\myvec{-2\\-3\\5} , \vec{C}=\myvec{5\\3\\-3}
\end{align}
\begin{align} 
\myvec{2 & 5 & -3 \\
-2 & -3 & 5  \\
5 & 3 & -3 }\vec{n}&=\myvec{1\\1\\1}\\
\myvec{2 & 5 & -3 & \vline & 1 \\
-2 & -3 & 5 & \vline & 1 \\
5 & 3 & -3 & \vline & 1}\nonumber
&\xrightarrow{R_2 \leftarrow R_2 + R_1,\; R_3 \leftarrow 2R_3 - 5R_1}
\myvec{2 & 5 & -3 & \vline & 1 \\
0 & 2 & 2 & \vline & 2 \\
0 & -19 & 9 & \vline & -3}\nonumber
\\[1em]
&\xrightarrow{R_3 \leftarrow 2R_3 + 19R_2}
\myvec{2 & 5 & -3 & \vline & 1 \\
0 & 2 & 2 & \vline & 2 \\
0 & 0 & 56 & \vline & 22}\nonumber
\\[1em]
&\xrightarrow{R_1 \leftarrow \tfrac{1}{2}R_1,\; R_2 \leftarrow \tfrac{1}{2}R_2,\; R_3 \leftarrow \tfrac{1}{56}R_3}
\myvec{1 & \tfrac{5}{2} & -\tfrac{3}{2} & \vline & \tfrac{1}{2} \\
0 & 1 & 1 & \vline & 1 \\
0 & 0 & 1 & \vline & \tfrac{11}{28}
}\nonumber
\\[1em]
&\xrightarrow{R_2 \leftarrow R_2 - R_3}
\myvec{1 & \tfrac{5}{2} & -\tfrac{3}{2} & \vline & \tfrac{1}{2} \\
0 & 1 & 0 & \vline & \tfrac{17}{28} \\
0 & 0 & 1 & \vline & \tfrac{11}{28}}\nonumber
\\[1em]
&\xrightarrow{R_1 \leftarrow R_1 + \tfrac{3}{2}R_3,\; R_1 \leftarrow R_1 - \tfrac{5}{2}R_2}
\myvec{1 & 0 & 0 & \vline & \tfrac{2}{7} \\
0 & 1 & 0 & \vline & \tfrac{3}{7} \\
0 & 0 & 1 & \vline & \tfrac{4}{7}}\nonumber
\end{align}
Hence the equation of the plane is 
\begin{align} 
\myvec{\tfrac{2}{7}&\tfrac{3}{7} &\tfrac{4}{7}}\vec{x}&=1
\end{align}

\textbf{The equation of the line passing through:}\\
\begin{align}
\vec{A}=\myvec{3\\1\\5},\vec{B}=\myvec{-1\\-3\\-1}\\
\end{align}
The direction vector of of the line
\begin{align}
\vec{m}&=\vec{A}-\vec{B}\\
&=\myvec{4\\4\\6}
\end{align}
\textbf{Vector equation of the line is}
\begin{align}
\vec{x}=\vec{A}+\lambda\vec{m}
\end{align}
Solving the equation of the plane ($\vec{n}^{T}\vec{x}=1$) and the line ($\vec{x}=\vec{A}+\lambda\vec{m}$),
\begin{align}
\vec{n}^{T}(\vec{A}+\lambda\vec{m})&=1\\
\vec{n}^{T}\vec{A}+\lambda\vec{n}^{T}\vec{m}&=1\\
\lambda&=\tfrac{1-\vec{n}^{T}\vec{A}}{\vec{n}^{T}\vec{m}}\\
\textbf{Substituting the $\vec{n},\vec{A},\vec{m}$}\\
\lambda&=\dfrac{1-\myvec{\tfrac{2}{7}\\\tfrac{3}{7} \\\tfrac{4}{7}}^{T}\myvec{3\\1\\5}}{\myvec{\tfrac{2}{7}\\\tfrac{3}{7} \\\tfrac{4}{7}}^{T}\myvec{4\\4\\6}}\\
\lambda&=\tfrac{1-(\tfrac{2}{7}.3+\tfrac{3}{7}.1+\tfrac{4}{7}.5)}{(\tfrac{2}{7}.4+\tfrac{3}{7}.4+\tfrac{4}{7}.6)}\\
\lambda&=\tfrac{1-\tfrac{29}{7}}{\tfrac{44}{7}}\\
\lambda&=\tfrac{-1}{2}\\
\text{From equation 8,}\\
\vec{x}&=\myvec{3+(\tfrac{-1}{2})4\\1+(\tfrac{-1}{2})4\\5+(\tfrac{-1}{2})6}\\
&=\myvec{1\\-1\\2}
\end{align}
the point of intersection is
\begin{align}
\vec{x}&=\myvec{1\\-1\\2}
\end{align}
\textbf{Therefore,}\\
the equation of the plane passing through the points (2, 5, -3) , (-2, -3, 5) and
(5, 3, -3) is $\myvec{\tfrac{2}{7}&\tfrac{3}{7} &\tfrac{4}{7}}\vec{x}=1$\\
and the point of intersection of the line with the plane is $\vec{x}=\myvec{1\\-1\\2}$

\begin{figure}[H]
    \centering
    \includegraphics[scale=0.7]{figs/4.4.12.jpg}
    \caption{}
    \label{fig:1}
\end{figure}

\end{document}

