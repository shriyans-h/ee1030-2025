\documentclass{beamer}
\usepackage[utf8]{inputenc}
\usetheme{Madrid}
\usecolortheme{default}
\usepackage{amsmath,amssymb,amsfonts,amsthm}
\usepackage{txfonts}
\usepackage{tkz-euclide}
\usepackage{listings}
\usepackage{adjustbox}
\usepackage{array}
\usepackage{tabularx}
\usepackage{gvv}
\usepackage{lmodern}
\usepackage{circuitikz}
\usepackage{tikz}
\usepackage{graphicx}

\setbeamertemplate{page number in head/foot}[totalframenumber]

\usepackage{tcolorbox}
\tcbuselibrary{minted,breakable,xparse,skins}



\definecolor{bg}{gray}{0.95}
\DeclareTCBListing{mintedbox}{O{}m!O{}}{%
  breakable=true,
  listing engine=minted,
  listing only,
  minted language=#2,
  minted style=default,
  minted options={%
    linenos,
    gobble=0,
    breaklines=true,
    breakafter=,,
    fontsize=\small,
    numbersep=8pt,
    #1},
  boxsep=0pt,
  left skip=0pt,
  right skip=0pt,
  left=25pt,
  right=0pt,
  top=3pt,
  bottom=3pt,
  arc=5pt,
  leftrule=0pt,
  rightrule=0pt,
  bottomrule=2pt,
  toprule=2pt,
  colback=bg,
  colframe=orange!70,
  enhanced,
  overlay={%
    \begin{tcbclipinterior}
    \fill[orange!20!white] (frame.south west) rectangle ([xshift=20pt]frame.north west);
    \end{tcbclipinterior}},
  #3,
}
\lstset{
    language=C,
    basicstyle=\ttfamily\small,
    keywordstyle=\color{blue},
    stringstyle=\color{orange},
    commentstyle=\color{green!60!black},
    numbers=left,
    numberstyle=\tiny\color{gray},
    breaklines=true,
    showstringspaces=false,
}
%------------------------------------------------------------
%This block of code defines the information to appear in the
%Title page
\title %optional
{4.4.12-Beamer}

%\subtitle{A short story}

\author % (optional)
{Varun-ai25btech11016}



\begin{document}


\frame{\titlepage}
\begin{frame}{Question}
Find the equation of the plane passing through the points $(2, 5, -3)$ , $(-2, -3, 5)$ and
$(5, 3, -3)$. Also find the point of intersection of this plane with the line passing
through points $(3, 1, 5)$ and $(-1, -3, -1)$.
\end{frame}
\begin{frame}{Theoretical Solution }
The points are 
\begin{align}
\vec{A}=\myvec{2\\5\\-3}  ,\vec{B}=\myvec{-2\\-3\\5} , \vec{C}=\myvec{5\\3\\-3}
\end{align}
\begin{align} 
\myvec{2 & 5 & -3 \\
-2 & -3 & 5  \\
5 & 3 & -3 }\vec{n}&=\myvec{1\\1\\1}\\
\myvec{2 & 5 & -3 & \vline & 1 \\
-2 & -3 & 5 & \vline & 1 \\
5 & 3 & -3 & \vline & 1}\nonumber
&\xrightarrow{R_2 \leftarrow R_2 + R_1,\; R_3 \leftarrow 2R_3 - 5R_1}
\myvec{2 & 5 & -3 & \vline & 1 \\
0 & 2 & 2 & \vline & 2 \\
0 & -19 & 9 & \vline & -3}\nonumber
\\[1em]
&\xrightarrow{R_3 \leftarrow 2R_3 + 19R_2}
\myvec{2 & 5 & -3 & \vline & 1 \\
0 & 2 & 2 & \vline & 2 \\
0 & 0 & 56 & \vline & 22}\nonumber\end{align}
\end{frame}
\begin{frame}{Theoretical Solution }
\begin{align}
\\[1em]
&\xrightarrow{R_1 \leftarrow \tfrac{1}{2}R_1,\; R_2 \leftarrow \tfrac{1}{2}R_2,\; R_3 \leftarrow \tfrac{1}{56}R_3}
\myvec{1 & \tfrac{5}{2} & -\tfrac{3}{2} & \vline & \tfrac{1}{2} \\
0 & 1 & 1 & \vline & 1 \\
0 & 0 & 1 & \vline & \tfrac{11}{28}
}\nonumber
\\[1em]
&\xrightarrow{R_2 \leftarrow R_2 - R_3}
\myvec{1 & \tfrac{5}{2} & -\tfrac{3}{2} & \vline & \tfrac{1}{2} \\
0 & 1 & 0 & \vline & \tfrac{17}{28} \\
0 & 0 & 1 & \vline & \tfrac{11}{28}}\nonumber
\\[1em]
&\xrightarrow{R_1 \leftarrow R_1 + \tfrac{3}{2}R_3,\; R_1 \leftarrow R_1 - \tfrac{5}{2}R_2}
\myvec{1 & 0 & 0 & \vline & \tfrac{2}{7} \\
0 & 1 & 0 & \vline & \tfrac{3}{7} \\
0 & 0 & 1 & \vline & \tfrac{4}{7}}\nonumber
\end{align}
\end{frame}
\begin{frame}{Theoretical Solution }
Hence the equation of the plane is 
\begin{align} 
\myvec{\tfrac{2}{7}&\tfrac{3}{7} &\tfrac{4}{7}}\vec{x}&=1
\end{align}

\textbf{The equation of the line passing through:}\\
\begin{align}
\vec{A}=\myvec{3\\1\\5},\vec{B}=\myvec{-1\\-3\\-1}\\
\end{align}
The direction vector of of the line
\begin{align}
\vec{m}&=\vec{A}-\vec{B}\\
&=\myvec{4\\4\\6}
\end{align}
\end{frame}
\begin{frame}{Theoretical Solution }
\textbf{Vector equation of the line is}
\begin{align}
\vec{x}=\vec{A}+\lambda\vec{m}
\end{align}
Solving the equation of the plane ($\vec{n}^{T}\vec{x}=1$) and the line ($\vec{x}=\vec{A}+\lambda\vec{m}$),
\begin{align}
\vec{n}^{T}(\vec{A}+\lambda\vec{m})&=1\\
\vec{n}^{T}\vec{A}+\lambda\vec{n}^{T}\vec{m}&=1\\
\lambda&=\tfrac{1-\vec{n}^{T}\vec{A}}{\vec{n}^{T}\vec{m}}\\
\textbf{Substituting the $\vec{n},\vec{A},\vec{m}$}\\
\lambda&=\dfrac{1-\myvec{\tfrac{2}{7}\\\tfrac{3}{7} \\\tfrac{4}{7}}^{T}\myvec{3\\1\\5}}{\myvec{\tfrac{2}{7}\\\tfrac{3}{7} \\\tfrac{4}{7}}^{T}\myvec{4\\4\\6}}\\
\end{align}
\end{frame}
\begin{frame}{Theoretical Solution }
\begin{align}
\lambda&=\tfrac{1-(\tfrac{2}{7}.3+\tfrac{3}{7}.1+\tfrac{4}{7}.5)}{(\tfrac{2}{7}.4+\tfrac{3}{7}.4+\tfrac{4}{7}.6)}\\
\lambda&=\tfrac{1-\tfrac{29}{7}}{\tfrac{44}{7}}\\
\lambda&=\tfrac{-1}{2}\\
\text{From equation 8,}\\
\vec{x}&=\myvec{3+(\tfrac{-1}{2})4\\1+(\tfrac{-1}{2})4\\5+(\tfrac{-1}{2})6}\\
&=\myvec{1\\-1\\2}
\end{align}
\end{frame}
\begin{frame}{Theoretical Solution }
the point of intersection is
\begin{align}
\vec{x}&=\myvec{1\\-1\\2}
\end{align}
\end{frame}
\begin{frame}{Conclusion}
\textbf{Therefore,}\\
the equation of the plane passing through the points (2, 5, -3) , (-2, -3, 5) and
(5, 3, -3) is $\myvec{\tfrac{2}{7}&\tfrac{3}{7} &\tfrac{4}{7}}\vec{x}=1$\\
and the point of intersection of the line with the plane is $\vec{x}=\myvec{1\\-1\\2}$
\end{frame}

\begin{frame}[fragile]
    \frametitle{C code}
    \begin{lstlisting}


#include <stdio.h>

typedef struct {
    double x, y, z;
} Vec3;

// Cross product
Vec3 cross(Vec3 a, Vec3 b) {
    Vec3 res;
    res.x = a.y*b.z - a.z*b.y;
    res.y = a.z*b.x - a.x*b.z;
    res.z = a.x*b.y - a.y*b.x;
    return res;
}
\end{lstlisting}
 
\end{frame}
\begin{frame}[fragile]
    \frametitle{C code}
    \begin{lstlisting}
    
// Dot product
double dot(Vec3 a, Vec3 b) {
    return a.x*b.x + a.y*b.y + a.z*b.z;
}

// Subtraction
Vec3 sub(Vec3 a, Vec3 b) {
    Vec3 res = {a.x-b.x, a.y-b.y, a.z-b.z};
    return res;
}

// Addition
Vec3 add(Vec3 a, Vec3 b) {
    Vec3 res = {a.x+b.x, a.y+b.y, a.z+b.z};
    return res;
}
\end{lstlisting}
 
\end{frame}
\begin{frame}[fragile]
    \frametitle{C code}
    \begin{lstlisting}
// Scalar multiply
Vec3 mul(Vec3 a, double s) {
    Vec3 res = {a.x*s, a.y*s, a.z*s};
    return res;
}

/*
   Function: find_intersection
   Input: points A, B, C (plane), P, Q (line)
   Output: intersection point (returned as Vec3)
*/
Vec3 find_intersection(Vec3 A, Vec3 B, Vec3 C, Vec3 P, Vec3 Q) {
    // Plane normal
    Vec3 AB = sub(B, A);
    Vec3 AC = sub(C, A);
    Vec3 n = cross(AB, AC);
\end{lstlisting}
 
\end{frame}
\begin{frame}[fragile]
    \frametitle{C code}
    \begin{lstlisting}
    // Plane constant
    double d = -dot(n, A);

    // Line direction
    Vec3 d_line = sub(Q, P);

    // Solve n.(P + t*d_line) + d = 0
    double t = -(dot(n, P) + d) / dot(n, d_line);

    // Intersection point
    Vec3 inter = add(P, mul(d_line, t));
    return inter;
}

\end{lstlisting}
 
\end{frame}
\begin{frame}[fragile]
    \frametitle{C plus Python code}
    \begin{lstlisting}

import ctypes
import numpy as np
import matplotlib.pyplot as plt
from mpl_toolkits.mplot3d import Axes3D

# Load the C shared library (compile first: gcc -shared -o geometry.so -fPIC geometry.c)
lib = ctypes.CDLL("./geometry.so")

# Define Vec3 struct (same as in C)
class Vec3(ctypes.Structure):
    _fields_ = [("x", ctypes.c_double),
                ("y", ctypes.c_double),
                ("z", ctypes.c_double)]
\end{lstlisting}
 
\end{frame}
\begin{frame}[fragile]
    \frametitle{C plus Python code}
    \begin{lstlisting}
# Configure function
lib.find_intersection.argtypes = [Vec3, Vec3, Vec3, Vec3, Vec3]
lib.find_intersection.restype = Vec3

# Define points
A = Vec3(2, 5, -3)
B = Vec3(-2, -3, 5)
C = Vec3(5, 3, -3)
P = Vec3(3, 1, 5)
Q = Vec3(-1, -3, -1)

# Call C function to get intersection
inter = lib.find_intersection(A, B, C, P, Q)
\end{lstlisting}
 
\end{frame}
\begin{frame}[fragile]
    \frametitle{C plus Python code}
    \begin{lstlisting}
# Convert to numpy arrays for plotting
A_np = np.array([A.x, A.y, A.z])
B_np = np.array([B.x, B.y, B.z])
C_np = np.array([C.x, C.y, C.z])
P_np = np.array([P.x, P.y, P.z])
Q_np = np.array([Q.x, Q.y, Q.z])
inter_np = np.array([inter.x, inter.y, inter.z])

# --- Compute plane for plotting ---
AB = B_np - A_np
AC = C_np - A_np
n = np.cross(AB, AC)   # normal
d = -np.dot(n, A_np)   # plane constant

xx, yy = np.meshgrid(range(-5, 8), range(-5, 8))
zz = (-n[0]*xx - n[1]*yy - d) / n[2]
\end{lstlisting}
 
\end{frame}
\begin{frame}[fragile]
    \frametitle{C plus Python code}
    \begin{lstlisting}
# --- Compute line for plotting ---
d_line = Q_np - P_np
t = np.linspace(-5, 5, 100)
line_points = P_np[:, None] + d_line[:, None] * t

# --- Plot ---
fig = plt.figure(figsize=(10, 7))
ax = fig.add_subplot(111, projection='3d')

# Plane
ax.plot_surface(xx, yy, zz, alpha=0.5, color='cyan')

# Line
ax.plot(line_points[0], line_points[1], line_points[2], color='red', label="Line")
\end{lstlisting}
 
\end{frame}
\begin{frame}[fragile]
    \frametitle{C plus Python code}
    \begin{lstlisting}
# Intersection
ax.scatter(*inter_np, color='black', s=60, label="Intersection")

# Points
ax.scatter(*A_np, color='blue', s=50, label='A')
ax.scatter(*B_np, color='green', s=50, label='B')
ax.scatter(*C_np, color='purple', s=50, label='C')
ax.scatter(*P_np, color='orange', s=50, label='P')
ax.scatter(*Q_np, color='brown', s=50, label='Q')

# Labels
ax.set_xlabel("X")
ax.set_ylabel("Y")
ax.set_zlabel("Z")
ax.legend()

plt.savefig("/sdcard/4.4.12.png")
plt.show()
\end{lstlisting}
 
\end{frame}
\begin{frame}[fragile]
    \frametitle{Python code}
    \begin{lstlisting}
    import numpy as np
import matplotlib.pyplot as plt
from mpl_toolkits.mplot3d import Axes3D

# Given points
A = np.array([2, 5, -3])
B = np.array([-2, -3, 5])
C = np.array([5, 3, -3])
P = np.array([3, 1, 5])
Q = np.array([-1, -3, -1])

# Step 1: Normal to the plane (AB x AC)
AB = B - A
AC = C - A
n = np.cross(AB, AC)
\end{lstlisting}
 
\end{frame}
\begin{frame}[fragile]
    \frametitle{Python code}
    \begin{lstlisting}

# Plane equation: n . (x - A) = 0
d = -np.dot(n, A)  # plane constant
print("Plane equation: {}x + {}y + {}z + {} = 0".format(n[0], n[1], n[2], d))

# Step 2: Line parametric equation
d_line = Q - P   # direction vector
t = np.linspace(-5, 5, 100)
line_points = P[:, None] + d_line[:, None] * t  # shape (3, len(t))
\end{lstlisting}
 
\end{frame}
\begin{frame}[fragile]
    \frametitle{Python code}
    \begin{lstlisting}

# Step 3: Find intersection of line with plane
# Solve n.(P + t*d_line) + d = 0
t_inter = -(np.dot(n, P) + d) / np.dot(n, d_line)
intersection = P + t_inter * d_line
print("Intersection point:", intersection)

# Step 4: Plot
fig = plt.figure(figsize=(10, 7))
ax = fig.add_subplot(111, projection='3d')
\end{lstlisting}
 
\end{frame}
\begin{frame}[fragile]
    \frametitle{Python code}
    \begin{lstlisting}

# Plot plane
xx, yy = np.meshgrid(range(-5, 8), range(-5, 8))
zz = (-n[0]*xx - n[1]*yy - d) / n[2]
ax.plot_surface(xx, yy, zz, alpha=0.5, color='cyan')

# Plot line
ax.plot(line_points[0], line_points[1], line_points[2], color='red', label="Line")

# Plot intersection
ax.scatter(*intersection, color='black', s=60, label="Intersection")
\end{lstlisting}
 
\end{frame}
\begin{frame}[fragile]
    \frametitle{Python code}
    \begin{lstlisting}

# Plot given points
ax.scatter(*A, color='blue', s=50, label='A')
ax.scatter(*B, color='green', s=50, label='B')
ax.scatter(*C, color='purple', s=50, label='C')
ax.scatter(*P, color='orange', s=50, label='P')
ax.scatter(*Q, color='brown', s=50, label='Q')
\end{lstlisting}
 
\end{frame}
\begin{frame}[fragile]
    \frametitle{Python code}
    \begin{lstlisting}

# Labels
ax.set_xlabel("X")
ax.set_ylabel("Y")
ax.set_zlabel("Z")
ax.legend()
ax.set_title("Plane, Line, and Intersection")
plt.savefig("\sdcard\4.4.12.png")
plt.show()

\end{lstlisting}
 \end{frame}

\begin{frame}[fragile]
    \frametitle{Plot}

\begin{figure}[H]
    \centering
    \includegraphics[scale=0.5]{figs/4.4.12.jpg}
    \caption{}
    \label{fig:1}
\end{figure}


 
\end{frame}
\end{document}