\let\negmedspace\undefined
\let\negthickspace\undefined
\documentclass[journal]{IEEEtran}
\usepackage[a5paper, margin=10mm, onecolumn]{geometry}
%\usepackage{lmodern} % Ensure lmodern is loaded for pdflatex
\usepackage{tfrupee} % Include tfrupee package

\setlength{\headheight}{1cm} % Set the height of the header box
\setlength{\headsep}{0mm}     % Set the distance between the header box and the top of the text

\usepackage{gvv-book}
\usepackage{gvv}
\usepackage{cite}
\usepackage{amsmath,amssymb,amsfonts,amsthm}
\usepackage{algorithmic}
\usepackage{graphicx}
\usepackage{textcomp}
\usepackage{xcolor}
\usepackage{txfonts}
\usepackage{listings}
\usepackage{enumitem}
\usepackage{mathtools}
\usepackage{gensymb}
\usepackage{comment}
\usepackage[breaklinks=true]{hyperref}
\usepackage{tkz-euclide} 
\usepackage{listings}
% \usepackage{gvv}                                        
\def\inputGnumericTable{}                                 
\usepackage[latin1]{inputenc}                                
\usepackage{color}                                            
\usepackage{array}                                            
\usepackage{longtable}                                       
\usepackage{calc}                                             
\usepackage{multirow}                                         
\usepackage{hhline}                                           
\usepackage{ifthen}                                           
\usepackage{lscape}
\usepackage{circuitikz}
\tikzstyle{block} = [rectangle, draw, fill=blue!20, 
    text width=4em, text centered, rounded corners, minimum height=3em]
\tikzstyle{sum} = [draw, fill=blue!10, circle, minimum size=1cm, node distance=1.5cm]
\tikzstyle{input} = [coordinate]
\tikzstyle{output} = [coordinate]


\begin{document}

\bibliographystyle{IEEEtran}
\vspace{3cm}

\title{1.7.2}
\author{AI25BTECH11016-Varun}
 \maketitle
% \newpage
% \bigskip
{\let\newpage\relax\maketitle}
\renewcommand{\thefigure}{\theenumi}
\renewcommand{\thetable}{\theenumi}
\setlength{\intextsep}{10pt} % Space between text and floats

\numberwithin{figure}{enumi}
\renewcommand{\thetable}{\theenumi}
\textbf{Question}:\\
If A(1, 2), O(0, 0), and C(a, 6) are collinear, then the value of a is

\solution \\
The given points are
\begin{align}
A = (1,2) \quad O = (0,0) \quad C = (a,6)
\end{align}

\begin{align}    
 \vec{A} - \vec{O} = \myvec{1 \\ 2}\\ 
 \vec{C} -\vec{O} = \myvec{a \\ 6}
\end{align}


Construct the matrix
\begin{align} 
M = \myvec{1 & a \\ 2 & 6}
\end{align}

For the points to be collinear, the two vectors $\Vec{OA}$ and $\vec{OC}$ must be linearly dependent.  
This means
\begin{align}
\operatorname{rank}(M) = 1 \quad \Leftrightarrow \quad \det(M) = 0 \
\end{align}


\begin{align}
\myvec{1 & a \\ 2 & 6}
&\xrightarrow{\;R_2 \gets R_2 - 2R_1\;}
\myvec{1 & a \\ 0 & 6 - 2a}

\end{align}


For the rank to drop,
\begin{align}
6 - 2a &= 0 \\
a &= 3
\end{align}

When $a=3$
\[
\myvec{1 & 3 \\ 0 & 0}
\]
is the reduced row-echelon form (rank $=1$)




The given points are collinear when

\begin{align}
a &= 3
\end{align}

\begin{figure}[H]
    \centering
    \includegraphics[scale=0.6]{figs/1.7.2.jpg}
    \caption{}
    \label{fig:1}
\end{figure}


\end{document}