\documentclass{beamer}
\usepackage[utf8]{inputenc}
\usetheme{Madrid}
\usecolortheme{default}
\usepackage{amsmath,amssymb,amsfonts,amsthm}
\usepackage{txfonts}
\usepackage{tkz-euclide}
\usepackage{listings}
\usepackage{adjustbox}
\usepackage{array}
\usepackage{tabularx}
\usepackage{gvv}
\usepackage{lmodern}
\usepackage{circuitikz}
\usepackage{tikz}
\usepackage{graphicx}

\setbeamertemplate{page number in head/foot}[totalframenumber]

\usepackage{tcolorbox}
\tcbuselibrary{minted,breakable,xparse,skins}



\definecolor{bg}{gray}{0.95}
\DeclareTCBListing{mintedbox}{O{}m!O{}}{%
  breakable=true,
  listing engine=minted,
  listing only,
  minted language=#2,
  minted style=default,
  minted options={%
    linenos,
    gobble=0,
    breaklines=true,
    breakafter=,,
    fontsize=\small,
    numbersep=8pt,
    #1},
  boxsep=0pt,
  left skip=0pt,
  right skip=0pt,
  left=25pt,
  right=0pt,
  top=3pt,
  bottom=3pt,
  arc=5pt,
  leftrule=0pt,
  rightrule=0pt,
  bottomrule=2pt,
  toprule=2pt,
  colback=bg,
  colframe=orange!70,
  enhanced,
  overlay={%
    \begin{tcbclipinterior}
    \fill[orange!20!white] (frame.south west) rectangle ([xshift=20pt]frame.north west);
    \end{tcbclipinterior}},
  #3,
}
\lstset{
    language=C,
    basicstyle=\ttfamily\small,
    keywordstyle=\color{blue},
    stringstyle=\color{orange},
    commentstyle=\color{green!60!black},
    numbers=left,
    numberstyle=\tiny\color{gray},
    breaklines=true,
    showstringspaces=false,
}
%------------------------------------------------------------
%This block of code defines the information to appear in the
%Title page
\title %optional
{1.7.2-Beamer}

%\subtitle{A short story}

\author % (optional)
{Varun-ai25btech11016}



\begin{document}


\frame{\titlepage}
\begin{frame}{Question}
If A(1, 2), O(0, 0), and C(a, 6) are collinear, then the value of a is

\end{frame}



\begin{frame}{Theoretical Solution }

 The given points are
\begin{align}
A = (1,2) \quad O = (0,0) \quad C = (a,6)
\end{align}

\begin{align}    
\Vec{A}-\Vec{O}& = \myvec{1 \\ 2}\\ 
 \Vec{C}-\Vec{O}& = \myvec{a \\ 6}
\end{align}
\end{frame}
\begin{frame}{Theoretical Solution }
Construct the matrix
\begin{align} 
M = \myvec{1 & a \\ 2 & 6}
\end{align}

For the points to be collinear, the two vectors $\Vec{OA}$ and $\vec{OC}$ must be linearly dependent



This means
\begin{align}
\operatorname{rank}(M) = 1 \quad \Leftrightarrow \quad \det(M) = 0 \
\end{align}



\begin{align}
\myvec{1 & a \\ 2 & 6}
&\xrightarrow{\;R_2 \gets R_2 - 2R_1\;}
\myvec{1 & a \\ 0 & 6 - 2a}
\end{align}
\end{frame}
\begin{frame}{Theoretical Solution }
For the rank to drop,
\begin{align}
6 - 2a &= 0 \\
a &= 3
\end{align}

When $a=3$,
\[
\myvec{1 & 3 \\ 0 & 0}
\]
is the reduced row-echelon form (rank $=1$)




The given points are collinear when

\begin{align}
a &= 3
\end{align}

\end{frame}


\begin{frame}[fragile]
    \frametitle{C Code}
    \begin{lstlisting}

#include <stdbool.h>

bool is_collinear(int a) {
    int det = 6 - 2*a;  // determinant
    return (det == 0);
}




\end{lstlisting}
 
\end{frame}
\begin{frame}[fragile]
    \frametitle{C plus Python code}
    \begin{lstlisting}

import ctypes
import matplotlib.pyplot as plt

# Load the shared library
lib = ctypes.CDLL("./collinear.so")
lib.is_collinear.argtypes = [ctypes.c_int]
lib.is_collinear.restype = ctypes.c_bool

# Points
O = (0, 0)
A = (1, 2)
a = 3   # try changing this value
C = (a, 6)

# Check collinearity using C function
print("Collinear?", lib.is_collinear(a))
\end{lstlisting}
 
\end{frame}
\begin{frame}[fragile]
    \frametitle{C plus Python code}
    \begin{lstlisting}

# Plot points
plt.figure(figsize=(6,6))
plt.scatter(*O, color='black', label="O(0,0)")
plt.scatter(*A, color='red', label="A(1,2)")
plt.scatter(*C, color='blue', label=f"C({a},6)")

# If collinear, draw line through O, A, C
if lib.is_collinear(a):
    plt.plot([O[0], A[0], C[0]], [O[1], A[1], C[1]], 'g--', label="Collinear line")
else:
    # If not collinear, just connect O-A and O-C separately
    plt.plot([O[0], A[0]], [O[1], A[1]], 'r--')
    plt.plot([O[0], C[0]], [O[1], C[1]], 'b--')
\end{lstlisting}
 
\end{frame}
\begin{frame}[fragile]
    \frametitle{C plus Python code}
    \begin{lstlisting}

# Formatting
plt.axhline(0, color='black', linewidth=0.5)
plt.axvline(0, color='black', linewidth=0.5)
plt.grid(True, linestyle="--", alpha=0.6)
plt.legend()
plt.xlabel("x-axis")
plt.ylabel("y-axis")
plt.title("Collinearity Check of A, O, and C")
pltsave.fig("/sdcard/Matrix/ee1030-2025/ai25btech11016/Matgeo/1.2.24/figs/1.7.2.png")
plt.show()
\end{lstlisting}
 
\end{frame}
\begin{frame}[fragile]
    \frametitle{Python plot code}
    \begin{lstlisting}



import numpy as np
import matplotlib.pyplot as plt

# Points
O = np.array([0, 0])
A = np.array([1, 2])
C = np.array([3, 6])  # since a = 3

# Plot the points
plt.figure(figsize=(6,6))
plt.scatter(*O, color='black', label='O(0,0)')
plt.scatter(*A, color='red', label='A(1,2)')
plt.scatter(*C, color='blue', label='C(3,6)')
\end{lstlisting}
\end{frame}
\begin{frame}[fragile]
    \frametitle{Python plot code}
    \begin{lstlisting}

# Draw lines between them
plt.plot([O[0], A[0], C[0]], [O[1], A[1], C[1]], 'g--', label='Collinear line')

# Labels and formatting
plt.axhline(0, color='black', linewidth=0.5)
plt.axvline(0, color='black', linewidth=0.5)
plt.grid(True, linestyle='--', alpha=0.6)
plt.legend()
plt.xlabel("x-axis")
plt.ylabel("y-axis")

plt.savefig("/sdcard/Matrix/ee1030-2025/ai25btech11016/Matgeo/1.2.24/figs/1.7.2.png"")
plt.show()
\end{lstlisting}
\end{frame}
\begin{frame}[fragile]
    \frametitle{Plot}
    


\begin{figure}[H]
    \centering
    \includegraphics[scale=0.4]{figs/1.7.2.jpg}
    \caption{}
    \label{fig:1}
\end{figure}
    

 \end{frame}
\end{document}