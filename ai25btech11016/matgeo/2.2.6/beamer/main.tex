\documentclass{beamer}
\usepackage[utf8]{inputenc}
\usetheme{Madrid}
\usecolortheme{default}
\usepackage{amsmath,amssymb,amsfonts,amsthm}
\usepackage{txfonts}
\usepackage{tkz-euclide}
\usepackage{listings}
\usepackage{adjustbox}
\usepackage{array}
\usepackage{tabularx}
\usepackage{gvv}
\usepackage{lmodern}
\usepackage{circuitikz}
\usepackage{tikz}
\usepackage{graphicx}

\setbeamertemplate{page number in head/foot}[totalframenumber]

\usepackage{tcolorbox}
\tcbuselibrary{minted,breakable,xparse,skins}



\definecolor{bg}{gray}{0.95}
\DeclareTCBListing{mintedbox}{O{}m!O{}}{%
  breakable=true,
  listing engine=minted,
  listing only,
  minted language=#2,
  minted style=default,
  minted options={%
    linenos,
    gobble=0,
    breaklines=true,
    breakafter=,,
    fontsize=\small,
    numbersep=8pt,
    #1},
  boxsep=0pt,
  left skip=0pt,
  right skip=0pt,
  left=25pt,
  right=0pt,
  top=3pt,
  bottom=3pt,
  arc=5pt,
  leftrule=0pt,
  rightrule=0pt,
  bottomrule=2pt,
  toprule=2pt,
  colback=bg,
  colframe=orange!70,
  enhanced,
  overlay={%
    \begin{tcbclipinterior}
    \fill[orange!20!white] (frame.south west) rectangle ([xshift=20pt]frame.north west);
    \end{tcbclipinterior}},
  #3,
}
\lstset{
    language=C,
    basicstyle=\ttfamily\small,
    keywordstyle=\color{blue},
    stringstyle=\color{orange},
    commentstyle=\color{green!60!black},
    numbers=left,
    numberstyle=\tiny\color{gray},
    breaklines=true,
    showstringspaces=false,
}
%------------------------------------------------------------
%This block of code defines the information to appear in the
%Title page
\title %optional
{2.2.6}

%\subtitle{A short story}

\author % (optional)
{Varun-ai25btech11016}



\begin{document}


\frame{\titlepage}
\begin{frame}{Question}
  Find the angle between the vectors 
$$
\vec{a} = 2\hat{i} - \hat{j} + \hat{k}, \quad \vec{b} = 3\hat{i} + 4\hat{j} - \hat{k}
$$

\end{frame}



\begin{frame}{Theoretical Solution }

 \begin{align}
\vec{a} &= \myvec{2 \\ -1 \\ 1} \\
\vec{b} &= \myvec{3 \\ 4 \\ -1}
\end{align}

From the formula,
\begin{align}
\cos \theta &= \frac{\vec{a}^{T} \vec{b}}{\lVert \vec{a} \rVert \, \lVert \vec{b} \rVert}
\end{align}
\end{frame}
\begin{frame}{Theoretical Solution }
Substituting,
\begin{align}
\cos \theta &= \frac{1}{\sqrt{6}\,\sqrt{26}} \\
&= \frac{1}{\sqrt{156}} \nonumber
\end{align}

Therefore,
\begin{align}
\theta &= \cos^{-1}\left(\frac{1}{\sqrt{156}}\right)
\end{align}
The angle between the given two vectors is $\cos^{-1}\left(\frac{1}{\sqrt{156}}\right)$
\end{frame}

\begin{frame}[fragile]
    \frametitle{C code}
    \begin{lstlisting}

#include <math.h>

// Function to compute angle (in radians) between two 3D vectors
double angle_between(double ax, double ay, double az,
                     double bx, double by, double bz) {
    double dot = ax*bx + ay*by + az*bz;

    double mag_a = sqrt(ax*ax + ay*ay + az*az);
    double mag_b = sqrt(bx*bx + by*by + bz*bz);

    if (mag_a == 0.0 || mag_b == 0.0) {
        return -1.0;  // invalid
    }
\end{lstlisting} 
\end{frame}
\begin{frame}[fragile]
    \frametitle{C code}
    \begin{lstlisting}
    double cos_theta = dot / (mag_a * mag_b);
    if (cos_theta > 1.0) cos_theta = 1.0;
    if (cos_theta < -1.0) cos_theta = -1.0;

    return acos(cos_theta);
}

\end{lstlisting} 
\end{frame}
\begin{frame}[fragile]
    \frametitle{C plus Python code}
    \begin{lstlisting}

import ctypes
import numpy as np
import matplotlib.pyplot as plt

# Load .so
lib = ctypes.CDLL("2.2.6fuction.so")
lib.angle_between.argtypes = [ctypes.c_double, ctypes.c_double, ctypes.c_double,
                              ctypes.c_double, ctypes.c_double, ctypes.c_double]
lib.angle_between.restype = ctypes.c_double

# Original 3D vectors
a = np.array([2, -1, 1])
b = np.array([3, 4, -1])
\end{lstlisting}
 
\end{frame}
\begin{frame}[fragile]
    \frametitle{C plus Python code}
    \begin{lstlisting}
# Call C function
theta = lib.angle_between(*a, *b)
print("Angle (radians):", theta)
print("Angle (degrees):", np.degrees(theta))

# ---- Project to 2D plane ----
# Orthonormal basis from vector a
u = a / np.linalg.norm(a)             # first basis vector
v = b - np.dot(b, u) * u              # make b orthogonal to u
v = v / np.linalg.norm(v)             # second basis vector

# Coordinates of a and b in this 2D plane
a2d = np.array([np.dot(a, u), np.dot(a, v)])
b2d = np.array([np.dot(b, u), np.dot(b, v)])
\end{lstlisting}
 
\end{frame}
\begin{frame}[fragile]
    \frametitle{C plus Python code}
    \begin{lstlisting}
# ---- Plot in 2D ----
plt.figure(figsize=(6,6))
plt.axhline(0, color='gray', lw=0.5)
plt.axvline(0, color='gray', lw=0.5)

plt.quiver(0, 0, a2d[0], a2d[1], angles='xy', scale_units='xy', scale=1, color="r", label="a = (2,-1,1)")
plt.quiver(0, 0, b2d[0], b2d[1], angles='xy', scale_units='xy', scale=1, color="b", label="b = (3,4,-1)")

plt.xlim(-5, 5)
plt.ylim(-5, 5)
plt.gca().set_aspect("equal")

plt.legend()
plt.savefig("/sdcard/Matrix/ee1030-2025/ai25btech11016/Matgeo/2.2.6/figs/2.2.6.png")
plt.show()
\end{lstlisting}
 
\end{frame}
\begin{frame}[fragile]
    \frametitle{Python plot code}
    \begin{lstlisting}

import numpy as np
import matplotlib.pyplot as plt

# Vectors in 3D
a = np.array([2, -1, 1])
b = np.array([3, 4, -1])

# Normalize a → treat as new x-axis
u = a / np.linalg.norm(a)

# Remove component of b along u → gives orthogonal direction in plane
b_proj = b - np.dot(b, u) * u
v = b_proj / np.linalg.norm(b_proj)  # normalize → new y-axis
\end{lstlisting}
 
\end{frame}
\begin{frame}[fragile]
    \frametitle{Python plot code}
    \begin{lstlisting}
# 2D coordinates in this new basis
a_2d = np.array([np.dot(a, u), np.dot(a, v)])
b_2d = np.array([np.dot(b, u), np.dot(b, v)])

# Plot in 2D
plt.figure(figsize=(6,6))
plt.axhline(0, color='black', linewidth=0.5)
plt.axvline(0, color='black', linewidth=0.5)

plt.quiver(0, 0, a_2d[0], a_2d[1], angles='xy', scale_units='xy', scale=1, color='r', label="a = (2,-1,1)")
plt.quiver(0, 0, b_2d[0], b_2d[1], angles='xy', scale_units='xy', scale=1, color='b', label="b = (3,4,-1)")
\end{lstlisting}
 
\end{frame}
\begin{frame}[fragile]
    \frametitle{Python plot code}
    \begin{lstlisting}
plt.legend()
plt.grid(True, linestyle="--", alpha=0.6)
plt.gca().set_aspect('equal', adjustable='box')
plt.savefig("/sdcard/Matrix/ee1030-2025/ai25btech11016/Matgeo/2.2.6/figs/2.2.6.png")
plt.show()

\end{lstlisting}
 
\end{frame}
\begin{frame}[fragile]
    \frametitle{Plot}
    
\begin{figure}[h]
    \centering
    \includegraphics[scale=0.5]{figs/2.2.6.jpg}
    \caption{}
    \label{fig:1}
\end{figure}


    
\end{frame}

\end{document}