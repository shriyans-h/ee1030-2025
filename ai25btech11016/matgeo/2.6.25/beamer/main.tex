\documentclass{beamer}
\usepackage[utf8]{inputenc}
\usetheme{Madrid}
\usecolortheme{default}
\usepackage{amsmath,amssymb,amsfonts,amsthm}
\usepackage{txfonts}
\usepackage{tkz-euclide}
\usepackage{listings}
\usepackage{adjustbox}
\usepackage{array}
\usepackage{tabularx}
\usepackage{gvv}
\usepackage{lmodern}
\usepackage{circuitikz}
\usepackage{tikz}
\usepackage{graphicx}

\setbeamertemplate{page number in head/foot}[totalframenumber]

\usepackage{tcolorbox}
\tcbuselibrary{minted,breakable,xparse,skins}



\definecolor{bg}{gray}{0.95}
\DeclareTCBListing{mintedbox}{O{}m!O{}}{%
  breakable=true,
  listing engine=minted,
  listing only,
  minted language=#2,
  minted style=default,
  minted options={%
    linenos,
    gobble=0,
    breaklines=true,
    breakafter=,,
    fontsize=\small,
    numbersep=8pt,
    #1},
  boxsep=0pt,
  left skip=0pt,
  right skip=0pt,
  left=25pt,
  right=0pt,
  top=3pt,
  bottom=3pt,
  arc=5pt,
  leftrule=0pt,
  rightrule=0pt,
  bottomrule=2pt,
  toprule=2pt,
  colback=bg,
  colframe=orange!70,
  enhanced,
  overlay={%
    \begin{tcbclipinterior}
    \fill[orange!20!white] (frame.south west) rectangle ([xshift=20pt]frame.north west);
    \end{tcbclipinterior}},
  #3,
}
\lstset{
    language=C,
    basicstyle=\ttfamily\small,
    keywordstyle=\color{blue},
    stringstyle=\color{orange},
    commentstyle=\color{green!60!black},
    numbers=left,
    numberstyle=\tiny\color{gray},
    breaklines=true,
    showstringspaces=false,
}
%------------------------------------------------------------
%This block of code defines the information to appear in the
%Title page
\title %optional
{2.6.25}

%\subtitle{A short story}

\author % (optional)
{Varun-ai25btech11016}



\begin{document}


\frame{\titlepage}
\begin{frame}{Question}
Find the area of a triangle formed by the points  $A(5,2)$, $B(4,7)$ and $C(7,-4)$

\end{frame}



\begin{frame}{Theoretical Solution }

\begin{align}
\vec{B} - \vec{A} &= \myvec{-1 \\ 5}
\end{align}
\begin{align}
\vec{C} - \vec{A} & = \myvec{2 \\ -6} 
\end{align}

\begin{align}
\norm{\vec{(B-A)} \times \vec{(C-A)}} = \norm{\,\myvec{|\vec{A_{23}} & \vec{B_{23}}| \\ |\vec{A_{31}} & \vec{B_{31}}| \\ |\vec{A_{12}} & \vec{B_{12}}|}\,} 
&=4 \nonumber\\\
\end{align} 
\begin{align}
\text{Area of the triangle ABC} &= \tfrac{1}{2} 
\norm{(\vec{B} - \vec{A}) \times (\vec{C} - \vec{A}) } \\
&=2 
\end{align}
\end{frame}
\begin{frame}{Conclusion}
\textbf{Therefore,
}
\begin{center}
    The area of triangle ABC is 2
\end{center}
\end{frame}
\begin{frame}[fragile]
    \frametitle{C Code}
\begin{lstlisting}

#include <math.h>

// Function to compute triangle area using cross product
double triangle_area(double x1, double y1, double x2, double y2, double x3, double y3) {
    // Vectors B-A and C-A
    double ux = x2 - x1;
    double uy = y2 - y1;
    double vx = x3 - x1;
    double vy = y3 - y1;

    // Cross product (2D)
    double cross = ux * vy - uy * vx;

    // Area is half magnitude of cross product
    return 0.5 * fabs(cross);}

\end{lstlisting}
\end{frame}
\begin{frame}[fragile]
    \frametitle{C plus Python code}
    \begin{lstlisting}
import ctypes
import matplotlib.pyplot as plt

# Load the shared library
lib = ctypes.CDLL("./triangle_area.so")

# Define the argument and return types
lib.triangle_area.argtypes = [ctypes.c_double, ctypes.c_double,
                              ctypes.c_double, ctypes.c_double,
                              ctypes.c_double, ctypes.c_double]
lib.triangle_area.restype = ctypes.c_double

# Triangle vertices
A = (5, 2)
B = (4, 7)
C = (7, -4)
\end{lstlisting}
\end{frame}
\begin{frame}[fragile]
    \frametitle{C plus Python code}
    \begin{lstlisting}
# Call the C function
area = lib.triangle_area(A[0], A[1], B[0], B[1], C[0], C[1])
print("Area of triangle ABC =", area)

# ---- Plotting ----
x_vals = [A[0], B[0], C[0], A[0]]
y_vals = [A[1], B[1], C[1], A[1]]

plt.plot(x_vals, y_vals, 'b-', linewidth=2, label="Triangle ABC")
plt.scatter([A[0], B[0], C[0]], [A[1], B[1], C[1]], color='red')

# Annotate points
plt.text(A[0], A[1], " A"+str(A))
plt.text(B[0], B[1], " B"+str(B))
plt.text(C[0], C[1], " C"+str(C))
\end{lstlisting}
\end{frame}
\begin{frame}[fragile]
    \frametitle{C plus Python code}
    \begin{lstlisting}
plt.xlabel("X-axis")
plt.ylabel("Y-axis")
plt.title(f"Triangle ABC (Area = {area})")
plt.grid(True)
plt.axis("equal")
plt.legend()
plt.savefig("/sdcard/Matrix/ee1030-2025/ai25btech11016/Matgeo/2.6.25/figs/2.6.25.png")
plt.show()
\end{lstlisting}
\end{frame}
\begin{frame}[fragile]
    \frametitle{Python code}
    \begin{lstlisting}


import numpy as np
import matplotlib.pyplot as plt

# Given points
A = np.array([5, 2])
B = np.array([4, 7])
C = np.array([7, -4])

# Compute area using determinant formula
area = 0.5 * abs(A[0]*(B[1]-C[1]) + B[0]*(C[1]-A[1]) + C[0]*(A[1]-B[1]))
print("Area of triangle:", area)
\end{lstlisting}
\end{frame}
\begin{frame}[fragile]
    \frametitle{Python code}
    \begin{lstlisting}

# Plot triangle
plt.figure(figsize=(6,6))
plt.plot([A[0], B[0], C[0], A[0]], [A[1], B[1], C[1], A[1]], 'b-', linewidth=2)
plt.fill([A[0], B[0], C[0]], [A[1], B[1], C[1]], color='skyblue', alpha=0.4)

# Mark points
plt.scatter(*A, color='r')
plt.text(A[0]+0.2, A[1], "A(5,2)", fontsize=10)
plt.scatter(*B, color='r')
plt.text(B[0]+0.2, B[1], "B(4,7)", fontsize=10)
plt.scatter(*C, color='r')
plt.text(C[0]+0.2, C[1], "C(7,-4)", fontsize=10)
\end{lstlisting}
\end{frame}
\begin{frame}[fragile]
    \frametitle{Python code}
    \begin{lstlisting}

# Axis setup
plt.axhline(0, color='gray', linewidth=0.5)
plt.axvline(0, color='gray', linewidth=0.5)
plt.gca().set_aspect("equal")
plt.title(f"Triangle ABC, Area = {area}")
plt.grid(True)
plt.savefig("/sdcard/Matrix/ee1030-2025/ai25btech11016/Matgeo/2.6.25/figs/2.6.25.png")
plt.show()
\end{lstlisting}
\end{frame}
\begin{frame}[fragile]
    \frametitle{Plot}
   
\begin{figure}[h]
    \centering
    \includegraphics[scale=0.5]{figs/2.6.25.jpg}
    \caption{}
    \label{fig:1}
\end{figure}

    
\end{frame}
\end{document}