\let\negmedspace\undefined
\let\negthickspace\undefined
\documentclass[journal]{IEEEtran}
\usepackage[a5paper, margin=10mm, onecolumn]{geometry}
%\usepackage{lmodern} % Ensure lmodern is loaded for pdflatex
\usepackage{tfrupee} % Include tfrupee package

\setlength{\headheight}{1cm} % Set the height of the header box
\setlength{\headsep}{0mm}     % Set the distance between the header box and the top of the text

\usepackage{gvv-book}
\usepackage{gvv}
\usepackage{cite}
\usepackage{amsmath,amssymb,amsfonts,amsthm}
\usepackage{algorithmic}
\usepackage{graphicx}
\usepackage{textcomp}
\usepackage{xcolor}
\usepackage{txfonts}
\usepackage{listings}
\usepackage{enumitem}
\usepackage{mathtools}
\usepackage{gensymb}
\usepackage{comment}
\usepackage[breaklinks=true]{hyperref}
\usepackage{tkz-euclide} 
\usepackage{listings}
% \usepackage{gvv}                                        
\def\inputGnumericTable{}                                 
\usepackage[latin1]{inputenc}                                
\usepackage{color}                                            
\usepackage{array}                                            
\usepackage{longtable}                                       
\usepackage{calc}                                             
\usepackage{multirow}                                         
\usepackage{hhline}                                           
\usepackage{ifthen}                                           
\usepackage{lscape}
\usepackage{circuitikz}
\tikzstyle{block} = [rectangle, draw, fill=blue!20, 
    text width=4em, text centered, rounded corners, minimum height=3em]
\tikzstyle{sum} = [draw, fill=blue!10, circle, minimum size=1cm, node distance=1.5cm]
\tikzstyle{input} = [coordinate]
\tikzstyle{output} = [coordinate]


\begin{document}

\bibliographystyle{IEEEtran}
\vspace{3cm}

\title{2.6.25}
\author{AI25BTECH11016-Varun}
 \maketitle
% \newpage
% \bigskip
{\let\newpage\relax\maketitle}
\renewcommand{\thefigure}{\theenumi}
\renewcommand{\thetable}{\theenumi}
\setlength{\intextsep}{10pt} % Space between text and floats
\numberwithin{figure}{enumi}
\renewcommand{\thetable}{\theenumi}
\textbf{Question:}\\

Find the area of a triangle formed by the points  $A(5,2)$, $B(4,7)$ and $C(7,-4)$



\textbf{Solution:}

\begin{align}
\vec{B} - \vec{A} &= \myvec{-1 \\ 5}
\end{align}
\begin{align}
\vec{C} - \vec{A} & = \myvec{2 \\ -6} 
\end{align}

\begin{align}
\norm{\vec{(B-A)} \times \vec{(C-A)}} = \norm{\,\myvec{|\vec{A_{23}} & \vec{B_{23}}| \\ |\vec{A_{31}} & \vec{B_{31}}| \\ |\vec{A_{12}} & \vec{B_{12}}|}\,} 
&=4 \nonumber\\\
\end{align} 
\begin{align}
\text{Area of the triangle ABC} &= \tfrac{1}{2} 
\norm{(\vec{B} - \vec{A}) \times (\vec{C} - \vec{A}) } \\
&=2 
\end{align}
\textbf{Therefore,
}
\begin{center}
    The area of triangle ABC is 2
\end{center}


\begin{figure}[h]
    \centering
    \includegraphics[scale=0.5]{figs/2.6.25.jpg}
    \caption{}
    \label{fig:1}
\end{figure}

\end{document}