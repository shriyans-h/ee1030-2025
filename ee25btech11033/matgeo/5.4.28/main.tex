\let\negmedspace\undefined
\let\negthickspace\undefined
\documentclass[journal]{IEEEtran}
\usepackage[a5paper, margin=10mm, onecolumn]{geometry}
%\usepackage{lmodern} % Ensure lmodern is loaded for pdflatex
\usepackage{tfrupee} % Include tfrupee package

\setlength{\headheight}{1cm} % Set the height of the header box
\setlength{\headsep}{0mm}     % Set the distance between the header box and the top of the text

\usepackage{gvv-book}
\usepackage{gvv}
\usepackage{cite}
\usepackage{amsmath,amssymb,amsfonts,amsthm}
\usepackage{algorithmic}
\usepackage{graphicx}
\usepackage{textcomp}
\usepackage{xcolor}
\usepackage{txfonts}
\usepackage{listings}
\usepackage{enumitem}
\usepackage{mathtools}
\usepackage{gensymb}
\usepackage{comment}
\usepackage[breaklinks=true]{hyperref}
\usepackage{tkz-euclide} 
\usepackage{listings}
% \usepackage{gvv}                                        
\def\inputGnumericTable{}                                 
\usepackage[latin1]{inputenc}                                
\usepackage{color}                                            
\usepackage{array}                                            
\usepackage{longtable}                                       
\usepackage{calc}                                             
\usepackage{multirow}                                         
\usepackage{hhline}                                           
\usepackage{ifthen}                                           
\usepackage{lscape}
\begin{document}

\bibliographystyle{IEEEtran}
\vspace{3cm}

\title{5.4.28}
\author{EE25BTECH11033 - Kavin}
% \maketitle
% \newpage
% \bigskip
{\let\newpage\relax\maketitle}

\renewcommand{\thefigure}{\theenumi}
\renewcommand{\thetable}{\theenumi}
\setlength{\intextsep}{10pt} % Space between text and floats
\textbf{Question}:\\
Using elementary transformations, find the inverse of the following matrix.
\begin{align*}
    \myvec{2 & 4\\-5 & -1}
\end{align*}
\bigskip


\textbf{Solution}:\\
Given the matrix,
\begin{align}
    \vec{A} = \myvec{2 & 4\\-5 & -1}
\end{align}
Let $\vec{A}^{-1}$ be the inverse of the matrix $\vec{A}$.\\
\\
We know that ,
\begin{align}
    \vec{A}\vec{A}^{-1} = \vec{I}
\end{align}
The augmented matrix of \augvec{1}{1}{\vec{A} & \vec{I}} is given by,
\begin{align}
    \augvec{2}{2}{2 & 4 & 1 & 0\\-5 & -1 & 0 & 1}
\end{align}
\begin{align}
    R_1\rightarrow\frac{1}{2}R_1\implies\augvec{2}{2}{1 & 2 & 1/2 & 0\\-5 & -1 & 0 & 1}
\end{align}
\begin{align}
    R_2 \rightarrow R_2 + 5R_1\implies\augvec{2}{2}{1 & 2 & 1/2 & 0\\0 & 9 & 5/2 & 1}
\end{align}
\begin{align}
    R_2 \rightarrow \frac{1}{9}R_2\implies\augvec{2}{2}{1 & 2 & 1/2 & 0\\0 & 1 & 5/18 & 1/9}
\end{align}
\begin{align}
    R_1 \rightarrow R_1 - 2R_2\implies\augvec{2}{2}{1 & 0 & -1/18 & -2/9\\0 & 1 & 5/18 & 1/9}
\end{align}\\
\bigskip

\begin{align}
    \implies \vec{A}^{-1} = \myvec{-1/18 & -2/9\\5/18 & 1/9}
\end{align}
\end{document}


