\let\negmedspace\undefined
\let\negthickspace\undefined
\documentclass[journal]{IEEEtran}
\usepackage[a5paper, margin=10mm, onecolumn]{geometry}
%\usepackage{lmodern} % Ensure lmodern is loaded for pdflatex
\usepackage{tfrupee} % Include tfrupee package

\setlength{\headheight}{1cm} % Set the height of the header box
\setlength{\headsep}{0mm}     % Set the distance between the header box and the top of the text

\usepackage{gvv-book}
\usepackage{gvv}
\usepackage{cite}
\usepackage{amsmath,amssymb,amsfonts,amsthm}
\usepackage{algorithmic}
\usepackage{graphicx}
\usepackage{textcomp}
\usepackage{xcolor}
\usepackage{txfonts}
\usepackage{listings}
\usepackage{enumitem}
\usepackage{mathtools}
\usepackage{gensymb}
\usepackage{comment}
\usepackage[breaklinks=true]{hyperref}
\usepackage{tkz-euclide} 
\usepackage{listings}
% \usepackage{gvv}                                        
\def\inputGnumericTable{}                                 
\usepackage[latin1]{inputenc}                                
\usepackage{color}                                            
\usepackage{array}                                            
\usepackage{longtable}                                       
\usepackage{calc}                                             
\usepackage{multirow}                                         
\usepackage{hhline}                                           
\usepackage{ifthen}                                           
\usepackage{lscape}
\begin{document}

\bibliographystyle{IEEEtran}
\vspace{3cm}

\title{4.11.11}
\author{EE25BTECH11033 - Kavin}
% \maketitle
% \newpage
% \bigskip
{\let\newpage\relax\maketitle}

\renewcommand{\thefigure}{\theenumi}
\renewcommand{\thetable}{\theenumi}
\setlength{\intextsep}{10pt} % Space between text and floats
\textbf{Question}:\\
Find the ratio in which the line $x - 3y = 0$ divides the line segment joining the points $(-2, -5)$ and $(6, 3)$. Find the coordinates of the point of intersection.\\
\bigskip


\textbf{Solution}:\\
Given the points,
\begin{align}
    \vec{A}=\begin{myvec}{-2\\-5}\end{myvec}\ \ 
    \vec{B}=\begin{myvec}{6\\3}\end{myvec}
\end{align}
and the line $L_1$,
\begin{align}
    L_1: \myvec{1 & -3}\vec{x} = 0
\end{align}
\begin{align}
    \implies \vec{n}^{\top}\vec{x}=0
\end{align}
\bigskip

Let the vector $\vec{P}$ be a point on the line $x - 3y = 0$ wihch divides the line segment joining the points $\vec{A}$ and $\vec{B}$.
\\
Section formula for a vector $\vec{P}$ which divides the line formed by vectors $\vec{A}$ and $\vec{B}$ in the ratio $k:1$ is given by
\begin{align}
    \vec{P}=\frac{k\vec{B}+\vec{A}}{k+1}
\end{align}
\begin{align}
    \vec{P}=\myvec{\vec{A} & \vec{B}}\myvec{\frac{1}{k+1}\\\frac{k}{k+1}}
\end{align}\\
\bigskip

Since $\vec{P}$ lies on line $L_1$,
\begin{align}
    \vec{n}^{\top}\vec{P}=0
\end{align}
\begin{align}
    \implies \myvec{1 & -3}\myvec{\vec{A} & \vec{B}}\myvec{\frac{1}{k+1}\\\frac{k}{k+1}}=0
\end{align}
\begin{align}
    \implies \myvec{1 & -3}\myvec{-2 & 6\\-5 & 3}\myvec{\frac{1}{k+1}\\\frac{k}{k+1}}=0
\end{align}
\begin{align}
    \implies \myvec{13 & -3}\myvec{\frac{1}{k+1}\\\frac{k}{k+1}}=0
\end{align}
\begin{align}
    \implies \frac{13-3k}{k+1} = 0
\end{align}
\begin{align}
    \implies k=\frac{13}{3}
\end{align}
Therefore the ratio in which $\vec{P}$ divides the line segment joining the points $\vec{A}$ and $\vec{B}$ is $13:3$\\
On substituting the value of $k$ in equation $\brak{5}$ we will get,
\begin{align}
    \vec{P}=\myvec{-2 & 6\\-5 & 3}\myvec{3/16 \\ 13/16}
\end{align}
\begin{align}
    \implies\vec{P}=\begin{myvec}{9/2\\3/2}\end{myvec}
\end{align}\\
\bigskip

\begin{figure}[H]
\begin{center}
\includegraphics[width=0.6\columnwidth]{figs/fig.png}
\end{center}
\label{fig:Fig1}
\end{figure}
\end{document}


