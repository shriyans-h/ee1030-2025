\let\negmedspace\undefined
\let\negthickspace\undefined
\documentclass[journal]{IEEEtran}
\usepackage[a5paper, margin=10mm, onecolumn]{geometry}

\usepackage{tfrupee} 

\setlength{\headheight}{1cm} 
\setlength{\headsep}{0mm}     

\usepackage{gvv-book}
\usepackage{gvv}
\usepackage{cite}
\usepackage{amsmath,amssymb,amsfonts,amsthm}
\usepackage{algorithmic}
\usepackage{graphicx}
\usepackage{textcomp}
\usepackage{xcolor}
\usepackage{txfonts}
\usepackage{listings}
\usepackage{enumitem}
\usepackage{mathtools}
\usepackage{gensymb}
\usepackage{comment}
\usepackage[breaklinks=true]{hyperref}
\usepackage{tkz-euclide} 
\usepackage{listings}
% \usepackage{gvv}                                        
\def\inputGnumericTable{}                                 
\usepackage[latin1]{inputenc}                                
\usepackage{color}                                            
\usepackage{array}                                            
\usepackage{longtable}                                       
\usepackage{calc}                                             
\usepackage{multirow}                                         
\usepackage{hhline}                                           
\usepackage{ifthen}                                           
\usepackage{lscape}
\begin{document}

\bibliographystyle{IEEEtran}
\vspace{3cm}

\title{4.7.25}
\author{EE25BTECH11033 - Kavin}
% \maketitle
% \newpage
% \bigskip
{\let\newpage\relax\maketitle}

\renewcommand{\thefigure}{\theenumi}
\renewcommand{\thetable}{\theenumi}
\setlength{\intextsep}{10pt} % Space between text and floats
\textbf{Question}:\\
Find the points on the line $x+y=4$ which lie at a unit distance from the line $4x+3y=10$.\\
\bigskip


\textbf{Solution}:\\
According to the question,\\
\begin{align}
    \text{Equation of line $L_1$:}\ \myvec{1&1}\myvec{x\\y}=4
\end{align}
and
\begin{align}
    \text{Equation of line $L_2$:}\ \myvec{4&3}\myvec{x\\y}=10
\end{align}
Any point $\vec{P}$ on line $L_1$ is given by ,
\begin{align}
    \vec{P} = \myvec{k\\4-k} 
\end{align}
The distance $\lambda$ of a vector $\vec{P}$ from the line $\vec{n}^{\top}\vec{x}=c$ is given by ,
\begin{align}
    \lambda = \frac{\abs{\vec{n}^{\top}\vec{P} - c}}{\norm{\vec{n}}}  
\end{align}
where,
\begin{align*}
    \vec{n}^\top = \myvec{4&3}\ ,c = 10\ \text{and}\ \lambda=1
\end{align*}\\
\bigskip

\begin{align}
    \implies \lambda\norm{\vec{n}} = \abs{\vec{n}^{\top}\vec{P} - c}
\end{align}
Also,
\begin{align}
    \norm{\vec{n}}=\sqrt{\vec{n}^{\top}\vec{n}} = \sqrt{25} = 5
\end{align}
\begin{align}
    \vec{n}^{\top}\vec{P} = k + 12
\end{align}
\newpage
\begin{align}
    \implies 5 = \abs{k + 12 - 10}
\end{align}
\begin{align}
    \implies 5 = \abs{k + 2}
\end{align}
\begin{align}
    \implies k = 3, -7
\end{align}
Therefore the points on $L_1$ which lie at a unit distance from the line $L_2$ are ,
\begin{align*}
    \myvec{3\\1} \ \text{and} \ \myvec{-7\\11}
\end{align*}

\begin{figure}[H]
    \centering
    \includegraphics[width=0.8\columnwidth]{figs/fig.png}
    \label{fig:1}
\end{figure}
\end{document}


