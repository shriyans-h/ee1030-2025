\documentclass{beamer}
\usepackage[utf8]{inputenc}

\usetheme{Madrid}
\usecolortheme{default}
\usepackage{amsmath,amssymb,amsfonts,amsthm}
\usepackage{txfonts}
\usepackage{tkz-euclide}
\usepackage{listings}
\usepackage{adjustbox}
\usepackage{array}
\usepackage{tabularx}
\usepackage{gvv}
\usepackage{lmodern}
\usepackage{circuitikz}
\usepackage{tikz}
\usepackage{graphicx}

\setbeamertemplate{page number in head/foot}[totalframenumber]

\usepackage{tcolorbox}
\tcbuselibrary{minted,breakable,xparse,skins}



\definecolor{bg}{gray}{0.95}
\DeclareTCBListing{mintedbox}{O{}m!O{}}{%
  breakable=true,
  listing engine=minted,
  listing only,
  minted language=#2,
  minted style=default,
  minted options={%
    linenos,
    gobble=0,
    breaklines=true,
    breakafter=,,
    fontsize=\small,
    numbersep=8pt,
    #1},
  boxsep=0pt,
  left skip=0pt,
  right skip=0pt,
  left=25pt,
  right=0pt,
  top=3pt,
  bottom=3pt,
  arc=5pt,
  leftrule=0pt,
  rightrule=0pt,
  bottomrule=2pt,
  toprule=2pt,
  colback=bg,
  colframe=orange!70,
  enhanced,
  overlay={%
    \begin{tcbclipinterior}
    \fill[orange!20!white] (frame.south west) rectangle ([xshift=20pt]frame.north west);
    \end{tcbclipinterior}},
  #3,
}
\lstset{
    language=C,
    basicstyle=\ttfamily\small,
    keywordstyle=\color{blue},
    stringstyle=\color{orange},
    commentstyle=\color{green!60!black},
    numbers=left,
    numberstyle=\tiny\color{gray},
    breaklines=true,
    showstringspaces=false,
}
\begin{document}

\title 
{4.7.25}
\date{September 2,2025}


\author 
{Kavin B-EE25BTECH11033}






\frame{\titlepage}
\begin{frame}{Question}
Find the points on the line $x+y=4$ which lie at a unit distance from the line $4x+3y=10$.\\
\end{frame}



\begin{frame}{Theoretical Solution}

According to the question,\\
\begin{align}
    \text{Equation of line $L_1$:}\ \myvec{1&1}\myvec{x\\y}=4
\end{align}
and
\begin{align}
    \text{Equation of line $L_2$:}\ \myvec{4&3}\myvec{x\\y}=10
\end{align}
Any point $\vec{P}$ on line $L_1$ is given by ,
\begin{align}
    \vec{P} = \myvec{k\\4-k} 
\end{align}
\end{frame}

\begin{frame}{Formulae}
The distance $\lambda$ of a vector $\vec{P}$ from the line $\vec{n}^{\top}\vec{x}=c$ is given by ,
\begin{align}
    \lambda = \frac{\abs{\vec{n}^{\top}\vec{P} - c}}{\norm{\vec{n}}}  
\end{align}
\end{frame}

\begin{frame}{Theoretical Solution}

where,
\begin{align*}
    \vec{n}^\top = \myvec{4&3}\ ,c = 10\ \text{and}\ \lambda=1
\end{align*}\\
\bigskip

\begin{align}
    \implies \lambda\norm{\vec{n}} = \abs{\vec{n}^{\top}\vec{P} - c}
\end{align}
Also,
\begin{align}
    \norm{\vec{n}}=\sqrt{\vec{n}^{\top}\vec{n}} = \sqrt{25} = 5
\end{align}
\begin{align}
    \vec{n}^{\top}\vec{P} = k + 12
\end{align}
\end{frame}


\begin{frame}{Theoretical Solution}
\begin{align}
    \implies 5 = \abs{k + 12 - 10}
\end{align}
\begin{align}
    \implies 5 = \abs{k + 2}
\end{align}
\begin{align}
    \implies k = 3, -7
\end{align}
Therefore the points on $L_1$ which lie at a unit distance from the line $L_2$ are ,
\begin{align*}
    \myvec{3\\1} \ \text{and} \ \myvec{-7\\11}
\end{align*}

\end{frame}

\begin{frame}{Plot}
    \centering
    \includegraphics[width=\columnwidth, height=0.8\textheight, keepaspectratio]{figs/fig.png}     
\end{frame}


\end{document}
