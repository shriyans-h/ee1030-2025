\let\negmedspace\undefined
\let\negthickspace\undefined
\documentclass[journal]{IEEEtran}
\usepackage[a5paper, margin=10mm, onecolumn]{geometry}
%\usepackage{lmodern} % Ensure lmodern is loaded for pdflatex
\usepackage{tfrupee} % Include tfrupee package

\setlength{\headheight}{1cm} % Set the height of the header box
\setlength{\headsep}{0mm}     % Set the distance between the header box and the top of the text

\usepackage{gvv-book}
\usepackage{gvv}
\usepackage{cite}
\usepackage{amsmath,amssymb,amsfonts,amsthm}
\usepackage{algorithmic}
\usepackage{graphicx}
\usepackage{textcomp}
\usepackage{xcolor}
\usepackage{txfonts}
\usepackage{listings}
\usepackage{enumitem}
\usepackage{mathtools}
\usepackage{gensymb}
\usepackage{comment}
\usepackage[breaklinks=true]{hyperref}
\usepackage{tkz-euclide} 
\usepackage{listings}
% \usepackage{gvv}                                        
\def\inputGnumericTable{}                                 
\usepackage[latin1]{inputenc}                                
\usepackage{color}                                            
\usepackage{array}                                            
\usepackage{longtable}                                       
\usepackage{calc}                                             
\usepackage{multirow}                                         
\usepackage{hhline}                                           
\usepackage{ifthen}                                           
\usepackage{lscape}
\begin{document}

\bibliographystyle{IEEEtran}
\vspace{3cm}

\title{5.13.59}
\author{EE25BTECH11033 - Kavin}
% \maketitle
% \newpage
% \bigskip
{\let\newpage\relax\maketitle}

\renewcommand{\thefigure}{\theenumi}
\renewcommand{\thetable}{\theenumi}
\setlength{\intextsep}{10pt} % Space between text and floats
\textbf{Question}:\\
 Let $\vec{P} = \myvec{a_{ij}}$ be a $3 \times 3$ matrix and let $\vec{Q} =  \myvec{b_{ij}}$, where $b_{ij} = 2^{i+j}a_{ij}$ for $1 \le i,j \le 3$. If the determinant of $\vec{P}$ is 2, then the determinant of the matrix $\vec{Q}$ is 
\begin{enumerate}
\begin{multicols}{4}
\item $2^{10}$ \columnbreak
\item $2^{11}$ \columnbreak
\item $2^{12}$\columnbreak
\item $2^{13}$
\end{multicols}
\end{enumerate}
\bigskip

\textbf{Solution}:\\
Let the matrix $\vec{P}$ be ,
\begin{align}
    \vec{P} = \myvec{a_{11} & a_{12} & a_{13}\\a_{21} & a_{22} & a_{23}\\a_{31} & a_{32} & a_{33}}
\end{align}
also,
\begin{align}
    \mydet{\vec{P}} = \mydet{a_{ij}} = 2
\end{align}
and the matrix $\vec{Q}$ be ,
\begin{align}
    \vec{Q} = \myvec{b_{11} & b_{12} & b_{13}\\b_{21} & b_{22} & b_{23}\\b_{31} & b_{32} & b_{33}}
\end{align}
Given that,
\begin{align}
    b_{ij} = 2^{i+j}a_{ij}
\end{align}
The determinant of the matrix $\vec{Q}$ is given by:

\begin{align}
\mydet{\vec{Q}} = \mydet{b_{ij}} = \mydet{2^{i+j} a_{ij}}
\end{align}

Split the exponent using the property $2^{i+j} = 2^i \cdot 2^j$:

\begin{align}
\mydet{\vec{Q}} = \mydet{2^i \cdot 2^j \cdot a_{ij}}
\end{align}

First, for each row $i$ (from $i=1$ to $3$),   factor out the common term $2^i$:

\begin{align}
\mydet{\vec{Q}} = (2^1)(2^2)(2^3) \cdot \mydet{2^j a_{ij}}
\end{align}

The product of these factors is:

\begin{align}
\prod_{i=1}^{3} 2^i = 2^{\sum_{i=1}^{3} i} = 2^{\frac{3(3+1)}{2}} = 2^6
\end{align}

This simplifies the expression for the determinant to:

\begin{align}
\mydet{\vec{Q}} = 2^{6} \mydet{2^j a_{ij}}
\end{align}

Now, look at the remaining determinant, $\mydet{2^j a_{ij}}$. For each column $j$ (from $j=1$ to $3$), factor out the common term $2^j$:

\begin{align}
\mydet{2^j a_{ij}} = (2^1)(2^2)(2^3) \cdot \mydet{a_{ij}} = 2^{6} \mydet{\vec{P}}
\end{align}

Substituting this back into our expression for $\mydet{\vec{Q}}$:

\begin{align}
\mydet{\vec{Q}} = 2^{6} \cdot \brak{2^{6} \mydet{\vec{P}}}
\end{align}

\begin{align}
\mydet{\vec{Q}} = 2^{12} \mydet{\vec{P}}
\end{align}
\begin{align}
\implies\mydet{\vec{Q}} = 2^{12}\cdot2
\end{align}
\begin{align}
\implies\mydet{\vec{Q}} = 2^{13}
\end{align}
\end{document}


