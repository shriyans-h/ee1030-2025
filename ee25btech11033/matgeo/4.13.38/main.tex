
\let\negmedspace\undefined
\let\negthickspace\undefined
\documentclass[journal]{IEEEtran}
\usepackage[a5paper, margin=10mm, onecolumn]{geometry}
%\usepackage{lmodern} % Ensure lmodern is loaded for pdflatex
\usepackage{tfrupee} % Include tfrupee package

\setlength{\headheight}{1cm} % Set the height of the header box
\setlength{\headsep}{0mm}     % Set the distance between the header box and the top of the text

\usepackage{gvv-book}
\usepackage{gvv}
\usepackage{cite}
\usepackage{amsmath,amssymb,amsfonts,amsthm}
\usepackage{algorithmic}
\usepackage{graphicx}
\usepackage{textcomp}
\usepackage{xcolor}
\usepackage{txfonts}
\usepackage{listings}
\usepackage{enumitem}
\usepackage{mathtools}
\usepackage{gensymb}
\usepackage{comment}
\usepackage[breaklinks=true]{hyperref}
\usepackage{tkz-euclide} 
\usepackage{listings}
% \usepackage{gvv}                                        
\def\inputGnumericTable{}                                 
\usepackage[latin1]{inputenc}                                
\usepackage{color}                                            
\usepackage{array}                                            
\usepackage{longtable}                                       
\usepackage{calc}                                             
\usepackage{multirow}                                         
\usepackage{hhline}                                           
\usepackage{ifthen}                                           
\usepackage{lscape}
\begin{document}

\bibliographystyle{IEEEtran}
\vspace{3cm}

\title{4.13.38}
\author{EE25BTECH11033 - Kavin}
% \maketitle
% \newpage
% \bigskip
{\let\newpage\relax\maketitle}

\renewcommand{\thefigure}{\theenumi}
\renewcommand{\thetable}{\theenumi}
\setlength{\intextsep}{10pt} % Space between text and floats
\textbf{Question}:\\
Let $PS$ be the median of the triangle with vertices $\vec{P}\brak{2, 2}$, $\vec{Q}\brak{6, -1}$ and $\vec{R}\brak{7,3}$. The equation of the line passing through $\brak{1, -1}$ and parallel to $PS$ is 
\begin{multicols}{2}
\begin{enumerate}
\item $4x+7y+3=0$
\item $2x-9y-11=0$
\item $4x-7y-11=0$
\item $2x+9y+7=0$
\end{enumerate}
\end{multicols}
\bigskip


\textbf{Solution}:\\
Given the points,\\
\begin{align}
\vec{P}=\myvec{2\\2}\ \ \vec{Q}=\myvec{6\\-1}\ \ \vec{R}=\myvec{7\\3}
\end{align}
$\vec{S}$ is the midpoint of the line segment joining points $\vec{Q}$ and $\vec{R}$.\\
If $\vec{S}$ divides $QR$ in the ratio $k : 1$,
		\begin{align}
			\vec{S}= \frac{k\vec{R}+\vec{Q}}{k+1}
		\end{align}
where,
\begin{align}
    k=1
\end{align}
\begin{align}
    \vec{S}= \frac{\vec{R}+\vec{Q}}{2}
\end{align}
\begin{align}
    \implies\vec{S}= \myvec{13/2\\1}
\end{align}
As $\vec{P}$ and $\vec{S}$ are collinear,
\begin{align}
    \vec{n}^{\top}\vec{P}=c\\
    \vec{n}^{\top}\vec{S}=c
\end{align}
which can be expressed as 
\begin{align}
	\myvec{ \vec{P} & \vec{S}}^{\top}\vec{n} = c\myvec{1 \\ 1}
	\\
	\equiv \myvec{ \vec{P} & \vec{S}}^{\top}\vec{n} = \myvec{1 \\ 1}
\end{align}
\begin{align}
    \implies \myvec{2 & 2\\13/2 & 1}\vec{n} = \myvec{1 \\ 1}
\end{align}
\begin{align}
	\implies 
	\augvec{2}{1}{ 
	2 & 2 & 1
	\\  
	13/2 & 1 & 1
	}
     \xleftrightarrow[]{R_2 \rightarrow 2R_2}
	\augvec{2}{1}{ 
	2 & 2 & 1
	\\  
	13 & 2 & 2
	}
\end{align}
\begin{align}
     \xleftrightarrow[]{R_2 \rightarrow 2R_2-13R_1}
	\augvec{2}{1}{ 
	2 & 2 & 1 
	\\ 
	0 & -22 & -9 
	}
	\xleftrightarrow[]{R_1 \rightarrow 1/2R_1}
	\augvec{2}{1}{ 
	1 & 1 & 1 
	\\ 
	0 & -22 & -9 
	}
\end{align}
\begin{align}
        \xleftrightarrow[]{R_2 \rightarrow -1/22R_1}
	\augvec{2}{1}{ 
	1 & 1 & 1/2 
	\\ 
	0 & 1 & 9/22 
	}
        \xleftrightarrow[]{R_1 \rightarrow R_1-R_2}
	\augvec{2}{1}{ 
	1 & 0 & 1/11 
	\\ 
	0 & 1 & 9/22 
	}
\end{align}
\begin{align}
\implies n = \myvec{1/11\\9/22}
\end{align}
$\therefore$ The equation of the line passing through $\myvec{1\\-1}$ and parallel to $PS$ is given by
\begin{align}
    \vec{n}^{\top}\brak{\vec{x}-\myvec{1\\-1}}=0
\end{align}
\begin{align}
    \myvec{1/11 & 9/22}\myvec{x-1\\y+1}=0
\end{align}
\begin{align}
    \implies 2x+9y+7=0
\end{align}
\begin{figure}[H]
\begin{center}
\includegraphics[width=0.7\columnwidth]{figs/fig.png}
\end{center}
\end{figure}
\end{document}

