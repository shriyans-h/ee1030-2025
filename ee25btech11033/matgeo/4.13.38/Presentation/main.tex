\documentclass{beamer}
\usepackage[utf8]{inputenc}

\usetheme{Madrid}
\usecolortheme{default}
\usepackage{amsmath,amssymb,amsfonts,amsthm}
\usepackage{txfonts}
\usepackage{tkz-euclide}
\usepackage{listings}
\usepackage{adjustbox}
\usepackage{array}
\usepackage{tabularx}
\usepackage{gvv}
\usepackage{lmodern}
\usepackage{circuitikz}
\usepackage{tikz}
\usepackage{graphicx}

\setbeamertemplate{page number in head/foot}[totalframenumber]

\usepackage{tcolorbox}
\tcbuselibrary{minted,breakable,xparse,skins}



\definecolor{bg}{gray}{0.95}
\DeclareTCBListing{mintedbox}{O{}m!O{}}{%
  breakable=true,
  listing engine=minted,
  listing only,
  minted language=#2,
  minted style=default,
  minted options={%
    linenos,
    gobble=0,
    breaklines=true,
    breakafter=,,
    fontsize=\small,
    numbersep=8pt,
    #1},
  boxsep=0pt,
  left skip=0pt,
  right skip=0pt,
  left=25pt,
  right=0pt,
  top=3pt,
  bottom=3pt,
  arc=5pt,
  leftrule=0pt,
  rightrule=0pt,
  bottomrule=2pt,
  toprule=2pt,
  colback=bg,
  colframe=orange!70,
  enhanced,
  overlay={%
    \begin{tcbclipinterior}
    \fill[orange!20!white] (frame.south west) rectangle ([xshift=20pt]frame.north west);
    \end{tcbclipinterior}},
  #3,
}
\lstset{
    language=C,
    basicstyle=\ttfamily\small,
    keywordstyle=\color{blue},
    stringstyle=\color{orange},
    commentstyle=\color{green!60!black},
    numbers=left,
    numberstyle=\tiny\color{gray},
    breaklines=true,
    showstringspaces=false,
}
\begin{document}

\title 
{4.13.38}
\date{September 08,2025}


\author 
{Kavin B-EE25BTECH11033}






\frame{\titlepage}
\begin{frame}{Question}
Let $PS$ be the median of the triangle with vertices $\vec{P}\brak{2, 2}$, $\vec{Q}\brak{6, -1}$ and $\vec{R}\brak{7,3}$. The equation of the line passing through $\brak{1, -1}$ and parallel to $PS$ is 
\begin{multicols}{2}
\begin{enumerate}
\item $4x+7y+3=0$
\item $2x-9y-11=0$
\item $4x-7y-11=0$
\item $2x+9y+7=0$
\end{enumerate}
\end{multicols}
\end{frame}



\begin{frame}{Theoretical Solution}

Given the points,\\
\begin{align}
\vec{P}=\myvec{2\\2}\ \ \vec{Q}=\myvec{6\\-1}\ \ \vec{R}=\myvec{7\\3}
\end{align}
$\vec{S}$ is the midpoint of the line segment joining points $\vec{Q}$ and $\vec{R}$.\\
If $\vec{S}$ divides $QR$ in the ratio $k : 1$,
\end{frame}

\begin{frame}{Formulae}
\textbf{Section formula for a vector $\vec{S}$ which divides the line formed by vectors $\vec{Q}$ and $\vec{R}$ in the ratio $k:1$ is given by}
\begin{align}
    \vec{S}=\frac{k\vec{R}+\vec{Q}}{k+1}
\end{align}
\end{frame}

\begin{frame}{Theoretical Solution}
where,
\begin{align}
    k=1
\end{align}
\begin{align}
    \vec{S}= \frac{\vec{R}+\vec{Q}}{2}
\end{align}
\begin{align}
    \implies\vec{S}= \myvec{13/2\\1}
\end{align}
The direction vector of line $PS$ is given by,
\begin{align}
\vec{m}=\vec{S}-\vec{P} \equiv \myvec{9/2\\-1}
\end{align}\\
Therefore, the normal vector of the desired line is given by,
\begin{align}
    \vec{n}=\myvec{1\\9/2}
\end{align}
\end{frame}


\begin{frame}{Theoretical Solution}
$\therefore$ The equation of the line passing through $\myvec{1\\-1}$ and parallel to $PS$ is given by
\begin{align}
    \vec{n}^{\top}\brak{\vec{x}-\myvec{1\\-1}}=0
\end{align}
\begin{align}
    \myvec{1 & 9/2}\myvec{x-1\\y+1}=0
\end{align}
\begin{align}
    \implies x-1 + \frac{9}{2}\brak{y+1}=0
\end{align}
\begin{align}
    \implies 2x+9y+7=0
\end{align}
\end{frame}

\begin{frame}{Plot}
    \centering
    \includegraphics[width=\columnwidth, height=0.8\textheight, keepaspectratio]{figs/fig.png}
\end{frame}


\end{document}
