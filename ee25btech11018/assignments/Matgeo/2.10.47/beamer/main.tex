\documentclass{beamer}
\usepackage[utf8]{inputenc}

\usetheme{Boadilla}
\usecolortheme{lily}
\usepackage{amsmath,amssymb,amsfonts,amsthm}
\usepackage{mathtools}
\usepackage{txfonts}
\usepackage{tkz-euclide}
\usepackage{listings}
\usepackage{adjustbox}
\usepackage{array}
\usepackage{tabularx}
\usepackage{lmodern}
\usepackage{gvv}
\usepackage[T1]{fontenc}
\usepackage{circuitikz}
\usepackage{tikz}
\usepackage{graphicx}

\setbeamertemplate{footline}
{
  \leavevmode%
  \hbox{%
  \begin{beamercolorbox}[wd=\paperwidth,ht=2.25ex,dp=1ex,right]{author in head/foot}%
    \insertframenumber{} / \inserttotalframenumber\hspace*{2ex} 
  \end{beamercolorbox}}%
  \vskip0pt%
}

\usepackage{tcolorbox}
\tcbuselibrary{minted,breakable,xparse,skins}




\providecommand{\nCr}[2]{\,^{#1}C_{#2}} % nCr
\providecommand{\nPr}[2]{\,^{#1}P_{#2}} % nPr
\providecommand{\mbf}{\mathbf}
\providecommand{\pr}[1]{\ensuremath{\Pr\left(#1\right)}}
\providecommand{\qfunc}[1]{\ensuremath{Q\left(#1\right)}}
\providecommand{\sbrak}[1]{\ensuremath{{}\left[#1\right]}}
\providecommand{\lsbrak}[1]{\ensuremath{{}\left[#1\right.}}
\providecommand{\rsbrak}[1]{\ensuremath{{}\left.#1\right]}}
\providecommand{\brak}[1]{\ensuremath{\left(#1\right)}}
\providecommand{\lbrak}[1]{\ensuremath{\left(#1\right.}}
\providecommand{\rbrak}[1]{\ensuremath{\left.#1\right)}}
\providecommand{\cbrak}[1]{\ensuremath{\left\{#1\right\}}}
\providecommand{\lcbrak}[1]{\ensuremath{\left\{#1\right.}}
\providecommand{\rcbrak}[1]{\ensuremath{\left.#1\right\}}}
\theoremstyle{remark}
\renewcommand{\sgn}{\mathop{\mathrm{sgn}}}
\renewcommand{\solution}{\noindent\textbf{Solution: }}
\providecommand{\abs}[1]{\left\vert#1\right\vert}
\providecommand{\res}[1]{\Res\displaylimits_{#1}} 
\providecommand{\norm}[1]{\lVert#1\rVert}
\providecommand{\mtx}[1]{\mathbf{#1}}
\providecommand{\mean}[1]{E\left[ #1 \right]}
\providecommand{\fourier}{\overset{\mathcal{F}}{ \rightleftharpoons}}
%\providecommand{\hilbert}{\overset{\mathcal{H}}{ \rightleftharpoons}}
\providecommand{\system}{\overset{\mathcal{H}}{ \longleftrightarrow}}
	%\newcommand{\solution}[2]{\textbf{Solution:}{#1}}
%\newcommand{\solution}{\noindent \textbf{Solution: }}
\providecommand{\dec}[2]{\ensuremath{\overset{#1}{\underset{#2}{\gtrless}}}}
\let\vec\mathbf

\lstset{
%language=C,
frame=single, 
breaklines=true,
columns=fullflexible
}

\numberwithin{equation}{section}

\lstset{
  language=Python,
  basicstyle=\ttfamily\small,
  keywordstyle=\color{blue},
  stringstyle=\color{orange},
  numbers=left,
  numberstyle=\tiny\color{gray},
  breaklines=true,
  showstringspaces=false
}

\title{Problem 2.10.47}
\author{EE25BTECH11018-Darisy Sreetej}

\date{\today} 
\begin{document}

\begin{frame}
\titlepage
\end{frame}


\section{Problem}

\begin{frame}
\frametitle{Problem}
The value of $a$ so that the volume of parallelopiped formed by $\hat{i} + a\hat{j} + \hat{k}, \hat{j} + a\hat{k} \text{ and } a\hat{i} + \hat{k}$ becomes minimum is
\begin{enumerate}
\item $-3$
\item $3$
\item $\frac{1}{\sqrt{3}}$
\item $\sqrt{3}$
\end{enumerate}
\end{frame}
%\subsection{Literature}
\section{Solution}

\subsection{Formula}
\setcounter{section}{1}
\begin{frame}
\frametitle{Formula}
Volume of the parallelopiped
\begin{align*}
 V=\vec{p} \cdot (\vec{q} \times \vec{r}) = \sqrt{\mydet{\vec{G}}}  
\end{align*}
where $\vec{G}$ is the Gram matrix of the vectors
\end{frame}
\subsection{Obtaining the Gram Matrix}
\begin{frame}
\frametitle{Obtaining the Gram Matrix}
Let us consider,
\begin{align*}
\vec{p}=\hat{i}+a\hat{j}+\hat{k}\\
\vec{q}=\hat{j}+a\hat{k}\\
\vec{r}=a\hat{i}+\hat{k}
\end{align*}
The Gram matrix $\vec{G} $ for the vectors $ \vec{p}, \vec{q}, \vec{r} $ is:
\begin{align}
\vec{G} = \myvec{
\vec{p}^\top \vec{p} & \vec{p}^\top \vec{q} & \vec{p}^\top \vec{r} \\
\vec{q}^\top \vec{p} & \vec{q}^\top \vec{q} & \vec{q}^\top \vec{r} \\
\vec{r}^\top \vec{p} & \vec{r}^\top \vec{q} & \vec{r}^\top \vec{r}
}
\end{align}

Now, calculate the dot products:

\begin{align}
\vec{p}^\top \vec{p} = 1^2 + a^2 + 1^2 = 2+a^2
\end{align}
\end{frame}
\begin{frame}
\frametitle{Obtaining the Gram Matrix}
\begin{align}
\vec{p}^\top \vec{q} = (1)(0) + (a)(1) + (1)(a) = 2a
\end{align}

\begin{align}
\vec{p}^\top \vec{r} = (1)(a) + (a)(0) + (1)(1) = a+1
\end{align}

\begin{align}
\vec{q}^\top \vec{p} = \vec{p}^\top \vec{q} = 2a
\end{align}

\begin{align}
\vec{q}^\top \vec{q} = a^2 + 1^2 = 1 + a^2
\end{align}

\begin{align}
\vec{q}^\top \vec{r} = (0)(a) + (1)(0) + (a)(1) = a
\end{align}

\begin{align}
\vec{r}^\top \vec{p} = \vec{p}^\top \vec{r} = a + 1
\end{align}

 \end{frame}
\begin{frame}
\frametitle{Obtaining the Gram Matrix}
\begin{align}
\vec{r}^\top \vec{q} = \vec{q}^\top \vec{r} = a
\end{align}

\begin{align}
\vec{r}^\top \vec{r} = a^2 + 1^2 = a^2 + 1 
\end{align}

Thus, the Gram matrix $ \vec{G} $ is:
\begin{align}
\vec{G} = \myvec{
2+a^2 & 2a & a+1 \\
2a & 1+a^2 & a \\
a+1 & a & 1+a^2
}
\end{align}
\end{frame}
\subsection{Calculating Volume}
\begin{frame}[fragile]
\frametitle{Calculating Volume}
    The characteristic equation is obtained by solving the determinant equation $ \mydet{\vec{G} - \lambda \vec{I}} = 0 $. The characteristic polynomial for the matrix is:

\begin{align}
\lambda^3 - (3a^2+4)\lambda^2 + (3a^4+2a^2+4)\lambda - (a^6-2a^4+2a^3+a^2-2a+1) = 0
\end{align}

The determinant of $ \vec{G} $ is the product of its eigenvalues:
\begin{align}
\mydet{\vec{G}} = \lambda_1 \lambda_2 \lambda_3 = (a^6-2a^4+2a^3+a^2-2a+1).
\end{align}


The scalar triple product is the square root of the determinant of \( \vec{G} \):
\begin{align}
\vec{p} \cdot (\vec{q} \times \vec{r}) = \sqrt{\mydet{\vec{G}}} = \sqrt{(a^6-2a^4+2a^3+a^2-2a+1)} = a^3-a+1
\end{align}
\begin{align}
    V=a^3-a+1
\end{align}
\end{frame}

\subsection{Finding a for minimum volume}
\begin{frame}[fragile]
\frametitle{Finding 'a' for minimum volume}
    Now , consider
\begin{align}
    f(a)=a^3-a+1\\
    f'(a)=3a^2+1
\end{align}
\begin{align*}
\text{Set }f'(a)=0  \Rightarrow a^2=\frac{1}{\sqrt{3}} \Rightarrow a=\frac{1}{\sqrt{3}} or -\frac{1}{\sqrt{3}} 
\end{align*}
\begin{align}
\text{Second derivative }f''(a)=6a\\
\text{At }a=\frac{1}{\sqrt{3}},f''>0 \Rightarrow minimum\\
\text{At }a=-\frac{1}{\sqrt{3}},f''<0 \Rightarrow maximum
\end{align}
Therefore , $a=\frac{1}{\sqrt{3}}$ for which the Volume of the parallelopiped becomes minimum.
\end{frame}
\subsection{Plot}
\begin{frame}[fragile]
\frametitle{Plot}

\begin{figure}[h!]
   \centering
   \includegraphics[width=0.7\columnwidth]{figs/fig.png}
	\caption*{Parallelopiped with Vectors $\vec{p},\vec{q},\vec{r}$ for which $a=\frac{1}{\sqrt{3}}$(Volume is minimum)}
   \label{}
\end{figure}
\end{frame}

\section{C Code}
\begin{frame}[fragile]
\frametitle{C Code}
\begin{lstlisting}[language=C]
// Determinant of Gram matrix calculation with inline dot product calculations
double gram_determinant(double a) {
    double G11 = 2 + a * a;
    double G12 = 2 * a;
    double G13 = a + 1;
    double G21 = 2 * a;
    double G22 = a * a + 1;
    double G23 = a;
    double G31 = a + 1;
    double G32 = a;
    double G33 = 1 + a * a;
    double det = G11 * (G22 * G33 - G23 * G32) - G12 * (G21 * G33 - G23 * G31)+ G13 * (G21 * G32 - G22 * G31);
               return det;
}

    \end{lstlisting}
\end{frame}
\begin{frame}[fragile]
    \frametitle{C code}
    \begin{lstlisting}[language=C]
        // Volume calculation (sqrt of determinant)
double volume(double a) {
    double det = gram_determinant(a);
    if (det < 0) det = -det; // use absolute value for volume
    return sqrt(det);
}
int main() {
    double options[] = {-3, 3, 1.0 / sqrt(3), sqrt(3)};
    int num_options = 4;
    double min_vol = 1e9;
    double min_a = 0;
    printf("%-10s %-15s %-15s\n", "a", "Determinant", "Volume");
    for (int i = 0; i < num_options; i++) {
        double a = options[i];
        double det = gram_determinant(a);
        double vol = volume(a);
        printf("%-10.6f %-15.6f %-15.6f\n", a, det, vol);

    \end{lstlisting}
\end{frame}
\begin{frame}[fragile]
    \frametitle{C code}
    \begin{lstlisting}[language=C]
if (vol < min_vol) {
            min_vol = vol;
            min_a = a;
        }
    }
    printf("\nMinimum volume is at a = %.6f with volume = %.6f\n", min_a, min_vol);
    return 0;
}
     \end{lstlisting}
\end{frame}
\section{Python Code}
\begin{frame}[fragile]
\frametitle{Python Code for Solving}
\begin{lstlisting}[language=Python]

import ctypes
import numpy as np

lib = ctypes.CDLL('./volume.so')

lib.volume.argtypes = [ctypes.c_double]
lib.volume.restype = ctypes.c_double
a_values = np.array([-3, 3, 1.0 / np.sqrt(3), np.sqrt(3)])
min_volume = float('inf')
min_a = None
for a in a_values:
    vol = lib.volume(a)
    if vol < min_volume:
        min_volume = vol
        min_a = a

print(f"The value of a for which the volume is minimum: {min_a:.6f}")
\end{lstlisting}
\end{frame}
\begin{frame}[fragile]
\frametitle{Python Code for Plotting}
\begin{lstlisting}[language=Python]
import numpy as np
import matplotlib.pyplot as plt
from mpl_toolkits.mplot3d import Axes3D
from mpl_toolkits.mplot3d.art3d import Poly3DCollection
import matplotlib.patches as mpatches

a_val = 1 / np.sqrt(3)
p = np.array([1, a_val, 1])
q = np.array([0, 1, a_val])
r = np.array([a_val, 0, 1])
origin = np.array([0, 0, 0])
vertices = [
    origin, p, q, r, p + q, q + r, r + p, p + q + r ]
\end{lstlisting}
\end{frame}
\begin{frame}[fragile]
\frametitle{Python Code for Plotting}
\begin{lstlisting}[language=Python]
faces = [
    [vertices[0], vertices[1], vertices[4], vertices[2]],
    [vertices[0], vertices[1], vertices[6], vertices[3]],
    [vertices[0], vertices[2], vertices[5], vertices[3]],
    [vertices[7], vertices[5], vertices[4], vertices[6]],
    [vertices[7], vertices[6], vertices[1], vertices[3]],
    [vertices[7], vertices[5], vertices[2], vertices[4]]
]

fig = plt.figure(figsize=(10, 8))
ax = fig.add_subplot(111, projection='3d')
color_p = '#FF7F0E'  # orange
color_q = '#9467BD'  # purple
color_r = '#17BECF'  # teal
shaded_color = '#ffc0cb'  # pink
# Plot vectors
ax.quiver(*origin, *p, color=color_p, linewidth=3, arrow_length_ratio=0.15)

\end{lstlisting}
\end{frame}
\begin{frame}[fragile]
\frametitle{Python Code for Plotting}
\begin{lstlisting}[language=Python]
ax.quiver(*origin, *q, color=color_q, linewidth=3, arrow_length_ratio=0.15)
ax.quiver(*origin, *r, color=color_r, linewidth=3, arrow_length_ratio=0.15)
# Plot shaded volume - no automatic label, so use patch for legend
poly3d = Poly3DCollection(faces, facecolors=shaded_color,edgecolors='gray', linewidths=1.2, alpha=0.7)
ax.add_collection3d(poly3d)


p_patch = mpatches.Patch(color=color_p, label='p')
q_patch = mpatches.Patch(color=color_q, label='q')
r_patch = mpatches.Patch(color=color_r, label='r')
vol_patch = mpatches.Patch(color=shaded_color, label='Volume', alpha=0.7)
\end{lstlisting}
\end{frame}

\begin{frame}[fragile]
\frametitle{Python Code for Plotting}
\begin{lstlisting}[language=Python]
ax.set_xlim([-0.5, 2.5])
ax.set_ylim([-0.5, 2.5])
ax.set_zlim([-0.5, 2.5])
ax.set_xlabel('X axis')
ax.set_ylabel('Y axis')
ax.set_zlabel('Z axis')
ax.legend(handles=[p_patch, q_patch, r_patch, vol_patch])
plt.show()
plt.savefig("fig.png")

\end{lstlisting}
\end{frame}

\end{document}