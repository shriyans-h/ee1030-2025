\documentclass{beamer}
\usepackage{listings}
\usepackage{color}
\usepackage{amsmath}
\usepackage{gvv}

\title{Angle Between Vectors Using Gram Matrix}
\author{EE25BTECH11008 - Anirudh M Abhilash}
\date{September 14, 2025}

\begin{document}

%----------------- Title -------------------
\begin{frame}
\titlepage
\end{frame}

%----------------- Problem -------------------
\begin{frame}{Problem Statement}
Find the acute angle between the planes 
\begin{align*}
x - 2y - 2z = 5 \\
\quad 3x - 6y + 2z = 7
\end{align*}
\end{frame}

%----------------- Solution -------------------
\begin{frame}{Solution}
The angle between two planes is the angle between their normals.
Let
\[
\vec{n}_1 = \myvec{1 \\ -2 \\ -2}, 
\quad \vec{n}_2 = \myvec{3 \\ -6 \\ 2}.
\]

Form the matrix\
\begin{align}
A = \myvec{1 & 3 \\ -2 & -6 \\ -2 & 2},
\end{align}
whose columns are the normals.  
The Gram matrix is
\begin{align}
G = A^\top A 
= \myvec{\vec{n}_1^\top \vec{n}_1 & \vec{n}_1^\top \vec{n}_2 \\ \vec{n}_2^\top \vec{n}_1 & \vec{n}_2^\top \vec{n}_2}.
\end{align}
\end{frame}

%----------------- Solution (cont) -------------------
\begin{frame}{Solution (cont..)}
Now,
\[
\vec{n}_1^\top \vec{n}_1 = 9, \quad 
\vec{n}_2^\top \vec{n}_2 = 49, \quad 
\vec{n}_1^\top \vec{n}_2 = 11.
\]

Thus,
\begin{align}
G = \myvec{9 & 11 \\ 11 & 49}.
\end{align}


Let
\begin{align}
D = \myvec{9 & 0 \\ 0 & 49}, 
\quad D^{-1/2} = \myvec{\tfrac{1}{3} & 0 \\ 0 & \tfrac{1}{7}}.
\end{align}
\end{frame}

\begin{frame}{Solution (cont..)}
The normalized Gram matrix is
\begin{align}
C = D^{-1/2} G D^{-1/2} 
= \myvec{1 & \tfrac{11}{21} \\ \tfrac{11}{21} & 1}.
\end{align}

The off-diagonal entry gives
\begin{align}
\cos\theta = \frac{11}{21}.
\end{align}

Hence, the acute angle between the planes is
\[
\boxed{\theta = \arccos\!\left(\tfrac{11}{21}\right) \approx 58.41^\circ}
\]
\end{frame}

\begin{frame}[fragile]{Python Code (Plotting Normals)}
\begin{lstlisting}[language=Python]
import numpy as np
import matplotlib.pyplot as plt
from mpl_toolkits.mplot3d import Axes3D

u = np.array([1, -2, -2])
v = np.array([3, -6,  2])
origin = np.zeros(3)

fig = plt.figure()
ax = fig.add_subplot(111, projection='3d')

ax.quiver(*origin, *u, color='r', arrow_length_ratio=0.1)
ax.text(u[0]*1.1, u[1]*1.1, u[2]*1.1, "u", color='r')
\end{lstlisting}
\end{frame}

\begin{frame}[fragile]{Python Code (cont..)}
\begin{lstlisting}[language=Python]
ax.quiver(*origin, *v, color='b', arrow_length_ratio=0.1)
ax.text(v[0]*1.1, v[1]*1.1, v[2]*1.1, "v", color='b')

all_points = np.vstack([origin, u, v])
ax.set_xlim([all_points[:,0].min()-1, all_points[:,0].max()+1])
ax.set_ylim([all_points[:,1].min()-1, all_points[:,1].max()+1])
ax.set_zlim([all_points[:,2].min()-1, all_points[:,2].max()+1])

ax.set_xlabel("X")
ax.set_ylabel("Y")
ax.set_zlabel("Z")
ax.set_title("Normal vectors U and V in 3D plot")

plt.show()
\end{lstlisting}
\end{frame}

%----------------- Plot -------------------
\begin{frame}{Plot (Python)}
\centering
\includegraphics[width=1\linewidth]{figs/plt.png}
\end{frame}

%----------------- C Code -------------------

\begin{frame}[fragile]{C Code (Matrix multiplication)}
\begin{lstlisting}[language=C]
#include <stdio.h>
#include <math.h>

void matmul(double A[2][2], double B[2][2], double C[2][2]) {
    for (int i = 0; i < 2; i++) {
        for (int j = 0; j < 2; j++) {
            C[i][j] = 0.0;
            for (int k = 0; k < 2; k++) {
                C[i][j] += A[i][k] * B[k][j];
            }
        }
    }
}

\end{lstlisting}
\end{frame}

\begin{frame}[fragile]{C Code (Finding G)}
\begin{lstlisting}[language=C]
void gram_matrix(double *u, double *v, int n, double G[2][2]) {
    double g11 = 0.0, g22 = 0.0, g12 = 0.0;
    for (int i = 0; i < n; i++) {
        g11 += u[i] * u[i];
        g22 += v[i] * v[i];
        g12 += u[i] * v[i];
    }
    G[0][0] = g11;
    G[0][1] = g12;
    G[1][0] = g12;
    G[1][1] = g22;
}
\end{lstlisting}
\end{frame}

\begin{frame}[fragile]{C Code (Normalising G)}
\begin{lstlisting}[language=C]
void normalize_gram(double G[2][2], double G_norm[2][2]) {
    double d11 = 1.0 / sqrt(G[0][0]);
    double d22 = 1.0 / sqrt(G[1][1]);

    double Dinv[2][2] = {{d11, 0.0}, {0.0, d22}};

    double temp[2][2];
    matmul(Dinv, G, temp); 
    matmul(temp, Dinv, G_norm);
}
\end{lstlisting}
\end{frame}

\begin{frame}[fragile]{C Code (Finding Angle)}
\begin{lstlisting}[language=C]
double angle_from_normalized(double G_norm[2][2]) {
    double cos_theta = G_norm[0][1]; 
    if (cos_theta > 1.0) {
        cos_theta = 1.0; 
    }
    if (cos_theta < -1.0) {
        cos_theta = -1.0;
    }
    return acos(cos_theta);
}
\end{lstlisting}
\end{frame}


%----------------- Python Code -------------------
\begin{frame}[fragile]{Python Code (Calling C)}
\begin{lstlisting}[language=Python]
import ctypes
import numpy

lib = ctypes.CDLL("./computations.so")

lib.gram_matrix.argtypes = [
    ctypes.POINTER(ctypes.c_double),
    ctypes.POINTER(ctypes.c_double),
    ctypes.c_int,
    ctypes.POINTER((ctypes.c_double * 2) * 2)
]
\end{lstlisting}
\end{frame}


\begin{frame}[fragile]{Python Code (cont..)}
\begin{lstlisting}[language=Python]
lib.normalize_gram.argtypes = [
    ctypes.POINTER((ctypes.c_double * 2) * 2),
    ctypes.POINTER((ctypes.c_double * 2) * 2)
]

lib.angle_from_normalized.argtypes = [
    ctypes.POINTER((ctypes.c_double * 2) * 2)
]
lib.angle_from_normalized.restype = ctypes.c_double
\end{lstlisting}
\end{frame}


\begin{frame}[fragile]{Python Code (cont..)}
\begin{lstlisting}[language=Python]
def compute_angle(u, v):
    n = len(u)
    u_arr = (ctypes.c_double * n)(*u)
    v_arr = (ctypes.c_double * n)(*v)

    G = ((ctypes.c_double * 2) * 2)()
    G_norm = ((ctypes.c_double * 2) * 2)()

    lib.gram_matrix(u_arr, v_arr, n, G)
    lib.normalize_gram(G, G_norm)
    theta = lib.angle_from_normalized(G_norm)
    return theta
\end{lstlisting}
\end{frame}


\begin{frame}[fragile]{Python Code (cont..)}
\begin{lstlisting}[language=Python]
u = [1, -2, -2]
v = [3, -6, 2]
theta = compute_angle(u, v)

print(f"Angle(radians):{theta}")
print(f"Angle(degrees):{theta*180/numpy.pi}")
\end{lstlisting}
\end{frame}

\end{document}