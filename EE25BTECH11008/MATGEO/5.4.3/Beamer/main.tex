\documentclass{beamer}
\usepackage{listings}
\usepackage{color}
\usepackage{amsmath}
\usepackage{gvv}

\title{Inverse of a Matrix Using Elementary Transformations}
\author{EE25BTECH11008 - Anirudh M Abhilash}
\date{October 4, 2025}

\begin{document}

%----------------- Title -------------------
\begin{frame}
\titlepage
\end{frame}

%----------------- Problem -------------------
\begin{frame}{Problem Statement}
Find the inverse of the matrix 
\[
A = \myvec{2 & 3 \\ 1 & 4}
\]
using elementary transformations.
\end{frame}

%----------------- Solution -------------------
\begin{frame}{Solution}
\[
A A^{-1} = I,
\]

We write the augmented matrix of $A$ with the identity matrix:
\[
[A | I] = \augvec{2}{2}{2 & 3 & 1 & 0 \\ 1 & 4 & 0 & 1}.
\]
\end{frame}

%----------------- Solution (cont) -------------------
\begin{frame}{Solution (cont..)}
\textbf{Step 1:}
\[
R_1 \to \frac{1}{2} R_1
\]
\[
\augvec{2}{2}{1 & 3/2 & 1/2 & 0 \\ 1 & 4 & 0 & 1}.
\]

\textbf{Step 2:}
\[
R_2 \to R_2 - R_1
\]
\[
\augvec{2}{2}{1 & 3/2 & 1/2 & 0 \\ 0 & 5/2 & -1/2 & 1}.
\]
\end{frame}

%----------------- Solution (cont) -------------------
\begin{frame}{Solution (cont..)}
\textbf{Step 3:}
\[
R_2 \to \frac{2}{5} R_2
\]
\[
\augvec{2}{2}{1 & 3/2 & 1/2 & 0 \\ 0 & 1 & -1/5 & 2/5}.
\]

\textbf{Step 4:}
\[
R_1 \to R_1 - \frac{3}{2} R_2
\]
\[
\augvec{2}{2}{1 & 0 & 4/5 & -3/5 \\ 0 & 1 & -1/5 & 2/5}.
\]
\end{frame}

%----------------- Solution (cont) -------------------
\begin{frame}{Solution (cont..)}
Hence, the inverse of $A$ is
\[
A^{-1} = \myvec{4/5 & -3/5 \\ -1/5 & 2/5}.
\]
\end{frame}

%----------------- C Code -------------------

\begin{frame}[fragile]{C Code (Inverse)}
\begin{lstlisting}[language=C]
#include <stdio.h>
#include <stdlib.h>

void inverse(double *mat, double *inv, int n) {
    int i, j, k;
    double temp;
    double **aug = (double **)malloc(n * sizeof(double *));
    for (i = 0; i < n; i++) {
        aug[i] = (double *)malloc(2 * n * sizeof(double));
        for (j = 0; j < n; j++) {
            aug[i][j] = mat[i*n + j];
            aug[i][j+n] = (i == j) ? 1.0 : 0.0; 
        }
    }
\end{lstlisting}
\end{frame}

\begin{frame}[fragile]{C Code (Cont..)}
\begin{lstlisting}[language=C]
    for (i = 0; i < n; i++) {
        temp = aug[i][i];
        for (j = 0; j < 2*n; j++)
            aug[i][j] /= temp;

        for (k = 0; k < n; k++) {
            if (k != i) {
                temp = aug[k][i];
                for (j = 0; j < 2*n; j++)
                    aug[k][j] -= temp * aug[i][j];
            }
        }
    }
\end{lstlisting}
\end{frame}

\begin{frame}[fragile]{C Code (Cont..)}
\begin{lstlisting}[language=C]
    for (i = 0; i < n; i++)
        for (j = 0; j < n; j++)
            inv[i*n + j] = aug[i][j+n];

    for (i = 0; i < n; i++)
        free(aug[i]);
    free(aug);
}
\end{lstlisting}
\end{frame}

%----------------- Python Code -------------------
\begin{frame}[fragile]{Python Code (Using C)}
\begin{lstlisting}[language=Python]
import ctypes
import numpy as np

lib = ctypes.CDLL('./inv.so')

lib.inverse.argtypes = [ctypes.POINTER(ctypes.c_double),
                        ctypes.POINTER(ctypes.c_double),
                        ctypes.c_int]
lib.inverse.restype = None

A = np.array([[2, 3],
              [1, 4]], dtype=np.float64)
n = A.shape[0]
\end{lstlisting}
\end{frame}

\begin{frame}[fragile]{Python Code (Cont..)}
\begin{lstlisting}[language=Python]
A_inv = np.zeros((n, n), dtype=np.float64)

lib.inverse(A.ctypes.data_as(ctypes.POINTER(ctypes.c_double)),
            A_inv.ctypes.data_as(ctypes.POINTER(ctypes.c_double)),
            n)

print("Original matrix:")
print(A)
print("Inverse matrix:")
print(A_inv)
\end{lstlisting}
\end{frame}

\end{document}