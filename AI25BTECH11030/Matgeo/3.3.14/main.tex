\let\negmedspace\undefined
\let\negthickspace\undefined
\documentclass[journal]{IEEEtran}
\usepackage[a5paper, margin=10mm, onecolumn]{geometry}
%\usepackage{lmodern} 
\usepackage{tfrupee} 

\setlength{\headheight}{1cm} 
\setlength{\headsep}{0mm}     

\usepackage{gvv-book}
\usepackage{gvv}
\usepackage{cite}
\usepackage{amsmath,amssymb,amsfonts,amsthm}
\usepackage{algorithmic}
\usepackage{graphicx}
\usepackage{textcomp}
\usepackage{xcolor}
\usepackage{txfonts}
\usepackage{listings}
\usepackage{enumitem}
\usepackage{mathtools}
\usepackage{gensymb}
\usepackage{comment}
\usepackage[breaklinks=true]{hyperref}
\usepackage{tkz-euclide} 
\usepackage{listings}                                        
\def\inputGnumericTable{}                                 
\usepackage[latin1]{inputenc}                                
\usepackage{color}                                            
\usepackage{array}                                            
\usepackage{longtable}                                       
\usepackage{calc}                                             
\usepackage{multirow}                                         
\usepackage{hhline}                                           
\usepackage{ifthen}                                           
\usepackage{lscape}

\begin{document}

\bibliographystyle{IEEEtran}
\vspace{3cm}

\title{3.3.14}
\author{AI25BTECH11030 -Sarvesh Tamgade}
{\let\newpage\relax\maketitle}

\renewcommand{\thefigure}{\theenumi}
\renewcommand{\thetable}{\theenumi}
\setlength{\intextsep}{10pt} 


\numberwithin{equation}{enumi}
\numberwithin{figure}{enumi}
\renewcommand{\thetable}{\theenumi}


\textbf{Question}: Construct a right triangle in which the sides, (other than the hypotenuse) are of length 6 cm and 8 cm.

\textbf{Solution:}\\
Let the two perpendicular sides have lengths 6 cm and 8 cm respectively.

Assume vertices:
\[
\vec{A} = \myvec{0 \\ 0}, \quad
\vec{B} = \myvec{6 \\ 0}, \quad
\vec{C} = \myvec{6 \cos 90^\circ \\ 8 \sin 90^\circ} = \myvec{0 \\ 8}
\]


This forms a right angle at vertex \(B\).
\begin{figure}[htbp]
    \centering
    \includegraphics[width=0.65\linewidth]{FIG/fig1.png}
    \caption{Vector Representation}
    \label{fig:FIG/fig1.png}
    \end{figure}

\end{document}  