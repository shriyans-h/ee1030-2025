\documentclass{beamer}
\mode<presentation>
\usepackage{amsmath}
\usepackage{amssymb}
%\usepackage{advdate}
\usepackage{graphicx}
\usepackage{adjustbox}
\usepackage{subcaption}
\usepackage{enumitem}
\usepackage{multicol}
\usepackage{mathtools}
\usepackage{listings}
\usepackage{url}
\def\UrlBreaks{\do\/\do-}
\usetheme{Boadilla}
\usecolortheme{lily}
\setbeamertemplate{footline}
{
  \leavevmode%
  \hbox{%
  \begin{beamercolorbox}[wd=\paperwidth,ht=2.25ex,dp=1ex,right]{author in head/foot}%
    \insertframenumber{} / \inserttotalframenumber\hspace*{2ex} 
  \end{beamercolorbox}}%
  \vskip0pt%
}
\setbeamertemplate{navigation symbols}{}

\providecommand{\nCr}[2]{\,^{#1}C_{#2}} % nCr
\providecommand{\nPr}[2]{\,^{#1}P_{#2}} % nPr
\providecommand{\mbf}{\mathbf}
\providecommand{\pr}[1]{\ensuremath{\Pr\left(#1\right)}}
\providecommand{\qfunc}[1]{\ensuremath{Q\left(#1\right)}}
\providecommand{\sbrak}[1]{\ensuremath{{}\left[#1\right]}}
\providecommand{\lsbrak}[1]{\ensuremath{{}\left[#1\right.}}
\providecommand{\rsbrak}[1]{\ensuremath{{}\left.#1\right]}}
\providecommand{\brak}[1]{\ensuremath{\left(#1\right)}}
\providecommand{\lbrak}[1]{\ensuremath{\left(#1\right.}}
\providecommand{\rbrak}[1]{\ensuremath{\left.#1\right)}}
\providecommand{\cbrak}[1]{\ensuremath{\left\{#1\right\}}}
\providecommand{\lcbrak}[1]{\ensuremath{\left\{#1\right.}}
\providecommand{\rcbrak}[1]{\ensuremath{\left.#1\right\}}}
\theoremstyle{remark}
\newtheorem{rem}{Remark}
\newcommand{\sgn}{\mathop{\mathrm{sgn}}}
\providecommand{\abs}[1]{\vert#1\vert}
\providecommand{\res}[1]{\Res\displaylimits_{#1}} 
\providecommand{\norm}[1]{\lVert#1\rVert}
\providecommand{\mtx}[1]{\mathbf{#1}}
\providecommand{\mean}[1]{E[ #1 ]}
\providecommand{\fourier}{\overset{\mathcal{F}}{ \rightleftharpoons}}
%\providecommand{\hilbert}{\overset{\mathcal{H}}{ \rightleftharpoons}}
\providecommand{\system}[1]{\overset{\mathcal{#1}}{ \longleftrightarrow}}
%\providecommand{\system}{\overset{\mathcal{H}}{ \longleftrightarrow}}
	%\newcommand{\solution}[2]{\vec{Solution:}{#1}}
%\newcommand{\solution}{\noindent \vec{Solution: }}
\providecommand{\dec}[2]{\ensuremath{\overset{#1}{\underset{#2}{\gtrless}}}}
\newcommand{\myvec}[1]{\ensuremath{\begin{pmatrix}#1\end{pmatrix}}}


\lstset{
%language=C,
frame=single, 
breaklines=true,
columns=fullflexible
}
\lstset{
  language=C,
  basicstyle=\ttfamily\footnotesize,
  keywordstyle=\color{blue}\bfseries,
  commentstyle=\color{gray}\itshape,
  stringstyle=\color{orange},
  numbers=left,
  numberstyle=\tiny\color{gray},
  breaklines=true,
  frame=single,
  showstringspaces=false,
  tabsize=4,
  captionpos=b
}
\numberwithin{equation}{section}
\lstset{
  language=Python,
  basicstyle=\ttfamily\small,
  keywordstyle=\color{blue},
  stringstyle=\color{orange},
  numbers=left,
  numberstyle=\tiny\color{gray},
  breaklines=true,
  showstringspaces=false
}



\title{Problem 2.10.20.}
\author{Sarvesh Tamgade}

\date{\today} 
\begin{document}

\begin{frame}
\titlepage
\end{frame}

\section{Question}
\begin{frame}{Question}
\textbf{Question}:
 Construct a right triangle in which the sides, (other than the hypotenuse) are of length 6 cm and 8 cm.
\end{frame}

\section{Solution}
\begin{frame}[fragile]
    \frametitle{Solution}
Let the two perpendicular sides have lengths 6 cm and 8 cm respectively.

Assume vertices:
\[
\vec{A} = \begin{pmatrix}0 \\ 0 \end{pmatrix},\quad
\vec{B} = \begin{pmatrix}6 \\ 0 \end{pmatrix}, \quad
\vec{C} = \begin{pmatrix}6 \cos 90^\circ \\ 8 \sin 90^\circ \end{pmatrix} = \begin{pmatrix}0 \\ 8 \end{pmatrix}
\]

This forms a right angle at vertex \(B\).



\end{frame}
\section{Graph}
\begin{frame}
    \frametitle{Graph}
    \begin{figure}[htbp]
    \centering
    \includegraphics[width=0.65\linewidth]{FIG/fig1.png}
    \caption{Vector Representation}
    \label{fig:FIG/fig1.png}
\end{figure}
\end{frame}
\section{ C Code}
\begin{frame}[fragile]
\frametitle{C Code }
\begin{lstlisting}[language=C]
#include <stdio.h>
#include "trianglefun.h"

int main() {
    Point A, B, C;

    construct_right_triangle(&A, &B, &C);

    printf("Coordinates of triangle vertices:\n");
    printf("A: (%.2f, %.2f)\n", A.x, A.y);
    printf("B: (%.2f, %.2f)\n", B.x, B.y);
    // Print symbolic expression alongside evaluated coordinate for C
    printf("C: (6 * cos(90°) = %.2f, 8 * sin(90°) = %.2f)\n", C.x, C.y);

    return 0;
}


    
\end{lstlisting}
\end{frame}


\begin{frame}[fragile]
\frametitle{Python Code for Plotting}
\begin{lstlisting}[language=Python]
import matplotlib.pyplot as plt
import numpy as np

# Define vertices A, B, C
A = np.array([0, 0])
B = np.array([6, 0])
C = np.array([6 * np.cos(np.radians(90)), 8 * np.sin(np.radians(90))])  # (0, 8)

# Create plot
plt.figure(figsize=(6,6))

# Draw triangle sides
plt.plot([A[0], B[0]], [A[1], B[1]], 'b-', label='AB = 6 cm')
plt.plot([B[0], C[0]], [B[1], C[1]], 'g-', label='BC = 8 cm')
plt.plot([C[0], A[0]], [C[1], A[1]], 'r-', label='AC (hypotenuse)')
\end{lstlisting}

\end{frame}
\begin{frame}[fragile]
\frametitle{Python Code for Plotting}
\begin{lstlisting}[language=Python]
# Mark points with labels and coordinates
plt.plot(A[0], A[1], 'ko')
plt.text(A[0]-1, A[1]-0.5, 'A (0, 0)', fontsize=12, fontweight='bold')

plt.plot(B[0], B[1], 'ko')
plt.text(B[0]+0.2, B[1]-0.5, 'B (6, 0)', fontsize=12, fontweight='bold')

plt.plot(C[0], C[1], 'ko')
plt.text(C[0]-3, C[1]+0.2, 'C (6 cos 90°, 8 sin 90°)', fontsize=12, fontweight='bold')

# Set axes limits and grid
plt.xlim(-4, 8)
plt.ylim(-2, 10)
plt.grid(True)
plt.gca().set_aspect('equal', adjustable='box')
\end{lstlisting}

\end{frame}
\begin{frame}[fragile]
\frametitle{Python Code for Plotting}
\begin{lstlisting}[language=Python]
# Title and legend
plt.title('Right Triangle ABC with sides 6 cm and 8 cm')
plt.legend()

# Save plot as PNG file
plt.savefig('triangle_abc_with_expr_coords.png')

# Close plot
plt.close()
\end{lstlisting}

\end{frame}


\end{document}
