\documentclass{beamer}
\mode<presentation>
\usepackage{amsmath}
\usepackage{amssymb}
%\usepackage{advdate}
\usepackage{graphicx}
\usepackage{adjustbox}
\usepackage{subcaption}
\usepackage{enumitem}
\usepackage{multicol}
\usepackage{mathtools}
\usepackage{listings}
\usepackage{url}
\def\UrlBreaks{\do\/\do-}
\usetheme{Boadilla}
\usecolortheme{lily}
\setbeamertemplate{footline}
{
  \leavevmode%
  \hbox{%
  \begin{beamercolorbox}[wd=\paperwidth,ht=2.25ex,dp=1ex,right]{author in head/foot}%
    \insertframenumber{} / \inserttotalframenumber\hspace*{2ex} 
  \end{beamercolorbox}}%
  \vskip0pt%
}
\setbeamertemplate{navigation symbols}{}

\providecommand{\nCr}[2]{\,^{#1}C_{#2}} % nCr
\providecommand{\nPr}[2]{\,^{#1}P_{#2}} % nPr
\providecommand{\mbf}{\mathbf}
\providecommand{\pr}[1]{\ensuremath{\Pr\left(#1\right)}}
\providecommand{\qfunc}[1]{\ensuremath{Q\left(#1\right)}}
\providecommand{\sbrak}[1]{\ensuremath{{}\left[#1\right]}}
\providecommand{\lsbrak}[1]{\ensuremath{{}\left[#1\right.}}
\providecommand{\rsbrak}[1]{\ensuremath{{}\left.#1\right]}}
\providecommand{\brak}[1]{\ensuremath{\left(#1\right)}}
\providecommand{\lbrak}[1]{\ensuremath{\left(#1\right.}}
\providecommand{\rbrak}[1]{\ensuremath{\left.#1\right)}}
\providecommand{\cbrak}[1]{\ensuremath{\left\{#1\right\}}}
\providecommand{\lcbrak}[1]{\ensuremath{\left\{#1\right.}}
\providecommand{\rcbrak}[1]{\ensuremath{\left.#1\right\}}}
\theoremstyle{remark}
\newtheorem{rem}{Remark}
\newcommand{\sgn}{\mathop{\mathrm{sgn}}}
\providecommand{\abs}[1]{\vert#1\vert}
\providecommand{\res}[1]{\Res\displaylimits_{#1}} 
\providecommand{\norm}[1]{\lVert#1\rVert}
\providecommand{\mtx}[1]{\mathbf{#1}}
\providecommand{\mean}[1]{E[ #1 ]}
\providecommand{\fourier}{\overset{\mathcal{F}}{ \rightleftharpoons}}
%\providecommand{\hilbert}{\overset{\mathcal{H}}{ \rightleftharpoons}}
\providecommand{\system}[1]{\overset{\mathcal{#1}}{ \longleftrightarrow}}
%\providecommand{\system}{\overset{\mathcal{H}}{ \longleftrightarrow}}
	%\newcommand{\solution}[2]{\vec{Solution:}{#1}}
%\newcommand{\solution}{\noindent \vec{Solution: }}
\providecommand{\dec}[2]{\ensuremath{\overset{#1}{\underset{#2}{\gtrless}}}}
\newcommand{\myvec}[1]{\ensuremath{\begin{pmatrix}#1\end{pmatrix}}}


\lstset{
%language=C,
frame=single, 
breaklines=true,
columns=fullflexible
}
\lstset{
  language=C,
  basicstyle=\ttfamily\footnotesize,
  keywordstyle=\color{blue}\bfseries,
  commentstyle=\color{gray}\itshape,
  stringstyle=\color{orange},
  numbers=left,
  numberstyle=\tiny\color{gray},
  breaklines=true,
  frame=single,
  showstringspaces=false,
  tabsize=4,
  captionpos=b
}
\numberwithin{equation}{section}
\lstset{
  language=Python,
  basicstyle=\ttfamily\small,
  keywordstyle=\color{blue},
  stringstyle=\color{orange},
  numbers=left,
  numberstyle=\tiny\color{gray},
  breaklines=true,
  showstringspaces=false
}



\title{Problem 2.10.20.}
\author{Sarvesh Tamgade}

\date{\today} 
\begin{document}

\begin{frame}
\titlepage
\end{frame}

\section{Question}
\begin{frame}{Question}
\textbf{Question}:
 Construct a right triangle in which the sides, (other than the hypotenuse) are of length 6 cm and 8 cm.
\end{frame}

\section{Solution}
\begin{frame}[fragile]
    \frametitle{Solution}
Let the two sides of the right triangle be represented as vectors:

\[
\mathbf{A} = \begin{bmatrix} 6 \\ 0 \end{bmatrix}, \quad
\mathbf{B} = \begin{bmatrix} 0 \\ 8 \end{bmatrix}.
\]

The hypotenuse vector is the sum of these two vectors:

\[
\mathbf{C} = \mathbf{A} + \mathbf{B} = \begin{bmatrix} 6 \\ 0 \end{bmatrix} + \begin{bmatrix} 0 \\ 8 \end{bmatrix} = \begin{bmatrix} 6 \\ 8 \end{bmatrix}.
\]

The length (magnitude) of the hypotenuse is calculated as

\begin{align*}
c &= \| \mathbf{C} \| = \sqrt{6^2 + 8^2} \\
  &= \sqrt{36 + 64} = \sqrt{100} = 10 \text{ cm}.
\end{align*}

Therefore, the sides of the right triangle are 6 cm, 8 cm, and 10 cm.


\end{frame}
\section{Graph}
\begin{frame}
    \frametitle{Graph}
    \begin{figure}[htbp]
    \centering
    \includegraphics[width=0.65\linewidth]{FIG/fig1.png}
    \caption{Vector Representation}
    \label{fig:FIG/fig1.png}
\end{figure}
\end{frame}
\section{ C Code}
\begin{frame}[fragile]
\frametitle{C Code }
\begin{lstlisting}[language=C]
#include <stdio.h>
#include "trianglefun.h"

int main() {
    double side1, side2;
    printf("Enter the lengths of two sides of the right triangle: ");
    scanf("%lf %lf", &side1, &side2);

    Vector2D A = vector_create(side1, 0);
    Vector2D B = vector_create(0, side2);

    Vector2D C = vector_add(A, B);
    double hypotenuse = vector_magnitude(C);

    printf("The hypotenuse length is: %.2f\n", hypotenuse);

    return 0;
}

    
\end{lstlisting}
\end{frame}


\begin{frame}[fragile]
\frametitle{Python Code for Plotting}
\begin{lstlisting}[language=Python]
import numpy as np
import matplotlib.pyplot as plt

a = 6
b = 8
c = (a**2 + b**2)**0.5  # Hypotenuse calculation

# Coordinates for triangle at (0,0), (a,0), (0,b)
pts = np.array([[0,0], [a,0], [0,b], [0,0]])

plt.figure(figsize=(5,5))
plt.plot(pts[:,0], pts[:,1], '-o', label='Triangle')
plt.text(a/2, -0.5, f'{a} cm', ha='center')
plt.text(-0.5, b/2, f'{b} cm', va='center', rotation=90)
plt.text(a/2, b/2, f'{c:.2f} cm', color='purple')
plt.axis('equal')
plt.grid(True)
plt.title('Right Triangle: sides 6 cm and 8 cm')
plt.xlabel('cm')
\end{lstlisting}

\end{frame}
\begin{frame}[fragile]
\frametitle{Python Code for Plotting}
\begin{lstlisting}[language=Python]
plt.ylabel('cm')
plt.legend()
plt.savefig('right_triangle.png')
plt.show()
\end{lstlisting}

\end{frame}

\end{document}
