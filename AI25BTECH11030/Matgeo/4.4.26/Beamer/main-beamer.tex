\documentclass{beamer}
\mode<presentation>
\usepackage{amsmath}
\usepackage{amssymb}
%\usepackage{advdate}
\usepackage{graphicx}
\usepackage{adjustbox}
\usepackage{subcaption}
\usepackage{enumitem}
\usepackage{multicol}
\usepackage{mathtools}
\usepackage{listings}
\usepackage{url}
\def\UrlBreaks{\do\/\do-}
\usetheme{Boadilla}
\usecolortheme{lily}
\setbeamertemplate{footline}
{
  \leavevmode%
  \hbox{%
  \begin{beamercolorbox}[wd=\paperwidth,ht=2.25ex,dp=1ex,right]{author in head/foot}%
    \insertframenumber{} / \inserttotalframenumber\hspace*{2ex} 
  \end{beamercolorbox}}%
  \vskip0pt%
}
\setbeamertemplate{navigation symbols}{}

\providecommand{\nCr}[2]{\,^{#1}C_{#2}} % nCr
\providecommand{\nPr}[2]{\,^{#1}P_{#2}} % nPr
\providecommand{\mbf}{\mathbf}
\providecommand{\pr}[1]{\ensuremath{\Pr\left(#1\right)}}
\providecommand{\qfunc}[1]{\ensuremath{Q\left(#1\right)}}
\providecommand{\sbrak}[1]{\ensuremath{{}\left[#1\right]}}
\providecommand{\lsbrak}[1]{\ensuremath{{}\left[#1\right.}}
\providecommand{\rsbrak}[1]{\ensuremath{{}\left.#1\right]}}
\providecommand{\brak}[1]{\ensuremath{\left(#1\right)}}
\providecommand{\lbrak}[1]{\ensuremath{\left(#1\right.}}
\providecommand{\rbrak}[1]{\ensuremath{\left.#1\right)}}
\providecommand{\cbrak}[1]{\ensuremath{\left\{#1\right\}}}
\providecommand{\lcbrak}[1]{\ensuremath{\left\{#1\right.}}
\providecommand{\rcbrak}[1]{\ensuremath{\left.#1\right\}}}
\theoremstyle{remark}
\newtheorem{rem}{Remark}
\newcommand{\sgn}{\mathop{\mathrm{sgn}}}
\providecommand{\abs}[1]{\vert#1\vert}
\providecommand{\res}[1]{\Res\displaylimits_{#1}} 
\providecommand{\norm}[1]{\lVert#1\rVert}
\providecommand{\mtx}[1]{\mathbf{#1}}
\providecommand{\mean}[1]{E[ #1 ]}
\providecommand{\fourier}{\overset{\mathcal{F}}{ \rightleftharpoons}}
%\providecommand{\hilbert}{\overset{\mathcal{H}}{ \rightleftharpoons}}
\providecommand{\system}[1]{\overset{\mathcal{#1}}{ \longleftrightarrow}}
%\providecommand{\system}{\overset{\mathcal{H}}{ \longleftrightarrow}}
	%\newcommand{\solution}[2]{\vec{Solution:}{#1}}
%\newcommand{\solution}{\noindent \vec{Solution: }}
\providecommand{\dec}[2]{\ensuremath{\overset{#1}{\underset{#2}{\gtrless}}}}
\newcommand{\myvec}[1]{\ensuremath{\begin{pmatrix}#1\end{pmatrix}}}


\lstset{
%language=C,
frame=single, 
breaklines=true,
columns=fullflexible
}
\lstset{
  language=C,
  basicstyle=\ttfamily\footnotesize,
  keywordstyle=\color{blue}\bfseries,
  commentstyle=\color{gray}\itshape,
  stringstyle=\color{orange},
  numbers=left,
  numberstyle=\tiny\color{gray},
  breaklines=true,
  frame=single,
  showstringspaces=false,
  tabsize=4,
  captionpos=b
}
\numberwithin{equation}{section}
\lstset{
  language=Python,
  basicstyle=\ttfamily\small,
  keywordstyle=\color{blue},
  stringstyle=\color{orange},
  numbers=left,
  numberstyle=\tiny\color{gray},
  breaklines=true,
  showstringspaces=false
}



\title{Problem 4.4.26}
\author{Sarvesh Tamgade}

\date{\today} 
\begin{document}

\begin{frame}
\titlepage
\end{frame}

\section{Question}
\begin{frame}{Question}
\textbf{Question}:
 Find the equation of the median through vertex \(\mathbf{A}\) of the triangle \(ABC\), having vertices
\[
\mathbf{A}(2,5), \quad \mathbf{B}(-4,9), \quad \mathbf{C}(-2,-1).
\]

\end{frame}

\section{Solution}
\begin{frame}[fragile]
    \frametitle{Solution}
Using the section formula, the midpoint \(\mathbf{M}\) of the side \(BC\) is 
\[
\mathbf{M} = \frac{\mathbf{B} + \mathbf{C}}{2} = 
\frac{1}{2} \begin{bmatrix} -4 \\ 9 \end{bmatrix} + 
\frac{1}{2} \begin{bmatrix} -2 \\ -1 \end{bmatrix} = 
\begin{bmatrix} -3 \\ 4 \end{bmatrix}.
\]

The median passes through points \(\mathbf{A}(2,5)\) and \(\mathbf{M}(-3,4)\).
Let the required line have the equation
\[
\mathbf{n}^\top \mathbf{x} = 1
\]
where \( \mathbf{n} = \begin{bmatrix} n_1 \\ n_2 \end{bmatrix} \) is the direction vector.

Since both the points \( A \) and \( M \) lie on the median, they satisfy the line equation. That is,
\[
\mathbf{n}^\top \mathbf{A} = 1, \quad \mathbf{n}^\top \mathbf{M} = 1
\]
or, writing explicitly for the points \( A(2,5) \), \( M(-3,4) \):
\[
\begin{pmatrix}
2 & 5 \\
-3 & 4
\end{pmatrix}
\begin{pmatrix}
n_1 \\ n_2
\end{pmatrix}
=
\begin{pmatrix}
1 \\ 1
\end{pmatrix}
\]

\end{frame}
\begin{frame}[fragile]
    \frametitle{Solution}

We want to find the vector \( \mathbf{n} = \begin{bmatrix} n_1 \\ n_2 \end{bmatrix} \) satisfying the system:
\[
\begin{pmatrix}
2 & 5 \\
-3 & 4
\end{pmatrix} \mathbf{n} = \mathbf{c}
\]

Set up the augmented matrix with right-hand side \(1\):
\[
\left(\begin{array}{cc|c}
2 & 5 & 1 \\
-3 & 4 & 1
\end{array}\right)
\]

Perform row operations:

\[
R_2 \to R_2 + \frac{3}{2} R_1:
\quad
\left(\begin{array}{cc|c}
2 & 5 & 1 \\
0 & \frac{23}{2} & \frac{5}{2}
\end{array}\right)
\]

\[
R_1 \to R_1 - \frac{10}{23} R_2:
\quad
\left(\begin{array}{cc|c}
2 & 0 & 1 - \frac{50}{46} \\
0 & \frac{23}{2} & \frac{5}{2}
\end{array}\right)
\]


\end{frame}
\begin{frame}[fragile]
    \frametitle{Solution}

So the augmented matrix is:
\[
\left(\begin{array}{cc|c}
2 & 0 & -\frac{2}{23} \\
0 & \frac{23}{2} & \frac{5}{2}
\end{array}\right)
\]

Solve the system:

\[
2 n_1 = -\frac{2}{23} \quad \Rightarrow \quad n_1 = -\frac{1}{23}
\]
\[
\frac{23}{2} n_2 = \frac{5}{2} \quad \Rightarrow \quad n_2 = \frac{5}{23}
\]

\[
\mathbf{n} = \frac{1}{23}\begin{bmatrix} -1 \\ 5 \end{bmatrix}
\]
\[
\mathbf{n}^\top \mathbf{x} = 1
\]
Substitute \( \mathbf{n} \):
\[
\left(
\frac{1}{23}
\begin{bmatrix}
-1 \\ 5
\end{bmatrix}
\right)^\top
\mathbf{x} = 1
\]

\end{frame}
\begin{frame}[fragile]
    \frametitle{Solution}

\[
\begin{bmatrix}
-1 & 5
\end{bmatrix}
\mathbf{x} = 23
\]


or equivalently,
\[
5y - x = 23.
\]
Therefore, equation of required line is  :\[
\boxed{5y - x = 23.}
\]

\end{frame}
\section{Graph}
\begin{frame}
    \frametitle{Graph}
    \begin{figure}[htbp]
    \centering
    \includegraphics[width=0.65\linewidth]{FIG/fig1.png}
    \caption{Vector Representation}
    \label{fig:FIG/fig1.png}
\end{figure}
\end{frame}
\section{ C Code}
\begin{frame}[fragile]
\frametitle{C Code }
\begin{lstlisting}[language=C]
#include <stdio.h>
#include "trianglefun.h"

int main() {
    // Vertices of triangle
    int Ax = 2, Ay = 5;
    int Bx = -4, By = 9;
    int Cx = -2, Cy = -1;

    char equation[50];

    // Calculate the median equation and store as string
    median_equation(Ax, Ay, Bx, By, Cx, Cy, equation);

    // Print the equation
    printf("Equation of the median from A: %s\n", equation);

    return 0;
}



    
\end{lstlisting}
\end{frame}


\begin{frame}[fragile]
\frametitle{Python Code for Plotting}
\begin{lstlisting}[language=Python]
import matplotlib.pyplot as plt
import numpy as np

# Vertices of the triangle
A = np.array([2, 5])
B = np.array([-4, 9])
C = np.array([-2, -1])

# Calculate midpoint M of BC
M = (B + C) / 2

# Plot triangle
plt.figure(figsize=(6,6))
triangle_points = np.array([A, B, C, A])
plt.plot(triangle_points[:,0], triangle_points[:,1], 'k-', label='Triangle ABC')

# Plot vertices
plt.plot(A[0], A[1], 'ro')

\end{lstlisting}

\end{frame}
\begin{frame}[fragile]
\frametitle{Python Code for Plotting}
\begin{lstlisting}[language=Python]
plt.plot(B[0], B[1], 'ro')
plt.plot(C[0], C[1], 'ro')

# Label vertices
plt.text(A[0]+0.2, A[1], 'A(2,5)', fontsize=12, color='red')
plt.text(B[0]+0.2, B[1], 'B(-4,9)', fontsize=12, color='red')
plt.text(C[0]+0.2, C[1], 'C(-2,-1)', fontsize=12, color='red')

# Plot median from A to midpoint M
plt.plot([A[0], M[0]], [A[1], M[1]], 'b--', linewidth=2, label='Median AM')

# Label midpoint M
plt.plot(M[0], M[1], 'go')
plt.text(M[0]+0.2, M[1], f'M({M[0]:.1f},{M[1]:.1f})', fontsize=12, color='green')

# Position to place equation on the median line midpoint
mid_x = (A[0] + M[0]) / 2

\end{lstlisting}

\end{frame}
\begin{frame}[fragile]
\frametitle{Python Code for Plotting}
\begin{lstlisting}[language=Python]

mid_y = (A[1] + M[1]) / 2

# Settings
plt.gca().set_aspect('equal', adjustable='box')
plt.grid(True)
plt.legend()
plt.title('Triangle ABC with Median from A')
plt.xlabel('X-axis')
plt.ylabel('Y-axis')
plt.xlim(-6, 4)
plt.ylim(-3, 11)

# Save the figure as PNG
filename = 'triangle_median_eqonline.png'
plt.savefig(filename)
plt.close()
\end{lstlisting}

\end{frame}


\end{document}
