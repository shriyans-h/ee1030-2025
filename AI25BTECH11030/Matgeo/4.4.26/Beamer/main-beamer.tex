\documentclass{beamer}
\mode<presentation>
\usepackage{amsmath}
\usepackage{amssymb}
%\usepackage{advdate}
\usepackage{graphicx}
\usepackage{adjustbox}
\usepackage{subcaption}
\usepackage{enumitem}
\usepackage{multicol}
\usepackage{mathtools}
\usepackage{listings}
\usepackage{url}
\def\UrlBreaks{\do\/\do-}
\usetheme{Boadilla}
\usecolortheme{lily}
\setbeamertemplate{footline}
{
  \leavevmode%
  \hbox{%
  \begin{beamercolorbox}[wd=\paperwidth,ht=2.25ex,dp=1ex,right]{author in head/foot}%
    \insertframenumber{} / \inserttotalframenumber\hspace*{2ex} 
  \end{beamercolorbox}}%
  \vskip0pt%
}
\setbeamertemplate{navigation symbols}{}

\providecommand{\nCr}[2]{\,^{#1}C_{#2}} % nCr
\providecommand{\nPr}[2]{\,^{#1}P_{#2}} % nPr
\providecommand{\mbf}{\mathbf}
\providecommand{\pr}[1]{\ensuremath{\Pr\left(#1\right)}}
\providecommand{\qfunc}[1]{\ensuremath{Q\left(#1\right)}}
\providecommand{\sbrak}[1]{\ensuremath{{}\left[#1\right]}}
\providecommand{\lsbrak}[1]{\ensuremath{{}\left[#1\right.}}
\providecommand{\rsbrak}[1]{\ensuremath{{}\left.#1\right]}}
\providecommand{\brak}[1]{\ensuremath{\left(#1\right)}}
\providecommand{\lbrak}[1]{\ensuremath{\left(#1\right.}}
\providecommand{\rbrak}[1]{\ensuremath{\left.#1\right)}}
\providecommand{\cbrak}[1]{\ensuremath{\left\{#1\right\}}}
\providecommand{\lcbrak}[1]{\ensuremath{\left\{#1\right.}}
\providecommand{\rcbrak}[1]{\ensuremath{\left.#1\right\}}}
\theoremstyle{remark}
\newtheorem{rem}{Remark}
\newcommand{\sgn}{\mathop{\mathrm{sgn}}}
\providecommand{\abs}[1]{\vert#1\vert}
\providecommand{\res}[1]{\Res\displaylimits_{#1}} 
\providecommand{\norm}[1]{\lVert#1\rVert}
\providecommand{\mtx}[1]{\mathbf{#1}}
\providecommand{\mean}[1]{E[ #1 ]}
\providecommand{\fourier}{\overset{\mathcal{F}}{ \rightleftharpoons}}
%\providecommand{\hilbert}{\overset{\mathcal{H}}{ \rightleftharpoons}}
\providecommand{\system}[1]{\overset{\mathcal{#1}}{ \longleftrightarrow}}
%\providecommand{\system}{\overset{\mathcal{H}}{ \longleftrightarrow}}
	%\newcommand{\solution}[2]{\vec{Solution:}{#1}}
%\newcommand{\solution}{\noindent \vec{Solution: }}
\providecommand{\dec}[2]{\ensuremath{\overset{#1}{\underset{#2}{\gtrless}}}}
\newcommand{\myvec}[1]{\ensuremath{\begin{pmatrix}#1\end{pmatrix}}}


\lstset{
%language=C,
frame=single, 
breaklines=true,
columns=fullflexible
}
\lstset{
  language=C,
  basicstyle=\ttfamily\footnotesize,
  keywordstyle=\color{blue}\bfseries,
  commentstyle=\color{gray}\itshape,
  stringstyle=\color{orange},
  numbers=left,
  numberstyle=\tiny\color{gray},
  breaklines=true,
  frame=single,
  showstringspaces=false,
  tabsize=4,
  captionpos=b
}
\numberwithin{equation}{section}
\lstset{
  language=Python,
  basicstyle=\ttfamily\small,
  keywordstyle=\color{blue},
  stringstyle=\color{orange},
  numbers=left,
  numberstyle=\tiny\color{gray},
  breaklines=true,
  showstringspaces=false
}



\title{Problem 2.10.20.}
\author{Sarvesh Tamgade}

\date{\today} 
\begin{document}

\begin{frame}
\titlepage
\end{frame}

\section{Question}
\begin{frame}{Question}
\textbf{Question}:
 Construct a right triangle in which the sides, (other than the hypotenuse) are of length 6 cm and 8 cm.
\end{frame}

\section{Solution}
\begin{frame}[fragile]
    \frametitle{Solution}
Using the section formula, the midpoint \(\mathbf{M}\) of the side \(BC\) is 
\[
\mathbf{M} = \frac{\mathbf{B} + \mathbf{C}}{2} = 
\frac{1}{2} \begin{bmatrix} -4 \\ 9 \end{bmatrix} + 
\frac{1}{2} \begin{bmatrix} -2 \\ -1 \end{bmatrix} = 
\begin{bmatrix} -3 \\ 4 \end{bmatrix}.
\]

The median passes through points \(\mathbf{A}(2,5)\) and \(\mathbf{M}(-3,4)\). The direction vector is 
\[
\mathbf{d} = \mathbf{M} - \mathbf{A} = \begin{bmatrix} -3 - 2 \\ 4 - 5 \end{bmatrix} = \begin{bmatrix} -5 \\ -1 \end{bmatrix}.
\]

Following the matrix approach:
\[
\left( \begin{array}{cc} -5 & -1 \end{array} \right) \mathbf{x} = c,
\]
where \(c\) is found by substituting point \(\mathbf{A}(2,5)\):
\[
-5 \times 2 - 1 \times 5 = -10 - 5 = -15.
\]

Thus, the equation of the median is
\[
-5 x - y = -15,
\]



\end{frame}
\begin{frame}[fragile]
    \frametitle{Solution}
or equivalently,
\[
5x + y = 15.
\]

Therefore, equation of required line is  :\[
\boxed{5x + y = 15}
\]



\end{frame}
\section{Graph}
\begin{frame}
    \frametitle{Graph}
    \begin{figure}[htbp]
    \centering
    \includegraphics[width=0.65\linewidth]{FIG/fig1.png}
    \caption{Vector Representation}
    \label{fig:FIG/fig1.png}
\end{figure}
\end{frame}
\section{ C Code}
\begin{frame}[fragile]
\frametitle{C Code }
\begin{lstlisting}[language=C]
#include <stdio.h>
#include "trianglefun.h"

int main() {
    // Vertices of triangle
    int Ax = 2, Ay = 5;
    int Bx = -4, By = 9;
    int Cx = -2, Cy = -1;

    char equation[50];

    // Calculate the median equation and store as string
    median_equation(Ax, Ay, Bx, By, Cx, Cy, equation);

    // Print the equation
    printf("Equation of the median from A: %s\n", equation);

    return 0;
}



    
\end{lstlisting}
\end{frame}


\begin{frame}[fragile]
\frametitle{Python Code for Plotting}
\begin{lstlisting}[language=Python]
import matplotlib.pyplot as plt
import numpy as np

# Vertices of the triangle
A = np.array([2, 5])
B = np.array([-4, 9])
C = np.array([-2, -1])

# Calculate midpoint M of BC
M = (B + C) / 2

# Plot triangle
plt.figure(figsize=(6,6))
triangle_points = np.array([A, B, C, A])
plt.plot(triangle_points[:,0], triangle_points[:,1], 'k-', label='Triangle ABC')

# Plot vertices
plt.plot(A[0], A[1], 'ro')

\end{lstlisting}

\end{frame}
\begin{frame}[fragile]
\frametitle{Python Code for Plotting}
\begin{lstlisting}[language=Python]
plt.plot(B[0], B[1], 'ro')
plt.plot(C[0], C[1], 'ro')

# Label vertices
plt.text(A[0]+0.2, A[1], 'A(2,5)', fontsize=12, color='red')
plt.text(B[0]+0.2, B[1], 'B(-4,9)', fontsize=12, color='red')
plt.text(C[0]+0.2, C[1], 'C(-2,-1)', fontsize=12, color='red')

# Plot median from A to midpoint M
plt.plot([A[0], M[0]], [A[1], M[1]], 'b--', linewidth=2, label='Median AM')

# Label midpoint M
plt.plot(M[0], M[1], 'go')
plt.text(M[0]+0.2, M[1], f'M({M[0]:.1f},{M[1]:.1f})', fontsize=12, color='green')

# Position to place equation on the median line midpoint
mid_x = (A[0] + M[0]) / 2

\end{lstlisting}

\end{frame}
\begin{frame}[fragile]
\frametitle{Python Code for Plotting}
\begin{lstlisting}[language=Python]

mid_y = (A[1] + M[1]) / 2

# Settings
plt.gca().set_aspect('equal', adjustable='box')
plt.grid(True)
plt.legend()
plt.title('Triangle ABC with Median from A')
plt.xlabel('X-axis')
plt.ylabel('Y-axis')
plt.xlim(-6, 4)
plt.ylim(-3, 11)

# Save the figure as PNG
filename = 'triangle_median_eqonline.png'
plt.savefig(filename)
plt.close()
\end{lstlisting}

\end{frame}


\end{document}
