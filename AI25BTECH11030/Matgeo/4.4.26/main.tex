\let\negmedspace\undefined
\let\negthickspace\undefined
\documentclass[journal]{IEEEtran}
\usepackage[a5paper, margin=10mm, onecolumn]{geometry}
%\usepackage{lmodern} 
\usepackage{tfrupee} 

\setlength{\headheight}{1cm} 
\setlength{\headsep}{0mm}     

\usepackage{gvv-book}
\usepackage{gvv}
\usepackage{cite}
\usepackage{amsmath,amssymb,amsfonts,amsthm}
\usepackage{algorithmic}
\usepackage{graphicx}
\usepackage{textcomp}
\usepackage{xcolor}
\usepackage{txfonts}
\usepackage{listings}
\usepackage{enumitem}
\usepackage{mathtools}
\usepackage{gensymb}
\usepackage{comment}
\usepackage[breaklinks=true]{hyperref}
\usepackage{tkz-euclide} 
\usepackage{listings}                                        
\def\inputGnumericTable{}                                 
\usepackage[latin1]{inputenc}                                
\usepackage{color}                                            
\usepackage{array}                                            
\usepackage{longtable}                                       
\usepackage{calc}                                             
\usepackage{multirow}                                         
\usepackage{hhline}                                           
\usepackage{ifthen}                                           
\usepackage{lscape}

\begin{document}

\bibliographystyle{IEEEtran}
\vspace{3cm}

\title{4.4.26}
\author{AI25BTECH11030 -Sarvesh Tamgade}
{\let\newpage\relax\maketitle}

\renewcommand{\thefigure}{\theenumi}
\renewcommand{\thetable}{\theenumi}
\setlength{\intextsep}{10pt} 


\numberwithin{equation}{enumi}
\numberwithin{figure}{enumi}
\renewcommand{\thetable}{\theenumi}


\textbf{Question}: Find the equation of the median through vertex \(\mathbf{A}\) of the triangle \(ABC\), having vertices
\[
\mathbf{A}(2,5), \quad \mathbf{B}(-4,9), \quad \mathbf{C}(-2,-1).
\]

\textbf{Solution:}\\
Using the section formula, the midpoint \(\mathbf{M}\) of the side \(BC\) is 
\[
\mathbf{M} = \frac{\mathbf{B} + \mathbf{C}}{2} = 
\frac{1}{2} \begin{bmatrix} -4 \\ 9 \end{bmatrix} + 
\frac{1}{2} \begin{bmatrix} -2 \\ -1 \end{bmatrix} = 
\begin{bmatrix} -3 \\ 4 \end{bmatrix}.
\]

The median passes through points \(\mathbf{A}(2,5)\) and \(\mathbf{M}(-3,4)\). The direction vector is 
\[
\mathbf{d} = \mathbf{M} - \mathbf{A} = \begin{bmatrix} -3 - 2 \\ 4 - 5 \end{bmatrix} = \begin{bmatrix} -5 \\ -1 \end{bmatrix}.
\]

Following the matrix approach :
\[
\left( \begin{array}{cc} -5 & -1 \end{array} \right) \mathbf{x} = c,
\]
 where \(c\) is found by substituting point \(\mathbf{A}(2,5)\):
\[
-5 \times 2 - 1 \times 5 = -10 - 5 = -15.
\]

Thus, the equation of the median is
\[
-5 x - y = -15,
\]
or equivalently,
\[
5x + y = 15.
\]

Therefore, equation of required line is  :\[
\boxed{5x + y = 15}
\]

\begin{figure}[htbp]
    \centering
    \includegraphics[width=0.65\linewidth]{FIG/fig1.png}
    \caption{Vector Representation}
    \label{fig:FIG/fig1.png}
    \end{figure}

\end{document}  