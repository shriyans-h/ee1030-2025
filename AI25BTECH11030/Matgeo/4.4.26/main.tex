\let\negmedspace\undefined
\let\negthickspace\undefined
\documentclass[journal]{IEEEtran}
\usepackage[a5paper, margin=10mm, onecolumn]{geometry}
%\usepackage{lmodern} 
\usepackage{tfrupee} 

\setlength{\headheight}{1cm} 
\setlength{\headsep}{0mm}     

\usepackage{gvv-book}
\usepackage{gvv}
\usepackage{cite}
\usepackage{amsmath,amssymb,amsfonts,amsthm}
\usepackage{algorithmic}
\usepackage{graphicx}
\usepackage{textcomp}
\usepackage{xcolor}
\usepackage{txfonts}
\usepackage{listings}
\usepackage{enumitem}
\usepackage{mathtools}
\usepackage{gensymb}
\usepackage{comment}
\usepackage[breaklinks=true]{hyperref}
\usepackage{tkz-euclide} 
\usepackage{listings}                                        
\def\inputGnumericTable{}                                 
\usepackage[latin1]{inputenc}                                
\usepackage{color}                                            
\usepackage{array}                                            
\usepackage{longtable}                                       
\usepackage{calc}                                             
\usepackage{multirow}                                         
\usepackage{hhline}                                           
\usepackage{ifthen}                                           
\usepackage{lscape}

\begin{document}

\bibliographystyle{IEEEtran}
\vspace{3cm}

\title{4.4.26}
\author{AI25BTECH11030 -Sarvesh Tamgade}
{\let\newpage\relax\maketitle}

\renewcommand{\thefigure}{\theenumi}
\renewcommand{\thetable}{\theenumi}
\setlength{\intextsep}{10pt} 


\numberwithin{equation}{enumi}
\numberwithin{figure}{enumi}
\renewcommand{\thetable}{\theenumi}


\textbf{Question}: Find the equation of the median through vertex \(\mathbf{A}\) of the triangle \(ABC\), having vertices
\[
\mathbf{A}(2,5), \quad \mathbf{B}(-4,9), \quad \mathbf{C}(-2,-1).
\]

\textbf{Solution:}\\
Using the section formula, the midpoint \(\mathbf{M}\) of the side \(BC\) is 
\[
\mathbf{M} = \frac{\mathbf{B} + \mathbf{C}}{2} = 
\frac{1}{2} \begin{bmatrix} -4 \\ 9 \end{bmatrix} + 
\frac{1}{2} \begin{bmatrix} -2 \\ -1 \end{bmatrix} = 
\begin{bmatrix} -3 \\ 4 \end{bmatrix}.
\]

The median passes through points \(\mathbf{A}(2,5)\) and \(\mathbf{M}(-3,4)\).
Let the required line have the equation
\[
\mathbf{n}^\top \mathbf{x} = 1
\]
where \( \mathbf{n} = \begin{bmatrix} n_1 \\ n_2 \end{bmatrix} \) is the direction vector.

Since both the points \( A \) and \( M \) lie on the median, they satisfy the line equation. That is,
\[
\mathbf{n}^\top \mathbf{A} = 1, \quad \mathbf{n}^\top \mathbf{M} = 1
\]
or, writing explicitly for the points \( A(2,5) \), \( M(-3,4) \):
\[
\begin{pmatrix}
2 & 5 \\
-3 & 4
\end{pmatrix}
\begin{pmatrix}
n_1 \\ n_2
\end{pmatrix}
=
\begin{pmatrix}
1 \\ 1
\end{pmatrix}
\]
We want to find the vector \( \mathbf{n} = \begin{bmatrix} n_1 \\ n_2 \end{bmatrix} \) satisfying the system:
\[
\begin{pmatrix}
2 & 5 \\
-3 & 4
\end{pmatrix} \mathbf{n} = \mathbf{c}
\]

Set up the augmented matrix with right-hand side \(1\):
\[
\left(\begin{array}{cc|c}
2 & 5 & 1 \\
-3 & 4 & 1
\end{array}\right)
\]

Perform row operations:

\[
R_2 \to R_2 + \frac{3}{2} R_1:
\quad
\left(\begin{array}{cc|c}
2 & 5 & 1 \\
0 & \frac{23}{2} & \frac{5}{2}
\end{array}\right)
\]

\[
R_1 \to R_1 - \frac{10}{23} R_2:
\quad
\left(\begin{array}{cc|c}
2 & 0 & 1 - \frac{50}{46} \\
0 & \frac{23}{2} & \frac{5}{2}
\end{array}\right)
\]



So the augmented matrix is:
\[
\left(\begin{array}{cc|c}
2 & 0 & -\frac{2}{23} \\
0 & \frac{23}{2} & \frac{5}{2}
\end{array}\right)
\]

Solve the system:

\[
2 n_1 = -\frac{2}{23} \quad \Rightarrow \quad n_1 = -\frac{1}{23}
\]
\[
\frac{23}{2} n_2 = \frac{5}{2} \quad \Rightarrow \quad n_2 = \frac{5}{23}
\]

\[
\mathbf{n} = \frac{1}{23}\begin{bmatrix} -1 \\ 5 \end{bmatrix}
\]
\[
\mathbf{n}^\top \mathbf{x} = 1
\]
Substitute \( \mathbf{n} \):
\[
\left(
\frac{1}{23}
\begin{bmatrix}
-1 \\ 5
\end{bmatrix}
\right)^\top
\mathbf{x} = 1
\]

\[
\begin{bmatrix}
-1 & 5
\end{bmatrix}
\mathbf{x} = 23
\]


or equivalently,
\[
5y - x = 23.
\]
Therefore, equation of required line is  :\[
\boxed{5y - x = 23.}
\]

\begin{figure}[htbp]
    \centering
    \includegraphics[width=0.65\linewidth]{FIG/fig1.png}
    \caption{Vector Representation}
    \label{fig:FIG/fig1.png}
    \end{figure}

\end{document}  