\documentclass{beamer}
\mode<presentation>
\usepackage{amsmath}
\usepackage{amssymb}
%\usepackage{advdate}
\usepackage{graphicx}
\usepackage{adjustbox}
\usepackage{subcaption}
\usepackage{enumitem}
\usepackage{multicol}
\usepackage{mathtools}
\usepackage{listings}
\usepackage{url}
\def\UrlBreaks{\do\/\do-}
\usetheme{Boadilla}
\usecolortheme{lily}
\setbeamertemplate{footline}
{
  \leavevmode%
  \hbox{%
  \begin{beamercolorbox}[wd=\paperwidth,ht=2.25ex,dp=1ex,right]{author in head/foot}%
    \insertframenumber{} / \inserttotalframenumber\hspace*{2ex} 
  \end{beamercolorbox}}%
  \vskip0pt%
}
\setbeamertemplate{navigation symbols}{}

\providecommand{\nCr}[2]{\,^{#1}C_{#2}} % nCr
\providecommand{\nPr}[2]{\,^{#1}P_{#2}} % nPr
\providecommand{\mbf}{\mathbf}
\providecommand{\pr}[1]{\ensuremath{\Pr\left(#1\right)}}
\providecommand{\qfunc}[1]{\ensuremath{Q\left(#1\right)}}
\providecommand{\sbrak}[1]{\ensuremath{{}\left[#1\right]}}
\providecommand{\lsbrak}[1]{\ensuremath{{}\left[#1\right.}}
\providecommand{\rsbrak}[1]{\ensuremath{{}\left.#1\right]}}
\providecommand{\brak}[1]{\ensuremath{\left(#1\right)}}
\providecommand{\lbrak}[1]{\ensuremath{\left(#1\right.}}
\providecommand{\rbrak}[1]{\ensuremath{\left.#1\right)}}
\providecommand{\cbrak}[1]{\ensuremath{\left\{#1\right\}}}
\providecommand{\lcbrak}[1]{\ensuremath{\left\{#1\right.}}
\providecommand{\rcbrak}[1]{\ensuremath{\left.#1\right\}}}
\theoremstyle{remark}
\newtheorem{rem}{Remark}
\newcommand{\sgn}{\mathop{\mathrm{sgn}}}
\providecommand{\abs}[1]{\vert#1\vert}
\providecommand{\res}[1]{\Res\displaylimits_{#1}} 
\providecommand{\norm}[1]{\lVert#1\rVert}
\providecommand{\mtx}[1]{\mathbf{#1}}
\providecommand{\mean}[1]{E[ #1 ]}
\providecommand{\fourier}{\overset{\mathcal{F}}{ \rightleftharpoons}}
%\providecommand{\hilbert}{\overset{\mathcal{H}}{ \rightleftharpoons}}
\providecommand{\system}[1]{\overset{\mathcal{#1}}{ \longleftrightarrow}}
%\providecommand{\system}{\overset{\mathcal{H}}{ \longleftrightarrow}}
	%\newcommand{\solution}[2]{\vec{Solution:}{#1}}
%\newcommand{\solution}{\noindent \vec{Solution: }}
\providecommand{\dec}[2]{\ensuremath{\overset{#1}{\underset{#2}{\gtrless}}}}
\newcommand{\myvec}[1]{\ensuremath{\begin{pmatrix}#1\end{pmatrix}}}


\lstset{
%language=C,
frame=single, 
breaklines=true,
columns=fullflexible
}
\lstset{
  language=C,
  basicstyle=\ttfamily\footnotesize,
  keywordstyle=\color{blue}\bfseries,
  commentstyle=\color{gray}\itshape,
  stringstyle=\color{orange},
  numbers=left,
  numberstyle=\tiny\color{gray},
  breaklines=true,
  frame=single,
  showstringspaces=false,
  tabsize=4,
  captionpos=b
}
\numberwithin{equation}{section}
\lstset{
  language=Python,
  basicstyle=\ttfamily\small,
  keywordstyle=\color{blue},
  stringstyle=\color{orange},
  numbers=left,
  numberstyle=\tiny\color{gray},
  breaklines=true,
  showstringspaces=false
}



\title{Problem 2.10.20.}
\author{Sarvesh Tamgade}

\date{\today} 
\begin{document}

\begin{frame}
\titlepage
\end{frame}

\section{Question}
\begin{frame}{Question}
\textbf{Question}:
 Which of the following expressions are meaningful?
\begin{multicols}{2}
\begin{enumerate}[label=(\alph*)]
     
\item $\vec{u} \cdot (\vec{v} \times \vec{w})$
\item $(\vec{u} \cdot \vec{v}) \cdot \vec{w}$
\item $(\vec{u} \cdot \vec{v})\ \vec{w}$
\item $\vec{u} \times (\vec{v} \cdot \vec{w})$

\end{enumerate}
\end{multicols}
\end{frame}

\section{Solution}
\begin{frame}[fragile]
    \frametitle{Solution}
Let $\mathbf{u}$, $\mathbf{v}$, $\mathbf{w}$ be vectors in $\mathbb{R}^3$.

\begin{itemize}
    \item[(a)] $\mathbf{u} \big(\mathbf{v} \times \mathbf{w}\big)$: \\
    The expression $\mathbf{v} \times \mathbf{w}$ is a vector (cross product), and the expression $\mathbf{u} \big(\mathbf{v} \times \mathbf{w}\big)$ denotes the scalar triple product (sometimes written as the inner product of $\mathbf{u}$ and the vector $\mathbf{v} \times \mathbf{w}$).\\
    \textbf{Meaningful.}
    
    \item[(b)] $\big(\mathbf{u} ^\top\mathbf{v}\big) \mathbf{w}$: \\
    Here, $\big(\mathbf{u}^\top \mathbf{v}\big)$ represents the inner (dot) product, which is a scalar. Multiplying a scalar by a vector $\mathbf{w}$ is valid. However, if it is interpreted as $(\mathbf{u}^\top  \mathbf{v}) ^\top \mathbf{w}$ having a dot between scalar and vector, that is not defined.\\
    \textbf{Not meaningful if interpreted as scalar dot vector.}
    
    \item[(c)] $\langle \mathbf{u}^\top  \mathbf{v} \rangle^\top  \mathbf{w}$: \\
    $\langle \mathbf{u}, \mathbf{v} \rangle$ denotes the inner product (a scalar) and multiplying this scalar by vector $\mathbf{w}$ is valid scalar multiplication of a vector.\\
    \textbf{Meaningful.} 
    
\end{itemize}



\end{frame}
\begin{frame}[fragile]
    \frametitle{Solution}
\begin{itemize}
    
    \item[(d)] $\mathbf{u} \times (\mathbf{v}^\top  \mathbf{w})$: \\
    $\mathbf{v}^\top \mathbf{w}$ inside parentheses denotes the inner product (scalar), and cross product between a vector and scalar is undefined.\\
    \textbf{Not meaningful.}
\end{itemize}

\textbf{Answer:}
\boxed{\text{Only (a) and (c) are meaningful }}


\end{frame}
\section{Graph}
\begin{frame}
    \frametitle{Graph}
    \begin{figure}[htbp]
    \centering
    \includegraphics[width=0.65\linewidth]{FIG/fig1.png}
    \caption{Vector Representation}
    \label{fig:FIG/fig1.png}
\end{figure}
\end{frame}
\section{ C Code}
\begin{frame}[fragile]
\frametitle{C Code }
\begin{lstlisting}[language=C]
#include <stdio.h>
#include "matfun.h"

void print_vector(const char* name, const double v[3]) {
    printf("%s = (%.2f, %.2f, %.2f)\n", name, v[0], v[1], v[2]);
}

int main() {
    double u[3] = {1, 2, 3};
    double v[3] = {4, 5, 6};
    double w[3] = {7, 8, 9};

    double cross_vw[3];
    cross_product(v, w, cross_vw);

    double dot_u_crossvw = dot_product(u, cross_vw);
    printf("u · (v × w) = %.2f\n", dot_u_crossvw);


   
    
\end{lstlisting}
\end{frame}
\begin{frame}[fragile]
\frametitle{C Code }
\begin{lstlisting}[language=C]
 double dot_uv = dot_product(u, v);
    printf("(u · v) = %.2f\n", dot_uv);

    printf("(u · v) · w is NOT meaningful as dot product of scalar and vector.\n");

    printf("(u · v) * w (scalar multiplication) = (%.2f, %.2f, %.2f)\n",
           dot_uv * w[0], dot_uv * w[1], dot_uv * w[2]);

    printf("v · w = %.2f\n", dot_product(v, w));
    printf("u × (v · w) is NOT meaningful as cross product of vector and scalar.\n");

    return 0;
}


\end{lstlisting}
\end{frame}

\begin{frame}[fragile]
\frametitle{Python Code for Plotting}
\begin{lstlisting}[language=Python]
import matplotlib.pyplot as plt
import numpy as np

# Define the three points
points = np.array([[1, -1], [0, 5], [3, 2]])

# Extract x and y coordinates
x = points[:, 0]
y = points[:, 1]

# Plot the points
plt.plot(x, y, 'ro')

# Annotate the points
for i, (xi, yi) in enumerate(points):
    plt.text(xi + 0.1, yi, f'({xi},{yi})')



\end{lstlisting}

\end{frame}
\begin{frame}[fragile]
\frametitle{Python Code for Plotting}
\begin{lstlisting}[language=Python]   
# Draw the triangle by connecting points and closing the loop
triangle = plt.Polygon(points, closed=True, fill=True, color='cyan', alpha=0.3)
plt.gca().add_patch(triangle)

# Set limits
plt.xlim(min(x)-1, max(x)+1)
plt.ylim(min(y)-1, max(y)+1)

# Title and labels
plt.title('Triangle formed by points')
plt.xlabel('X-axis')
plt.ylabel('Y-axis')

# Save the figure
plt.savefig('triangle_area.png')

plt.show()
\end{lstlisting}

\end{frame}

\end{document}
