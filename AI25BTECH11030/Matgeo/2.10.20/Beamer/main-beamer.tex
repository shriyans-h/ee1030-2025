\documentclass{beamer}
\mode<presentation>
\usepackage{amsmath}
\usepackage{amssymb}
%\usepackage{advdate}
\usepackage{graphicx}
\usepackage{adjustbox}
\usepackage{subcaption}
\usepackage{enumitem}
\usepackage{multicol}
\usepackage{mathtools}
\usepackage{listings}
\usepackage{url}
\def\UrlBreaks{\do\/\do-}
\usetheme{Boadilla}
\usecolortheme{lily}
\setbeamertemplate{footline}
{
  \leavevmode%
  \hbox{%
  \begin{beamercolorbox}[wd=\paperwidth,ht=2.25ex,dp=1ex,right]{author in head/foot}%
    \insertframenumber{} / \inserttotalframenumber\hspace*{2ex} 
  \end{beamercolorbox}}%
  \vskip0pt%
}
\setbeamertemplate{navigation symbols}{}

\providecommand{\nCr}[2]{\,^{#1}C_{#2}} % nCr
\providecommand{\nPr}[2]{\,^{#1}P_{#2}} % nPr
\providecommand{\mbf}{\mathbf}
\providecommand{\pr}[1]{\ensuremath{\Pr\left(#1\right)}}
\providecommand{\qfunc}[1]{\ensuremath{Q\left(#1\right)}}
\providecommand{\sbrak}[1]{\ensuremath{{}\left[#1\right]}}
\providecommand{\lsbrak}[1]{\ensuremath{{}\left[#1\right.}}
\providecommand{\rsbrak}[1]{\ensuremath{{}\left.#1\right]}}
\providecommand{\brak}[1]{\ensuremath{\left(#1\right)}}
\providecommand{\lbrak}[1]{\ensuremath{\left(#1\right.}}
\providecommand{\rbrak}[1]{\ensuremath{\left.#1\right)}}
\providecommand{\cbrak}[1]{\ensuremath{\left\{#1\right\}}}
\providecommand{\lcbrak}[1]{\ensuremath{\left\{#1\right.}}
\providecommand{\rcbrak}[1]{\ensuremath{\left.#1\right\}}}
\theoremstyle{remark}
\newtheorem{rem}{Remark}
\newcommand{\sgn}{\mathop{\mathrm{sgn}}}
\providecommand{\abs}[1]{\vert#1\vert}
\providecommand{\res}[1]{\Res\displaylimits_{#1}} 
\providecommand{\norm}[1]{\lVert#1\rVert}
\providecommand{\mtx}[1]{\mathbf{#1}}
\providecommand{\mean}[1]{E[ #1 ]}
\providecommand{\fourier}{\overset{\mathcal{F}}{ \rightleftharpoons}}
%\providecommand{\hilbert}{\overset{\mathcal{H}}{ \rightleftharpoons}}
\providecommand{\system}[1]{\overset{\mathcal{#1}}{ \longleftrightarrow}}
%\providecommand{\system}{\overset{\mathcal{H}}{ \longleftrightarrow}}
	%\newcommand{\solution}[2]{\vec{Solution:}{#1}}
%\newcommand{\solution}{\noindent \vec{Solution: }}
\providecommand{\dec}[2]{\ensuremath{\overset{#1}{\underset{#2}{\gtrless}}}}
\newcommand{\myvec}[1]{\ensuremath{\begin{pmatrix}#1\end{pmatrix}}}


\lstset{
%language=C,
frame=single, 
breaklines=true,
columns=fullflexible
}
\lstset{
  language=C,
  basicstyle=\ttfamily\footnotesize,
  keywordstyle=\color{blue}\bfseries,
  commentstyle=\color{gray}\itshape,
  stringstyle=\color{orange},
  numbers=left,
  numberstyle=\tiny\color{gray},
  breaklines=true,
  frame=single,
  showstringspaces=false,
  tabsize=4,
  captionpos=b
}
\numberwithin{equation}{section}
\lstset{
  language=Python,
  basicstyle=\ttfamily\small,
  keywordstyle=\color{blue},
  stringstyle=\color{orange},
  numbers=left,
  numberstyle=\tiny\color{gray},
  breaklines=true,
  showstringspaces=false
}



\title{Problem 2.10.20.}
\author{Sarvesh Tamgade}

\date{\today} 
\begin{document}

\begin{frame}
\titlepage
\end{frame}
\section{Question}
\begin{frame}{Question}
\textbf{Question}:
 Which of the following expressions are meaningful?
\begin{multicols}{2}
\begin{enumerate}[label=(\alph*)]
     
\item $\mathbf{u} \cdot (\mathbf{v} \times \mathbf{w})$
\item $(\mathbf{u} \cdot \mathbf{v}) \cdot \mathbf{w}$
\item $(\mathbf{u} \cdot \mathbf{v})\ \mathbf{w}$
\item $\mathbf{u} \times (\mathbf{v} \cdot \mathbf{w})$

\end{enumerate}
\end{multicols}
\end{frame}

\section{Solution}
\begin{frame}[fragile]
    \frametitle{Solution}
Let $\mathbf{u}$, $\mathbf{v}$, $\mathbf{w}$ be vectors in $\mathbb{R}^3$.

   Let \begin{align}
    \mathbf{u} &= \begin{bmatrix} 2 \\ 3 \end{bmatrix}, \quad
  \mathbf{v} = \begin{bmatrix} 4 \\ 1 \end{bmatrix}, \quad
  \mathbf{w} = \begin{bmatrix} 0 \\ 5 \end{bmatrix}.
  \\
\end{align}
\begin{enumerate}[label=\alph*)]
  \item \( \mathbf{u}^\top (\mathbf{v} \times \mathbf{w}) \)

\[
\mathbf{v} \times \mathbf{w} =
\begin{pmatrix}
v_{23} & w_{23} \\
v_{31} & w_{31} \\
v_{12} & w_{12}
\end{pmatrix}
=
\begin{pmatrix}
v_2 w_3 - v_3 w_2 \\
v_3 w_1 - v_1 w_3 \\
v_1 w_2 - v_2 w_1
\end{pmatrix}
=
\begin{pmatrix}
1 \times 0 - 0 \times 5 \\
0 \times 0 - 4 \times 0 \\
4 \times 5 - 1 \times 0
\end{pmatrix}
\]
\[
=
\begin{pmatrix}
0 \\
0 \\
20
\end{pmatrix}
\]

\end{enumerate}
\end{frame}
\begin{frame}[fragile]
    \frametitle{Solution}
\begin{enumerate}[label=\alph*) ]
\setcounter{enumi}{1} 
    \[
\mathbf{u}^\top (\mathbf{v} \times \mathbf{w}) =
\begin{bmatrix} 2 & 3 & 0 \end{bmatrix}
\begin{bmatrix}
0 \\
0 \\
20
\end{bmatrix} = 0
\]
Since the scalar (dot) product of two vectors is defined, the expression  \(\mathbf{u} ^\top(\mathbf{v} \times \mathbf{w})\) is meaningful.
  \item \( (\mathbf{u}^\top \mathbf{v})^\top \mathbf{w} \)

  \begin{align*}
  \mathbf{u}^\top \mathbf{v} &= \begin{bmatrix} 2 & 3 \end{bmatrix} \begin{bmatrix} 4 \\ 1 \end{bmatrix} = 2 \times 4 + 3 \times 1 = 11,
  \\
  (\mathbf{u}^\top \mathbf{v})^\top \mathbf{w} &= 11^\top \mathbf{w} \quad \text{(scalar dot vector)} \quad \text{undefined}.
  \end{align*}

  \item \( (\mathbf{u}^\top \mathbf{v}) \mathbf{w} \)

  \begin{align*}
  (\mathbf{u}^\top \mathbf{v}) \mathbf{w} &= 11 \times \begin{bmatrix} 0 \\ 5 \end{bmatrix} = \begin{bmatrix} 0 \\ 55 \end{bmatrix}.
  \end{align*}
\end{enumerate}

\end{frame}
\begin{frame}[fragile]
    \frametitle{Solution}
\begin{enumerate}[label=\alph*) ]
\setcounter{enumi}{3} 
 

  This is meaningful scalar multiplication.

  \item \( \mathbf{u} \times (\mathbf{v}^\top \mathbf{w}) \)

  \begin{align*}
  \mathbf{v}^\top \mathbf{w} &= \begin{bmatrix} 4 & 1 \end{bmatrix} \begin{bmatrix} 0 \\ 5 \end{bmatrix} = 0 + 5 = 5,
  \\
  \mathbf{u} \times 5 &= \text{cross product of vector and scalar -- undefined}.
  \end{align*}
  
\end{enumerate}

\textbf{Answer:}
\boxed{\text{Only (a) and (c) are meaningful }}


\end{frame}
\section{Graph}
\begin{frame}
    \frametitle{Graph}
    \begin{figure}[htbp]
    \centering
    \includegraphics[width=0.65\linewidth]{FIG/fig1.png}
    \caption{Vector Representation}
    \label{fig:FIG/fig1.png}
\end{figure}
\end{frame}
\section{ C Code}
\begin{frame}[fragile]
\frametitle{C Code }
\begin{lstlisting}[language=C]
#include <stdio.h>
#include "matfun.h"

void print_vector(const char* name, const double v[3]) {
    printf("%s = (%.2f, %.2f, %.2f)\n", name, v[0], v[1], v[2]);
}

int main() {
    double u[3] = {1, 2, 3};
    double v[3] = {4, 5, 6};
    double w[3] = {7, 8, 9};

    double cross_vw[3];
    cross_product(v, w, cross_vw);

    double dot_u_crossvw = dot_product(u, cross_vw);
    printf("u · (v × w) = %.2f\n", dot_u_crossvw);


   
    
\end{lstlisting}
\end{frame}
\begin{frame}[fragile]
\frametitle{C Code }
\begin{lstlisting}[language=C]
 double dot_uv = dot_product(u, v);
    printf("(u · v) = %.2f\n", dot_uv);

    printf("(u · v) · w is NOT meaningful as dot product of scalar and vector.\n");

    printf("(u · v) * w (scalar multiplication) = (%.2f, %.2f, %.2f)\n",
           dot_uv * w[0], dot_uv * w[1], dot_uv * w[2]);

    printf("v · w = %.2f\n", dot_product(v, w));
    printf("u × (v · w) is NOT meaningful as cross product of vector and scalar.\n");

    return 0;
}


\end{lstlisting}
\end{frame}

\begin{frame}[fragile]
\frametitle{Python Code for Plotting}
\begin{lstlisting}[language=Python]
import matplotlib.pyplot as plt
import numpy as np

# Vectors u, v, w in 2D (using first two components)
u = np.array([2, 3])
v = np.array([4, 1])
w = np.array([0, 5])

# Origin point
origin = np.array([0, 0])

# Plotting the vectors
plt.figure(figsize=(7, 7))
plt.quiver(*origin, *u, angles='xy', scale_units='xy', scale=1, color='red', label='u = [2, 3]')
plt.quiver(*origin, *v, angles='xy', scale_units='xy', scale=1, color='green', label='v = [4, 1]')
plt.quiver(*origin, *w, angles='xy', scale_units='xy', scale=1, color='blue', label='w = [0, 5]')

\end{lstlisting}

\end{frame}
\begin{frame}[fragile]
\frametitle{Python Code for Plotting}
\begin{lstlisting}[language=Python]
# Setting the limits
plt.xlim(-1, 5)
plt.ylim(-1, 6)

# Adding labels and title
plt.xlabel('X')
plt.ylabel('Y')
plt.title('2D vectors: u, v, w')
plt.grid()
plt.legend()
plt.gca().set_aspect('equal')

# Save the figure as a PNG file
plt.savefig('2D_vectors.png')
plt.close()

\end{lstlisting}

\end{frame}

\end{document}
