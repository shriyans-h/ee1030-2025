\let\negmedspace\undefined
\let\negthickspace\undefined
\documentclass[journal]{IEEEtran}
\usepackage[a5paper, margin=10mm, onecolumn]{geometry}
%\usepackage{lmodern} 
\usepackage{tfrupee} 

\setlength{\headheight}{1cm} 
\setlength{\headsep}{0mm}     

\usepackage{gvv-book}
\usepackage{gvv}
\usepackage{cite}
\usepackage{amsmath,amssymb,amsfonts,amsthm}
\usepackage{algorithmic}
\usepackage{graphicx}
\usepackage{textcomp}
\usepackage{xcolor}
\usepackage{txfonts}
\usepackage{listings}
\usepackage{enumitem}
\usepackage{mathtools}
\usepackage{gensymb}
\usepackage{comment}
\usepackage[breaklinks=true]{hyperref}
\usepackage{tkz-euclide} 
\usepackage{listings}                                        
\def\inputGnumericTable{}                                 
\usepackage[latin1]{inputenc}                                
\usepackage{color}                                            
\usepackage{array}                                            
\usepackage{longtable}                                       
\usepackage{calc}                                             
\usepackage{multirow}                                         
\usepackage{hhline}                                           
\usepackage{ifthen}                                           
\usepackage{lscape}

\begin{document}

\bibliographystyle{IEEEtran}
\vspace{3cm}

\title{4.12.44}
\author{AI25BTECH11030 -Sarvesh Tamgade}
{\let\newpage\relax\maketitle}

\renewcommand{\thefigure}{\theenumi}
\renewcommand{\thetable}{\theenumi}
\setlength{\intextsep}{10pt} 


\numberwithin{equation}{enumi}
\numberwithin{figure}{enumi}
\renewcommand{\thetable}{\theenumi}


\textbf{Question}: Find the equation of the set of points which are equidistant from the points \(\vec{A} = \myvec{1 \\ 2 \\ 3}\) and \(\vec{B} = \myvec{3 \\ 2 \\ -1}\).

\textbf{Solution}:
Let \(\vec{X} = \myvec{a \\ b \\ c}\) be the position vector of any point equidistant from \(\vec{A}\) and \(\vec{B}\).

The condition for \(\vec{X}\) to be equidistant is:
\begin{equation}
\|\vec{X} - \vec{A}\| = \|\vec{X} - \vec{B}\|
\end{equation}

Squaring both sides we get:
\begin{equation}
(\vec{X} - \vec{A})^\top (\vec{X} - \vec{A}) = (\vec{X} - \vec{B})^\top (\vec{X} - \vec{B})
\end{equation}

Expanding,
\begin{equation}
\vec{X}^\top \vec{X} - 2 \vec{A}^\top \vec{X} + \vec{A}^\top \vec{A} = \vec{X}^\top \vec{X} - 2 \vec{B}^\top \vec{X} + \vec{B}^\top \vec{B}
\end{equation}

Simplifying,
\begin{equation}
-2 \vec{A}^\top \vec{X} + \vec{A}^\top \vec{A} = -2 \vec{B}^\top \vec{X} + \vec{B}^\top \vec{B}
\end{equation}

Rearranging,
\begin{equation}
2(\vec{B} - \vec{A})^\top \vec{X} = \vec{B}^\top \vec{B} - \vec{A}^\top \vec{A}
\end{equation}

Calculate \(\vec{B} - \vec{A}\):
\begin{equation}
\vec{B} - \vec{A} = \myvec{3 - 1 \\ 2 - 2 \\ -1 - 3} = \myvec{2 \\ 0 \\ -4}
\end{equation}

Calculate \(\vec{B}^\top \vec{B}\) and \(\vec{A}^\top \vec{A}\):
\begin{equation}
\vec{B}^\top \vec{B} = 3^2 + 2^2 + (-1)^2 = 14, \quad
\vec{A}^\top \vec{A} = 1^2 + 2^2 + 3^2 = 14
\end{equation}

Thus,
\begin{equation}
2 \myvec{2 & 0 & -4} \myvec{a \\ b \\ c} = 14 - 14 = 0
\end{equation}

Simplifying,
\begin{equation}
\myvec{4 & 0 & -8} \myvec{a \\ b \\ c} = 0
\end{equation}

This matrix equation represents the plane:
\begin{equation}
4a - 8c = 0
\end{equation}

or equivalently,
\begin{equation}
a - 2c = 0
\end{equation}

\textbf{Final Answer:} The set of points equidistant from \(\vec{A}\) and \(\vec{B}\) lies on the plane defined by
\[
\boxed{
\myvec{4 & 0 & -8} \vec{x} = 0
}
\]


\begin{figure}[htbp]
    \centering
    \includegraphics[width=0.65\linewidth]{FIG/fig1.png}
    \caption{Vector Representation}
    \label{fig:FIG/fig1.png}
    \end{figure}

\end{document}  