\documentclass{beamer}
\usepackage[utf8]{inputenc}

\usetheme{Madrid}
\usecolortheme{default}
\usepackage{amsmath,amssymb,amsfonts,amsthm}
\usepackage{txfonts}
\usepackage{tkz-euclide}
\usepackage{listings}
\usepackage{adjustbox}
\usepackage{array}
\usepackage{tabularx}
\usepackage{gvv}
\usepackage{lmodern}
\usepackage{circuitikz}
\usepackage{tikz}
\usepackage{graphicx}
\usepackage{multicol}

\setbeamertemplate{page number in head/foot}[totalframenumber]

\usepackage{tcolorbox}
\tcbuselibrary{minted,breakable,xparse,skins}
{
  \hbox{%
  \begin{beamercolorbox}[wd=\paperwidth,ht=2.25ex,dp=1ex,right]{author in head/foot}%
    \insertframenumber{} / \inserttotalframenumber\hspace*{2ex} 
  \end{beamercolorbox}}%
  \vskip0pt%
}
\setbeamertemplate{navigation symbols}{}

\providecommand{\nCr}[2]{\,^{#1}C_{#2}} % nCr
\providecommand{\nPr}[2]{\,^{#1}P_{#2}} % nPr
\providecommand{\mbf}{\mathbf}
\providecommand{\pr}[1]{\ensuremath{\Pr\left(#1\right)}}
\providecommand{\qfunc}[1]{\ensuremath{Q\left(#1\right)}}
\providecommand{\sbrak}[1]{\ensuremath{{}\left[#1\right]}}
\providecommand{\lsbrak}[1]{\ensuremath{{}\left[#1\right.}}
\providecommand{\rsbrak}[1]{\ensuremath{{}\left.#1\right]}}
\providecommand{\brak}[1]{\ensuremath{\left(#1\right)}}
\providecommand{\lbrak}[1]{\ensuremath{\left(#1\right.}}
\providecommand{\rbrak}[1]{\ensuremath{\left.#1\right)}}
\providecommand{\cbrak}[1]{\ensuremath{\left\{#1\right\}}}
\providecommand{\lcbrak}[1]{\ensuremath{\left\{#1\right.}}
\providecommand{\rcbrak}[1]{\ensuremath{\left.#1\right\}}}
\theoremstyle{remark}
\newtheorem{rem}{Remark}
\newcommand{\sgn}{\mathop{\mathrm{sgn}}}
\providecommand{\abs}[1]{\vert#1\vert}
\providecommand{\res}[1]{\Res\displaylimits_{#1}} 
\providecommand{\norm}[1]{\lVert#1\rVert}
\providecommand{\mtx}[1]{\mathbf{#1}}
\providecommand{\mean}[1]{E[ #1 ]}
\providecommand{\fourier}{\overset{\mathcal{F}}{ \rightleftharpoons}}
%\providecommand{\hilbert}{\overset{\mathcal{H}}{ \rightleftharpoons}}
\providecommand{\system}[1]{\overset{\mathcal{#1}}{ \longleftrightarrow}}
%\providecommand{\system}{\overset{\mathcal{H}}{ \longleftrightarrow}}
	%\newcommand{\solution}[2]{\vec{Solution:}{#1}}
%\newcommand{\solution}{\noindent \vec{Solution: }}
\providecommand{\dec}[2]{\ensuremath{\overset{#1}{\underset{#2}{\gtrless}}}}
\newcommand{\myvec}[1]{\ensuremath{\begin{pmatrix}#1\end{pmatrix}}}


\lstset{
%language=C,
frame=single, 
breaklines=true,
columns=fullflexible
}
\lstset{
  language=C,
  basicstyle=\ttfamily\footnotesize,
  keywordstyle=\color{blue}\bfseries,
  commentstyle=\color{gray}\itshape,
  stringstyle=\color{orange},
  numbers=left,
  numberstyle=\tiny\color{gray},
  breaklines=true,
  frame=single,
  showstringspaces=false,
  tabsize=4,
  captionpos=b
}
\numberwithin{equation}{section}
\lstset{
  language=Python,
  basicstyle=\ttfamily\small,
  keywordstyle=\color{blue},
  stringstyle=\color{orange},
  numbers=left,
  numberstyle=\tiny\color{gray},
  breaklines=true,
  showstringspaces=false
}

\title{Problem 2.7.2}
\author{Sarvesh Tamgade}

\date{\today} 
\begin{document}

\begin{frame}
\titlepage
\end{frame}

\section*{Outline}
\begin{frame}
\tableofcontents
\end{frame}
\section{Question}
\begin{frame}{Question}
\textbf{Question}:
\centering
The area of a triangle with vertices A(-1,1), B(0,5) and C(3,2) is?
\end{frame}

\section{Solution}
\begin{frame}[fragile]
    \frametitle{Solution}
Given: $A(-1,1),\; B(0,5),\; C(3,2).$\\

$
\vec{B}-\vec{A}=\myvec{0-(-1)\\5-1}
=\myvec{1\\4},\qquad
\vec{C}-\vec{A}=\myvec{3-(-1)\\2-1} = \myvec{4\\1}.
$\\

$\norm{\vec{(B-A)} \times \vec{(C-A)}} = \norm{\,\myvec{|\vec{A_{23}} & \vec{B_{23}}| \\ |\vec{A_{31}} & \vec{B_{31}}| \\ |\vec{A_{12}} & \vec{B_{12}}|}\,} = 7.5 $\\\\


$
\text{Area}=\frac{1}{2}\norm{\vec{(B-A)} \times \vec{(C-A)}} = 7.5
$
\textbf{Answer:}


\begin{align}
    \centering
    \boxed{Area \, of \, Triangle \, ABC \,= \,7.5\,sq.units}
\end{align}

\end{frame}
\begin{frame}
    \frametitle{Graph}
    \begin{figure}[htbp]
    \centering
    \includegraphics[width=0.65\linewidth]{FIG/fig1.png}
    \caption{Vector Representation}
    \label{fig:FIG/fig1.png}
\end{figure}
\end{frame}
\section{ C Code}
\begin{frame}[fragile]
\frametitle{C Code }
\begin{lstlisting}[language=C]
#include <stdio.h>
#include <stdlib.h>
#include <math.h>
#ifndef M_PI
#define M_PI 3.14159265358979323846
#endif
#include "matfun.h"

int main(void) {
    // Allocate 2x1 matrices for points
    double **A = createMat(2,1);
    double **B = createMat(2,1);
    double **C = createMat(2,1);

    // Set points: A(-1,1), B(0,5), C(3,2)
    A[0][0] = -1.0; A[1][0] = 1.0;
    B[0][0] = 0.0; B[1][0] = 5.0;
    C[0][0] = 3.0; C[1][0] = 2.0;

   
    
\end{lstlisting}
\end{frame}
\begin{frame}[fragile]
\frametitle{C Code }
\begin{lstlisting}[language=C]
 // Vectors B-A and C-A
    double **BA = Matsub(B, A, 2, 1);
    double **CA = Matsub(C, A, 2, 1);

    // Extract components
    double BAx = BA[0][0], BAy = BA[1][0];
    double CAx = CA[0][0], CAy = CA[1][0];

    // Cross product magnitude |(B-A) x (C-A)| = |BAx*CAy - BAy*CAx|
    double cp = fabs(BAx*CAy - BAy*CAx);
    double area = 0.5 * cp; // Triangle area

    // Save to points.dat
    FILE *fp = fopen("points.dat", "w");
    if (!fp) {
        perror("points.dat");
        freeMat(BA, 2); freeMat(CA, 2);



\end{lstlisting}
\end{frame}
\begin{frame}[fragile]
\frametitle{C Code }
\begin{lstlisting}[language=C]
        freeMat(A, 2); freeMat(B, 2); freeMat(C, 2);
        return 1;
    }
    fprintf(fp, "# Point_Name X Y\n");
    fprintf(fp, "A %.1f %.1f\n", A[0][0], A[1][0]);
    fprintf(fp, "B %.1f %.1f\n", B[0][0], B[1][0]);
    fprintf(fp, "C %.1f %.1f\n", C[0][0], C[1][0]);
    fclose(fp);
    printf("Wrote points.dat\n");
    printf("Triangle area = %.2f\n", area);

    // Clean-up
    freeMat(BA, 2); freeMat(CA, 2);
    freeMat(A, 2); freeMat(B, 2); freeMat(C, 2);
    return 0;
}

\end{lstlisting}
\end{frame}
\begin{frame}[fragile]
\frametitle{Python Code for Plotting}
\begin{lstlisting}[language=Python]
import matplotlib.pyplot as plt
import numpy as np

# Define the three points
points = np.array([[1, -1], [0, 5], [3, 2]])

# Extract x and y coordinates
x = points[:, 0]
y = points[:, 1]

# Plot the points
plt.plot(x, y, 'ro')

# Annotate the points
for i, (xi, yi) in enumerate(points):
    plt.text(xi + 0.1, yi, f'({xi},{yi})')



\end{lstlisting}

\end{frame}
\s                                                                    ection{Python Code}
\begin{frame}[fragile]
\frametitle{Python Code for Plotting}
\begin{lstlisting}[language=Python]   
# Draw the triangle by connecting points and closing the loop
triangle = plt.Polygon(points, closed=True, fill=True, color='cyan', alpha=0.3)
plt.gca().add_patch(triangle)

# Set limits
plt.xlim(min(x)-1, max(x)+1)
plt.ylim(min(y)-1, max(y)+1)

# Title and labels
plt.title('Triangle formed by points')
plt.xlabel('X-axis')
plt.ylabel('Y-axis')

# Save the figure
plt.savefig('triangle_area.png')

plt.show()
\end{lstlisting}

\end{frame}

\end{document}
