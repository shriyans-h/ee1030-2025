\documentclass{beamer}
\mode<presentation>
\usepackage{amsmath}
\usepackage{amssymb}
%\usepackage{advdate}
\usepackage{graphicx}
\usepackage{adjustbox}
\usepackage{subcaption}
\usepackage{enumitem}
\usepackage{multicol}
\usepackage{mathtools}
\usepackage{listings}
\usepackage{url}
\def\UrlBreaks{\do\/\do-}
\usetheme{Boadilla}
\usecolortheme{lily}
\setbeamertemplate{footline}
{
  \leavevmode%
  \hbox{%
  \begin{beamercolorbox}[wd=\paperwidth,ht=2.25ex,dp=1ex,right]{author in head/foot}%
    \insertframenumber{} / \inserttotalframenumber\hspace*{2ex} 
  \end{beamercolorbox}}%
  \vskip0pt%
}
\setbeamertemplate{navigation symbols}{}

\providecommand{\nCr}[2]{\,^{#1}C_{#2}} % nCr
\providecommand{\nPr}[2]{\,^{#1}P_{#2}} % nPr
\providecommand{\mbf}{\mathbf}
\providecommand{\pr}[1]{\ensuremath{\Pr\left(#1\right)}}
\providecommand{\qfunc}[1]{\ensuremath{Q\left(#1\right)}}
\providecommand{\sbrak}[1]{\ensuremath{{}\left[#1\right]}}
\providecommand{\lsbrak}[1]{\ensuremath{{}\left[#1\right.}}
\providecommand{\rsbrak}[1]{\ensuremath{{}\left.#1\right]}}
\providecommand{\brak}[1]{\ensuremath{\left(#1\right)}}
\providecommand{\lbrak}[1]{\ensuremath{\left(#1\right.}}
\providecommand{\rbrak}[1]{\ensuremath{\left.#1\right)}}
\providecommand{\cbrak}[1]{\ensuremath{\left\{#1\right\}}}
\providecommand{\lcbrak}[1]{\ensuremath{\left\{#1\right.}}
\providecommand{\rcbrak}[1]{\ensuremath{\left.#1\right\}}}
\theoremstyle{remark}
\newtheorem{rem}{Remark}
\newcommand{\sgn}{\mathop{\mathrm{sgn}}}
\providecommand{\abs}[1]{\vert#1\vert}
\providecommand{\res}[1]{\Res\displaylimits_{#1}} 
\providecommand{\norm}[1]{\lVert#1\rVert}
\providecommand{\mtx}[1]{\mathbf{#1}}
\providecommand{\mean}[1]{E[ #1 ]}
\providecommand{\fourier}{\overset{\mathcal{F}}{ \rightleftharpoons}}
%\providecommand{\hilbert}{\overset{\mathcal{H}}{ \rightleftharpoons}}
\providecommand{\system}[1]{\overset{\mathcal{#1}}{ \longleftrightarrow}}
%\providecommand{\system}{\overset{\mathcal{H}}{ \longleftrightarrow}}
	%\newcommand{\solution}[2]{\vec{Solution:}{#1}}
%\newcommand{\solution}{\noindent \vec{Solution: }}
\providecommand{\dec}[2]{\ensuremath{\overset{#1}{\underset{#2}{\gtrless}}}}
\newcommand{\myvec}[1]{\ensuremath{\begin{pmatrix}#1\end{pmatrix}}}

\lstset{
%language=C,
frame=single, 
breaklines=true,
columns=fullflexible
}

\numberwithin{equation}{section}

\lstset{
  language=Python,
  basicstyle=\ttfamily\small,
  keywordstyle=\color{blue},
  stringstyle=\color{orange},
  numbers=left,
  numberstyle=\tiny\color{gray},
  breaklines=true,
  showstringspaces=false
}
\usepackage{listings}
\usepackage{xcolor}

\lstset{
  language=C,
  basicstyle=\ttfamily\footnotesize,
  keywordstyle=\color{blue}\bfseries,
  commentstyle=\color{gray}\itshape,
  stringstyle=\color{orange},
  numbers=left,
  numberstyle=\tiny\color{gray},
  breaklines=true,
  frame=single,
  showstringspaces=false,
  tabsize=4,
  captionpos=b
}

\numberwithin{equation}{section}
\title{1.3.9}
\author{AI25BTECH11030 - SARVESH TAMGADE}
% \maketitle
% \newpage
% \bigskip
\begin{document}

\begin{frame}
\titlepage
\end{frame}

\begin{frame}
\frametitle{Question}

Find the coordinates of a point on Y axis which is at a distance of \( 5\sqrt{2} \) from the point \( P(3, -2, 5) \).

\end{frame}

\section{Solution}

\begin{frame}
\frametitle{Solution}

Let the required point on Y axis be \(\vec{Q} = \begin{pmatrix}0 \\ y \\ 0\end{pmatrix}\).

So,
\[
\vec{P} - \vec{Q} = 
\begin{pmatrix} 3 \\ -2 \\ 5 \end{pmatrix} - 
\begin{pmatrix} 0 \\ y \\ 0 \end{pmatrix} = 
\begin{pmatrix} 3 \\ -2 - y \\ 5 \end{pmatrix}
\]

The desired distance is:
\[
d = \|\vec{P} - \vec{Q}\| = 5\sqrt{2}
\]

So,
\[
(\vec{P} - \vec{Q})^T (\vec{P} - \vec{Q}) = 3^2 + (-2 - y)^2 + 5^2 = 9 + (y + 2)^2 + 25
\]
\[
9 + (y+2)^2 + 25 = 50
\]
\[
(y+2)^2 = 16
\]
\[
y + 2 = \pm 4
\]
\[
\text{Thus, } y = 2 \text{ or } y = -6
\]

\end{frame}


\begin{frame}
\frametitle{Answer}

The required coordinates are:
\[
\vec{Q_1} = \myvec{0 \\ 2 \\ 0} \qquad \text{and} \qquad \vec{Q_2} = \myvec{0 \\ -6 \\ 0}
\]

\end{frame}
\begin{frame}
    \frametitle{Graph}
    \begin{figure}[h!]
        \centering
        \includegraphics[width=0.7\linewidth]{FIG/graph.png}
        \caption{3D Visualization of Point P and Points on Y-axis Q1,Q2}
    \end{figure}
\end{frame}
\begin{frame}[fragile]
\frametitle{C Code }
\begin{lstlisting}[language=C]
#include <stdio.h>
#include <math.h>
#include "matfun.h"

int main() {
    double P[3] = {3.0, -2.0, 5.0};
    double distance = 5.0 * sqrt(2.0);

    double roots[2];
    solve_y_coordinate(P, distance, roots);

    if (isnan(roots[0]) || isnan(roots[1])) {
        printf("No real solutions exist for the given distance.\n");
    } else {
        printf("The points on the Y-axis at distance %.2f from P(3, -2, 5) are:\n", distance);
        printf("Q1 = (0, %.2f, 0)\n", roots[0]);
        printf("Q2 = (0, %.2f, 0)\n", roots[1]);
    }
    return 0;
}

\end{lstlisting}
\end{frame}
\begin{frame}[fragile]
\frametitle{Python Plot }
\begin{lstlisting}[language=Python]
import numpy as np
import matplotlib.pyplot as plt
from mpl_toolkits.mplot3d import Axes3D

# Points
P = np.array([3, -2, 5])
Q1 = np.array([0, 2, 0])
Q2 = np.array([0, -6, 0])

# Plot
fig = plt.figure()
ax = fig.add_subplot(111, projection='3d')

# Plot points
ax.scatter(*P, color='red', label='P(3,-2,5)', s=50)
ax.scatter(*Q1, color='green', label='Q1(0,2,0)', s=50)
ax.scatter(*Q2, color='blue', label='Q2(0,-6,0)', s=50)


\end{lstlisting}
\end{frame}
\begin{frame}[fragile]
\frametitle{Python Plot }
\begin{lstlisting}[language=Python]
# Plot Y-axis line for reference
y_axis = np.array([[0, y, 0] for y in np.linspace(-8, 4, 100)])
ax.plot(y_axis[:,0], y_axis[:,1], y_axis[:,2], color='orange', linestyle='--', label='Y-axis')

# Lines from P to Q1 and Q2
ax.plot([P[0], Q1[0]], [P[1], Q1[1]], [P[2], Q1[2]], color='purple', linestyle='-', label='Distance PQ1')
ax.plot([P[0], Q2[0]], [P[1], Q2[1]], [P[2], Q2[2]], color='cyan', linestyle='-', label='Distance PQ2')

# Labels and legend
ax.set_xlabel('X')
ax.set_ylabel('Y')
ax.set_zlabel('Z')
ax.legend()


\end{lstlisting}
\end{frame}
\begin{frame}[fragile]
\frametitle{Python Plot }
\begin{lstlisting}[language=Python]
# Set aspect ratio equal for better visualization
ax.set_box_aspect([1,2,1])

plt.title('3D Visualization of Point P and Points on Y-axis Q1, Q2')
plt.savefig('3d_points_plot.png', dpi=300)

plt.show()
\end{lstlisting}
\end{frame}
\end{document}