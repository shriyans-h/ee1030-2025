\documentclass{beamer}
\usepackage[utf8]{inputenc}

\usetheme{Madrid}
\usecolortheme{default}
\usepackage{amsmath,amssymb,amsfonts,amsthm}
\usepackage{txfonts}
\usepackage{tkz-euclide}
\usepackage{listings}
\usepackage{adjustbox}
\usepackage{array}
\usepackage{tabularx}
\usepackage{gvv}
\usepackage{lmodern}
\usepackage{circuitikz}
\usepackage{tikz}
\usepackage{graphicx}
\usepackage{multicol}

\setbeamertemplate{page number in head/foot}[totalframenumber]

\usepackage{tcolorbox}
\tcbuselibrary{minted,breakable,xparse,skins}


{
  \leavevmode%
  \hbox{%
  \begin{beamercolorbox}[wd=\paperwidth,ht=2.25ex,dp=1ex,right]{author in head/foot}%
    \insertframenumber{} / \inserttotalframenumber\hspace*{2ex} 
  \end{beamercolorbox}}%
  \vskip0pt%
}
\setbeamertemplate{navigation symbols}{}

\providecommand{\nCr}[2]{\,^{#1}C_{#2}} % nCr
\providecommand{\nPr}[2]{\,^{#1}P_{#2}} % nPr
\providecommand{\mbf}{\mathbf}
\providecommand{\pr}[1]{\ensuremath{\Pr\left(#1\right)}}
\providecommand{\qfunc}[1]{\ensuremath{Q\left(#1\right)}}
\providecommand{\sbrak}[1]{\ensuremath{{}\left[#1\right]}}
\providecommand{\lsbrak}[1]{\ensuremath{{}\left[#1\right.}}
\providecommand{\rsbrak}[1]{\ensuremath{{}\left.#1\right]}}
\providecommand{\brak}[1]{\ensuremath{\left(#1\right)}}
\providecommand{\lbrak}[1]{\ensuremath{\left(#1\right.}}
\providecommand{\rbrak}[1]{\ensuremath{\left.#1\right)}}
\providecommand{\cbrak}[1]{\ensuremath{\left\{#1\right\}}}
\providecommand{\lcbrak}[1]{\ensuremath{\left\{#1\right.}}
\providecommand{\rcbrak}[1]{\ensuremath{\left.#1\right\}}}
\theoremstyle{remark}
\newtheorem{rem}{Remark}
\newcommand{\sgn}{\mathop{\mathrm{sgn}}}
\providecommand{\abs}[1]{\vert#1\vert}
\providecommand{\res}[1]{\Res\displaylimits_{#1}} 
\providecommand{\norm}[1]{\lVert#1\rVert}
\providecommand{\mtx}[1]{\mathbf{#1}}
\providecommand{\mean}[1]{E[ #1 ]}
\providecommand{\fourier}{\overset{\mathcal{F}}{ \rightleftharpoons}}
%\providecommand{\hilbert}{\overset{\mathcal{H}}{ \rightleftharpoons}}
\providecommand{\system}[1]{\overset{\mathcal{#1}}{ \longleftrightarrow}}
%\providecommand{\system}{\overset{\mathcal{H}}{ \longleftrightarrow}}
	%\newcommand{\solution}[2]{\vec{Solution:}{#1}}
%\newcommand{\solution}{\noindent \vec{Solution: }}
\providecommand{\dec}[2]{\ensuremath{\overset{#1}{\underset{#2}{\gtrless}}}}
\newcommand{\myvec}[1]{\ensuremath{\begin{pmatrix}#1\end{pmatrix}}}

\lstset{
%language=C,
frame=single, 
breaklines=true,
columns=fullflexible
}

\numberwithin{equation}{section}

\lstset{
  language=Python,
  basicstyle=\ttfamily\small,
  keywordstyle=\color{blue},
  stringstyle=\color{orange},
  numbers=left,
  numberstyle=\tiny\color{gray},
  breaklines=true,
  showstringspaces=false
}
\usepackage{listings}
\usepackage{xcolor}

\lstset{
  language=C,
  basicstyle=\ttfamily\footnotesize,
  keywordstyle=\color{blue}\bfseries,
  commentstyle=\color{gray}\itshape,
  stringstyle=\color{orange},
  numbers=left,
  numberstyle=\tiny\color{gray},
  breaklines=true,
  frame=single,
  showstringspaces=false,
  tabsize=4,
  captionpos=b
}

\numberwithin{equation}{section}
\title{1.3.9}
\author{AI25BTECH11030 - SARVESH TAMGADE}
% \maketitle
% \newpage
% \bigskip
\begin{document}

\begin{frame}
\titlepage
\end{frame}

\begin{frame}
\frametitle{Question}

Find the coordinates of a point on Y axis which is at a distance of \( 5\sqrt{2} \) from the point \( P(3, -2, 5) \).

\end{frame}

\section{Solution}

\begin{frame}
\frametitle{Solution}
Let 
\begin{align}
\vec{P} &\in \mathbb{R}^3, \quad \vec{Q} = y\,\vec{e}_2, \quad \text{where} \quad \vec{e}_2 = \begin{pmatrix} 0 \\ 1 \\ 0 \end{pmatrix}
\end{align}

The required distance condition is
\begin{align}
\|\vec{P} - \vec{Q}\| &= d \\
\implies (\vec{P} - y\vec{e}_2)^T (\vec{P} - y\vec{e}_2) &= d^2
\end{align}

Expanding the quadratic form:
\begin{align}
\vec{P}^T \vec{P} - 2y\,\vec{e}_2^T \vec{P} + y^2 \vec{e}_2^T \vec{e}_2 &= d^2
\end{align}

Since \(\vec{e}_2^T \vec{e}_2 = 1\), this leads to the quadratic equation in \( y \):
\begin{align}
y^2 - 2(\vec{e}_2^T \vec{P}) y + \left(\vec{P}^T \vec{P} - d^2\right) = 0
\end{align}


\end{frame}
\begin{frame}
\frametitle{Solution}
Applying the quadratic formula, the solution for \( y \) is:
\begin{align}
y = \vec{e}_2^T \vec{P} \pm \sqrt{\left(\vec{e}_2^T \vec{P}\right)^2 - \left(\vec{P}^T \vec{P} - d^2\right)}
\end{align}
\begin{align}
\vec{P} &= \begin{pmatrix}3 \\ -2 \\ 5\end{pmatrix}, \quad d = 5\sqrt{2}
\end{align}

Calculate intermediate terms:
\begin{align}
\vec{e}_2^T \vec{P} &= -2 \\
\vec{P}^T \vec{P} &= 3^2 + (-2)^2 + 5^2 = 38 \\
d^2 &= (5\sqrt{2})^2 = 50
\end{align}

Substitute into the general formula:

\end{frame}
\begin{frame}
\frametitle{Solution}
\begin{align}
y &= -2 \pm \sqrt{(-2)^2 - (38 - 50)} \\
  &= -2 \pm \sqrt{4 + 12} \\
  &= -2 \pm 4
\end{align}

Solutions are:
\begin{align}
y_1 &= 2, \qquad y_2 = -6
\end{align}
\end{frame}

\begin{frame}
\frametitle{Answer}

Therefore, the required points on the Y-axis are:
\begin{align}
\vec{Q}_1 &= \begin{pmatrix} 0 \\ 2 \\ 0 \end{pmatrix}, \qquad
\vec{Q}_2 = \begin{pmatrix} 0 \\ -6 \\ 0 \end{pmatrix}
\end{align}


\end{frame}
\begin{frame}
    \frametitle{Graph}
    \begin{figure}[h!]
        \centering
        \includegraphics[width=0.7\linewidth]{FIG/graph.png}
        \caption{3D Visualization of Point P and Points on Y-axis Q1,Q2}
    \end{figure}
\end{frame}
\begin{frame}[fragile]
\frametitle{C Code }
\begin{lstlisting}[language=C]
#include <stdio.h>
#include <math.h>
#include "matfun.h"

int main() {
    double P[3] = {3.0, -2.0, 5.0};
    double distance = 5.0 * sqrt(2.0);

    double roots[2];
    solve_y_coordinate(P, distance, roots);

    if (isnan(roots[0]) || isnan(roots[1])) {
        printf("No real solutions exist for the given distance.\n");
    } else {
        printf("The points on the Y-axis at distance %.2f from P(3, -2, 5) are:\n", distance);
        printf("Q1 = (0, %.2f, 0)\n", roots[0]);
        printf("Q2 = (0, %.2f, 0)\n", roots[1]);
    }
    return 0;
}

\end{lstlisting}
\end{frame}
\begin{frame}[fragile]
\frametitle{Python Plot }
\begin{lstlisting}[language=Python]
import numpy as np
import matplotlib.pyplot as plt
from mpl_toolkits.mplot3d import Axes3D

# Points
P = np.array([3, -2, 5])
Q1 = np.array([0, 2, 0])
Q2 = np.array([0, -6, 0])

# Plot
fig = plt.figure()
ax = fig.add_subplot(111, projection='3d')

# Plot points
ax.scatter(*P, color='red', label='P(3,-2,5)', s=50)
ax.scatter(*Q1, color='green', label='Q1(0,2,0)', s=50)
ax.scatter(*Q2, color='blue', label='Q2(0,-6,0)', s=50)


\end{lstlisting}
\end{frame}
\begin{frame}[fragile]
\frametitle{Python Plot }
\begin{lstlisting}[language=Python]
# Plot Y-axis line for reference
y_axis = np.array([[0, y, 0] for y in np.linspace(-8, 4, 100)])
ax.plot(y_axis[:,0], y_axis[:,1], y_axis[:,2], color='orange', linestyle='--', label='Y-axis')

# Lines from P to Q1 and Q2
ax.plot([P[0], Q1[0]], [P[1], Q1[1]], [P[2], Q1[2]], color='purple', linestyle='-', label='Distance PQ1')
ax.plot([P[0], Q2[0]], [P[1], Q2[1]], [P[2], Q2[2]], color='cyan', linestyle='-', label='Distance PQ2')

# Labels and legend
ax.set_xlabel('X')
ax.set_ylabel('Y')
ax.set_zlabel('Z')
ax.legend()


\end{lstlisting}
\end{frame}
\begin{frame}[fragile]
\frametitle{Python Plot }
\begin{lstlisting}[language=Python]
# Set aspect ratio equal for better visualization
ax.set_box_aspect([1,2,1])

plt.title('3D Visualization of Point P and Points on Y-axis Q1, Q2')
plt.savefig('3d_points_plot.png', dpi=300)

plt.show()
\end{lstlisting}
\end{frame}
\end{document}