\let\negmedspace\undefined
\let\negthickspace\undefined
\documentclass{article}
\usepackage{gvv-book}
\usepackage{gvv}
\usepackage{amsmath}
\usepackage{amsfonts}
\usepackage{tikz}
\usepackage{setspace}
\usepackage{gensymb}
\usepackage[cmex10]{amsmath}
\usepackage{amsthm}
\usepackage{mathrsfs}
\usepackage{txfonts}
\usepackage{stfloats}
\usepackage{bm}
\usepackage{cite}
\usepackage{cases}
\usepackage{subfig}
\usepackage{longtable}
\usepackage{multirow}
\usepackage{enumitem}
\usepackage{mathtools}
\usepackage{tikz}
\usepackage{circuitikz}
\usepackage{verbatim}
\usepackage[breaklinks=true]{hyperref}
\usepackage{tkz-euclide}
\usepackage{listings}
\usepackage{color}    
\usepackage{array}    
\usepackage{longtable}
\usepackage{calc}     
\usepackage{multirow} 
\usepackage{hhline}   
\usepackage{ifthen}   
\usepackage{lscape}     
\usepackage{chngcntr}
\usepackage{graphicx}
\usepackage{float}
\usepackage{multicol}
\usepackage[a4paper, left = 1.5cm, right = 1.5cm]{geometry}

%\newtheorem{theorem}{Theorem}[section]
%\newtheorem{theorem}{Theorem}[section]
%\newtheorem{problem}{Problem}
%\newtheorem{proposition}{Proposition}[section]
%\newtheorem{lemma}{Lemma}[section]
%\newtheorem{corollary}[theorem]{Corollary}
%\newtheorem{example}{Example}[section]
%\newtheorem{definition}[problem]{Definition}
\title{1.8.4}
\author{AI25BTECH110030 - SARVESH TAMGADE}
\begin{document}

{\let\newpage\relax\maketitle}

\textbf{Question}:

Find the coordinates of a point on Y axis which is at a distance of \( 5\sqrt{2} \) from the point \( P(3, -2, 5) \).

\textbf{Solution:}

Given point:
\begin{align}
\vec{P} &= \begin{pmatrix} 3 \\ -2 \\ 5 \end{pmatrix}
\end{align}

We seek a point on the Y-axis:
\begin{align}
\vec{Q} &= y\,\vec{e}_2 = \begin{pmatrix} 0 \\ y \\ 0 \end{pmatrix}
\end{align}

Let the standard basis vectors be:
\begin{align}
\vec{e}_1 &= \begin{pmatrix} 1 \\ 0 \\ 0 \end{pmatrix}, \quad
\vec{e}_2 = \begin{pmatrix} 0 \\ 1 \\ 0 \end{pmatrix}, \quad
\vec{e}_3 = \begin{pmatrix} 0 \\ 0 \\ 1 \end{pmatrix}
\end{align}

The required distance:
\begin{align}
\|\vec{P} - \vec{Q}\| &= d = 5\sqrt{2}
\end{align}

So, equating squared norms:
\begin{align}
\|\vec{P} - y\vec{e}_2\|^2 &= d^2 \\
(\vec{P} - y\vec{e}_2)^T (\vec{P} - y\vec{e}_2) &= d^2 \\
\vec{P}^T \vec{P} - 2y \vec{P}^T \vec{e}_2 + y^2 \vec{e}_2^T \vec{e}_2 &= d^2 \\
\vec{P}^T \vec{P} - 2y (\vec{P} \cdot \vec{e}_2) + y^2 &= d^2
\end{align}

Rearrange:
\begin{align}
y^2 - 2(\vec{P} \cdot \vec{e}_2) y + (\vec{P}^T \vec{P} - d^2) = 0
\end{align}

With values:
\begin{align}
\vec{P}^T \vec{P} &= 3^2 + (-2)^2 + 5^2 = 38 \\
\vec{P} \cdot \vec{e}_2 &= -2 \\
d^2 &= (5\sqrt{2})^2 = 50
\end{align}

So:
\begin{align}
y^2 + 4y - 12 = 0
\end{align}

Quadratic formula:
\begin{align}
y &= \frac{-4 \pm \sqrt{4^2 - 4 \cdot 1 \cdot (-12)}}{2} \\
  &= \frac{-4 \pm \sqrt{16 + 48}}{2} \\
  &= \frac{-4 \pm 8}{2}
\end{align}

Thus,
\begin{align}
y_1 &= 2, \qquad y_2 = -6
\end{align}
\vspace{2mm}
\textbf{Answer:}
\\

Therefore, the required points on the Y-axis are:
\begin{align}
\vec{Q}_1 &= \begin{pmatrix} 0 \\ 2 \\ 0 \end{pmatrix}, \qquad
\vec{Q}_2 = \begin{pmatrix} 0 \\ -6 \\ 0 \end{pmatrix}
\end{align}


\textbf{Graph:}
\begin{figure}[H]
	\centering
	\includegraphics[width=0.8\columnwidth]{FIG/graph.png}
	\caption{3D Visualization of Point P and Points on Y-axis Q1,Q2}
	\label{img}
\end{figure}
\end{document}
