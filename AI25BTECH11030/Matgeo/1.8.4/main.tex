\let\negmedspace\undefined
\let\negthickspace\undefined
\documentclass{article}
\usepackage{gvv-book}
\usepackage{gvv}
\usepackage{amsmath}
\usepackage{amsfonts}
\usepackage{tikz}
\usepackage{setspace}
\usepackage{gensymb}
\usepackage[cmex10]{amsmath}
\usepackage{amsthm}
\usepackage{mathrsfs}
\usepackage{txfonts}
\usepackage{stfloats}
\usepackage{bm}
\usepackage{cite}
\usepackage{cases}
\usepackage{subfig}
\usepackage{longtable}
\usepackage{multirow}
\usepackage{enumitem}
\usepackage{mathtools}
\usepackage{tikz}
\usepackage{circuitikz}
\usepackage{verbatim}
\usepackage[breaklinks=true]{hyperref}
\usepackage{tkz-euclide}
\usepackage{listings}
\usepackage{color}    
\usepackage{array}    
\usepackage{longtable}
\usepackage{calc}     
\usepackage{multirow} 
\usepackage{hhline}   
\usepackage{ifthen}   
\usepackage{lscape}     
\usepackage{chngcntr}
\usepackage{graphicx}
\usepackage{float}
\usepackage{multicol}
\usepackage[a4paper, left = 1.5cm, right = 1.5cm]{geometry}

%\newtheorem{theorem}{Theorem}[section]
%\newtheorem{theorem}{Theorem}[section]
%\newtheorem{problem}{Problem}
%\newtheorem{proposition}{Proposition}[section]
%\newtheorem{lemma}{Lemma}[section]
%\newtheorem{corollary}[theorem]{Corollary}
%\newtheorem{example}{Example}[section]
%\newtheorem{definition}[problem]{Definition}
\title{1.8.4}
\author{AI25BTECH110030 - SARVESH TAMGADE}
\begin{document}

{\let\newpage\relax\maketitle}

\textbf{Question}:

Find the coordinates of a point on Y axis which is at a distance of \( 5\sqrt{2} \) from the point \( P(3, -2, 5) \).

\textbf{Solution:}

Let 
\begin{align}
\vec{P} &\in \mathbb{R}^3, \quad \vec{Q} = y\,\vec{e}_2, \quad \text{where} \quad \vec{e}_2 = \begin{pmatrix} 0 \\ 1 \\ 0 \end{pmatrix}
\end{align}

The required distance condition is
\begin{align}
\|\vec{P} - \vec{Q}\| &= d \\
\implies (\vec{P} - y\vec{e}_2)^T (\vec{P} - y\vec{e}_2) &= d^2
\end{align}

Expanding the quadratic form:
\begin{align}
\vec{P}^T \vec{P} - 2y\,\vec{e}_2^T \vec{P} + y^2 \vec{e}_2^T \vec{e}_2 &= d^2
\end{align}

Since \(\vec{e}_2^T \vec{e}_2 = 1\), this leads to the quadratic equation in \( y \):
\begin{align}
y^2 - 2(\vec{e}_2^T \vec{P}) y + \left(\vec{P}^T \vec{P} - d^2\right) = 0
\end{align}

Applying the quadratic formula, the solution for \( y \) is:
\begin{align}
y = \vec{e}_2^T \vec{P} \pm \sqrt{\left(\vec{e}_2^T \vec{P}\right)^2 - \left(\vec{P}^T \vec{P} - d^2\right)}
\end{align}
\begin{align}
\vec{P} &= \begin{pmatrix}3 \\ -2 \\ 5\end{pmatrix}, \quad d = 5\sqrt{2}
\end{align}

Calculate intermediate terms:
\begin{align}
\vec{e}_2^T \vec{P} &= -2 \\
\vec{P}^T \vec{P} &= 3^2 + (-2)^2 + 5^2 = 38 \\
d^2 &= (5\sqrt{2})^2 = 50
\end{align}

Substitute into the general formula:
\begin{align}
y &= -2 \pm \sqrt{(-2)^2 - (38 - 50)} \\
  &= -2 \pm \sqrt{4 + 12} \\
  &= -2 \pm 4
\end{align}

Solutions are:
\begin{align}
y_1 &= 2, \qquad y_2 = -6
\end{align}

\vspace{2mm}
\textbf{Answer:}
\\

Therefore, the required points on the Y-axis are:
\begin{align}
\vec{Q}_1 &= \begin{pmatrix} 0 \\ 2 \\ 0 \end{pmatrix}, \qquad
\vec{Q}_2 = \begin{pmatrix} 0 \\ -6 \\ 0 \end{pmatrix}
\end{align}


\textbf{Graph:}
\begin{figure}[H]
	\centering
	\includegraphics[width=0.8\columnwidth]{FIG/graph.png}
	\caption{3D Visualization of Point P and Points on Y-axis Q1,Q2}
	\label{img}
\end{figure}
\end{document}
