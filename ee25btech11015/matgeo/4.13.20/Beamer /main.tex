\documentclass{beamer}
\usepackage[utf8]{inputenc}

\usetheme{Madrid}
\usecolortheme{default}
\usepackage{amsmath,amssymb,amsfonts,amsthm}
\usepackage{txfonts}
\usepackage{tkz-euclide}
\usepackage{listings}
\usepackage{adjustbox}
\usepackage{array}
\usepackage{tabularx}
\usepackage{gvv}
\usepackage{lmodern}
\usepackage{circuitikz}
\usepackage{tikz}
\usepackage{graphicx}
\usepackage[T1]{fontenc}
\usepackage[utf8]{inputenc}

\lstset{
  language=Python,
  basicstyle=\ttfamily\small,
  breaklines=true,
  literate={λ}{{$\lambda$}}1
}



\setbeamertemplate{page number in head/foot}[totalframenumber]

\usepackage{tcolorbox}
\tcbuselibrary{minted,breakable,xparse,skins}



\definecolor{bg}{gray}{0.95}
\DeclareTCBListing{mintedbox}{O{}m!O{}}{%
  breakable=true,
  listing engine=minted,
  listing only,
  minted language=#2,
  minted style=default,
  minted options={%
    linenos,
    gobble=0,
    breaklines=true,
    breakafter=,,
    fontsize=\small,
    numbersep=8pt,
    #1},
  boxsep=0pt,
  left skip=0pt,
  right skip=0pt,
  left=25pt,
  right=0pt,
  top=3pt,
  bottom=3pt,
  arc=5pt,
  leftrule=0pt,
  rightrule=0pt,
  bottomrule=2pt,
  toprule=2pt,
  colback=bg,
  colframe=orange!70,
  enhanced,
  overlay={%
    \begin{tcbclipinterior}
    \fill[orange!20!white] (frame.south west) rectangle ([xshift=20pt]frame.north west);
    \end{tcbclipinterior}},
  #3,
}
\lstset{
    language=C,
    basicstyle=\ttfamily\small,
    keywordstyle=\color{blue},
    stringstyle=\color{orange},
    commentstyle=\color{green!60!black},
    numbers=left,
    numberstyle=\tiny\color{gray},
    breaklines=true,
    showstringspaces=false,
}
\begin{document}

\title 
{4.13.20}
\date{october 1,2025}


\author 
{Bhoomika V - EE25BTECH11015}




\frame{\titlepage}
\begin{frame}{Question}
A ray of light along $x + 3y = 3$ gets reflected upon reaching the $X$-axis.  
The equation of the reflected ray is:  

\[
\begin{array}{cc}
\text{(a)} \; y = x + 3 & \text{(b)} \; 3y = x - 3 \\[8pt]
\text{(c)} \; y = 3x - 3 & \text{(d)} \; 3y = x - 1
\end{array}
\]
\end{frame}

\begin{frame}{Parametric form of line}
The given line in parametric (matrix) form

\[
x+3y=3.
\]
The normal vector is 
\[
\vec{n}=\begin{pmatrix}1\\3\end{pmatrix}.
\]
A direction vector $\vec{d}$ satisfies $\vec{n}^\top \vec{d}=0$.  
\[
\vec{d}=\begin{pmatrix}-3\\1\end{pmatrix}, 
\quad 
\]
A point on the line is 
\[
\vec{p}=\begin{pmatrix}0\\1\end{pmatrix}, \quad (0+3\cdot 1=3).
\]

Hence, the parametric form is
\[
\vec{r}(t)=\vec{p}+t\vec{d}
=\begin{pmatrix}0\\1\end{pmatrix}+t\begin{pmatrix}-3\\1\end{pmatrix}.
\]
\end{frame}

\begin{frame}
Point of incidence (intersection with the $x$-axis)


For incidence with the $x$-axis, set $y=0$.  
From the second component:
\[
1+t=0 \;\;\Rightarrow\;\; t=-1.
\]

Thus,
\[
\vec{P}=\vec{r}(-1)
=\begin{pmatrix}0\\1\end{pmatrix} -1\begin{pmatrix}-3\\1\end{pmatrix}
=\begin{pmatrix}3\\0\end{pmatrix}.
\]

\end{frame}

\begin{frame}{Reflection}
Reflection in the $x$-axis is represented by the matrix
\[
R=\begin{pmatrix}1 & 0\\[4pt] 0 & -1\end{pmatrix}.
\]

So,
\[
\vec{d}'=R\vec{d}
=\begin{pmatrix}1 & 0\\ 0 & -1\end{pmatrix}
\begin{pmatrix}-3\\1\end{pmatrix}
=\begin{pmatrix}-3\\-1\end{pmatrix}.
\]

(Equivalently, we can take $\vec{d}'=(3,1)$.)


\end{frame}

\begin{frame}{Reflected ray}
Equation of the reflected ray


The reflected ray is
\[
\vec{r}'(s)=\vec{P}+s\vec{d}'
=\begin{pmatrix}3\\0\end{pmatrix}+s\begin{pmatrix}3\\1\end{pmatrix}.
\]

Coordinates:
\[
x=3+3s, \quad y=0+s.
\]

Thus,
\[
x-3=3y \quad \Rightarrow \quad 3y=x-3.
\]

 \text{Equation of the reflected ray: } \; 3y=x-3 
\end{frame}

\begin{frame}[fragile]
    \begin{lstlisting}
#include <stdio.h>

// Reflect a vector (dx,dy) about the X-axis
void reflect_ray(float dx, float dy, float *rx, float *ry) {
    *rx = dx;      // x component unchanged
    *ry = -dy;     // y component flipped
}

     \end{lstlisting}
\end{frame}

\begin{frame}[fragile]
    \frametitle{Python Code}
    \begin{lstlisting}
import ctypes
import os
import numpy as np
import matplotlib.pyplot as plt

# --- Load the C library ---
lib_path = os.path.abspath("./reflection.so")
try:
    c_lib = ctypes.CDLL(lib_path)
except OSError:
    print("reflection.so not found. Compile with: gcc -shared -o reflection.so -fPIC reflection.c")
    exit()

# Define function signature
c_lib.reflect_ray.argtypes = [ctypes.c_float, ctypes.c_float,
                              ctypes.POINTER(ctypes.c_float),
                              ctypes.POINTER(ctypes.c_float)]

    \end{lstlisting}
\end{frame} 

\begin{frame}[fragile]
    \frametitle{Python Code}
    \begin{lstlisting}
# --- Incident ray direction (line x+3y=3 has direction (-3,1)) ---
dx, dy = -3.0, 1.0
rx, ry = ctypes.c_float(), ctypes.c_float()

# Call C function
c_lib.reflect_ray(ctypes.c_float(dx), ctypes.c_float(dy),
                  ctypes.byref(rx), ctypes.byref(ry))

print(f"Incident direction = ({dx}, {dy})")
print(f"Reflected direction = ({rx.value}, {ry.value})")

# --- Geometry ---
# Point of incidence (intersection with x-axis)
P = (3, 0)
    \end{lstlisting}
\end{frame} 

\begin{frame}[fragile]
    \frametitle{Python Code}
    \begin{lstlisting}
# Parametric plotting
t = np.linspace(-1, 2, 100)
incident_x = P[0] + dx * t
incident_y = P[1] + dy * t

reflected_x = P[0] + rx.value * t
reflected_y = P[1] + ry.value * t

# --- Plot ---
plt.figure(figsize=(6,6))
plt.plot(incident_x, incident_y, "b", label="Incident Ray")
plt.plot(reflected_x, reflected_y, "r", label="Reflected Ray")

    \end{lstlisting}
\end{frame} 

\begin{frame}[fragile]
    \frametitle{Python Code}
    \begin{lstlisting}
# Mark incidence point
plt.scatter(*P, color="black", s=60, zorder=5)
plt.text(P[0]+0.1, P[1]+0.2, "P(3,0)", fontsize=10)

# Axes formatting
plt.axhline(0, color="black", linewidth=1)
plt.axvline(0, color="black", linewidth=1)
plt.gca().set_aspect("equal")
plt.xlim(-2, 6)
plt.ylim(-3, 3)

plt.title("Reflection of Ray $x+3y=3$ at the X-axis (via C function)")
plt.xlabel("x")
plt.ylabel("y")
plt.legend()
plt.grid(True)
plt.show()
    \end{lstlisting}
\end{frame} 


\end{document}