\documentclass{beamer}
\usepackage[utf8]{inputenc}

\usetheme{Madrid}
\usecolortheme{default}
\usepackage{amsmath,amssymb,amsfonts,amsthm}
\usepackage{txfonts}
\usepackage{tkz-euclide}
\usepackage{listings}
\usepackage{adjustbox}
\usepackage{array}
\usepackage{tabularx}
\usepackage{gvv}
\usepackage{lmodern}
\usepackage{circuitikz}
\usepackage{tikz}
\usepackage{graphicx}
\usepackage[T1]{fontenc}
\usepackage[utf8]{inputenc}
\lstset{
  language=Python,
  basicstyle=\ttfamily\small,
  breaklines=true,
  literate={λ}{{$\lambda$}}1
}



\setbeamertemplate{page number in head/foot}[totalframenumber]

\usepackage{tcolorbox}
\tcbuselibrary{minted,breakable,xparse,skins}



\definecolor{bg}{gray}{0.95}
\DeclareTCBListing{mintedbox}{O{}m!O{}}{%
  breakable=true,
  listing engine=minted,
  listing only,
  minted language=#2,
  minted style=default,
  minted options={%
    linenos,
    gobble=0,
    breaklines=true,
    breakafter=,,
    fontsize=\small,
    numbersep=8pt,
    #1},
  boxsep=0pt,
  left skip=0pt,
  right skip=0pt,
  left=25pt,
  right=0pt,
  top=3pt,
  bottom=3pt,
  arc=5pt,
  leftrule=0pt,
  rightrule=0pt,
  bottomrule=2pt,
  toprule=2pt,
  colback=bg,
  colframe=orange!70,
  enhanced,
  overlay={%
    \begin{tcbclipinterior}
    \fill[orange!20!white] (frame.south west) rectangle ([xshift=20pt]frame.north west);
    \end{tcbclipinterior}},
  #3,
}
\lstset{
    language=C,
    basicstyle=\ttfamily\small,
    keywordstyle=\color{blue},
    stringstyle=\color{orange},
    commentstyle=\color{green!60!black},
    numbers=left,
    numberstyle=\tiny\color{gray},
    breaklines=true,
    showstringspaces=false,
}
\begin{document}

\title 
{2.10.44}
\date{September 15,2025}


\author 
{Bhoomika V - EE25BTECH11015}




\frame{\titlepage}
\begin{frame}{Question}
If $\vec{a}, \vec{b}, \vec{c}$ are unit vectors, then
\[
\lVert \vec{a} - \vec{b} \rVert^2
+ \lVert \vec{b} - \vec{c} \rVert^2
+ \lVert \vec{a} - \vec{b} \rVert^2
\]
does not exceed
\begin{enumerate}[label=\alph*)] % makes options a), b), c), d)
    \item 4
    \item 9
    \item 8
    \item 6
\end{enumerate}
\end{frame}

\begin{frame}{Solution-Gram matrix}
Let 
\[
x = \vec{a}\cdot \vec{b}, \quad y = \vec{a}\cdot \vec{c}, \quad z = \vec{b}\cdot \vec{c}.
\]
Since $\vec{a},\vec{b},\vec{c}$ are unit vectors, their Gram matrix is
\[
G = \begin{pmatrix}
\vec{a}\cdot \vec{a} & \vec{a}\cdot \vec{b} & \vec{a}\cdot \vec{c} \\
\vec{a}\cdot \vec{b} & \vec{b}\cdot \vec{b} & \vec{b}\cdot \vec{c} \\
\vec{a}\cdot \vec{c} & \vec{b}\cdot \vec{c} & \vec{c}\cdot \vec{c}
\end{pmatrix}. 
\]
\[
 = \begin{pmatrix}
1 & x & y \\
x & 1 & z \\
y & z & 1
\end{pmatrix}.
\]
\end{frame}

\begin{frame}{SOlution}
Now consider
\[
(1,1,1) \, G \, (1,1,1)^T = (\vec{a}+\vec{b}+\vec{c}) \cdot (\vec{a}+\vec{b}+\vec{c}) \geq 0.
\]
Expanding,
\[
|\vec{a}|^2 + |\vec{b}|^2 + |\vec{c}|^2 + 2(x+y+z) = 3 + 2(x+y+z) \geq 0,
\]
\begin{equation}
\implies \quad x+y+z \geq -\tfrac{3}{2}.
\label{1}
\end{equation}
\end{frame}

\begin{frame}{Solution}
Now ,
\[
|\vec{a}-\vec{b}|^2 + |\vec{b}-\vec{c}|^2 + |\vec{c}-\vec{a}|^2
= (2-2x) + (2-2z) + (2-2y).
\]
So,
\[
= 6 - 2(x+y+z).
\]
From Equation~\eqref{1}
\[
6 - 2(x+y+z) \leq 6 - 2\left(-\tfrac{3}{2}\right) = 9.
\]

\end{frame}

\begin{frame}{Answer}
\[
\text{Thus, } 
\lVert \vec{a} - \vec{b} \rVert^2
+ \lVert \vec{b} - \vec{c} \rVert^2
+ \lVert \vec{a} - \vec{b} \rVert^2
\text{ does not exceed } 9.
\]
\end{frame}

\begin{frame}[fragile]
    \frametitle{C Code - A function to find max value}

    \begin{lstlisting}
#include <math.h>

// Function to check inequality
// Returns 1 if inequality holds, 0 otherwise
int is_within_bound(float ax, float ay, float az,
                    float bx, float by, float bz,
                    float cx, float cy, float cz) {
    
    // Compute squared distances
    float ab = (ax - bx)*(ax - bx) + (ay - by)*(ay - by) + (az - bz)*(az - bz);
    float bc = (bx - cx)*(bx - cx) + (by - cy)*(by - cy) + (bz - cz)*(bz - cz);
    float ca = (cx - ax)*(cx - ax) + (cy - ay)*(cy - ay) + (cz - az)*(cz - az);
     \end{lstlisting}
\end{frame}

\begin{frame}[fragile]
    \frametitle{C Code - A function to find max value  }

    \begin{lstlisting}
float sum = ab + bc + ca;

    if (sum <= 9.0f) {
        return 1;  // Inequality holds
    } else {
        return 0;  // Inequality violated
    }
}

     \end{lstlisting}
\end{frame}

\begin{frame}[fragile]
    \frametitle{Python Code}
    \begin{lstlisting}
import numpy as np
import matplotlib.pyplot as plt
from mpl_toolkits.mplot3d.art3d import Poly3DCollection
import ctypes
import os

# --- Load the C library ---
try:
    c_lib = ctypes.CDLL('./code.so')
except OSError:
    print(" Error: 'code.so' not found. Compile with:")
    print("   gcc -shared -o code.so -fPIC inequality.c -lm")
    exit()

# Define argument and return types
c_lib.is_within_bound.argtypes = [
    ctypes.c_float, ctypes.c_float, ctypes.c_float,
    ctypes.c_float, ctypes.c_float, ctypes.c_float,
    ctypes.c_float, ctypes.c_float, ctypes.c_float]
    \end{lstlisting}
\end{frame}

\begin{frame}[fragile]
    \frametitle{Python Code}
    \begin{lstlisting}
c_lib.is_within_bound.restype = ctypes.c_int

# --- Function to generate random unit vector ---
def random_unit_vector():
    vec = np.random.randn(3)
    return vec / np.linalg.norm(vec)

# --- Generate unit vectors a, b, c ---
a = random_unit_vector()
b = random_unit_vector()
c = random_unit_vector()

# --- Call C function ---
result = c_lib.is_within_bound(a[0], a[1], a[2],
                               b[0], b[1], b[2],
                               c[0], c[1], c[2])
    \end{lstlisting}
\end{frame}

\begin{frame}[fragile]
    \frametitle{Python Code}
    \begin{lstlisting}
if result == 1:
    print(" The inequality holds (sum ≤ 9).")
else:
    print(" The inequality is violated.")

# --- Plotting ---
fig = plt.figure(figsize=(8,6))
ax = fig.add_subplot(111, projection='3d')

# Plot points
ax.scatter(*a, color="red", s=50)
ax.scatter(*b, color="blue", s=50)
ax.scatter(*c, color="green", s=50)

# Draw triangle
triangle = np.array([a, b, c])
ax.add_collection3d(Poly3DCollection([triangle], alpha=0.2, facecolor='cyan'))
    \end{lstlisting}
\end{frame}

\begin{frame}[fragile]
    \frametitle{Python Code}
    \begin{lstlisting}
# Edges
ax.plot(*zip(a,b), color="black")
ax.plot(*zip(b,c), color="black")
ax.plot(*zip(c,a), color="black")

# Labels
ax.text(*a, "a", color="red")
ax.text(*b, "b", color="blue")
ax.text(*c, "c", color="green")

# Axes labels
ax.set_xlabel("X-axis")
ax.set_ylabel("Y-axis")
ax.set_zlabel("Z-axis")
ax.set_title("Triangle formed by unit vectors a, b, c")

plt.show()
    \end{lstlisting}
\end{frame}

\end{document}