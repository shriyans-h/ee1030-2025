\documentclass{beamer}
\usepackage[utf8]{inputenc}

\usetheme{Madrid}
\usecolortheme{default}
\usepackage{amsmath,amssymb,amsfonts,amsthm}
\usepackage{txfonts}
\usepackage{tkz-euclide}
\usepackage{listings}
\usepackage{adjustbox}
\usepackage{array}
\usepackage{tabularx}
\usepackage{gvv}
\usepackage{lmodern}
\usepackage{circuitikz}
\usepackage{tikz}
\usepackage{graphicx}
\usepackage[T1]{fontenc}
\usepackage[utf8]{inputenc}

\lstset{
  language=Python,
  basicstyle=\ttfamily\small,
  breaklines=true,
  literate={λ}{{$\lambda$}}1
}



\setbeamertemplate{page number in head/foot}[totalframenumber]

\usepackage{tcolorbox}
\tcbuselibrary{minted,breakable,xparse,skins}



\definecolor{bg}{gray}{0.95}
\DeclareTCBListing{mintedbox}{O{}m!O{}}{%
  breakable=true,
  listing engine=minted,
  listing only,
  minted language=#2,
  minted style=default,
  minted options={%
    linenos,
    gobble=0,
    breaklines=true,
    breakafter=,,
    fontsize=\small,
    numbersep=8pt,
    #1},
  boxsep=0pt,
  left skip=0pt,
  right skip=0pt,
  left=25pt,
  right=0pt,
  top=3pt,
  bottom=3pt,
  arc=5pt,
  leftrule=0pt,
  rightrule=0pt,
  bottomrule=2pt,
  toprule=2pt,
  colback=bg,
  colframe=orange!70,
  enhanced,
  overlay={%
    \begin{tcbclipinterior}
    \fill[orange!20!white] (frame.south west) rectangle ([xshift=20pt]frame.north west);
    \end{tcbclipinterior}},
  #3,
}
\lstset{
    language=C,
    basicstyle=\ttfamily\small,
    keywordstyle=\color{blue},
    stringstyle=\color{orange},
    commentstyle=\color{green!60!black},
    numbers=left,
    numberstyle=\tiny\color{gray},
    breaklines=true,
    showstringspaces=false,
}
\begin{document}

\title 
{5.4.10}
\date{October 3,2025}


\author 
{Bhoomika V - EE25BTECH11015}




\frame{\titlepage}
\begin{frame}{Question}
\[
\text{Find the area of the triangle } ABC \text{ whose vertices are }\\ \\
\vec{A}(2,5),\; \vec{B}(4,7),\; \vec{C}(6,2).
\]
\end{frame}

\begin{frame}{solution} 
\[
A = \begin{bmatrix}
1 & -1 & 2 \\
2 & 3 & 5 \\
-2 & 0 & 1
\end{bmatrix}.
\]

We form the augmented matrix $[A \mid I]$:

\[
\left[\begin{array}{ccc|ccc}
1 & -1 & 2 & 1 & 0 & 0\\
2 & 3 & 5 & 0 & 1 & 0\\
-2 & 0 & 1 & 0 & 0 & 1
\end{array}\right]
\]
\end{frame}

\begin{frame}
\[
\overset{\substack{R_2 \to R_2 - 2R_1 \\ R_3 \to R_3 + 2R_1}}{\longrightarrow}
\left[\begin{array}{ccc|ccc}
1 & -1 & 2 & 1 & 0 & 0\\
0 & 5 & 1 & -2 & 1 & 0\\
0 & -2 & 5 & 2 & 0 & 1
\end{array}\right]
\]

\[
\overset{R_2 \to \tfrac{1}{5}R_2}{\longrightarrow}
\left[\begin{array}{ccc|ccc}
1 & -1 & 2 & 1 & 0 & 0\\
0 & 1 & \tfrac{1}{5} & -\tfrac{2}{5} & \tfrac{1}{5} & 0\\
0 & -2 & 5 & 2 & 0 & 1
\end{array}\right]
\]
\end{frame}

\begin{frame}
\[
\overset{\substack{R_1 \to R_1 + R_2 \\ R_3 \to R_3 + 2R_2}}{\longrightarrow}
\left[\begin{array}{ccc|ccc}
1 & 0 & \tfrac{11}{5} & \tfrac{3}{5} & \tfrac{1}{5} & 0\\
0 & 1 & \tfrac{1}{5} & -\tfrac{2}{5} & \tfrac{1}{5} & 0\\
0 & 0 & \tfrac{27}{5} & \tfrac{6}{5} & \tfrac{2}{5} & 1
\end{array}\right]
\]

\[
\overset{R_3 \to \tfrac{5}{27}R_3}{\longrightarrow}
\left[\begin{array}{ccc|ccc}
1 & 0 & \tfrac{11}{5} & \tfrac{3}{5} & \tfrac{1}{5} & 0\\
0 & 1 & \tfrac{1}{5} & -\tfrac{2}{5} & \tfrac{1}{5} & 0\\
0 & 0 & 1 & \tfrac{2}{9} & \tfrac{2}{27} & \tfrac{5}{27}
\end{array}\right]
\]
\end{frame}

\begin{frame}
\[
\overset{\substack{R_1 \to R_1 - \tfrac{11}{5}R_3 \\ R_2 \to R_2 - \tfrac{1}{5}R_3}}{\longrightarrow}
\left[\begin{array}{ccc|ccc}
1 & 0 & 0 & \tfrac{1}{9} & \tfrac{1}{27} & -\tfrac{11}{27}\\
0 & 1 & 0 & -\tfrac{4}{9} & \tfrac{5}{27} & -\tfrac{1}{27}\\
0 & 0 & 1 & \tfrac{2}{9} & \tfrac{2}{27} & \tfrac{5}{27}
\end{array}\right]
\]

Thus the inverse is

\[
A^{-1} =
\begin{bmatrix}
\dfrac{1}{9} & \dfrac{1}{27} & -\dfrac{11}{27}\\[6pt]
-\dfrac{4}{9} & \dfrac{5}{27} & -\dfrac{1}{27}\\[6pt]
\dfrac{2}{9} & \dfrac{2}{27} & \dfrac{5}{27}
\end{bmatrix}.
\]
\end{frame}

\begin{frame}[fragile]
    \frametitle

    \begin{lstlisting}
 #include <stdio.h>

// Compute determinant of 3x3 matrix
float determinant(float A[3][3]) {
    return A[0][0]*(A[1][1]*A[2][2] - A[1][2]*A[2][1])
         - A[0][1]*(A[1][0]*A[2][2] - A[1][2]*A[2][0])
         + A[0][2]*(A[1][0]*A[2][1] - A[1][1]*A[2][0]);
}

// Function to compute inverse of 3x3 matrix
int matrix_inverse(float A[3][3], float inverse[3][3]) {
    float det = determinant(A);
    if (det == 0) return 0; // singular, no inverse

    float adj[3][3];

    // Cofactor matrix
    adj[0][0] =  (A[1][1]*A[2][2] - A[1][2]*A[2][1]);
    adj[0][1] = -(A[1][0]*A[2][2] - A[1][2]*A[2][0]);
    adj[0][2] =  (A[1][0]*A[2][1] - A[1][1]*A[2][0]);

     \end{lstlisting}
\end{frame}

\begin{frame}[fragile]
    \frametitle

    \begin{lstlisting}
   adj[1][0] = -(A[0][1]*A[2][2] - A[0][2]*A[2][1]);
    adj[1][1] =  (A[0][0]*A[2][2] - A[0][2]*A[2][0]);
    adj[1][2] = -(A[0][0]*A[2][1] - A[0][1]*A[2][0]);

    adj[2][0] =  (A[0][1]*A[1][2] - A[0][2]*A[1][1]);
    adj[2][1] = -(A[0][0]*A[1][2] - A[0][2]*A[1][0]);
    adj[2][2] =  (A[0][0]*A[1][1] - A[0][1]*A[1][0]);

    // Transpose adjoint and divide by determinant → inverse
    for (int i = 0; i < 3; i++) {
        for (int j = 0; j < 3; j++) {
            inverse[i][j] = adj[j][i] / det;
        }
    }
    return 1;
}
     \end{lstlisting}
\end{frame}

\begin{frame}[fragile]
    \frametitle{Python Code}
    \begin{lstlisting}
import ctypes
import os
import numpy as np

# Load C library
lib = ctypes.CDLL(os.path.abspath("./matrix_inverse.so"))

# Define function signatures
ArrayType = ctypes.c_float * 9  # 3x3 = 9 elements

lib.matrix_inverse.argtypes = [ArrayType, ArrayType]
lib.matrix_inverse.restype = ctypes.c_int

def c_matrix_inverse(A):
    A = np.array(A, dtype=np.float32).reshape(9)
    A_c = ArrayType(*A)
    inv_c = ArrayType()
    \end{lstlisting}
\end{frame}

\begin{frame}[fragile]
    \frametitle{Python Code}
    \begin{lstlisting}
 success = lib.matrix_inverse(A_c, inv_c)
    if not success:
        raise ValueError("Matrix is singular, no inverse exists.")

    inv = np.array(list(inv_c), dtype=np.float32).reshape(3, 3)
    return inv

# --- Example ---
A = [
    [1, -1, 2],
    [2,  3, 5],
    [-2, 0, 1]
]

    \end{lstlisting}
\end{frame}
\begin{frame}[fragile]
    \frametitle{Python Code}
    \begin{lstlisting}
inv = c_matrix_inverse(A)

print("Original matrix A:")
print(np.array(A, dtype=np.float32))

print("\nInverse of A (from C function):")
print(inv)

print("\nCheck with NumPy:")
print(np.linalg.inv(np.array(A, dtype=np.float32)))
    \end{lstlisting}
\end{frame}

\end{document}