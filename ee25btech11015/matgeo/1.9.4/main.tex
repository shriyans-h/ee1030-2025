\let\negmedspace\undefined
\let\negthickspace\undefined
\documentclass[journal]{IEEEtran}
\usepackage[a5paper, margin=10mm, onecolumn]{geometry}
%\usepackage{lmodern} % Ensure lmodern is loaded for pdflatex
\usepackage{tfrupee} % Include tfrupee package

\setlength{\headheight}{1cm} % Set the height of the header box
\setlength{\headsep}{0mm}     % Set the distance between the header box and the top of the text

\usepackage{gvv-book}
\usepackage{gvv}
\usepackage{cite}
\usepackage{amsmath,amssymb,amsfonts,amsthm}
\usepackage{algorithmic}
\usepackage{graphicx}
\usepackage{textcomp}
\usepackage{xcolor}
\usepackage{txfonts}
\usepackage{listings}
\usepackage{enumitem}
\usepackage{mathtools}
\usepackage{gensymb}
\usepackage{comment}
\usepackage[breaklinks=true]{hyperref}
\usepackage{tkz-euclide} 
\usepackage{listings}
% \usepackage{gvv}                                        
\def\inputGnumericTable{}                                 
\usepackage[latin1]{inputenc}                                
\usepackage{color}                                            
\usepackage{array}                                            
\usepackage{longtable}                                       
\usepackage{calc}                                             
\usepackage{multirow}                                         
\usepackage{hhline}                                           
\usepackage{ifthen}                                           
\usepackage{lscape}
\begin{document}

\bibliographystyle{IEEEtran}
\vspace{3cm}

\title{1.9.4}
\author{EE25BTECH11015 - Bhoomika V}
% \maketitle
% \newpage
% \bigskip
{\let\newpage\relax\maketitle}

\renewcommand{\thefigure}{\theenumi}
\renewcommand{\thetable}{\theenumi}
\setlength{\intextsep}{10pt} % Space between text and floats


\numberwithin{equation}{enumi}
\numberwithin{figure}{enumi}
\renewcommand{\thetable}{\theenumi}
\parindent 0px 
{Question :-} \\ 
If $\lVert \vec{a} \rVert = 4$ and $-3 \le \lambda \le 2$, then $\lVert \lambda \vec{a} \rVert$ lies in
\begin{enumerate}
  \item $[0,12]$
  \item $[2,3]$
  \item $[8,12]$
  \item $[-12,8]$
\end{enumerate} 
\vspace{0.5cm}

\solution \\

Using matrix definition of the norm:
\begin{equation}
  \|\vec{a}\| = \sqrt{\vec{a}^T \vec{a}}, 
  \qquad \text{hence} \qquad 
  \vec{a}^T \vec{a} = \|\vec{a}\|^2 = 4^2 = 16.
  \label{eq1}
\end{equation}
  The squared norm of \(\lambda \vec{a}\) using matrix notation is:
  \[
    \|\lambda \vec{a}\|^2 = (\lambda \vec{a})^T(\lambda \vec{a}) 
    = \lambda^2 (\vec{a}^T \vec{a}).
  \]
  Substituting from Equation~\eqref{eq1}:

  \[
    \|\lambda \vec{a}\|^2 = 16 \lambda^2.
  \]

  Taking square roots (norms are nonnegative) gives
  \[
    \|\lambda \vec{a}\| = \sqrt{16\lambda^2} = 4\,|\lambda|.
  \]  
  The range of \(|\lambda|\) given \(-3 \le \lambda \le 2\).
  \[
    0 \le |\lambda| \le \max\{|-3|,|2|\} = 3.
  \]
  Multiplying by \(4\) yields
  \[
    0 \le 4|\lambda| \le 12.
  \]

  Therefore
  \[
    \|\lambda \vec{a}\| = 4|\lambda| \in [0,12].
  \]
  \[
    \boxed{\; \|\lambda \vec{a}\| \in [0,12]\; }
  \]


\begin{center}
$\implies k = 2$
\end{center}
\begin{figure}[H]
\begin{center}
\includegraphics[width=0.6\columnwidth]{Figs/Fig1.png}
\end{center}
\caption{}
\label{fig:Fig.1}
\end{figure}


\end{document}