
\documentclass{beamer}
\usepackage[utf8]{inputenc}

\usetheme{Madrid}
\usecolortheme{default}
\usepackage{amsmath,amssymb,amsfonts,amsthm}
\usepackage{txfonts}
\usepackage{tkz-euclide}
\usepackage{listings}
\usepackage{adjustbox}
\usepackage{array}
\usepackage{tabularx}
\usepackage{gvv}
\usepackage{lmodern}
\usepackage{circuitikz}
\usepackage{tikz}
\usepackage{graphicx}
\usepackage[T1]{fontenc}
\usepackage[utf8]{inputenc}
\lstset{
  language=Python,
  basicstyle=\ttfamily\small,
  breaklines=true,
  literate={λ}{{$\lambda$}}1
}



\setbeamertemplate{page number in head/foot}[totalframenumber]

\usepackage{tcolorbox}
\tcbuselibrary{minted,breakable,xparse,skins}



\definecolor{bg}{gray}{0.95}
\DeclareTCBListing{mintedbox}{O{}m!O{}}{%
  breakable=true,
  listing engine=minted,
  listing only,
  minted language=#2,
  minted style=default,
  minted options={%
    linenos,
    gobble=0,
    breaklines=true,
    breakafter=,,
    fontsize=\small,
    numbersep=8pt,
    #1},
  boxsep=0pt,
  left skip=0pt,
  right skip=0pt,
  left=25pt,
  right=0pt,
  top=3pt,
  bottom=3pt,
  arc=5pt,
  leftrule=0pt,
  rightrule=0pt,
  bottomrule=2pt,
  toprule=2pt,
  colback=bg,
  colframe=orange!70,
  enhanced,
  overlay={%
    \begin{tcbclipinterior}
    \fill[orange!20!white] (frame.south west) rectangle ([xshift=20pt]frame.north west);
    \end{tcbclipinterior}},
  #3,
}
\lstset{
    language=C,
    basicstyle=\ttfamily\small,
    keywordstyle=\color{blue},
    stringstyle=\color{orange},
    commentstyle=\color{green!60!black},
    numbers=left,
    numberstyle=\tiny\color{gray},
    breaklines=true,
    showstringspaces=false,
}
\begin{document}

\title 
{1.9.4}
\date{September 9,2025}


\author 
{Bhoomika V - EE25BTECH11015}




\frame{\titlepage}
\begin{frame}{Question}
If $\lVert \vec{a} \rVert = 4$ and $-3 \le \lambda \le 2$, then $\lVert \lambda \vec{a} \rVert$ lies in
\begin{enumerate}
  \item $[0,12]$
  \item $[2,3]$
  \item $[8,12]$
  \item $[-12,8]$
\end{enumerate} 
\end{frame}



\begin{frame}{Theoretical Solution}
\begin{equation}
  \|\vec{a}\| = \sqrt{\vec{a}^T \vec{a}}, 
  \qquad \text{hence} \qquad 
  \vec{a}^T \vec{a} = \|\vec{a}\|^2 = 4^2 = 16.
  \label{eq1}
\end{equation}
\end{frame}




\begin{frame}{Theoretical Solution}
  The squared norm of \(\lambda \vec{a}\) using matrix notation is:
  \[
    \|\lambda \vec{a}\|^2 = (\lambda \vec{a})^T(\lambda \vec{a}) 
    = \lambda^2 (\vec{a}^T \vec{a}).
  \]
  Substituting from Equation~\eqref{eq1}:

  \[
    \|\lambda \vec{a}\|^2 = 16 \lambda^2.
  \]

  Taking square roots (norms are nonnegative) gives
  \[
    \|\lambda \vec{a}\| = \sqrt{16\lambda^2} = 4\,|\lambda|.
  \]  


\end{frame}

\begin{frame}{Answer}
 The range of \(|\lambda|\) given \(-3 \le \lambda \le 2\).
  \[
    0 \le |\lambda| \le \max\{|-3|,|2|\} = 3.
  \]
  Multiplying by \(4\) yields
  \[
    0 \le 4|\lambda| \le 12.
  \]

  Therefore
  \[
    \|\lambda \vec{a}\| = 4|\lambda| \in [0,12].
  \]
  \[
    \boxed{\; \|\lambda \vec{a}\| \in [0,12]\; }
  \]

\end{frame}


\begin{frame}[fragile]
    \frametitle{C Code - A function to find the value of $\lVert \lambda \vec{a} \rVert$ }

    \begin{lstlisting}
#include <stdio.h>
#include <math.h>

int main() {
    float norm_a = 4;           // ||a|| = 4
    float lambda_min = -3, lambda_max = 2;

    // Compute max |lambda|
    float max_abs_lambda = fmax(fabs(lambda_min), fabs(lambda_max));
    float min_abs_lambda = 0;   // since lambda can be 0 in [-3, 2]

    
     \end{lstlisting}
\end{frame}


    
\begin{frame}[fragile]
\frametitle{C Code - A function to find the value of $\lVert \lambda \vec{a} \rVert$ }

\begin{lstlisting}

 // Corresponding ||lambda * a|| values
 float min_norm = norm_a * min_abs_lambda;
 float max_norm = norm_a * max_abs_lambda;    
 printf( "||lambda * a|| lies in [%.0f, %.0f]\n " , min_norm, max_norm);
 return 0;
\end{lstlisting}

\end{frame}




\begin{frame}[fragile]
    \frametitle{Python Code}
    \begin{lstlisting}
    import numpy as np
import matplotlib.pyplot as plt
import ctypes
import os

# --- Load the C library ---
try:
    c_lib = ctypes.CDLL('./code.so')
except OSError:
    print("Error: 'code.so' not found.")
    print("Please compile code.c using: gcc -shared -o code.so -fPIC code.c")
    exit()

# Define argument and return types for the C function
c_lib.norm_lambda_a.argtypes = [ctypes.c_float, ctypes.c_float]
c_lib.norm_lambda_a.restype = ctypes.c_float
    \end{lstlisting}
\end{frame}



\begin{frame}[fragile]
    \frametitle{Python Code}
    \begin{lstlisting}
    # --- Given ---
norm_a = 4.0
lam_min, lam_max = -3.0, 2.0

# --- Generate λ values and call C function ---
lambdas = np.linspace(lam_min, lam_max, 200)
y = np.array([c_lib.norm_lambda_a(ctypes.c_float(l), ctypes.c_float(norm_a)) for l in lambdas])

# --- Range ---
y_min, y_max = np.min(y), np.max(y)
print(f" The values of ||λa|| lie in the interval [{y_min:.0f}, {y_max:.0f}]")

# --- Plotting ---
plt.plot(lambdas, y, label="||λa|| = 4|λ|", color="blue")
    \end{lstlisting}
\end{frame}

\begin{frame}[fragile]
    \frametitle{Python Code}
    \begin{lstlisting}
    # Mark endpoints
plt.scatter([lam_min, lam_max],
            [c_lib.norm_lambda_a(lam_min, norm_a), c_lib.norm_lambda_a(lam_max, norm_a)],
            color=['red','green'], zorder=5)

# Labels
plt.text(lam_min, c_lib.norm_lambda_a(lam_min, norm_a)+0.5,
         f"({lam_min:.0f},{c_lib.norm_lambda_a(lam_min, norm_a):.0f})")
plt.text(lam_max, c_lib.norm_lambda_a(lam_max, norm_a)+0.5,
         f"({lam_max:.0f},{c_lib.norm_lambda_a(lam_max, norm_a):.0f})")
    \end{lstlisting}
\end{frame}

\begin{frame}[fragile]
    \frametitle{Python Code}
    \begin{lstlisting}
    # Axes and grid
plt.axhline(0, color='gray', linewidth=1)
plt.axvline(0, color='gray', linewidth=1)
plt.xlabel("lambda")
plt.ylabel("||lambda a||")
plt.title("Graph of ||lambda a|| = 4|lambda|")
plt.legend(loc='best')
plt.grid(True)
plt.show()
    \end{lstlisting}
\end{frame}


\begin{frame}{Plot}
    \centering
    \includegraphics[width=\columnwidth, height=0.8\textheight, keepaspectratio]{Figs/Fig1.png}     
\end{frame}




\end{document}