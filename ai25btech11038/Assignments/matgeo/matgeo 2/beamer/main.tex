\documentclass{beamer}
\mode<presentation>
\usepackage{amsmath}
\usepackage{amssymb}
\usepackage[utf8]{inputenc}
\usepackage{fontenc}
%\usepackage{advdate}
\usepackage{adjustbox}
\usepackage{subcaption}
\usepackage{enumitem}
\usepackage{multicol}
\usepackage{mathtools}
\usepackage{listings}
\usepackage{url}
\def\UrlBreaks{\do\/\do-}
\usetheme{Boadilla}
\usecolortheme{lily}
\setbeamertemplate{footline}
{
  \leavevmode%
  \hbox{%
  \begin{beamercolorbox}[wd=\paperwidth,ht=2.25ex,dp=1ex,right]{author in head/foot}%
    \insertframenumber{} / \inserttotalframenumber\hspace*{2ex} 
  \end{beamercolorbox}}%
  \vskip0pt%
}
\setbeamertemplate{navigation symbols}{}

\providecommand{\nCr}[2]{\,^{#1}C_{#2}} % nCr
\providecommand{\nPr}[2]{\,^{#1}P_{#2}} % nPr
\providecommand{\mbf}{\mathbf}
\providecommand{\pr}[1]{\ensuremath{\Pr\left(#1\right)}}
\providecommand{\qfunc}[1]{\ensuremath{Q\left(#1\right)}}
\providecommand{\sbrak}[1]{\ensuremath{{}\left[#1\right]}}
\providecommand{\lsbrak}[1]{\ensuremath{{}\left[#1\right.}}
\providecommand{\rsbrak}[1]{\ensuremath{{}\left.#1\right]}}
\providecommand{\brak}[1]{\ensuremath{\left(#1\right)}}
\providecommand{\lbrak}[1]{\ensuremath{\left(#1\right.}}
\providecommand{\rbrak}[1]{\ensuremath{\left.#1\right)}}
\providecommand{\cbrak}[1]{\ensuremath{\left\{#1\right\}}}
\providecommand{\lcbrak}[1]{\ensuremath{\left\{#1\right.}}
\providecommand{\rcbrak}[1]{\ensuremath{\left.#1\right\}}}
\theoremstyle{remark}
\newtheorem{rem}{Remark}
\newcommand{\sgn}{\mathop{\mathrm{sgn}}}
%\providecommand{\abs}[1]{\left\vert#1\right\vert}
\providecommand{\res}[1]{\Res\displaylimits_{#1}} 
\providecommand{\norm}[1]{\lVert#1\rVert}
\providecommand{\mtx}[1]{\mathbf{#1}}
%\providecommand{\mean}[1]{E\left[ #1 \right]}
\providecommand{\fourier}{\overset{\mathcal{F}}{ \rightleftharpoons}}
%\providecommand{\hilbert}{\overset{\mathcal{H}}{ \rightleftharpoons}}
\providecommand{\system}{\overset{\mathcal{H}}{ \longleftrightarrow}}
	%\newcommand{\solution}[2]{\textbf{Solution:}{#1}}
%\newcommand{\solution}{\noindent \textbf{Solution: }}
\providecommand{\dec}[2]{\ensuremath{\overset{#1}{\underset{#2}{\gtrless}}}}
\newcommand{\myvec}[1]{\ensuremath{\begin{pmatrix}#1\end{pmatrix}}}
\let\vec\mathbf

\lstset{
language=C,
frame=single, 
breaklines=true,
columns=fullflexible
}

\numberwithin{equation}{section}

\title{Presentation - Matgeo}
\author{Tejas Uppala - AI25BTECH11038}

\begin{document}

\begin{frame}
\titlepage
\end{frame}

\section*{Outline}


\begin{frame}
\frametitle{Problem Statement}
Find the points on the X axis which are at a distance on $2 \sqrt{5}$ from the point $\brak{7, {-4}}$. How many such points are there?
\end{frame}

\begin{frame}{Solution}
 Given that the point $\brak{7, {-4}}$ is at a distance $2 \sqrt{5}$ from, assume a point P that lies on the X axis,\\ 
   Let the given point be denoted A and its position vector \textbf{a} and the position vector of P will be 
\begin{equation}
\textbf{p} = x\cdot \mathbf{e_1}
\end{equation}

    \noindent The distance between the two given points will be,
\begin{equation}
||\textbf{a} - \textbf{p}|| = 2 \sqrt{5}
\end{equation}

\begin{equation}
||\textbf{a} - x\cdot \mathbf{e_1} || = 2 \sqrt{5}
\end{equation}

    \noindent We know that, 
\begin{equation}
||H||^2 = H \cdot H^T
\end{equation}

    \noindent So,
\begin{equation}
(\textbf{a} - x\cdot \mathbf{e_1})\cdot (\textbf{a} - x\cdot \mathbf{e_1})^T = (2\sqrt{5})^2
\end{equation}

\begin{equation}
(\textbf{a} - x\cdot \mathbf{e_1})\cdot (\textbf{a}^T - x\cdot \mathbf{e_1}^T) = 20
\end{equation}
\end{frame}

\begin{frame}
\begin{equation}
(\textbf{a}\cdot \textbf{a}^T) - (x\cdot \textbf{a}\cdot \mathbf{e_1}^T) - (x\cdot \textbf{a}^T\cdot \mathbf{e_1}) + (x^2\cdot \mathbf{e_1}\cdot \mathbf{e_1}^T) = 20
\end{equation}

\begin{equation}
(x^2\cdot \mathbf{e_1}\cdot \mathbf{e_1}^T) - ((\textbf{a}\cdot \mathbf{e_1}^T + \textbf{a}^T\cdot \mathbf{e_1} )\cdot x) + (\textbf{a}\cdot \textbf{a}^T) - 20 = 0
\end{equation}

\begin{equation}
(\textbf{a}\cdot \mathbf{e_1}^T + \textbf{a}^T\cdot \mathbf{e_1} ) = 2\cdot \textbf{a}\cdot \mathbf{e_1}^T
\end{equation}

    \noindent On solving the quadratic for x, 
\begin{equation}
x = \textbf{a}\cdot \mathbf{e_1}^T \pm \sqrt{(\textbf{a}\cdot \mathbf{e_1}^T)^2 - ||\textbf{a}||^2 + 20}
\end{equation}

    \noindent On substituting the values of \textbf{a} and $\mathbf{e_1}$,
\begin{equation}
x = 7 \pm \sqrt{7^2 - 7^2 - 4^2 + 20}
\end{equation}
\end{frame}

\begin{frame}
\begin{equation}
x = 7 \pm 2 = 9 , 5
\end{equation}

   \noindent Hence, there exist two values if x i.e, there exist two ponits P on the X axis for the distance between the given point and P to be $2\sqrt{5}$ 

\begin{figure}[h]
    \centering
    \includegraphics[width=0.5\linewidth]{figs/image.png}
    \caption{The plot of the points A and the two points on the X axis}
\end{figure}
\end{frame}

\begin{frame}[fragile]{Code - C}
\begin{lstlisting}
#include <math.h>
#include <stdio.h>

void find_x_axis_points(double x0, double y0, double d, double *x1, double *x2) {
    double rhs = d*d - y0*y0;
    if (rhs < 0) {
        *x1 = NAN;
        *x2 = NAN;
        return;
    }
    double root = sqrt(rhs);
    *x1 = x0 + root;
    *x2 = x0 - root;
}

int main() {
    double x1, x2;
    find_x_axis_points(7.0, -4.0, 2.0*sqrt(5.0), &x1, &x2);
    printf("X-axis points: (%.2f, 0) and (%.2f, 0)\n", x1, x2);
    return 0;
}    
\end{lstlisting}  
\end{frame}



\end{document}
