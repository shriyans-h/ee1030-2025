\let\negmedspace\undefined
\let\negthickspace\undefined
\documentclass[journal,12pt,onecolumn]{IEEEtran}
\usepackage{cite}
\usepackage{amsmath,amssymb,amsfonts,amsthm}
\usepackage{algorithmic}
\usepackage{graphicx}
\graphicspath{{./figs/}}
\usepackage{textcomp}
\usepackage{xcolor}
\usepackage{txfonts}
\usepackage{listings}
\usepackage{enumitem}
\usepackage{mathtools}
\usepackage{gensymb}
\usepackage{comment}
\usepackage{caption}
\usepackage[breaklinks=true]{hyperref}
\usepackage{tkz-euclide} 
\usepackage{listings}
\usepackage{gvv}                                        
\usepackage{amsmath}                                 
\usepackage[latin1]{inputenc}     
\usepackage{xparse}
\usepackage{color}                                            
\usepackage{array}                                            
\usepackage{longtable}                                       
\usepackage{calc}                                             
\usepackage{multirow}
\usepackage{multicol}
\usepackage{hhline}                                           
\usepackage{ifthen}                                           
\usepackage{lscape}
\usepackage{tabularx}
\usepackage{array}
\usepackage{float}

\begin{document}

\title{1.4.13}
\author{AI25BTECH11038 -- Tejas Uppala}
{\let\newpage\relax\maketitle}

\textbf{Question:} 

   \noindent Given that \textbf{P}$\brak{3, 2, {-4}}$, \textbf{Q}$\brak{5, 4, {-6}}$ and \textbf{R}$\brak{9,8,{-10}}$ are collinear. Find the ratio in which \textbf{Q} divides PR.

\textbf{Solution:}

\begin{equation}
P=\myvec{3\\2\\-4},\qquad
Q=\myvec{5\\4\\-6},\qquad
R=\myvec{9\\8\\-10}
\end{equation}

\noindent From the section formula,

\begin{equation}
\textbf{Q} = \frac{k\textbf{P} + \textbf{R}}{k + 1} 
\end{equation}

\noindent for some scalar $k$. Where \textbf{Q} divides PR in the ratio $k : 1$. \\
\noindent From equation (1): 

\begin{equation}
 \brak{\textbf{R}-\textbf{P}}t=\brak{\textbf{Q}-\textbf{P}}  
\end{equation}

\begin{equation}
   k = \frac{\brak{\textbf{Q}-\textbf{P}}\brak{\textbf{R}-\textbf{P}}^T}{||\textbf{R}-\textbf{P}||^2}
\end{equation}

\begin{equation}
\brak{\textbf{R}-\textbf{P}}=\myvec{6\\6\\-6},\qquad
\brak{\textbf{Q}-\textbf{P}}=\myvec{2\\2\\-2}.
\end{equation}

\begin{equation}
\mathbf{R}-\mathbf{P} = \myvec{6 \\ 6 \\ -6}, \qquad 
\mathbf{Q}-\mathbf{P} = \myvec{2 \\ 2 \\ -2}
\end{equation}

\begin{equation}
\brak{\mathbf{Q}-\mathbf{P}}\brak{\mathbf{R}-\mathbf{P}}^T
= \myvec{2 & 2 & -2}\myvec{6 \\ 6 \\ -6}
= 2\cdot 6 + 2\cdot 6 + (-2)(-6) 
= 36
\end{equation}

\begin{equation}
\brak{\mathbf{R}-\mathbf{P}}\brak{\mathbf{R}-\mathbf{P}}^T
= \myvec{6 & 6 & -6}\myvec{6 \\ 6 \\ -6}
= 6^2 + 6^2 + \brak{-6}^2
= 108
\end{equation}

\begin{equation}
\therefore \quad k = \frac{36}{108} = \frac{1}{3}
\end{equation}

\begin{equation}
\text{Thus, } {P}{Q}:{Q}{R} = k:\brak{1-k} = \tfrac{1}{3}:\tfrac{2}{3} = 1:2
\end{equation}

\begin{figure}[h]
    \centering
    \includegraphics[width=0.5\linewidth]{figs/points_plot.png}
    \caption{Plot of the points P, Q and R}
\end{figure}


\end{document}