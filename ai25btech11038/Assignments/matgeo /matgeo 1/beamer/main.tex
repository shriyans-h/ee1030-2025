\documentclass{beamer}
\mode<presentation>
\usepackage{amsmath}
\usepackage{amssymb}
\usepackage[utf8]{inputenc}
\usepackage{fontenc}
%\usepackage{advdate}
\usepackage{adjustbox}
\usepackage{subcaption}
\usepackage{enumitem}
\usepackage{multicol}
\usepackage{mathtools}
\usepackage{listings}
\usepackage{url}
\def\UrlBreaks{\do\/\do-}
\usetheme{Boadilla}
\usecolortheme{lily}
\setbeamertemplate{footline}
{
  \leavevmode%
  \hbox{%
  \begin{beamercolorbox}[wd=\paperwidth,ht=2.25ex,dp=1ex,right]{author in head/foot}%
    \insertframenumber{} / \inserttotalframenumber\hspace*{2ex} 
  \end{beamercolorbox}}%
  \vskip0pt%
}
\setbeamertemplate{navigation symbols}{}

\providecommand{\nCr}[2]{\,^{#1}C_{#2}} % nCr
\providecommand{\nPr}[2]{\,^{#1}P_{#2}} % nPr
\providecommand{\mbf}{\mathbf}
\providecommand{\pr}[1]{\ensuremath{\Pr\left(#1\right)}}
\providecommand{\qfunc}[1]{\ensuremath{Q\left(#1\right)}}
\providecommand{\sbrak}[1]{\ensuremath{{}\left[#1\right]}}
\providecommand{\lsbrak}[1]{\ensuremath{{}\left[#1\right.}}
\providecommand{\rsbrak}[1]{\ensuremath{{}\left.#1\right]}}
\providecommand{\brak}[1]{\ensuremath{\left(#1\right)}}
\providecommand{\lbrak}[1]{\ensuremath{\left(#1\right.}}
\providecommand{\rbrak}[1]{\ensuremath{\left.#1\right)}}
\providecommand{\cbrak}[1]{\ensuremath{\left\{#1\right\}}}
\providecommand{\lcbrak}[1]{\ensuremath{\left\{#1\right.}}
\providecommand{\rcbrak}[1]{\ensuremath{\left.#1\right\}}}
\theoremstyle{remark}
\newtheorem{rem}{Remark}
\newcommand{\sgn}{\mathop{\mathrm{sgn}}}
%\providecommand{\abs}[1]{\left\vert#1\right\vert}
\providecommand{\res}[1]{\Res\displaylimits_{#1}} 
\providecommand{\norm}[1]{\lVert#1\rVert}
\providecommand{\mtx}[1]{\mathbf{#1}}
%\providecommand{\mean}[1]{E\left[ #1 \right]}
\providecommand{\fourier}{\overset{\mathcal{F}}{ \rightleftharpoons}}
%\providecommand{\hilbert}{\overset{\mathcal{H}}{ \rightleftharpoons}}
\providecommand{\system}{\overset{\mathcal{H}}{ \longleftrightarrow}}
	%\newcommand{\solution}[2]{\textbf{Solution:}{#1}}
%\newcommand{\solution}{\noindent \textbf{Solution: }}
\providecommand{\dec}[2]{\ensuremath{\overset{#1}{\underset{#2}{\gtrless}}}}
\newcommand{\myvec}[1]{\ensuremath{\begin{pmatrix}#1\end{pmatrix}}}
\let\vec\mathbf

\lstset{
language=C,
frame=single, 
breaklines=true,
columns=fullflexible
}

\numberwithin{equation}{section}

\title{Presentation - Matgeo}
\author{Tejas Uppala - AI25BTECH11038}

\begin{document}

\begin{frame}
\titlepage
\end{frame}

\section*{Outline}


\begin{frame}
\frametitle{Problem Statement}
Given that \textbf{P}$\brak{3, 2, {-4}}$, \textbf{Q}$\brak{5, 4, {-6}}$ and \textbf{R}$\brak{9,8,{-10}}$ are collinear. Find the ratio in which \textbf{Q} divides PR.
\end{frame}

\begin{frame}
\frametitle{Solution:}
Given that, P , Q and R are 3 coincident points 
\begin{equation}
P=\myvec{3\\2\\-4},\qquad
Q=\myvec{5\\4\\-6},\qquad
R=\myvec{9\\8\\-10}
\end{equation}

From the section formula,
\begin{equation}
\textbf{Q} = \frac{k\textbf{P} + \textbf{R}}{k + 1} 
\end{equation}

for some scalar $k$. Where \textbf{Q} divides PR in the ratio $k : 1$. \\
From equation 1.2 : 
\begin{equation}
 \brak{\textbf{R}-\textbf{P}}k=\brak{\textbf{Q}-\textbf{P}}  
\end{equation}
\begin{equation}
   k = \frac{\brak{\textbf{Q}-\textbf{P}}\brak{\textbf{R}-\textbf{P}}^T}{||\textbf{R}-\textbf{P}||^2}
\end{equation}
\end{frame}

\begin{frame}
\frametitle{Calculation}
\begin{equation}
\brak{\textbf{R}-\textbf{P}}=\myvec{6\\6\\-6},\qquad
\brak{\textbf{Q}-\textbf{P}}=\myvec{2\\2\\-2}.
\end{equation}
\begin{equation}
\brak{\mathbf{Q}-\mathbf{P}}\brak{\mathbf{R}-\mathbf{P}}^T
= \myvec{2 & 2 & -2}\myvec{6 \\ 6 \\ -6}
= 2\cdot 6 + 2\cdot 6 + (-2)(-6) 
= 36
\end{equation}
\begin{equation}
\brak{\mathbf{R}-\mathbf{P}}\brak{\mathbf{R}-\mathbf{P}}^T
= \myvec{6 & 6 & -6}\myvec{6 \\ 6 \\ -6}
= 6^2 + 6^2 + \brak{-6}^2
= 108
\end{equation}
\begin{equation}
\therefore \quad k = \frac{36}{108} = \frac{1}{3}
\end{equation}
\end{frame}

\begin{frame}
\frametitle{Result}
Thus,
\begin{equation}
 {P}{Q}:{Q}{R} = k:\brak{1-k} = \tfrac{1}{3}:\tfrac{2}{3} = 1:2
\end{equation}
The line PR is divided in the ratio $1 : 2$ by the point Q.
\end{frame}

\begin{frame}
\frametitle{Plot}
 \begin{figure}[h]
    \centering
    \includegraphics[width=0.5\linewidth]{figs/points_plot.png}
    \caption{Plot of the points P, Q and R}
\end{figure}   
\end{frame}

\begin{frame}[fragile]{Code - C}{Helper Functions}
\begin{lstlisting}
#include <stdio.h>
#include <math.h>

float dotProduct(float a[3], float b[3]) {
    return a[0]*b[0] + a[1]*b[1] + a[2]*b[2];
}

float normSquared(float a[3]) {
    return dotProduct(a, a);
}
\end{lstlisting}
\end{frame}

\begin{frame}[fragile]{Code-C}{findK Function}
\begin{lstlisting}
float findK(float P[3], float Q[3], float R[3]) {
    float QP[3], RP[3];

    for (int i = 0; i < 3; i++) {
        QP[i] = Q[i] - P[i];
        RP[i] = R[i] - P[i];
    }
    float numerator = dotProduct(QP, RP);
    float denominator = normSquared(RP);

    if (denominator == 0) {
        return 0.0f;   // if P and R are same point, k is undefined - return 0
    }
    return numerator / denominator;
}
\end{lstlisting}
\end{frame}

\begin{frame}[fragile]{Code - Python}
\begin{lstlisting}
import matplotlib.pyplot as plt
from mpl_toolkits.mplot3d import Axes3D

# Points
P = (3, 2, -4)
Q = (5, 4, -6)
R = (9, 8, -10)

# Plotting
fig = plt.figure()
ax = fig.add_subplot(111, projection='3d')

# Scatter points
ax.scatter(*P, color='red', label='P(3,2,-4)')
ax.scatter(*Q, color='blue', label='Q(5,4,-6)')
ax.scatter(*R, color='green', label='R(9,8,-10)')
\end{lstlisting}
\end{frame}

\begin{frame}[fragile]{Code-Python}
\begin{lstlisting}
# Line PR (whole line)
ax.plot([P[0], R[0]], [P[1], R[1]], [P[2], R[2]], color='black', linestyle='--', label='PR')

# Subsegments PQ and QR
ax.plot([P[0], Q[0]], [P[1], Q[1]], [P[2], Q[2]], color='red', linewidth=2, label='PQ')
ax.plot([Q[0], R[0]], [Q[1], R[1]], [Q[2], R[2]], color='green', linewidth=2, label='QR')

# Labels on points
ax.text(*P, "P", color='red')
ax.text(*Q, "Q", color='blue')
ax.text(*R, "R", color='green')
\end{lstlisting}
\end{frame}

\begin{frame}[fragile]{Code-Python}
\begin{lstlisting}
# Axis labels
ax.set_xlabel("X-axis")
ax.set_ylabel("Y-axis")
ax.set_zlabel("Z-axis")
ax.legend()

plt.show()

plt.savefig('../figs/img.png')
\end{lstlisting}
\end{frame}

\end{document}
