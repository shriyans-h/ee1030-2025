\documentclass{beamer}
\usepackage[utf8]{inputenc}

\usetheme{Madrid}
\usecolortheme{default}
\usepackage{amsmath,amssymb,amsfonts,amsthm}
\usepackage{mathtools}
\usepackage{txfonts}
\usepackage{tkz-euclide}
\usepackage{listings}
\usepackage{adjustbox}
\usepackage{array}
\usepackage{gensymb}
\usepackage{tabularx}
\usepackage{gvv}
\usepackage{lmodern}
\usepackage{circuitikz}
\usepackage{tikz}
\lstset{literate={·}{{$\cdot$}}1 {λ}{{$\lambda$}}1 {→}{{$\to$}}1}
\usepackage{graphicx}

\setbeamertemplate{page number in head/foot}[totalframenumber]

\usepackage{tcolorbox}
\tcbuselibrary{minted,breakable,xparse,skins}



\definecolor{bg}{gray}{0.95}
\DeclareTCBListing{mintedbox}{O{}m!O{}}{%
  breakable=true,
  listing engine=minted,
  listing only,
  minted language=#2,
  minted style=default,
  minted options={%
    linenos,
    gobble=0,
    breaklines=true,
    breakafter=,,
    fontsize=\small,
    numbersep=8pt,
    #1},
  boxsep=0pt,
  left skip=0pt,
  right skip=0pt,
  left=25pt,
  right=0pt,
  top=3pt,
  bottom=3pt,
  arc=5pt,
  leftrule=0pt,
  rightrule=0pt,
  bottomrule=2pt,
  toprule=2pt,
  colback=bg,
  colframe=orange!70,
  enhanced,
  overlay={%
    \begin{tcbclipinterior}
    \fill[orange!20!white] (frame.south west) rectangle ([xshift=20pt]frame.north west);
    \end{tcbclipinterior}},
  #3,
}
\lstset{
    language=C,
    basicstyle=\ttfamily\small,
    keywordstyle=\color{blue},
    stringstyle=\color{orange},
    commentstyle=\color{green!60!black},
    numbers=left,
    numberstyle=\tiny\color{gray},
    breaklines=true,
    showstringspaces=false,
}
%------------------------------------------------------------
%This block of code defines the information to appear in the
%Title page
\title %optional
{8.4.29}
\date{September 9,2025}
%\subtitle{A short story}

\author % (optional)
{Harsha-EE25BTECH11026}



\begin{document}


\frame{\titlepage}


\begin{frame}{Question}
A hyperbola, having the transverse axis of length $2\sin{\theta}$, is confocal with the ellipse $3x^2+4y^2=12$. Then the equation is
\begin{enumerate}
    \item $x^2\csc^2{\theta}-y^2\sec^2{\theta}=1$
    \item $x^2\sec^2{\theta}-y^2\csc^2{\theta}=1$
    \item $x^2\sin^2{\theta}-y^2\cos^2{\theta}=1$
    \item $x^2\cos^2{\theta}-y^2\sin^2{\theta}=1$
\end{enumerate}
\end{frame}

\begin{frame}{Theoretical Solution}
From the data given,
\begin{align}
    \text{Equation of ellipse is given by : }\vec{x}^{\top}\vec{M_e}\vec{x}=1
\end{align}
where,
\begin{align}
    \vec{M_e}=\myvec{\tfrac{1}{4}&&0\\0&&\tfrac{1}{3}}
\end{align}
Focal length of the ellipse $\brak{f}$ is given by,
\begin{align}
    f_e^2=\frac{\lambda_2-\lambda_1}{\|M_e\|}
\end{align}
where, $\lambda_1$ and $\lambda_2$ are the eigen values of the matrix $\vec{M_e}$.For a diagonal matrix it's eigen values are given by their diagonal elements.
\end{frame}

\begin{frame}{Theoretical Solution}
\begin{align}
    \therefore f_e^2=\frac{\frac{1}{3}-\frac{1}{4}}{\frac{1}{12}}=1
    \implies f_e=1
\end{align}
As ellipse and hyperbola are confocal,their focal lengths are same.Let the equation of hyperbola be
\begin{align}
    \vec{x}^{\top}\vec{M_H}\vec{x}=1
\end{align}
where,
\begin{align}
    \vec{M_H}=\myvec{\mu_1&&0\\0&&\mu_2}
\end{align}
where $\mu_1$ and $\mu_2$ are the eigen values of matrix $\vec{M_H}$.
\end{frame}

\begin{frame}{Theoretical Solution}
The focal length of hyperbola $f_H$ is given by,
\begin{align}
    f_H^2=-\frac{\mu_1-\mu_2}{\|M_H\|}
\end{align}
As the value of transverse axis is $2\sin{\theta}$,
\begin{align}
    \mu_1=\csc^2{\theta}
\end{align}
Also,
\begin{align}
    \mu_2-\mu_1=\mu_1\mu_2
\end{align}
\begin{align}
    \implies \mu_2=-sec^2{\theta}
\end{align}
Thus, the desired equation is
\begin{align}
    \vec{x}^{\top}\vec{M_H}\vec{x}=1
\end{align}
where, $\vec{M_H}=\myvec{\csc^2{\theta}&&0\\0&&-\sec^2{\theta}}$.
\end{frame}

\begin{frame}[fragile]
    \frametitle{C Code -Finding the equation of hyperbola}

    \begin{lstlisting}[language=C]
#include <stdio.h>
#include <math.h>

void hyperbola_params(double theta, double *arr) {
    double a = sin(theta);   // transverse semi-axis
    double b = cos(theta);   // conjugate semi-axis
    arr[0] = a * a;          // a^2
    arr[1] = b * b;          // b^2
}
    \end{lstlisting}
\end{frame}



\begin{frame}[fragile]
    \frametitle{Python+C code}

    \begin{lstlisting}[language=Python]
import ctypes
import numpy as np
import matplotlib.pyplot as plt
from math import pi, sqrt

# Load the shared library
lib = ctypes.CDLL("./libconfocal.so")

# Define argument and return types for the C function
lib.hyperbola_params.argtypes = [ctypes.c_double, ctypes.POINTER(ctypes.c_double)]
lib.hyperbola_params.restype = None

def get_hyperbola_params(theta):
    arr = (ctypes.c_double * 2)()
    lib.hyperbola_params(theta, arr)
    return arr[0], arr[1]   # returns (a^2, b^2)

    \end{lstlisting}
\end{frame}

\begin{frame}[fragile]
    \frametitle{Python+C code}

    \begin{lstlisting}[language=Python]
    
def plot_confocal(theta=pi/6):
    # Ellipse parameters
    a_e = 2.0
    b_e = sqrt(3.0)
    c = sqrt(a_e*a_e - b_e*b_e)   # foci distance = 1
    # Get hyperbola parameters from C
    a2, b2 = get_hyperbola_params(theta)
    a_h = sqrt(a2)
    b_h = sqrt(b2)
    # Print hyperbola equation
    print(f"Hyperbola: x^2/{a2:.3f} - y^2/{b2:.3f} = 1")
    # Ellipse parametric curve
    t = np.linspace(0, 2*np.pi, 800)
    x_ell = a_e * np.cos(t)
    y_ell = b_e * np.sin(t)


    \end{lstlisting}
\end{frame}

\begin{frame}[fragile]
    \frametitle{Python+C code}

    \begin{lstlisting}[language=Python]
    # Hyperbola parametric curve
    u = np.linspace(-1.2, 1.2, 600)
    sec_u = 1.0/np.cos(u)
    tan_u = np.tan(u)
    x_h_pos = a_h * sec_u
    y_h_pos = b_h * tan_u
    x_h_neg = -a_h * sec_u
    y_h_neg = b_h * tan_u
    mask_pos = np.isfinite(x_h_pos) & np.isfinite(y_h_pos) & (np.abs(x_h_pos)<50) & (np.abs(y_h_pos)<50)
    mask_neg = np.isfinite(x_h_neg) & np.isfinite(y_h_neg) & (np.abs(x_h_neg)<50) & (np.abs(y_h_neg)<50)

    \end{lstlisting}
\end{frame}

\begin{frame}[fragile]
    \frametitle{Python+C code}

    \begin{lstlisting}[language=Python]
    fig, ax = plt.subplots(figsize=(8,6))
    ax.plot(x_ell, y_ell, label="Ellipse")
    ax.plot(x_h_pos[mask_pos], y_h_pos[mask_pos], linestyle='--', label="Hyperbola (right branch)")
    ax.plot(x_h_neg[mask_neg], y_h_neg[mask_neg], linestyle='--', label="Hyperbola (left branch)")
    ax.scatter([c, -c], [0,0], marker='x', s=80, label="Foci")
    ax.set_aspect('equal', 'box')
    ax.grid(True)
    ax.set_xlabel('x')
    ax.set_ylabel('y')
    ax.set_title('Ellipse and Confocal Hyperbola (via C + Python)')
    ax.legend()
    plt.savefig("/home/user/Matrix Theory: workspace/Matgeo_assignments/8.4.29/figs/figure_1.png")
    plt.show()
plot_confocal(theta=pi/6)
    \end{lstlisting}
\end{frame}


\begin{frame}[fragile]
    \frametitle{Python code}
    \begin{lstlisting}[language=Python]
import numpy as np
import matplotlib.pyplot as plt
from math import sin, cos, pi, sqrt

def plot_confocal_hyperbola(theta=pi/6):
    # Ellipse parameters
    a_e = 2.0
    b_e = sqrt(3.0)
    c = sqrt(a_e*a_e - b_e*b_e)   # foci distance = 1
    
    # Hyperbola parameters
    a_h = abs(sin(theta))  # transverse semi-axis
    b_h = abs(cos(theta))  # conjugate semi-axis
    if a_h == 0:
        raise ValueError("theta leads to a_h = 0 (sin(theta)=0). Choose a different theta.")
    \end{lstlisting}   
\end{frame}

\begin{frame}[fragile]
    \frametitle{Python code}
    \begin{lstlisting}[language=Python]
 eq_hyperbola = f"Hyperbola: x^2/{a_h**2:.3f} - y^2/{b_h**2:.3f} = 1"
    print(eq_hyperbola)

    # Parametric ellipse
    t = np.linspace(0, 2*np.pi, 800)
    x_ell = a_e * np.cos(t)
    y_ell = b_e * np.sin(t)

    # Parametric hyperbola branches
    u = np.linspace(-1.2, 1.2, 600)
    sec_u = 1.0/np.cos(u)
    tan_u = np.tan(u)
    x_h_pos = a_h * sec_u
    y_h_pos = b_h * tan_u
    x_h_neg = -a_h * sec_u
    y_h_neg = b_h * tan_u

    \end{lstlisting}   
\end{frame}

\begin{frame}[fragile]
    \frametitle{Python code}
    \begin{lstlisting}[language=Python]
mask_pos = np.isfinite(x_h_pos) & np.isfinite(y_h_pos) & (np.abs(x_h_pos)<50) & (np.abs(y_h_pos)<50)
    mask_neg = np.isfinite(x_h_neg) & np.isfinite(y_h_neg) & (np.abs(x_h_neg)<50) & (np.abs(y_h_neg)<50)

    # Plot
    fig, ax = plt.subplots(figsize=(8,6))
    ax.plot(x_ell, y_ell, label="Ellipse")
    ax.plot(x_h_pos[mask_pos], y_h_pos[mask_pos], linestyle='--', label="Hyperbola (right branch)")
    ax.plot(x_h_neg[mask_neg], y_h_neg[mask_neg], linestyle='--', label="Hyperbola (left branch)")
    ax.scatter([c, -c], [0,0], marker='x', s=80, label="Foci")
    \end{lstlisting}   
\end{frame}

\begin{frame}[fragile]
    \frametitle{Python code}
    \begin{lstlisting}[language=Python]
ax.set_aspect('equal', 'box')
    ax.grid(True)
    ax.set_xlabel('x')
    ax.set_ylabel('y')
    ax.set_title('Ellipse and Confocal Hyperbola')
    ax.legend()
    plt.savefig("/home/user/Matrix Theory: workspace/Matgeo_assignments/8.4.29/figs/Figure_1.png")
    plt.show()

# Example run with theta = pi/6
plot_confocal_hyperbola(theta=pi/6)

    \end{lstlisting}   
\end{frame}

\begin{frame}{Plot}
    \begin{figure}[H]
    \centering
    \includegraphics[width=0.8\columnwidth]{figs/Figure_1.png}
    \label{fig:1}
\end{figure}
\end{frame}

\end{document}