\let\negmedspace\undefined
\let\negthickspace\undefined
\documentclass[journal]{IEEEtran}
\usepackage[a5paper, margin=10mm, onecolumn]{geometry}
%\usepackage{lmodern} % Ensure lmodern is loaded for pdflatex
\usepackage{tfrupee} % Include tfrupee package

\setlength{\headheight}{1cm} % Set the height of the header box
\setlength{\headsep}{0mm}     % Set the distance between the header box and the top of the text

\usepackage{gvv-book}
\usepackage{gvv}
\usepackage{cite}
\usepackage{amsmath,amssymb,amsfonts,amsthm}
\usepackage{algorithmic}
\usepackage{graphicx}
\usepackage{textcomp}
\usepackage{xcolor}
\usepackage{txfonts}
\usepackage{listings}
\usepackage{enumitem}
\usepackage{mathtools}
\usepackage{gensymb}
\usepackage[breaklinks=true]{hyperref}
\usepackage{tkz-euclide} 
\usepackage{listings}
% \usepackage{gvv}                                        
\def\inputGnumericTable{}                                 
\usepackage[latin1]{inputenc}                                
\usepackage{color}                                            
\usepackage{array}                                            
\usepackage{longtable}                                       
\usepackage{calc}                                             
\usepackage{multirow}                                         
\usepackage{hhline}                                           
\usepackage{ifthen}                                           
\usepackage{lscape}
\usepackage{circuitikz}
\usepackage{comment}
\tikzstyle{block} = [rectangle, draw, fill=blue!20, 
    text width=4em, text centered, rounded corners, minimum height=3em]
\tikzstyle{sum} = [draw, fill=blue!10, circle, minimum size=1cm, node distance=1.5cm]
\tikzstyle{input} = [coordinate]
\tikzstyle{output} = [coordinate]


\begin{document}

\bibliographystyle{IEEEtran}
\vspace{3cm}

\title{8.4.29}
\author{EE25BTECH11026-Harsha}
 \maketitle
% \newpage
% \bigskip
{\let\newpage\relax\maketitle}

\renewcommand{\thefigure}{\theenumi}
\renewcommand{\thetable}{\theenumi}
\setlength{\intextsep}{10pt} % Space between text and floats


\numberwithin{equation}{enumi}
\numberwithin{figure}{enumi}
\renewcommand{\thetable}{\theenumi}

\textbf{Question}:\\
A hyperbola, having the transverse axis of length $2\sin{\theta}$, is confocal with the ellipse $3x^2+4y^2=12$. Then the equation is
\begin{multicols}{2}
\begin{enumerate}
    \item $x^2\csc^2{\theta}-y^2\sec^2{\theta}=1$
    \item $x^2\sec^2{\theta}-y^2\csc^2{\theta}=1$
    \item $x^2\sin^2{\theta}-y^2\cos^2{\theta}=1$
    \item $x^2\cos^2{\theta}-y^2\sin^2{\theta}=1$
\end{enumerate}
\end{multicols}
\solution \\
Let us solve the given question theoretically and then verify the solution computationally.\\
\\
From the data given,
\begin{align}
    \text{Equation of ellipse is given by : }\vec{x}^{\top}\vec{M_e}\vec{x}=1
\end{align}
where,
\begin{align}
    \vec{M_e}=\myvec{\tfrac{1}{4}&&0\\0&&\tfrac{1}{3}}
\end{align}
Focal length of the ellipse $\brak{f}$ is given by,
\begin{align}
    f_e^2=\frac{\lambda_2-\lambda_1}{\|M_e\|}
\end{align}
where, $\lambda_1$ and $\lambda_2$ are the eigen values of the matrix $\vec{M_e}$.For a diagonal matrix it's eigen values are given by their diagonal elements.
\begin{align}
    \therefore f_e^2=\frac{\frac{1}{3}-\frac{1}{4}}{\frac{1}{12}}=1
    \implies f_e=1
\end{align}
As ellipse and hyperbola are confocal,their focal lengths are same.Let the equation of hyperbola be
\begin{align}
    \vec{x}^{\top}\vec{M_H}\vec{x}=1
\end{align}
where,
\begin{align}
    \vec{M_H}=\myvec{\mu_1&&0\\0&&\mu_2}
\end{align}
where $\mu_1$ and $\mu_2$ are the eigen values of matrix $\vec{M_H}$.
\newpage
\vspace*{0.25cm}
The focal length of hyperbola $f_H$ is given by,
\begin{align}
    f_H^2=-\frac{\mu_1-\mu_2}{\|M_H\|}
\end{align}
As the value of transverse axis is $2\sin{\theta}$,
\begin{align}
    \mu_1=\csc^2{\theta}
\end{align}
Also,
\begin{align}
    \mu_2-\mu_1=\mu_1\mu_2
\end{align}
\begin{align}
    \implies \mu_2=-sec^2{\theta}
\end{align}
Thus, the desired equation is
\begin{align}
    \vec{x}^{\top}\vec{M_H}\vec{x}=1
\end{align}
where, $\vec{M_H}=\myvec{\csc^2{\theta}&&0\\0&&-\sec^2{\theta}}$.\\
\\
From the figure, it is clearly verified that the theoretical solution matches with the computational solution.\\
\begin{figure}[H]
    \centering
    \includegraphics[width=0.8\columnwidth]{figs/Figure_1.png}
    \label{fig:1}
\end{figure}


\end{document}
