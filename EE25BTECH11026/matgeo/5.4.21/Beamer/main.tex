\documentclass{beamer}
\usepackage[utf8]{inputenc}

\usetheme{Madrid}
\usecolortheme{default}
\usepackage{amsmath,amssymb,amsfonts,amsthm}
\usepackage{mathtools}
\usepackage{txfonts}
\usepackage{tkz-euclide}
\usepackage{listings}
\usepackage{adjustbox}
\usepackage{array}
\usepackage{gensymb}
\usepackage{tabularx}
\usepackage{gvv}
\usepackage{lmodern}
\usepackage{circuitikz}
\usepackage{tikz}
\lstset{literate={·}{{$\cdot$}}1 {λ}{{$\lambda$}}1 {→}{{$\to$}}1}
\usepackage{graphicx}

\setbeamertemplate{page number in head/foot}[totalframenumber]

\usepackage{tcolorbox}
\tcbuselibrary{minted,breakable,xparse,skins}



\definecolor{bg}{gray}{0.95}
\DeclareTCBListing{mintedbox}{O{}m!O{}}{%
  breakable=true,
  listing engine=minted,
  listing only,
  minted language=#2,
  minted style=default,
  minted options={%
    linenos,
    gobble=0,
    breaklines=true,
    breakafter=,,
    fontsize=\small,
    numbersep=8pt,
    #1},
  boxsep=0pt,
  left skip=0pt,
  right skip=0pt,
  left=25pt,
  right=0pt,
  top=3pt,
  bottom=3pt,
  arc=5pt,
  leftrule=0pt,
  rightrule=0pt,
  bottomrule=2pt,
  toprule=2pt,
  colback=bg,
  colframe=orange!70,
  enhanced,
  overlay={%
    \begin{tcbclipinterior}
    \fill[orange!20!white] (frame.south west) rectangle ([xshift=20pt]frame.north west);
    \end{tcbclipinterior}},
  #3,
}
\lstset{
    language=C,
    basicstyle=\ttfamily\small,
    keywordstyle=\color{blue},
    stringstyle=\color{orange},
    commentstyle=\color{green!60!black},
    numbers=left,
    numberstyle=\tiny\color{gray},
    breaklines=true,
    showstringspaces=false,
}
%------------------------------------------------------------
%This block of code defines the information to appear in the
%Title page
\title %optional
{5.4.21}
\date{September 5,2025}
%\subtitle{A short story}

\author % (optional)
{Harsha-EE25BTECH11026}



\begin{document}


\frame{\titlepage}


\begin{frame}{Question}
Using elementary transformations, find the inverse of the following matrix. 
\begin{align*}
    \myvec{2&&-6\\1&&-2}
\end{align*}
\end{frame}

\begin{frame}{Theoretical Solution}
To solve for the inverse of a matrix, we can employ the Gauss-Jordan approach.
\begin{align}
\begin{aligned}
  \augvec{2}{2}{2 & -6 & 1 & 0\\ 1 & -2 & 0 & 1}
  \xleftrightarrow[\,R_2 \gets R_2- R_1]{\,R_1 \gets \tfrac{R_1}{2}}
  \augvec{2}{2}{1 & -3 & \tfrac{1}{2} & 0\\ 0 & 1 & -\tfrac{1}{2} & 1} \\
  \xleftrightarrow{\,R_1 \gets R_1+ 3R_2}
  \augvec{2}{2}{1 & 0 & -1 & 3 \\ 0 & 1 & -\tfrac{1}{2} & 1}
\end{aligned}
\end{align}
\begin{align}
    \therefore \text{Inverse of the given Matrix:}\myvec{-1&&3\\-\frac{1}{2}&&1}
\end{align}
\end{frame}



\begin{frame}[fragile]
    \frametitle{C Code -Finding Inverse of a Matrix}

    \begin{lstlisting}[language=C]
#include <stdio.h>

#define N 2   // matrix size (you can generalize)

void inverse(double A[N][N], double inv[N][N]) {
    // Step 1: Create augmented matrix [A|I]
    double aug[N][2*N];
    for (int i = 0; i < N; i++) {
        for (int j = 0; j < N; j++) {
            aug[i][j] = A[i][j];          // copy A
            aug[i][j+N] = (i == j) ? 1 : 0; // identity
        }
    }
    \end{lstlisting}
\end{frame}

\begin{frame}[fragile]
    \frametitle{C Code -Finding Inverse of a Matrix}

    \begin{lstlisting}[language=C]
  // Step 2: Gauss–Jordan elimination
    for (int i = 0; i < N; i++) {
        // Make pivot = 1
        double pivot = aug[i][i];
        for (int j = 0; j < 2*N; j++) {
            aug[i][j] /= pivot;
        }

        // Eliminate other rows
        for (int k = 0; k < N; k++) {
            if (k != i) {
                double factor = aug[k][i];
                for (int j = 0; j < 2*N; j++) {
                    aug[k][j] -= factor * aug[i][j];
                }
            }
        }
    }


    \end{lstlisting}
\end{frame}

\begin{frame}[fragile]
    \frametitle{C Code -Finding Inverse of a Matrix}

    \begin{lstlisting}[language=C]
    // Step 3: Extract inverse from augmented matrix
    for (int i = 0; i < N; i++) {
        for (int j = 0; j < N; j++) {
            inv[i][j] = aug[i][j+N];
        }
    }
}
    \end{lstlisting}
\end{frame}

\begin{frame}[fragile]
    \frametitle{Python+C code}

    \begin{lstlisting}[language=Python]
import ctypes
import numpy as np
import sympy as sp

# Load C library
lib = ctypes.CDLL("./libinverse.so")

# Define function signature
lib.inverse.argtypes = [ctypes.POINTER((ctypes.c_double * 2) * 2),
                        ctypes.POINTER((ctypes.c_double * 2) * 2)]

# Input matrix
A = np.array([[2, -6],
              [1, -2]], dtype=np.double)

inv = np.zeros((2,2), dtype=np.double)



    \end{lstlisting}
\end{frame}

\begin{frame}[fragile]
    \frametitle{Python+C code}

    \begin{lstlisting}[language=Python]
# Call C function
lib.inverse(A.ctypes.data_as(ctypes.POINTER((ctypes.c_double * 2) * 2)),
            inv.ctypes.data_as(ctypes.POINTER((ctypes.c_double * 2) * 2)))

inverse=sp.Matrix(inv)
sp.pprint(inverse)
    \end{lstlisting}
\end{frame}


\begin{frame}[fragile]
    \frametitle{Python code}
    \begin{lstlisting}[language=Python]
import sympy as sp

A = sp.Matrix([[2, -6], [1, -2]])
A_inv = A.inv()
sp.pprint(A_inv) 
    \end{lstlisting}   
\end{frame}

\end{document}