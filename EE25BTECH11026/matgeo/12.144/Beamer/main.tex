\documentclass{beamer}
\usepackage[utf8]{inputenc}

\usetheme{Madrid}
\usecolortheme{default}
\usepackage{amsmath,amssymb,amsfonts,amsthm}
\usepackage{mathtools}
\usepackage{txfonts}
\usepackage{tkz-euclide}
\usepackage{listings}
\usepackage{adjustbox}
\usepackage{array}
\usepackage{gensymb}
\usepackage{tabularx}
\usepackage{gvv}
\usepackage{lmodern}
\usepackage{circuitikz}
\usepackage{tikz}
\lstset{literate={·}{{$\cdot$}}1 {λ}{{$\lambda$}}1 {→}{{$\to$}}1}
\usepackage{graphicx}
\usepackage{multicol}

\setbeamertemplate{page number in head/foot}[totalframenumber]

\usepackage{tcolorbox}
\tcbuselibrary{minted,breakable,xparse,skins}



\definecolor{bg}{gray}{0.95}
\DeclareTCBListing{mintedbox}{O{}m!O{}}{%
  breakable=true,
  listing engine=minted,
  listing only,
  minted language=#2,
  minted style=default,
  minted options={%
    linenos,
    gobble=0,
    breaklines=true,
    breakafter=,,
    fontsize=\small,
    numbersep=8pt,
    #1},
  boxsep=0pt,
  left skip=0pt,
  right skip=0pt,
  left=25pt,
  right=0pt,
  top=3pt,
  bottom=3pt,
  arc=5pt,
  leftrule=0pt,
  rightrule=0pt,
  bottomrule=2pt,
  toprule=2pt,
  colback=bg,
  colframe=orange!70,
  enhanced,
  overlay={%
    \begin{tcbclipinterior}
    \fill[orange!20!white] (frame.south west) rectangle ([xshift=20pt]frame.north west);
    \end{tcbclipinterior}},
  #3,
}
\lstset{
    language=C,
    basicstyle=\ttfamily\small,
    keywordstyle=\color{blue},
    stringstyle=\color{orange},
    commentstyle=\color{green!60!black},
    numbers=left,
    numberstyle=\tiny\color{gray},
    breaklines=true,
    showstringspaces=false,
}
%------------------------------------------------------------
%This block of code defines the information to appear in the
%Title page
\title %optional
{12.144}
\date{September 16,2025}
%\subtitle{A short story}

\author % (optional)
{Harsha-EE25BTECH11026}



\begin{document}


\frame{\titlepage}


\begin{frame}{Question}
Let $\vec{A}=\myvec{3&&0&&0\\0&&6&&2\\0&&2&&6}$ and let $\lambda_1 \geq \lambda_2 \geq \lambda_3$ be the eigen values of $\vec{A}$.\\
\\
$\brak{a}$ The triple $\brak{\lambda_1,\lambda_2,\lambda_3}$ equals
\begin{enumerate}

    \item $\brak{9,4,2}$
    \item $\brak{8,4,3}$
    \item $\brak{9,3,3}$
    \item $\brak{7,5,3}$
\end{enumerate}
\end{frame}

\begin{frame}{Question}
$\brak{b}$ The Matrix $\vec{P}$ such that
\begin{align*}
    \vec{P}^{-1}\vec{A}\vec{P}=\myvec{\lambda_1&&0&&0\\0&&\lambda_2&&0\\0&&0&&\lambda_3}
\end{align*}
is

\begin{enumerate}
\begin{multicols}{2}
    \item $\myvec{\frac{1}{\sqrt{3}}&&0&&\frac{-2}{\sqrt{6}}\\\frac{1}{\sqrt{3}}&&\frac{1}{\sqrt{2}}&&\frac{1}{\sqrt{6}}\\\frac{1}{\sqrt{3}}&&\frac{-1}{\sqrt{2}}&&\frac{1}{\sqrt{6}}}$
    \item $\myvec{\frac{1}{\sqrt{3}}&&\frac{-2}{\sqrt{6}}&&0\\\frac{1}{\sqrt{3}}&&\frac{1}{\sqrt{6}}&&\frac{1}{\sqrt{2}}\\\frac{1}{\sqrt{3}}&&\frac{1}{\sqrt{6}}&&\frac{-1}{\sqrt{2}}}$
    \item $\myvec{0&&0&&1\\\frac{1}{\sqrt{2}}&&\frac{1}{\sqrt{2}}&&0\\\frac{1}{\sqrt{2}}&&\frac{-1}{\sqrt{2}}&&0}$
    \item $\myvec{0&&1&&0\\\frac{1}{\sqrt{2}}&&0&&\frac{1}{\sqrt{2}}\\\frac{1}{\sqrt{2}}&&0&&\frac{-1}{\sqrt{2}}}$
\end{multicols}
\end{enumerate}
\end{frame}
\begin{frame}{Theoretical solution}
$\brak{a}$ The eigen values of $\vec{A}$ is obtained using characteristic polynomial, which is given by,
\begin{align}
    det|\vec{A}-\lambda\vec{I}|=0
\end{align}
\begin{align}
    \therefore \mydet{3-\lambda&0&0\\0&6-\lambda&2\\0&2&6-\lambda}=0
\end{align}
\begin{align}
    \therefore \brak{3-\lambda}\brak{\brak{6-\lambda}^2-4}=0
\end{align}
\begin{align}
    \implies  \brak{\lambda-3}\brak{\lambda-4}\brak{\lambda-8}=0
\end{align}
\begin{align}
    \therefore \brak{\lambda_1,\lambda_2,\lambda_3}=\brak{8,4,3}
\end{align}
\end{frame}

\begin{frame}{Theoretical Solution}
$\brak{b}$ The given relation can be computed as ,
\begin{align}
    \vec{A}=\vec{P}\vec{D}\vec{P}^{-1} \label{eq:3}
\end{align}
where $\vec{D}$ is the diagonal matrix of eigenvalues of $\vec{A}$.\\
\\
From~\eqref{eq:3}, we can infer that it is the Eigen-value decomposition of matrix $\vec{A}$.\\
\\
Therefore, $\vec{P}$ is the ortho-normalized matrix of collection of eigen vectors of $\vec{A}$.
\begin{align}
    \vec{P}=\myvec{\vec{v_1}&&\vec{v_2}&&\vec{v_3}}
\end{align}
where $\vec{v_1},\vec{v_2} \,and\, \vec{v_3}$ are the normalized eigen vectors of $\vec{A}$.
\end{frame}

\begin{frame}{Definition}
Eigenvectors $\vec{v}$ for any square matrix $\vec{A}$ is defined as 
\begin{align}
    \vec{A}\vec{v}=\lambda\vec{v}
\end{align}
where $\lambda$ is a scalar and is called the eigen value of $\vec{A}$.
\end{frame}

\begin{frame}{Theoretical Solution}
As we could observe that matrix $\vec{A}$ has zeroes along the first row and first column except the the first pivot,
\begin{align}
     \implies \myvec{3&&0&&0\\0&&6&&2\\0&&2&&6}\myvec{1\\0\\0}=3\myvec{1\\0\\0}
\end{align}
\begin{align}
    \therefore \vec{e_1}=\myvec{1\\0\\0}
\end{align}
\end{frame}

\begin{frame}{Theoretical Solution}
To obtain the other eigen vectors of $\vec{A}$, we can use the fact that the $\vec{A}$ is symmetric.\\
\\
Let us consider two eigen vectors of symmetric matrix $\vec{A}$ to be $\vec{u}$ and $\vec{w}$ such that,
\begin{align}
    \vec{A}\vec{u}=\lambda\vec{u} \quad and \quad \vec{A}\vec{w}=\mu \vec{w}
\end{align}
Consider the scalar $\vec{u}^{\top}\vec{A}\vec{w}$.Because $\vec{A}$ is symmetric,
\begin{align}
    \vec{u}^{\top}\vec{A}\vec{w}=\brak{\vec{A}\vec{u}}^{\top}\vec{w}=\brak{\lambda \vec{u}}^{\top}\vec{w}=\lambda\vec{u}^{\top}\vec{w}
    \label{eq:2}
\end{align}
Similarly,
\begin{align}
    \vec{u}^{\top}\brak{\vec{A}\vec{w}}=\mu\vec{u}^{\top}\vec{w}
    \label{eq:3}
\end{align}

\end{frame}

\begin{frame}{Theoretical Solution}
From ~\eqref{eq:2} and ~\eqref{eq:3},
\begin{align}
    \lambda\vec{u}^{\top}\vec{w}=\mu \vec{u}^{\top}\vec{w}
    \implies \brak{\lambda-\mu}\vec{u}^{\top}\vec{w}=0
\end{align}
As $\lambda$ and $\mu$ are distinct,
\begin{align}
    \vec{u}^{\top}\vec{w}=0 \label{eq:4}
\end{align}
$\implies$ $\vec{u}$ and $\vec{w}$ are orthogonal.\\
Therefore, the other eigenvectors of $\vec{A}$ would be orthogonal to the eigen vector $\myvec{1\\0\\0}$.
\end{frame}

\begin{frame}{Theoretical Solution}
Any vector of form $\myvec{0\\a\\b}$ will be orthogonal to $\myvec{1\\0\\0}$.\\
\\
As we could observe that the $2\times 2$ block from $\vec{A}$, i.e,
\begin{align}
    \vec{B}=\myvec{6&&2\\2&&6}
\end{align}
is also symmetric,
\begin{align}
    \therefore \myvec{1\\1} \text{ is an eigen-vector of $\vec{B}$ }
\end{align}
\begin{align}
      \implies \text{Eigen vector of $\vec{A}$ $\brak{\vec{e_2}}$}=\myvec{0\\1\\1}
\end{align}
\end{frame}

\begin{frame}{Theoretical solution}
    As we know that from ~\eqref{eq:4}, we could say that the other eigen-vector is orthogonal to both $\myvec{1\\0\\0}$ and $\myvec{0\\1\\1}$.\\
\\
The third eigen-vector $\vec{e_3}$ is the vector-product of $\myvec{1\\0\\0}$ and $\myvec{0\\1\\1}$.
\begin{align}
    \implies \vec{e_3}=\vec{e_2} \times \vec{e_1}
\end{align}
\begin{align}
    \therefore \vec{e_3}=\myvec{0\\1\\-1}
\end{align}
\end{frame}

\begin{frame}{Theoretical Solution}

\begin{align}
    \therefore \text{The eigen-vectors of A:}\quad \myvec{1\\0\\0},\quad \myvec{0\\1\\-1},\quad \myvec{0\\1\\1}
\end{align}
As we require unit eigen-vectors,

\begin{align}
    \implies \vec{v_1}=\myvec{1\\0\\0} \qquad \vec{v_2}=\myvec{0\\\frac{1}{\sqrt{2}}\\\frac{1}{\sqrt{2}}} \qquad \vec{v_3}=\myvec{0\\\frac{1}{\sqrt{2}}\\\frac{-1}{\sqrt{2}}}
\end{align}
\begin{align}
    \therefore \vec{P}=\myvec{1&&0&&0\\0&&\frac{1}{\sqrt{2}}&&\frac{1}{\sqrt{2}}\\0&&\frac{1}{\sqrt{2}}&&\frac{-1}{\sqrt{2}}}
\end{align}
\end{frame}

\begin{frame}[fragile]
    \frametitle{C Code -Finding eigen values and eigen vectors of Matrix}

    \begin{lstlisting}[language=C]
#include <stdio.h>
#include <math.h>

void solve_eigen(double *eigenvalues, double *eigenvectors) {
    eigenvalues[0] = 3.0;
    eigenvalues[1] = 8.0;
    eigenvalues[2] = 4.0;

    eigenvectors[0] = 1.0; eigenvectors[1] = 0.0; eigenvectors[2] = 0.0;

    eigenvectors[3] = 0.0; eigenvectors[4] = 1.0/sqrt(2.0); eigenvectors[5] = 1.0/sqrt(2.0);

    eigenvectors[6] = 0.0; eigenvectors[7] = 1.0/sqrt(2.0); eigenvectors[8] = -1.0/sqrt(2.0);
}


    \end{lstlisting}
\end{frame}




\begin{frame}[fragile]
    \frametitle{Python+C code}

    \begin{lstlisting}[language=Python]
import ctypes
import numpy as np

lib = ctypes.CDLL("./libeigen_solver.so")

# Prepare result arrays
eigenvalues = (ctypes.c_double * 3)()
eigenvectors = (ctypes.c_double * 9)()

# Call function
lib.solve_eigen(eigenvalues, eigenvectors)

eigvals = np.array(eigenvalues)
eigvecs = np.array(eigenvectors).reshape(3,3, order="F")  

print("Eigenvalues:", np.round(eigvals,0))
print("Eigenvectors:\n", np.round(eigvecs,3))
    \end{lstlisting}
\end{frame}

\begin{frame}[fragile]
    \frametitle{Python code}
    \begin{lstlisting}[language=Python]
import numpy as np
import sympy as sp

A = np.array([[3,0,0],[0,6,2],[0,2,6]])
eigvals, P = np.linalg.eigh(A)
B=sp.Matrix(np.round(P, 3))
print("P=")
sp.pprint(B)

print(np.flip(eigvals))
    \end{lstlisting}   
\end{frame}

\end{document}