\let\negmedspace\undefined
\let\negthickspace\undefined
\documentclass[journal]{IEEEtran}
\usepackage[a5paper, margin=10mm, onecolumn]{geometry}
%\usepackage{lmodern} % Ensure lmodern is loaded for pdflatex
\usepackage{tfrupee} % Include tfrupee package

\setlength{\headheight}{1cm} % Set the height of the header box
\setlength{\headsep}{0mm}     % Set the distance between the header box and the top of the text

\usepackage{gvv-book}
\usepackage{gvv}
\usepackage{cite}
\usepackage{amsmath,amssymb,amsfonts,amsthm}
\usepackage{algorithmic}
\usepackage{graphicx}
\usepackage{textcomp}
\usepackage{xcolor}
\usepackage{txfonts}
\usepackage{listings}
\usepackage{enumitem}
\usepackage{mathtools}
\usepackage{gensymb}
\usepackage[breaklinks=true]{hyperref}
\usepackage{tkz-euclide} 
\usepackage{listings}
% \usepackage{gvv}                                        
\def\inputGnumericTable{}                                 
\usepackage[latin1]{inputenc}                                
\usepackage{color}                                            
\usepackage{array}                                            
\usepackage{longtable}                                       
\usepackage{calc}                                             
\usepackage{multirow}                                         
\usepackage{hhline}                                           
\usepackage{ifthen}                                           
\usepackage{lscape}
\usepackage{circuitikz}
\usepackage{comment}
\tikzstyle{block} = [rectangle, draw, fill=blue!20, 
    text width=4em, text centered, rounded corners, minimum height=3em]
\tikzstyle{sum} = [draw, fill=blue!10, circle, minimum size=1cm, node distance=1.5cm]
\tikzstyle{input} = [coordinate]
\tikzstyle{output} = [coordinate]


\begin{document}

\bibliographystyle{IEEEtran}
\vspace{3cm}

\title{7.4.33}
\author{EE25BTECH11026-Harsha}
 \maketitle
% \newpage
% \bigskip
{\let\newpage\relax\maketitle}

\renewcommand{\thefigure}{\theenumi}
\renewcommand{\thetable}{\theenumi}
\setlength{\intextsep}{10pt} % Space between text and floats


\numberwithin{equation}{enumi}
\numberwithin{figure}{enumi}
\renewcommand{\thetable}{\theenumi}

\textbf{Question}:\\
A circle C of radius 1 unit is inscribed in an equilateral triangle PQR. The points of contact of C with sides PQ, QR, RP are D, E, F respectively.The line PQ is given by the equation
$\sqrt{3}x+y-6=0$ and the point $\vec{D}$ is $\brak{\frac{3\sqrt{3}}{2},\frac{3}{2}}$. Further, it is given
that the origin and the centre of C are on same side of line PQ. The equation of circle C is
\begin{multicols}{2}
\begin{enumerate}
    \item $\brak{x-2\sqrt{3}}^2+\brak{y-1}^2=1$
    \item $\brak{x-2\sqrt{3}}^2+\brak{y+\frac{1}{2}}^2=1$
    \item $\brak{x-\sqrt{3}}^2+\brak{y-1}^2=1$
    \item $\brak{x-\sqrt{3}}^2+\brak{y+1}^2=1$
\end{enumerate}
\end{multicols}
\solution \\
Let us solve the given question theoretically and then verify the solution computationally.\\
\\
According to the question,
\begin{align}
    \text{Equation of tangent PQ :\,}\vec{n}^{\top}\vec{x}=c
\end{align}
where $\vec{n}=\myvec{\sqrt{3}&&1}^{\top}$ and $c=6$
\begin{align}
    \text{Point of tangency}\brak{\vec{D}}:\myvec{\frac{3\sqrt{3}}{2}\\\frac{3}{2}}
\end{align}
\begin{align}
    radius(r)=1
\end{align}
As the point of tangency $\vec{D}$ and centre of circle $\vec{u}$  are along the direction of the vector $\vec{n}$,
\begin{align}
    \therefore \vec{D}-\vec{u}=\lambda\vec{n}\;\text{, for some scalar $\lambda$}
\end{align}
\begin{align}
    \implies \vec{u}=\vec{D}-\lambda\vec{n}
\end{align}
Also,
\begin{align}
    \frac{|\vec{n}^{\top}\vec{u}-c|}{\|\vec{n}\|}=r
\end{align}
\begin{align}
    |\vec{n}^{\top}\vec{u}-c|=r\|\vec{n}\|
\end{align}
\begin{align}
    \vec{n}^{\top}\vec{u}=c \pm r\|\vec{n}\|
\end{align}
\newpage
\vspace*{0.25cm}
To decide the sign , we need to use the fact that the origin and centre of circle are on the same side of the line PQ.
\begin{align}
    \therefore \brak{\vec{n}^{\top}\vec{u}-c}\brak{\vec{n}^{\top}\myvec{0\\0}-c}>0
\end{align}
\begin{align}
    \implies \vec{n}^{\top}\vec{u}<c
\end{align}
\begin{align}
    \therefore \vec{n}^{\top}\vec{u}=c-r\|\vec{n}\|
\end{align}
Substituting value of $\vec{u}$,
\begin{align}
\vec{n}^{\top}\brak{\vec{D}-\lambda\vec{n}}=c-r\|\vec{n}\|
\end{align}
\begin{align}
    \implies \lambda=\frac{\vec{n}^{\top}\vec{D}+r\|n\|-c}{\vec{n}^{\top}\vec{n}}
\end{align}
\begin{align}
    \vec{u}=\vec{D}-\frac{\vec{n}^{\top}\vec{D}+r\|n\|-c}{\vec{n}^{\top}\vec{n}}\;\vec{n}
\end{align}
Substituting the values,
\begin{align}
    \therefore \vec{u}=\myvec{\frac{3\sqrt{3}}{2}\\\frac{3}{2}}
-\frac{1}{2}\myvec{\sqrt{3}\\1}=\myvec{\sqrt{3}\\1}
\end{align}
\begin{align}
    \therefore \text{Required equation of circle :\,}\|\vec{x}\|^2-2\myvec{\sqrt{3}&&1}\vec{x}+3=0
\end{align}
From the figure, it is clearly verified that the theoretical solution matches with the computational solution.\\
\begin{figure}[H]
    \centering
    \includegraphics[width=0.4\columnwidth]{figs/Figure_1.png}
    \label{fig:1}
\end{figure}
\end{document}
