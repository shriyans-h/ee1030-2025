\documentclass{beamer}
\usepackage[utf8]{inputenc}

\usetheme{Madrid}
\usecolortheme{default}
\usepackage{amsmath,amssymb,amsfonts,amsthm}
\usepackage{mathtools}
\usepackage{txfonts}
\usepackage{tkz-euclide}
\usepackage{listings}
\usepackage{adjustbox}
\usepackage{array}
\usepackage{gensymb}
\usepackage{tabularx}
\usepackage{gvv}
\usepackage{lmodern}
\usepackage{circuitikz}
\usepackage{tikz}
\lstset{literate={·}{{$\cdot$}}1 {λ}{{$\lambda$}}1 {→}{{$\to$}}1}
\usepackage{graphicx}

\setbeamertemplate{page number in head/foot}[totalframenumber]

\usepackage{tcolorbox}
\tcbuselibrary{minted,breakable,xparse,skins}



\definecolor{bg}{gray}{0.95}
\DeclareTCBListing{mintedbox}{O{}m!O{}}{%
  breakable=true,
  listing engine=minted,
  listing only,
  minted language=#2,
  minted style=default,
  minted options={%
    linenos,
    gobble=0,
    breaklines=true,
    breakafter=,,
    fontsize=\small,
    numbersep=8pt,
    #1},
  boxsep=0pt,
  left skip=0pt,
  right skip=0pt,
  left=25pt,
  right=0pt,
  top=3pt,
  bottom=3pt,
  arc=5pt,
  leftrule=0pt,
  rightrule=0pt,
  bottomrule=2pt,
  toprule=2pt,
  colback=bg,
  colframe=orange!70,
  enhanced,
  overlay={%
    \begin{tcbclipinterior}
    \fill[orange!20!white] (frame.south west) rectangle ([xshift=20pt]frame.north west);
    \end{tcbclipinterior}},
  #3,
}
\lstset{
    language=C,
    basicstyle=\ttfamily\small,
    keywordstyle=\color{blue},
    stringstyle=\color{orange},
    commentstyle=\color{green!60!black},
    numbers=left,
    numberstyle=\tiny\color{gray},
    breaklines=true,
    showstringspaces=false,
}
%------------------------------------------------------------
%This block of code defines the information to appear in the
%Title page
\title %optional
{5.13.52}
\date{September 6,2025}
%\subtitle{A short story}

\author % (optional)
{Harsha-EE25BTECH11026}



\begin{document}


\frame{\titlepage}


\begin{frame}{Question}
If the system of equations $x + ay = 0$,$az + y = 0$ and $ax + z = 0$ has infinite solutions, then the value of a is
\begin{enumerate}
    \item -1
    \item 1
    \item 0
    \item no real values
\end{enumerate}
\end{frame}

\begin{frame}{Theoretical Solution}
From the given,
\begin{align}
    \myvec{1&&a&&0}\vec{x}=0\\
    \myvec{0&&1&&a}\vec{x}=0\\
    \myvec{a&&0&&1}\vec{x}=0
\end{align}
\begin{align}
    \therefore \myvec{1&&a&&0\\0&&1&&a\\a&&0&&1}\vec{x}=0
\end{align}
\end{frame}

\begin{frame}{Theoretical Solution}
To solve for a, we can use the fact that of rank of coefficient matrix should be less than 3.
\begin{align}
    \myvec{1&&a&&0\\0&&1&&a\\a&&0&&1}
    \xleftrightarrow[\,R_3 \gets R_3+ a^2 \times R_2]{\, R_3 \gets R_3- a \times R_1}
    \myvec{1&&a&&0\\0&&1&&a\\0&&0&&a^3+1}
\end{align}
\\
As the rank of the matrix should be less than 3, we require the last pivot to be zero.
\begin{align}
    \therefore a^3+1=0 \implies a=-1,-\omega,-\omega^2
\end{align}
\end{frame}

\begin{frame}[fragile]
    \frametitle{C Code -Finding the determinant of the matrix}

    \begin{lstlisting}[language=C]
#include <stdio.h>

double det3x3(double a) {
    double det = 1 + a*a*a;
    return det;
}
    \end{lstlisting}
\end{frame}



\begin{frame}[fragile]
    \frametitle{Python+C code}

    \begin{lstlisting}[language=Python]
import ctypes
# Load the shared C library
lib = ctypes.CDLL("./libmatrix_solver.so")
lib.det3x3.argtypes = [ctypes.c_double]
lib.det3x3.restype = ctypes.c_double
# Real solution directly
a = -1.0
det_val = lib.det3x3(a)
tol = 1e-6

if abs(det_val) < tol:
    solutions = [a]
else:
    solutions = []

print("Real values of a for infinite solutions:")
print(solutions)


    \end{lstlisting}
\end{frame}

\begin{frame}[fragile]
    \frametitle{Python code}
    \begin{lstlisting}[language=Python]
import sympy as sp
a = sp.symbols('a')
A = sp.Matrix([
    [1, a, 0],
    [0, 1, a],
    [a, 0, 1]
])
# Solve det(A) = 0 for exact solution
solutions = sp.solve(A.det(), a)
print("Value(s) of a for infinite solutions:", solutions) 
    \end{lstlisting}   
\end{frame}

\end{document}