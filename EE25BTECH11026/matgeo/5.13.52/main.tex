\let\negmedspace\undefined
\let\negthickspace\undefined
\documentclass[journal]{IEEEtran}
\usepackage[a5paper, margin=10mm, onecolumn]{geometry}
%\usepackage{lmodern} % Ensure lmodern is loaded for pdflatex
\usepackage{tfrupee} % Include tfrupee package

\setlength{\headheight}{1cm} % Set the height of the header box
\setlength{\headsep}{0mm}     % Set the distance between the header box and the top of the text

\usepackage{gvv-book}
\usepackage{gvv}
\usepackage{cite}
\usepackage{amsmath,amssymb,amsfonts,amsthm}
\usepackage{algorithmic}
\usepackage{graphicx}
\usepackage{textcomp}
\usepackage{xcolor}
\usepackage{txfonts}
\usepackage{listings}
\usepackage{enumitem}
\usepackage{mathtools}
\usepackage{gensymb}
\usepackage[breaklinks=true]{hyperref}
\usepackage{tkz-euclide} 
\usepackage{listings}
% \usepackage{gvv}                                        
\def\inputGnumericTable{}                                 
\usepackage[latin1]{inputenc}                                
\usepackage{color}                                            
\usepackage{array}                                            
\usepackage{longtable}                                       
\usepackage{calc}                                             
\usepackage{multirow}                                         
\usepackage{hhline}                                           
\usepackage{ifthen}                                           
\usepackage{lscape}
\usepackage{circuitikz}
\usepackage{comment}
\tikzstyle{block} = [rectangle, draw, fill=blue!20, 
    text width=4em, text centered, rounded corners, minimum height=3em]
\tikzstyle{sum} = [draw, fill=blue!10, circle, minimum size=1cm, node distance=1.5cm]
\tikzstyle{input} = [coordinate]
\tikzstyle{output} = [coordinate]


\begin{document}

\bibliographystyle{IEEEtran}
\vspace{3cm}

\title{5.13.52}
\author{EE25BTECH11026-Harsha}
 \maketitle
% \newpage
% \bigskip
{\let\newpage\relax\maketitle}

\renewcommand{\thefigure}{\theenumi}
\renewcommand{\thetable}{\theenumi}
\setlength{\intextsep}{10pt} % Space between text and floats


\numberwithin{equation}{enumi}
\numberwithin{figure}{enumi}
\renewcommand{\thetable}{\theenumi}

\textbf{Question}:\\
If the system of equations $x + ay = 0$,$az + y = 0$ and $ax + z = 0$ has infinite solutions, then the value of a is
\begin{multicols}{4}
\begin{enumerate}
    \item -1
    \item 1
    \item 0
    \item no real values
\end{enumerate}
\end{multicols}
\solution \\
Let us solve the given question theoretically and then verify the solution computationally.\\
\\
From the given,
\begin{align}
    \myvec{1&&a&&0}\vec{x}=0\\
    \myvec{0&&1&&a}\vec{x}=0\\
    \myvec{a&&0&&1}\vec{x}=0
\end{align}
\begin{align}
    \therefore \myvec{1&&a&&0\\0&&1&&a\\a&&0&&1}\vec{x}=0
\end{align}
To solve for a, we can use the fact that of rank of coefficient matrix should be less than 3.
\begin{align}
    \myvec{1&&a&&0\\0&&1&&a\\a&&0&&1}
    \xleftrightarrow[\,R_3 \gets R_3+ a^2 \times R_2]{\, R_3 \gets R_3- a \times R_1}
    \myvec{1&&a&&0\\0&&1&&a\\0&&0&&a^3+1}
\end{align}
\\
As the rank of the matrix should be less than 3, we require the last pivot to be zero.
\begin{align}
    \therefore a^3+1=0 \implies a=-1,-\omega,-\omega^2
\end{align}

\end{document}

