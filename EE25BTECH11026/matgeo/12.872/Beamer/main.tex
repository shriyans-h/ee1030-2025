\documentclass{beamer}
\usepackage[utf8]{inputenc}

\usetheme{Madrid}
\usecolortheme{default}
\usepackage{amsmath,amssymb,amsfonts,amsthm}
\usepackage{mathtools}
\usepackage{txfonts}
\usepackage{tkz-euclide}
\usepackage{listings}
\usepackage{adjustbox}
\usepackage{array}
\usepackage{gensymb}
\usepackage{tabularx}
\usepackage{gvv}
\usepackage{lmodern}
\usepackage{circuitikz}
\usepackage{tikz}
\lstset{literate={·}{{$\cdot$}}1 {λ}{{$\lambda$}}1 {→}{{$\to$}}1}
\usepackage{multicol}
\usepackage{graphicx}

\setbeamertemplate{page number in head/foot}[totalframenumber]

\usepackage{tcolorbox}
\tcbuselibrary{minted,breakable,xparse,skins}



\definecolor{bg}{gray}{0.95}
\DeclareTCBListing{mintedbox}{O{}m!O{}}{%
  breakable=true,
  listing engine=minted,
  listing only,
  minted language=#2,
  minted style=default,
  minted options={%
    linenos,
    gobble=0,
    breaklines=true,
    breakafter=,,
    fontsize=\small,
    numbersep=8pt,
    #1},
  boxsep=0pt,
  left skip=0pt,
  right skip=0pt,
  left=25pt,
  right=0pt,
  top=3pt,
  bottom=3pt,
  arc=5pt,
  leftrule=0pt,
  rightrule=0pt,
  bottomrule=2pt,
  toprule=2pt,
  colback=bg,
  colframe=orange!70,
  enhanced,
  overlay={%
    \begin{tcbclipinterior}
    \fill[orange!20!white] (frame.south west) rectangle ([xshift=20pt]frame.north west);
    \end{tcbclipinterior}},
  #3,
}
\lstset{
    language=C,
    basicstyle=\ttfamily\small,
    keywordstyle=\color{blue},
    stringstyle=\color{orange},
    commentstyle=\color{green!60!black},
    numbers=left,
    numberstyle=\tiny\color{gray},
    breaklines=true,
    showstringspaces=false,
}
%------------------------------------------------------------
%This block of code defines the information to appear in the
%Title page
\title %optional
{12.872}
\date{October 11,2025}
%\subtitle{A short story}

\author % (optional)
{Harsha-EE25BTECH11026}



\begin{document}


\frame{\titlepage}


\begin{frame}{Question}
Let $\vec{A}=\myvec{1&&1\\1&&3\\-2&&-3}$ and $\vec{b}=\myvec{b_1\\b_2\\b_3}$. For $\vec{A}\vec{x}=\vec{b}$ to be solvable, which one of the following options is the correct condition on $b_1,b_2,\,and\,b_3$.
\begin{enumerate}
\begin{multicols}{2}
    \item $b_1+b_2+b_3=1$
    \item $3b_1+b_2+2b_3=0$
    \item $b_1+3b_2+b_3=2$
    \item $b_1+b_2+b_3=2$
\end{multicols}
\end{enumerate}
\end{frame}


\begin{frame}{Theoretical solution}
Given,
\begin{align}
    \vec{A}\vec{x}=\vec{b} \label{eq:1}
\end{align}
Multiplying $\vec{A}^{\top}$ on both sides,
\begin{align}
    \myvec{1&&1&&-2\\1&&3&&-3}\myvec{1&&1\\1&&3\\-2&&-3}\vec{x}=\myvec{1&&1&&-2\\1&&3&&-3}\myvec{b_1\\b_2\\b_3}
\end{align}
\begin{align}
    \implies \myvec{6&&10\\10&&19}\vec{x}=\myvec{b_1+b_2-2b_3\\b_1+3b_2-3b_3}
\end{align}
\end{frame}

\begin{frame}{Theoretical solution}
Forming the augmented matrix,
\begin{align}
    \augvec{2}{1}{6&10&b_1+b_2-2b_3\\10&19&b_1+3b_2-3b_3}
    \xleftrightarrow{\, R_2 \gets R_2-\frac{5}{3}R_1}
    \augvec{2}{1}{6&10&b_1+b_2-2b_3\\0&\frac{7}{3}&-\frac{2}{3}b_1+\frac{4}{3}b_2+\frac{1}{3}b_3}
\end{align}
\begin{align}
    \xleftrightarrow[\, R_1 \gets R_1-10R_2]{\, R_2 \gets \frac{3}{7}R_2}
    \augvec{2}{1}{6&0&\frac{27}{7}b_1-\frac{33}{7}b_2-\frac{24}{7}b_3\\0&1&-\frac{2}{7}b_1+\frac{4}{7}b_2+\frac{1}{7}b_3}
    \xleftrightarrow{\, R_1 \gets \frac{1}{6}R_1}
\end{align}
\begin{align}
    \augvec{2}{1}{1&0&\frac{9}{14}b_1-\frac{11}{14}b_2-\frac{4}{7}b_3\\0&1&-\frac{2}{7}b_1+\frac{4}{7}b_2+\frac{1}{7}b_3}
\end{align}
\end{frame}

\begin{frame}{Theoretical solution}
\begin{align}
    \therefore \vec{x}=\myvec{x_1\\x_2}=\myvec{\frac{9}{14}b_1-\frac{11}{14}b_2-\frac{4}{7}b_3\\-\frac{2}{7}b_1+\frac{4}{7}b_2+\frac{1}{7}b_3} 
\end{align}
From ~\eqref{eq:1},
\begin{align}
    x_1+x_2=b_1 \Rightarrow \; \frac{9}{14}b_1-\frac{11}{14}b_2-\frac{4}{7}b_3\, +\, \brak{-\frac{2}{7}b_1+\frac{4}{7}b_2+\frac{1}{7}b_3} =b_1
\end{align}
\begin{align}
    \therefore 3b_1+b_2+2b_3=0 
\end{align}
\end{frame}


\end{document}