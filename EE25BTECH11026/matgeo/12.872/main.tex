\let\negmedspace\undefined
\let\negthickspace\undefined
\documentclass[journal]{IEEEtran}
\usepackage[a5paper, margin=10mm, onecolumn]{geometry}
%\usepackage{lmodern} % Ensure lmodern is loaded for pdflatex
\usepackage{tfrupee} % Include tfrupee package

\setlength{\headheight}{1cm} % Set the height of the header box
\setlength{\headsep}{0mm}     % Set the distance between the header box and the top of the text

\usepackage{gvv-book}
\usepackage{gvv}
\usepackage{cite}
\usepackage{amsmath,amssymb,amsfonts,amsthm}
\usepackage{algorithmic}
\usepackage{graphicx}
\usepackage{textcomp}
\usepackage{xcolor}
\usepackage{txfonts}
\usepackage{listings}
\usepackage{enumitem}
\usepackage{mathtools}
\usepackage{gensymb}
\usepackage[breaklinks=true]{hyperref}
\usepackage{tkz-euclide} 
\usepackage{listings}
% \usepackage{gvv}                                        
\def\inputGnumericTable{}                                 
\usepackage[latin1]{inputenc}                                
\usepackage{color}                                            
\usepackage{array}                                            
\usepackage{longtable}                                       
\usepackage{calc}                                             
\usepackage{multirow}                                         
\usepackage{hhline}                                           
\usepackage{ifthen}                                           
\usepackage{lscape}
\usepackage{circuitikz}
\usepackage{comment}
\tikzstyle{block} = [rectangle, draw, fill=blue!20, 
    text width=4em, text centered, rounded corners, minimum height=3em]
\tikzstyle{sum} = [draw, fill=blue!10, circle, minimum size=1cm, node distance=1.5cm]
\tikzstyle{input} = [coordinate]
\tikzstyle{output} = [coordinate]


\begin{document}

\bibliographystyle{IEEEtran}
\vspace{3cm}

\title{12.872}
\author{EE25BTECH11026-Harsha}
 \maketitle
% \newpage
% \bigskip
{\let\newpage\relax\maketitle}

\renewcommand{\thefigure}{\theenumi}
\renewcommand{\thetable}{\theenumi}
\setlength{\intextsep}{10pt} % Space between text and floats


\numberwithin{equation}{enumi}
\numberwithin{figure}{enumi}
\renewcommand{\thetable}{\theenumi}

\textbf{Question}:\\
Let $\vec{A}=\myvec{1&&1\\1&&3\\-2&&-3}$ and $\vec{b}=\myvec{b_1\\b_2\\b_3}$. For $\vec{A}\vec{x}=\vec{b}$ to be solvable, which one of the following options is the correct condition on $b_1,b_2,\,and\,b_3$.
\begin{enumerate}
\begin{multicols}{2}
    \item $b_1+b_2+b_3=1$
    \item $3b_1+b_2+2b_3=0$
    \item $b_1+3b_2+b_3=2$
    \item $b_1+b_2+b_3=2$
\end{multicols}
\end{enumerate}

\solution \\
Let us solve the given question theoretically and then verify the solution computationally.\\
\\
Let $\vec{x}=\myvec{x_1\\x_2}$.
Forming the augmented matrix of $\vec{A}$ and $\vec{b}$,
\begin{align}
    \augvec{2}{1}{1&1&b_1\\1&3&b_2\\-2&-3&b_3}
    \xleftrightarrow[\, R_2 \gets R_2-R_1]{\, R_3 \gets R_3+R_2}
    \augvec{2}{1}{1&1&b_1\\0&2&b_2-b_1\\-1&0&b_2+b_3}
\end{align}
\begin{align}
    \implies \myvec{x_1+x_2\\2x_2\\-x_1}=\myvec{b_1\\b_2-b_1\\b_2+b_3}
\end{align}
\begin{align}
    \therefore x_1=-\brak{b_2+b_3}  \qquad x_2=\frac{b_2-b_1}{2}
\end{align}
Substituting the above, yielding,
\begin{align}
    \therefore -\brak{b_2+b_3}+\frac{b_2-b_1}{2}=b_1
\end{align}
\begin{align}
    \implies 3b_1+b_2+2b_3=0
\end{align}

\end{document}