\let\negmedspace\undefined
\let\negthickspace\undefined
\documentclass[journal]{IEEEtran}
\usepackage[a5paper, margin=10mm, onecolumn]{geometry}
%\usepackage{lmodern} % Ensure lmodern is loaded for pdflatex
\usepackage{tfrupee} % Include tfrupee package

\setlength{\headheight}{1cm} % Set the height of the header box
\setlength{\headsep}{0mm}     % Set the distance between the header box and the top of the text

\usepackage{gvv-book}
\usepackage{gvv}
\usepackage{cite}
\usepackage{amsmath,amssymb,amsfonts,amsthm}
\usepackage{algorithmic}
\usepackage{graphicx}
\usepackage{textcomp}
\usepackage{xcolor}
\usepackage{txfonts}
\usepackage{listings}
\usepackage{enumitem}
\usepackage{mathtools}
\usepackage{gensymb}
\usepackage[breaklinks=true]{hyperref}
\usepackage{tkz-euclide} 
\usepackage{listings}
% \usepackage{gvv}                                        
\def\inputGnumericTable{}                                 
\usepackage[latin1]{inputenc}                                
\usepackage{color}                                            
\usepackage{array}                                            
\usepackage{longtable}                                       
\usepackage{calc}                                             
\usepackage{multirow}                                         
\usepackage{hhline}                                           
\usepackage{ifthen}                                           
\usepackage{lscape}
\usepackage{circuitikz}
\usepackage{comment}
\tikzstyle{block} = [rectangle, draw, fill=blue!20, 
    text width=4em, text centered, rounded corners, minimum height=3em]
\tikzstyle{sum} = [draw, fill=blue!10, circle, minimum size=1cm, node distance=1.5cm]
\tikzstyle{input} = [coordinate]
\tikzstyle{output} = [coordinate]


\begin{document}

\bibliographystyle{IEEEtran}
\vspace{3cm}

\title{12.872}
\author{EE25BTECH11026-Harsha}
 \maketitle
% \newpage
% \bigskip
{\let\newpage\relax\maketitle}

\renewcommand{\thefigure}{\theenumi}
\renewcommand{\thetable}{\theenumi}
\setlength{\intextsep}{10pt} % Space between text and floats


\numberwithin{equation}{enumi}
\numberwithin{figure}{enumi}
\renewcommand{\thetable}{\theenumi}

\textbf{Question}:\\
Let $\vec{A}=\myvec{1&&1\\1&&3\\-2&&-3}$ and $\vec{b}=\myvec{b_1\\b_2\\b_3}$. For $\vec{A}\vec{x}=\vec{b}$ to be solvable, which one of the following options is the correct condition on $b_1,b_2,\,and\,b_3$.
\begin{enumerate}
\begin{multicols}{2}
    \item $b_1+b_2+b_3=1$
    \item $3b_1+b_2+2b_3=0$
    \item $b_1+3b_2+b_3=2$
    \item $b_1+b_2+b_3=2$
\end{multicols}
\end{enumerate}

\solution \\
Let us solve the given question theoretically and then verify the solution computationally.\\
\\
Given,
\begin{align}
    \vec{A}\vec{x}=\vec{b} \label{eq:1}
\end{align}
Multiplying $\vec{A}^{\top}$ on both sides,
\begin{align}
    \myvec{1&&1&&-2\\1&&3&&-3}\myvec{1&&1\\1&&3\\-2&&-3}\vec{x}=\myvec{1&&1&&-2\\1&&3&&-3}\myvec{b_1\\b_2\\b_3}
\end{align}
\begin{align}
    \implies \myvec{6&&10\\10&&19}\vec{x}=\myvec{b_1+b_2-2b_3\\b_1+3b_2-3b_3}
\end{align}
Forming the augmented matrix,
\begin{align}
    \augvec{2}{1}{6&10&b_1+b_2-2b_3\\10&19&b_1+3b_2-3b_3}
    \xleftrightarrow{\, R_2 \gets R_2-\frac{5}{3}R_1}
    \augvec{2}{1}{6&10&b_1+b_2-2b_3\\0&\frac{7}{3}&-\frac{2}{3}b_1+\frac{4}{3}b_2+\frac{1}{3}b_3}
    \xleftrightarrow[\, R_1 \gets R_1-10R_2]{\, R_2 \gets \frac{3}{7}R_2}
\end{align}
\begin{align}
    \augvec{2}{1}{6&0&\frac{27}{7}b_1-\frac{33}{7}b_2-\frac{24}{7}b_3\\0&1&-\frac{2}{7}b_1+\frac{4}{7}b_2+\frac{1}{7}b_3}
    \xleftrightarrow{\, R_1 \gets \frac{1}{6}R_1}
    \augvec{2}{1}{1&0&\frac{9}{14}b_1-\frac{11}{14}b_2-\frac{4}{7}b_3\\0&1&-\frac{2}{7}b_1+\frac{4}{7}b_2+\frac{1}{7}b_3}
\end{align}
\begin{align}
    \therefore \vec{x}=\myvec{x_1\\x_2}=\myvec{\frac{9}{14}b_1-\frac{11}{14}b_2-\frac{4}{7}b_3\\-\frac{2}{7}b_1+\frac{4}{7}b_2+\frac{1}{7}b_3} 
\end{align}
From ~\eqref{eq:1},
\begin{align}
    x_1+x_2=b_1 \Rightarrow \; \frac{9}{14}b_1-\frac{11}{14}b_2-\frac{4}{7}b_3\, +\, \brak{-\frac{2}{7}b_1+\frac{4}{7}b_2+\frac{1}{7}b_3} =b_1
\end{align}
\begin{align}
    \therefore 3b_1+b_2+2b_3=0 
\end{align}


\end{document}