\documentclass{beamer}
\usepackage[utf8]{inputenc}

\usetheme{Madrid}
\usecolortheme{default}
\usepackage{amsmath,amssymb,amsfonts,amsthm}
\usepackage{txfonts}
\usepackage{tkz-euclide}
\usepackage{listings}
\usepackage{adjustbox}
\usepackage{array}
\usepackage{tabularx}
\usepackage{gvv}
\usepackage{lmodern}
\usepackage{circuitikz}
\usepackage{tikz}
\usepackage{graphicx}

\setbeamertemplate{page number in head/foot}[totalframenumber]

\usepackage{tcolorbox}
\tcbuselibrary{minted,breakable,xparse,skins}



\definecolor{bg}{gray}{0.95}
\DeclareTCBListing{mintedbox}{O{}m!O{}}{%
  breakable=true,
  listing engine=minted,
  listing only,
  minted language=#2,
  minted style=default,
  minted options={%
    linenos,
    gobble=0,
    breaklines=true,
    breakafter=,,
    fontsize=\small,
    numbersep=8pt,
    #1},
  boxsep=0pt,
  left skip=0pt,
  right skip=0pt,
  left=25pt,
  right=0pt,
  top=3pt,
  bottom=3pt,
  arc=5pt,
  leftrule=0pt,
  rightrule=0pt,
  bottomrule=2pt,
  toprule=2pt,
  colback=bg,
  colframe=orange!70,
  enhanced,
  overlay={%
    \begin{tcbclipinterior}
    \fill[orange!20!white] (frame.south west) rectangle ([xshift=20pt]frame.north west);
    \end{tcbclipinterior}},
  #3,
}
\lstset{
    language=C,
    basicstyle=\ttfamily\small,
    keywordstyle=\color{blue},
    stringstyle=\color{orange},
    commentstyle=\color{green!60!black},
    numbers=left,
    numberstyle=\tiny\color{gray},
    breaklines=true,
    showstringspaces=false,
}
%------------------------------------------------------------
%This block of code defines the information to appear in the
%Title page
\title %optional
{3.3.2}
\date{September 9,2025}
%\subtitle{A short story}

\author % (optional)
{Hemanth Reddy-AI25BTECH11018}



\begin{document}


\frame{\titlepage}
\begin{frame}{Question}
Construct a triangle with sides 5cm, 6cm and 7cm. 
\end{frame}



\begin{frame}{Theoretical Solution}
\textbf{Solution:}\\
Let triangle be $\triangle ABC $ \\
Let AB=5cm BC=6cm CA=7cm \\Take\\

$\vec{A}$ \myvec{0\\0}  ,$\vec{B}$ \myvec{5\\0} ,  $\vec{C}$\myvec{7\cos A\\7 \sin A}
\begin{align}
    cos \vec{A} =\frac{AB^{2}+AC^{2}-BC^{2}}{2\cdot AB\cdot AC}\\
    cos \vec{A} =\frac{5^{2}+7^{2}-6^{2}}{ 2\cdot 5\cdot 7}=\frac{19}{35}\\
           \sin A = \frac{12\sqrt{6}}{35}     
\end{align}
 Therefore
\begin{align}
    \vec{C}\myvec{7\cdot \frac{19}{35} & 7\cdot \frac{12\sqrt{6}}{35} }
    \end{align}


\end{frame}

\begin{frame}{Theoretical Solution}
\begin{align}
    \vec{C}\myvec{\frac{19}{5} & ,\frac{12\sqrt{6}}{5} }
\end{align}


\end{frame}


\begin{frame}[fragile]
    \frametitle{C Code }
    \begin{lstlisting}

#include <stdio.h>
#include <math.h>

int main() {
    // Side lengths
    double AB = 5.0; // between A and B
    double BC = 6.0; // between B and C
    double CA = 7.0; // between C and A

    // Coordinates of points
    double Ax = 0.0, Ay = 0.0;
    double Bx = AB, By = 0.0;

    // Calculate cosA using the law of cosines
    double cosA = (AB*AB + CA*CA - BC*BC) / (2 * AB * CA);

    // Calculate sinA using identity sin^2 A + cos^2 A = 1
    double sinA = sqrt(1 - cosA*cosA);




    \end{lstlisting}
\end{frame}




\begin{frame}[fragile]
    \frametitle{C Code}
    \begin{lstlisting}

    // Coordinates of C
    double Cx = CA * cosA;
    double Cy = CA * sinA;

    printf("Coordinates of A: (%.2f, %.2f)\n", Ax, Ay);
    printf("Coordinates of B: (%.2f, %.2f)\n", Bx, By);
    printf("Coordinates of C: (%.2f, %.2f)\n", Cx, Cy);

    return 0;
}

    \end{lstlisting}
\end{frame}







\begin{frame}[fragile]
    \frametitle{Python Code}
    \begin{lstlisting}

import numpy as np
import matplotlib.pyplot as plt

# Side lengths
AB = 5
BC = 6
CA = 7

# Place A at (0,0), B at (5,0)
A = (0, 0)
B = (AB, 0)

# Calculate cosA using Law of Cosines
cosA = (AB**2 + CA**2 - BC**2) / (2 * AB * CA)
sinA = np.sqrt(1 - cosA**2)



    \end{lstlisting}
\end{frame}


\begin{frame}[fragile]
    \frametitle{Python Code}
    \begin{lstlisting}

# Coordinates of C
Cx = CA * cosA
Cy = CA * sinA
C = (Cx, Cy)

print('Coordinates of A:', A)
print('Coordinates of B:', B)
print('Coordinates of C:', (round(Cx, 2), round(Cy, 2)))

# 2D graph
plt.figure(figsize=(7,7))
plt.plot([A[0], B[0]], [A[1], B[1]], 'bo-', label='AB (5 cm)')
plt.plot([B[0], C[0]], [B[1], C[1]], 'go-', label='BC (6 cm)')
plt.plot([C[0], A[0]], [C[1], A[1]], 'ro-', label='CA (7 cm)')

for point, label in zip([A, B, C], ['A', 'B', 'C']):
    plt.text(point[0], point[1], label, fontsize=14, fontweight='bold', ha='right', color='black')



    \end{lstlisting}
\end{frame}

\begin{frame}[fragile]
    \frametitle{Python Code}
    \begin{lstlisting}

plt.xlabel('X (cm)')
plt.ylabel('Y (cm)')
plt.title('Triangle with sides 5 cm, 6 cm, 7 cm')
plt.legend()
plt.grid(True)
plt.axis('equal')
plt.tight_layout()
plt.savefig('triangle_5_6_7.png', dpi=200)
plt.close()
    \end{lstlisting}
\end{frame}


  
\begin{frame}{Plot}
\begin{figure}
    \centering
    \includegraphics[width=0.5\linewidth]{Beamer/figs/triangle_5_6_7.png}
    \caption{Caption}
    \label{fig:placeholder}
\end{figure}
\end{frame}




\end{document}
