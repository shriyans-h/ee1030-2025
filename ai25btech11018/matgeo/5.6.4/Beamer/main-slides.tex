\documentclass{beamer}
\usepackage[utf8]{inputenc}

\usetheme{Madrid}
\usecolortheme{default}
\usepackage{amsmath,amssymb,amsfonts,amsthm}
\usepackage{txfonts}
\usepackage{tkz-euclide}
\usepackage{listings}
\usepackage{adjustbox}
\usepackage[T1]{fontenc}
\usepackage{array}
\usepackage{tabularx}
\usepackage{gvv}
\usepackage{lmodern}
\usepackage{circuitikz}
\usepackage{tikz}
\usepackage{graphicx}

\setbeamertemplate{page number in head/foot}[totalframenumber]

\usepackage{tcolorbox}
\tcbuselibrary{minted,breakable,xparse,skins}



\definecolor{bg}{gray}{0.95}
\DeclareTCBListing{mintedbox}{O{}m!O{}}{%
  breakable=true,
  listing engine=minted,
  listing only,
  minted language=#2,
  minted style=default,
  minted options={%
    linenos,
    gobble=0,
    breaklines=true,
    breakafter=,,
    fontsize=\small,
    numbersep=8pt,
    #1},
  boxsep=0pt,
  left skip=0pt,
  right skip=0pt,
  left=25pt,
  right=0pt,
  top=3pt,
  bottom=3pt,
  arc=5pt,
  leftrule=0pt,
  rightrule=0pt,
  bottomrule=2pt,
  toprule=2pt,
  colback=bg,
  colframe=orange!70,
  enhanced,
  overlay={%
    \begin{tcbclipinterior}
    \fill[orange!20!white] (frame.south west) rectangle ([xshift=20pt]frame.north west);
    \end{tcbclipinterior}},
  #3,
}
\lstset{
    language=C,
    basicstyle=\ttfamily\small,
    keywordstyle=\color{blue},
    stringstyle=\color{orange},
    commentstyle=\color{green!60!black},
    numbers=left,
    numberstyle=\tiny\color{gray},
    breaklines=true,
    showstringspaces=false,
}
%------------------------------------------------------------
%This block of code defines the information to appear in the
%Title page
\title %optional
{ 5.6.4}

%\subtitle{A short story}

\author % (optional)
{Hemanth Reddy-AI25BTECH11018}



\begin{document}


\frame{\titlepage}
\begin{frame}{Question}
If $\vec{A}$ = $\myvec{3&-2\\4&-2}$ and $\vec{I}$ = $\myvec{1&0\\0&1}$, find k so that $\vec{A}^{2}=k\vec{A}-2\vec{I}$.


\end{frame}



\begin{frame}{Theoretical Solution}
\textbf{Solution:}\\

The characteristic equation for a matrix $\vec{A}$ is
f($\lambda$) = $\vec{A}$ - $\lambda$ $\vec{I}$ = 0 \\
\begin{align}
    \vec{A} - \lambda \vec{I} =\begin{vmatrix}
3 - \lambda & -2  \\
4& -2 - \lambda
\end{vmatrix} = 0
\end{align}

Upon expanding we get \qquad $\lambda^2$- $\lambda $+2=0\\
\begin{align}
     \lambda^2 = \lambda  -2
\end{align}
 Using the Cayley-Hamilton theorem f($\lambda$) = f($\vec{A}$ )=0
 \begin{align}
      \vec{A}^2 =  \vec{A}  -2\vec{I}
\end{align}
\centering{Value of k=1}




\end{frame}



\begin{frame}[fragile]
    \frametitle{C Code }
    \begin{lstlisting}

#include <stdio.h>

// Function to multiply two 2x2 matrices (A * A)
void multiply_matrices(double result[2][2], double A[2][2], double B[2][2]) {
    result[0][0] = A[0][0] * B[0][0] + A[0][1] * B[1][0];
    result[0][1] = A[0][0] * B[0][1] + A[0][1] * B[1][1];
    result[1][0] = A[1][0] * B[0][0] + A[1][1] * B[1][0];
    result[1][1] = A[1][0] * B[0][1] + A[1][1] * B[1][1];
}

// Function to multiply a 2x2 matrix by a scalar
void scalar_multiply(double result[2][2], double matrix[2][2], double k) {
    result[0][0] = k * matrix[0][0];
    result[0][1] = k * matrix[0][1];
    result[1][0] = k * matrix[1][0];
    result[1][1] = k * matrix[1][1];
}



    \end{lstlisting}
\end{frame}

\begin{frame}[fragile]
    \frametitle{C Code }
    \begin{lstlisting}

// Function to subtract two 2x2 matrices with a scalar multiplication
void subtract_matrices(double result[2][2], double A[2][2], double I[2][2]) {
    result[0][0] = A[0][0] - 2 * I[0][0];
    result[0][1] = A[0][1] - 2 * I[0][1];
    result[1][0] = A[1][0] - 2 * I[1][0];
    result[1][1] = A[1][1] - 2 * I[1][1];
}

int main() {
    // Define the matrix A and the identity matrix I
    double A[2][2] = {{3.0, -2.0}, {4.0, -2.0}};
    double I[2][2] = {{1.0, 0.0}, {0.0, 1.0}};

    // Calculate A^2
    double A_squared[2][2];
    multiply_matrices(A_squared, A, A);

   
    \end{lstlisting}
\end{frame}

\begin{frame}[fragile]
    \frametitle{C Code }
    \begin{lstlisting}
 // Solve for k using one element of the matrix equation
    // Equating the (0,0) elements: (A^2)[0][0] = k*A[0][0] - 2*I[0][0]
    // A^2[0][0] = k*3 - 2*1
    // A_squared[0][0] + 2 = 3*k
    double k_numerator = A_squared[0][0] + 2.0;
    double k_denominator = A[0][0];
    
    double k_val = k_numerator / k_denominator;

    // Verify the solution with another element for consistency
    // Equating the (1,1) elements: (A^2)[1][1] = k*A[1][1] - 2*I[1][1]
    // A^2[1][1] = k*(-2) - 2*1
    // A^2[1][1] + 2 = -2*k
    double k_val_check = (A_squared[1][1] + 2.0) / A[1][1];

   

    \end{lstlisting}
\end{frame}

\begin{frame}[fragile]
    \frametitle{C Code }
    \begin{lstlisting}
 printf("Calculated value of k from (0,0) element: %.2f\n", k_val);
    printf("Calculated value of k from (1,1) element: %.2f\n\n", k_val_check);

    // Print the final result and verify the equation
    printf("The value of k that satisfies A^2 = kA - 2I is k = %.2f\n", k_val);
    
    return 0;
}

    \end{lstlisting}
\end{frame}

\begin{frame}[fragile]
    \frametitle{Python Code }
    \begin{lstlisting}



import sys
#for path to external scripts
sys.path.insert(0, '/sdcard/github/matgeo/codes/CoordGeo')
#path to my scripts
import numpy as np
import numpy.linalg as LA
import matplotlib.pyplot as plt
import matplotlib.image as mpimg

#local imports
#from line.funcs import *
#from triangle.funcs import *
#from matrix.funcs import *

    \end{lstlisting}
\end{frame}

\begin{frame}[fragile]
    \frametitle{Python Code }
    \begin{lstlisting}

#from conics.funcs import circ_gen


#if using termux
import subprocess
import shlex
#end if

def solve_for_k_cayley_hamilton(A, I):
    """
    Solves for the scalar k in the matrix equation A^2 = kA - 2I
    by using the Cayley-Hamilton Theorem.

    Args:
        A (np.array): A 2x2 NumPy array representing matrix A.
        I (np.array): A 2x2 NumPy array representing the identity matrix I.

    
    \end{lstlisting}
\end{frame}

\begin{frame}[fragile]
    \frametitle{Python Code }
    \begin{lstlisting}
Returns:
        float: The value of k that satisfies the equation.
    """
    # Step 1: Find the characteristic polynomial coefficients of A
    # The characteristic polynomial is lambda^2 - (tr(A))lambda + det(A) = 0
    coeffs = np.poly(A)
    # The coefficients will be [1, -tr(A), det(A)]

    # Step 2: Apply the Cayley-Hamilton Theorem
    # The matrix A satisfies its characteristic equation:
    # A^2 + coeffs[1]*A + coeffs[2]*I = 0
    
    # Rearranging the Cayley-Hamilton equation:
    # A^2 = -coeffs[1]*A - coeffs[2]*I

    
    \end{lstlisting}
\end{frame}

\begin{frame}[fragile]
    \frametitle{Python Code }
    \begin{lstlisting}
# Step 3: Compare with the given equation
    # Given: A^2 = kA - 2I
    # By comparing the coefficients of A and I, we find:
    # k = -coeffs[1]
    # -2 = -coeffs[2]  =>  coeffs[2] = 2

    # The value of k is the negative of the second coefficient from np.poly(A)
    k_value = -coeffs[1]
    
    # Verification check to see if the determinant is correct
    if np.isclose(coeffs[2], 2):
        print("Verification successful: The determinant from the characteristic polynomial matches the problem statement.")
    else:
        print("Verification failed: The determinant does not match the problem statement.")
        
 


    \end{lstlisting}
\end{frame}

\begin{frame}[fragile]
    \frametitle{Python Code }
    \begin{lstlisting}
   return k_value

# Given matrices
A = np.array([[3, -2],
              [4, -2]])
I = np.array([[1, 0],
              [0, 1]])

# Find the value of k
k_value = solve_for_k_cayley_hamilton(A, I)

# Print the final result
if k_value is not None:
    print(f"\nThe value of k is: {k_value}")


        \end{lstlisting}
\end{frame}


\end{document}
