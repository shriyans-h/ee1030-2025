
\let\negmedspace\undefined
\let\negthickspace\undefined
\documentclass[journal]{IEEEtran}
\usepackage[a5paper, margin=10mm, onecolumn]{geometry}
%\usepackage{lmodern} % Ensure lmodern is loaded for pdflatex
\usepackage{tfrupee} % Include tfrupee package

\setlength{\headheight}{1cm} % Set the height of the header box
\setlength{\headsep}{0mm}     % Set the distance between the header box and the top of the text

\usepackage{gvv-book}
\usepackage{gvv}
\usepackage{cite}
\usepackage{amsmath,amssymb,amsfonts,amsthm}
\usepackage{amsmath}
\usepackage{algorithmic}
\usepackage{graphicx}
\usepackage{textcomp}
\usepackage{xcolor}
\usepackage{txfonts}
\usepackage{listings}
\usepackage{enumitem}
\usepackage{mathtools}
\usepackage{gensymb}
\usepackage{comment}
\usepackage[breaklinks=true]{hyperref}
\usepackage{tkz-euclide} 
\usepackage{listings}
% \usepackage{gvv}                                        
\def\inputGnumericTable{}                                 
\usepackage[latin1]{inputenc}                                
\usepackage{color}                                            
\usepackage{array}                                            
\usepackage{longtable}                                       
\usepackage{calc}                                             
\usepackage{multirow}                                         
\usepackage{hhline}                                           
\usepackage{ifthen}                                           
\usepackage{lscape}
\usepackage{circuitikz}
\tikzstyle{block} = [rectangle, draw, fill=blue!20, 
    text width=4em, text centered, rounded corners, minimum height=3em]
\tikzstyle{sum} = [draw, fill=blue!10, circle, minimum size=1cm, node distance=1.5cm]
\tikzstyle{input} = [coordinate]
\tikzstyle{output} = [coordinate]


\begin{document}

\bibliographystyle{IEEEtran}
\vspace{3cm}

\title{5.6.4}
\author{AI25BTECH11018-Hemanth Reddy}
 \maketitle
% \newpage
% \bigskip
{\let\newpage\relax\maketitle}

\renewcommand{\thefigure}{\theenumi}
\renewcommand{\thetable}{\theenumi}
\setlength{\intextsep}{10pt} % Space between text and floats


\numberwithin{equation}{enumi}
\numberwithin{figure}{enumi}
\renewcommand{\thetable}{\theenumi}

\textbf{Question:}\\
If $\vec{A}$ = $\myvec{3&-2\\4&-2}$ and $\vec{I}$ = $\myvec{1&0\\0&1}$, find k so that $\vec{A}^{2}=k\vec{A}-2\vec{I}$.


\textbf{Solution:}\\

The characteristic equation for a matrix $\vec{A}$ is
f($\lambda$) = $\vec{A}$ - $\lambda$ $\vec{I}$ = 0 \\
\begin{align}
    \vec{A} - \lambda \vec{I} =\begin{vmatrix}
3 - \lambda & -2  \\
4& -2 - \lambda
\end{vmatrix} = 0
\end{align}

Upon expanding we get \qquad $\lambda^2$- $\lambda $+2=0\\
\begin{align}
     \lambda^2 = \lambda  -2
\end{align}
 Using the Cayley-Hamilton theorem f($\lambda$) = f($\vec{A}$ )=0
 \begin{align}
      \vec{A}^2 =  \vec{A}  -2\vec{I}
\end{align}
\centering{Value of k=1}

\end{document}






