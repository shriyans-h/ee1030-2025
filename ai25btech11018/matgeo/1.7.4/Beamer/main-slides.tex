\documentclass{beamer}
\usepackage[utf8]{inputenc}

\usetheme{Madrid}
\usecolortheme{default}
\usepackage{amsmath,amssymb,amsfonts,amsthm}
\usepackage{txfonts}
\usepackage{tkz-euclide}
\usepackage{listings}
\usepackage{adjustbox}
\usepackage{array}
\usepackage{tabularx}
\usepackage{gvv}
\usepackage{lmodern}
\usepackage{circuitikz}
\usepackage{tikz}
\usepackage{graphicx}

\setbeamertemplate{page number in head/foot}[totalframenumber]

\usepackage{tcolorbox}
\tcbuselibrary{minted,breakable,xparse,skins}



\definecolor{bg}{gray}{0.95}
\DeclareTCBListing{mintedbox}{O{}m!O{}}{%
  breakable=true,
  listing engine=minted,
  listing only,
  minted language=#2,
  minted style=default,
  minted options={%
    linenos,
    gobble=0,
    breaklines=true,
    breakafter=,,
    fontsize=\small,
    numbersep=8pt,
    #1},
  boxsep=0pt,
  left skip=0pt,
  right skip=0pt,
  left=25pt,
  right=0pt,
  top=3pt,
  bottom=3pt,
  arc=5pt,
  leftrule=0pt,
  rightrule=0pt,
  bottomrule=2pt,
  toprule=2pt,
  colback=bg,
  colframe=orange!70,
  enhanced,
  overlay={%
    \begin{tcbclipinterior}
    \fill[orange!20!white] (frame.south west) rectangle ([xshift=20pt]frame.north west);
    \end{tcbclipinterior}},
  #3,
}
\lstset{
    language=C,
    basicstyle=\ttfamily\small,
    keywordstyle=\color{blue},
    stringstyle=\color{orange},
    commentstyle=\color{green!60!black},
    numbers=left,
    numberstyle=\tiny\color{gray},
    breaklines=true,
    showstringspaces=false,
}
%------------------------------------------------------------
%This block of code defines the information to appear in the
%Title page
\title %optional
{1.7.4}
\date{September 8,2025}
%\subtitle{A short story}

\author % (optional)
{Hemanth Reddy-AI25BTECH11018}



\begin{document}


\frame{\titlepage}
\begin{frame}{Question}
Using vectors, prove that the points(2,-1,3), (3,-5,1),and(-1,11,9) are collinear.
\end{frame}



\begin{frame}{Theoretical Solution}
\textbf{Solution:}\\


Let $\vec{A}$ \myvec{2\\-1\\3}  $\vec{B}$ \myvec{3\\-5\\1}   $\vec{C}$\myvec{-1\\11\\9} be vectors \\
Points $\vec{A}$,$\vec{B}$, $\vec{C}$ are defined to be collinear if\\
\begin{center}
  
        
    
  \begin{align}
\text{rank}\left(\textbf{B} - \textbf{A} \quad \textbf{C} - \textbf{A}\right) = 1
\end{align}

\begin{align}
    \text{Let }\left(\textbf{B} - \textbf{A} \quad \textbf{C} - \textbf{A}\right) = \textbf{D} 
\end{align}

\begin{align}
    \text{rank} \textbf{D} = \text{rank} \textbf{D}^{T}
\end{align}
\end{center}


\end{frame}

\begin{frame}{Theoretical Solution}

\begin{align}
    \textbf{D}^{T}=\myvec{
1 & -4 & -2 \\
-3 & 12 & 6
}
\end{align}

\begin{align}
    R_{2} = R_{2} + 3R_{2}
\end{align}

\begin{align}
    \textbf{D}^{T}=\myvec{
1 & -4 & -2 \\
0 & 0 & 0
}
\end{align}

 which has rank 1.So we can conclude that the given points are collinear.
 

\end{frame}


\begin{frame}[fragile]
    \frametitle{C Code }
    \begin{lstlisting}

#include <stdio.h>

// Function to compute cross product of two vectors in 3D
void crossProduct(int v1[3], int v2[3], int cross[3]) {
    cross[0] = v1[1]*v2[2] - v1[2]*v2[1];
    cross[1] = v1[2]*v2[0] - v1[0]*v2[2];
    cross[2] = v1[0]*v2[1] - v1[1]*v2[0];
}

// Function to check if three points are collinear in 3D
int areCollinear(int A[3], int B[3], int C[3]) {
    int AB[3], AC[3], cross[3];
    for (int i = 0; i < 3; i++) {
        AB[i] = B[i] - A[i];
        AC[i] = C[i] - A[i];
    }


    \end{lstlisting}
\end{frame}

\begin{frame}[fragile]
    \frametitle{C Code }
    \begin{lstlisting}

    crossProduct(AB, AC, cross);
    return (cross[0] == 0 && cross[1] == 0 && cross[2] == 0);
}

int main() {
    int A[3] = {2, -1, 3};
    int B[3] = {3, -5, 1};
    int C[3] = {-1, 11, 9};
    if (areCollinear(A, B, C))
        printf("The points are collinear.\n");
    else
        printf("The points are not collinear.\n");
    return 0;
}
    \end{lstlisting}
\end{frame}







\begin{frame}[fragile]
    \frametitle{Python Code}
    \begin{lstlisting}

import matplotlib.pyplot as plt
from mpl_toolkits.mplot3d import Axes3D
import numpy as np

# Points
A = np.array([2, -1, 3])
B = np.array([3, -5, 1])
C = np.array([-1, 11, 9])

# Check collinearity (print for reference)
AB = B - A
AC = C - A

print("Vector AB:", AB)
print("Vector AC:", AC)








    \end{lstlisting}
\end{frame}


\begin{frame}[fragile]
    \frametitle{Python Code}
    \begin{lstlisting}

# Verify if AC is a scalar multiple of AB
if np.allclose(AC, (AC[0]/AB[0]) * AB):
    print("Points are collinear")
else:
    print("Points are not collinear")

# 3D Plot
fig = plt.figure()
ax = fig.add_subplot(111, projection='3d')

# Plot points
ax.scatter(*A, color='red', label='A(2, -1, 3)')
ax.scatter(*B, color='green', label='B(3, -5, 1)')
ax.scatter(*C, color='blue', label='C(-1, 11, 9)')

# Plot line through A in direction of AB (which also passes through B and C)

    \end{lstlisting}
\end{frame}
\begin{frame}[fragile]
    \frametitle{Python Code}
    \begin{lstlisting}

t = np.linspace(-2, 2, 100)
line = A[:, None] + np.outer(AB, t)

ax.plot(line[0], line[1], line[2], 'k--', label='Line through A, B, and C')

# Labels and legend
ax.set_xlabel('X')
ax.set_ylabel('Y')
ax.set_zlabel('Z')
ax.legend()
ax.set_title('Collinear Points in 3D')

# Save figure as PNG
plt.savefig("collinear_points.png")

# Show plot
plt.show()

    \end{lstlisting}
\end{frame}
  
\begin{frame}{Plot}

\begin{figure}
    \centering
    \includegraphics[width=0.5\linewidth]{Beamer/figs/collinear_points.png}
    \caption{}
    \label{fig:placeholder}
\end{figure}



\end{frame}




\end{document}
