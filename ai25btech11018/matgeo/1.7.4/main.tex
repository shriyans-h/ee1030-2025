\let\negmedspace\undefined
\let\negthickspace\undefined
\documentclass[journal]{IEEEtran}
\usepackage[a5paper, margin=10mm, onecolumn]{geometry}
%\usepackage{lmodern} % Ensure lmodern is loaded for pdflatex
\usepackage{tfrupee} % Include tfrupee package

\setlength{\headheight}{1cm} % Set the height of the header box
\setlength{\headsep}{0mm}     % Set the distance between the header box and the top of the text

\usepackage{gvv-book}
\usepackage{gvv}
\usepackage{cite}
\usepackage{amsmath,amssymb,amsfonts,amsthm}
\usepackage{amsmath}
\usepackage{algorithmic}
\usepackage{graphicx}
\usepackage{textcomp}
\usepackage{xcolor}
\usepackage{txfonts}
\usepackage{listings}
\usepackage{enumitem}
\usepackage{mathtools}
\usepackage{gensymb}
\usepackage{comment}
\usepackage[breaklinks=true]{hyperref}
\usepackage{tkz-euclide} 
\usepackage{listings}
% \usepackage{gvv}                                        
\def\inputGnumericTable{}                                 
\usepackage[latin1]{inputenc}                                
\usepackage{color}                                            
\usepackage{array}                                            
\usepackage{longtable}                                       
\usepackage{calc}                                             
\usepackage{multirow}                                         
\usepackage{hhline}                                           
\usepackage{ifthen}                                           
\usepackage{lscape}
\usepackage{circuitikz}
\tikzstyle{block} = [rectangle, draw, fill=blue!20, 
    text width=4em, text centered, rounded corners, minimum height=3em]
\tikzstyle{sum} = [draw, fill=blue!10, circle, minimum size=1cm, node distance=1.5cm]
\tikzstyle{input} = [coordinate]
\tikzstyle{output} = [coordinate]


\begin{document}

\bibliographystyle{IEEEtran}
\vspace{3cm}

\title{1.2.26}
\author{AI25BTECH11018-Hemanth Reddy}
 \maketitle
% \newpage
% \bigskip
{\let\newpage\relax\maketitle}

\renewcommand{\thefigure}{\theenumi}
\renewcommand{\thetable}{\theenumi}
\setlength{\intextsep}{10pt} % Space between text and floats


\numberwithin{equation}{enumi}
\numberwithin{figure}{enumi}
\renewcommand{\thetable}{\theenumi}

\textbf{Question:}\\
Using vectors, prove that the points(2,-1,3), (3,-5,1),and(-1,11,9) are collinear.


\textbf{Solution:}\\
Let $\vec{A}$ \myvec{2\\-1\\3}  $\vec{B}$ \myvec{3\\-5\\1}   $\vec{C}$\myvec{-1\\11\\9} be vectors \\
Points $\vec{A}$,$\vec{B}$, $\vec{C}$ are defined to be collinear if\\

\begin{center}
    
$
\text{rank}\left(\textbf{B} - \textbf{A} \quad \textbf{C} - \textbf{A}\right) = 1
$

$
\text{rank} \textbf{A} = \text{rank} \textbf{A}^{T}
$\\
\vspace{0.5cm}
$\textbf{A}^{T}=$\myvec{
1 & -4 & -2 \\
-3 & 12 & 6
}\\
\vspace{0.4cm}
$R_{2} = R_{2} + 3R_{2}$\\
\vspace{0.4cm}
$\textbf{A}^{T}=$\myvec{
1 & -4 & -2 \\
0 & 0 & 0
}\\
\vspace{0.4cm}
 which has rank 1.So we can conclude that the given points are collinear.
\begin{figure}
    \centering
    \includegraphics[width=0.5\linewidth]{Beamer/figs/collinear_points.png}
    \caption{}
    \label{fig:placeholder}
\end{figure}


\end{center}


\end{document}